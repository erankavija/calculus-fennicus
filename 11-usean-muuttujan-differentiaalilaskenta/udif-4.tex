\section{Vektorikentät: Divergenssi ja roottori} \label{divergenssi ja roottori}
\sectionmark{Divergenssi ja roottori}
\alku
\index{osittaisdifferentiaalioperaattori!vektorioperaattorit|vahv}

Tässä luvussa tarkastellaan funktioita, jotka riippuvat kahdesta tai kolmesta fysikaalisesta
p\pain{aikka}muuttujasta ($x,y$ tai $x,y,z$), mahdollisesti lisäksi fysikaalisesta
\pain{aika}muuttujasta $t$. Funktiot voivat olla joko \pain{skalaarifunktioita}, eli
reaaliarvoisia (tai skalaariarvoisia, kirjaimellisesti 'skaalaajafunktioita') tai
vektoriarvoisia, eli
\index{vektorikenttä} \index{funktio A!i@vektorikenttä}%
\kor{vektorikenttiä} (engl.\ vector field). Jatkossa ajatellaan ainoastaan fysikaalisia
vektorikenttiä, joissa on joko kaksi (tasokenttä) tai kolme (avaruuskenttä) komponenttia.
Skalaarifunktiot, joita
\index{skalaarikenttä}%
kutsutaan myös \kor{skalaarikentiksi}, voivat siis olla tyyppiä $f(x,y)$, $f(x,y,z)$,
$f(x,y,t)$ tai $f(x,y,z,t)$ (joskus aikamuuttuja halutaan kirjoittaa ensimmäiseksi,
esim.\ $f(t,x,y,z)$) ja vektorikentät esim.\ tyyppiä
\[ 
\vec F(x,y,z,t) = F_1(x,y,z,t)\,\vec i + F_2(x,y,z,t)\,\vec j + F_3(x,y,z,t)\,\vec k. 
\]
Fysikaalisia vektorikenttiä ovat esim. virtaavan nesteen, kaasun, yms.\ nopeusvektoreiden 
muodostama \pain{virtauskenttä}, (esim.\ gravitaatioon liittyvä) \pain{voimakenttä}, 
sähkömagneettisiin ilmiöihin liittyvät \pain{sähkökenttä} ja \pain{ma}g\pain{neettikenttä}, ym.
Tällaiset fysikaaliset kentät ovat joko \pain{staattisia} tai \pain{d}y\pain{naamisia}, riippuen
siitä onko muuttujien joukossa myös aika ($t$) vai ei. Vektorikenttä kuvataan usein graafisesti
'nuolikenttänä' (kuvassa virtauskenttä).

\begin{figure}[H]
\setlength{\unitlength}{1cm}
\begin{center}
\begin{picture}(12,5.5)
\path(0,0.5)(5,0.5)(5,2.5)(7,2.5)(7,0.5)(12,0.5)
\path(0,5)(12,5)
\Thicklines
\multiput(0,1)(0,0.5){8}{\vector(1,0){1}}
\multiput(1.5,1)(0,0.5){8}{\vector(1,0){1}}
\multiput(3,3.5)(0,0.5){3}{\vector(1,0){1}}
\multiput(3,1)(0,0.5){3}{\vector(4,1){0.5}}
\multiput(3,2.5)(0,0.5){2}{\vector(4,1){1}}
\multiput(3.8,1.2)(0,0.5){2}{\vector(1,1){0.5}}
\put(3.8,2.2){\vector(2,1){0.5}}
\multiput(5,3)(0,0.5){4}{\vector(1,0){2}}
\multiput(8,3.5)(0,0.5){3}{\vector(1,0){1}}
\multiput(8,2.5)(0,0.5){2}{\vector(4,-1){1}}
\put(8,2){\vector(2,-1){0.5}}
\put(8.8,1.6){\vector(1,-4){0.15}}
\put(8.6,0.8){\vector(-4,-1){0.5}}
\put(8,1){\vector(-1,3){0.15}}
\multiput(9.5,2.5)(0,0.5){5}{\vector(1,0){1}}
\put(9.5,2){\vector(4,-1){1}}
\multiput(11,1)(0,0.5){8}{\vector(1,0){1}}
\path(5,0.5)(7,0.5)
\multiput(1,0.5)(1,0){11}{\line(-1,-1){0.25}}
\end{picture}
%\caption{Esimerkki virtauskentästä}
\end{center}
\end{figure}
\index{differentiaalioperaattori!c@osittaisderivoinnin}%
Vektorikenttään kohdistuvina differentiaalioperaattorit 
$\,\partial_x,\partial_y,\partial_z,\partial_t\,$ ymmärretään samalla tavoin kuin skalaarin ja
vektorin kertolaskussa, eli operaattorit kohdistuvat vektorikentän jokaiseen komponenttiin. 
Esim.
\[ 
\partial_x\vec F(x,y,z,t) 
        = \pder{F_1}{x}\,\vec i + \pder{F_2}{x}\,\vec j + \pder{F_3}{x}\,\vec k.
\]

\subsection{Gradienttikentät}
\index{gradienttikenttä|vahv}

Jos $f$ on skalaarifunktio (skalaarikenttä) edellä mainittua tyyppiä, niin funktion
gradientti ymmärretään vain paikkamuuttujiin $x,y$ tai $x,y,z$ kohdistuvana. Gradientti toimii
tällöin differentiaalioperaattorina, joka muuntaa skalaarifunktion vektoriarvoiseksi funktioksi
eli vektorikentäksi. Monet fysikaaliset vektorikentät ovat tällaisia \kor{gradienttikenttiä}.
Jos $\vec F$ on gradienttikenttä ja
\[
\vec F=-\nabla u
\]
(miinusmerkki fysikaalisista mukavuussyistä), niin sanotaan, että $u$ on kentän $\vec F$ 
\index{potentiaali (vektorikentän)} \index{skalaaripotentiaali}%
\kor{(skalaari)potentiaali}.
\begin{Exa} Fysikaalisia esimerkkejä potentiaaleista ja gradienttikentistä ovat esimerkiksi\,:

\vspace{3mm}
\begin{tabular}{ll}
\pain{Potentiaali} & \pain{Gradienttikenttä} \\ \\
$u=$ gravitaatiopotentiaali \hspace{2cm} & $\vec G=-\nabla u=$ gravitaatiokenttä \\ \\
$u=$ sähköpotentiaali & $\vec E=-\nabla u=$ sähkökenttä \\ \\
$u=$ lämpötila & $\vec J=-\lambda\nabla u=$ lämpövirran tiheys
\end{tabular}
\vspace{2mm}

\index{Fourierb@Fourier'n laki}%
Yhteyttä $\vec J=-\lambda\nabla u$ sanotaan lämpöopissa \pain{Fourier'n} \pain{laiksi}. Sen 
mukaan siis lämpöenergia virtaa lämpötilan negatiivisen gradientin $-\nabla u$ suuntaan ja 
virtatiheys (yksikkö W/m$^2$) on gradientin itseisarvoon verrannollinen. 
Verrannollisuuskerrointa sanotaan materiaalin \pain{lämmön}j\pain{ohtavuudeksi}. \loppu
\end{Exa}

\subsection{Divergenssi ja roottori. Laplacen operaattori}
\index{differentiaalioperaattori!e@divergenssi $\nabla\cdot$|vahv}
\index{differentiaalioperaattori!f@roottori $\nabla\times$|vahv}
\index{differentiaalioperaattori!g@Laplacen operaattori|vahv} 
\index{divergenssi|vahv} \index{roottori|vahv} \index{Laplacen operaattori|vahv}

Differentiaalioperaattori $\nabla=\partial_x\vec i+\partial_y\vec i+\partial_z\vec k$ voidaan 
myös ymmärtää operaattorina, jonka kohteena ovat vektoriarvoiset funktiot eli vektorikentät. 
Tällöin operaattori $\nabla$ operoi kohdefunktioon $\vec F$ joko pistetulon (skalaaritulon) tai
ristitulon (vektoritulon) välityksellä. Kun kohdefunktioksi oletetaan
\[
\vec F(x,y,z)=F_1(x,y,z)\vec i + F_2(x,y,z)\vec j + F_3(x,y,z)\vec k,
\]
niin tulevat näin määritellyksi kentän $\vec F$ \kor{divergenssi} (engl. divergence)
\[
\boxed{\kehys\quad \nabla\cdot\vec F
           =\partial_x F_1+\partial_y F_2+\partial_z F_3 \quad\text{(divergenssi)}\quad}
\]
ja \kor{roottori} (engl.\ curl)
\[
\boxed{\quad \nabla\times\vec F=\begin{vmatrix}
\vec i & \vec j & \vec k \\
\partial_x & \partial_y & \partial_z \\
F_1 & F_2 & F_3
\end{vmatrix} \quad\text{(roottori)}.\quad}
\]
Roottorin laskukaavan purettu muoto on
\[ \nabla \times \vec F = (\partial_y F_3 - \partial_z F_2)\vec i 
                        - (\partial_x F_3 - \partial_z F_1)\vec j
                        + (\partial_x F_2 - \partial_y F_1)\vec k. 
\]
Divergenssiä ja roottoria merkitään myös symboleilla div ja rot (tai $\Vect{\text{rot}}$). 
Tavallisia lukutapoja ovat 'nabla piste' ja 'nabla risti', vastaten gradientin $\nabla f$ 
lukutapaa 'nabla f'.
\begin{Exa} Funktion $\vec F(x,y,z)=x^2\vec i+y^2z\vec j-(2xz+yz^2)\vec k$ divergenssi ja 
roottori ovat
\begin{align*}
\nabla\cdot\vec F  &= 2x+2yz-2x-2yz=0, \\
\nabla\times\vec F &= (\partial_y F_3-\partial_z F_2)\vec i
                     +(\partial_z F_1-\partial_x F_3)\vec j
                     +(\partial_x F_2-\partial_y F_1)\vec k \\
                   &= -(y^2+z^2)\vec i+2z\vec j. \loppu
\end{align*}
\end{Exa}
Piste- ja ristitulojen kautta tulevat määritellyksi myös mm.\ operaattoritulot 
\[
(1)\ \ \nabla\cdot\nabla \qquad (2)\ \ \nabla(\nabla\cdot\,) \qquad
(3)\ \ \nabla\times(\nabla\times\,) \qquad (4)\ \ \nabla\times\nabla
\]
Näistä viimeinen on nollaoperaattori:
\[
\boxed{\kehys\quad \nabla\times\nabla=\vec 0. \quad }
\]
Nollaoperaattori on myös $\nabla\cdot(\nabla\times\,)$, ts.\ pätee (vrt.\ Luku \ref{ristitulo})
\[
\boxed{\kehys\quad \nabla\cdot\nabla\times\vec F = (\nabla\times\nabla)\cdot\vec F = 0. \quad}
\]
Operaattorien (1), (2) ja (3) välillä on yhteys
(Harj.teht.\,\ref{H-udif-4: nablaamissääntöjä}f --- vrt.\ Luku \ref{ristitulo})
\[
\boxed{\kehys\quad \nabla\times(\nabla\times\vec F)=\nabla(\nabla\cdot\vec F)
                                                   -(\nabla\cdot\nabla)\vec F. \quad}
\]
Operaattoria $\nabla\cdot\nabla$, joka voi operoida sekä skalaariin että vektoriin, sanotaan 
\kor{Laplacen operaattoriksi}. Tämä on hyvin yleinen differentiaalioperaattori erilaisissa 
fysiikan laeissa (ks.\ huomautukset jäljempänä), siksi sille on vakiintunut erillinen symboli 
$\Delta$. Myös merkintää $\nabla^2$ käytetään. Lukutapoja ovat 'Laplace' tai 'nabla toiseen'. 
Määritelmä on siis
\begin{align*}
\nabla\cdot\nabla &= (\vec i \, \partial_x+\vec j \, \partial_y +\vec k \, \partial_z)\cdot
                     (\vec i \, \partial_x+\vec j \, \partial_y +\vec k \, \partial_z) \\
                  &=\partial_x^2+\partial_y^2+\partial_z^2=\Delta.
\end{align*}

\subsection{Operaattorit $\nabla\star$ ja $\vec u\cdot\nabla$}
\index{differentiaalioperaattori!h@$\nabla\star$, $\vec u\cdot\nabla$|vahv}

Yleisiä 'nablaamissääntöjä' voidaan usein yksinkertaistaa käyttämällä operaattorimerkintää
$\nabla\star$, missä '$\star$' voi olla 'piste' ($\nabla\cdot$), 'risti' ($\nabla\times$) tai
'tyhjä' ($\nabla$). Viimeksi mainitussa tapauksessa tulkitaan kohde aina skalaarikentäksi,
ts.\ $\nabla\vec F=\nabla F$. Operaattorimerkintä $\nabla\star$ on kätevä esimerkiksi tulon
derivoimissäännön yleistyksessä: Jos $f=f(x,y,z)$ on skalaarikenttä ja $\vec F$
kolmiulotteinen vektorikenttä, niin pätee
\[
\boxed{\kehys\quad \nabla\star(f\vec F)=\nabla f\star\vec F + f\nabla\star\vec F. \quad}
\] 
(Tapauksessa $\star=$ 'tyhjä' tämä tulkitaan: $\nabla(fF)=F\nabla f+f\nabla F$.)
Perustelu jätetään harjoitustehtäväksi (Harj.teht.\,\ref{H-udif-4: nablaamissääntöjä}abc). 

\index{muuttuvakertoiminen diff.-oper.}%
\kor{Muuttuvakertoimisista} vektoridifferentiaalioperaattoreista on syytä mainita
sovelluksissa yleinen operaattori
\[
\vec u\cdot\nabla  = u_1\partial_x + u_2\partial_y + u_3\partial_z,
\]
missä $\vec u=\vec u(x,y,z)$ (tai $\vec u=\vec u(x,y,z,t)$) on vektorikenttä. Tämä on
skalaarinen operaattori, joka voi kohdistua skalaari- tai vektorikenttään. Edellisessä
(ei jälkimmäisessä) tapauksessa $\vec u\cdot\nabla f$ on laskettavissa myös kaksivaiheisesti:
\[
f \map \nabla f \map \vec u\cdot\nabla f.
\]
Muuttuvakertoimisia operaattoreita yhdisteltäessä on huomioitava, että derivointi kohdistuu 
tällöin myös kertoimiin (vrt.\ Luku \ref{osittaisderivaatat}).

\subsection{Vectorikentän lähde. Poissonin ja Laplacen yhtälöt}

Kun $\vec F$ on jokin fysikaalinen kenttä, kuten sähkö-, magneetti-, virtaus- ym.\ kenttä, 
sanotaan divergenssiä
\[
\nabla\cdot\vec F=\rho
\]
\index{lzy@lähde (vektorikentän)}%
kentän $\vec F$ \kor{lähteeksi}. Kentän ja lähteen $\rho$ välinen yhteys on tällaisissa 
sovellutuksissa luonnonlaki, jossa lähde tulkitaan kentän $\vec F$ aiheuttajaksi --- siitä nimi
'lähde'. Tyypillinen lähtökohta onkin, että lähde on tunnettu ja kenttä halutaan määrätä. 
Esimerkiksi gravitaatiokenttää määrättäessä on $\rho$ verrannollinen massatiheyteen ja 
sähkökentän tapauksessa varaustiheyteen.

Jos vektorikentän $\vec F$ lähde $\rho$ on tunnettu ja lisäksi tiedetään, että $\vec F$ on 
gradienttikenttä, eli on olemassa potentiaali $u$ siten, että $\vec F = -\nabla u$ 
(näin on esim.\ gravitaatio- ja sähkökentän tapauksessa), niin kentän ja lähteen välinen yhteys
$\nabla\vec F=\rho$ voidaan kirjoittaa muotoon
\[
-\Delta u=\rho.
\]
\index{Poissonin yhtälö}%
Tätä sanotaan \kor{Poissonin yhtälöksi}. Sellaisissa pisteissä, joissa kenttä $\vec F$ on 
\index{lzy@lähteetön vektorikenttä} \index{vektorikenttä!a@lähteetön}%
\kor{lähteetön}, eli $\nabla\cdot\vec F=0$, toteuttaa kentän potentiaali
\index{Laplacen yhtälö}% 
\kor{Laplacen yhtälön}\footnote[2]{Laplacen ja Poissonin yhtälöiden nimet viittaavat 
ranskalaisiin matemaatikko-fyysikkoihin \hist{P.S. de Laplace} (1749-1827) ja 
\hist{S.D. Poisson} (1781-1840), jotka tutkivat tämän tyyppisten yhtälöiden teoriaa. Yhtälöiden
käytännöllisen ratkaisemisen perustan loi samoihin aikoihin vaikuttanut \hist{J.B.J. Fourier} 
(1768-1830). Myös Fourier oli matemaatikko-fysiikko, joka tutki erityisesti 
lämmönjohtumisilmiöitä. Näistä tutkimuksista alkunsa saaneet \kor{Fourier-sarjat} ja 
\kor{Fourier-muunnokset} ovat tehneet Fourier'n nimestä kuolemattoman.
\index{Laplace, P. S. de|av} \index{Poisson, S. D.|av} \index{Fouriera@Fourier, F. B. J.|av} }
\[
\Delta u=0.
\]
\index{harmoninen funktio}%
Funktiota, joka toteuttaa Laplacen yhtälön, sanotaan \kor{harmoniseksi}.
 
Sovellustilanteissa Poissonin tai Laplacen yhtälö esiintyy tyypillisesti
\index{reuna-arvotehtävä}% 
\kor{reuna-arvo\-tehtävässä} (engl.\ boundary value problem), jossa tarkasteltavan (avoimen)
joukon $A$ --- tai niinkuin tällaisissa yhteyksissä useammin sanotaan, \kor{alueen} $A$ 
($A\subset\Rkaksi$ tai $A\subset\Rkolme$) 
\index{reuna (joukon, alueen)} \index{reunaehto}% 
--- \kor{reunalla} $\partial A$ asetetaan jokin \kor{reunaehto}. Sovellustilanteessa reunaehto
määräytyy fysikaalisin perustein. Esimerkiksi jos reunalla asetetaan nk.\ (homogeeninen) 
\index{Dirichletb@Dirichlet'n (reuna)ehto}%
\kor{Dirichlet'n ehto}, niin saadaan reuna-arvotehtävän perusmuoto:
\[
\begin{cases} 
-\Delta u=\rho &\text{$A$:n sisäpisteissä}, \\ \,u=0 &\text{reunalla $\partial A$.}
\end{cases}
\]
Tämä on osoitettavissa tietyin (funktiota $\rho$ ja reunaviivaa/pintaa $S=\partial A$ koskevin) 
säännöllisyysehdoin yksikäsitteisesti ratkeavaksi. Itse ratkaisemisen on yleensä tapahduttava
numeerisin (likimääräisin) keinoin, muutamia poikkeustilanteita lukuun ottamatta.
\begin{Exa}
Jos $A$ on $R$-säteinen kiekko ($A\subset\Rkaksi$) tai kuula ($A\subset\Rkolme$) ja $\rho=Q=$ 
vakio, niin reuna-arvotehtävän
\[
\begin{cases} 
-\Delta u=Q &\text{$A$:n sisäpisteissä}, \\ \,u=0 &\text{reunalla $\partial A$}
\end{cases}
\]
ratkaisu on kummassakin tapauksessa toisen asteen polynomi:
\begin{align*}
A\subset\Rkaksi: \qquad u(x,y)   &= \frac{Q}{4}(R^2-x^2-y^2). \\
A\subset\Rkolme: \quad  u(x,y,z) &= \frac{Q}{6}(R^2-x^2-y^2-z^2). \loppu
\end{align*}
\end{Exa}

Poissonin ja Laplacen yhtälöt ovat esimerkkejä yhtälöistä, joista käytetään yleisnimeä 
\index{osittaisdifferentiaaliyhtälö}
\index{differentiaaliyhtälö!r@osittaisdifferentiaaliyhtälö}%
\kor{osittaisdifferentiaaliyhtälö} (engl.\ partial differential equation, lyh.\ PDE).
Paitsi Poissonin ja Laplacen yhtälöissä, Laplacen operaattori esiintyy monissa muissakin 
fysiikan osittaisdifferentiaaliyhtälöissä. Mainittakoon tässä ainoastaan ajasta riippuvia 
ilmiöitä kuvaavat \kor{diffuusioyhtälö} ja \kor{aaltoyhtälö}, joiden perusmuodot ovat 
(sovelluksissa näistä esiintyy erilaisia variaatioita):
\index{diffuusioyhtälö} \index{aaltoyhtälö}%
\begin{align*}
\text{Diffuusioyhtälö}: \quad   &\,u_t=\Delta u. \\
\text{Aaltoyhtälö}: \qquad\,\   &u_{tt}=\Delta u.
\end{align*}
Diffuusioyhtälön tyyppiä on esimerkiksi (aikariippuva) 
\pain{lämmön}j\pain{ohtumis}y\pain{htälö}. \index{lzy@lämmönjohtumisyhtälö}%

\subsection{Pyörrekenttä.  Maxwellin yhtälöt}

Fysikaalisen vektorikentän roottoria
\[
\nabla\times\vec F=\vec \omega
\]
\index{pyzz@pyörrekenttä}%
sanotaan kentän $\vec F$ \kor{pyörrekentäksi}. Pyörrekentillä on fysikaalinen merkitys 
esimerkiksi virtausmekaniikassa, mutta erityisen keskeisellä sijalla ne ovat sähkömagneettisten
kenttien teoriassa. Tämä nähdään jo sähkömagneettisten kenttien perusyhtälöistä, 
\kor{Maxwellin yhtälöistä} (\vahv{J.C. Maxwell}, 1864): \index{Maxwellin yhtälöt}
\[
\left\{\begin{aligned}
\nabla\times\vec E &= -\frac{\partial \vec B}{\partial t}\,, \\
\nabla\times\vec H &=  \frac{\partial \vec D}{\partial t}+\vec J, \\[2mm]
\nabla\cdot\vec D &=\rho, \\[3mm]
\nabla\cdot\vec B &= 0.
\end{aligned}\right.
\]
Tässä esiintyvien kenttien nimet ovat:
\begin{align*}
\vec E: \quad &\text{sähkökenttä} \qquad\qquad \vec D: \quad \text{sähkövuon tiheys} \\
\vec H: \quad &\text{magneettikenttä} \qquad\, \vec B: \quad \text{magneettivuon tiheys} \\
\rho  : \quad &\text{varaustiheys} \qquad\qquad \vec J: \quad \text{virtatiheys}
\end{align*}
Maxwellin yhtälöitä ratkaistaessa on huomioitava kenttien $\vec E,\vec D$ ja $\vec H,\vec B$
keskinäinen riippuvuus, joka on väliaineelle ominainen. Yksinkertisimmillaan riippuvuudet ovat
ilmaistavissa kahden materiaalivakion $\epsilon,\mu$ avulla muodossa 
\[
\vec D=\epsilon\vec E, \quad \vec B=\mu\vec H.
\]

\index{pyzz@pyörteetön vektorikenttä} \index{vektorikenttä!b@pyörteetön}%
Vektorikenttää $\vec F$ sanotaan \kor{pyörteettömäksi} sellaisessa alueessa, jossa 
$\nabla\times\vec F=\vec 0$. Kaikki gradienttikentät, eli skalaaripotentiaalin avulla 
ilmaistavat kentät, ovat pyörteettömiä (koska $\nabla\times\nabla=\vec 0$). Maxwellin
yhtälöiden mukaan sähkökenttä $\vec E$ on pyörteetön \pain{stationaarisessa} tilanteessa, jossa
kentät ovat ajasta riippumattomia. Sekä lähteetön että pyörteetön on esimerkiksi 
gravitaatiokenttä tyhjässä (massattomassa) avaruudessa, ja yleisemminkin gradienttikenttä 
lähteettömässä alueessa.

\subsection{Roottori tasossa}
\index{differentiaalioperaattori!f@roottori $\nabla\times$|vahv}
\index{roottori|vahv}

Jos tason vektorikenttä tulkitaan ($z$-koordinaatista riippumattomaksi ja $xy$-tason
suuntaiseksi) avaruuden vektorikentäksi, niin kentän roottori on
\[
\nabla\times[F_1(x,y)\,\vec i+F_2(x,y)\,\vec j\,] =(\partial_x F_2-\partial_y F_1)\,\vec k.
\]
Tässä on tapana kirjoittaa
\[
\text{rot}\,\vec F = \frac{\partial F_2}{\partial x}-\frac{\partial F_1}{\partial y}
                   \quad (\text{tason vektorikenttä}),
\]
jolloin roottori tasossa tulee määritellyksi skalaarisena operaattorina (kuten div). 
\begin{Exa} Maxwellin yhtälöiden mukaan stationaarisessa tilanteessa magneettikentän $\vec H$
pyörrekenttä on sähkövirran tiheys $\vec J$. Jos oletetetaan tyypillinen sovellustilanne,
jossa virta kulkee suorassa, $z$-akselin suuntaisessa johtimessa, niin $\vec J$ on muotoa
\[
\vec J(x,y,z)=\begin{cases} 
              \,I(x,y)\vec k, &\text{kun}\ (x,y) \in A, \\ \,\vec 0, &\text{muulloin}.
              \end{cases}
\]
Tässä $A\subset\Rkaksi$ edustaa johtimen poikkileikkausta ja $I$ on (skalaarinen) virran tiheys
johtimessa. Magneettikenttä on $xy$-tason suuntainen ja $z$-koordinaatista riippumaton,
ts.\ muotoa $\vec H = H_1(x,y)\vec i + H_2(x,y)\vec j$, jolloin Maxwellin yhtälö 
$\nabla\times\vec H=\vec J$ pelkistyy johtimen poikkipinnalla skalaariseksi yhtälöksi
\[
\text{rot}\,\vec H = \partial_x H_2 - \partial_y H_1 = I(x,y).
\]
Johtimen ulkopuolella magneettikenttä on pyörteetön: $\text{rot}\,\vec H=0$. \loppu
\end{Exa}

\subsection{*Navier--Stokesin virtausyhtälöt}
\index{Navier--Stokesin yhtälöt|vahv}

Nesteen tai kaasun (esim.\ veden tai ilman) virtausta hallitsevat perusyhtälöt ovat nk.\
\index{szy@säilymislaki}%
\kor{säilymislakeja} (engl.\ conservation laws), jotka ilmaisevat matemaattisesti
(osittaisdifferentiaaliyhtälöinä) \pain{massan}, \pain{liikemäärän} ja \pain{ener}g\pain{ian}
säilymisen virtauksessa. Massan ja liikemäärän säilymislaeissa esiintyvät vektori- ja
skalaarikentät ovat
\begin{align*}
\vec u: \quad &\text{virtausnopeus} \\
     p: \quad &\text{paine} \\
  \rho: \quad &\text{(massa)tiheys}
\end{align*}
Jos virtaavan aineen massatiheys voidaan olettaa vakioksi (esim.\ vesi, monissa
virtaustilanteissa myös ilma) ja lisäksi virtaus on ulkoisista voimista vapaa, niin liikemäärän
ja massan säilymislait voidaan kirjoittaa \kor{Navier--Stokesin yhtälöinä}
(\vahv{C.-L. Navier, G.G. Stokes}, 1822) 
\[
\left\{ \begin{aligned}
\vec u_t-\nu\Delta\vec u+(\vec u\cdot\nabla)\vec u+\rho^{-1}\nabla p &= \vec 0, \\
                                               \nabla\cdot\vec u &=0.
\end{aligned} \right.
\]
Tässä $\nu$ on virtaavan aineen (kinemaattinen) \pain{viskositeetti}.
%\footnote[2]{Virtausmekaniikan yhtälöiden numeerinen ratkaiseminen
%tietokoneilla on nykyisin arkipäivää lentokoneiden aerodynamiikassa, laivojen
%hydrodynamiikassa, meteorologiassa, ym.\ sovelluksissa. Yhtälöihin liittyy silti myös avoimia
%matemaattisia ongelmia. Esimerkiksi Navier--Stokesin yhtälöiden nk.\ globaalia (ajassa ja
%paikassa rajoittamatonta) ratkeavuutta (tai vastaesimerkillä: ratkeamattomuutta) säännöllisestä
%alkutilanteesta ei ole pystytty osoittamaan matemaattisesti. --- Kyseessä on eräs tunnettu
%''miljoonan dollarin ongelma'', jonka ratkaisjalle on yhdysvaltalainen Clay-instituutti
%luvannut mainitun suuruisen palkkion.}

\Harj
\begin{enumerate}

\item
Laske divergenssi ja roottori: \vspace{2mm}\newline
a) \ $x\vec i+y\vec j+z\vec k \qquad$
b) \ $yz\vec i+xz\vec j+xy\vec k \qquad$
c) \ $xy^2\vec i-yz^2\vec j+zx^2\vec k$ \vspace{1mm}\newline
d) \ $f(x)\vec i+g(y)\vec j+h(z)\vec k \qquad$
e) \ $f(y,z)\vec i+g(x,z)\vec j+h(x,y)\vec k$
 
\item 
Olkoon
$\vec F(x,y,z)=(x y-z^2)\,\vec i + x y z \,\vec j + (x-y^2-z^2)\,\vec k$. \vspace{1mm}\newline
a) Laske $\,\nabla\cdot\vec F$, $\,\nabla(\nabla\cdot\vec F)$, $\,\nabla\times\vec F$, 
$\,\nabla\times(\nabla\times\vec F)\,$ ja $\,\Delta \vec F$ ja tarkista, että pätee
$\,\nabla\times(\nabla\times\vec F) 
            = \nabla(\nabla\cdot\vec F)-\Delta\vec F$. \vspace{1mm}\newline
b) Laske $(\vec F\cdot\nabla)\vec F$ ja $(\vec F\times\nabla)\cdot\vec F$.

\item \label{H-udif-4: nablaamissääntöjä}
Olkoon $\vec F$ ja $\vec G$ tason tai avaruuden (mahdollisesti ajasta riippuvia) vektorikenttiä
ja $f$ ja $g$ skalaarikenttiä. Perustele seuraavat säännöt
(oletetaan riittävä derivoituvuus): \vspace{1mm}\newline
a) \ $\nabla(fg)=g\nabla f+f\nabla g\qquad\qquad\qquad\ \ \ $
b) \ $\nabla\cdot(f\vec F)=\vec F\cdot\nabla f+f\,\nabla\cdot\vec F$ \vspace{1mm}\newline
c) \ $\nabla\times(f\vec F)=\nabla f\times\vec F+f\,\nabla\times\vec F\qquad\ $
d) \ $\D\partial_t(\nabla\star\vec F)=\nabla\star\vec F_t$ \vspace{1mm}\newline
e) \ $\nabla\cdot(\vec F\times\vec G)=(\nabla\times\vec F)\cdot\vec G
                                        -(\nabla\times\vec G)\cdot\vec F$ \vspace{1mm}\newline
f) \ $\nabla\times(\nabla\times\vec F)
                  =\nabla(\nabla\cdot\vec F)-\Delta\vec F$ \vspace{1mm}\newline
g) \ $\square(\vec F\circ\vec G) = (\square\vec F)\circ\vec G+\vec F\circ\square\vec G \,\ $
     ja $\,\ \square(f\vec F)=(\square f)\vec F+f\square\vec F$, \newline
     kun $\,\ \square\in\{\partial_x,\partial_y,\partial_z,\partial_t\}\,\ $ ja
     $\,\ \circ\in\{\,\cdot\,,\times\,\}$ \vspace{1mm}\newline
h) \ $(\vec u\,\square)\circ f\vec F 
      = \square f\,\vec u\circ\vec F+f\vec u\circ\,\square\vec F$,\, kun
     $\,\ \square\in\{\partial_x,\partial_y,\partial_z,\partial_t\}\,\ $ ja
     $\,\ \circ\in\{\,\cdot\,,\times\,\}$ \newline

\item
a) Näytä, että seuraavat funktiot ovat määrittelyjoukossaan harmonisia:
\begin{align*}
&xy,\ \ x^2-y^2,\ \ x^3-3xy^2,\ \ x^3y-xy^3,\ \ \ln(x^2+y^2),\ \ \Arctan\frac{y}{x} \\
&e^{ax}(A\sin ay+B\cos ay)\ \ (a,A,B\in\R)
\end{align*}
b) Monen muuttujan Laplacen operaattori määritellään
$\Delta=\partial_1^2+ \ldots + \partial_n^2$. Tutki, millaisille reaalifunktioille $f$ pätee: 
$f\bigl(\abs{\mx})=f(\sqrt{x_1^2 + \ldots + x_n^2}\,\bigr)$ on harmoninen muualla kuin origossa.

\item
Seuraavilla osittaisdifferentiaaliyhtälöillä väitetään olevan annettua muotoa olevia ratkaisuja.
Tarkista väittämät (funktiot $f$ ja $g$ oletetaan kahdesti derivoituviksi).
\vspace{1mm}\newline
a) \ $u_{tt}=c^2u_{xx}\,,\ \ u(x,t)=f(x+ct)+g(x-ct)$ \vspace{1mm}\newline
b) \ $u_{xx}+u_{yy}=2u_{xy}\,,\ \ u(x,y)=xf(x+y)+yg(x+y)$ \vspace{1mm}\newline
c) \ $u_t=u_{xx}\,,\ \ u(x,t)=e^{-\omega^2 t}(A\sin\omega x+B\sin\omega x)\ \ (\omega,A,B\in\R)$
\vspace{1mm}\newline
d) \ $u_t=u_{xx}\,,\ \ u(x,t)=t^{-1/2}e^{-x^2/4t}$ \vspace{1mm}\newline
e) \ $u_t=u_{xx}+u_{yy}\,,\ \ u(x,y,t)=t^{-1}e^{-(x^2+y^2)/4t}$

\item
Näytä, että osittaisdifferentiaaliyhtälön $u_{xy}=0$ yleinen ratkaisu $\R^2$:ssa on
$u(x,y)=f(x)+g(y)$, missä $f:\,\R\kohti\R$ ja $g:\,\R\kohti\R$ ovat mitä tahansa $\R$:ssä 
derivoituvia funktioita.

\item (*)
Todista (olettaen riittävä derivoituvuuus)\,:
\[
\nabla\times(\vec F\times\vec G)=(\nabla\cdot\vec G)\vec F+(\vec G\cdot\nabla)\vec F
                                -(\nabla\cdot\vec F)\vec G-(\vec F\cdot\nabla)\vec G.
\]

\item (*)
Olkoon $A\subset\R^2$ tasasivuinen kolmio, jonka kärjet ovat $(0,0)$, $(a,0)$ ja
$(a/2,\sqrt{3}a/2)$ $(a>0)$. Näytä, että reuna-arvotehtävällä
\[
\begin{cases} 
-\Delta u=Q &\text{$A$:n sisäpisteissä}, \\ \,u=0 &\text{reunalla $\partial A$}
\end{cases}
\]
($Q=$ vakio) on polynomiratkaisu.

\item (*)
Tarkastellaan Maxwellin yhtälöitä homogeenisessa väliaineessa, jossa $\rho=0$, $\vec J=\vec 0$
ja kenttien $\vec D,\vec E$ ja $\vec B,\vec H$ välillä on yhteydet $\vec D=\epsilon\vec E$ ja 
$\vec B=\mu\vec H$ ($\epsilon$ ja $\mu$ materiaalivakioita). Näytä, että kaikki mainitut
kentät, sikäli kuin riittävän säännöllisiä, toteuttavat vektorimuotoisen aaltoyhtälön
\[
\vec F_{tt}=c^2\Delta\vec F,
\]
missä $c=(\epsilon\mu)^{-1/2}$ (= valon nopeus väliaineessa). 

\end{enumerate}