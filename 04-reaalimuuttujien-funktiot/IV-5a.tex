%poistokamaa vanhasta Luvusta IV-5: Parametriset käyrät
\begin{Exa} Kuvan piirtämistä varten halutaan parametrisoida avaruuden pistejoukko $S$, joka 
koostuu pisteistä
\[
P \vastaa (x,y,z): \quad \begin{cases} x^2 + y^2 + z^2 = 1 \\ x+y+z = 1 \end{cases}
\]
Määrittele jokin parametrisaatio. 
\end{Exa}
\ratk Kyseessä on origokeskisen pallon ja tason $\,T:\ x+y+z = 1\,$ leikkausviiva, joten $S$ on
ympyräviiva tasolla $T$. Ympyräviivan keskipiste on $T$:n piste $P$, joka on lähinnä origoa $O$.
Tällöin vektorin $\Vect{OP}$ on oltava $T$:n normaalivektorin suuntainen, eli 
$\Vect{OP} = c(\vec{i}+\vec{j}+\vec{k})$ jollakin $c$. Koska $A = (1,0,0) \in S$, niin $c$ 
määräytyy ehdosta
\[ 
\Vect{OP} \cdot \Vect{PA} 
     = c(\vec{i}+\vec{j}+\vec{k}) \cdot [\vec{i}-c(\vec{i}+\vec{j}+\vec{k}] = c-3c^2 = 0. 
\]
Käypä ratkaisu on $c=1/3$, joten eräs tason $T$ suuntavektori on
\[ 
\vec{v}_1 = \overrightarrow{PA} = \frac{1}{3}(2\vec{i}-\vec{j}-\vec{k}). 
\]
Tässä $P$ on ympyräviivan $S$ keskipiste ja $A \in S$, joten ympyräviivan säde on
\[ 
R = \abs{\vec{v}_1} = \frac{\sqrt{6}}{3}\,. 
\]
Haetaan tasolle $T$ toinen suuntavektori $\vec{v}_2 = \overrightarrow{OB}$ siten että 
$\,\vec{v}_1 \cdot \vec{v}_2 = 0$ ja $\abs{\vec{v}_2}=R$ (jolloin $B \in S$). Ehdot ovat
\begin{align*}
\vec{v}_2\ =\ a\,(\Vect{OP} \times \vec{v}_1)\ 
                  &=\ \dfrac{a}{3}\,(\vec{j}-\vec{k}) \quad \ja \quad \abs{\vec{v}_2} 
                        = \dfrac{\sqrt{6}}{3}\\
\qimpl \vec{v}_2\ &= \pm \dfrac{1}{\sqrt{3}}(\vec{j}-\vec{k}).
\end{align*}
Kun tässä valitaan etumerkki $+$, niin ympyräviivalle $S$ saadaan parametrisaatio 
(vrt.\ Esimerkki \ref{ympyrän parametrisaatio})
\[ 
\vec{r}(t) = \Vect{OP} + R\cos t\,\vec{v}_1 + R\sin t\,\vec{v}_2, \quad t \in [0,2\pi]. 
\]
Komponenttimuodossa tämä on
\[ \begin{cases}
x = \dfrac{1}{3} + \dfrac{2\sqrt{6}}{9} \cos t, \\[2mm]
y = \dfrac{1}{3} - \dfrac{\sqrt{6}}{9} \cos t + \dfrac{\sqrt{2}}{3} \sin t, \\[2mm]
z = \dfrac{1}{3} - \dfrac{\sqrt{6}}{9} \cos t - \dfrac{\sqrt{2}}{3} \sin t, 
                                                           \quad t \in [0,2\pi] \loppu
\end{cases} \]