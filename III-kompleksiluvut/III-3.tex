\section{Algebran peruslause} \label{III-3}
\alku

\index{funktio A!e@kompleksifunktio}%
Kompleksianalyysin alalaji \kor{funktioteoria} tutkii kompleksimuuttujan kompleksiarvoisia 
funktioita tyyppiä
\[
f: \C \kohti \C \quad \text{tai} \quad f: A \kohti \C, \quad A\subset\C.
\]
Yksinkertaisin (vaan ei vähäisin) esimerkki tällaisesta funktiosta on edellisessä luvussa
(Propositio \ref{polynomin kasvu}) tarkasteltu, koko $\C$:ssä määritelty polynomifunktio
\begin{equation} \label{polynomi}
p(z) = \sum_{k=0}^n c_k z^k, \quad c_k \in \C, \ c_n \neq 0.
\end{equation}

Jokainen kompleksimuuttujan kompleksiarvoinen funktio on esitettävissä muodossa
\[
f(z) = \re f(z) + i\,\im f(z) = u(x,y) + iv(x,y), \quad z = x + iy,
\]
missä $u,v$ ovat funktioita tyyppiä $u,v: \Rkaksi\kohti\R$ 
(vrt.\ Luku \ref{trigonometriset funktiot}). Itse $f$ on siis myös tulkittavissa funktioksi
tyyppiä $f: \Rkaksi\kohti\C$. Kompleksimuuttujan funktioita onkin luontevaa ajatella
geometrisesti tasossa määritellyiksi. Tasoa sanotaan tällaisissa tarkasteluissa
\index{kompleksitaso}%
\kor{kompleksitasoksi}. Kompleksitasossa siis piste $(x,y)$ tarkoittaa kompleksilukua
$z = x +iy$.
\begin{figure}[H]
\setlength{\unitlength}{1cm}
\begin{center}
\begin{picture}(8,6)(-1,-1)
\put(-1,0){\vector(1,0){8}} \put(5.5,-0.5){$x=\text{Re}\,z$}
\put(0,-1){\vector(0,1){6}} \put(0.2,4.8){$y=\text{Im}\,z$}
\put(3.9,2.9){$\bullet$} \put(4.2,2.9){$z=x+iy$}
\dashline{0.2}(0,3)(4,3) \dashline{0.2}(4,3)(4,0)
\put(3.9,-0.4){$x$} \put(-0.4,2.9){$y$}
\end{picture}
\end{center}
\end{figure}
Kompleksifunktion
\index{juuri (kompleksifunktion)}%
$f$ \kor{juureksi} (engl. root) tai yksinkertaisesti \kor{nollakohdaksi} 
sanotaan jokaista $z \in \C$, jolle
\[
f(z) = 0.
\]
Reaali- ja imaginaariosiin hajotetusta muodosta juuret $z = x +iy$ saadaan yhtälöryhmän
\[
\begin{cases}
 u(x,y) = 0 \\ v(x,y) = 0
\end{cases}
\]
ratkaisuista.
\begin{Exa} \label{kompleksipolynomin juuret} $f(z) = z^3 +1$. Juuret? \end{Exa}
\ratk Tässä
\begin{align*}
f(x+iy) &= (x+iy)^3 + 1 \\
&= (x^3 -3xy^2 +1) + (3x^2y -y^3)i,
\end{align*}
joten $f$:n nollakohdat ratkeavat yhtälöryhmästä
\[
\left\{ \begin{array}{ll}
x^3 -3xy^2 +1 = 0, \\
3x^2y -y^3 = 0.
\end{array} \right.
\]
Jokaista yhtälöryhmän reaalista ratkaisua $(x,y) \in \Rkaksi$ vastaa $f$:n juuri $x+iy \in \C$.
Juuret löydetään tässä helposti, koska yhtälöryhmän jälkimmäinen yhtälö on kirjoitettavissa 
muotoon $(3x^2-y^2)y=0$. Näin ollen on oltava joko $y=0$ tai 
$y^2 = 3x^2\ \ekv\ y = \pm\sqrt{3}\,x$. Sijoittamalla nämä edelliseen yhtälöön saadaan juuriksi
\begin{multicols}{2} \raggedcolumns
\begin{align*}
(x_1,y_1) = (-1,0)\qquad\,\ &\vastaa\ z_1 = -1, \\[3mm]
(x_2,y_2) = \left(\frac{1}{2}\,,\frac{\sqrt{3}}{2}\right)\quad\,  
                            &\vastaa\ z_2 = \frac{1}{2}\left(1 +i\sqrt{3}\right), \\
(x_3,y_3) = \left(\frac{1}{2}\,,-\frac{\sqrt{3}}{2}\right)\ 
                            &\vastaa\ z_3 = \frac{1}{2}\left(1 -i\sqrt{3}\right).
\end{align*}
\begin{figure}[H]
\setlength{\unitlength}{1cm}
\begin{center}
\begin{picture}(6,6)(-3.5,-3)
\put(-3,0){\vector(1,0){6}} \put(2.5,-0.5){$\text{Re}\,z$}
\put(0,-3){\vector(0,1){6}} \put(0.2,2.8){$\text{Im}\,z$}
\Thicklines
\put(2,0){\line(0,-1){0.3}} \put(1.9,-0.8){$1$}
\put(-2,0){\line(0,-1){0.3}} \put(-2.43,-0.8){$-1$}
\thinlines
\put(0,0){\bigcircle{4}}
\put(-2.1,-0.1){$\bullet$} \put(-1.8,0.2){$z_1$}
\put(0.9,1.63){$\bullet$} \put(1.1,1.8){$z_2$}
\put(0.9,-1.83){$\bullet$} \put(1.1,-2){$z_3$}
\end{picture}
\end{center}
\end{figure}
\end{multicols}

Esimerkissä $f$ oli polynomi astetta 3 ja sille löydettiin kolme juurta. Tulos on
erikoistapaus lauseesta, joka on sekä laajakantoinen että syvällinen:
\begin{*Lause} \vahv{(Algebran peruslause)} \label{algebran pl}
\index{Algebran peruslause|emph} Jos $p \, $ on polynomi muotoa \eqref{polynomi} ja astetta
$n \geq 1$, niin on olemassa $n$ kompleksilukua $z_1, \ldots, z_n$ siten, että pätee
\begin{align*}
p(z) &= c_n(z-z_1) \cdot \cdots \cdot (z-z_n) \\
     &= c_n \prod_{i=1}^n (z-z_i).
\end{align*}
\end{*Lause} 
Lauseen \ref{algebran pl} mukaisessa tulohajotelmassa luvut $z_i$ eivät välttämättä ole eri
suuret, joten polynomissa voi olla tekijä muotoa
\[
(z -z_k)^m, \quad 1 \leq m \leq n,
\]
liittyen juureen $z_k$. Jos $p$:llä on tämä tekijä mutta ei tekijää $(z -z_k)^{m+1}$, niin 
sanotaan, että juuri $z_k$ on \kor{$m$-kertainen} (engl. $m$-fold) tai että juuren 
\index{kertaluku!a@juuren (nollakohdan)}
\kor{kertaluku} (engl. order) on $m$. Jos kertaluku on $m=1$, sanotaan, että juuri on 
\index{yksinkertainen!a@nollakohta (juuri)}%
\kor{yksinkertainen} (engl. simple). Jos samaa juurta edustavat tekijät kootaan yhteen,
saadaan Lauseen \ref{algebran pl} tulohajotelmalle muoto
\begin{equation} \label{tulohajotelma}
p(z) = c_n \prod_{i=1}^ \nu (z -z_i)^{m_i},
\end{equation}
missä nyt $z_i \neq z_j$ kun $i \neq j$, $m_i \in \N$ ja
\[
\sum_{i=1}^ \nu m_i = n.
\]
Erityisesti jos hajotelmassa \eqref{tulohajotelma} on $\nu = n$, eli polynomilla on $n$ eri 
juurta, on kaikkien juurien oltava yksinkertaisia. Tämä tilanne oli Esimerkissä 
\ref{kompleksipolynomin juuret}.

Esimerkissä \ref{kompleksipolynomin juuret} aidosti kompleksiset juuret
($\text{Im}\,z \neq 0$) esiintyivät \kor{konjugaattiparina}. Näin on yleisemmin silloin, kun
(kuten esimerkissä) polynomi on \kor{reaalikertoiminen}, ts. $c_k \in \R$ lausekkeessa
\eqref{polynomi}. Tämä perustuu siihen, että reaalikertoimisen polynomin tapauksessa pätee
(vrt. edellisen luvun kaavat (1)--(2))
\[
\overline{p(z)} = p(\overline{z}) \quad (\text{reaalikertoiminen polynomi}).
\]
Tällöin
\[
p(z) = 0 \ \impl \ p(\overline{z}) = 0,
\]
ja juuret siis 'pariutuvat'. Jos konjugaattiparin muodostavat juuret
\[
z= a \pm ib,
\]
niin polynomissa on tekijänä
\[
(z-a-ib)(z-a+ib) = (z-a)^2 +b^2.
\]
Reaalikertoiminen polynomi voidaan siis aina hajottaa reaalikertoimisiin tekijöihin, jotka
ovat joko muotoa
\begin{align*}
&\text{(a)}\qquad (z-a_k)^m, \quad a_k \in \R,
\intertext{vastaten $m$-kertaista reaalijuurta $z_k=a_k$, tai muotoa}
&\text{(b)}\qquad [(z-a_k)^2+b_k^2\,]^m \quad a_k, b_k \in \R, \ b_k \neq 0,
\end{align*}
vastaten $m$-kertaista konjugaattijuuriparia $z_k=a_k \pm ib_k$.
\jatko \begin{Exa} (jatko) Esimerkin polynomi hajoaa muotoon
\begin{align*}
z^3+1 &= (z+1)(z^2-z+1) \\
&= (z+1)\left[\left(z-\frac{1}{2}\right)^2+\left(\frac{\sqrt{3}}{2}\right)^2\right]. \loppu
\end{align*}
\end{Exa}
'Algebran peruslause' edellä esitetyssä muodossa on itse asiassa oikean 
\kor{Algebran peruslauseen} tekninen seuraamus. Oikea peruslause muotoillaan:
\begin{*Lause} \vahv{(Algebran peruslause -- lyhyt muoto)} \label{algebran peruslause}
\index{Algebran peruslause|emph} Jos  $n \ge 1$, niin polynomilla \eqref{polynomi} on juuri
kompleksitasossa.
\end{*Lause} 
Lauseen \ref{algebran peruslause} perusteella polynomi purkautuu heti tekijöihin Lauseen
\ref{algebran pl} mukaisesti. Nimittäin jos yksi juuri $z_1$ on löydettävissä, niin $p(z)$ 
voidaan kirjoittaa muotoon
\[
p(z) = p(z) - p(z_1) = \sum_{k=1}^n c_k(z^k-z_1^k).
\]
Tässä on kunta-algebran perusteella (Propositio \ref{kuntakaava})
\[
z^k-z_1^k = (z-z_1)(z^{k-1} + z_1z^{k-2} + \ldots + z_1^{k-1}), \quad k = 1 \ldots n,
\]
joten nähdään, että $p(z)$ on kirjoitettavissa muotoon
\[
p(z) = (z-z_1)\,p_1(z),
\]
missä $p_1$ on polynomi astetta $n-1$ ja muotoa 
\[
p_1(z) = c_n z^{n-1} + [\,\text{alempiasteisia termejä}\,]. 
\]
Mikäli $n \geq 2$, on $p_1$:llä puolestaan edelleen juuri Lauseen \ref{algebran peruslause} 
perusteella, jolloin $p_1$:stä on erotettavissa tekijä $(z-z_2)$ jne, ja Lauseen 
\ref{algebran pl} väittämä siis seuraa.

Algebran peruslause on seuraamuksiltaan sen verran mittava, että matematiikkaa --- edes 
sovellettua matematiikkaa --- ilman sitä on nykyisin vaikea kuvitella. Lause on myös hyvin 
elegantti em. 'lyhyen kaavan' mukaan esitettynä. Edes todistusperiaatteen taakse kurkistaminen 
ei eleganssia vähennä, pikemminkin päinvastoin: Algebran peruslauseen voi nykyisin nähdä 
kompleksifunktioiden yleisemmän teorian seurauksena. Tarkemmin sanoen kyse on nk.\ 
\kor{analyyttisten} funktioiden teoriasta. Käsite 'analyyttinen' määritellään myöhemmin
Luvussa \ref{analyyttiset funktiot}; tässä yhteydessä todettakoon vain, että kyse on
äärimmäisen säännöllisistä kompleksifunktioista, esimerkiksi sellaisista kuin juuri polynomit.

Algebran peruslauseen todistuksen perusidea analyyttisten funktioiden teorian avulla
esitettynä on seuraava\footnote[2]{Algebran peruslauseen todistamiseen paneutui mm.\ 
'matemaatikkojen ruhtinaaksi' sanottu saksalainen \hist{Carl Friedrich Gauss} (1777-1855).
Erään hyvin uskottavan todistuksen Gauss esitti väitöskirjassaan v. 1799. Loogisesti täysin
auktoton todistus pystyttiin esittämään vasta myöhemmin 1800-luvulla reaali- ja
kompleksianalyysin kehityttyä riittävästi. --- Myöhemmin Luvussa V.10 esitetään Algebran
peruslauseelle melko suoraviivainen todistus, joka perustuu vain reaalianalyysiin, Propositioon
\ref{polynomin kasvu} ja toiseen kompleksialgebran väittämään, joka on esitetty
harjoitustehtävässä \ref{H-III-3: avaintulos}. \index{Gauss, C.F.|av}}: Tehdään vastaoletus,
että polynomilla $p$ ei ole nollakohtia. Siinä tapauksessa funktio
\[
f(z) = \frac{1}{p(z)}
\]
on koko kompleksitasossa säännöllinen. Itse asiassa $f$ on 'äärimmäisen säännöllinen' eli 
analyyttinen. Kun lisäksi käytetään Proposition \ref{polynomin kasvu} tulosta, niin voidaan 
todeta: Jos $p$:n aste on $n \ge 1$ ja $p$:llä ei ole nollakohtia, niin funktio $f=1/p$
toteuttaa
\begin{enumerate}
\item $f$ on analyyttinen koko kompleksitasossa.
\item $f(z) \rightarrow 0$, kun $\abs{z} \rightarrow \infty$.
\end{enumerate}
On olemassa ainakin yksi funktio, joka toteuttaa nämä molemmat ehdot, nimittäin $f(z) = 0$.
Kysymys kuulukin: Onko muita? Analyyttisten funktioiden teorian antama --- hieman yllättävä ---
vastaus tähän kysymykseen on:
\begin{itemize}
\item[] Ei!
\end{itemize}
Vastaoletus on näin ollen johtanut loogiseen ristiriitaan ja lause siis on tosi.

\subsection{Kompleksiluvun juuret}
\index{juuri (kompleksiluvun)|vahv}
\index{\ohje!a@--luvun tai osaluvun otsikkoon|vahv}
\index{\ohje!b@--lauseeseen tai määritelmään|emph}
\index{\ohje!c@--tekstiin}
\index{\ohje!d@--alaviitteeseen|av}

Algebran peruslauseen mukaan polynomiyhtälöllä
\begin{equation} \label{kompleksiluvun juuret}
z^n = a, \quad a \in \C,\ n \in \N,\ n \ge 2
\end{equation}
on ainakin yksi juuri. Itse asiassa osoittautuu, että jos $a \neq 0$ (mikä oletetaan), niin 
juuria on tasan $n$ kappaletta, eli kaikki juuret ovat yksinkertaisia. Merkitään niitä kaikkia 
symbolilla
\[
z=a^{1/n}=\sqrt[n]{a},
\]
ja sanotaan, että kyseessä ovat \kor{kompleksiluvun $a$ juuret}. Näitä siis tulee olemaan 
$n$ kpl, ja halutaan laskea ne perusmuotoon
\[
z_k = x_k + iy_k, \quad k=0 \ldots n-1.
\]
Tähän päästään helpoiten polaariesityksen kautta: Kirjoitetaan
\[
a= r \vkulma{\varphi} = r (\cos \varphi + i \sin \varphi),
\]
missä siis $r= \abs{a} \neq 0$. Kun merkitään $z=\abs{z}\vkulma{\psi}$, niin
\begin{align*}
z^n = a &\qekv \abs{z}^n = r \quad\ja\quad n\psi=\varphi+k\cdot 2\pi,\ k\in\Z \\
        &\qekv \abs{z}=\sqrt[n]{r} \quad\ja\quad 
               \psi=\frac{\varphi}{n}+k\cdot\frac{2\pi}{n},\ k\in\Z.
\end{align*}
Erilaisia ratkaisuja saadaan tästä indeksin arvoilla $k=0 \ldots n-1$. Nämä voidaan esittää
muodossa
\begin{align*}
z_k &= \sqrt[n]{r}\vkulma{\tfrac{\varphi}{n}+k\cdot\tfrac{2\pi}{n}} \\
    &= \left(\sqrt[n]{r}\vkulma{\tfrac{\varphi}{n}}\right)
       \left(1\vkulma{k\cdot\tfrac{2\pi}{n}}\right) \\
    &= \left(\sqrt[n]{r}\vkulma{\tfrac{\varphi}{n}}\right)
       \left(1\vkulma{\tfrac{2\pi}{n}}\right)^k \\
    &= z_0\rho^k, \quad k=0 \ldots n-1,
\end{align*}
missä siis
\begin{align*}
z_0  &\,=\,\sqrt[n]{r}\vkulma{\frac{\varphi}{n}} 
      \,=\,\sqrt[n]{r}\left(\cos\frac{\varphi}{n}+i\sin\frac{\varphi}{n}\right), \quad
                             r=|a|,\ \varphi=\arg a, \\
\rho &\,=\, 1\vkulma{\frac{2\pi}{n}} = \cos\frac{2\pi}{n}+i\sin\frac{2\pi}{n}\,.
\end{align*}
Perusmuodossa yhtälön (\ref{kompleksiluvun juuret}) ratkaisut ovat
\[
z_k = \sqrt[n]{r}\left[\cos\left(\frac{\varphi}{n} + k \cdot \frac{2 \pi}{n}\right) + 
 i \sin\left(\frac{\varphi}{n} + k\cdot\frac{2\pi}{n}\right)\right], \quad k = 0 \ldots n-1.
\]
Lähtien perusjuuresta $z_0$ juuret sijaitsevat tasavälein kompleksitason ympyrällä, jonka säde
$= \sqrt[n]{\abs{a}}$. Kuvassa on tapaus $n=8$.
\begin{figure}[H]
\setlength{\unitlength}{1cm}
\begin{center}
\begin{picture}(8,8)(-4,-4)
\put(-4,0){\vector(1,0){8}} \put(3.5,-0.5){$\text{Re}\,z$}
\put(0,-4){\vector(0,1){8}} \put(0.2,3.8){$\text{Im}\,z$}
\put(0,0){\bigcircle{6}}
\put(0,0){\vector(2,1){2.62}} \put(2.57,1.26){$\bullet$} \put(2.85,1.45){$z_0$}
\put(0,0){\vector(1,3){0.92}} \put(0.85,2.74){$\bullet$}
\put(0,0){\vector(-1,2){1.32}} \put(-1.45,2.58){$\bullet$}
\put(0,0){\vector(-3,1){2.82}} \put(-2.94,0.84){$\bullet$}
\put(0,0){\vector(-2,-1){2.62}} \put(-2.77,-1.46){$\bullet$}
\put(0,0){\vector(-1,-3){0.92}} \put(-1.05,-2.94){$\bullet$}
\put(0,0){\vector(1,-2){1.32}} \put(1.25,-2.78){$\bullet$}
\put(0,0){\vector(3,-1){2.82}} \put(2.74,-1.04){$\bullet$}
\put(0,0){\arc{2}{-0.464}{0}} \put(1.2,0.2){$\varphi/n$}
\put(0,0){\arc{1}{-1.249}{-0.464}} \put(0.4,0.55){$\frac{2\pi}{n}$}
\end{picture}
\end{center}
\end{figure}
\jatko \begin{Exa} (jatko) Tässä $a = -1 = 1 \vkulma{\pi}\,$, siis $r=1$, $\varphi = \pi$. 
Juuret ovat
\[
z_k = \cos\left(\frac{\pi}{3}+ k \cdot \frac{2\pi}{3}\right) + i \sin\left(\frac{\pi}{3}
                              + k \cdot \frac{2\pi}{3}\right), \quad k = 0,1,2,
\]
\begin{multicols}{2} \raggedcolumns
\begin{align*}
\text{eli} \quad &\left\{ \begin{array}{ll}
z_0 = \dfrac{1}{2} + \dfrac{\sqrt{3}}{2} i, \\
z_1 = -1, \\
z_2 = \dfrac{1}{2} - \dfrac{\sqrt{3}}{2} i. \quad \loppu
\end{array} \right.
\end{align*}
\begin{figure}[H]
\setlength{\unitlength}{1cm}
\begin{center}
\begin{picture}(6,6)(-3,-3)
\put(-3,0){\vector(1,0){6}} \put(2.5,-0.5){$\text{Re}\,z$}
\put(0,-3){\vector(0,1){6}} \put(0.2,2.8){$\text{Im}\,z$}
\Thicklines
\put(2,0){\line(0,-1){0.3}} \put(1.9,-0.8){$1$}
\put(-2,0){\line(0,-1){0.3}} \put(-2.43,-0.8){$-1$}
\thinlines
\put(0,0){\bigcircle{4}}
\put(-2.1,-0.1){$\bullet$} \put(-1.8,0.2){$z_1$}
\put(1.1,1.47){$\bullet$} \put(1.2,1.7){$z_0$}
\put(1.0,-1.76){$\bullet$} \put(1.1,-2){$z_2$}
\put(0,0){\vector(3,4){1.2}}
\put(0,0){\vector(-1,0){2}}
\put(0,0){\vector(2,-3){1.1}}
\put(0,0){\arc{1.4}{-0.93}{0}} \put(0.7,0.3){$\frac{\pi}{3}$}
\put(0,0){\arc{1}{-3.14}{-0.93}} \put(-0.6,0.8){$\frac{2\pi}{3}$}
\end{picture}
\end{center}
\end{figure}
\end{multicols}
\end{Exa}
\begin{Exa} $\sqrt{i} =\,?$
\end{Exa}
\ratk
Kyseessä on yhtälön $z^2 = i$ ratkaiseminen. Koska $\,i = 1\vkulma{\pi/2}\,$, niin ratkaisut
ovat
\begin{align*}
z_k              &= \cos\left(\frac{\pi}{4} + k \pi\right) + i \sin\left(\frac{\pi}{4}
                                                           + k \pi\right), \quad k=0,1 \\
\impl \ \sqrt{i} &= \pm \frac{1}{\sqrt{2}}(1+i).
\end{align*}
\pain{Tarkistus:} 
\[
(\sqrt{i})^2 = \frac{1}{2}(1+i)^2 = \frac{1}{2}(1-1+2i) = i. \quad \text{OK!} \loppu
\]
\begin{Exa} Ratkaise yhtälö $\ z^2+(2+2i)z-i=0$. \end{Exa}
\ratk Toisen asteen yhtälön ratkaisukaava perustuu vain kunta-algebraan, joten se on pätevä 
myös kompleksikertoimiselle polynomille. Ratkaisut siis ovat
\[
z_{1,2} = -(1+i) \pm\sqrt{(1+i)^2+i} = -(1+i)\pm\sqrt{3i},
\]
eli edellisen esimerkin perusteella
\begin{align*}
z_1 &= -(1+i) + \sqrt{\frac{3}{2}}\,(1+i) = \left(\sqrt{\frac{3}{2}}-1\right)(1+i), \\
z_2 &= -(1+i) - \sqrt{\frac{3}{2}}\,(1+i) = -\left(\sqrt{\frac{3}{2}}+1\right)(1+i). \loppu
\end{align*}

\Harj
\begin{enumerate}

\item
Määritä seuraavien funktioiden kaikki juuret kompleksitasossa jakamalla funktiot reaali- 
ja imaginaariosiin. \newline
a) \ $f(z)=z^2+i \quad$ 
b) \ $f(z)=z^3+8i \quad$ 
c) \ $f(z)=z^2+4\overline{z}-1$

\item
Jos $\re z=\im z=a$, niin millä $a$:n arvoilla pätee $\abs{z-i}<\abs{z-3}$\,? Kuvio!

\item
Tutki, millaiset kompleksitason pistejoukot tulevat määritellyiksi seuraavilla ehdoilla. 
Piirrä kuviot. \newline
a) \ $\abs{z+1+i}^2=2 \quad$ 
b) \ $\abs{z-i} \le 2 \quad$ 
c) \ $\abs{z+i}=\abs{z-1-i}$ \newline
d) \ $\text{Re}[(z-i)/(z+i)]=0$

\item
Jaa polynomi enintään toisen asteen reaalikertoimisiin tekijöihin: \newline
a) \ $z^3+2z^2+3z+2 \quad$ 
b) \ $z^4+2 \quad$ 
c) \ $z^4+2z^3+z^2-2z-2$ \newline
d) \ $z^8-256 \quad$ 
e) \ $(z^4+1)^2-z^4$

\item
Määritä a) kaikki reaalikertoimiset, b) kaikki polynomit $p$ astetta $\le 5$, joilla on
ominaisuudet: $p(0)=1$ ja $p$:n nollakohdat ovat $1$, $i$ ja $-i$.

\item
Määritä seuraavien juurien kaikki arvot: \newline
a) \ $\sqrt[4]{-4} \quad$ 
b) \ $\sqrt[6]{-64} \quad$
c) \ $\sqrt[3]{i-1} \quad$
d) \ $\sqrt{3+4i} \quad$ 
e) \ $\sqrt{-7+24i}$

\item
Juuren $\sqrt[6]{2+3i}\,$ eräs likiarvo on $1.2217+0.2019i$. Piirrä kuva, jossa kaikki juuren
arvot on sijoitettu kompleksitasoon (laskematta erikseen muiden juurien likiarvoja).

\item
Määritä seuraavien yhtälöiden kaikki ratkaisut perusmuodossa $x+iy$. \newline
a) \ $z^2+2iz-i-1=0 \quad$ 
b) \ $z^2-4iz-4+i=0 \quad$
c) \ $z^2-(3+5i)z=4-7i$ \newline
d) \ $z^4-2z^2+4=0 \quad $ 
e) \ $z^4+(1-2i\sqrt{3})z^2-3-i\sqrt{3}=0$

\item
Määritä $a\in\C$ siten, että $z=1+i$ on polynomin $p(z)=z^3+az+1$ juuri, ja laske ja sen
jälkeen muut juuret.

\item (*) a) Juuret $\sqrt[4]{3+4i}\,$ on mahdollista laskea tarkasti perusmuodossa
$z_k=x_k+iy_k\,$ siten, että $x_k$ ja $y_k$ ovat geometrisia lukuja. Laske! \vspace{1mm}\newline
b) Määritä tarkasti ne kompleksiluvut, joiden viides potenssi $=1$. Piirrä kuvio lukujen
sijainnista kompleksitasossa.

\item (*)
Olkoon $p(z)=c_n z^n + \ldots + c_0$ reaalikertoiminen polynomi, $c_n \neq 0$. Näytä, että 
polynomilla on ainakin yksi reaalinen nollakohta, jos joko $n$ on pariton tai $c_n c_0 \le 0$.

\item (*)
Todista, että jos kahdella polynomilla astetta $\le n$ on samat arvot $n+1$ eri kompleksitason
pisteessä, niin polynomit ovat samat.

\item (*) \label{H-III-3: avaintulos}
Olkoon $p(z)$ kompleksimuuttujan polynomi, joka ei ole vakio, ja olkoon $c\in\C$ ja 
$p(c)=a\in\C$. Näytä: \vspace{1mm}\newline
a) On olemassa $m\in\N$ ja $b\in\C,\ b \neq 0$ sekä polynomi $q(z)$ siten, että 
\[
p(z)=a+b(z-c)^m+(z-c)^{m+1}q(z)\ \ \forall z\in\C.
\]
b) Jos a)-kohdan hajotelmassa on $a\neq 0$ ja $q(z)=0$, niin $\abs{p(z)}$ pienenee johonkin
suuntaan $c$:stä lähdettäessä, t.s.\ on olemassa $\rho\in\C,\ \abs{\rho}=1$ ja $\delta>0$
siten, että
\[
\abs{p(c+t\rho)} < \abs{p(c)}, \quad \text{kun}\ 0<t<\delta.
\]
c) Jos a)-kohdan hajotelmassa on $a \neq 0$ ja $q(z) \ne 0$, niin b)-kohdan väittämä on
edelleen tosi. \newline
d) Jos $\abs{p(z)}$ saavuttaa paikallisen minimiarvon $c$:ssä, niin $p(c)=a=0$.

\end{enumerate}