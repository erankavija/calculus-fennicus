\section{Äärettömät desimaaliluvut} \label{desimaaliluvut}
\alku
\index{zyzy@ääretön desimaaliluku|vahv}

Matematiikan arkipäivässä hyvin yleisesti kohdattava lukujono on
\kor{ääretön desimaaliluku}\footnote[2]{Vaihtoehtoinen nimitys on \kor{päättymätön}
desimaaliluku. \index{pzyzy@päättymätön desimaaliluku|av}} (engl.\ infinite decimal). Tämä on
ilmiasultaan merkkijono muotoa
\[
e d_{-m} d_{-m+1} \ldots d_0 . d_1 d_2 d_3 \ldots\ =\ x_0.d_1 d_2 d_3 \ldots\,,
\]
missä
\index{desimaali, -luku, -piste} \index{kokonaisluvut!a@kokonaislukuosa (desim.-luvun)}%
\begin{itemize}
\item[-] $\quad$ '$e$' on \kor{etumerkki}, joko $e = +$ tai $e = -\ $ (\ 'tyhjä' = +\,)
\item[-] $\quad$ '$.$' on \kor{desimaalipiste} 
                           (tai 'desimaalipilkku', $d_0$:n ja $d_1$:n välissä)
\item[-] $\quad \ \ m\in\N\cup\{0\}$
\item[-] $\quad \ \ x_0\ =\ e d_{-m} d_{-m+1} \ldots d_0\,\in\,\Z\ =$\ \ desimaaliluvun 
                    \kor{kokonaislukuosa}
\item[-] $\quad \ \ d_n,\,\ n=1,2,\ldots\ =$\ \ desimaaliluvun \kor{desimaalit}
                    ($d_n\,\in\,\{0, \ldots, 9\}$)
\end{itemize}
\begin{Exa}
\begin{align*}
141&.42135623730950488 \ldots \qquad (\,e = +,\ m = 2,\ x_0 = 141\,) \\
- 0&.00314159265358979 \ldots \qquad (\,e = -,\ m = 0,\ x_0 =0\,) \qquad \loppu
\end{align*}
\end{Exa}

Ääretön desimaaliluku on siis pääosin numerojono, jossa numeroiden lisäksi on erillinen 
etumerkki ja desimaalipiste, jonka sijainti numerojonossa määrittää kokonaislukuosan $x_0$. 
Tällaisen merkkijonon \pain{tulkitaan} \pain{tarkoittavan} rationaalilukujonoa
\[
\{\,x_0, x_1, \ldots\,\} = \{\,x_n\,\}_{n=0}^{\infty}\,,
\]
missä $x_0$ on kokonaislukuosa ja
\[
x_n\ =\ x_0 \pm \bigl(\, d_1 \cdot 10^{-1} + \cdots + d_n \cdot 10^{-n}\,\bigr)
     =\ x_0 \pm \sum_{k=0}^n d_k \cdot 10^{-k}, \quad n\in\N,
\]
missä summalausekkeen etumerkki valitaan $e$:n mukaan. Tulkinnan mukaisesti ääretön
desimaaliluku on siis itse asiassa \pain{sar}j\pain{a}, jonka ensimmäinen termi on $x_0$ ja
muut $\pm d_k \cdot 10^{-k}$, $k=1,2,\ldots\,$ Sarjan osasummat ovat indeksistä $n=1$
lähtien nk.\ 
\index{zyzy@äärellinen desimaaliluku}%
\kor{äärellisiä} (tai päättyviä, ks.\ alaviite) \kor{desimaalilukuja}, jotka jatkossa merkitään
\[
x_n\ =\ x_0. d_1 \ldots d_n \quad (n\in\N).
\]
Tällä tarkoitetaan siis lukua $\,\pm(|x_0| + \sum_{k=0}^n d_k \cdot 10^{-k})$, missä etumerkki
on erikseen asetettu (sama kuin $x_0$:ssa, jos $x_0 \neq 0$). Kirjoittamalla luku muotoon
\[
x_n\ =\ \pm 10^{-n} \left(|x_0| \cdot 10^n + \sum_{k=0}^n d_k \cdot 10^{n-k}\right)
\]
nähdään, että $x_n=p/10^{n}$, missä $p = e d_{-m} \ldots d_0 d_1 \ldots d_n \in \Z$ on 
kokonaisluku. Toisaalta jos lähtökohtana on luku $x=p/10^n,\ p\in\Z$, niin esittämällä $p$ 
kymmenjärjestelmässä nähdään, että $x$ on saatettavissa muotoon $x = x_0.d_1 \ldots d_n$.
Siis ko.\ muotoa olevat äärelliset desimaaliluvut muodostavat täsmälleen $\Q$:n osajoukon 
\[
\Q_n = \{\,p/10^n \mid p \in \Z\,\}.
\]
Kun tämän mukaisesti asetetaan $\Q_0=\Z$, niin pätee
\[
\Z = \Q_0 \subset \Q_1 \subset \Q_2 \subset \ldots \subset \Q.
\]
Lukujoukossa $\Q_n$ luvut ovat tasavälein siten, että peräkkäisten lukujen $x,y$ etäisyys on
$|x-y|=10^{-n}$. Kuvassa on lukujoukkoja $\Q_0,\Q_1,\Q_2$ 
havainnollistettu geometrisesti.\footnote[2]{Lukujen 'geometrisointia' tarkastellaan myöhemmin 
Luvussa \ref{geomluvut}. Tässä vaiheessa pidetään geometrisen havainnollistamisen perusideoita 
tunnettuina.}
\begin{figure}[H]
\setlength{\unitlength}{1cm}
\begin{center}
\begin{picture}(12,3)(0,0)
\path(0,0)(12,0)
\path(0,1)(12,1)
\path(0,2)(12,2)
\multiput(1,0)(5,0){3}{\line(0,1){0.2}}
\multiput(1,1)(5,0){3}{\line(0,1){0.2}}
\multiput(0,1)(0.5,0){25}{\line(0,1){0.1}}
\multiput(1,2)(5,0){3}{\line(0,1){0.2}}
\multiput(0,2)(0.5,0){25}{\line(0,1){0.1}}
\multiput(0,2)(0.05,0){241}{\line(0,1){0.05}}
\put(12.5,0){$\Q_0$}
\put(12.5,1){$\Q_1$}
\put(12.5,2){$\Q_2$}
\put(5.9,-0.5){$0$} \put(0.6,-0.5){$-1$} \put(10.9,-0.5){$1$}
\end{picture}
\end{center}
\end{figure}
Jatkossa äärettömiin desimaalilukuihin viitataan joko tavanomaisilla lukusymboleilla
$x,y,z$ jne., tai vaihtoehtoisesti symboleilla $\x,\y,\z$ jne., kun halutaan tehdä ero 
'oikeisiin' lukuihin, erityisesti rationaalilukuihin. Äärettömien desimaalilukujen joukkoa
merkitään symbolilla $\DD\,$:
\[
\DD = \{\text{äärettömät desimaaliluvut}\}.
\]
Yksittäisen desimaaliluvun tapauksessa pidetään merkkijonoesitystä ko.\ desimaaliluvun 
'nimenä', jolloin merkinnässä
\[
\x\ =\ x_0. d_1 d_2 \ldots\ =\ \{\,x_n\,\}_{n=0}^{\infty} 
\]
ensimmäinen '$=$' nimeää ja toinen tulkitsee (antaa merkityssisällön).
\jatko  \begin{Exa} (jatko) Tehdyin merkintäsopimuksin voidaan kirjoittaa
\begin{align*}
x\ &=\ 141.421356 \ldots \\ 
   &=\ \{\,141,141.4,141.42, \ldots\,\} \\
   &=\ \{\,141,\ \dfrac{1414}{10}\,,\ \dfrac{14142}{100}\,,\ \ldots\,\} \\
   &=\ \{\,x_0,\,x_1,\,x_2,\,\ldots\,\} \\[3mm]
y\ &=\ -0.00314159 \ldots \\
   &=\ \{\,0,\,0.0,\,0.00,\,-0.003,\,-0.0031,\, \ldots\,\} \\
   &=\ \{\,0,\ 0,\ 0,\ -\dfrac{3}{1000}\,,\ -\dfrac{31}{10000}\,,\ \ldots\,\} \\
   &=\ \{\,y_0,\,y_1,\,y_2,\,\ldots\,\} \quad \loppu
\end{align*}
\end{Exa}

Jokainen ääretön desimaaliluku $x\in\DD$ on siis merkitykseltään lukujono 
$\seq{x_n}_{n=0}^\infty\,$, missä $x_n\in\Q_n\ \forall n$. Määritelmänsä mukaisesti lukujono 
$\seq{x_n}$ on määrättävissä asettamalla ensin $x_0\in\Z$ ja sitten $x_1,x_2, \ldots\,$ 
palautuvasti muodossa
\[
x_n\ =\ \begin{cases}
         x_{n-1} + d_n \cdot 10^{-n}, \quad \text{jos $e\ =\ +$}\,, \\
         x_{n-1} - d_n \cdot 10^{-n}, \quad \text{jos $e\ =\,\ $ --}\ ,
        \end{cases}
\]
missä $e$ on $x$:n etumerkki ja $d_n\in\{0, \ldots ,9\}\ \forall n\in\N$. Tämän mukaan on
joko $x_n \ge x_{n-1}\ \forall n\in\N\,$ tai $\,x_n \le x_{n-1}\ \forall n\in\N$, riippuen 
etumerkistä $e$. Tällaiset 'yksitoikkoiset' jonot ovat huomattavan tärkeä lukujonojen
erikoistapaus, joten asetetaan niitä varten
\begin{Def} \label{monotoninen jono} \index{lukujono!c@monotoninen|emph}
\index{monotoninen!a@lukujono|emph}
\index{aidosti kasvava, vähenevä, monotoninen!a@lukujono|emph}
Lukujono $\{\,a_n,\ n=m,m+1, \ldots\,\}$ on
\begin{itemize}
\item[-] \kor{kasvava},\ \ \ \ jos $\ a_{n+1} \ge a_n\ \ \forall n \ge m$
\item[-] \kor{vähenevä},\ \    jos $\ a_{n+1} \le a_n\ \ \forall n \ge m$
\end{itemize}
Lukujono on \kor{aidosti} kasvava, jos $\,a_{n+1} > a_n\ \forall n \ge m$ ja aidosti vähenevä,
jos $\,a_{n+1} < a_n\ \forall n \ge m$. Jos lukujono on (aidosti) kasvava tai vähenevä, se on 
(aidosti) \kor{monotoninen}.
\end{Def}
Ääretön desimaaliluku on siis lukujonona monotoninen: kasvava etumerkin ollessa $e=+$ ja 
vähenevä, kun etumerkki on $e=-$. Seuraavan tuloksen mukaan tämä kasvu tai väheneminen kuitenkin
'hiipuu' nopeasti $n$:n kasvaessa.
\begin{Prop} \label{desim} Jos $x = \{\,x_n\,\}_{n=0}^{\infty} \in \DD$, niin jokaisella
$k\in\N\cup\{0\}$ ja jokaisella $n\in\N$, $n>k$ pätee $\abs{x_n-x_k}<10^{-k}$, tarkemmin
\[
\abs{x_n - x_k}\ \le\ 10^{-k}-10^{-n}.
\]
\end{Prop}
\tod Kun $n > k \ge 0$, on
\begin{align*}
\abs{x_n - x_k}\ =\ \sum_{j=k+1}^n d_j \cdot 10^{-j}\
                 &\le\ \sum_{j=k+1}^n 9 \cdot 10^{-j} \\
                 &=\ 9 \cdot 10^{-k-1} \sum_{i=0}^{n-k-1} \Bigl(\frac{1}{10}\Bigr)^i \\
                 &=\ 10^{-k} \cdot \dfrac{9}{10} \cdot 
                                    \dfrac{1 - (\frac{1}{10})^{n-k}}{1-\frac{1}{10}} \\
                 &=\ 10^{-k} - 10^{-k}\left(\frac{1}{10}\right)^{n-k} \\[2mm] 
                 &=\ 10^{-k}-10^{-n}. \loppu
\end{align*}
Kun Proposition \ref{desim} tuloksen ohella huomioidaan lukujonon $\seq{x_n}$ monotonisuus, niin
nähdään, että jokaiselle äärettömälle desimaaliluvulle $\x=\seq{x_n}$ pätee
\[
0 \le k < n \qimpl \begin{cases} 
                   \,x_k \le x_n < x_k + 10^{-k}, \quad &\text{jos etumerkki $e\ =\ +$}\,, \\
                   \,x_k - 10^{-k} < x_n \le x_k, \quad &\text{jos etumerkki $e\ =\,\ $ --}\,.
                   \end{cases}
\]
Tämän mukaan termien $x_n$ liikkumavara supistuu nopeasti indeksin $n$ kasvaessa.

\subsection{Jakokulma-algoritmi}
\index{jakokulma-algoritmi|vahv}%

Edellä on todettu, että äärettömät desimaaliluvut ovat lukujonoina $\seq{x_n}$ sekä monotonisia
että 'hiipuvia'. Voi myös käydä niin, että jono $\{x_n\}$ ei ainoastaan 'hiivu', vaan 
\kor{lähestyy rationaalilukua} $x \in \Q$ siinä mielessä, että $\abs{x_n-x}$ tulee yhä 
pienemmäksi $n$:n kasvaessa. Milloin tällainen 'lähestyminen' on yleensä mahdollista, 
selvitetään myöhemmissä luvuissa. Tässä vaiheessa asetetaan kysymys toisin päin: Jos on annettu
$x\in\Q$, niin miten löydetään ääretön desimaaliluku $\x=\seq{x_n}$ siten, että $\abs{x_n-x}$ 
tulee yhä pienemmäksi $n$:n kasvaessa\,? Vastauksen antaa tunnettu 
\kor{jakokulma-algoritmi}\footnote[2]{\kor{Algoritmi} on täsmällinen ja toteutuskelpoinen
(ohjelmointikelpoinen) toimintaohje jonkin laskennallisen päämäärän saavuttamiseksi. Jos 
päämääränä on lukujono $\{\,a_n,\ n=1,2, \ldots\,\}$, on algoritmia seuraamalla saatava 
äärellisellä määrällä laskutoimituksia vastaus kysymykseen $a_n=\,$?, olipa indeksi $n \in \N$
mikä hyvänsä annettu luku. \index{algoritmi|av}}. Algoritmissa konstruoidaan desimaalivastine
luvulle $|x|$ ja varustetaan tämä $x$:n mukaisella etumerkillä, joten voidaan rajoittua
tapaukseen $x \ge 0$. Valitaan tällöin $x_n\in\Q_n,\ n=0,1,2,\ldots$ siten, että jokaisella
$n$ toteutuu ehto
\begin{equation} \label{jakoehto 1}
x_n \le x < x_n + 10^{-n}.
\end{equation}
Jos $x$ ei ole äärellinen desimaaliluku, ts.\ $x\not\in\Q_n\ \forall n$, niin ehdon
\eqref{jakoehto 1} mukaisesti on $x_n < x < x_n+10^{-n}\,\ \forall n$. Jos $x\in\Q_k$
jollakin $k$, niin $x_n\in\Q_n$ myös kun $n>k$, jolloin ehdosta \eqref{jakoehto 1}
seuraa, että $x_n=x$ jokaisella $n \ge k$. Molemmissa tapauksissa $x_n+10^{-n}$ on $x$:ää
lähinnä oleva, $x$:ää aidosti suurempi luku joukossa $\Q_n$. On ilmeistä, että näin
konstruoitu lukujono $\seq{x_n}$ on yksikäsitteinen, toteuttaa
\begin{equation}  \label{jakoehto 2}
\abs{x_n - x}\ <\ 10^{-n}, \quad n=0,1, \ldots
\end{equation}
ja määräytyy yksikäsitteisesti myös tästä ehdosta $x$:n etumerkistä riippumatta.

Jos $x \ge 0$, niin ehdon \eqref{jakoehto 1} toteuttavan lukujonon konstruoiminen 
jakokulma-algoritmilla käy seuraavasti: Olkoon $x = r/s$, missä $r\in\N\cup\{0\}$ ja 
$s \in \N$. Tällöin voidaan ensinnäkin kirjoittaa
\[
\dfrac{r}{s} = x_0 + \dfrac{r_1}{s}\,,
\]
missä $x_0 \in \N \cup \{0\}$ ja $r_1 \in \{\,0, \ldots, s-1\,\}$. Luvut $x_0$ ja $r_1$ ovat
ilmeisen yksikäsitteiset. Koska $0\,\le\,r_1/s\,<\,1\ \impl\ 0\,\le\,10\,r_1/s\,<\,10$, on 
edelleen löydettävissä yksikäsitteiset $d_1 \in \{\,0, \ldots, 9\,\}$ ja 
$r_2 \in \{\,0, \ldots, s-1\,\}$ siten, että
\[
\dfrac{10\,r_1}{s} = d_1 + \dfrac{r_2}{s}\,.
\]
Jatkamalla näin saadaan yleiseksi algoritmiksi
\[
\boxed{\quad \dfrac{10\,r_k}{s} = d_k + \dfrac{r_{k+1}}{s}, \quad 
                     k = 1,2, \ldots \quad \quad \text{(jakokulma-algoritmi)} \quad}
\]
Tässä $d_k \in \{\,0, \ldots, 9\,\}$ ja luvut $r_{k+1} \in \{\,0, \ldots, s-1\,\}$ ovat 
\index{jakojäännös}%
nk.\ \kor{jakojäännöksiä}. Jos algoritmi katkaistaan indeksiin $k=n$, on luvulle $x$ saatu
esitysmuoto
\begin{align*}
x\ =\ x_0 + \dfrac{r_1}{s}\ 
  &=\ x_0 + d_1 \cdot 10^{-1} + 10^{-1} \cdot \dfrac{r_2}{s} \\ 
  &\ \ \vdots \\
  &=\ (\,x_0 + d_1 \cdot 10^{-1} + \ldots + d_n \cdot 10^{-n}\,) 
                                          + 10^{-n} \cdot \dfrac{r_{n+1}}{s} \\
  &=\ x_n + 10^{-n} \cdot \dfrac{r_{n+1}}{s}\ .
\end{align*}
Tässä on $\ 0\,\le\,10^{-n}\,r_{n+1}/s\,<\,10^{-n}$, joten ehto \eqref{jakoehto 1} toteutuu
algoritmin määrittämälle lukujonolle $\{\,x_0,\,x_0.d_1,\,x_0.d_1d_2,\,\ldots\,\} = \seq{x_n}$.

Jos $\x\in\DD$ on jakokulma-algoritmin tuottama desimaalilukuvastine rationaaliluvulle $x$, on
tapana muitta mutkitta kirjoittaa $x = \x$, eli \pain{samastaa} $x$ ja $\x$. Jatkossa
noudatetaan tätä sopimusta. Tämän mukaisesti siis jokaiselle rationaaliluvulle luotu 'kopio'
äärettömien desimaalilukujen joukossa, eli pätee $\DD \supset \Q$.
\begin{Exa} Tehdyn sopimuksen mukaan
\begin{align*}
50/3    &= 16.66666666666666666666666666666666666666 \ldots\,\in \DD, \\
311/125 &= 2.488000000000000000000000000000000000000 \ldots\,\in \DD, \\
245/17  &= 14.41176470588235294117647058823529411764 \ldots\,\in \DD. \quad\loppu 
\end{align*} \end{Exa}
Esimerkissä luku $311/125$ on äärellinen desimaaliluku. Tällaisessa tapauksessa jakojäännös
$r_k = 0$ jollakin $k$ (tässä kun $k=4$), jolloin algoritmi antaa $r_n = d_n = 0$ kun $n \ge k$.
\index{jaksollinen desimaaliluku}%
Muissakin esimerkeissä algoritmin antama desimaaliluku on \kor{jaksollinen}, eli jollakin
$k,m \in \N$ muotoa
\[
\x\ =\ x_0.d_1 d_2 \ldots d_k d_{k+1} \ldots d_{k+m} d_{k+1} \ldots d_{k+m} \ldots,
\]
jolloin jaksollisuus alkaa indeksistä $k+1$ ja jakson pituus $=m$. Jakokulma-algoritmi antaa 
itse asiassa \pain{aina} jaksollisen tuloksen, sillä koska jakojäännöksellä $r_n$ on vain $s-1$
erilaista $0$:sta poikkeavaa arvoa, niin jollakin $n=k+1\in \{\,1, \ldots,s\,\}$ ja 
$m \in \{\,1, \ldots, s-1\,\}$ on oltava $r_{n+m} = r_{n}$. Tällöin algoritmi antaa 
$r_{n+m} = r_n$ ja $d_{n+m} = d_n$ myös kun $n > k+1$, jolloin tulos on ym.\ jaksollista muotoa.
Jaksollisuus alkaa siis viimeistään indeksistä $n=s$ ja jakson pituus on enintään $m=s-1$. 

Jatkossa merkitään jaksollisten desimaalilukujen joukkoa symbolilla $\DD_p\,$:
\[
\DD_p = \{\text{jaksolliset desimaaliluvut}\}.
\]
\index{jaksoton desimaaliluku}%
Jos $x\in\DD$ ja $x\not\in\DD_p$, niin sanotaan, että $x$ on \kor{jaksoton}.

\subsection{Äärettömät binaariluvut}
\index{zyzy@ääretön desimaaliluku!48@ääretön binaariluku|vahv}%

\kor{Äärettömillä binaariluvuilla} tarkoitetaan merkki/lukujonoja muotoa
\[
x\ =\ eb_m b_{m+1} \ldots b_0 . b_1 b_2 \ldots\ =\ \{x_n\}_{n=0}^{\infty},
\]
\index{bitti}%
missä $b_n$:t ovat binaarijärjestelmän numeroita eli \kor{bittejä}, $b_n \in \{\,0,1\,\}$, ja
\[
x_n\ =\ x_0 \pm \sum_{k=1}^n b_k \cdot 2^{-k},
\]
missä $x_0 = e b_m b_{m+1} \ldots b_0$ on kokonaislukuosa (binaarimuodossa) ja summalausekkeen
etumerkki on sama kuin $x$:ssä. Jos lukujärjestelmän kantalukuna on $8$ tai $16$, puhutaan 
\index{oktaaliluku} \index{heksadesimaaliluku}%
vastaavasti (äärettömistä) \kor{oktaali}- ja \kor{heksadesimaaliluvuista}. Tietokoneiden ja 
laskinten käyttäminä lukujärjestelminä nämä mainitut ovat tavallisia. --- Kaikki mitä edellä
on sanottu äärettömistä desimaaliluvuista, on helposti muunnetavissa mainittuja (ja muitakin)
lukujärjestelmiä koskevaksi.

\subsection{Desimaaliluvun skaalaus}
\index{skaalaus (desimaaliluvun)|vahv}

Toistaiseksi äärettömille desimaaliluvuille ei määritellä muita laskuoperaatioita kuin
\kor{merkin vaihto} ja \kor{skaalaus} = kertominen luvuilla $10^m,\ m\in\Z$. Skaalaus
luvulla $10^m$ tarkoittaa yksinkertaisesti desimaalipisteen paikan siirtämistä joko $m$
desimaalia eteenpäin ($m>0$) tai $|m|$ numeroa taaksepäin ($m<0$). Jälkimmäisessä tapauksessa
lisätään kokonaislukuosaan tarvittaessa riittävä määrä 'etunollia'. Luvulla $10^0=1$ 
skaalattaessa jää desimaaliluku ennalleen. Skaalaukselle pätevät vaihdanta- ja
liitäntälait 
\[
10^m \cdot (10^n \cdot x) \,=\, 10^{m+n} \cdot x 
                          \,=\, 10^n \cdot (10^m \cdot x), \quad m,n\in\Z.
\]
Kun merkin vaihto tulkitaan samaksi kuin kertominen luvulla $-1$ (kuten kunta-algebrassa),
niin merkin vaihto ja skaalaus yhdistämällä tulee määritellyksi skaalaus luvuilla
$\pm 10^m,\ m\in\Z$. 
\jatko\jatko \begin{Exa} (jatko)
\begin{align*}
 10^{-4} \cdot 141.421356 \ldots\  &=\ 0.0141421356 \ldots \\
 -10^4 \cdot (-0.00314159 \ldots)\ &=\ 31.4159 \ldots
\end{align*}
\end{Exa} \seur

\subsection{Liukuluvut}
\index{liukuluku|vahv}%

\kor{Liukuluvuiksi} (engl.\ floating point number, ruots.\ flyttal) sanotaan 
kymmenjärjestelmässä äärellisiä desimaalilukuja muotoa
\[
\pm 10^{\beta} \cdot 0 . d_1 \ldots d_n,
\]
missä $\beta\in\Z$ ja $n\in\N$ on \pain{kiinteä}. Numeerisessa laskennassa liukulukuja
käytetään äärettömien desimaalilukujen \kor{approksimaatioina}, eli likimääräisinä
vastineina. Approksimointi tapahtuu kirjoittamalla desimaaliluku ensin skaalaustekijän
avulla muotoon $x=\pm 10^\beta \cdot 0.d_1d_2..$, missä $\beta\in\Z$ ja $d_1 \neq 0$
(mahdollista aina kun $x \neq 0$) ja katkaisemalla näin saatu luku desimaaliin $d_n$.
Sanotaan, että näin saadut numerot $d_1 \ldots d_n$ ovat desimaaliluvun $n$ ensimmäistä 
\index{merkitsevät numerot}%
\kor{merkitsevää numeroa} (engl.\ significant digits), ja että ko.\ liukuluku
on desimaaliluvun
\index{katkaistu desimaaliluku}%
\kor{katkaistu} (engl.\ truncated) muoto. Katkaisun vaihtoehtona 
\index{pyzz@pyöristys (desimaaliluvun)}%
approksimaatiossa voidaan käyttää \kor{pyöristystä} (engl. rounding), yleensä lähimpään
samaa tyyppiä olevaan liukulukuun (nk.\ normaalipyöristys). Tieto desimaalipisteen paikasta
suhteessa ensimmäiseen merkitsevään numeroon sisältyy lukuun $\beta \in \Z$. Kun ääretön
desimaaliluku ilmoitetaan $n$ \kor{merkitsevän numeron tarkkuudella}, tarkoitetaan luvun
(katkaisutua tai pyöristettyä) liukulukuapproksimaatiota.
\begin{Exa} Yhdeksän ($n=9$) merkitsevän numeron tarkkuudella on
\begin{align*}
0.0001414213562373095   \ldots \quad \approx \quad\, 10^{-3} &\cdot 0.141421356 \\
-172/3\ = \ -57.333333333 \ldots \quad \approx \quad   -10^2 &\cdot 0.573333333 
\end{align*} 
Tässä katkaisu ja normaalipyöristys antavat saman tuloksen. \loppu
\end{Exa}

Tietokoneet ja laskimet operoivat numeerisissa laskuissa aina liukuluvuilla, esim.\ 
binaarijärjestelmään perustuvilla. Tällöin sekä merkitsevien numeroiden määrä että $\beta$:n
sisältämien numeroiden määrä on rajoitettu koneelle ominaisesti. Mahdollisia
\index{koneluku}%
\kor{konelukuja} (engl.\ machine numbers) on siten aina äärellinen määrä.
\begin{Exa} Koneluvut ovat muotoa $\ \pm 2^{\beta} \cdot 0 .\,b_1 \ldots b_{30}$, missä 
$\beta = \pm \beta_1 \ldots \beta_{10}$, $b_k, \beta_k \in \{\,0,1\,\}$. Yhden luvun 
tallettamiseen tarvitaan $42$ bittiä (numerot ja etumerkit), ja erilaisia lukuja on
\[
2^{30} \cdot (2^{11} - 1) + 1\ =\ 2197949513729\ \ \text{kpl.} \quad \quad \loppu
\]
\end{Exa}

Tietokoneen/laskimen suorittamia peruslaskuoperaatioita (yhteen-, vähennys-, kerto- ja 
jakolasku) koneluvuilla sanotaan \kor{liukulukuoperaatioiksi} (engl.\ floating point operation,
lyh.\ flop). Nämä poikkeavat hiukan tavanomaisista laskuoperaatioista, koska tulos on yleensä
(etenkin kerto- ja jakolaskuissa) katkaistava tai pyöristettävä koneluvuksi. Laskutoimitusten
\index{pyzz@pyöristys (desimaaliluvun)!pyöristysvirhe}%
likimääräisyydestä aiheutuvat \kor{pyöristysvirheet} (engl.\ roundoff errors) voivat vahvistua
tai kasautua peräkkäisissä laskutoimituksissa. Yksittäisessäkin laskutoimituksessa voi tapahtua
kohtalokas
\index{merkitsevät numerot!merkitsevien numeroiden kato}%
\kor{merkitsevien numeroiden kato}, kuten seuraava yksinkertainen esimerkki osoittaa.
\begin{Exa} Olkoon $a$ ja $b$ annettuja lukuja. Halutaan laskea
\[
x=\frac{1}{a}\left(1-\frac{1}{1+b}\right).
\]
Mikä arvo saadaan $x$:lle, jos $a=10^{-20},\ b=2 \cdot 10^{-20}$, ja koneluvut ovat muotoa 
$\pm 10^{\beta} \cdot 0 . d_1 \ldots d_{15}$ (kymmenjärjestelmä), missä $-99 \le \beta \le 99$\,?
\end{Exa}
\ratk Suoraan annettuun lausekkeeseen perustuva algoritmi on:
\[
y=1+b,\ \ z=1/y,\ \ u=1-z,\ \ x=u/a.
\]
Koska $y$:n tarkka arvo on $\,y = 1.00000000000000000002$,
niin kone sijoittaa muistipaikkaan '$y$' tämän luvun katkaistun (tai pyöristetyn)
liukulukumuodon $y = 10^1 \cdot 0.100000000000000 = 1$. Laskun muut vaiheet ovat tämän jälkeen
virheettömiä, ja lopputulokseksi saadaan $x=0$. \loppu

Esimerkissä vaarattoman näköinen pyöristys johtaa suureen virheeseen lopputuloksessa. Ongelma 
poistuu, jos laskut suoritetaan 
\index{kaksoistarkkuus}%
\kor{kaksoistarkkuudella} (engl.\ double precision), eli käyttämällä liukulukuesitystä, jossa
talletetaan $15$:n sijasta $30$ merkitsevää numeroa. Tällöin saadaan tulos $x=2$, joka on 
varsin lähellä tarkkaa arvoa $x=1.999999999999999999960000000000000000000799999\ldots$
Kaksoistarkkuuteen (tai useampikertaiseen tarkkuuteen) siirtyminen on yleinen menettely
pyöristysvirheiden uhatessa. Toinen vaihtoehto on algoritmin parempi suunnittelu.
\jatko \begin{Exa} (jatko) Laskettava lauseke yksinkertaistuu kunta-algebran keinoin muotoon
\[
x=\frac{b}{a(1+b)}\,.
\]
Tässä approksimaatio $1+b \approx 1$ ei aiheuta merkittävää virhettä. \loppu
\end{Exa}

\Harj
\begin{enumerate}

\item
a) Näytä, että jos luvuilla $p,q\in\N$ ei ole yhteisiä tekijöitä, niin $x=p/q\in\Q$ on 
äärellinen desimaaliluku täsmälleen kun on olemassa $\,m,n\in\N\cup\{0\}\,$ siten, että 
$q=2^m \cdot 5^n$. \ b) Ilmoita luku $x=(2/5)^{10}$ äärellisenä desimaalilukuna. \ 
c) Millaisia rationaalilukuja ovat äärellisten desimaalilukujen vastineet
lukujärjestelmässä, jonka kantaluku on $60$\,? \ d) Luvulla $1111111/5400$ on
$60$-järjestelmässä äärellistä desimaalilukua vastaava esitysmuoto. Määritä tämä
käyttämällä luvuille $n=0 \ldots 59$ symboleja $(n)$ $60$-järjestelmässä.

\item \label{H-I-5: monotonisten jonojen yhdistely}
Näytä, että jos lukujonot $\seq{a_n}$ ja $\seq{b_n}$ ($n\in\N$) ovat kasvavia, niin \
\mbox{a) myös} lukujono $\seq{a_n+b_n}$ on kasvava, \ b) jos lisäksi $a_n \ge 0\ \forall n$ ja
$b_n \ge 0\ \forall n$, niin lukujono $\seq{a_n b_n}$ on kasvava. \ c) Näytä esimerkillä,
että jos b)-kohdassa lisäehto $b_n \ge 0\ \forall n$ poistetaan, niin lukujono $\seq{a_n b_n}$
voi olla aidosti vähenevä.

\item
Desimaaliluvuista $x=\seq{x_n}\in\DD$ ja $y=\seq{y_n}\in\DD$ tiedetään, että
$x=23.4569 \ldots$ ja $y=23.4567 \ldots$ Kuinka suuri luku $x_n-y_n$ on tämän tiedon
perusteella vähintään ja kuinka suuri enintään, kun $n=10$\,?

\item
Muotoile ja todista Proposition \ref{desim} vastine äärettömille binaariluvuille.

\item
Laske jakokulma-algoritmin avulla seuraavien rationaalilukujen vastineet jaksollisina 
desimaalilukuina:
\ a) $1/11$, \ b) $31/13$, \ c) $19/17$.

\item
Jakokulma-algoritmin mukainen luvun $x\in\Q$ vastine äärettömänä desimaalilukuna olkoon
jaksollinen siten, että jakson pituus $=m$ ja jaksollisuus alkaa desimaalista $d_k$. Anna 
esimerkki luvusta $x=r/s,\ r,s\in\N$, jolle pätee \ 
a) $\,m=1,\ k=5$, \,\ b) $\,m=s-1,\ k=s$, \,\ c) $\,m=s-1,\ k=1$.

\item
Kymmenjärjestelmän luku $7/11$ samastuu $2$-kantaisessa lukujärjestelmässä äärettömään 
binaarilukuun ja $3$-kantaisessa järjestelmässä äärettömään triaarilukuun. Määritä nämä 
jakokulma-algoritmilla. \kor{Huomaa}, että jakojäännöstä ei algoritmissa kerrota kymmenellä vaan
kantaluvulla, siis $2$:lla tai $3$:lla. Jatka laskua, kunnes jaksollisuus ilmenee\,!

\item
Halutaan laskea $y=(x+a)^2-(x+b)^2$ kun $x=1/7,\ a=0.1457 \cdot 10^{-6}$ ja 
$b=0.6973 \cdot 10^{-6}$. \ a) Määritä $y$ neljän merkitsevän numeron tarkkuudella muokkaamalla
$y$:n lauseke ensin sellaiseen muotoon, että merkitsevien numeroiden katoa ei tapahdu. \ b) Jos
käytetään alkuperäistä lauseketta, niin millä tarkkuudella laskut on suoritettava, jotta
päästään mainittuun tarkkuuteen lopputuloksessa\,?

\item
Jos $x=7 \cdot 10^{-8}$, niin monenko merkitsevän numeron tarkkuudella on
\[
\text{a)} \quad \frac{1}{8+3x} \approx 0.125 \qquad \text{b)} \quad (1+x)^3 \approx 1+3x\ ?
\]

\item
Muokkaa seuraavat lausekkeet niin, että alttius merkitsevien numeroiden kadolle poistuu. Laske
sitten lausekkeiden arvot kymmenen merkitsevän numeron tarkkuudella.
\begin{align*}
&\text{a)} \quad 10^{20}\left(\frac{1}{1+x}-\frac{1}{1+2x}\right), \quad 
                                                 x = \frac{6}{7} \cdot 10^{-20} \\
&\text{b)} \quad 10^{40}\left(\frac{1}{1+x}-\frac{1}{1-2x}+3x\right), \quad 
                                                 x = \frac{2}{7} \cdot 10^{-20} \\
&\text{c)} \quad 10^{60}\left(\frac{1+x}{1-x}-\frac{1-x}{1+x}-4x\right), \quad
                                                 x = \frac{5}{7} \cdot 10^{-20} \\
&\text{d)} \quad 10^{80}\left(\frac{1+x}{1-x}+\frac{1-x}{1+x}+\frac{2}{1+2x^2}-4\right), \quad
                                                 x = \frac{3}{7} \cdot 10^{-20} \\
&\text{e)} \quad 10^{120}\left(\frac{1+x}{1-x}+\frac{1-x}{1+x}
                                              -\frac{2}{1-2x^2+2x^4}\right), \quad
                                                 x = \frac{4}{7} \cdot 10^{-20}
\end{align*}

\item (*) \index{zzb@\nim!Mummon nelilaskin} 
(Mummon nelilaskin) Nelilaskin pystyy näyttämään luvun, jossa on kahdeksan numeroa, etumerkki
ja mahdollinen desimaalipiste numeroiden välissä, esim.\ $-0.0000999$. Montako erilaista
rationaalilukua laskin kykenee näyttämään tarkasti?

\item (*) \label{H-I-4.8}
Olkoon $x=\seq{x_n}\in\DD$ ja $y=\seq{y_n}\in\DD$. Näytä, että jos $x_k < y_k$ jollakin
$k\in\N\cup\{0\}$, niin a) $x_n < y_n$ jokaisella $n \ge k$, \ b) $x_n \le y_n$ jokaisella $n$.

\item (*)
Ikuisesti laskeva tietokone tuottaa bittijonon $\{b_n\}_{n=1}^\infty$ seuraavalla algoritmilla:

1. Asetetaan $A=0$, $B=1$, $n=0$.\newline
2. Asetetaan 
\begin{align*}
&n\leftarrow n+1 \text{ (uusi arvo=vanha arvo}+1\text{)} \\
&C=\frac{1}{2}(A+B)\quad\text{ja} \\
&b_n=1\,\ \text{ja}\,\ A=C, \quad \text{jos}\ \ 3 \cdot C \le 1 \\
&b_n=0\,\ \text{ja}\,\ B=C, \quad \text{jos}\ \ 3 \cdot C > 1
\end{align*}
Ilmoitetaan $b_n$ ja palataan kohtaan 2.

a) Millaiset luvut $b_n,$ $n=1\ldots 4,$ kone ilmoittaa? \newline
b) Jos laskun lopputulos (iäisen odottelun jälkeen) tulkitaan äärettömänä binaarilukuna 
$\x=0.b_1b_2\ldots$, niin minkä rationaaliluvun $x$ binaariesitys on kyseessä\,? Perustele\,!

\end{enumerate}