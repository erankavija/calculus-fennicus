\section{Katsaus rationaalilukuihin}  \label{ratluvut}
\alku
\sectionmark{Rationaaliluvut}

\kor{Luonnollisten lukujen} joukkoa merkitsemme
\[
\N = \{1,2,3,4,\ldots\}.
\]
Yleisemmin \kor{joukko} (engl. set) koostuu \kor{alkioista} (engl. element), joita joukon \N\ 
tapauksessa ovat luvut $1$, $2$, $3$ jne. Merkinnät
\[
8 \in \N, \quad \infty \not\in \N
\]
luetaan 'kahdeksan kuuluu \N:ään, 'ääretön ei kuulu \N:ään', tarkoittaen yksinkertaisesti, että
\N:ssä on alkio nimeltä 'kahdeksan' mutta ei alkiota nimeltä 'ääretön'.

Joukko $A = \{1,2,3\}$ on \N\ :n \kor{osajoukko} (engl. subset). Merkitään
\[
A \subset \N \quad \text{tai} \quad \N \supset A
\]
ja luetaan '$A$ kuuluu/sisältyy \N\ :ään' tai '\N\ sisältää $A$:n'. Yleisesti merkintä 
$A \subset B$ tarkoittaa, että jokainen $A$:n alkio on myös $B$:n alkio, toisin sanoen pätee
\[
x \in A \qimpl x \in B.
\]
Tässä '$\impl$' on logiikan symboleihin kuuluva nk. \kor{implikaationuoli}: Merkintä $P \impl Q$
luetaan '$P$:stä seuraa $Q$', tai 'jos $P$, niin $Q$'. Symboli '$\subset$' (kuuluminen 
osajoukkona) määritellään siis symbolin '$\in$' (kuuluminen alkiona) avulla. Jälkimmäinen 
symboli on nk.\ \kor{primitiivi}, jota ei voi enää määritellä muiden symbolien avulla. Joukoista 
puhuttaessa voi siis merkintöjä $x \in A$ tai $x \not\in A$ pitää kaiken loogisen ajattelun 
lähtökohtana. 
\begin{Exa} Jos $A = \{2,1,1,2,1\}$ ja $B = \{1,2\}$, niin $A \subset B$, sillä jos $x \in A$,
niin on joko $x=1$, jolloin $x \in B$, tai $x=2$, jolloin myöskin $x \in B$. Samalla päättelyllä
nähdään, että pätee myös $B \subset A$. \loppu \end{Exa} 
Jos joukoille $A$ ja $B$ pätee sekä $A \subset B$ että $B \subset A$, kuten esimerkissä, niin 
sanotaan, että $A$ ja $B$ ovat (joukkoina) \kor{samat}, ja merkitään $A=B$.\footnote[2]{Jos 
$A \subset B$, ja $A$ ei ole sama kuin $B$, sanotaan että $A$ on $B$:n \kor{aito} 
(engl.\ proper) osajoukko. Joissakin matemaattisissa teksteissä symbolilla '$\subset$' on tämä
rajatumpi merkitys, jolloin mahdollisuus $A=B$ varataan merkinnällä $A \subseteq B$. Tässä 
tekstissä siis pätee $A \subset A$.} 
Logiikan ja joukko-opin merkintöjä käsitellään hieman laajemmin Luvussa \ref{logiikka}. 
Todettakoon tässä yhteydessä vielä osajoukon määrittelyssä tavallinen merkintä
\[ 
B\ =\ \{\,x \in A \mid E\,\}, 
\]
missä $E$ on jokin \kor{ehto}, joka jokaisen alkion $x \in A$ kohdalla joko toteutuu, jolloin 
$x \in B$, tai ei toteudu, jolloin $x \not\in B$.
\begin{Exa} \kor{Parilliset} ja \kor{parittomat} luonnolliset luvut määritellään $\N$:n osajoukkoina
\begin{align*}
A &= \{\,x \in \N \mid x=2\cdot y\,\ \text{jollakin}\ y\in\N\,\}\ =\ \{\,2,4,6,\ \ldots\,\}, \\
B &= \{\,x\in\N \mid x\not\in A\,\}\ =\ \{\,1,3,5,\ \ldots\,\} \loppu
\end{align*}
\end{Exa}
Esimerkissä on jo viitattu luonnollisten lukujen joukossa (tunnetulla tavalla) määriteltyihin 
\kor{laskuoperaatioihin}, joita on kaksi:
\begin{align*}
\text{yhteenlasku:} \quad &x \in \N\ \ja\ y \in \N \quad \map \quad 
                             x + y \in \N \quad \quad \text{'$x$ plus $y$'} \\
\text{kertolasku:}  \quad &x \in \N\ \ja\ y \in \N \quad \map \quad 
                             x \cdot y\ \,\in \N \quad \quad \text{'$x$ kertaa $y$'}
\end{align*}
Tässä '$\&$' on logiikan merkintä, luetaan 'ja' (formaalimpi merkintä '$\wedge$'), ja 
'$\map$' tarkoittaa \kor{liittämistä}: Lukuihin $x$ ja $y$, eli \N:n \kor{lukupariin}, liitetään
(yksikäsitteinen) luku $x + y \in \N$, nimeltään $x$:n ja $y$:n \kor{summa}, ja 
$x \cdot y \in \N$, nimeltään $x$:n ja $y$:n \kor{tulo}. Laskuoperaatio tarkoittaa siis 
yksittäisten lukujen $x,y$ tapauksessa 'liittämissääntöä', yleisemmin 'liittämissäännöstöä'. 
Säännöt voidaan ajatella joko joistakin yleisemmistä periaatteista (laskusäännöistä) 
johdetuiksi, tai ne voidaan ymmärtää pelkästään 'luettelona', joka kattaa kaikki lukuparit. Kun
yhteen- ja kertolaskun sisältö on yhteisesti sovittu (ja peruskoulun oppikirjoihin painettu), on
luonnollisten lukujen perustalle rakennettu \kor{algebra}\footnote[2]{Algebra eli 'kirjainlaskenta'
on numeroilla laskemisen eli \kor{aritmetiikan} abstraktimpi muoto. Algebran kehitys alkoi
arabikulttuurin piirissä 1.\ vuosituhannella jKr.}, jota merkittäköön $(\N,+,\cdot)$.

Seuraavat hyvin tunnetut peruslait ovat voimassa luonnollisten lukujen laskuoperaatioille:
\begin{itemize}
\item[(L1)] $\quad x+y = y+x$
\item[(L2)] $\quad x \cdot y = y \cdot x$
\item[(L3)] $\quad x+(y+z) = (x+y)+z$
\item[(L4)] $\quad x \cdot (y \cdot z) = (x \cdot y) \cdot z$
\item[(L5)] $\quad x \cdot (y+z) = x \cdot y + x \cdot z$
\item[(L6)] $\quad x \cdot 1 = x \ \ \forall x \in \N$
\end{itemize}
Tässä (L1),\,(L2) ovat \kor{vaihdantalait}, (L3),\,(L4) ovat \kor{liitäntälait} ja (L5) on 
\kor{osittelulaki}. Viimeksi mainitulla on myös rinnakkainen muoto
\begin{itemize}
\item[(L5')] $\quad\!\! (x+y) \cdot z = x \cdot z + y \cdot z$,
\end{itemize}
joka seuraa yhdistämällä (L2) ja (L5). Laissa (L6) '$\,\forall$' on logiikan symboli, joka 
luetaan 'kaikille'. Tämä laki ilmaisee, että luku  $1$ on kertolaskun nk. \kor{ykkösalkio}. 
Kuten nähdään jäljempänä, laskulait (L1)--(L6) ovat yhteisiä monille algebroille. Yleisemmissä tarkasteluissa
nämä lait otetaankin usein oletetuiksi peruslaeiksi eli \kor{aksioomiksi}.\footnote[2]{Luonnollisten lukujen 
tapauksessa laskulait (L1)--(L6) ovat johdettavissa luonnollisten lukujen perustana olevista
\vahv{Peanon aksioomista}. Aksioomat esitti italialainen matemaatikko \hist{Giuseppe Peano} (1858--1932) vuonna
1889. Peanon aksioomat luonnollisille luvuille ovat: (P1) $1\in\N$. \,(P2) Jokaisella $x\in\N$ on yksikäsitteinen
\kor{seuraaja} $x'\in\N$. \,(P3) Luku $1$ ei ole minkään $x\in\N$ seuraaja. \,(P4) Jokainen $x\in\N$ on
korkeintaan yhden luvun seuraaja, ts.\ jos $x,y\in\N$ ja $x'=y'$, niin $x=y$. \,(P5) Jos $S\subset\N$ ja pätee
(i) $1\in\N$ ja (ii) $x' \in S$ aina kun $x \in S$, niin $S=\N$. Aksioomassa (P2) esiintyvä luvun seuraaja on 
tavanomaisen kielenkäytön mukaan 'seuraava luku'. Aksiooman (P3) mukaan luku $1$ on luonnollisista luvuista
'ensimmäinen', sen sijaan 'viimeistä' lukua ei ole, koska jokaisella luvulla on seuraaja (aksiooma (P2)) eikä 
mikään luku voi esiintyä seuraajaketjussa kahdesti (aksioomat (P3) ja (P4)). Luonnollisia lukuja on siis
äärettömän monta. Aksiooma (P5) asettaa nk.\ \kor{induktioperiaatteen}, ks.\ Luku \ref{jono} jäljempänä. 

Luonnollisten lukujen yhteenlasku ja kertolasku määritellään aksioomiin (P1)--(P5) perustuen asettamalla
kummallekin laskuoperaatiolle kaksi aksioomaa. Näiden perusteella minkä tahansa halutun laskuoperaation tulos
on määrättävissä lukujen $\,1$, $1' = 2$, $2' = 3$, jne avulla. Yhteenlaskun aksioomat: \,(Y1) $\,x+1=x'$, 
\,(Y2) $\,x+y'=(x+y)'$. Kertolaskun aksioomat: \,(K1) = (L6), \,(K2) $\,x \cdot y' = x \cdot y + x$.}

Laskulaeissa (L1)--(L6) on oletettu normaalit järjestyssäännöt laskuoperaatioiden yhdistelylle,
eli ensin lasketaan sulkeiden sisällä olevat lausekkeet, ja kertolaskut aina ennen yhteenlaskuja
mikäli sulkeita ei ole. Huomattakoon, että tässä ei ole kyse mistään ylimääräisistä 
olettamuksista, vaan esimerkiksi sääntö
\[
x + y \cdot z\ =\ x + (y \cdot z)
\]
on \pain{sulkeiden} p\pain{ois}j\pain{ättösääntö}, eli kyse on merkintäsopimuksesta. Samaa operaatiota
yhdisteltäessä on  liitäntälakien (L3),\,(L4) perusteella tulos aina sama operaatioiden järjestyksestä
riippumatta, joten sulkeet voidaan jättää pois (merkintäsopimus!) ja kirjoittaa
\[
x + y + x + \ldots, \quad x \cdot y \cdot z \cdot \ldots
\]
Tavanomaisen lukujen algebran merkintäsopimuksiin kuuluu myös vapaus jättää kertomerkki merkitsemättä
sikäli kuin sekaannuksen vaaraa ei ole:
\[
x \cdot y = xy, \quad x \cdot 2 = 2x, \quad  2\cdot(1+2) = 2(1+2) = 2 \cdot 3 \neq 23
\]
Peräkkäisiä tuloja laskettaessa on saman luvun $n$-kertaisesta ($n \in \N$) kertolaskusta tapana
käyttää nimitystä \kor{potenssiin korotus} ja käyttää merkintää
\[
\underset{(n\ \text{kpl})}{x \cdot x \cdots x} \quad = \quad x^n \quad 
                              \quad \text{'$x$ potenssiin $n$'} \quad (n \in \N).
\]
   
\subsection{Kokonaisluvut $\Z$}

\kor{Kokonaislukujen} joukko \Z\ on ensimmäinen askel siinä 'ihmistyössä', jossa 
lukujärjestelmää pyritään laajentamaan lähtökohtana luonnolliset luvut. Laajennus koostuu 
kahdesta osa-askeleesta, joista ensimmäinen on luvun \kor{nolla} ('ei mitään') ja toinen 
\kor{vastaluvun} ('montako puuttuu') käyttöönotto. Näiden ajatusten tuloksena syntyvää 
lukujoukkoa merkitään
\[
\Z = \{\ \ldots,-2,-1,0,1,2,\ \ldots\ \} \quad = \quad \{\ 0,\pm 1, \pm 2,\ \ldots\ \}.
\]
Lukualueen laajennus tehdään aksiomaattisesti olettamalla seuraavat lisäaksioomat, jotka siis
ovat voimassa \Z:ssa mutteivät \N:ssä.
\begin{itemize}
\item[(L7)] $\quad \text{On olemassa \kor{luku}}\ 0,\ 
                   \text{jolle pätee}\ \ x+0 = x\ \ \forall\ x \in \Z.$
\item[(L8)] $\quad \text{Jokaisella}\ x \in \Z\ \text{on \kor{vastaluku}}\ -x,\ 
                   \text{jolle pätee}\ \ x+(-x) = 0.$
\end{itemize}
Lyhennysmerkintä
\[
x + (-y) = x - y \quad \quad \text{'$x$ miinus $y$'}
\]
tuo \Z:aan uuden laskuoperaation, \kor{vähennyslaskun}, jonka tulos on nimeltään lukujen $x$ ja
$y$ \kor{erotus}.  Tässä on siis kyse yhteenlaskun ja 'merkinvaihdon' $x \map -x$ (jonka voi 
myös tulkita laskuoperaatioksi) yhdistämisestä.

Kokonaisluvuilla laskettaessa pidetään selviönä, että luku $0$ samoin kuin vastaluvut $-1,\ -2, \ldots\,$
ovat yksikäsitteisiä. Tuttuja ovat myös laskusäännöt $\,-(-x)=x$, $\,0 \cdot x=0\,$ ja $\,(-1) \cdot x = -x$.
Näitä oletuksia\,/\,laskusääntöjä ei tarvitse kuitenkaan sisällyttää kokonaislukujen aksioomiin, sillä ne
ovat aksioomien seurauksia:
\begin{Lause}\footnote[2]{Kun matemaattisessa tekstissä halutaan nostaa esiin jokin selkeästi 
muotoiltu, tosi väittämä, esim. siksi että väittämään halutaan myöhemmin viitata tai kun 
halutaan korostaa tuloksen uutuusarvoa, käytetään nimityksiä \kor{Lause} eli \kor{Teoreema} 
(engl.\ Theorem), \kor{Propositio} (engl.\ Proposition), \kor{Apulause} eli \kor{Lemma} 
(engl.\ Lemma) ja \kor{Seurauslause} eli \kor{Korollaari} (engl.\ Corollary). Tässä tekstissä 
käytetään (hieman epäjohdonmukaisesti) nimityksiä Lause, Propositio, Lemma ja Korollaari. 
Propositio on lausetta vähäarvoisempi tai sisällöltään teknisempi väittämä --- nimitystä 
käytetään lähinnä, kun halutaan luokitella lauseita niiden painoarvon mukaan. Lemma on yleensä
välietappi jonkin lauseen/proposition todistamisessa ja korollaari jo todistetun lauseen tai 
proposition suoraviivainen seuraamus.} \label{Z-tuloksia} Kokonaisluvuille pätee aksioomien
(L1)--(L8) perusteella: \vspace{1mm}\newline
(a) \ Luku $0$ on yksikäsitteinen, samoin vastaluku $-x$ jokaisella $x\in\Z$. \newline
(b) \ $-(-x)=x\,\ \forall x\in\Z$. \newline
(c) \ $0 \cdot x = 0\,\ \forall x\in\Z$. \newline
(d) \ $(-1) \cdot x = -x\,\ \forall x\in\Z$. 
\end{Lause}
Lause \ref{Z-tuloksia} tulee todistetuksi seuraavassa luvussa osana yleisempää lausetta,
joten tyydytään tässä todistamaan ainoastaan helpoin osaväittämä (b): Koska aksioomien
(L1) ja (L8) mukaan on
\[
-x+x = x+(-x) = 0,
\]
niin vastaluvun määritelmän mukaisesti on $-(-x)=x$. \loppu
\begin{Exa} Lauseen \ref{Z-tuloksia} laskusäännöt (b) ja (d) yhdistämällä nähdään, että
$(-1)\cdot(-1)=-(-1)=1.$ \loppu \end{Exa}

\subsection{Lukujärjestelmät}
Kokonaisluvut on käytännössä ilmaistava jossakin \kor{lukujärjestelmässä}. Tavanomaisin, 
matematiikan 'äidinkieleksi' vakiintunut on \kor{kymmenjärjestelmä} (engl.\ decimal system), 
jossa kirjoitetaan
\begin{align*}
1+1\ &=\ 2 \quad\ \,\text{'kaksi'}, \\
2+1\ &=\ 3 \quad\ \,\text{'kolme'}, \\
     &\ \vdots \\
8+1\ &=\ 9 \quad\ \,\text{'yhdeksän'}, \\
9+1\ &=\ 10 \quad \text{'kymmenen'}.
\end{align*}
Tässä merkit $0 \ldots 9$ ovat \kor{kymmenjärjestelmän numerot} (engl.\ digits, lat.\ 
digitus = sormi,varvas). Luku 'kymmenen' on luonnollisista luvuista ensimmäinen, jolla ei ole 
ko.\ järjestelmässä omaa symbolia, vaan se merkitään yhdistelmänä 'yksi nolla'. Tämä on 
järjestelmän \kor{kantaluku}. Kymmenjärjestelmän luvut ilmaistaan tämän jälkeen symbolisesti 
muodossa 
\[
x\ =\ eabcd \ldots\ \in\ \Z,
\]
missä $e$ on \kor{etumerkki} (engl.\ sign), $e=+$ tai $e=-$ (\,'tyhjä' = $+$\,), ja 
$\,a,b,c,d, \ldots$ ovat kymmenjärjestelmän numeroita. Numeroiden lukumäärästä riippuen tulkinta
on
\[
ea = \pm a, \quad eab = \pm (a \cdot 10 + b), \quad 
                 eabc = \pm(a \cdot 10^2 + b \cdot 10 + c), \quad \text{jne.}
\]
Tässä etumerkki on $e$:n mukainen.\footnote[2]{Lukumerkinnän tulkinnan mukaisesti on
$\pm 0abc .. = \pm abc ..$, ts.\ 'etunollat' eivät vaikuta luvun tulkintaan.} 

Yleisemmin jos lukujärjestelmän kantaluku on luku $k \in \N,\ k\neq 1$, niin tämän 
$k$-\kor{järjestelmän} luvuilla $0,1, \ldots, k-1$ on oltava kullakin oma 
numeromerkki.\footnote[2]{Lukujärjestelmien idea keksittiin Kaksoisvirranmaassa noin 4000 vuotta
sitten. Varhaisimmissa järjestelmissä kantaluku oli $60$. Numeromerkit olivat nuolenpäämerkkien
yhdistelmiä.} Jos $a,b,c, \ldots$ ovat tällaisia merkkeja, niin luku $x = eabcd \ldots$ 
tulkitaan samoilla säännöillä kuin edellä. On huomioitava ainoastaan, että luku '10' ei nyt ole
'kymmenen', vaan tällä tarkoitetaan lukua $1\cdot k + 0 = k$. Kantaluvun merkki on siis aina 
'$10$', sen sijaan ääntämistapa (esim.\ 'kymmenen','tusina','tiu') on sovittava erikseen 
kussakin lukujärjestelmässä (ja kussakin kielessä). Ellei mitään ole sovittu, on luettava 
'yksi nolla'. Lukujärjestelmistä kaikkein yksinkertaisin numeromerkistöltään on 
\kor{binaarijärjestelmä}, jossa kantalukuna on 'ykkösestä seuraava' eli $2$. Binaarijärjestelmän
numeroita eli \kor{bittejä} ovat vain $0$ ja $1$.
\begin{Exa} Kymmenjärjestelmän luku $1253$ voidaan lukea 'yksi kaksi viisi kolme', jolloin 
ainoastaan äännetään luvun 'nimi'. Lukutavassa 'tuhat kaksi sataa viisikymmentä kolme' annetaan
jo luvulle tulkinta: $1253 = $ 'tuhat plus kaksi kertaa sata plus viisi kertaa kymmenen plus 
kolme'. Tässä 'sata' ja 'tuhat' ovat suomen kielessä sovittuja (ja kouluissa opetettuja) 
ääntämistapoja luvuille $100=k^2$ ja $1000=k^3$, kun $k =$ 'kymmenen'. \loppu 
\end{Exa}
\begin{Exa} Lausu kymmenjärjestelmän luku 11 binaarijärjestelmässä. \end{Exa}
\ratk Kymmenjärjestelmässä voidaan kirjoittaa $11 = 1 \cdot 2^3 + 0 \cdot 2^2 + 1 \cdot 2 + 1$. 
Binaarijärjestelmän esitysmuoto saadaan tästä kirjoittamalla merkin '$2$' tilalle '$10$'\,:
\[
11\ =\ 1 \cdot 10^3 + 0 \cdot 10^2 + 1 \cdot 10 + 1\ =\ 1011 \quad 
                                  \text{(binaarijärjestelmä)}. \quad \loppu
\]
\begin{Exa} Laske $7$-järjestelmän lukujen $145$ ja $66$ summa ja muunna tulos 
kymmenjärjestelmään. \end{Exa}
\ratk Kun merkitään $k=7$, niin laskulakeja (L1)--(L6) soveltaen saadaan
\begin{align*}
145+66 &= (1 \cdot k^2 + 4 \cdot k + 5) + (6 \cdot k + 6) \\
       &= 1 \cdot k^2 + (4+6) \cdot k + (5+6) \\
       &= 1 \cdot k^2 + (1 \cdot k + 3) \cdot k + (1 \cdot k + 4) \\
       &= (1+1) \cdot k^2 + (3+1) \cdot k + 4 \\
       &= 2 \cdot k^2 + 4 \cdot k + 4 \\
       &= 244.
\end{align*}
Kymmenjärjestelmässä ilmaistuna lopputulos on $2 \cdot 7^2 + 4 \cdot 7 + 4 = 130$. \loppu

\pagebreak

\subsection{Rationaaliluvut $\Q$}

Siirryttäessä \Z:sta \kor{rationaalilukujen} joukkoon \Q\ tarvitaan enää yksi lisäaksiooma:
\begin{itemize}
\item[(L9)] $\quad \text{Jokaisella}\ x \neq 0\ \text{on \kor{käänteisluku}}\ x^{-1}\ 
                                                 \text{siten, että}\ \ x \cdot x^{-1} = 1.$
\end{itemize}
Lyhennysmerkintä
\[
x \cdot y^{-1} = x / y = \frac{x}{y} \quad  \quad \text{'$x$ per $y$'}
\]
määrittelee neljännen laskuoperaation, \kor{jakolaskun}, jonka tulos on nimeltään lukujen $x$ ja
$y$ \kor{osamäärä}. Jakolaskussa on siis kyse 'kääntämisoperaation' $y \map y^{-1}$ ja 
kertolaskun $x,y^{-1} \map x \cdot y^{-1}$ yhdistämisestä. Sanotaan, että luku $x$ on tässä
operaatiossa \kor{osoittaja} ja luku $y$ on \kor{nimittäjä} (engl.\ numerator, denominator). 
Jakolaskun määritelmän ja aksioomien (L2),\,(L6) perusteella on
\[
x^{-1} = x^{-1} \cdot 1 = 1 \cdot x^{-1} = 1 / x.
\]
Kun lukujoukkoa \Z\ laajennetaan siten, että laajennetussa joukossa \Q\ pätevät aksioomat 
(L1)-(L9), on tuloksena uusi algebra, jota merkitään $(\Q,+,\cdot)$. Aksiooman (L9) ja jakolaskun
määritelmän mukaisesti \Q:n alkioita ovat ainakin luvut muotoa $x = p/q$, missä $p \in \Z$ ja $q \in \N$.
Osoittautuu (ks.\ Esimerkki\,\ref{kaksi Q-lakia} alla ja Harj.teht.\,\ref{H-I-1: Q-lakeja}), että
lähdettäessä mainittua muotoa olevista luvuista ovat laskuoperaatioiden $\,x \map -x$, $\,x \map x^{-1}$,
$\,x,y \map x+y\,$ ja $\,x,y \map x \cdot y\,$ aina samaa muotoa, eli muun tyyppisiä lukuja ei \Q:ssa ole.
Rationaalilukujen joukko \Q\ voidaan siten määritellä
\[
\Q = \{\ x = p/q\ \vert\ p \in \Z\ \ja\ q \in \N\ \}.
\]

Aksioomista (L1)-(L9) voidaan johtaa rationaalilukujen kaikki normaalit laskusäännöt. Seuraavassa
kaksi esimerkkiä (ks.\ myös Harj.teht.\,\ref{H-I-1: Q-lakeja}).
\begin{Exa} \label{kaksi Q-lakia}
Vedoten aksioomiin (L1)--(L9) ja Lauseeseen \ref{Z-tuloksia} perustele rationaalilukujen
laskusäännöt
\[
\text{a)}\,\ \frac{0}{q}=0\,\ (q\in\N) \qquad 
\text{b)}\,\ \frac{1}{q_1}\cdot\frac{1}{q_2} = \frac{1}{q_1q_2}\,\ (q_1,q_2\in\N)
\]
\end{Exa}
\ratk a) Lähtien rationaaliluvun määritelmästä ja vedoten Lauseen \ref{Z-tuloksia}
väittämään (c) sekä aksioomiin (L4), (L9) ja (L6) päätellään:
\begin{align*}
0/q = 0 \cdot q^{-1} &= (0 \cdot q)\cdot q^{-1} \\
                     &= 0\cdot(q \cdot q^{-1}) \\
                     &= 0 \cdot 1 \\
                     &= 0.
\end{align*}
b) Väitetään, että $q_1^{-1}q_2^{-1}=(q_1q_2)^{-1}$, ts.\ että $y=q_1^{-1}q_2^{-1}$ on luvun
$x=q_1q_2$ käänteisluku. Tarkistus:
\begin{align*}
x \cdot y = (q_1q_2)\cdot(q_1^{-1}q_2^{-1}) &= (q_1q_2)\cdot(q_2^{-1}q_1^{-1}) \\
                                            &= q_1(q_2q_2^{-1})q_1^{-1} \\
                                            &= q_1 \cdot 1 \cdot q_1^{-1} \\
                                            &= q_1q_1^{-1} = 1. 
\end{align*}
Tässä vedottiin aksioomiin (L2), (L4), (L9) ja (L6). \loppu

Kun käänteisluvun potenssiin korotuksessa sovitaan merkinnästä
\[
(x^{-1})^n = \quad x^{-n}, \quad n \in \N,
\]
voidaan $x^{-1}$ lukea myös '$x$ potenssiin miinus $1$'. Aksioomat (L2),\,(L4),\,(L6) ja (L9)
huomioiden seuraa myös 
\[
x \neq 0\ \ja\ m,n \in \Z\ \ja\ m,n \neq 0 
                         \qimpl x^m \cdot x^n = \begin{cases}
                                                \,x^{m+n}, \quad \text{kun}\,\ m+n \neq 0 \\
                                                \,1, \quad\quad\,\ \ \text{kun}\,\ m+n = 0
                                                \end{cases}
\]
Tämän säännön yksinkertaistamiseksi on luontevaa \pain{määritellä}:
\[
\boxed{\kehys\quad x^0 = 1, \quad \text{kun}\ x \neq 0\ \ (x\in\Q) \quad }
\]
Tällöin saadaan yleiset laskusäännöt
\[
\boxed{\kehys \quad x^mx^n=x^{m+n}, \quad x^ny^n=(xy)^n, \quad (x^m)^n=x^{mn}, 
                                                         \quad x,y \neq 0,\ m,n\in\Z \quad} 
\]
Myös $0^n = 0 \ \ \forall\ n \in \N$. Sen sijaan määrittelemättä jäävät $0^0$ ja 
$0^{-n},\ n \in \N$, syystä että $0$:lla ei ole käänteislukua. 

\subsection{$\Q$:n järjestysrelaatio}

Rationaaliluvut voidaan tunnetulla tavalla myös \kor{järjestää}, eli voidaan määritellä
\kor{järjestysrelaatio}
\[
<\ : \quad \text{'pienempi kuin'}.
\]
Tällöin syntyvää algebraa voidaan merkitä $(\Q,+,\cdot,<)$. Järjestysrelaatioon liitetään seuraavat
yleiset aksioomat (eli oletetut peruslait):

\pagebreak

\begin{itemize}
\item[(J1)] $\quad \text{Vaihtoehdoista}\ \{x<y,\ x=y,\ x>y\}\ 
                                          \text{on voimassa täsmälleen yksi}.$
\item[(J2)] $\quad x<y\ \ja\ y<z \,\ \impl\,\ x<z$
\item[(J3)] $\quad x<y \,\ \impl\,\ x+z < y+z \ \ \forall z$
\item[(J4)] $\quad x > 0 \ \ja\ y > 0 \,\ \impl\,\ x \cdot y\, > 0$
\end{itemize}
Tässä merkintä $x>y$ ('$x$ suurempi kuin $y$') on merkinnän $y<x$ vaihtoehtoinen muoto. Käytännössä 
vertailusta selvitään nopeimmin käyttämällä laskulakeja (L7),\,(L8) ja (L1), sillä näistä ja 
aksioomasta (J3) on pääteltävissä, että pätee
\[ 
\begin{cases} \begin{aligned}
x<y &\qekv x-y < 0 \\ x=y &\qekv x-y = 0 \\ x>y &\qekv x-y > 0
\end{aligned} \end{cases}
\] 
Tässä '$\ekv$' on logiikan symboleihin kuuluva \kor{ekvivalenssinuoli}, joka voidaan tässä 
yhteydessä lukea
\[
\text{P} \quad \ekv \quad \text{Q}\ : \quad \text{'P täsmälleen kun Q'}.
\]
Ym.\ kriteerillä voidaan siis vertailukysymys $\,x$ ? $y\,$ selvittää tutkimalla, onko luku 
$x-y$ \kor{positiivinen} ($>0$), \kor{negatiivinen} ($<0$), vai $\,=0$. Perusmuotoisesta
rationaaliluvusta $\,x = p/q,\ (p\in\Z,\ q\in \N)$ puolestaan sovitaan, että vaihtoehdot
$x>0$ ja $x=0$ vastaavat vaihtoehtoja $p\in\N$ ja $p=0$, muulloin on $x<0$. Tämä sopimus on
mahdollinen, koska Esimerkin\,\ref{kaksi Q-lakia} laskusäännön (a) mukaan luvut $0/q$ samastuvat
lukuun $0$\,:
\[
0\ =\ \dfrac{0}{1}\ =\ \dfrac{0}{2}\ =\ \ldots
\]
Toisaalta muita luvun $0$ esiintymismuotoja ei $\Q$:ssa ole, sillä jos $x=p/q=0$, niin seuraa
$\,(p/q) \cdot q = p = 0 \cdot q = 0$ (aksioomat (L4),\,(L2),\,(L9),\,(L6) ja Lauseen \ref{Z-tuloksia}
väittämä (c)). Aksiooma (J1) on näin ollen voimassa. Myös aksioomien (J2)--(J4) voimassaolo on
helposti pääteltävissä (Harj.teht.\,\ref{H-I-1: Q-järjestys}).

Järjestys ja samastus ovat esimerkkejä (joukko--opillisista) \kor{relaatioista}, joissa on kyse joukon
kahden alkion välisestä 'suhteesta'. Järjestykseen\,/\,samastukseen liittyviä relaatiosymboleja ovat 
mainittujen lisäksi myös '$\neq$', '$\le$' ja '$\ge$', jotka määritellään
\begin{align*}
x \neq y \quad &\ekv \quad \text{ei päde}\ \ x=y \\
x \le y  \quad &\ekv \quad x < y\,\ \text{tai}\,\ x = y \\
x \ge y  \quad &\ekv \quad x > y\,\ \text{tai}\,\ x = y
\end{align*}
Merkintä $P \ekv Q$ voidaan tässä lukea '$P$ tarkoittaa samaa kuin $Q$'.

\subsection{Summa- ja tulomerkinnät}

Laskettaessa peräkkäin yhteen tai kerrottaessa useita erilaisia lukuja on luvut kätevää 
numeroida eli \kor{indeksoida}, jolloin voidaan käyttää lyhennysmerkintöjä. Nämä ovat 
\begin{align*}
&\text{\kor{Summamerkintä:}} \qquad\, x_1 + x_2 + \ \ldots\  + x_n\ =\ \sum_{i=1}^n x_i \\
&\text{\kor{Tulomerkintä:}} \qquad\qquad\quad x_1 \cdot x_2 \cdot\ \ldots\ \cdot x_n\ 
                                                                    =\ \prod_{i=1}^n x_i
\end{align*}
Summamerkinnässä symboli $i$ on nimeltään \kor{summausindeksi}. Tämä saa yleensä (ei aina)
kokonaislukuarvoja, jolloin symboli valitaan tavallisimmin joukosta $\{i,j,k,l,m,n\}$.
Koska kyse on vain lyhennysmerkinnästä, ei symbolin valinta luonnollisesti vaikuta itse 
laskutoimitukseen:
\[
\sum_{i=1}^n a_i = \sum_{j=1}^n a_j = \sum_{k=1}^n a_k = \ldots
\]
Peräkkäisissä tuloissa sanotaan kerrottavia lukuja tulon \kor{tekijöiksi} (engl.\ factor).
Kokonaislukujen peräkkäisiin tuloihin liittyy termi $n$-\kor{kertoma} (engl.\ $n$-factorial),
joka merkitään $\,n!\,$ ja määritellään
\[
\quad 0! = 1, \quad 1! = 1, \quad n!\,=\,1 \cdot 2 \ \ldots\ \cdot n\ 
                                      =\ \prod_{k=1}^n k, \quad n = 2,3,\ \ldots
\]

%\subsection{Reaaliluvut $\R$}

%Rationaalilukujen joukko \Q\ on edelleen laajennettavissa \kor{reaalilukujen} joukoksi \R, 
%jolloin koko laajennusketju on
%\[
%\N \quad \ext \quad \Z \quad \ext \quad \Q \quad \ext \quad \R.
%\]
%Viimeinen laajennusvaihe on kuitenkin käsitteellisesti huomattavasti vaikeampi kuin muut. Luvun
%\Roman{chapter} eräänä päätavoitteena --- ja punaisena lankana jatkossa --- on juuri tämän 
%laajennusvaiheen toteuttaminen.

\Harj
\begin{enumerate}

\item
Näytä, että seuraaville joukoille pätee $A=B$, eli $A$ ja $B$ ovat joukkoina samat:
\begin{align*}
&\text{a)}\ \ A=\{3n+2 \mid n\in\Z\}, \quad B=\{3n-7 \mid n\in\Z\} \\
&\text{b)}\ \ A=\{7n+3 \mid n\in\Z\}, \quad B=\{7n-32 \mid n\in\Z\}
\end{align*}

\item
Vedoten aksioomiin (L1)--(L8) ja Lauseen \ref{Z-tuloksia} väittämiin perustele kokonaislukujen
laskusäännöt \ a) $-x+(-y)=-(x+y)$, \ b) $(-x) \cdot y = - x\cdot y$, \
c) $(-x) \cdot (-y) = x \cdot y$.

\item 
Muodosta kantalukuja $k=2$, $k=3$ ja $k=4$ vastaavien lukujärjestelmien kertotaulut.

\item 
Tusinajärjestelmässä on kymmenjärjestelmän merkkien lisäksi käytössä numeromerkit (vasemmalla 
kymmenjärjestelmän merkintä)
\begin{align*}
&10=\diamondsuit \quad \text{'ruutu'} \\
&11=\heartsuit \quad \text{'hertta'} \\
&12=10 \quad \text{'tusina' (kantaluku)}
\end{align*}
Muunna kymmenjärjestelmän luku 155 tusinajärjestelmään ja tusinajärjestelmän luku 
$9\diamondsuit 07\heartsuit$ kymmenjärjestelmään.

\item 
Suorita kymmenjärjestelmään muuntamatta seuraavat tusinajärjestelmän (ks. edellinen tehtävä) 
laskuoperaatiot:
\[
\text{a)}\ \ \heartsuit\heartsuit\diamondsuit + \diamondsuit\diamondsuit\heartsuit \qquad 
\text{b)}\ \  247-19\diamondsuit \qquad 
\text{c)}\ \ 23 \cdot 34
\]

\item
Rusinajärjestelmässä kymmenjärjestelmän luku $7$ äännetään 'rusina', luvut $1 \ldots 6$ merkitään ja
äännetään samoin kuin kymmenjärjestelmässä, ja luvut $100$ ja $1000$ äännetään 'pulla' ja 'kakku'.
Miten merkitään ja lausutaan kymmenjärjestelmän luku $2331$ rusinajärjestelmässä?



\item  \label{H-I-1: Q-lakeja}
Seuraavissa rationaalilukujen laskusäännöissä on tulos esitetty rationaaliluvun perusmuodossa
olettaen, että $p,p_1,p_2\in\Z$ ja $q,q_1,q_2\in\N$. Perustele säännöt vedoten aksioomiin (L1)--(L9),
Lauseeseen \ref{Z-tuloksia} ja Esimerkin \ref{kaksi Q-lakia} laskusääntöihin.
\begin{align*}
&\text{a) \,Supistus\,(lavennus):} \quad (p \cdot m)/(q \cdot m) = p/q, \quad m\in\Z,\ m\neq 0 \\
&\text{b)   Vastaluku:} \quad -(p/q) = (-p)/q \\
&\text{c) \,Käänteisluku:} \quad (p/q)^{-1} = \begin{cases}
                                              \,q/p,       &\text{jos}\ p\in\N \\
                                              \,(-q)/(-p), &\text{jos}\ p\not\in\N,\ p \neq 0
                                              \end{cases} \\
&\text{d)   Yhteenlasku:} \quad p_1/q_1+p_2/q_2 = (p_1q_2+p_2q_1)/(q_1q_2) \\
&\text{e) \,Kertolasku:} \quad (p_1/q_1)(p_2/q_2) = (p_1p_2)/(q_1q_2)
\end{align*}

\item \label{H-I-1: Q-järjestys}
Vedoten tehtävän \ref{H-I-1: Q-lakeja} laskusääntöihin ja rationaalilukujen järjestysrelaation
määritelmään näytä, että järjestysrelaatiolle ovat voimassa aksioomat (J2), (J3) ja (J4). 

\item
Ratkaise $\Q$:ssa seuraavat yhtälöt\,/\,epäyhtälöt mahdollisimman yksinkertaiseen muotoon. 
Perustele laskun vaiheet aksioomilla (L1)--(L9), (J1)--(J4)\,! \vspace{1mm}\newline
a) \ $2x+7=9x-4 \qquad\quad\,\ $
b) \ $\tfrac{x}{5}+\tfrac{2}{3}<\tfrac{x}{3}+\tfrac{1}{2}$ \newline
c) \ $(x+1)/(x-2)=4 \qquad$
d) \ $(2x-1)/(3x+2)>3$

\item
a) Näytä, että seuraavat päättelysäännöt ovat päteviä sekä kokonaisluvuille ($x_k\in\Z$) että
rationaaliluvuille ($x_k\in\Q$). Voit vedota sekä aksioomiin että niistä johdettuihin laskusääntöihin
(Lause \ref{Z-tuloksia}, Tehtävä \ref{H-I-1: Q-lakeja}).
\begin{align*}
&\quad\ \ \prod_{k=1}^n x_k = 0 \qekv x_k=0\,\ \text{jollakin}\,\ k\in\{1,\ldots,n\} \\
&\quad\ \ \prod_{k=1}^n x_k < 0 \qekv \text{pariton määrä lukuja $x_k$ on negatiivisia, muut}\ >0
\end{align*}
Soveltaen a-kohdan päätelmiä ratkaise $\Q$:ssa:
\vspace{1mm}
\begin{align*}
&\text{b)}\,\ 2x^2-3x-2 \ge 0 \qquad\,
 \text{c)}\,\ 63x^2-32x \le 63 \qquad
 \text{d)}\,\ 9x^3 \ge 169x \\[1mm]
&\text{e)}\,\ \frac{x+1}{x-1}\,\ge\,\frac{x}{x+1} \qquad\quad
 \text{f)}\,\ \frac{5}{6x-7}\,\le\,x \qquad\qquad
 \text{g)}\,\ 6x+\frac{2x+12}{x^2+x}\,\ge\,13
\end{align*}

\item
Sievennä (ol.\ $n\in\N$)
\[
\text{a)}\ \ \sum_{k=1}^n n \qquad
\text{b)}\ \ \sum_{i=1}^{n^2} \frac{2}{n} - \sum_{j=1}^n 1 - \sum_{k=1}^n 1 \qquad
\text{c)}\ \ \prod_{k=1}^n \left(1+\frac{1}{k}\right)
\]

\item 
Kaksoissumma $\sum_{i=1}^n\sum_{j=1}^m a_{ij}\ (n,m\in\N,\ a_{ij}\in\Q)$ määritellään
\[
\sum_{i=1}^n\sum_{j=1}^m a_{ij} = \sum_{i=1}^n\Bigl(\sum_{j=1}^m a_{ij}\Bigr) 
                                = \sum_{j=1}^m\Bigl(\sum_{i=1}^n a_{ij}\Bigr)
                                = \sum_{j=1}^m\sum_{i=1}^n a_{ij}\,.
\]
a) Näytä, että tässä sulkeiden poisto ja järjestyksen vaihto ovat todella mahdollisia, ts.\ 
että kaksi keskimmäistä lauseketta ovat samat. \vspace{1mm}\newline
b)\,\ Sievennä: $\D \ \sum_{i=1}^n\sum_{j=1}^n (a_i + a_j). \quad\ $
c)\,\ Näytä: $\D \ \Bigl(\,\sum_{i=1}^n a_i \Bigr)^2 = \sum_{i=1}^n\sum_{j=1}^n a_i a_j\,$.

\item (*)  \label{H-I-1: teleskooppisumma}
Summaa $\sum_{k=0}^n a_k$ sanotaan \kor{teleskooppisummaksi}, jos $a_k=b_{k+1}-b_k\ \forall k$, 
missä luvut $b_k,\ k=0 \ldots n+1$,
ovat tunnettuja. \newline
a) Määritä teleskooppisumman arvo lukujen $b_k$ avulla.\newline
b) Mistä nimitys teleskooppisumma?\newline
c) Laske teleskooppi-idealla summat
\[
\sum_{k=0}^n (2k+1) \quad \text{ja} \quad \sum_{k=1}^n\frac{1}{k(k+1)}\,, \quad n\in\N.
\]
d) Näytä teleskooppisummien avulla:\, $\sum_{k=1}^n k^2 = \tfrac{1}{6}n(2n^2+3n+1),\,\ n\in\N$. 

\end{enumerate}