\section{Separoituva differentiaaliyhtälö} \label{separoituva DY}
\alku
\index{differentiaaliyhtälö!h@separoituva|vahv}
\index{separoituva DY|vahv}

Ensimmäisen kertaluvun normaalimuotoinen differentiaaliyhtälö on \kor{separoituva}, jos se on
muotoa
\[
y'=\frac{f(x)}{g(y)}\,,
\]
eli jos muuttujat $x$ ja $y$ erottuvat (separoituvat) oikealla puolella. Kirjoittamalla $g(y)$:n
tilalle $1/g(y)$ ja/tai $f(x)$:n tilalle $1/f(x)$ saadaan separoituvuudelle muita 
ilmenemismuotoja. Oletettu muoto on kätevä lähinnä ratkeavuusoletuksien muotoilun kannalta.
Jatkossa oletetaan funktioista $f$ ja $g$, että joillakin $x_1<x_2$ ja $y_1<y_2$ pätee
\begin{itemize}
\item[(i)] $f$ on jatkuva välillä $[x_1,x_2]$ ja $g$ on jatkuva välillä $[y_1,y_2]$,
\item[(ii)] $g(y)\neq 0\quad\forall y\in [y_1,y_2]$.
\end{itemize}
Valitaan nyt $x_0\in (x_1,x_2)$, $y_0\in (y_1,y_2)$ (avoimet välit!) ja tarkastellaan 
alkuarvotehtävää
\begin{equation} \label{separ 1}
\begin{cases} 
\,y'=f(x)/g(y), \quad x\in(x_0-\delta,x_0+\delta) \\ \,y(x_0)=y_0
\end{cases} \tag{$\star$}
\end{equation}
Tässä valitaan $\delta$ ensinnäkin niin, että toteutuu
\[
0 < \delta \le \delta_1 = \min \{x_0-x_1,x_2-x_0\},
\]
jolloin $[x_0-\delta,x_0+\delta]\subset [x_1,x_2]$. Jatkossa tehdään myös toinen oletus, joka
koskee alkuarvotehtävän \eqref{separ 1} ratkeavuutta. Tämäkin ehto rajoittaa $\delta$:n
valintaa, mutta toistaiseksi jätetään avoimeksi, miten. Oletus muotoillaan seuraavasti:
\begin{itemize}
\item[(iii)] Alkuarvotehtävällä \eqref{separ 1} on välillä $(x_0-\delta,x_0+\delta)$ 
             derivoituva ratkaisu $y(x)$, ja pätee 
             $y(x)\in[y_1,y_2]\,\ \forall x\in(x_0-\delta,x_0+\delta)$.
\end{itemize}

Tehtyjen oletuksien (i)--(iii) perusteella funktiot $f(x)$, $y(x)$ ja $g(y(x))$ ovat
jatkuvia välillä $(x_0-\delta,x_0+\delta)$ ja $g(y(x)) \neq 0$ ko.\ välillä. Tähän ja Analyysin
peruslauseeseen vedoten voidaan päätellä
\begin{align*}
y'(x)=\frac{f(x)}{g(y(x))} 
           &\qekv g(y(x))y'(x)=f(x) \\
           &\qekv \frac{d}{dx}\int_{y_0}^{y(x)} g(t)\,dt
                     =\frac{d}{dx}\int_{x_0}^x f(t)\,dt,\quad x\in (x_0-\delta,x_0+\delta)
\end{align*}
eli
\begin{align*}
y'(x)= \frac{f(x)}{g(y(x))} \qekv
       &\frac{d}{dx}[G(y(x))-F(x)]=0,\quad x\in (x_0-\delta,x_0+\delta), \\
       &G(y)=\int_{y_0}^y g(t)dt,\quad F(x)=\int_{x_0}^x f(t)dt.
\end{align*}
Tämän mukaan on oltava $G(y(x))-F(x)=C=$ vakio välillä $(x_0-\delta,x_0+\delta)$
(Lause \ref{Integraalilaskun peruslause}). Toisaalta koska $G(y(x_0))=G(y_0)=0$ ja $F(x_0)=0$,
niin $C=0$, joten on päätelty:
\[
G(y(x))=F(x),\quad x\in (x_0-\delta,x_0+\delta).
\]
Varmistetaan nyt, että tämä yhtälö määrittelee implisiittisesti funktion $y(x)$. Ensinnäkin
oletuksista (i)--(ii) funktiolle $g$ ja Lauseesta \ref{Weierstrassin peruslause} seuraa, että
\[
\text{joko:}\ \  g(y) \ge c>0 \ \ \forall y\in [y_1,y_2], \qquad
\text{tai:}\ \   g(y) \le -c<0\ \ \forall y\in [y_1,y_2].
\]
(Tässä $c$:n suurin arvo $=|g(y)|$:n minimiarvo välillä $[y_1,y_2]$.) Tämän perusteella funktio
$G(y)$ on joko aidosti kasvava tai aidosti vähenevä välillä $[y_1,y_2]$, siis $G$ on 1--1 ko.\
välillä.
\begin{figure}[H]
\setlength{\unitlength}{1cm}
\begin{center}
\begin{picture}(14,4)(-1,0)
\multiput(0,0)(7,0){2}{
\put(0,0){\vector(1,0){6}} \put(5.8,0.2){$x$}
\put(0,0){\vector(0,1){4}} \put(0.2,3.8){$y$}
}
\curve(1,1.5,3,2,5,3)
\dashline{0.2}(3,0)(3,2) \dashline{0.2}(3,2)(0,2)
\put(4.55,3.2){$x=G(y)$}
\put(-0.4,1.9){$y_0$} \put(-0.4,1.4){$y_1$} \put(-0.4,2.9){$y_2$}
\put(0,1.5){\line(1,0){0.1}} \put(0,3){\line(1,0){0.1}} 
\put(2.9,-0.5){$x_0$} 
\curve(8,3,10,2,12,1.5)
\dashline{0.2}(10,0)(10,2) \dashline{0.2}(10,2)(7,2)
\put(7.5,3.2){$x=G(y)$}
\put(6.6,1.9){$y_0$} \put(6.6,1.4){$y_1$} \put(6.6,2.9){$y_2$}
\put(7,1.5){\line(1,0){0.1}} \put(7,3){\line(1,0){0.1}}
\put(9.9,-0.5){$x_0$} 
\end{picture}
\end{center}
\end{figure}
Koska siis $G$ on 1--1 välillä $[y_1,y_2]$ ja koska oletuksen (iii) mukaan $y(x)\in[y_1,y_2]$
kun $x\in(x_0-\delta,x_0+\delta)$, niin voidaan kirjoittaa
\[
G(y(x))=F(x) \ \ekv \ y(x)=\inv{G}(F(x)),\quad x\in(x_0-\delta,x_0+\delta).
\]
Oletuksien (i)--(iii) voimassa ollessa on näin muodoin päätelty:
\[
y'(x)=\frac{f(x)}{g(y(x))} \qekv y(x)=\inv{G}(F(x)),\quad x\in(x_0-\delta,x_0+\delta).
\]
Lähtemällä saadusta ratkaisukaavasta nähdään em.\ päättelyketjusta, että kaavan mukainen
$y(x)$ on alkuarvotehtävän \eqref{separ 1} yksikäsitteinen (derivoituva) ratkaistu, sikäli kuin
oletukseen (iii) sisältyvä ehto $y(x)\in[y_1,y_2]\ \forall x\in(x_0-\delta,x_0+\delta)$ on 
voimassa. On siis enää varmistettava, että myös tämä ehto toteutuu saadulle (ainoalle 
mahdolliselle) ratkaisulle $y(x)=G^{-1}(F(x))$.

Todetaan ensinnäkin, että em.\ arvioista funktiolle $g(y)$ sekä integraalien 
vertailuperiaatteesta (Lause \ref{integraalien vertailuperiaate}) seuraa
\[
\Bigl|\int_{y_0}^{y(x)} g(t)dt\Bigr| \,\ge\, c\,\abs{y(x)-y_0} \quad 
                                            \forall x\in (x_0-\delta,x_0+\delta).
\]
Toisaalta oletuksen (i) ja Lauseiden \ref{Weierstrassin peruslause} ja
\ref{integraalien vertailuperiaate} perusteella pätee
\[
\Bigl|\int_{x_0}^x f(t)dt\Bigr|\,\le\,M\abs{x-x_0}, \quad x\in[x_1,x_2],
\]
missä $M=\max_{x\in[x_1,x_2]}|f(x)|$. Yhdistämällä arviot seuraa
\begin{align*}
c\,\abs{y(x)-y_0}        &\,\le\, \abs{G(y(x))} \,=\, \abs{F(x)} \,\le\, M\abs{x-x_0} \\
  \impl \ \abs{y(x)-y_0} &\,\le\, \frac{M}{c}\abs{x-x_0} \,<\, \frac{M\delta}{c}
                                                   \quad \forall x\in (x_0-\delta,x_0+\delta).
\end{align*}
Näin ollen kun valitaan $\delta$ siten, että toteutuu
\[
0<\delta\le\min\{\delta_1,\delta_2\}, \quad \delta_2 = \frac{c}{M}\min \{y_0-y_1,y_2-y_0\},
\]
niin oletuksen (iii) toteutuminen on varmistettu ja näin todistettu
\begin{Lause} \label{separoituvan DY:n ratkaisu} \index{differentiaaliyhtälön!h@ratkeavuus|emph}
Jos $x_0\in (x_1,x_2)$ ja $\,y_0\in (y_1,y_2)$,
niin oletuksien (i),\,(ii) ollessa voimassa on olemassa $\delta>0$ siten, että 
alkuarvotehtävällä \eqref{separ 1} on yksikäsitteinen ratkaisu $y(x)$ välillä 
$(x_0-\delta,x_0+\delta)\subset [x_1,x_2]$. Ratkaisukaava ko.\ välillä on
\[
\int_{y_0}^{y(x)} g(t)dt = \int_{x_0}^x f(t)dt.
\]
\end{Lause}
Helposti muistettava, muodollinen menettely separoituvan differentiaaliyhtälön yleisen ratkaisun
hakemiseksi on kaavio
\begin{align*}
\frac{dy}{dx} = \frac{f(x)}{g(y)} \quad &\impl \quad g(y)\,dy = f(x)\,dx \\
                                        &\impl\,\ \int g(y)\,dy = \int f(x)\,dx.
\end{align*}
Tässä ensimmäinen vaihe on 'separointi' ja toinen 'integrointi'. Menettely on pätevä Lauseen
\ref{separoituvan DY:n ratkaisu} ehdoin, ja alkuehdon $y(x_0)=y_0$ toteuttava ratkaisu saadaan
siis muuttamalla määräämättömät integraalit määrätyiksi.
\begin{Exa} Aiemmin Luvussa \ref{exp(x) ja ln(x)} ratkaistussa alkuarvotehtävässä
\[
\left\{ \begin{aligned}
&y' = y,\quad x\in\R, \\
&y(0) = y_0
\end{aligned} \right.
\]
differentiaaliyhtälö on em.\ separoituvaa tyyppiä ($f(x)=1,\ g(y)=1/y$), joten ratkaisukaava on
Lauseen \ref{separoituvan DY:n ratkaisu} mukaan
\[
\int_{y_0}^{y(x)} \frac{1}{t}\,dt = \int_0^x dt.
\]
Kaava on pätevä, jos joko $y_0,y(x)\in(0,\infty)$ tai $y_0,y(x)\in(-\infty,0)$, jolloin
ratkaisuksi saadaan
\[
\ln (y/y_0)=x \ \ekv \ y(x)=y_0e^x, \quad x\in\R.
\]
Saatu ratkaisu on käypä myös tapauksessa $y_0=0$, johon  separointimenettely ei sovellu. \loppu
\end{Exa}
\begin{Exa}
Määritä differentiaaliyhtälön $\,1+y^2+xyy'=0\,$ yleinen ratkaisu.
\end{Exa}
\ratk Separoimalla ja integroimalla saadaan
\begin{align*}
\frac{ydy}{1+y^2} = -\frac{dx}{x} \ &\impl\ \int \frac{ydy}{1+y^2} = -\int \frac{dx}{x} \\
                                    &\impl\ \frac{1}{2}\ln (1+y^2) = - \ln\abs{x} +C.
\end{align*}
Kun kirjoitetaan $C$:n tilalle $\ln C,\ C>0$, saadaan sieventämällä
\[
x^2(1+y^2)=C^2.
\]
Ratkaisemalla tästä $y=y(x)$ saadaan kaksihaarainen ratkaisu
\[
y(x)=\pm\frac{1}{x}\sqrt{C^2-x^2},\quad C>0.
\]
Tämä on pätevä väleillä $(-C,0)$ ja $(0,C)$. \loppu

Esimerkissä vaihto $C\ext\ln C$ perustui siihen, että $\ln C$ saa kaikki mahdolliset
reaaliarvot, kun $C\in\R_+$. Tällaisilla manipulaatioilla voidaan ratkaisu usein saattaa
yksinkertaisempaan muotoon.
\begin{Exa} \label{muuan separoituva dy}
Ratkaise differentiaaliyhtälö $\,y'=\sin y$.
\end{Exa}
\begin{align*}
\text{\ratk}\quad \frac{dy}{\sin y} = dx 
                       &\qimpl \int\frac{dy}{\sin y} = \int dx \\
                       &\qimpl \ln\Bigl|\tan\frac{1}{2}y\Bigr| = x +\ln C,\quad C>0 \\
                       &\qekv \Bigl|\tan\frac{1}{2}y\Bigr| = Ce^x,\quad C>0 \\
                       &\qekv \tan\frac{1}{2} y = Ce^x,\quad C\neq 0 \quad 
                                                       (\,\pm C \hookrightarrow C) \\[2mm]
                       &\qekv y(x) = 2\arctan(Ce^x), \quad C \neq 0 \\[3mm]
                       &\qekv y(x) = 2\Arctan(Ce^x)+2n\pi, \quad C \neq 0,\ n\in\Z.
\end{align*}

Saatu ratkaisu on käypä myös kun $C=0$, sillä $y(x)=2n\pi$ on alkuperäisen
differentiaaliyhtälön ratkaisu jokaisella $n\in\Z$. Nämä ratkaisut voidaan lukea yleiseen
ratkaisuun kuuluviksi, vaikka separointimenettely ei niihin suoraan ulotu. Yleiseen ratkaisuun
kuulumattomia erikoisratkaisuja ovat lisäksi
\[
y(x)=(2n+1)\pi, \quad n\in\Z.
\]
Nämä ovat em.\ ratkaisufunktioiden raja-arvoja, kun $C\kohti\pm\infty$.
\loppu

\subsection{Autonominen differentiaaliyhtälö}
\index{differentiaaliyhtälö!i@autonominen|vahv}
\index{autonominen DY|vahv}

Separoituvaa differentiaaliyhtälöä muotoa
\[
y'=f(y),
\]
missä siis $f$ ei riipu vapaasta muuttujasta $x$, sanotaan \kor{autonomiseksi}. Kuten 
Esimerkissä 3 edellä, jokaista $f$:n nollakohtaa $y_i$ vastaa vakioratkaisu
\[
y(x)=y_i.
\]
Tämä on erikoisratkaisu tai mahdollisesti yleiseen ratkaisulausekkeeseen sisällytettävissä
oleva, vrt.\ esimerkit edellä. Separointimenettelyllä ei tällaisia ratkaisuja saada suoraan,
koska funktio $g(y)=1/f(y)$ ei ole jatkuva $f$:n nollakohtien ympäristössä 
(vrt.\ Lause \ref{separoituvan DY:n ratkaisu}). 
\begin{Exa} Ratkaise differentiaaliyhtälöt $\,y'=y(1-y)$.
\end{Exa}
\ratk Separoimalla ja integroimalla saadaan
\begin{align*}
dx = \frac{dy}{y(1-y)}\ &\impl\ \int dx = x = \int \frac{1}{y(1-y)}\,dy 
                                            = \int \left(\frac{1}{y}+\frac{1}{1-y}\right)dy \\
                        &\impl\ x = \ln\left|\frac{y}{1-y}\right|+\ln C, \quad C>0.
\end{align*}
Tämän mukaan on
\begin{align*}
\frac{y}{1-y} = \pm C^{-1}e^x &\qekv y(x) = \frac{e^x}{e^x \pm C}\,, \quad C>0 \\
                              &\qekv y(x) = \frac{e^x}{e^x+C}\,,\quad C\neq 0.
\end{align*}
Rajoitus $C \neq 0$ voidaan poistaa saadusta yleisen ratkaisun lausekkeesta, koska myös $y(x)=1$
on ratkaisu. Tämä on kaikkien ratkaisujen yhteinen raja-arvo (asymptoottinen ratkaisu), kun
$x\kohti\infty$. Ratkaisu on edelleen myös $y(x)=0$, joka saadaan yleisen ratkaisulausekkeen
raja-arvona, kun $C\kohti\infty$. \loppu
\begin{Exa}: \vahv{Logistinen kasvumalli}. \index{zza@\sov!Logistinen kasvumalli} \ Jos
fysikaalisen suureen (esim.\ väkiluvun) kasvua ajan $t$ funktiona kuvaa autonominen
differentiaaliyhtälö
\[
y'=ay-by^2,
\]
missä $a$ ja $b$ ovat (dimensiottomina) positiivisia vakioita, niin kasvumallia 
(myös differentiaaliyhtälöä) sanotaan
\index{differentiaaliyhtälö!j@logistinen} \index{logistinen DY}%
\kor{logistiseksi}. Sijoituksella $y(t)=(a/b)\,u(t)$ logistinen DY muuntuu muotoon
\[
u'=a(u-u^2).
\]
Ratkaisu separoimalla (vrt.\ edellinen esimerkki):
\[
a\,dt = \frac{du}{u(1-u)} \qimpl u(t)=\frac{e^{at}}{e^{at} + C}\,, \quad C\in\R. 
\]
Ratkaisu on fysikaalisesti järkevä (kasvava) vain kun $C>0$. Tässä tapauksessa voidaan 
kirjoittaa $C=e^{at_0},\ t_0\in\R$. Kun vielä merkitään $\tau=1/a$ (aikavakio), niin ratkaisut
saadaan muotoon
\[
y(t)=A\,Y\left(\frac{t-t_0}{\tau}\right), \quad A=a/b, \quad \tau=1/a, \quad 
Y(t)=\frac{e^{t}}{e^{t}+1} \quad (t_0\in\R). \loppu
\]
\end{Exa}
\begin{figure}[H]
\setlength{\unitlength}{1cm}
\begin{center}
\begin{picture}(10,5)(0,-0.5)
\put(-1,0){\vector(1,0){12}} \put(10.8,-0.5){$t$}
\put(0,-1){\vector(0,1){5}} \put(0.2,3.8){$y(t)$}
\put(0,3){\line(-1,0){0.1}} \put(-0.6,2.9){$A$}
\put(5,0){\line(0,-1){0.1}} \put(4.9,-0.5){$t_0$}
\put(3,0){\line(0,-1){0.1}} \put(2.6,-0.5){$t_0-\tau$}
\put(7,0){\line(0,-1){0.1}} \put(6.6,-0.5){$t_0+\tau$}
\curve(
-1.0,  0.1423,
 0.0,  0.2276,
 1.0,  0.3576,
 2.0,  0.5473,
 3.0,  0.8068,
 4.0,  1.1326,
 5.0,  1.5000,
 6.0,  1.8674,
 7.0,  2.1932,
 8.0,  2.4527,
 9.0,  2.6424,
10.0,  2.7724,
11.0,  2.8577)
\end{picture}
\end{center}
\end{figure}

\subsection{Differentiaaliyhtälö $\,y'=f(y/x)$}
\index{differentiaaliyhtälö!k@tasa-asteinen|vahv}
\index{tasa-asteinen DY|vahv}

\kor{Tasa-asteinen} differentiaaliyhtälö $y'=f(y/x)$ muuntuu sijoituksella
\[
u(x)=y(x)/x \ \impl \ y(x)=xu(x) \ \impl \ y'=xu'+u
\]
yhtälöksi
\[
xu'=f(u)-u.
\]
Tämä on separoituva, joten tasa-asteinen yhtälö on \pain{se}p\pain{aroituvaksi} 
p\pain{alautuva}. Jos $f(u_0)-u_0=0$, niin erikoisratkaisu on
\[
u(x)=u_0 \qimpl y(x)=u_0x.
\]
Yleinen ratkaisu löydetään separoimalla.
\begin{Exa}
Ratkaise differentiaaliyhtälö
\[
y'=\frac{x+y}{x-y}\,.
\]
\end{Exa}
\ratk Tämä on tasa-asteinen, joten tehdään sijoitus $u=y/x$\,:
\[
\impl\ x\frac{du}{dx} \,=\, \frac{1+u}{1-u}-u \,=\, \frac{1+u^2}{1-u}\,.
\]
Separoimalla ja integroimalla seuraa
\begin{align*}
\int\frac{dx}{x} = \ln\abs{x} &= \int\frac{1-u}{1+u^2}\,du \\
                              &= \Arctan u-\frac{1}{2}\ln (1+u^2)+\ln C \\
              \ekv\ \Arctan u &= \ln\left(\frac{\abs{x}\sqrt{1+u^2}}{C}\right) \\
     \ekv\ \Arctan\frac{y}{x} &= \ln\left(\frac{\sqrt{x^2+y^2}}{C}\right)\quad (C>0).
\end{align*}
Siirtymällä napakoordinaatistoon saadaan ratkaisulle helpompi muoto
\[
\varphi = \ln\frac{r}{C} \ \ekv \ r=Ce^\varphi\quad (C>0).
\]
Ratkaisut ovat nk.\
\index{logaritminen spiraali}%
\kor{logaritmisia spiraaleja}. \loppu
\input{plots/logspiraalit.tex}

\subsection{Differentiaaliyhtälö $\,y'=f(ax+by+c)$}
\index{differentiaaliyhtälö!l@$y'=f(ax+by+c)$|vahv}

Differentiaaliyhtälö $y'=f(ax+by+c)$, missä $a,b,c\in\R$ ja $b \neq 0$, on myös separoituvaksi
palautuva. Tässä luonteva sijoitus on
\[
u=ax+by+c \ \impl \ y=\frac{1}{b}(u-ax-c),
\]
jolloin
\[
\frac{dy}{dx}=\frac{1}{b}\left(\frac{du}{dx}-a\right),
\]
ja näin ollen $u$:lle saadaan separoituva (itse asiassa autonominen) DY
\[
u'=bf(u)+a.
\]
\begin{Exa}
Ratkaise differentiaaliyhtälö
\[
y'=(x-y)^2+1.
\]
\end{Exa}
\ratk Sijoituksella
\[
x-y=u \ \ekv \ y=x-u \ \impl \ y'=1-u'
\]
saadaan differentiaaliyhtälöksi
\[
u'=-u^2.
\]
Separoimalla saadaan yleiseksi ratkaisuksi $u(x)=1/(x+C)$, joten alkuperäisen yhtälön yleinen
ratkaisu on
\[
y(x)=x-\frac{1}{x+C}\,.
\]
Tämän lisäksi on muunnetun yhtälön ratkaisua $u(x)=0$ vastaava erikoisratkaisu
\[
y(x)=x. \loppu
\]

\Harj
\begin{enumerate}

\item 
Määritä seuraavien separoituvien tai sellaisiksi palautuvien differentiaaliyhtälöiden
yleiset ratkaisut tai ratkaise alkuarvotehtävä. Alkuarvotehtävän tapauksessa selvitä myös,
millä välillä ratkaisu on pätevä.
\begin{align*}
&\text{a)}\ \ x^2 y'=y^2 \qquad 
 \text{b)}\ \ y'=-2xy \qquad 
 \text{c)}\ \ y'=(1-y)^2 \qquad
 \text{d)}\ \ y'=y^2-1 \\[2mm]
&\text{e)}\ \ (1+x)y'=1+y \qquad
 \text{f)}\ \ 1+y^2-xy'=0 \qquad
 \text{g)}\ \ (1+x^3)y'=x^2y \\[2mm]
&\text{h)}\ \ (1-x^2)y'=1-y^2 \qquad
 \text{i)}\ \ y'=\tan y \qquad
 \text{j)}\ \ xy'=\cot y \\[2mm]
&\text{k)}\ \ (x-3)y'=-y,\ y(-1)=1 \qquad
 \text{l)}\ \ y'=y^2,\ y(0)=4 \\[2mm]
&\text{m)}\ \ y'+5x^4y^2=0,\ y(0)=1 \qquad
 \text{n)}\ \ y'-5x^4y^2=0,\ y(0)=1 \\[2mm] 
&\text{o)}\ \ y'\Arctan y=1,\ y(1)=1 \qquad
 \text{p)}\ \ y'\sin x=y\ln y,\ y(\pi/2)=1 \\[2mm]
&\text{q)}\ \ (1+e^x)yy'=e^x,\ y(1)=1 \qquad
 \text{r)}\ \ \cos^2 x\cos(\ln y)y'=y,\ y(\pi/4)=2 \\[1mm]
&\text{s)}\ \ y'=\frac{y}{x-y} \qquad
 \text{t)}\ \ (3x+y)y'=y \qquad
 \text{u)}\ \ xyy'=x^2+y^2 \\
&\text{v)}\ \ (2x^2+y^2)y'=2xy \qquad
 \text{x)}\ \ xy'=y+\tan\frac{y-x}{x} \qquad
 \text{y)}\ \ y'=e^{x-2y} \\  
&\text{z)}\ \ xy'=xe^{y/x}+y,\ y(1)=0 \qquad
 \text{å)}\ \ xy'=y\ln\frac{y}{x},\ y(1)=1 \\
&\text{ä)}\ \ (x+y)y'=1-x-y,\ y(0)=1 \qquad 
 \text{ö)}\ \ y'=(2x+y+3)^2,\ y(0)=0
\end{align*}

\item
Ratkaise alkuarvotehtävä $\,e^y y'=x+e^y-1,\ y(0)=y_0$ sijoituksella $u=x+e^y$. Millä $y_0$:n
arvoilla ratkaisu on pätevä koko $\R$:ssä?

\item
Määritä differentiaaliyhtälön $y'=2x\abs{y-1}$ ratkaisukäyrät, jotka sivuavat $x$-akselia.

\item
Määritä seuraavien käyräparvien kohtisuorat leikkaajat.
\[
\text{a)}\,\ y=Cx^2 \quad\
\text{b)}\,\ y=e^x+C \quad\
\text{c)}\,\ x^2+2y^2=C^2 \quad\
\text{d)}\,\ e^{x+y}=(x+C)^2
\]

\item
Määritä käyrät, joilla on ominaisuus: käyrän ja sen normaalin leikkauspisteen etäisyys ko.\
normaalin ja $x$-akselin leikkauspisteestä $=a=$ vakio.

\item \index{integraaliyhtälö}
Määritä välillä $[0,\infty)$ jatkuva funktio $y(x)$, joka toteuttaa \kor{integraaliyhtälön}
\[
1+\int_0^x \frac{y(t)}{t^2+1}\,dt = y(x), \quad x \ge 0.
\]

\item \index{zzb@\nim!Hypetia, Utopia ja Apatia}
(Hypetia, Utopia ja Apatia) H:n, U:n ja A:n valtakunnissa oli elintaso $=1$ vuonna 2000 ja
$=1.01$ vuonna 2001. Ennusta ko.\ valtakuntien elintasot vuonna $2100$, kun tiedetään, että
elintason kasvunopeus on H:ssa suoraan verrannollinen elintason neliöön, U:ssa suoraan
verrannollinen elintasoon ja A:ssa kääntäen verrannollinen elintasoon.

\item 
Vuonna 1960 oli maapallon väkiluku 3.0 mrd ja ja kasvunopeus tuolloin 50 milj./v. Jos
oletetaan, että väkiluku $y(t) \kohti 12$ mrd kun $t\kohti\infty$, niin mikä olisi väkiluvun
pitänyt olla vuonna 2000 logistisen kasvumallin mukaan? (Todellinen väkiluku oli 6.1 mrd.)

\item
Laskuvarjohyppääjän putoamisnopeus noudattaa varjon auettua liikelakia $\,mv'=mg-av-bv^2$,
missä $m=80$ kg, $g=9.8\ \text{m/s}^2$, $a=120$ kg/s ja $b=4.0$ kg/m. Määritä $v(t),\ t \ge 0$
(aikayksikkö = s), kun varjon aukeamishetkellä $t=0$ putoamisnopeus on $20$ m/s. 
Miten rajanopeus $v_\infty=\lim_{t\kohti\infty}v(t)$ saadaan helpoimmin selville? Hahmottele
$v(t)$ aikaväleillä $[0,1]$ ja $[0,60]$.

\item
Avaruudessa suuntaan $-\vec k$ etenevät valonsäteet heijastuvat pyörähdyspinnasta
$S:\,z=u(r),\ r=\sqrt{x^2+y^2}\,$ siten, että jokainen heijastunut säde kulkee origon
kautta. Funktion $u(r)$ on tällöin toteutettava tasa-asteinen differentiaaliyhtälö
$\,ru'=\sqrt{r^2+u^2}+u$
(vrt.\ Harj.teht.\,\ref{derivaatta geometriassa}:\ref{H-dif-2: tutka}). Ratkaise!

\item (*)
Differentiaaliyhtälö
\[
y'=f\left(\frac{2x-y+1}{x-2y+1}\right)
\]
palautuu tasa-asteiseksi sijoituksilla $x=x_0+t$, $y=y_0+u$, kun $x_0$ ja $y_0$ valitaan
sopivasti. Ratkaise yhtälö tällä tavoin, kun $f(x)=x$.

\item (*)
Seuraavat differentiaaliyhtälöt palautuvat tasa-asteisiksi sijoituksilla \newline
$x=t^\alpha$, $y=z^\beta$, kun $\alpha$ ja $\beta$ valitaan sopivasti. Ratkaise! \newline
a)\ \  $42(x^2-xy^2)y'+y^3=0 \quad$ b)\ \ $(x^2y^2-1)y'+2xy^3=0$

\item (*)
Lieriön muotoisessa astiassa on vettä $200$ litraa. Astian pohjaan avataan aukko, jonka
pinta-ala $=20$ cm$^2$. Vesi purkautuu aukosta nopeudella $v=\sqrt{2gh}$, missä
$g=9.8$ m/s$^2$ ja $h=$ veden korkeus astiassa kyseisellä hetkellä. (Purkausnopeus oletetaan
samaksi aukon eri kohdissa.) Laske, missä ajassa astia tyhjenee, kun veden korkeus on aluksi
$\,h=1$ m.

\item (*)
Käyrällä $S: y=y(x),\ x \ge 0$, missä $y(x)$ on jatkuvasti derivoituva välillä $[0,\infty)$,
on ominaisuus: Käyrän kaarenpituus välillä $[0,x]$ on sama kuin käyrän ja $x$-akselin välisen
alueen pinta-ala ko.\ välillä. Määritä kaikki käyrät, jotka toteuttavat tämän ehdon ja lisäksi
alkuehdon $y(0)=2$.

\end{enumerate}