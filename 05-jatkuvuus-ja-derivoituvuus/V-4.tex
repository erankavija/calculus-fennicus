\section{Trigonometristen funktioiden derivointi} \label{kaarenpituus}
\alku

Trigonometriset funktiot reaalifunktioina perustuvat kaarenpituuden käsitteeseen
(yksikköympyrällä, ks.\ Luku \ref{trigonometriset funktiot}). Käsite on aiemmin määritelty
lyhyesti Luvussa \ref{geomluvut}. Seuraavassa tarkastellaan aluksi kaarenpituutta hieman
yleisemmältä kannalta ja johdetaan näiden tarkastelujen pohjalta raja-arvotulos
\[
\lim_{x \kohti 0} \frac{\sin x}{x} = 1.
\]
Trigonometristen funktioiden kaikki derivoimissäännöt ovat tästä perustuloksesta ja Luvun
\ref{derivaatta} säännöistä johdettavissa.

\subsection{Kaarenpituus}
\index{kaarenpituus|vahv}

Olkoon $f:\DF_f\kohti\R$, $\DF_f\subset\R$, jatkuva suljetulla välillä $[a,b]\subset \DF_f$.
Merkitään
\[
A=(a,f(a)), \quad B=(b,f(b))
\]
ja sanotaan, että euklidisen tason käyrä
\[
S=\{P=(x,y)\in\Ekaksi \mid x\in [a,b] \ \ja \ y=f(x)\}
\]
\index{kaari (käyrän)} \index{yksinkertainen!c@käyrä, kaari}%
on pisteitä $A,B$ yhdistävä (yksinkertainen) \kor{kaari} (tai käyrän kaari, engl.\ arc).
\begin{figure}[H]
\setlength{\unitlength}{1cm}
\begin{center}
\begin{picture}(10,5)(-1,-1)
\put(-1,0){\vector(1,0){10}} \put(8.8,-0.4){$x$}
\put(0,-1){\vector(0,1){5}} \put(0.2,3.8){$y$}
\put(0.9,1.9){$\bullet \ A$} \put(7.9,2.9){$\bullet \ B$}
\curve(
    1.0000,    2.0000,
    1.5000,    2.7635,
    2.0000,    3.0000,
    2.5000,    2.8848,
    3.0000,    2.5679,
    3.5000,    2.1733,
    4.0000,    1.8000,
    4.5000,    1.5211,
    5.0000,    1.3843,
    5.5000,    1.4117,
    6.0000,    1.6000,
    6.5000,    1.9202,
    7.0000,    2.3179,
    7.5000,    2.7130,
    8.0000,    3.0000)
\put(4.5,1.8){$S$}
\end{picture}
\end{center}
\end{figure}
Olkoon edelleen $\{P_0,\ldots,P_n\}$, $n\in\N$, äärellinen, järjestetty pistejoukko kaarella
$S$ siten, että
\[
P_i=(x_i,f(x_i)),\quad i=0,\ldots,n,
\]
missä
\[
a=x_0<x_1<\ldots <x_n=b,
\]
jolloin siis $P_0=A$, $P_n=B$. Merkitään tällaista pistejoukkoa symbolisesti $\mathcal{P}$:llä,
\[
\mathcal{P}=\{P_0,\ldots,P_n\}\subset S,
\]
ja kaikkien tällaisten pistejoukkojen joukkoa $\mathcal{M}$:llä. Joukko $\mathcal{M}$ on 
ylinumeroituva, sillä jo pisteellä $P_1$ on ylinumeroituva määrä vaihtoehtoja.

Pistejoukon $\mathcal{P}\in\mathcal{M}$ kautta kulkevaa murtoviivaa, päätepisteinä $A$ ja $B$, 
sanottakoon pisteistön $\mathcal{P}$ kautta kulkevaksi kaaren $S$ \kor{oikopoluksi}.
\begin{figure}[H]
\setlength{\unitlength}{1cm}
\begin{center}
\begin{picture}(10,5)(-1,-1)
\put(-1,0){\vector(1,0){10}} \put(8.8,-0.4){$x$}
\put(0,-1){\vector(0,1){5}} \put(0.2,3.8){$y$}
\put(0.9,1.9){$\bullet \ A$} \put(7.9,2.9){$\bullet \ B$}
\curve(
    1.0000,    2.0000,
    1.5000,    2.7635,
    2.0000,    3.0000,
    2.5000,    2.8848,
    3.0000,    2.5679,
    3.5000,    2.1733,
    4.0000,    1.8000,
    4.5000,    1.5211,
    5.0000,    1.3843,
    5.5000,    1.4117,
    6.0000,    1.6000,
    6.5000,    1.9202,
    7.0000,    2.3179,
    7.5000,    2.7130,
    8.0000,    3.0000)
\put(3.7,1.2){$S$}
\put(2.4,2.8){$\bullet$} \put(2.4,3.1){$P_1$}
\put(5.4,1.3){$\bullet$} \put(5.4,0.9){$P_2$}
\path(1,2)(2.5,2.88)(5.5,1.41)(8,3)
\end{picture}
\end{center}
\end{figure}
Jokaisella oikopolulla on (geometrinen) \kor{pituus}, jota merkitään symbolilla
$s_\mathcal{P}$ ja määritellään
\[
s_\mathcal{P}=\sum_{i=1}^n \sqrt{(x_i-x_{i-1})^2+[f(x_i)-f(x_{i-1})]^2}\,.
\]
\index{suoristuva (käyrä)}%
Sanotaan, että kaari $S$ on \kor{suoristuva} (engl. rectifiable), jos joukko
\[
\mathcal{S}=\{s_\mathcal{P} \ | \ \mathcal{P}\in\mathcal{M}\}
\]
on (reaalilukujoukkona) \pain{ra}j\pain{oitettu}. Tässä tapauksessa kaarelle $S$ voidaan
määritellä  \kor{kaarenpituus}. Nimittäin, koska rajoitetulla joukolla on pienin yläraja eli
supremum (vrt.\ Luku \ref{reaalilukujen ominaisuuksia}), niin luontevalta tuntuva määritelmä on
\[
\boxed{\kehys\quad S\text{:n pituus} = s = \sup\mathcal{S}. \quad}
\]
Pelkästään $f$:n jatkuvuudesta \pain{ei} seuraa, että kaari $S$ on suoristuva (vaikka
vasta\-esimerkkejä ei ole aivan helppo löytää). Jatkossa tehdään funktiosta $f$ sen vuoksi 
hieman voimakkaampi oletus, joka takaa kaarenpituuden olemassaolon. Oletus on lievennettävissä
koskemaan välin $[a,b]$ osavälejä (ks.\ alaviite jäljempänä), joten tavanomaisia funktioita 
ajatellen lisäoletus on melko viaton.
\begin{Lause} \label{kaarenpituuslause}
Jos $f$ on välillä $[a,b]$ jatkuva ja lisäksi monotoninen, niin kaari
$S=\{P=(x,y)\in\Ekaksi \mid x\in [a,b] \ja y=f(x)\}$ on suoristuva, ja kaarenpituudelle $s$
pätee $\,b-a \,\le\, s \,\le\, b-a + \abs{f(b)-f(a)}$.
\end{Lause}
\tod  Koska $f$ on monotoninen, niin $s_\mathcal{P}$:n lausekkeessa luvut $f(x_i)-f(x_{i-1})$
ovat samanmerkkisiä (tai nollia), jolloin epäyhtälöä $\,\sqrt{x^2+y^2}\le|x|+|y|\,$ ensin
soveltaen seuraa
\begin{align*}
s_\mathcal{P}\, &\le\, \sum_{i=1}^n [\,(x_i-x_{i-1})+|f(x_i)-f(x_{i-1})|\,] \\
               &=\, \sum_{i=1}^n (x_i-x_{i-1})+\left|\sum_{i=1}^n [f(x_i)-f(x_{i-1})]\right|
             \,=\, b-a+\abs{f(b)-f(a)}.
\end{align*}
Todetaan myös, että $s_\mathcal{P} \ge b-a\ \forall\ \mathcal{P}$, sillä pienin mahdollinen
$s_\mathcal{P}$ (jokaisella $\mathcal{P}$) saadaan, kun $f$ on vakio. Tehdyin oletuksin 
siis pätee
\[
b-a\ \le\ s_\mathcal{P}\ \le\ b-a+\abs{f(b)-f(a)} \quad \forall\ s_\mathcal{P}\in\mathcal{S}.
\]
Joukolle $\mathcal{S}$ on näin saatu alaraja ja yläraja, joten luku $s=\sup\mathcal{S}$ on
olemassa ja saadut rajat pätevät myös tälle luvulle. \loppu

Ellei $S$ itse satu olemaan murtoviiva (jolloin $s\in\mathcal{S}$), on kaarenpituus määrättävä
käytännössä konstruoimalla jono $\mathcal{P}_n\in\mathcal{M}$, $n=1,2,\ldots$ niin, että 
$s_{\mathcal{P}_n}=s_n\kohti s$, kun $n\kohti\infty$. Esimerkiksi jos valitaan 
$\mathcal{P}_n = \{a + k(b-a)/n,\ k = 0 \ldots n\}$, niin algoritmi toimiikin tavanomaisissa
tapauksissa (vrt.\ Harj.teht.\,\ref{H-V-4: numeerinen
kaarenpituus}).\footnote[2]{Kaarenpituuden laskemiseen palataan myöhemmin toisessa
asiayhteydessä. Tällöin myös näytetään, että kaarenpituus on tarkasteltavan välin suhteen
\kor{additiivinen}, ts.\ jos $a<c<b$, niin $s=s_1+s_2$, missä $s$, $s_1$ ja $s_2$ ovat
vastaavasti kaarenpituudet välillä $[a,b]$, $[a,c]$ ja $[c,b]$. Additiivisuuden ja Lauseen
\ref{kaarenpituuslause} perusteella käyrän $S: y=f(x)$ kaaren suoristuvuuteen välillä $[a,b]$
riittää jatkuvuusoletuksen lisäksi, että väli $[a,b]$ on jaettavissa äärellisen moneen
osaväliin, joilla $f$ on monotoninen.}

Jos koko välin $[a,b]$ sijasta tarkastellaan väliä $[a,x]$, $a<x \le b$, ja määritellään
\[
S_x=\{P=(t,f(t)) \ | \ a\leq t\leq x\},
\]
niin $S_x$:n kaarenpituudesta $s(x)$ tulee välillä $[a,b]$ määritelty funktio, kun vielä 
asetetaan $s(a)=0$. Koska $s(x_2)-s(x_1)$ tällöin merkitsee kaarenpituutta välillä $[x_1,x_2]$
(ks.\ alaviite), on Lauseen \ref{kaarenpituuslause} mukaan
\[
s(x_2)-s(x_1)\,\ge\,x_2-x_1,\quad a \le x_1 < x_2 \le b,
\]
joten $s(x)$ on välillä $[a,b]$ aidosti kasvava. Lisäksi nähdään Lauseen
\ref{kaarenpituuslause} yläraja-arviosta, että $s(x)$ on jatkuva välillä $[a,b]$.

Siirrytään nyt tarkastelemaan jatkon kannalta tärkeää erikoistapausta, missä $[a,b]=[0,1]$ ja 
funktio $f$ määritellään
\[
f(x)=1-\sqrt{1-x^2}, \quad x\in [0,1].
\]
Lauseen \ref{kaarenpituuslause} oletukset toteutuvat tällöin, ja $S$ on $x$-akselia origossa
sivuavan 1-säteisen ympyrän kaari.
\begin{figure}[H]
\setlength{\unitlength}{1cm}
\begin{center}
\begin{picture}(6,5)(-1,-1)
\put(-1,0){\vector(1,0){6}} \put(4.8,-0.4){$x$}
\put(0,-1){\vector(0,1){5}} \put(0.2,3.8){$y$}
\put(3,0){\line(0,-1){0.15}} \put(2.92,-0.5){$\scriptstyle{1}$}
\put(0,3){\line(-1,0){0.15}} \put(-0.4,2.9){$\scriptstyle{1}$}
\put(0,3){\arc{6}{0}{1.57}}
\put(-0.1,-0.1){$\bullet$} \put(2.9,2.9){$\bullet$}
\path(0,3)(2.25,1.05)(0,1.05)
\dashline{0.1}(2.25,1.05)(2.25,0)
\put(0,1.05){$\overbrace{\hspace{2.25cm}}^x$}
\put(1,0.5){$\scriptstyle{t}$} \put(3,2){$S$} \put(2.18,-0.2){$\scriptstyle{x}$}
\end{picture}
\end{center}
\end{figure}
Kun merkitään $t=$ kaarenpituus välillä $[0,x]$, niin (ks.\ kuvio)
\[
x=\sin t.
\]
Kun $x$:n ja $t$:n välille oletetaan tämä yhteys välillä 
$x\in[0,1]\ \ekv\ t\in[0,\tfrac{\pi}{2}]$, niin ko.\ välillä voidaan kirjoittaa
\[
f(x) \,=\, 1+\sqrt{1-x^2} \,=\, 1-\cos t
\]
ja arvioida Lauseen \ref{kaarenpituuslause} perusteella
\[
\begin{cases} \,t \ge x, \\ \,t \le x+f(x) \end{cases} \qekv
\begin{cases} \,t \ge \sin t, \\ \,t \le \sin t + 1-\cos t \end{cases}
\]
eli
\[
t-(1-\cos t) \,\le\, \sin t \,\le\, t, \quad t\in[0,\tfrac{\pi}{2}].
\]

Tässä on viitattu aiemmin (Luku \ref{geomluvut}) sovittuun merkintään ($\pi$:n määritelmä!)
\[
S\text{:n pituus}=\tfrac{\pi}{2}.
\]
Kun saadun epäyhtälön vasemmalla puolella käytetään trigonometrian kaavaa
$\,1-\cos t = 2\sin^2\tfrac{t}{2}$ (ks.\ Luku \ref{trigonometriset funktiot}) ja sovelletaan
epäyhtälön jälkimmäistä osaa, seuraa
\[
t-(1-\cos t) \,=\, t-2\sin^2\tfrac{t}{2} 
             \,\ge\, t-2\left(\tfrac{t}{2}\right)^2 
             \,=\, t-\tfrac{1}{2}t^2, \quad t\in[0,\pi].
\]
Yhdistämällä epäyhtälöt on tultu päätelmään
\begin{align*}
t-\tfrac{1}{2}t^2                &\,\le\ \sin t \,\le\ t, \quad t\in[0,\tfrac{\pi}{2}] \\[1mm]
\impl\quad\ 1-\tfrac{1}{2}\,t    &\,\le\, \frac{\sin t}{t} \,\le\, 1,
                                    \quad t\in(0,\tfrac{\pi}{2}] \\
\impl\quad 1-\tfrac{1}{2}\abs{t} &\,\le\, \frac{\sin t}{t} \,\le\, 1, 
                                    \quad 0<\abs{t}\le\tfrac{\pi}{2}.
\end{align*}
Tässä viimeinen päättely perustui funktion $\sin t/t$ parillisuuteen.

Jos nyt $t_n\in [-\tfrac{\pi}{2},\tfrac{\pi}{2}]$, $t_n\neq 0$ ja $t_n\kohti 0$, niin viimeksi
kirjoitetun epäyhtälön perusteella
\[
1\ \ge\ \frac{\sin t_n}{t_n}\ \ge\ 1-\tfrac{1}{2}\abs{t_n} \kohti 1.
\]
Raja-arvon määritelmän (Määritelmä \ref{funktion raja-arvon määritelmä}) perusteella on 
johdettu raja-arvotulos
\begin{equation} \label{sinin raja-arvotulos}
\boxed{\quad \lim_{t\kohti 0}\,\frac{\ykehys\sin t}{\akehys t}=1. \quad}
\end{equation}
Kun sovelletaan tätä tulosta, kaavaa $1-\cos t=2\sin^2\tfrac{t}{2}$ ja raja-arvojen
yhdistelysääntöjä (Lause \ref{funktion raja-arvojen yhdistelysäännöt}), niin seuraa
\begin{equation} \label{kosinin raja-arvotulos}
\boxed{\quad \lim_{t\kohti 0}\,\frac{\ykehys 1-\cos t}{\akehys t^2}=\frac{1}{2}\,. \quad}
\end{equation}
Edellä olennaisesti johdettiin myös seuraavat, kaikilla $t\in\R$ pätevät epäyhtälöt:
\[
\boxed{\kehys\quad \abs{\sin t}\,\le\,\abs{t}, \quad 
                   0\,\le\,1-\cos t \le \frac{1}{2}\,t^2, \quad t\in\R. \quad}
\]
Kun raja-arvotulokset \eqref{sinin raja-arvotulos},\,\eqref{kosinin raja-arvotulos} yhdistetään
trigonometrisiin yhteenlaskukaavoihin
\begin{align*}
\sin (x+t) &=\sin x\cos t+\cos x\sin t, \\
\cos (x+t) &=\cos x\cos t-\sin x\sin t
\end{align*}
(ks.\ Luku \ref{trigonometriset funktiot}) saadaan yleisemmät raja-arvotulokset
(Harj.teht.\,\ref{H-V-4: sinin ja kosinin erotusosamäärät})
\begin{equation} \label{sinin ja kosinin erotusosamäärät}
\boxed{\begin{aligned}
\quad \lim_{t\kohti 0}\,\frac{\ykehys \sin(x+t)-\sin x}{t} &= \cos x, \\
       \lim_{t\kohti 0}\,\frac{\cos(x+t)-\cos x}{\akehys t}&= -\sin x. \quad
\end{aligned}}
\end{equation}

\subsection{Trigonometristen funktioiden derivaatat}
\index{derivoimissäännöt!f@trigonometriset funktiot|vahv}

Raja-arvotulosten \eqref{sinin ja kosinin erotusosamäärät} perusteella funktioiden $\sin$ ja 
$\cos$ derivoimissäännöt ovat
\[ \boxed{ \begin{aligned}
\quad \dif\sin x &= \cos x, \quad x\in\R, \\
      \dif\cos x &= -\sin x, \quad x\in\R. \quad
\end{aligned} } \]
Näiden sääntöjen ja Luvun \ref{derivaatta} sääntöjen perusteella on
\[
\dif\left(\frac{\sin x}{\cos x}\right)=1+\frac{\sin^2 x}{\cos^2 x}\,.
\]
Tämä ja vastaava $\cot$-funktion derivoimissääntö annetaan yleensä muodossa
\[ \boxed{ \begin{aligned}
\quad \dif\tan x\ &=\ 1+\tan^{\ykehys 2}x\ =\ \dfrac{1}{\cos^2 x}\,, \\
      \dif\cot x\ &=\ -(1+\cot^2 x)\ =\ -\dfrac{1}{\akehys \sin^2 x}\,. \quad
\end{aligned} } \]

Trigonometristen käänteisfunktioiden derivaatat voidaan laskea edellisen luvun säännöllä
\eqref{D5}. Esim.\ jos $y\in(-1,1)$ ja $\Arcsin y=x\in(-\tfrac{\pi}{2},\tfrac{\pi}{2})$,
niin mainittu sääntö yhdessä trigonometrian kaavojen kanssa antaa
\[
\frac{d}{dy}\Arcsin y = \frac{1}{\dif\sin x} 
                      = \frac{1}{\cos x} 
                      = \frac{1}{\sqrt{1-\sin^2 x}}
                      = \frac{1}{\sqrt{1-y^2}}\,.
\]
Kun $\Arccos$ ja $\Arctan$ derivoidaan vastaavalla tavalla on tulos
(Harj.teht.\,\ref{H-V-4: Arccos ja Arctan})\,:
\[ \boxed{ \begin{aligned}
\quad \dif\Arcsin x\ &=\ \dfrac{1}{\sqrt{1-x^2}}\,, \quad x\in (-1,1), \\
      \dif\Arccos x\ &=\ -\dfrac{1}{\sqrt{1-x^2}}\,, \quad x\in (-1,1), \quad \\
      \dif\Arctan x\ &=\ \dfrac{1}{1+x^2}\,, \quad x\in\R.
\end{aligned} } 
\]

Em.\ derivoimissääntöjen antamia funktioita edelleen derivoimalla nähdään, että kaikki 
säännöissä mainitut funktiot ovat mielivaltaisen monta kertaa derivoituvia koko
määrittelyjoukossaan, jos tämä koostuu avoimista väleistä, muuten kaikilla määrittelyjoukon
avoimilla osaväleillä. Sama pätee myös yleisemmille trigonometrisille funktioille, jotka
saadaan yhdistelemällä perusfunktioita laskuoperaatioilla tai yhdistettyinä funtioina.
\begin{Exa} Funktioiden $\,\sin\,$ ja $\,\cos\,$ sekä edellisen luvun dervioimissääntöjen
nojalla on
\[
\dif\,\frac{\sin x}{1+\cos x} 
       \,=\, \frac{\cos x}{1+\cos x}+\frac{\sin^2x}{(1+\cos x)^2}
       \,=\, \frac{\cos x+\cos^2x+\sin^2x}{(1+\cos x)^2}
       \,=\, \frac{1}{1+\cos x}\,.
\]
Lasku on pätevä jokaisella $x\in\R$, jolla $\cos x \neq -1$, eli derivoitavan funktion
koko määrittelyjoukossa. \loppu
\end{Exa}
\begin{Exa} Funktiot $\,u(x)=\cos x\,$ ja $\,v(x)=\sin x\,$ ovat (ainakin eräs) ratkaisu
ongelmaan
\[
\begin{cases} \,u'=-v,\ v'=u, \quad x\in\R \\ \,u(0)=1,\ v(0)=0 \end{cases}
\]
(Ongelma on ratkaistavissa toisellakin tavalla, ks.\
Harj.teht.\,\ref{derivaatta}:\ref{H-V-3: cos ja sin potenssisarjoina}b.) \loppu
\end{Exa}
 

\Harj
\begin{enumerate}

\item \label{H-V-4: numeerinen kaarenpituus}
Tutki, kuinka tarkasti kaaren
\[
S=\{P=(x,y) \mid y=x^2\ \ja\ x\in[0,1]\}
\]
pituus saadaan lasketuksi käyttämällä pisteistöä $\{(x_i,f(x_i)),\ i=0 \ldots 10\}$, missä 
$x_i=i/10$. Tarkka arvo on $s=1.47894..$

\item
Origosta siirrytään käyrää $r=4\cos\varphi$ (napakoordinaatit) seuraten pisteeseen, jonka
$x$-koordinaatti $=3$. Mikä on matkan pituus lyhintä reittiä?

\item \label{H-V-4: sinin ja kosinin erotusosamäärät}
Johda raja-arvotulokset \eqref{sinin ja kosinin erotusosamäärät}.

\item
Näytä, että jos $a,b\in\R$ ja $b \neq 0$, niin 
$\,\displaystyle{\lim_{x \kohti 0}\frac{\sin ax}{\sin bx}=\frac{a}{b}}\,$.

\item
Määritä raja-arvot
\begin{align*}
&\text{a)}\ \lim_{x \kohti 0} \frac{x\sin x}{1-\cos x} \qquad
 \text{b)}\ \lim_{x \kohti 0} \frac{x\sin 2x}{1-\cos 2x} \qquad
 \text{c)}\ \lim_{x \kohti 0^+} \frac{x-\sqrt{x}}{\sqrt{\sin x}} \\
&\text{d)}\ \lim_{x\kohti\frac{\pi}{2}}(1+\cos 2x)\tan^2 x \qquad
 \text{e)}\ \lim_{x\kohti\infty} x\sin\frac{1}{x}
\end{align*}

\item \label{H-V-4: Arccos ja Arctan}
Johda funktioiden $\Arccos$ ja $\Arctan$ derivoimissäännöt.

\item
Derivoi ja sievennä lopputulos:
\begin{align*}
&\text{a)}\ \frac{1-\cos x}{\sin x} \qquad
 \text{b)}\ \frac{1+\cos x}{1-\cos x} \qquad
 \text{c)}\ \frac{\sin x-\cos x}{\sin x+\cos x} \\[1mm]
&\text{d)}\ \Arcsin(\cos x) \qquad
 \text{e)}\ \Arccos(\sin x) \qquad
 \text{f)}\ \Arctan(\cot x)
\end{align*} 

\item
Määritä implisiittifunktion $y(x)$ derivaatta annetussa pisteessä:
\begin{align*}
&\text{a)}\,\ y+2\sin y+\cos y=x,\ \ (x,y)=(1,0) \\
&\text{b)}\,\ 2x+y-\sqrt{2}\sin(xy)=\frac{\pi}{2}\,,\ \ (x,y)=\left(\frac{\pi}{4}\,,1\right) \\
&\text{c)}\,\ x\sin(xy-y^2)=x^2-1,\ \ (x,y)=(1,1) \\
&\text{d)}\,\ \tan(xy^2)=\frac{2xy}{\pi}\,,\ \ (x,y)=\left(-\pi,\frac{1}{2}\right)
\end{align*}

\item
Näytä induktiolla, että $\dif^n\tan x=p(\tan x)$, missä $p$ on polynomi astetta $n+1$ ja
muotoa $\,p(t)=n!\,t^{n+1}\,+\,c_{n-1}\,t^{n-1}\,+\,c_{n-3}\,t^{n-3}\,+\,\ldots$

\item
Totea funktio
\[
f(x)=\begin{cases} 
     x^2\sin\dfrac{1}{x^2}\,, &\text{kun}\ x \neq 0, \\ 0, &\text{kun}\ x=0
     \end{cases} \]
esimerkiksi funktiosta, joka on derivoituva jokaisessa pisteessä $x\in\R$, mutta derivaatta
ei ole jokaisessa pisteessä jatkuva. Hahmottele $f$:n ja $f'$:n kuvaajat!

\item (*)
Olkoon $s_n$ käyrän $\,S: y=x^n\,$ kaarenpituus välillä $[0,1]$, kun $n\in\N$. Näytä, että
$\lim_n s_n = 2$.

\item (*)
Näytä sopivalla muuttujan vaihdolla: \vspace{1mm}\newline
$\D
\text{a)}\ \ \lim_{x\kohti\infty} x\left(\frac{\pi}{2}-\Arctan x\right)=1 \qquad 
\text{b)}\ \ \lim_{x \kohti 1^-} \frac{\pi-2\Arcsin x}{\sqrt{1-x}}=2\sqrt{2}$

\end{enumerate}