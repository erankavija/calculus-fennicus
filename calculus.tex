\documentclass[12pt,a4paper,finnish]{book}
\usepackage[finnish]{babel}
\usepackage[utf8x]{inputenc}
\usepackage[T1]{fontenc}
\usepackage{amsmath,amssymb}
\usepackage{theorem}
\usepackage{graphicx}
\usepackage{apropos}
\usepackage{epsfig}
\usepackage{multicol}
\usepackage{amscd}
\usepackage{curves}
\usepackage{epic}
\usepackage{eepic}
\usepackage{float}
\usepackage{import}
\usepackage{rotating}
\usepackage{makeidx}

% Riittävä sidontamarginaali. Huom! Textwidth määriteltävä täällä,
% jottei sotke geometry-paketin toimintaa.
\usepackage[hcentering, textwidth=35em,bindingoffset=23mm]{geometry}

\setlength{\parindent}{0mm}
\setlength{\parskip}{\medskipamount}

\usepackage{jpheader}

% Hyperlinkit
\usepackage[colorlinks=true,linkcolor=blue,breaklinks=true]{hyperref}
\setcounter{secnumdepth}{1}
\setcounter{tocdepth}{2}

\renewcommand{\thechapter}{\Roman{chapter}}
\renewcommand{\thefootnote}{\fnsymbol{footnote}}
\renewcommand{\theequation}{\arabic{equation}}

\theoremheaderfont{\rmfamily\scshape}

%\includeonly{etukansi,esipuhe}
%\includeonly{I-0,I-1,I-2,I-3,I-4,I-5,I-6,I-7,I-8,I-9,I-10,I-11,I-12,
%II-0,II-1,II-2,II-3,II-3,II-4,II-5,II-6,II-7,II-8,
%III-0,III-1,III-2,III-3,III-4,
%IV-0,IV-1,IV-2,IV-3,IV-4,IV-5,
%V-0,V-1,V-2,V-3,V-4,V-5,V-6,V-7,V-8,V-9,
%exp-0,exp-1,exp-2,exp-3,exp-4,
%dif-0,dif-1,dif-2,dif-3,dif-4,dif-5,dif-6,
%int-0,int-1,int-2,int-3,int-4,int-5,int-6,int-7,int-8,int-9,
%dy-0,dy-1,dy-2,dy-3,dy-4,dy-5,dy-6,dy-7,dy-8,
%m-0,m-1,m-2,m-3,m-4,m-5,m-6,m-7,m-8,
%udif-0,udif-1,udif-2,udif-3,udif-4,udif-5,udif-6,udif-7,udif-8,udif-9,
%eig-0,eig-1,eig-2,eig-3,eig-4,
%uint-0,uint-1,uint-2,uint-3,uint-4,uint-5,uint-6,uint-7,
%Gauss-0,Gauss-1,Gauss-2,Gauss-3,Gauss-4,Gauss-5}

\makeindex

\begin{document}

\frontmatter

\include{etukansi}
\include{esipuhe}

\tableofcontents

\mainmatter

%Esimerkkien numeroinnin makroja
\newcommand{\alku}{\setcounter{Exa}{0} \setcounter{equation}{0}}
\newcommand{\jatko}{\addtocounter{Exa}{-1}}
\newcommand{\seur}{\addtocounter{Exa}{1}}

%Hakemison merkintöjä
\index{alkeisfunktio!a@nimetty|see{funktio C}}
\index{avaruuskäyrä|see{käyrä}}
\index{differentiaaliyhtälö!p@lineaarinen|see{lineaarinen DY}}
\index{divergoiminen|see{hajaantuminen}}
\index{funktio A!k@operaattori|see{differentiaaliop.}}
\index{homogeeninen DY|see{lineaarinen DY}}
\index{kasvava|see{monotoninen}}
\index{konvergoiminen|see{suppeneminen}}
\index{kosini ($\cos$)|see{funktio C}}
\index{kosekantti ($\csc$)|see{funktio C}}
\index{kotangentti ($\cot$)|see{funktio C}}
\index{kuvaus|see{funktio A}}
\index{kzyyrzy@käyräintegraali|see{viivaintegraali}}
\index{lineaarikombinaatio|see{l.-yhdistely}}
\index{mittasuhde|see{muuntosuhde}}
\index{nabla|see{differentiaalioperaattori}}
\index{operaattori!b@muu|see{differentiaalioperaattori}}
\index{ortonormeerattu kanta|see{kanta}}
\index{sekantti ($\sec$)|see{funktio C}}
\index{sijoitus|see{muuttujan vaihto}}
\index{sini ($\sin$)|see{funktio C}}
\index{siszy@sisätulo|see{skalaaritulo}}
\index{suurennussuhde|see{muuntosuhde}}
\index{tangentti ($\tan$)|see{funktio C}}
\index{tasokäyrä|see{käyrä}}
\index{tieintegraali|see{polkuintegraali}}
\index{tzy@täydellinen DY|see{lineaarinen DY}}
\index{vzy@vähenevä|see{monotoninen}}

\chapter{Luvut ja lukujonot}

''Jumala loi luonnolliset luvut --- kaikki muu on ihmistyötä'' oli saksalaisen matemaatikon 
\index{Kronecker L.}%
\hist{Leopold Kronecker}in (1823-1891) aforismi ja kannanotto koskien matematiikan alkuperää.
--- Epäilemättä lukumäärää tai järjestystä kuvaavien sanojen 'yksi', 'kaksi', jne eriytyminen
luonnollisessa kielessä on ollut eräs lähtökohta ja perusedellytys matematiikan syntymiselle.
Merkkejä lukujen ja suhteiden tajusta on säilynyt jopa 30 000 vuoden takaa, joten matematiikan
voi perustellusti katsoa olevan ihmisen kulttuurissa \kor{sui generis}, omaa lajiaan.
Kulttuuri-ilmiönä matematiikkaa voi jossain määrin ymmärtää rinnastamalla se muihin
kulttuurin lajeihin. Esimerkiksi jos tarkastellaan matematiikan historiaa ja kumulatiivista
rakennetta, tai käyttötapoja, voi nähdä monia yhtymäkohtia muihinkin mahtaviin ihmishengen
ilmentymiin, erityisesti luonnollisiin kieliin ja tekniikkaan.

Matematiikkaa voi kuitenkin syvemmin oppia ymmärtämään vain matematiikkaa opiskelemalla.
Opiskelu aloitetaan tässä luvussa 'alusta', eli (matemaattisen) \kor{luvun} käsitteestä. Luvun
punaisena lankana on \kor{reaalilukujen} konstruoiminen lähtien 'järjellisistä' eli
\kor{rationaalisista} luvuista. Matematiikan historiassa reaaliluvun käsite on täsmentynyt
lopullisesti vasta 1800-luvulla. Historiallisena ongelmana --- ja edelleen ongelmana
opetuksessa --- on käsitteeseen väistämättä (muodossa tai toisessa) sisältyvä 'loputtomuuden'
ajatus. Tässä tekstissä reaalilukuja lähestytään rationaalilukujen muodostamien
\kor{lukujonojen}, ja näiden erikoistapausten, \kor{äärettömien desimaalilukujen} kautta.
Reaaliluvun käsite tulee tällä tavoin liitetyksi 'oikeaan laskemiseen' ja myös tavanomaisiin
numeromerkintöihin. Päättymättömät lukujonot tulevat jatkossa käyttöön monessa muussakin
yhteydessä. Niihin perustuvat viime kädessä sekä monet laskentamenetelmät (algoritmit) että
reaalilukuihin perustuvan matemaattisen analyysin eli \kor{reaalianalyysin} käsitteet.  % Luvut ja lukujonot

\section{Katsaus rationaalilukuihin}  \label{ratluvut}
\index{rationaaliluvut|vahv}
\alku
\sectionmark{Rationaaliluvut}

\index{luonnolliset luvut}%
\kor{Luonnollisten lukujen} joukkoa merkitsemme
\[
\N = \{1,2,3,4,\ldots\}.
\]
Yleisemmin 
\index{joukko} \index{alkio (joukon)}%
\kor{joukko} (engl. set) koostuu \kor{alkioista} (engl. element), joita joukon \N\ 
tapauksessa ovat luvut $1$, $2$, $3$ jne. Merkinnät
\[
8 \in \N, \quad \infty \not\in \N
\]
luetaan 'kahdeksan kuuluu \N:ään, 'ääretön ei kuulu \N:ään', tarkoittaen yksinkertaisesti, että
\N:ssä on alkio nimeltä 'kahdeksan' mutta ei alkiota nimeltä 'ääretön'.

Joukko $A = \{1,2,3\}$ on \N\ :n 
\index{osajoukko}%
\kor{osajoukko} (engl. subset). Merkitään
\[
A \subset \N \quad \text{tai} \quad \N \supset A
\]
ja luetaan '$A$ kuuluu/sisältyy \N\ :ään' tai '\N\ sisältää $A$:n'. Yleisesti merkintä 
$A \subset B$ tarkoittaa, että jokainen $A$:n alkio on myös $B$:n alkio, toisin sanoen pätee
\[
x \in A \qimpl x \in B.
\]
Tässä '$\impl$' on logiikan symboleihin kuuluva nk. 
\index{implikaatio}
\kor{implikaationuoli}: Merkintä $P \impl Q$
luetaan '$P$:stä seuraa $Q$', tai 'jos $P$, niin $Q$'. Symboli '$\subset$' (kuuluminen 
osajoukkona) määritellään siis symbolin '$\in$' (kuuluminen alkiona) avulla. Jälkimmäinen 
symboli on nk.\ 
\index{primitiivi}
\kor{primitiivi}, jota ei voi enää määritellä muiden symbolien avulla. Joukoista 
puhuttaessa voi siis merkintöjä $x \in A$ tai $x \not\in A$ pitää kaiken loogisen ajattelun 
lähtökohtana. 
\begin{Exa} Jos $A = \{2,1,1,2,1\}$ ja $B = \{1,2\}$, niin $A \subset B$, sillä jos $x \in A$,
niin on joko $x=1$, jolloin $x \in B$, tai $x=2$, jolloin myöskin $x \in B$. Samalla päättelyllä
nähdään, että pätee myös $B \subset A$. \loppu \end{Exa} 
Jos joukoille $A$ ja $B$ pätee sekä $A \subset B$ että $B \subset A$, kuten esimerkissä, niin 
sanotaan, että $A$ ja $B$ ovat (joukkoina)
\index{samastus '$=$'!a@joukkojen}%
\kor{samat}, ja merkitään $A=B$.\footnote[2]{Jos 
$A \subset B$, ja $A$ ei ole sama kuin $B$, sanotaan että $A$ on $B$:n \kor{aito} 
(engl.\ proper) osajoukko. Joissakin matemaattisissa teksteissä symbolilla '$\subset$' on tämä
rajatumpi merkitys, jolloin mahdollisuus $A=B$ varataan merkinnällä $A \subseteq B$. Tässä 
tekstissä siis pätee $A \subset A$. \index{aito osajoukko|av}} 
Logiikan ja joukko-opin merkintöjä käsitellään hieman laajemmin Luvussa \ref{logiikka}. 
Todettakoon tässä yhteydessä vielä osajoukon määrittelyssä tavallinen merkintä
\[ 
B\ =\ \{\,x \in A \mid E\,\}, 
\]
missä $E$ on jokin \kor{ehto}, joka jokaisen alkion $x \in A$ kohdalla joko toteutuu, jolloin 
$x \in B$, tai ei toteudu, jolloin $x \not\in B$.
\begin{Exa} \kor{Parilliset} ja \kor{parittomat} luonnolliset luvut määritellään $\N$:n osa-
joukkoina \index{parillinen, pariton!a@luku}%
\begin{align*}
A &= \{\,x \in \N \mid x=2\cdot y\,\ \text{jollakin}\ y\in\N\,\}\ =\ \{\,2,4,6,\ \ldots\,\}, \\
B &= \{\,x\in\N \mid x\not\in A\,\}\ =\ \{\,1,3,5,\ \ldots\,\} \loppu
\end{align*} 
\end{Exa}
Esimerkissä on jo viitattu luonnollisten lukujen joukossa (tunnetulla tavalla) määriteltyihin 
\index{laskuoperaatiot!a@rationaalilukujen|(}%
\kor{laskuoperaatioihin}, joita on kaksi:
\begin{align*}
\text{yhteenlasku:} \quad &x \in \N\ \ja\ y \in \N \quad \map \quad 
                             x + y \in \N \quad \quad \text{'$x$ plus $y$'} \\
\text{kertolasku:}  \quad &x \in \N\ \ja\ y \in \N \quad \map \quad 
                             x \cdot y\ \,\in \N \quad \quad \text{'$x$ kertaa $y$'}
\end{align*}
Tässä '$\&$' on logiikan merkintä, luetaan 'ja' (formaalimpi merkintä '$\wedge$'), ja 
'$\map$' tarkoittaa \kor{liittämistä}: Lukuihin $x$ ja $y$, eli \N:n \kor{lukupariin}, liitetään
(yksikäsitteinen) luku $x + y \in \N$, nimeltään $x$:n ja $y$:n \kor{summa}, ja 
$x \cdot y \in \N$, nimeltään $x$:n ja $y$:n \kor{tulo}. Laskuoperaatio tarkoittaa siis 
yksittäisten lukujen $x,y$ tapauksessa 'liittämissääntöä', yleisemmin 'liittämissäännöstöä'. 
Säännöt voidaan ajatella joko joistakin yleisemmistä periaatteista (laskusäännöistä) 
johdetuiksi, tai ne voidaan ymmärtää pelkästään 'luettelona', joka kattaa kaikki lukuparit. Kun
yhteen- ja kertolaskun sisältö on yhteisesti sovittu (ja peruskoulun oppikirjoihin painettu), on
luonnollisten lukujen perustalle rakennettu
\index{algebra}%
\kor{algebra}\footnote[2]{Algebra eli 
'kirjainlaskenta' on numeroilla laskemisen eli \kor{aritmetiikan} abstraktimpi muoto. Algebran
kehitys alkoi arabikulttuurin piirissä 1.\ vuosituhannella jKr.}, jota merkittäköön
$(\N,+,\cdot)$.

Seuraavat hyvin tunnetut peruslait ovat voimassa luonnollisten lukujen laskuoperaatioille:
\begin{itemize}
\item[(L1)] $\quad x+y = y+x$
\item[(L2)] $\quad x \cdot y = y \cdot x$
\item[(L3)] $\quad x+(y+z) = (x+y)+z$
\item[(L4)] $\quad x \cdot (y \cdot z) = (x \cdot y) \cdot z$
\item[(L5)] $\quad x \cdot (y+z) = x \cdot y + x \cdot z$
\item[(L6)] $\quad x \cdot 1 = x \ \ \forall x \in \N$
\end{itemize}
\index{vaihdantalaki} \index{liitzy@liitäntälaki} \index{osittelulaki}%
Tässä (L1),\,(L2) ovat \kor{vaihdantalait}, (L3),\,(L4) ovat \kor{liitäntälait} ja (L5) on 
\kor{osittelulaki}. Viimeksi mainitulla on myös rinnakkainen muoto
\begin{itemize}
\item[(L5')] $\quad\!\! (x+y) \cdot z = x \cdot z + y \cdot z$,
\end{itemize}
joka seuraa yhdistämällä (L2) ja (L5). Laissa (L6) '$\,\forall$' on logiikan symboli, joka 
luetaan 'kaikille'. Tämä laki ilmaisee, että luku  $1$ on kertolaskun nk. \kor{ykkösalkio}. 
Kuten nähdään jäljempänä, laskulait (L1)--(L6) ovat yhteisiä monille algebroille. Yleisemmissä
tarkasteluissa nämä lait otetaankin usein oletetuiksi peruslaeiksi eli
\index{aksiooma}%
\kor{aksioomiksi}.\index{Peano, G.|av} \index{Peanon aksioomat|av}%
\footnote[2]{Luonnollisten lukujen tapauksessa laskulait (L1)--(L6) ovat
johdettavissa luonnollisten lukujen perustana olevista \vahv{Peanon aksioomista}. Aksioomat
esitti italialainen matemaatikko \hist{Giuseppe Peano} (1858--1932) vuonna 1889. Peanon
aksioomat luonnollisille luvuille ovat: (P1) $1\in\N$. \,(P2) Jokaisella $x\in\N$ on
yksikäsitteinen \kor{seuraaja} $x'\in\N$. \,(P3) Luku $1$ ei ole minkään $x\in\N$ seuraaja. \,
(P4) Jokainen $x\in\N$ on korkeintaan yhden luvun seuraaja, ts.\ jos $x,y\in\N$ ja $x'=y'$, niin
$x=y$. \,(P5) Jos $S\subset\N$ ja pätee (i) $1\in S$ ja (ii) $x' \in S$ aina kun $x \in S$,
niin $S=\N$. Aksioomassa (P2) esiintyvä luvun seuraaja on tavanomaisen kielenkäytön mukaan
'seuraava luku'. Aksiooman (P3) mukaan luku $1$ on luonnollisista luvuista 'ensimmäinen', sen
sijaan 'viimeistä' lukua ei ole, koska jokaisella luvulla on seuraaja (aksiooma (P2)) eikä 
mikään luku voi esiintyä seuraajaketjussa kahdesti (aksioomat (P3) ja (P4)). Luonnollisia lukuja
on siis äärettömän monta. Aksiooma (P5) asettaa nk.\ \kor{induktioperiaatteen}, ks.\ Luku
\ref{jono} jäljempänä. 

Luonnollisten lukujen yhteenlasku ja kertolasku määritellään aksioomiin (P1)--(P5) perustuen
asettamalla kummallekin laskuoperaatiolle kaksi aksioomaa. Näiden perusteella minkä tahansa
halutun laskuoperaation tulos on määrättävissä lukujen $\,1$, $1' = 2$, $2' = 3$, jne avulla.
Yhteenlaskun aksioomat: \,(Y1) $\,x+1=x'$, \,(Y2) $\,x+y'=(x+y)'$. Kertolaskun aksioomat: \,
(K1) = (L6), \,(K2) $\,x \cdot y' = x \cdot y + x$.}

Laskulaeissa (L1)--(L6) on oletettu normaalit järjestyssäännöt laskuoperaatioiden yhdistelylle,
eli ensin lasketaan sulkeiden sisällä olevat lausekkeet, ja kertolaskut aina ennen yhteenlaskuja
mikäli sulkeita ei ole. Huomattakoon, että tässä ei ole kyse mistään ylimääräisistä 
olettamuksista, vaan esimerkiksi sääntö
\[
x + y \cdot z\ =\ x + (y \cdot z)
\]
on \pain{sulkeiden} p\pain{ois}j\pain{ättösääntö}, eli kyse on merkintäsopimuksesta. Samaa
operaatiota yhdisteltäessä on  liitäntälakien (L3),\,(L4) perusteella tulos aina sama
operaatioiden järjestyksestä riippumatta, joten sulkeet voidaan jättää pois (merkintäsopimus!)
ja kirjoittaa
\[
x + y + x + \ldots, \quad x \cdot y \cdot z \cdot \ldots
\]
Tavanomaisen lukujen algebran merkintäsopimuksiin kuuluu myös vapaus jättää kertomerkki
merkitsemättä sikäli kuin sekaannuksen vaaraa ei ole:
\[
x \cdot y = xy, \quad x \cdot 2 = 2x, \quad  2\cdot(1+2) = 2(1+2) = 2 \cdot 3 \neq 23.
\]
Peräkkäisiä tuloja laskettaessa on saman luvun $n$-kertaisesta ($n \in \N$) kertolaskusta tapana
käyttää nimitystä \kor{potenssiin korotus} ja käyttää merkintää
\[
\underset{(n\ \text{kpl})}{x \cdot x \cdots x} \quad = \quad x^n \quad 
                              \quad \text{'$x$ potenssiin $n$'} \quad (n \in \N).
\]
   
\subsection{Kokonaisluvut $\Z$}
\index{kokonaisluvut|vahv}%

\kor{Kokonaislukujen} joukko \Z\ on ensimmäinen askel siinä 'ihmistyössä', jossa 
lukujärjestelmää pyritään laajentamaan lähtökohtana luonnolliset luvut. Laajennus koostuu 
kahdesta osa-askeleesta, joista ensimmäinen on luvun \kor{nolla} ('ei mitään') ja toinen 
\kor{vastaluvun} ('montako puuttuu') käyttöönotto. Näiden ajatusten tuloksena syntyvää 
lukujoukkoa merkitään
\[
\Z = \{\ \ldots,-2,-1,0,1,2,\ \ldots\ \} \quad = \quad \{\ 0,\pm 1, \pm 2,\ \ldots\ \}.
\]
Lukualueen laajennus tehdään aksiomaattisesti olettamalla seuraavat lisäaksioomat, jotka siis
ovat voimassa \Z:ssa mutteivät \N:ssä.
\begin{itemize}
\item[(L7)] $\quad \text{On olemassa \kor{luku}}\ 0,\ 
                   \text{jolle pätee}\ \ x+0 = x\ \ \forall\ x \in \Z.$
\item[(L8)] $\quad \text{Jokaisella}\ x \in \Z\ \text{on \kor{vastaluku}}\ -x,\ 
                   \text{jolle pätee}\ \ x+(-x) = 0.$
\end{itemize}
Lyhennysmerkintä
\[
x + (-y) = x - y \quad \quad \text{'$x$ miinus $y$'}
\]
tuo \Z:aan uuden laskuoperaation, \kor{vähennyslaskun}, jonka tulos on nimeltään lukujen $x$ ja
$y$ \kor{erotus}.  Tässä on siis kyse yhteenlaskun ja 'merkinvaihdon' $x \map -x$ (jonka voi 
myös tulkita laskuoperaatioksi) yhdistämisestä.

Kokonaisluvuilla laskettaessa pidetään selviönä, että luku $0$ samoin kuin vastaluvut
$-1,\ -2, \ldots\,$ ovat yksikäsitteisiä. Tuttuja ovat myös laskusäännöt
$\,-(-x)=x$, $\,0 \cdot x=0\,$ ja $\,(-1) \cdot x = -x$. Näitä oletuksia\,/\,laskusääntöjä ei
tarvitse kuitenkaan sisällyttää kokonaislukujen aksioomiin, sillä ne ovat aksioomien seurauksia:
\begin{Lause}\footnote[2]{Kun matemaattisessa tekstissä halutaan nostaa esiin jokin selkeästi 
muotoiltu, tosi väittämä, esim. siksi että väittämään halutaan myöhemmin viitata tai kun 
halutaan korostaa tuloksen uutuusarvoa, käytetään nimityksiä \kor{Lause} eli \kor{Teoreema} 
(engl.\ Theorem), \kor{Propositio} (engl.\ Proposition), \kor{Apulause} eli \kor{Lemma} 
(engl.\ Lemma) ja \kor{Seurauslause} eli \kor{Korollaari} (engl.\ Corollary). Tässä tekstissä 
käytetään (hieman epäjohdonmukaisesti) nimityksiä Lause, Propositio, Lemma ja Korollaari. 
Propositio on lausetta vähäarvoisempi tai sisällöltään teknisempi väittämä --- nimitystä 
käytetään lähinnä, kun halutaan luokitella lauseita niiden painoarvon mukaan. Lemma on yleensä
välietappi jonkin lauseen/proposition todistamisessa ja korollaari jo todistetun lauseen tai 
proposition suoraviivainen seuraamus. \index{Lause (teoreema)|av}
\index{Propositio (tosi väittämä)|av} \index{Lemma (apulause)|av}
\index{Korollaari (seurauslause)|av}} \label{Z-tuloksia}
Kokonaisluvuille pätee aksioomien (L1)--(L8) perusteella: \vspace{1mm}\newline
(a) \ Luku $0$ on yksikäsitteinen, samoin vastaluku $-x$ jokaisella $x\in\Z$. \newline
(b) \ $-(-x)=x\,\ \forall x\in\Z$. \newline
(c) \ $0 \cdot x = 0\,\ \forall x\in\Z$. \newline
(d) \ $(-1) \cdot x = -x\,\ \forall x\in\Z$. 
\end{Lause}
Lause \ref{Z-tuloksia} tulee todistetuksi seuraavassa luvussa osana yleisempää lausetta,
joten tyydytään tässä todistamaan ainoastaan helpoin osaväittämä (b): Koska aksioomien
(L1) ja (L8) mukaan on
\[
-x+x = x+(-x) = 0,
\]
niin vastaluvun määritelmän mukaisesti on $-(-x)=x$. \loppu
\begin{Exa} Lauseen \ref{Z-tuloksia} laskusäännöt (b) ja (d) yhdistämällä nähdään, että
$(-1)\cdot(-1)=-(-1)=1.$ \loppu \end{Exa}

\subsection{Lukujärjestelmät}
\index{lukujzy@lukujärjestelmät|vahv}%

Kokonaisluvut on käytännössä ilmaistava jossakin \kor{lukujärjestelmässä}. Tavanomaisin, 
\index{kymmenjzy@kymmenjärjestelmä}%
matematiikan 'äidinkieleksi' vakiintunut on \kor{kymmenjärjestelmä} (engl.\ decimal system), 
jossa kirjoitetaan
\begin{align*}
1+1\ &=\ 2 \quad\ \,\text{'kaksi'}, \\
2+1\ &=\ 3 \quad\ \,\text{'kolme'}, \\
     &\ \vdots \\
8+1\ &=\ 9 \quad\ \,\text{'yhdeksän'}, \\
9+1\ &=\ 10 \quad \text{'kymmenen'}.
\end{align*}
Tässä merkit $0 \ldots 9$ ovat \kor{kymmenjärjestelmän numerot} (engl.\ digits, lat.\ 
digitus = sormi,varvas). Luku 'kymmenen' on luonnollisista luvuista ensimmäinen, jolla ei ole 
ko.\ järjestelmässä omaa symbolia, vaan se merkitään yhdistelmänä 'yksi nolla'. Tämä on 
\index{kantaluku (lukujärjestelmän)}%
järjestelmän \kor{kantaluku}. Kymmenjärjestelmän luvut ilmaistaan tämän jälkeen symbolisesti 
muodossa 
\[
x\ =\ eabcd \ldots\ \in\ \Z,
\]
missä $e$ on \kor{etumerkki} (engl.\ sign), $e=+$ tai $e=-$ (\,'tyhjä' = $+$\,), ja 
$\,a,b,c,d, \ldots$ ovat kymmenjärjestelmän numeroita. Numeroiden lukumäärästä riippuen tulkinta
on
\[
ea = \pm a, \quad eab = \pm (a \cdot 10 + b), \quad 
                 eabc = \pm(a \cdot 10^2 + b \cdot 10 + c), \quad \text{jne.}
\]
Tässä etumerkki on $e$:n mukainen.\footnote[2]{Lukumerkinnän tulkinnan mukaisesti on
$\pm 0abc .. = \pm abc ..$, ts.\ 'etunollat' eivät vaikuta luvun tulkintaan.} 

Yleisemmin jos lukujärjestelmän kantaluku on luku $k \in \N,\ k\neq 1$, niin tämän 
$k$-\kor{järjestelmän} luvuilla $0,1, \ldots, k-1$ on oltava kullakin oma 
numeromerkki.\footnote[2]{Lukujärjestelmien idea keksittiin Kaksoisvirranmaassa noin 4000 vuotta
sitten. Varhaisimmissa järjestelmissä kantaluku oli $60$. Numeromerkit olivat nuolenpäämerkkien
yhdistelmiä.} Jos $a,b,c, \ldots$ ovat tällaisia merkkeja, niin luku $x = eabcd \ldots$ 
tulkitaan samoilla säännöillä kuin edellä. On huomioitava ainoastaan, että luku '10' ei nyt ole
'kymmenen', vaan tällä tarkoitetaan lukua $1\cdot k + 0 = k$. Kantaluvun merkki on siis aina 
'$10$', sen sijaan ääntämistapa (esim.\ 'kymmenen','tusina','tiu') on sovittava erikseen 
kussakin lukujärjestelmässä (ja kussakin kielessä). Ellei mitään ole sovittu, on luettava 
'yksi nolla'. Lukujärjestelmistä kaikkein yksinkertaisin numeromerkistöltään on 
\index{binaarijärjestelmä}%
\kor{binaarijärjestelmä}, jossa kantalukuna on 'ykkösestä seuraava' eli $2$. Binaarijärjestelmän
\index{bitti}%
numeroita eli \kor{bittejä} ovat vain $0$ ja $1$.
\begin{Exa} Kymmenjärjestelmän luku $1253$ voidaan lukea 'yksi kaksi viisi kolme', jolloin 
ainoastaan äännetään luvun 'nimi'. Lukutavassa 'tuhat kaksi sataa viisikymmentä kolme' annetaan
jo luvulle tulkinta: $1253 = $ 'tuhat plus kaksi kertaa sata plus viisi kertaa kymmenen plus 
kolme'. Tässä 'sata' ja 'tuhat' ovat suomen kielessä sovittuja (ja kouluissa opetettuja) 
ääntämistapoja luvuille $100=k^2$ ja $1000=k^3$, kun $k =$ 'kymmenen'. \loppu 
\end{Exa}
\begin{Exa} Lausu kymmenjärjestelmän luku 11 binaarijärjestelmässä. \end{Exa}
\ratk Kymmenjärjestelmässä voidaan kirjoittaa $11 = 1 \cdot 2^3 + 0 \cdot 2^2 + 1 \cdot 2 + 1$. 
Binaarijärjestelmän esitysmuoto saadaan tästä kirjoittamalla merkin '$2$' tilalle '$10$'\,:
\[
11\ =\ 1 \cdot 10^3 + 0 \cdot 10^2 + 1 \cdot 10 + 1\ =\ 1011 \quad 
                                  \text{(binaarijärjestelmä)}. \quad \loppu
\]
\begin{Exa} Laske $7$-järjestelmän lukujen $145$ ja $66$ summa ja muunna tulos 
kymmenjärjestelmään. \end{Exa}
\ratk Kun merkitään $k=7$, niin laskulakeja (L1)--(L6) soveltaen saadaan
\begin{align*}
145+66 &= (1 \cdot k^2 + 4 \cdot k + 5) + (6 \cdot k + 6) \\
       &= 1 \cdot k^2 + (4+6) \cdot k + (5+6) \\
       &= 1 \cdot k^2 + (1 \cdot k + 3) \cdot k + (1 \cdot k + 4) \\
       &= (1+1) \cdot k^2 + (3+1) \cdot k + 4 \\
       &= 2 \cdot k^2 + 4 \cdot k + 4 \\
       &= 244.
\end{align*}
Kymmenjärjestelmässä ilmaistuna lopputulos on $2 \cdot 7^2 + 4 \cdot 7 + 4 = 130$. \loppu

\pagebreak

\subsection{Rationaaliluvut $\Q$}

Siirryttäessä \Z:sta \kor{rationaalilukujen} joukkoon \Q\ tarvitaan enää yksi lisäaksiooma:
\begin{itemize}
\item[(L9)] $\quad \text{Jokaisella}\ x \neq 0\ \text{on \kor{käänteisluku}}\ x^{-1}\ 
                                                 \text{siten, että}\ \ x \cdot x^{-1} = 1.$
\end{itemize}
Lyhennysmerkintä
\[
x \cdot y^{-1} = x / y = \frac{x}{y} \quad  \quad \text{'$x$ per $y$'}
\]
määrittelee neljännen laskuoperaation, \kor{jakolaskun}, jonka tulos on nimeltään lukujen $x$ ja
$y$ \kor{osamäärä}. Jakolaskussa on siis kyse 'kääntämisoperaation' $y \map y^{-1}$ ja 
kertolaskun $x,y^{-1} \map x \cdot y^{-1}$ yhdistämisestä. Sanotaan, että luku $x$ on tässä
operaatiossa \kor{osoittaja} ja luku $y$ on \kor{nimittäjä} (engl.\ numerator, denominator). 
Jakolaskun määritelmän ja aksioomien (L2),\,(L6) perusteella on
\[
x^{-1} = x^{-1} \cdot 1 = 1 \cdot x^{-1} = 1 / x.
\]
Kun lukujoukkoa \Z\ laajennetaan siten, että laajennetussa joukossa \Q\ pätevät aksioomat 
(L1)-(L9), on tuloksena uusi algebra, jota merkitään $(\Q,+,\cdot)$. Aksiooman (L9) ja
jakolaskun määritelmän mukaisesti \Q:n alkioita ovat ainakin luvut muotoa $x = p/q$, missä
$p \in \Z$ ja $q \in \N$. Osoittautuu (ks.\ Esimerkki\,\ref{kaksi Q-lakia} alla ja
Harj.teht.\,\ref{H-I-1: Q-lakeja}), että lähdettäessä mainittua muotoa olevista luvuista ovat
laskuoperaatioiden $\,x \map -x$, $\,x \map x^{-1}$, $\,x,y \map x+y\,$ ja
$\,x,y \map x \cdot y\,$ tulokset aina samaa muotoa, eli muun tyyppisiä lukuja ei \Q:ssa ole.
Rationaalilukujen joukko \Q\ voidaan siten määritellä
\[
\Q = \{\ x = p/q\ \vert\ p \in \Z\ \ja\ q \in \N\ \}.
\]

Aksioomista (L1)-(L9) voidaan johtaa rationaalilukujen kaikki normaalit laskusäännöt.
Seuraavassa kaksi esimerkkiä (ks.\ myös Harj.teht.\,\ref{H-I-1: Q-lakeja}).
\begin{Exa} \label{kaksi Q-lakia}
Vedoten aksioomiin (L1)--(L9) ja Lauseeseen \ref{Z-tuloksia} perustele rationaalilukujen
laskusäännöt
\[
\text{a)}\,\ \frac{0}{q}=0\,\ (q\in\N), \qquad 
\text{b)}\,\ \frac{1}{q_1}\cdot\frac{1}{q_2} = \frac{1}{q_1q_2}\,\ (q_1,q_2\in\N).
\]
\end{Exa}
\ratk a) Lähtien rationaaliluvun määritelmästä ja vedoten Lauseen \ref{Z-tuloksia}
väittämään (c) sekä aksioomiin (L4), (L9) ja (L6) päätellään:
\begin{align*}
0/q = 0 \cdot q^{-1} &= (0 \cdot q)\cdot q^{-1} \\
                     &= 0\cdot(q \cdot q^{-1}) \\
                     &= 0 \cdot 1 \\
                     &= 0.
\end{align*}
b) Väitetään, että $q_1^{-1}q_2^{-1}=(q_1q_2)^{-1}$, ts.\ että $y=q_1^{-1}q_2^{-1}$ on luvun
$x=q_1q_2$ käänteisluku. Tarkistus:
\begin{align*}
x \cdot y = (q_1q_2)\cdot(q_1^{-1}q_2^{-1}) &= (q_1q_2)\cdot(q_2^{-1}q_1^{-1}) \\
                                            &= q_1(q_2q_2^{-1})q_1^{-1} \\
                                            &= q_1 \cdot 1 \cdot q_1^{-1} \\
                                            &= q_1q_1^{-1} = 1. 
\end{align*}
Tässä vedottiin aksioomiin (L2), (L4), (L9) ja (L6). \loppu

Kun käänteisluvun potenssiin korotuksessa sovitaan merkinnästä
\[
(x^{-1})^n = \quad x^{-n}, \quad n \in \N,
\]
voidaan $x^{-1}$ lukea myös '$x$ potenssiin miinus $1$'. Aksioomat (L2),\,(L4),\,(L6) ja (L9)
huomioiden seuraa myös 
\[
x \neq 0\ \ja\ m,n \in \Z\ \ja\ m,n \neq 0 
                         \qimpl x^m \cdot x^n = \begin{cases}
                                                \,x^{m+n}, \quad \text{kun}\,\ m+n \neq 0, \\
                                                \,1, \quad\quad\,\ \ \text{kun}\,\ m+n = 0.
                                                \end{cases}
\]
Tämän säännön yksinkertaistamiseksi on luontevaa \pain{määritellä}:
\[
\boxed{\kehys\quad x^0 = 1, \quad \text{kun}\ x \neq 0\ \ (x\in\Q). \quad }
\]
Tällöin saadaan yleiset laskusäännöt
\[
\boxed{\kehys \quad x^mx^n=x^{m+n}, \quad x^ny^n=(xy)^n, \quad (x^m)^n=x^{mn}, 
                                                         \quad x,y \neq 0,\ m,n\in\Z. \quad} 
\]
Myös $0^n = 0 \ \ \forall\ n \in \N$. Sen sijaan määrittelemättä jäävät $0^0$ ja 
$0^{-n},\ n \in \N$, syystä että $0$:lla ei ole käänteislukua.
\index{laskuoperaatiot!a@rationaalilukujen|)}% 

\subsection{$\Q$:n järjestysrelaatio}
\index{jzy@järjestysrelaatio!b@$\Q$:n|vahv}%

Rationaaliluvut voidaan tunnetulla tavalla myös \kor{järjestää}, eli voidaan määritellä
\kor{järjestysrelaatio}
\[
<\ : \quad \text{'pienempi kuin'}.
\]
Tällöin syntyvää algebraa voidaan merkitä $(\Q,+,\cdot,<)$. Järjestysrelaatioon liitetään
seuraavat yleiset aksioomat (eli oletetut peruslait):

\pagebreak

\index{jzy@järjestysrelaatio} \index{samastus '$=$'!b@rationaalilukujen}%
\begin{itemize}
\item[(J1)] $\quad \text{Vaihtoehdoista}\ \{x<y,\ x=y,\ x>y\}\ 
                                          \text{on voimassa täsmälleen yksi}.$
\item[(J2)] $\quad x<y\ \ja\ y<z \,\ \impl\,\ x<z$.
\item[(J3)] $\quad x<y \,\ \impl\,\ x+z < y+z \ \ \forall z$.
\item[(J4)] $\quad x > 0 \ \ja\ y > 0 \,\ \impl\,\ x \cdot y\, > 0$.
\end{itemize}
Tässä merkintä $x>y$ ('$x$ suurempi kuin $y$') on merkinnän $y<x$ vaihtoehtoinen muoto.
Käytännössä vertailusta selvitään nopeimmin käyttämällä laskulakeja (L7),\,(L8) ja (L1), sillä
näistä ja aksioomasta (J3) on pääteltävissä, että pätee
\[ 
\begin{cases} \begin{aligned}
x<y &\qekv x-y < 0, \\ x=y &\qekv x-y = 0, \\ x>y &\qekv x-y > 0.
\end{aligned} \end{cases}
\] 
Tässä '$\ekv$' on logiikan symboleihin kuuluva
\index{ekvivalenssi}
\kor{ekvivalenssinuoli}, joka voidaan tässä 
yhteydessä lukea
\[
\text{P} \quad \ekv \quad \text{Q}\ : \quad \text{'P täsmälleen kun Q'}.
\]
Ym.\ kriteerillä voidaan siis vertailukysymys $\,x$ ? $y\,$ selvittää tutkimalla, onko luku 
$x-y$ \kor{positiivinen} ($>0$), \kor{negatiivinen} ($<0$), vai $\,=0$. Perusmuotoisesta
rationaaliluvusta $\,x = p/q,\ (p\in\Z,\ q\in \N)$ puolestaan sovitaan, että vaihtoehdot
$x>0$ ja $x=0$ vastaavat vaihtoehtoja $p\in\N$ ja $p=0$, muulloin on $x<0$. Tämä sopimus on
mahdollinen, koska Esimerkin\,\ref{kaksi Q-lakia} laskusäännön (a) mukaan luvut $0/q$ samastuvat
lukuun $0$\,:
\[
0\ =\ \dfrac{0}{1}\ =\ \dfrac{0}{2}\ =\ \ldots
\]
Toisaalta muita luvun $0$ esiintymismuotoja ei $\Q$:ssa ole, sillä jos $x=p/q=0$, niin seuraa
$\,(p/q) \cdot q = p = 0 \cdot q = 0$ (aksioomat (L4),\,(L2),\,(L9),\,(L6) ja Lauseen
\ref{Z-tuloksia} väittämä (c)). Aksiooma (J1) on näin ollen voimassa. Myös aksioomien (J2)--(J4)
voimassaolo on helposti pääteltävissä (Harj.teht.\,\ref{H-I-1: Q-järjestys}).

\index{relaatio}%
Järjestys ja samastus ovat esimerkkejä (joukko--opillisista) \kor{relaatioista}, joissa on kyse
joukon kahden alkion välisestä 'suhteesta'. Järjestykseen\,/\,samastukseen liittyviä
relaatiosymboleja ovat mainittujen lisäksi myös '$\neq$', '$\le$' ja '$\ge$', jotka määritellään
\begin{align*}
x \neq y \quad &\ekv \quad \text{ei päde}\ \ x=y, \\
x \le y  \quad &\ekv \quad x < y\,\ \text{tai}\,\ x = y, \\
x \ge y  \quad &\ekv \quad x > y\,\ \text{tai}\,\ x = y.
\end{align*}
Merkintä $P \ekv Q$ voidaan tässä lukea '$P$ tarkoittaa samaa kuin $Q$'.

\subsection{Summa- ja tulomerkinnät}
\index{summamerkintä|vahv} \index{tulomerkintä|vahv}%

Laskettaessa peräkkäin yhteen tai kerrottaessa useita erilaisia lukuja on luvut kätevää 
numeroida eli \kor{indeksoida}, jolloin voidaan käyttää lyhennysmerkintöjä. Nämä ovat 
\begin{align*}
&\text{\kor{Summamerkintä:}} \qquad\, x_1 + x_2 + \ \ldots\  + x_n\ =\ \sum_{i=1}^n x_i \\
&\text{\kor{Tulomerkintä:}} \qquad\qquad\quad x_1 \cdot x_2 \cdot\ \ldots\ \cdot x_n\ 
                                                                    =\ \prod_{i=1}^n x_i
\end{align*}
\index{indeksi!a@summa- ja tulomerkinnän}%
Summamerkinnässä symboli $i$ on nimeltään \kor{summausindeksi}. Tämä saa yleensä (ei aina)
kokonaislukuarvoja, jolloin symboli valitaan tavallisimmin joukosta $\{i,j,k,l,m,n\}$.
Koska kyse on vain lyhennysmerkinnästä, ei symbolin valinta luonnollisesti vaikuta itse 
laskutoimitukseen:
\[
\sum_{i=1}^n a_i = \sum_{j=1}^n a_j = \sum_{k=1}^n a_k = \ldots
\]
Peräkkäisissä tuloissa sanotaan kerrottavia lukuja tulon \kor{tekijöiksi} (engl.\ factor).
Kokonaislukujen peräkkäisiin tuloihin liittyy termi 
\index{kertoma} \index{n@$n$-kertoma}%
$n$-\kor{kertoma} (engl.\ $n$-factorial),
joka merkitään $\,n!\,$ ja määritellään
\[
\quad 0! = 1, \quad 1! = 1, \quad n!\,=\,1 \cdot 2 \ \ldots\ \cdot n\ 
                                      =\ \prod_{k=1}^n k, \quad n = 2,3,\ \ldots
\]

\Harj
\begin{enumerate}

\item
Näytä, että seuraaville joukoille pätee $A=B$, eli $A$ ja $B$ ovat joukkoina samat:
\begin{align*}
&\text{a)}\ \ A=\{3n+2 \mid n\in\Z\}, \quad B=\{3n-7 \mid n\in\Z\} \\
&\text{b)}\ \ A=\{7n+3 \mid n\in\Z\}, \quad B=\{7n-32 \mid n\in\Z\}
\end{align*}

\item
Vedoten aksioomiin (L1)--(L8) ja Lauseen \ref{Z-tuloksia} väittämiin perustele kokonaislukujen
laskusäännöt \ a) $-x+(-y)=-(x+y)$, \ b) $(-x) \cdot y = - x\cdot y$, \
c) $(-x) \cdot (-y) = x \cdot y$.

\item 
Muodosta kantalukuja $k=2$, $k=3$ ja $k=4$ vastaavien lukujärjestelmien kertotaulut.

\item 
Tusinajärjestelmässä on kymmenjärjestelmän merkkien lisäksi käytössä numeromerkit (vasemmalla 
kymmenjärjestelmän merkintä)
\begin{align*}
&10=\diamondsuit \quad \text{'ruutu'} \\
&11=\heartsuit \quad \text{'hertta'} \\
&12=10 \quad \text{'tusina' (kantaluku)}
\end{align*}
Muunna kymmenjärjestelmän luku 155 tusinajärjestelmään ja tusinajärjestelmän luku 
$9\diamondsuit 07\heartsuit$ kymmenjärjestelmään.

\item 
Suorita kymmenjärjestelmään muuntamatta seuraavat tusinajärjestelmän (ks. edellinen tehtävä) 
laskuoperaatiot:
\[
\text{a)}\ \ \heartsuit\heartsuit\diamondsuit + \diamondsuit\diamondsuit\heartsuit \qquad 
\text{b)}\ \  247-19\diamondsuit \qquad 
\text{c)}\ \ 23 \cdot 34
\]

\item
Rusinajärjestelmässä kymmenjärjestelmän luku $7$ äännetään 'rusina', luvut $1 \ldots 6$
merkitään ja äännetään kuten kymmenjärjestelmässä, ja rusinajärjestelmän luvut $100$ ja
$1000$ äännetään 'pulla' ja 'kakku'. Miten merkitään ja lausutaan kymmenjärjestelmän luku $2331$
rusinajärjestelmässä?

\item  \label{H-I-1: Q-lakeja} \index{laskuoperaatiot!a@rationaalilukujen}%
Seuraavissa rationaalilukujen laskusäännöissä on tulos esitetty rationaaliluvun perusmuodossa
olettaen, että $p,p_1,p_2\in\Z$ ja $q,q_1,q_2\in\N$. Perustele säännöt vedoten aksioomiin
(L1)--(L9), Lauseeseen \ref{Z-tuloksia} ja Esimerkin \ref{kaksi Q-lakia} laskusääntöihin.
\begin{align*}
&\text{a) \,Supistus\,(lavennus):} \quad (p \cdot m)/(q \cdot m) = p/q, 
                    \              \quad m\in\Z,\ m\neq 0. \\
&\text{b)   Vastaluku:} \quad -(p/q) = (-p)/q. \\
&\text{c) \,Käänteisluku:} \quad (p/q)^{-1} = \begin{cases}
                                              \,q/p,       &\text{jos}\ p\in\N, \\
                                              \,(-q)/(-p), &\text{jos}\ p\not\in\N,\ p \neq 0.
                                              \end{cases} \\
&\text{d)   Yhteenlasku:} \quad p_1/q_1+p_2/q_2 = (p_1q_2+p_2q_1)/(q_1q_2). \\
&\text{e) \,Kertolasku:} \quad (p_1/q_1)(p_2/q_2) = (p_1p_2)/(q_1q_2).
\end{align*}

\item \label{H-I-1: Q-järjestys}
Vedoten tehtävän \ref{H-I-1: Q-lakeja} laskusääntöihin ja rationaalilukujen järjestysrelaation
määritelmään näytä, että järjestysrelaatiolle ovat voimassa aksioomat (J2), (J3) ja (J4). 

\item
Ratkaise $\Q$:ssa seuraavat yhtälöt\,/\,epäyhtälöt mahdollisimman yksinkertaiseen muotoon. 
Perustele laskun vaiheet aksioomilla (L1)--(L9), (J1)--(J4)\,! \vspace{1mm}\newline
a) \ $2x+7=9x-4 \qquad\quad\,\ $
b) \ $\tfrac{x}{5}+\tfrac{2}{3}<\tfrac{x}{3}+\tfrac{1}{2}$ \newline
c) \ $(x+1)/(x-2)=4 \qquad$
d) \ $(2x-1)/(3x+2)>3$

\item
a) Näytä, että seuraavat päättelysäännöt ovat päteviä sekä kokonaisluvuille ($x_k\in\Z$) että
rationaaliluvuille ($x_k\in\Q$). Voit vedota sekä aksioomiin että niistä johdettuihin
laskusääntöihin
(Lause \ref{Z-tuloksia}, Tehtävä \ref{H-I-1: Q-lakeja}).
\begin{align*}
&\quad\ \ \prod_{k=1}^n x_k = 0 \qekv x_k=0\ \ \text{jollakin}\ \ k\in\{1,\ldots,n\}. \\
&\quad\ \ \prod_{k=1}^n x_k < 0 \qekv 
                               \text{pariton määrä lukuja $x_k$ on negatiivisia, muut}\ >0.
\end{align*}
Soveltaen a-kohdan päätelmiä ratkaise $\Q$:ssa:
\vspace{1mm}
\begin{align*}
&\text{b)}\,\ 2x^2-3x-2 \ge 0 \qquad\,
 \text{c)}\,\ 63x^2-32x \le 63 \qquad
 \text{d)}\,\ 9x^3 \ge 169x \\[1mm]
&\text{e)}\,\ \frac{x+1}{x-1}\,\ge\,\frac{x}{x+1} \qquad\quad
 \text{f)}\,\ \frac{5}{6x-7}\,\le\,x \qquad\qquad
 \text{g)}\,\ 6x+\frac{2x+12}{x^2+x}\,\ge\,13
\end{align*}

\item
Sievennä (ol.\ $n\in\N$)
\[
\text{a)}\ \ \sum_{k=1}^n n \qquad
\text{b)}\ \ \sum_{i=1}^{n^2} \frac{2}{n} - \sum_{j=1}^n 1 - \sum_{k=1}^n 1 \qquad
\text{c)}\ \ \prod_{k=1}^n \left(1+\frac{1}{k}\right)
\]

\item 
Kaksoissumma $\sum_{i=1}^n\sum_{j=1}^m a_{ij}\ (n,m\in\N,\ a_{ij}\in\Q)$ määritellään
\[
\sum_{i=1}^n\sum_{j=1}^m a_{ij} = \sum_{i=1}^n\Bigl(\sum_{j=1}^m a_{ij}\Bigr) 
                                = \sum_{j=1}^m\Bigl(\sum_{i=1}^n a_{ij}\Bigr)
                                = \sum_{j=1}^m\sum_{i=1}^n a_{ij}\,.
\]
a) Näytä, että tässä sulkeiden poisto ja järjestyksen vaihto ovat todella mahdollisia, ts.\ 
että kaksi keskimmäistä lauseketta ovat samat. \vspace{1mm}\newline
b)\,\ Sievennä: $\D \ \sum_{i=1}^n\sum_{j=1}^n (a_i + a_j). \quad\ $
c)\,\ Näytä: $\D \ \Bigl(\,\sum_{i=1}^n a_i \Bigr)^2 = \sum_{i=1}^n\sum_{j=1}^n a_i a_j\,$.

\item (*)  \label{H-I-1: teleskooppisumma} \index{teleskooppisumma}
Summaa $\sum_{k=0}^n a_k$ sanotaan \kor{teleskooppisummaksi}, jos $a_k=b_{k+1}-b_k\ \forall k$, 
missä luvut $b_k,\ k=0 \ldots n+1$,
ovat tunnettuja. \newline
a) Määritä teleskooppisumman arvo lukujen $b_k$ avulla.\newline
b) Mistä nimitys teleskooppisumma?\newline
c) Laske teleskooppi-idealla summat
\[
\sum_{k=0}^n (2k+1) \quad \text{ja} \quad \sum_{k=1}^n\frac{1}{k(k+1)}\,, \quad n\in\N.
\]
d) Näytä teleskooppisummien avulla:\, $\sum_{k=1}^n k^2 = \tfrac{1}{6}n(2n^2+3n+1),\,\ n\in\N$. 

\end{enumerate}  % Rationaaliluvut
\section{Kunta} \label{kunta}
\alku

Rationaaliluvut laskuoperaatioineen ovat esimerkki (toistaiseksi ainoa) algebrallisesta
rakennelmasta nimeltä \kor{kunta} (engl.\ field, ruots.\ kropp, saks.\ Körper). Jos \K\ on jokin
lukujoukko (voisi olla yleisempikin joukko), niin sanotaan, että $(\K,+,\cdot )$ on kunta, jos
ensinnäkin
\index{laskuoperaatiot!ab@kunnan}% 
\begin{itemize}
\item Laskuoperaatiot $\,x,y \map x+y$ ja $\,x,y \map x \cdot y$ on määritelty yksikäsitteisesti
$\forall x,y \in \K$ ja pätee $x+y \in \K$ ja $x \cdot y \in \K$,
\end{itemize}
ja lisäksi seuraavat \vahv{kunta-aksioomat} ovat voimassa:
\index{kunta-aksioomat}%
\begin{itemize}
\item Vaihdanta-, liitäntä- ja osittelulait
   \begin{itemize}
   \item[(K1)] $\quad x+y = y+x$
   \item[(K2)] $\quad x \cdot y = y \cdot x$
   \item[(K3)] $\quad x+(y+z) = (x+y)+z$
   \item[(K4)] $\quad x \cdot (y \cdot z) = (x \cdot y) \cdot z$
   \item[(K5)] $\quad x \cdot (y+z) = x \cdot y + x \cdot z$
   \end{itemize}
\item Nolla--alkion ja vasta--alkion olemassaolo
   \begin{itemize}
   \index{nolla-alkio (kunnan)}%
   \item[(K6)] $\quad \text{On olemassa \kor{nolla--alkio}}\ 0 \in \K,\ 
                          \text{jolle pätee}\ \ x+0 = x\ \ \forall\ x \in \K.$
   \index{vasta-alkio (kunnan)}%   
   \item[(K7)] $\quad \text{Jokaisella}\ x \in \K\ \text{on \kor{vasta--alkio}}\ -x \in \K,\ 
                          \text{jolle pätee}\ \ x+(-x) = 0.$
   \end{itemize}
\item Ykkösalkion ja käänteisalkion olemassaolo
   \begin{itemize}
   \index{ykkösalkio (kunnan)}%
   \item[(K8)] $\quad \text{On olemassa \kor{ykkösalkio}}\ 1 \in \K,\ \ 
                          \text{jolle pätee}\ \ x \cdot 1 = x \ \ \forall x \in \K$.
   \index{kzyzy@käänteisalkio (kunnan)}%
   \item[(K9)] $\quad \text{Jokaisella}\ x \in \K,\ x \neq 0\ 
                          \text{on \kor{käänteisalkio}}\ x^{-1} \in \K,\
                          \text{jolle pätee}$\newline
               \phantom{ai} $x \cdot x^{-1} = 1.$
   \end{itemize}
\item Erotteluaksiooma
   \begin{itemize}
   \item[(K10)] $\quad 0 \neq 1$
   \end{itemize}
\end{itemize}
Aksioomista erillinen alkuoletus katsotaan yleensä $\K$:n laskuoperaatioiden määritelmään
sisältyväksi. Erillisenä tämä oletus on huomioitava lähinnä silloin, kun laskuoperaatiot on
alunperin määritelty jossakin suuremmassa joukossa $\A\supset\K$, tai kun tarkasteltavaa joukkoa
$\K$ halutaan laajentaa. Tällöin on varmistettava, että laskuoperaatioiden tulos pysyy joukossa
$\K$, jota tarkastellaan, ks.\ esimerkit jäljempänä.
Peruslaskuoperaatioiden $(+,\cdot)$ lisäksi kunnassa voidaan aina määritellä
yhdistetyt laskuoperaatiot $x,y \map x-y = x+(-y)$ (vähennyslasku) ja
$x,y \map x/y = x \cdot y^{-1}$ (jakolasku, $y \neq 0$), jolloin on $-x=0-x$ ja $x^{-1}=1/x$.
Koska aksiooma (K10) tarkoittaa 'ei päde $0=1$', niin kunnassa tarvitaan yleisesti vain
samastusrelaatio (\K:n alkioiden erotteluperiaate), ei
järjestysrelaatiota.
\index{samastus '$=$'!c@kunnan|av}%
\footnote[2]{Samastusrelaatiolta '$=$' edellytetään aina aksioomat \
(S1) $x=x\ \forall x$, \ (S2) $x=y \ \impl\ y=x$ ja (S3) $x=y \,\ja\, y=z \ \impl\ x=z$. Kunnan
samastusrelaatiolta vaaditaan lisäksi yhteensopivuus laskuoperaatioiden kanssa siten, että
laskuoperaation tulos on aina yksikäsitteinen. Vaihdantalakien (K1)--(K2) ja aksiooman (S3)
perusteella tämä vaatimus toteutuu olettamalla:\newline
(SK1) $x=y \ \impl\ x+z=y+z\ \forall z\,$ ja
(SK2) $x=y \ \impl\ x \cdot z = y \cdot z\ \forall z$.}
\begin{Exa} \label{yksinkertaisin kunta}
Yksinkertaisin mahdollinen kunta saadaan, kun valitaan $\K = \{0,1\}$ (missä $0\neq 1$) ja 
sovitaan laskusäännöstä $1+1 = 0$ (!). Tällöin $-1 = 1$ ja muut laskusäännöt ovat pääteltävissä
aksioomista (K1),\,(K2), (K5),\,(K6) ja (K8):
\begin{align*}
0+0 &= 0, \quad 0+1 = 1+0 = 1, \quad 1\,\cdot\,1 = 1, \quad 0\,\cdot\,1 = 1\,\cdot\,0 = 0, \\
0\,\cdot\,0 &= 0\,\cdot\,(1+1) = 0\,\cdot\,1 + 0\,\cdot\,1 = 0.
\end{align*}
Näillä säännöillä aksioomat (K1)-(K9) ovat kaikki voimassa (samoin oletus (K0)), joten kyseessä
on kunta. 
\loppu
\end{Exa}
\begin{Lause} \label{kuntatuloksia} Jokaisessa kunnassa $(\K,+,\cdot)$ pätee
\begin{itemize}
\item[(a)] $\quad \text{Nolla--alkio, vasta--alkio, ykkösalkio ja käänteisalkio ovat
yksikäsitteisiä.}$
\item[(b)] $\quad -(-x) = x \quad \forall x \in \K, \quad\quad 
                  (x^{-1})^{-1} = x \quad \forall x \in \K,\ x \neq 0.$
\item[(c)] $\quad 0 \cdot x = 0 \quad \forall x \in \K$.
\item[(d)] $\quad -x = (-1) \cdot x \quad \forall x \in \K$.
\end{itemize}
\end{Lause}

\tod (a) Oletetaan, että $0 \in \K$ ja $\theta \in \K$ ovat kaksi nolla--alkiota. Tällöin
aksiooman (K6) mukaan on
\[
x = x + 0 \quad \forall x \in \K, \quad \quad y + \theta = y \quad \forall y \in \K.
\]
Kun valitaan $x = \theta$ ja $y = 0$ ja käytetään aksioomaa (K1), seuraa
\[
\theta = \theta + 0 = 0 + \theta = 0,
\]
eli $\theta = 0$. Siis nolla--alkio on yksikäsitteinen. Vasta--alkion yksikäsitteisyyden 
toteamiseksi oletetaan, että $a\in\K$ ja $b\in\K$ ovat saman alkion $x \in \K$ 
vasta--alkioita. Tällöin aksioomien (K6),\,(K3) ja (K1) perusteella on
\[
a=a+0=a+(x+b)=(a+x)+b=(x+a)+b=0+b=b+0=b,
\]
eli $a=b$. Siis vasta--alkiokin on yksikäsitteinen. Muut väitetyt yksikäsitteisyydet 
seuraavat samanlaisella päättelyllä.

(b) Väittämän ensimmäinen osa todistettiin edellisessä luvussa (Lause \ref{Z-tuloksia}(b)).
Toisen osan todistamiseksi sovelletaan aksioomia (K2) ja (K9):
\[
x^{-1} \cdot x = x \cdot x^{-1} = 1\ \ \impl\ \ (x^{-1})^{-1}=x.
\]

(c) Kun merkitään $0 \cdot x = y\in\K$, niin aksioomien (K6),\,(K2) ja (K5) perusteella
\begin{equation}
y = 0 \cdot x = (0 + 0) \cdot x = 0 \cdot x + 0 \cdot x = y+y.  \tag{$\star$}
\end{equation}
Käyttäen tätä tulosta ja kunta--aksioomia päätellään:
\begin{align}
0 &= y+(-y)      \tag{K7} \\
  &= (y+y)+(-y)  \tag{$\star$} \\
  &= y+[y+(-y)]  \tag{K3} \\
  &= y+0         \tag{K7} \\
  &= y.          \tag{K6}
\end{align}

(d) Tuloksen (c) ja kunta--aksioomien perusteella
\begin{align}
x + (-1) \cdot x &= x \cdot 1 + x \cdot (-1)  \tag*{(K8),\,(K2)} \\
                 &= x \cdot [1 + (-1)]        \tag{K5} \\
                 &= x \cdot 0                 \tag{K7} \\
                 &= 0. \loppu                 \tag*{(K2),\,(c)}
\end{align}

Yleisessä kunnassa voidaan potenssiin korotus määritellä samalla tavoin kuin rationaalilukujen 
kunnassa, vrt.\ edellinen luku.
\begin{Exa} Johda binomikaavat lausekkeille $\ (x+y)^2\ $ ja $\ (x+y)^3\ $ sellaisessa kunnassa
$(\K,+,\cdot)$, jossa pätee $1+1 = \spadesuit$, $\spadesuit+1 = \clubsuit$. \end{Exa}
\ratk Potenssiin korotuksen määritelmän ja kunta-aksioomien perusteella
\begin{align*}
(x+y)^2\ =\ (x+y) \cdot (x+y)\ &=\ x \cdot x + x \cdot y + y \cdot x + y \cdot y \\
                               &=\ x^2 + (x \cdot y + x \cdot y) + y^2.
\end{align*}
Tässä on edelleen kunta--aksioomien ja oletuksen perusteella
\[
x \cdot y + x \cdot y\ =\ (x \cdot y) \cdot (1+1)\ 
                       =\ (x \cdot y) \cdot \spadesuit\ =\ \spadesuit \cdot x \cdot y,
\]
joten pyydetty ensimmäinen binomikaava on
\[
(x+y)^2\ =\ x^2 + \spadesuit \cdot x \cdot y + y^2.
\]
Vastaavalla päättelyllä saadaan toiseksi kaavaksi
\[
(x+y)^3\ =\ x^3 + \clubsuit \cdot x^2 \cdot y + \clubsuit \cdot x \cdot y^2 + y^3. \loppu
\]
%\begin{Exa} Päättele, että kunnan nolla--alkiolla ei ole käänteisalkiota. \end{Exa}
%\ratk Lauseen \ref{kuntatuloksia} väittämän (c) ja erotteluaksiooman (K10) perusteella on 
%$0 \cdot x = 0 \neq 1\ \forall x \in \K$, eli mikään $x \in \K$ ei täytä $0$:n käänteisalkiolle
%asetettavaa vaatimusta $0 \cdot x = 1$. \loppu

Seuraava kunta--algebran tulos osoittautuu jatkossa hyödylliseksi
(Harj.teht.\,\ref{H-I-2: kuntakaavat}a).
\begin{Prop} \label{kuntakaava} Jokaisessa kunnassa pätee
\begin{align*}
x^n-y^n\, &=\, (x-y)\left(x^{n-1}+x^{n-2}y + \ldots + y^{n-1}\right) \\
          &=\, (x-y)\sum_{k=0}^{n-1} x^{n-1-k}y^k, \quad n\in\N.
\end{align*} \end{Prop}
 
\subsection{Järjestetty kunta}
\index{jzy@järjestetty kunta|vahv}%

Jos kunnassa on määritelty järjestysrelaatio edellisen luvun aksioomien (J1)-(J4) mukaisesti,
niin sanotaan, että kyseessä on \kor{järjestetty kunta} (engl. ordered field). Järjestetyssä
kunnassa jokainen alkio on aksiooman (J1) mukaisesti joko positiivinen ($x>0$), negatiivinen
($x<0$), tai $=0$. Toistaiseksi ainoa esimerkki järjestetystä kunnasta on rationaalilukujen
kunta. 
\addtocounter{Thm}{-2}
\begin{Lause} (jatko) Jokaisessa järjestetyssä kunnassa $(\K,+,\cdot,<)$ pätee
\begin{itemize}
\item[(e)] $\quad x>0\ \ \impl\ \ -x<0, \quad\quad x>0\ \ \impl\ \ x^{-1} > 0$.
\item[(f)] $\quad x>0\ \ \&\ \ y>0 \ \ \impl\ \ x+y>0$.
\item[(g)] $\quad 0 < 1$.
\end{itemize} \end{Lause}
\addtocounter{Thm}{1}
\tod (e) \ Oletetaan, että $0<x$. Tällöin on aksiooman (J3) mukaan myös 
\[
0+(-x)\ <\ x + (-x), 
\]
mikä kunta-aksioomien mukaan pelkistyy ensimmäiseksi väittämäksi $-x < 0$. Väittämän toisen 
osan todistamiseksi suljetaan pois aksiooman (J1) jättämät muut vaihtoehdot. Jos $x^{-1} = 0$,
niin aksioomien (K8),\,(K2) ja tuloksen (c) mukaan on $1 = x \cdot x^{-1} = x \cdot 0 = 0$ eli
$0=1$, mikä on ristiriidassa aksiooman (K10) kanssa. Siis mahdollisuus $x^{-1} = 0$ on pois
suljettu. Jos $x^{-1} < 0$, niin tuloksen (d) ja väittämän (e) jo todistetun ensimmäisen osan
mukaan on $(-1) \cdot x^{-1} > 0$. Tällöin aksiooman (J4) ja oletuksen $x>0$ mukaan on myös 
$(-1) \cdot x^{-1} \cdot x > 0$, mikä sievenee kunta-aksioomien ja mainittujenen tulosten 
perusteella muotoon $1<0$. Tämäkin on mahdotonta aksiooman (J1) ja vielä todistamatta olevan 
väittämän (g) mukaan, joten sikäli kuin (g) on tosi, jää ainoaksi vaihtoehdoksi $x^{-1} > 0$.   

(f) Päätellään ensin aksioomien (J3), (K1) ja (K6) perusteella:
\[
0<y\ \ \impl\ \ x<x+y.
\]
Jatketaan tästä soveltaen aksioomaa (J3):
\[
0<x\ \ \&\ \ x<x+y\ \ \impl\ \ 0<x+y.
\]

(g) Aksioomien (J1) ja (K10) mukaan on oltava joko $0<1$ tai $1<0$. Jos oletetaan jälkimmäinen
vaihtoehto, niin aksiooman (J3) mukaan on siinä tapauksessa
\[
1 + (-1) < 0 + (-1),
\]
mikä sievenee aksioomien (K7),\,(K1) ja (K6) perusteella muotoon
\[
0 < -1.
\]
Tällöin on aksiooman (J4) perusteella oltava myös
\[
0 < (-1) \cdot (-1).
\]
Mutta väittämien (b),\,(d) perusteella on $1 = -(-1) = (-1) \cdot (-1)$, joten on päätelty,
että oletetussa vaihtoehdossa $1<0$ pätee myös $0<1$. Aksiooman (J1) mukaan tämä on kuitenkin 
mahdotonta, joten ainoaksi vaihtoehdoksi jää $0<1$. Päättelyssä ei tarvittu vielä avoimena 
olevaa väittämää (e), joten myös tämän väittämän todistus tuli samalla loppuun viedyksi. 
\loppu

Järjestetyssä kunnassa voidaan jokaiseen kunnan alkioon liittää 'itseisalkio', eli ko.\ alkion
\index{itseisarvo}%
\kor{itseisarvo} (engl.\ absolute value) määritelmällä
\[
\abs{x}\ =\ \max \{x,-x\}\ =\ \begin{cases}
                                 \ x,  &\text{jos $x \ge 0$,} \\
                                  -x,   &\text{jos $x < 0$.}
                              \end{cases}
\]       
Itseisarvon avulla on edelleen määriteltävissä kahden luvun välinen \kor{etäisyys}
\[ d(x,y)\ =\ \abs{x-y}, \]
jolloin sellaiset sanonnat kuin '$y$ on lähempänä $x$:ää kuin $z$' saavat (algebrallisen) 
sisällön. Seuraavaa itseisarvoon liittyvää tulosta tarvitaan matemaattisessa analyysissä hyvin
usein.
\begin{Lause} \label{kolmioepäyhtälö} (\vahv{Kolmioepäyhtälö})
\index{kolmioepäyhtälö!a@järjestetyn kunnan|emph} Järjestetyssä kunnassa $(\K,+,\cdot,<)$ pätee
\[
\boxed{\kehys \quad \abs{\,\abs{x} - \abs{y}\,}\ \ 
             \le\ \ \abs{x+y}\ \ \le\ \ \abs{x} + \abs{y}, \quad x,y \in \K. \quad}
\] 
\end{Lause}
\tod Päätulos on epäyhtälöistä jälkimmäinen, sillä edellinen seuraa kunta--aksioomista,
jälkimmäisestä epäyhtälöstä ja itseisarvon määritelmästä päättelyllä
\begin{align*}
&\begin{cases}
 \,\abs{x}\ =\ \abs{\,(x+y) + (-y)\,}\ \le\ \abs{x+y} + \abs{-y}\ =\ \abs{x+y} + \abs{y} \\
 \,\abs{y}\ =\ \abs{\,(x+y) + (-x)\,}\ \le\ \abs{x+y} + \abs{-x}\ =\ \abs{x+y} + \abs{x}
\end{cases} \\
& \quad\ \ \qimpl \pm\,(\abs{x}-\abs{y})\ \le\ \abs{x+y}
           \qimpl  \abs{\abs{x}-\abs{y}}\ \le\ \abs{x+y}.
\end{align*}
Kolmioepäyhtälön jälkimmäisessä osassa väitetään itseisarvon määritelmän perusteella, että
\[
\pm(x+y)\ \le\ \abs{x} + \abs{y}.
\]
Koska saman määritelmän mukaan on
\[
\pm x \le \abs{x} = a, \quad \pm y \le \abs{y} =b,
\]
niin nähdään, että väitetty epäyhtälö seuraa väittämästä
\begin{Lem} Järjestetyssä kunnassa pätee 
\[  
x \le a\ \ \ja\ \ y \le b \qimpl x+y \le a+b. 
\] 
\end{Lem}
\tod Väittämässä on neljä vaihtoehtoa:\ a) $x<a,\ y<b$,\ b) $x<a,\ y=b$,\ c) $x=a,\ y<b$,\ 
d) $x=a,\ y=b$. Vaihtoehdossa a) on $\,x<a\ \ekv\ a-x>0\,$ ja $\,y<b\ \ekv\ b-y>0$, jolloin
Lauseen \ref{kuntatuloksia} väittämästä (f) ja kunta-aksioomista seuraa
$(a-x)+(b-y)>0\ \ekv\ x+y<a+b$. Myös vaihtoehdoissa b) ja c) väite toteutuu tässä muodossa ja
vaihtoehdossa d) muodossa '$=$', kuten nähdään helposti. \loppu

\subsection{Alikunta ja kuntalaajennus}
\index{alikunta|vahv} \index{kuntalaajennus|vahv}%

Sanotaan, että kunta $(\A,+,\cdot)$ on kunnan $(\K,+,\cdot)$ \kor{alikunta} (engl.\ subfield),
jos $\A \subset \K$ ja kunnilla on samat laskuoperaatiot, ts.\ kunnan $(\A,+,\cdot)$ 
laskuoperaatiot $=$ kunnan $(\K,+,\cdot)$ operaatiot osajoukkoon $\A$ rajoitettuina. Tällöin 
kunnilla on myös yhteiset nolla- ja ykkösalkiot, eli kunnan $(\K,+,\cdot)$ nolla- ja 
ykkösalkioille pätee $\ 0,1 \in \A$. Nimittäin jos \K:n nolla--alkio on $0 \in \K$ ja \A:n 
nolla--alkio on $\theta \in \A$, niin jokaisella $x \in \A$ (esim.\ $x=\theta$) voidaan päätellä
\[
x = x + \theta \ \ \impl \ \ -x + x = -x + x + \theta \ \ 
                     \impl \ \ 0 = 0+\theta = \theta+0 = \theta.
\]
Tässä $-x \in \K$ on $x$:n vasta--alkio \K:ssa, jolloin päättely perustui ensin kunnan 
$(\A,+,\cdot)$ aksioomaan (K6) ja sen jälkeen kunnan $(\K,+,\cdot)$ aksioomiin (K1),\,(K3), (K6)
ja (K7). Vastaavaan tapaan nähdään, että ykkösalkio on kunnissa sama. 

Jos kunta $(\A,+,\cdot)$ on kunnan $(\K,+,\cdot)$ alikunta, sanotaan vastaavasti, että 
jälkimmäinen on edellisen \kor{kuntalaajennus}. Myöhemmissä luvuissa tehdään useita lukualueen
laajennuksia tyyppiä $\Q \ext \K$. Näille on yhteistä, että laajennuksissa syntyy uusia kuntia,
joiden kaikkien yhteinen alikunta on $(\Q,+,\cdot)$. Seuraavassa hyvin varovainen esimerkki
tällaisesta laajennuksesta.

\begin{Exa} \label{muuan kunta} Sovitaan, että on olemassa luku $a$ jolla on ominaisuus
\[
a^2 = a \cdot a = 2.
\]
(Pidetään tunnettuna, että $a\not\in\Q$.) Luvun $a$ ja rationaalilukujen välisistä
laskuoperaatioista sovitaan, että $1 \cdot a = a$ ja $0 \cdot a = 0\in\Q$ ja lisäksi
sovitaan, että vaihdanta-, liitäntä- ja osittelulait (K1)--(K5) ovat voimassa myös, kun
luku $a$ on operaatioissa mukana. Näiden sopimusten nojalla voidaan muodostaa lukujoukko
\[
\K = \{\, z = x + ya \mid x,y \in \Q\,\}.
\]
Tämä on $\Q$:n aito laajennus, sillä $x\in\Q\ \impl x=x+0a\in\K$, mutta $0+1a=a\not\in\Q$.
Tehtyjen sopimusten mukaan on erityisesti $0+0a=0$. Luvulla $0$ ei ole $\K$:ssa muita
esitysmuotoja, sillä jos $x+ya=0$, niin on oltava $y=0\ \impl\ x=0$, koska muuten olisi
$a=-x/y\in\Q$. (Yleisemmin voidaan päätellä, että jokaisen luvun $z\in\K$ esitysmuoto $z=x+ya$
on yksikäsitteinen.)

Jos $z_1 = x_1 + y_1a \in \K$ ja $z_2 = x_2 + y_2a \in \K$ ($x_1,x_2,y_1,y_2 \in \Q$), niin
oletettujen aksioomien (K1)--(K5) ja laskusäännön $a^2=2$ nojalla
\begin{align*}
z_1+z_2 &= (x_1+x_2) + (y_1+y_2)a, \\
z_1z_2  &= (x_1x_2 + y_1y_2a^2) + (x_1y_2 + x_2y_1)a = (x_1x_2+2y_2y_2) + (x_1y_2 + x_2y_1)a,
\end{align*}
joten $z_1+z_2\in\K$ ja $z_1z_2\in\K$. Laskuoperaatioita koskeva perusoletus (ensimmäinen
oletus edellä) on siis voimassa. Edelleen nähdään, että kunta--aksioomista ovat oletettujen
lisäksi voimassa myös (K6), (K8) ja (K10) ($0,1\in\Q$) ja (K7) (jos $z = x + ya$, niin 
$-z = -x -ya$). Lopuksi päätellään, että myös (K9) on voimassa. Nimittäin $z z^{-1} = 1$, kun
määritellään 
\[
z^{-1} = (x^2-2y^2)^{-1}\,(x-ya) = \dfrac{x}{x^2-2y^2} - \dfrac{y}{x^2-2y^2}\,a. 
\]
Tässä on $x^2-2y^2=0$ vain kun $x=y=0$ (koska $(x/y)^2 \neq 2$ kun $x,y\in\Q$ ja $y\neq 0$),
joten jokaisella $z\in\K,\ z \neq 0$ on käänteisluku $z^{-1}$. On päätelty, että $(\K,+,\cdot)$
on kunta ja siis rationaalilukujen kunnan $(Q,+,\cdot)$ aito laajennus.

Kunta $(\K,+,\cdot)$ voidaan myös järjestää. Sovitaan ensinnäkin, että $a>0$ (mahdollista,
koska $a^2=(-a)^2$). Yleisemmin nähdään, että vertailu $z_1<z_2$ ($z_1,z_2\in\K$) palautuu
kunta--algebran avulla aina viime kädessä luvun $a$ ja rationaaliluvun vertailuksi.
(Esim.\ $7-2a>1-a\ \ekv\ 6-3a>0\ \ekv\ 3/2>a$.) Jos $x\in\Q$, niin ilmeisesti on $x<a$ aina kun
$x \le 0$. Jos taas on $x>0$, niin on myös $x+a>0\ \ekv\ (x+a)^{-1}>0$, jolloin vertailukysymys
$x\,?\,a$ ratkeaa käyttämällä järjestysrelaation aksioomaa (J4)\,:
\[
x>a \qekv x-a\,=\,\frac{x^2-a^2}{x+a}\,>\,0 \qekv x^2-a^2>0 \qekv x^2>2.
\]   
Järjestetyssä kunnassa $(\K,+,\cdot,<)$ luku $a$ sijoittuu siis rationaalilukujen 'väliin'\,:
Jokaisen luvun $x \in \Q$ kohdalla voidaan ratkaista, onko $a<x$ vai $a>x$, ja tämä vertailu
perustuu vain rationaalilukujen $0,x,2$ ja $x^2$ vertailuun. \loppu
\end{Exa}

\subsection{Juuriluvut ja murtopotenssit}

Esimerkin \ref{muuan kunta} kuntaa huomattavasti käyttökelpoisempi kuntalaajennus saadaan
aikaan, kun rationaalilukujen joukko laajennetaan lukujoukoksi, jossa kuntaoperaatioiden
$x,y \map x+y,\,x-y,\,xy,\,x/y$ lisäksi sallitaan laskuoperaatiot
\[
x \map \sqrt[m]{x}, \quad x>0, \quad m\in\N,\ m \ge 2.
\]
\index{juuriluku}%
Tässä luvulla $\sqrt[m]{x}$ tarkoitetaan \kor{juurilukua} $y$, joka toteuttaa ehdot
\[
y^m = x\ \ \ja\ \ y>0.
\]
Sopimuksella $\sqrt[1]{x}=x$ tulee juuriluku $\sqrt[m]{x}$ määritellyksi $\forall m\in\N$.
Sanotaan, että $\sqrt[m]{x}$ on luvun $x$ \kor{neliöjuuri}, jos $m=2$
(lyhennysmerkintä $\sqrt{x}$), \kor{kuutiojuuri}, jos $m=3$, tai  yleisemmin $m$:\kor{s juuri},
luetaan '$m$:s juuri $x$'. Tällainen luku voidaan yksinkertaisesti sopia 'olemassa olevaksi'
samaan tapaan kuin meneteltiin Esimerkissä \ref{muuan kunta} luvun $a\ (=\sqrt{2})$ kohdalla.
Lisäehto $y>0$ (joka jo viittaa järjestysrelaatioon) tarvitaan takaamaan luvun $y$
yksikäsitteisyys, sillä $m$:n ollessa parillinen on $(-y)^m=y^m$. Yksikäsitteisyys mainitulla
lisäehdolla seuraa identiteetistä (ks.\ Propositio \ref{kuntakaava})
\[
(y_1-y_2)(y_1^{m-1}+y_1^{m-2}y_2+ \ldots + y_2^{m-1}) = y_1^m-y_2^m.
\]
Jos tässä oletetetaan, että $y_1^m=y_2^m=x$ ja $y_1,\,y_2>0$, niin oikea puoli $=0$ ja vasemmalla
puolella on jälkimmäinen tekijä $>0$, joten on oltava $y_1-y_2=0$.  

Laskuoperaatioilla $x \map \sqrt[m]{x}$ laajennettua lukujoukkoa, lähtökohtana rationaaliluvut,
merkittäköön symbolilla $\J$. Joukko $\J$ määritellään yksinkertaisesti koostuvaksi luvuista,
jotka saadaan äärellisellä määrällä kuntaoperaatioita ja operaatioita $x \map \sqrt[m]{x}$
lähtien luvuista $0,1\in\Z$. Tällöin on ilmeistä, että jos $x,y\in\J$, niin myös
$x \pm y,\,xy\,,x/y\in\J\ (y \neq 0)$, joten laskuoperaatioita koskeva perusoletus on voimassa.
Lukujen $x\in\J$ 'ulkonäkö' voi kylläkin olla konstikas.
\begin{Exa}
\[
\left(3+\sqrt{2}
-\sqrt[8]{\frac{\sqrt[6]{5-\sqrt[4]{3}}}{\sqrt[4]{3+\sqrt[3]{2}}}+\frac{1}{\sqrt[7]{7}}}\right)/
\left(2+\sqrt{3}
+\sqrt[8]{\frac{\sqrt[6]{3-\sqrt[4]{5}}}{\sqrt[4]{2+\sqrt[3]{3}}}+\sqrt[7]{7}}\right)
                                                        \ \in\ \J. \loppu
\]
\end{Exa}
Juuriluvuilla laskettaessa oletetaan kunta--algebran laskusäännöt päteviksi, ts.\ oletetaan,
että $(\J,+,\cdot)$ on kunta. Tästä oletuksesta sekä juuriluvun määritelmästä voidaan johtaa
seuraavat yleiset laskusäännöt (ol.\ $x,y>0,\ m,n\in\N$):
\[
\text{a)}\ \ \sqrt[m]{x} = \sqrt[mn]{x^n}. \qquad
\text{b)}\ \ \sqrt[m]{x}\sqrt[m]{y} = \sqrt[m]{xy}. \qquad
\text{c)}\ \ \left(\sqrt[m]{x}\right)^{-1} = \sqrt[m]{x^{-1}}.
\]
Sääntöjen perustelemiseksi merkitään $a=\sqrt[m]{x}$, $b=\sqrt[m]{y}\,$, jolloin on
$a,b>0\ \impl ab,a^{-1}>0$. Juuriluvun määritelmään ja kunta--algebraan vedoten voidaan
tällöin päätellä:
\begin{align*}
&\text{a)}\ \ a^m=x\ \ \impl\ \ \ (a^m)^n = a^{mn} = x^n\ \  \impl\ \ a=\sqrt[mn]{x^n}. \\[1mm]
&\text{b)}\ \ a^m=x\ \ja\ b^m=y\ \ \impl\ \  a^mb^m =(ab)^m = xy\ \ \impl\ \ ab=\sqrt[m]{xy}. \\
&\text{c)}\ \ a^m=x\ \ \impl\ \ (a^m)^{-1} = (a^{-1})^m=x^{-1}\ \ \impl\ \ a^{-1}=\sqrt[m]{x^{-1}}.
\end{align*}

Juuriluvuilla laskemisen säännöt saadaan kätevästi yhdistetyksi potenssien laskusääntöihin
\index{murtopotenssi}%
(vrt.\ edellinen luku), kun määritellään luvun $x>0$ \kor{murtopotenssi} asettamalla
\[
x^{p/q} = \sqrt[q]{x^p}, \quad p\in\Z,\ q\in\N,\ q \ge 2.
\]
Tämä määrittelee yksiselitteisesti luvun $x^r$ jokaisella $r\in\Q$, sillä jos $r$ esitetään
perusmuodossa $r=p/q,\ q\in\N$, niin määritelmän ja säännön a) perusteella on
$x^{(pn)/(qn)}=\sqrt[qn]{x^{pn}}=\sqrt[nq]{(x^p)^n}=\sqrt[q]{x^p}=x^{p/q}\ \forall n\in\N$, eli
$x^r$ ei riipu $r$:n esitysmuodosta. Säännöistä a)--c) voidaan nyt johtaa murtopotensseille
samat laskusäännöt kuin kokonaislukupotensseille (Harj.teht.\,\ref{H-I-2: murtopotenssit})\,:
\[
\boxed{\kehys \quad x^rx^s=x^{r+s}, \quad x^ry^r=(xy)^r, \quad (x^r)^s=x^{rs}, 
                                                         \quad x,y>0,\ r,s\in\Q. \quad} 
\]

Luvut $x\in\J,\ x\not\in\Q$ on myös mahdollista sijoitella rationaalilukujen 'väleihin' niin,
että syntyy järjestetty kunta $(\J,+,\cdot,<)$. Järjestysrelaation määrittely yleisessä
tapauksessa vaatisi kuitenkin tähänastista vakavampia lukuteoreettisia pohdiskeluja, siksi
asiaan ei toistaiseksi puututa. Todettakoon ainoastaan, että suotuisissa erikoistapauksissa
vertailu on mahdollista palauttaa suoraan rationaalilukujen vertailuksi samalla tavoin kuin
Esimerkissä \ref{muuan kunta} edellä. Ideana on tällöin purkaa juurilausekkeet päättelyllä: Jos
$x,y>0$ ja $m\in\N$, niin $x<y\ \ekv\ x^m<y^m$ (Harj.teht.\,\ref{H-I-2: potenssien järjestys}).
\begin{Exa} Kumpi on suurempi, $\,x=\sqrt{\sqrt{2}+5}\,$ vai $\,y=\sqrt[4]{41}$\,? \end{Exa}
\ratk Koska $\,x<y\ \ekv\ x^4<y^4\ (x,y>0)$, niin päätellään
\[
x<y \qekv (\sqrt{2}+5)^2 < 41 \qekv 10\sqrt{2}+27 < 41 \qekv 10\sqrt{2} < 14.
\]
Jatkamalla tästä päättelyllä $\,a<b\ \ekv\ a^2<b^2\ (a,b>0)$ nähdään, että
\[
10\sqrt{2} < 14 \qekv 200 < 196.
\]
Koska viimeinen epäyhtälö ei toteudu, vaihdetaan epäyhtälöiden suunnat ja päätellään:
$x>y\ \ekv\ 200>196$. Siis $x$ on suurempi. \loppu

\Harj
\begin{enumerate}

\item
Lukujoukossa $\K$ on vain luvut 0,1 ja $\diamondsuit=1+1$ (kaikki keskenään eri suuria).
Määrittele (jos mahdollista) $\K$:n laskusäännöt (yhteenlasku, kertolasku, vastaluvut, 
käänteisluvut) siten, että $(\K,+,\cdot)$ on kunta.

\item
\index{nollasääntö!a@tulon}%
Vedoten kunta--aksioomiin tai Lauseen \ref{kuntatuloksia} väittämiin näytä, että jokaisessa
kunnassa pätee: \ a) \ $x \cdot y = 0$ $\ \ekv\ $ $x=0$ $\tai$ $y=0$
(\kor{tulon nollasääntö}), \ \ b) \ $(-x)\cdot (y)=-(x\cdot y)$, \ 
c) \ $(-x)\cdot (-y)=x\cdot y$, \ d) $0$:lla ei ole käänteisalkiota.

\item \label{H-I-2: kuntakaavat}
Näytä, että jokaisessa kunnassa pätee \vspace{1mm}\newline
a)\,\ $x^n-y^n = (x-y)\sum_{k=0}^{n-1} x^{n-1-k}y^k,\,\ n\in\N$ \vspace{1mm}\newline
b)\,\ $x^n+y^n = (x+y)\sum_{k=0}^{n-1} (-1)^kx^{n-1-k}y^k,\,\ $ 
      jos $n\in\N$ ja $n$ on pariton \vspace{1mm}\newline
c)\,\ $1+x + \ldots + x^n = (x^{n+1}-1)(x-1)^{-1},\,\ n\in\N,\ x \neq 1$

\item \label{H-I-2: järjestysaksioomat}
Näytä, että järjestetyn kunnan aksiooma (J3) voidaan korvata aksioomalla
(J3') $x<y \,\ekv\, x-y<0$.

\item \label{H-I-2: järjestetyn kunnan väitteitä}
Näytä, että jokaisessa järjestetyssä kunnassa pätee: \newline
a) \ $x>1\,\ \impl\,\ 0 < x^{-1} < 1$ \newline
b) \ $x>0\ \ja\ y>1\ \impl\ 0 < x/y < x$ \vspace{0.2mm}\newline
c) \ $\abs{x \cdot y}=\abs{x}\cdot\abs{y}$ \vspace{0.2mm}\newline
d) \ $\abs{x^{-1}}=\abs{x}^{-1}$ \newline
e) \ $x^2<y^2\ \ekv\ \abs{x} < \abs{y}$ \newline
f) \ $0 \le x \le a\ \ja\ 0 \le y \le b\ \impl\ 0 \le xy \le ab$

\item \label{H-I-2: potenssien järjestys}
Lähtien tehtävän \ref{H-I-2: kuntakaavat} kaavasta a) näytä, että järjestetyssä kunnassa pätee:
Jos $x>0$, $y>0$ ja $n\in\N,\ n \ge 2$, niin $x<y\ \ekv\ x^n<y^n$.

\item
Aseta luvut $a$, $17/12$ ja $72/51$ suuruusjärjestykseen Esimerkin \ref{muuan kunta} kunnassa.
Suorita vertailut tarkasti!

\item \label{H-I-2: murtopotenssit}
Perustele murtopotenssien laskusäännöt \,\ a) $x^rx^s=x^{r+s}$,\,\ b) $x^ry^r=(xy)^r$,
c) $(x^r)^s=x^{rs}$ \ ($x,y>0,\ r,s\in\Q$).

\item
Selvitä ilman laskinta lukujen $x,y\in\J$ suuruusjärjestys palauttamalla vertailu
rationaalilukujen vertailuksi\,:\vspace{1mm}\newline 
a)\,\ $x=\sqrt[3]{1020},\ \ y=\sqrt{102}, \quad$
b)\,\ $x=\sqrt[3]{3},\ \ y=\sqrt[4]{43/10},$\vspace{1mm}\newline
c)\,\ $x=2-\sqrt{3},\ \ y=1/\sqrt{4\sqrt{3}+7}.$

\item
Olkoon $(\K,+,\cdot,<)$ järjestetty kunta. Näytä: \ a) \ $x+1>x\ \forall x\in\K$. \newline
b) Joukossa $\K$ on äärettömän monta eri alkiota. 

\item (*)
Näytä käsinlaskulla, että pätee
\[
\text{a)}\ \ \frac{577}{408}-\frac{1}{400000}\ <\ \sqrt{2}\ <\ \frac{577}{408}\,, \qquad
\text{b)}\ \ \sqrt{2}\ <\ \frac{665857}{470832}\,.
\]

\item (*)
Näytä, että järjestetyn kunnan aksiooma (J2) voidaan korvata aksioomalla
(J2') $x>0 \,\ja\, y>0 \,\impl\, x+y>0$.

\item (*)
Halutaan määritelllä pienin mahdollinen rationaalilukujen kunnan laajennus $(\K,+,\cdot)$ siten,
että $\sqrt{2}\in\K$ ja $\sqrt{3}\in\K$. Näytä, että \ a) $\K$ koostuu luvuista muotoa
$\,x+y\sqrt{2}+z\sqrt{3}+u\sqrt{6},\ x,y,z,u\in\Q$, \ b) kunnassa $(\K,+,\cdot)$ on
määriteltävissä järjestysrelaatio, joka perustuu vain rationaalilukujen vertailuun.

\item(*) \label{H-I-2: Big Ben} \index{zzb@\nim!Big Ben}
(Big Ben) Tarkastellaan joukkoa $\K=$\{kellon viisarit\}. Jokainen viisari $v\in\K$ on 
\kor{lukupari} $(r,\theta)$, missä $r=$ viisarin pituus (yksikkö m) ja $\theta =$ viisarin 
suunta asteina, mitattuna klo 12:sta positiivisena myötäpäivään tai negatiivisena vastapäivään.
Sovitaan, että \ $(r_1,\theta _1)=(r_2,\theta _2)$, jos joko (i) $r_1=r_2=0$ tai
(ii) $r_1=r_2>0$ ja $\theta _1-\theta _2=k \cdot 360,\ k\in\Z$. Jos $r=0$, sanotaan viisaria
$(r,\theta)$ \kor{nollaviisariksi}, merkitään\ $0_v$. Määritellään viisareiden kertolasku
seuraavasti:
\[
(r_1,\theta _1)\cdot (r_2, \theta _2) = (r_1r_2,\,\theta _1+\theta _2).
\]
a) Näytä, että kertolaskulle pätee sekä vaihdantalaki että liitäntälaki.
\newline
b) Määrittele ykkösviisari sekä viisarin $(r, \theta) \neq 0_v$ käänteisviisari.
\newline
c) Big Ben, jonka minuuttiviisarin pituus on 4 ja tuntiviisarin pituus on 2,
näyttää aikaa noin klo 3. Mikä kellonaika on tarkemmin kyseessä (sekunnin tarkkuus\,!), kun
tiedetään, että myös Big Benin käänteiskello Small Ben näyttää samaan aikaan aivan järkevää
(vaikkakin toista) kellonaikaa?  --- Huomaa, että myös Small Benin minuuttiviisari on pidempi\,!

\end{enumerate}  % Kunta
\section{Logiikan ja joukko-opin merkinnöistä} \label{logiikka}
\alku
\sectionmark{Logiikka ja joukko-oppi}
\index{logiikka|vahv}

\kor{Logiikassa} (ja yleisemminkin matematiikassa) tarkastellaan väittämiä eli 
\index{propositio (väitelause)}%
\kor{propositioita}\footnote[2]{Proposition sijasta käytetään suomenkielisissä teksteissä usein
termiä \kor{väitelause} tai vain 'lause'. Tässä tekstissä termillä 'lause' on erikoismerkitys 
(= 'teoreema'), termiä 'propositio' sen sijaan käytetään sekä yleis- että erikoismerkityksessä,
ks.\ alaviite Luvussa \ref{ratluvut}. \index{vzy@väitelause|av} \index{lause (väitelause)|av}},
joilla on tietty
\index{totuusarvo}%
\kor{totuusarvo}, joko 'tosi' (totuusarvo = 1, tai T = True) tai 'epätosi'
(totuusarvo = 0, tai F = False).
\begin{Exa} Lausumista
\begin{align*}
A &= \text{'eilen satoi'} \\
B &= \text{'eilen oli pouta'} \\
C &= \text{'huomenna sataa'} \\
D &= \text{'huomenna saisi jo sataa'} \\
E &= \text{'kolme per neljä on pienempi kuin kaksi per kolme'}
\end{align*}
kolmea ensimmäistä voidaan pitää propositioina, sikäli kuin lausumat rajataan tiettyä päivää ja
paikkaa koskeviksi (ja jätetään huomiotta mahdolliset mittausongelmat). Samoin $E$ on 
propositio. Lausuma $D$ ei ole propositio. \loppu \end{Exa} 
Esimerkkiväittämistä $A$ ja $B$ ovat toistensa
\index{looginen!negaatio (komplementti)} \index{negaatio (looginen)}%
\kor{negaatioita} eli \kor{komplementteja}. 
Merkitään $B = \neg A,\ A = \neg B$, tai vähemmän muodollisesti $B = ei(A)$, $A = ei(B)$.
Väittämän $C$ totuusarvo ei ole (tänään) tiedossa --- tämä ei siis ole ongelma logiikassa.

\index{looginen!operaattori} \index{operaattori!a@looginen}%
\kor{Loogisten operaattorien} $\wedge$ ('ja'), $\tai$ ('tai'), $\impl$ ('seuraa'), $\ekv$
('täsmälleen kun') avulla voidaan propositioista johtaa uusia propositioita. Erityisesti jos 
$A$ ja $B$ ovat propositioita, niin $A \wedge B$, $A \tai B$, $A \impl B$ ja $A \ekv B$ ovat
myös propositioita. Näistä kahden ensimmäisen tulkinta on  ilmeinen: $A \wedge B$ on tosi kun
$A$ ja $B$ ovat molemmat tosia, muulloin epätosi, ja $A \tai B$ on tosi täsmälleen kun ainakin
toinen  väittämistä $A,B$ on tosi.  Väittämä $A \wedge \neg A$ on \kor{identtisesti epätosi} 
(epätosi jokaisella $A$), luonnollisella kielellä 'mahdoton'. Tällainen väittämä on
\index{looginen!ristiriita} \index{ristiriita (looginen)} \index{identtisesti (epä)tosi}%
\kor{loogisen ristiriidan} perusmuoto. Väittämä $A \tai \neg A$ on puolestaan 
\kor{identtisesti tosi} (tosi jokaisella $A$), eli suhteessa väittämään $A$ 'mitäänsanomaton'.
\begin{Exa} Jos $A$ on epätosi ja $B$ ja $C$ molemmat tosia, niin $(A \wedge B) \tai C$ on tosi
ja $A \wedge (B \tai C)$ on epätosi. Jälkimmäiset kaksi väittämää eivät siis ole yleisesti 
samanarvoiset (eli sulkeita ei voida poistaa). \loppu \end{Exa}

\subsection{Implikaatio}
\index{implikaatio|vahv}

Implikaatioväittämän $A \impl B$ looginen tulkinta ei ole aivan ilmeinen. Mahdollisia lukutapoja
ovat ensinnäkin: \index{riittävä ehto} \index{vzy@välttämätön ehto}%
\begin{align*}
A \impl B\ : \quad &\text{$A$:sta seuraa $B$}                 \\
                   &\text{$A$ \kor{implikoi} $B$:n}           \\
                   &\text{jos $A$, niin $B$}                  \\
                   &\text{aina kun $A$, niin $B$}             \\
                   &\text{aina $B$, kun $A$}                  \\
                   &\text{$A$ vain, kun $B$}                  \\
                   &\text{$A$ on $B$:n \kor{riittävä ehto}}   \\
                   &\text{$B$ on $A$:n \kor{välttämätön ehto}}
\end{align*}
Sovelluksissa, myös luonnollisessa kielessä, voidaan implikaatioväittämä usein tulkita niin, 
että $A$ on 'syy' ja $B$ on 'seuraus'. Logiikassa ei mitään 'syyllisyyttä' kuitenkaan 
edellytetä, vaan implikaatio voi olla puhdas sattumakin ('sattumoisin aina $B$ kun $A$'). 
Loogisessa kalkyylissä väittämän $A \impl B$ totuusarvo voidaan laskea tulkinnoista
(ks.\ myös Harj.teht.\,\ref{H-I-3: tautologioita}a)
\begin{align}
A \impl B \quad &\ekv \quad \neg\,(A \wedge \neg B)    \tag{a} \\
                &\ekv \quad \neg A \tai (A \wedge B).  \tag{b}
\end{align}
Tässä $P \ekv Q$ luetaan '$P$:llä ja $Q$:lla on sama totuusarvo', ks.\ ekvivalenssinuolen 
tulkinnat jäljempänä. Implikaation tulkinnan (b) mukaan $A \impl B$ on tosi täsmälleen, kun 
joko $A$ on epätosi tai $A$ ja $B$ ovat molemmat tosia. Erityisesti siis 'mahdottomasta seuraa 
mitä tahansa', eli väittämä $A \impl B$ on tosi jokaisella $B$ (eli $B$:n suhteen 
'mitäänsanomaton'), jos $A$ on epätosi. 
\begin{Exa} Tulkinnoista (a)--(b) nähdään, että väittämä $A \impl A$ on identtisesti tosi, eli
väittämä 'ei sano mitään' $A$:sta.
\loppu \end{Exa}
\begin{Exa} Jos $0$ ja $1$ ovat jonkin kunnan nolla- ja ykkösalkiot, niin propositio
[\,$0 = 1\,\ \impl\,\ \text{kirjoittaja on mainio matemaatikko}$\,] on (loogisesti) tosi. \loppu 
\end{Exa}
Matematiikan lauseet, propositiot, lemmat ja korollaarit ovat tyypillisesti muotoa 'Jos .. 
[oletukset], niin .. [väitös]', eli muotoa $A \impl B$, missä $A$ on \pain{oletus} (oletukset)
ja $B$ on \pain{väitös}. Loogista päättelyä, joka osoittaa lauseen tai vastaavan todeksi 
(sillä matematiikan lauseet ovat tosia!) sanotaan
\index{todistus}%
\kor{todistukseksi} (engl.\ proof). Jos lause
on muotoa $A \impl B$, niin todistaminen (eli proposition $A \impl B$ todeksi näyttäminen)
tapahtuu \pain{olettamalla}, että $A$ on tosi ja \pain{nä}y\pain{ttämällä}, että tällöin myös
$B$ on tosi (riittää, koska $A \impl B$ on tosi, jos $A$ on epätosi!). Tämän nk.
\index{todistus!a@suora, epäsuora}%
\kor{suoran todistuksen} ohella toinen mahdollinen todistustapa on 
\index{epzys@epäsuora todistus}%
\kor{epäsuora todistus}. Epäsuoran todistuksen ideana on näyttää, että jos todistettavan
väittämän  \pain{ne}g\pain{aatio} \pain{on} \pain{tosi}, niin seuraa
\pain{loo}g\pain{inen} \pain{ristiriita} (muotoa '$C$ tosi ja epätosi', missä $C$ on
propositio), jolloin päätellään, että negaation on oltava epätosi ja väittämän siis tosi
(ks.\ Harj.teht. \ref{H-I-3: modus tollens}). Implikaatioväittämän tapauksessa epäsuoran
todistuksen rakenne on seuraava, vrt.\ em.\ tulkinta (a).
\begin{itemize}
\item[(1)] Oletetaan, että $A$ on tosi ja $B$ epätosi (eli $A \impl B$ epätosi).
\item[(2)] Näytetään, että oletus (1) johtaa loogiseen ristiriitaan. Päätellään, että oletus oli
           väärä ja siis $A \impl B$ tosi.
\end{itemize}
Oletusta (1) (tai osaoletusta '$B$ epätosi') sanotaan
\index{vastaoletus}%
\kor{vastaoletukseksi}, ja koko 
todistustavasta käytetään myös nimitystä \kor{todistus vastaoletuksella} (engl. proof by 
contradiction). --- Huomattakoon, että vastaoletus ei nimestään huolimatta ole oletuksen
negaatio vaan pikemminkin 'vastaväitös'. 

Epäsuoraan todistustapaan turvauduttiin itse asiassa jo edellä Lauseen \ref{kuntatuloksia}  
kohdassa (g). Tässä väittämä oli muotoa $A \impl B$, missä
\begin{align*}
A \quad &= \quad \text{aksioomat (K1)--(K10) ja (J1)--(J4)}, \\
B \quad &= \quad 0 < 1.
\end{align*}
Vastaoletus, että $B$ on epätosi, johti loogiseen ristiriitaan '$C$ tosi ja epätosi',
missä $C=$ (K10)$\,\wedge\,$(J1). 

\subsection{Ekvivalenssi}
\index{ekvivalenssi|vahv}%
 
Jos $A$ ja $B$ ovat propositioita, niin $A \ekv B$ on propositio, joka kertoo, että $A$ ja $B$ 
ovat samanarvoiset eli \kor{ekvivalentit} väittämät. Tämä tarkoittaa yksinkertaisesti, että
väittämien $A$ ja $B$ totuusarvot ovat samat, eli joko molemmat ovat tosia tai molemmat ovat 
epätosia. Ekvivalenssin yleisiä lukutapoja ovat
\begin{align*}
A \ekv B: \quad &\text{$A$ ja $B$ ekvivalentit, samanarvoiset} \\
                &\text{$A$ ja $B$ yhtäpitävät} \\
                &\text{$A$ \kor{silloin ja vain silloin kun} (engl. if and only if) $B$} \\
                &\text{$A$ täsmälleen kun $B$} \\
                &\text{$A$ \kor{joss} (engl. iff) $B$}
\end{align*} 
Ekvivalenssinuolen avulla voidaan esim. ilmaista jokin väittämä toisin sanoin tai merkinnöin, 
tai vain toisessa muodossa. Kyse voi tällöin olla esim.\ jonkin merkintätavan 
\kor{määritelmästä}, tai yleispätevästä toisinnosta eli
\index{looginen!tautologia} \index{tautologia (looginen)}%
\kor{tautologiasta}. 
\begin{Exa} \label{ekvivalensseja}
\begin{align*}
0< x < 1 \quad\quad          &\ekv \quad\quad 0 < x\ \ \wedge\ \ x < 1 \\
A \impl B \quad\quad         &\ekv \quad\quad \neg B \impl \neg A \\
A \impl B \impl C \quad\quad &\ekv \quad\quad A \impl B\ \wedge\ B \impl C \\
A \ekv B \quad\quad          &\ekv \quad\quad A \impl B\ \wedge\ B \impl A \quad \loppu
\end{align*} \end{Exa}
Ensimmäinen esimerkki määrittelee merkinnän $0<x<1$. Toisessa esimerkissä on kyse yleisestä 
(myös hyödyllisestä!) tautologiasta (Harj.teht. \ref{H-I-3: tautologioita}b). Kolmannessa 
esimerkissä määritellään kahden implikaatioväittämän muodostama 
\index{pzyzy@päättelyketju}%
\kor{päättelyketju}\footnote[2]{Jos matemaattinen lause yms.\ on implikaatioväittämä muotoa 
$A \impl B$, niin todistus on tyypillisesti päättelyketju muotoa
$\,A \impl C_1 \impl C_2 \impl \ldots \impl C_n \impl B$, missä osaväittämät
$A \impl C_1,\ C_1 \impl C_2,\ \ldots\ C_n \impl B$ joko ovat ilmeisen tosia, tai tosia muiden 
tunnettujen lauseiden (tai erikseen todistettavien aputulosten) perusteella. Päättelyketjun
määrittely noudattaa tässä matematiikan käytäntöä --- formaalissa logiikassa tällaista
sopimusta ei ole.}. 
Viimeisessä esimerkissä tulkitaan itse väittämä $A \ekv B$. Tämän tulkinnan mukaisesti 
ekvivalenssiväittämä todistetaan osoittamalla erikseen \fbox{\impl} $(\ A \impl B\ )$ ja
\fbox{\limp} $(\ A \limp B\ )$ eli näyttämällä implikaatiot tosiksi molempiin suuntiin.
Jos kyseessä on yksinkertainen tautologia, voidaan myös muodostaa
\index{totuustaulu}%
\kor{totuus}(arvo)\kor{taulu}(kko), jossa käydään läpi kaikki mahdollisuudet.
\begin{Exa} \label{de Morgan logiikassa} Näytä identtisesti tosiksi
\vahv{de Morganin lait}: \vspace{1mm}\newline
a)\,\ $\neg\,(A \tai B)\ \ekv\ \neg A \wedge \neg B \qquad$
b)\,\ $\neg\,(A \wedge B)\ \ekv\ \neg A \tai \neg B$
\end{Exa}
\ratk Totuustaulussa (alla) on merkitty $P = A \tai B$, $Q = A \wedge B$ ja käyty läpi
propositioiden $A,B$ kaikki totuusarvoyhdistelmät  ($4$ kpl). Taulukosta nähdään, että
propositioiden $\neg P$ ja $\neg A \wedge \neg B$, samoin propositioiden $\neg Q$ ja 
\mbox{$\neg A \tai \neg B$} totuusarvot ovat kaikissa tapauksissa samat. Tämä todistaa
väitteet. \loppu

\begin{tabular}{cccccccccccccc}
$A$ & $B$ & $\neg A$ & $\neg B$ &  &  & $P$ & $\neg P$ & $\neg A \wedge \neg B$ & & & 
$Q$ & $\neg Q$ & $\neg A \tai \neg B$ \\ \hline
1 & 1 & 0 & 0 & & & 1 & 0 & 0 & & & 1 & 0 & 0 \\
1 & 0 & 0 & 1 & & & 1 & 0 & 0 & & & 0 & 1 & 1 \\
0 & 1 & 1 & 0 & & & 1 & 0 & 0 & & & 0 & 1 & 1 \\
0 & 0 & 1 & 1 & & & 0 & 1 & 1 & & & 0 & 1 & 1
\end{tabular}

Loogisten operaattorien, samoin järjestys- ja samastusrelaatioiden yms.\ yhteydessä negaation 
merkkinä käytetään yleisesti päälleviivausta.
\begin{Exa}
\begin{align*}
A \not\impl B \,\ &\ekv \,\ \neg\,( A \impl B ) \,\ \ekv \,\ A\,\wedge\,\neg B \\
A \not\impl B \,\ &\not\ekv \,\ A \impl \neg B \\
x \neq y \,\      &\ekv \,\ \neg\,(x=y) \loppu
\end{align*}
\end{Exa}

\subsection{Predikaatti ja kvanttorit}
\index{predikaatti|vahv} \index{kvanttori|vahv}%

Logiikassa \kor{predikaatti} on sellainen lausuma, jossa on yksi tai useampia vapaita 
\kor{muuttujia}. Predikaatista tulee propositio, kun muuttujat \kor{sidotaan} --- sellaisenaan
predikaatti ei ole propositio. Esimerkiksi jos tarkastellaan rationaalilukuja ja kirjoitetaan
\[
P(x): \quad x>2, \quad\quad\quad Q(x,y): \quad x=y^2,
\]
\index{yksipaikkainen predikaatti} \index{kaksipaikkainen predikaatti}%
niin $P(x)$ on \kor{yksipaikkainen} (so.\ yhden muuttujan sisältävä) predikaatti, nimeltään 
\kor{epäyhtälö} (engl.\ inequality), ja $Q(x,y)$ on \kor{kaksipaikkainen} predikaatti, nimeltään
\kor{yhtälö} (engl.\ equation). Jokaisella muuttujan $x$ arvolla $P(x)$ on propositio, samoin 
$Q(x,y)$ kun molemmat muuttujat kiinnitetään. Esim.\ $P(5/2)$ ja $Q(4,2)$ ovat tosia, $P(1)$ ja
$Q(0,1)$ epätosia. Muuttujien sitominen voi tapahtua myös \kor{kvanttorien} avulla. Kvanttoreita
ovat symbolit '$\forall$' ja '$\exists$', jotka luetaan
\begin{align*}
&\forall \quad \text{'kaikille', 'jokaiselle'}, \\
&\exists \quad \text{'on olemassa'}.
\end{align*}
Esimerkiksi jos $X \subset \Q$, niin ym.\ predikaattiin $P(x)$ liittyviä propostitioita ovat
\[
A: \quad \exists x \in X\ (\,P(x)\,), \quad\quad\quad B: \quad \forall x \in X\ (\,P(x)\,).
\]
Tässä kvanttorien ulottuvuus on merkitty sulkeilla. Jos sulkeet halutaan välttää, niin $A$:n 
vähemän formaali muoto on
\[
A: \quad \exists x \in X\ \,\text{siten, että}\ \,P(x),
\]
missä \,'siten, että'\, voidaan haluttaessa lyhentää muotoon \,'s.e.'\,. Kummassakin 
propositiossa tullaan toimeen ilman sulkeita myös, kun kirjoitetaan yksinkertaisemmin
\[
A: \quad P(x)\ \ \text{jollakin}\ x \in X, \quad\quad\quad B: \quad P(x)\ \ \forall x \in X.
\]
\begin{Exa} Jos $X=\Q$, niin predikaatista $P(x):\,x^2 \neq 2$ johdettu propositio 
$B:\,x^2 \neq 2\ \forall x\in\Q$ on tosi (Harj.teht. \ref{H-I-3: sqrt 2}). \loppu
\end{Exa}
Em.\ propositioiden $A,B$ negaatiot saadaan ymmärrettävämpään muotoon suorittamalla
\pain{ne}g\pain{aation} p\pain{urku} seuraavasti:
\begin{align*}
\neg\,[\,\exists x \in X\,(\,P(x)\,)\,] \quad 
                 &\ekv \quad \forall x \in X\ (\,\neg\,P(x)\,), \\
\neg\,[\,\forall x \in X\ (\,P(x)\,)\,] \quad 
                 &\ekv \quad \exists x \in X\ (\,\neg\,P(x)\,).
\end{align*}
Jälkimmäisen säännön mukaisesti proposition $B:\ P(x)\,\forall x \in X$ näyttämiseen 
\pain{e}p\pain{ätodeksi} riittää löytää yksikin \pain{vastaesimerkki} $ x\in X$, jolle
$P(x)$ on epätosi.
\begin{Exa} Jos $X = \{x \in \Q \mid x < 1\ \}$, niin propositio
\[
A: \quad \forall x \in X\ \exists y \in X\ (y>x)
\]
on tosi. Jos $X = \{x \in \Q \mid x \le 1\}$, niin $A$:n negaatio
\[
\neg A: \quad \exists x \in X\ [\,\not\exists y \in X\ (y>x)\,] \,\ \ekv\,\
              \exists x \in X\ [\,\forall  y\in X\,(\,y \le x\,)\,]
\]
on tosi. \loppu 
\end{Exa}
Formaali kvanttorimerkintä $\forall x \in X$ jätetään matemaattisissa teksteissä usein
merkitsemättä silloin kun on selvää, että kyse on koko tiettyä joukkoa $X$ (esim.\ $X=\Q$)
koskevasta päättelystä.
\begin{Exa} \label{näkymätön kvanttori} Rationaalilukuja koskeva päätelmä
\[
2x^2+x<1\,\ \ekv\,\ -1<x<\tfrac{1}{2}
\]
ratkaisee $\Q$:ssa (oikein!) epäyhtälön $2x^2+x<1$. Päätelmä on muodoltaan predikaatti mutta
tulkittavissa propositioksi, jossa sidonta $\forall x\in\Q$ on sujuvuussyistä 'näkymätön'.
\loppu
\end{Exa} 

\subsection{Joukko-oppi}
\index{joukko-oppi|vahv}%

Logiikan ohella abstraktin ajattelun perimmäisiä perusteita käsittelee matematiikan laji nimeltä
\kor{joukko-oppi} (engl. set theory). Tässä todettakoon ainoastaan lyhyesti eräiden logiikan ja
joukko-opin perusideoiden välinen yhteys. Ensinnäkin 'mahdottoman väittämän' muotoa 
$A \wedge \neg A$ vastine joukko-opissa on
\index{tyhjä joukko}%
\kor{tyhjä joukko}, jonka symboli on '$\emptyset$'. 
Tyhjässä joukossa ei ole alkioita, eli $x \in \emptyset$ on aina epätosi. 
 
Loogisen negaation $\neg A$ vastine joukko-opissa on joukon $A$ 
\index{komplementti (joukon)}%
\kor{komplementti}, joka merkitään $\complement(A)$ ja määritellään
\[
x \in \complement(A) \quad \ekv \quad x \not\in A.
\]
Käytännössä komplementti on määriteltävä jonkin
\index{universaalijoukko}%
\kor{universaalijoukon}\footnote[2]{Termi 
'universaalijoukko' saattaa herättää mielikuvan joukosta, joka sisältää kirjaimellisesti 
'kaiken'. Tämän tyyppisiä ajatelmia joukko-opissa olikin sen teorian alkuvaiheissa, mutta niiden
huomattiin johtavan paradokseihin, ts.\ loogisiin mahdottomuuksiin. Sittemmin täsmentyneistä 
joukko-opin aksioomista seuraakin, että 'mikään ei sisällä kaikkea'.} $U$ suhteen. Tällä
ymmärretään sellaista joukkoa, johon kaikki tarkasteltavat joukot (mukaan lukien joukkojen 
komplementit) sisältyvät osajoukkoina. 
\begin{Exa} Olkoon $P(x)$ jokin rationaalilukujoukkoon \Q\ liittyvä predikaatti, esim.\ 
$P(x) = x<2$. Tällöin ehto '$P(x)$ tosi' määrittelee \Q:n osajoukon $A$, jota merkitään
$\,A\,=\,\{\,x \in \Q \mid P(x)\,\}$. Universaalijoukoksi on tässä luonnollista ajatella
$U=\Q$, jolloin $A$:n komplementti on
\[
\complement(A)\ =\ \{\,x \in \Q \mid \neg P(x)\,\}\ 
                =\ \{\,x \in \Q \mid x \ge 2\,\}. \loppu 
\] \end{Exa}

Loogisten yhdistelyjen $A \tai B$, $A \wedge B$, $A \impl B$ ja $A \ekv B$ joukko-opilliset
vastineet ovat $A \cup B$, $A \cap B$, $A \subset B$ ja $A=B$, kuten nähdään määritelmistä:
\begin{align*}
x\in A \cup B   &\qekv x \in A\ \tai\ x \in B  \\
x\in A \cap B   &\qekv x \in A\,\ \wedge\,\ x \in B  \\
A \subset B     &\qekv x \in A\ \impl\ x \in B \\
A = B           &\qekv x \in A\ \ekv\ x \in B
\end{align*}
Joukko $A \cup B$ on nimeltään $A$:n ja $B$:n 
\index{unioni (yhdiste)} \index{yhdiste (unioni)}%
\kor{yhdiste} eli \kor{unioni} (engl.\ union)
ja $A \cap B$ on $A$:n ja $B$:n 
\index{leikkaus (joukkojen)}%
\kor{leikkaus} (engl.\ intersection). Jos kahdella joukolla
$A,B$ ei ole yhteisiä alkioita, eli $x \in A\,\wedge\,x \in B$ on epätosi jokaisella $x$, niin 
tämä voidaan ilmaista lyhyesti merkinnällä $A \cap B = \emptyset$. Sanotaan tällöin, että $A$ ja
$B$ ovat \kor{erillisiä} eli 
\index{pistevieraat joukot}%
\kor{pistevieraita} (engl. disjoint).
\begin{Exa} Esimerkin \ref{näkymätön kvanttori} päättely tulkittiin propositioksi ajattelemalla
kvanttorimerkintä $\forall x\in\Q$ lisätyksi. Vielä luontevampi on joukko-opillinen tulkinta:
$\ \{\,x\in\Q \mid 2x^2+x<1\,\}\ =\ \{\,x\in\Q \mid -1<x<\tfrac{1}{2}\,\}$. \loppu
\end{Exa}
\index{de Morganin lait}
\begin{Exa} Todista joukko-opin de Morganin lait (vrt.\ Esimerkki\,\ref{de Morgan logiikassa})
\begin{align*}
\complement(A \cup B) \quad &= \quad \complement(A) \cap \complement(B), \\
\complement(A \cap B) \quad &= \quad \complement(A) \cup \complement(B).
\end{align*}
\end{Exa}
\ratk \ Päättelyketjussa
\begin{align*}
x \in \complement(A \cup B) \quad &\ekv \quad x \not\in (A \cup B) \\
                                  &\ekv \quad \neg\,(\,x \in A\,\tai\,x \in B\,) \\
                                  &\ekv \quad x \not\in A\,\wedge\,x\not\in B \qquad\qquad 
                                         \text{[\,Esim.\,\ref{de Morgan logiikassa}\,a)\,]} \\
                                  &\ekv \quad x\in\complement(A)\,\wedge\,x\in\complement(B) \\
                                  &\ekv \quad x\in\complement(A)\cap\complement(B)
\end{align*}
voidaan edetä molempiin suuntiin, joten joukoilla $\complement(A \cup B)$ ja 
$\complement(A) \cap \complement(B)$ on samat alkiot, ja ensimmäinen väittämä on siis 
todistettu. Toisen väittämän todistus saadaan tästä vaihdoilla $\cup \ext \cap$, 
$\cap \ext\cup$, $\tai \ext \wedge$ ja $\wedge \ext \tai$. \loppu 

\subsection{Ekvivalenssirelaatio}
\index{ekvivalenssirelaatio|vahv}%

\index{relaatio}%
\kor{Relaatio} on joukko-opillinen 'suhteen' käsite. Jos $R$ on joukossa $A$ määritelty relaatio
ja $x,y \in A$, niin merkintä $x\,R\,y$ luetaan '$x$ on relaatiossa $R$ $y$:n kanssa', tai
sujuvammin vain '$x$ [relaation nimi] $y$'. Relaatioista ovat jo tuttuja käytännössä tärkeimmät, 
eli järjestys- ja samastusrelaatio.
\index{samastus '$=$'!d@yleinen}%
Samastusrelaation on aina oltava \kor{ekvivalenssirelaatio}, jonka aksioomat ovat seuraavat:
\begin{itemize}
\item[(E1)] \ $x\,R\,x\ \ \forall x$
\item[(E2)] \ $x\,R\,y\ \impl\ y\,R\,x$
\item[(E3)] \ $x\,R\,y\ \wedge\ y\,R\,z\ \impl\ x\,R\,z$
\end{itemize}
\index{symmetrisyys!a@relaation} \index{refleksiivisyys (relaation)}
\index{transitiivisuus (relaation)}%
Vaaditut ominaisuudet ovat nimeltään \kor{refleksiivisyys}, \kor{symmetrisyys} ja 
\kor{transitiivisuus}. Esimerkiksi relaatio '$\le$' on refleksiivinen ja transitiivinen muttei 
symmetrinen, ja relaatio '$\neq$' on ainoastaan symmetrinen. Samastusrelaatio 'sama kuin' sen 
sijaan on mitä ilmeisimmin ekvivalenssirelaatio. Esimerkiksi kun kirjoitetaan $x=y=z$ 
(tarkoittaen: \ $x=y$ ja $y=z$), niin on ilmeistä, että $x=z$. Tässä on siis kyse 
transitiivisuudesta.\footnote[2]{Samastusrelaation yleiset aksioomat ovat samat kuin 
ekvivalenssirelaation, vrt.\ alaviite edellisessä luvussa.}
\begin{Exa} Jos $A =$ \{suomalaiset\}, niin seuraavat $A$:ssa määritellyt relaatiot ovat
ekvivalenssirelaatioita:
\begin{align*}
x \sim y         \quad &\ekv \quad \text{$x$ ja $y$ ovat syntyneet samana vuonna}, \\
x\ \heartsuit\ y \quad &\ekv \quad \text{$y=x\,\ $ tai $\,\ y$ on $x$:n puoliso}. \loppu
\end{align*}
\end{Exa}
Joukossa $A$ määritellyllä ekvivalenssirelaatiolla on se ominaisuus, että se jakaa $A$:n
\index{ekvivalenssiluokka}%
\kor{ekvivalenssiluokkiin}. Nämä ovat $A$:n osajoukkoja, joiden sisältämät alkiot ovat kaikki
relaatiossa keskenään. Refleksiivisyysominaisuuden (E1) vuoksi jokainen $A$:n alkio kuuluu 
ainakin yhteen ekvivalenssiluokkaan (mahdollisesti yksinään). Toisaalta 
transitiivisuusominaisuudesta (E3) seuraa, että kaksi ekvivalenssiluokkaa ovat joko täysin samat
tai ne ovat pistevieraita --- siis jokainen $A$:n alkio kuuluu täsmälleen yhteen
ekvivalenssiluokkaan. Jos kyseessä on samastusrelaatio, niin kunkin ekvivalenssiluokan 
sisältämät alkiot 'luokitellaan samoiksi' eli samastetaan keskenään. Tällöin voidaan puhua myös
\index{samastusluokka}%
\kor{samastusluokista}. 
\jatko \begin{Exa} (jatko) Relaation '$\sim$' määräämiä ekvivalenssiluokkia sanotaan 
ikäluokiksi. Relaatio '$\heartsuit$' jakaa $A$:n ekvivalenssiluokkiin joissa on joko yksi alkio
(alaikäiset, sinkut, ym.) tai kaksi alkiota (rekisteröidyt parit). \loppu 
\end{Exa}

\Harj
\begin{enumerate}

\item
Olkoon $A$ propositio 'Sataa', $B$ propositio 'Menen lenkille' ja $C$ propositio 'Minulla on
aikaa'. Kirjoita näiden ja loogisten operaattoreiden avulla mahdollisimman pelkistetysti: \newline
a) \ 'Väsyttää, on pimeää ja TV:stä tulee BB, joten en mene lenkille.' \newline
b) \ 'Menen lenkille, satoi tai paistoi!' \newline
c) \ 'Sataa, joten en mene lenkille.' \newline
d) \ 'Sataa enkä mene lenkille.' \newline
e) \ 'Menen lenkille vain, jos minulla on aikaa.' \newline
f) \ 'Minulla on aikaa eikä sadakaan, mutta en mene lenkille.' \newline
g) \ 'Jos ei sada ja minulla on aikaa, niin menen lenkille.' \newline
h) \ 'Minulla on aikaa vain, jos sateen vuoksi en mene lenkille.'

\item Olkoon $A=$ 'Tänään sataa', $B=$ 'Huomenna on pouta' ja $C=$ 'Tänään on pouta'. a) Jos 
tänään alkanut sade jatkuu huomiseen, niin mitkä ovat propositioiden $\,A \impl B\,$ ja 
$\,B \impl A\,$ totuusarvot\,? b) Riippuuko proposition
$\neg(\neg A \wedge B) \impl \neg C$ totuusarvo siitä, mikä päivä on 'tänään'\,?

\item \label{H-I-3: modus tollens} \index{modus tollens}
Osoita joko totuustaulun avulla tai muuten päättelemällä seuraava propositio 
(nk.\ \kor{modus tollens}) identtisesti todeksi:
$\ [\,(\neg P \impl Q) \wedge \neg Q\,] \impl P$. \newline
Miten tulokseen vedotaan epäsuorassa todistuksessa\,?

\item \label{H-I-3: tautologioita}
Näytä totuustaulun avulla tai muuten päättelemällä identtisesti todeksi: 
\vspace{1mm}\newline
a) \ $\neg\,(A \wedge \neg B)\ \ekv\ \neg A \tai (A \wedge B) \qquad$
b) \ $(A \impl B)\ \ekv\ (\neg B \impl \neg A)$ \newline
c) \ $[\,(A \impl B)\,\wedge\,(B \impl C)\,]\ \impl\ (A \impl C)$ \newline
d) \ $[\,(A \impl C)\wedge(B \impl C)\,]\ \impl\ [\,(A \tai B) \impl C\,]$ \newline
e) \ $[\,(A \wedge B) \impl C\,]\ \ekv\ [\,A \impl (B \impl C)\,]$ \vspace{1mm}\newline
Vertaile totuustaulun avulla: \vspace{1mm}\newline
f) \ $(A \wedge B) \tai C\,\ \text{ja}\,\ A \wedge (B \tai C) \quad\ $  
g) \ $A \not\impl B\,\ \text{ja}\,\ A\impl\neg B$ \vspace{1mm}\newline
Tutki totuustaulun avulla, voiko seuraavat propositiot ilmaista jollakin yksinkertaisemmalla
(ekvivalentilla) tavalla: \vspace{1mm}\newline
h) \ $(A \impl B) \wedge B \quad\ $ 
i) \ $(A \impl B) \impl A \quad\ $ 
j) \ $[\,(A \impl B) \wedge B\,]\,\impl\,A$

\item
Jos seuraavat predikaatit $P(x)$ tulkitaan väittäminä $P(x)\ \forall x\in\Q$, niin mitkä 
väittämistä ovat tosia ja mitkä epätosia\,? \vspace{1mm}\newline
a) \ $2x>3\,\ \impl\,\ x>4/3 \qquad\qquad\qquad\quad$
b) \ $x<-2\,\ \ekv\,\ x^2>4$ \newline
c) \ $3x<4\,\ \impl\,\ x<1\ \tai\ 2x<3 \qquad\quad$
d) \ $x>1\,\ \not\ekv\,\ x>0\ \wedge\ x>1$ \newline
e) \ $x=4/3\ \wedge\ 3x^2<x+4\,\ \impl\,\ x^2+2x+1<0$ \newline
f) \ $2x^2+x\,\le\,1\,\le\,3x-2x^2\,\ \ekv\,\ x=4/3$ \newline

\item
Olkoot $x$ ja $y$ rationaalilukuja. Mitkä seuraavista propositioista ovat tosia\,? \newline
a) \ $\forall x \exists y\ (x \cdot y = 0) \qquad$ 
b) \ $\forall x \exists y\ (x \cdot y = 1)$ \newline
c) \ $\exists y \forall x\ (x \cdot y = 0) \qquad$ 
d) \ $\exists y \forall x\ (x \cdot y = 1)$ \newline
e) \ $\exists y \forall x\ (x \cdot y = x)$ \qquad 
f) \ $\exists y \forall x\ (x \cdot y = y)$ \newline
g) \ $\forall x \exists y\ (y<x) \qquad\quad\,$
h) \ $\exists x \forall y\ (y \ge x)$ \newline
Miten tilanne muuttuu, jos $x$ ja $y$ ovat positiivisia rationaalilukuja\,?

\item
Olkoon $x\in\Q$. Muodosta seuraavien kahden proposition negaatiot auki purettuina. Mitkä näin
syntyvistä neljästä propositiosta ovat tosia? \vspace{1mm}\newline
a) \ $\forall (\eps>0) \exists (x \neq 1) (\abs{x-1}<\eps) \quad$
b) \ $\exists (x \neq 1) \forall (\eps>0) (\abs{x-1}<\eps)$

\item \label{H-I-3: sqrt 2} 
Näytä epäsuoralla todistustavalla: \ a) Ei ole olemassa pienintä positiivista 
rationaalilukua. \ b) $\,x^2 \neq 2\ \forall x\in\Q$. \ c) $\,x^2 \neq 3\ \forall x\in\Q$.

\item
Universaalijoukko $U$ olkoon kymmenen ensimmäisen luonnollisen luvun muodostama joukko 
$\,\{1,2,3,4,5,6,7,8,9,10\}$. Olkoon $A=\{2,5,7,8,10\}$ ja $B=\{2,4,6,8,10\}$. Määritä 
$A \cup B$, $A \cap B$, $\complement(A)$, $\complement(A \cup B)$ ja
$\complement(A) \cap \complement(B)$. 

\item
Todista seuraavat joukko-opilliset väittämät: \newline
a) \ $A \subset B\ \ekv\ A \cup B = B\ \ekv\ A \cap B = A$ \newline
b) \ $A \subset B\ \wedge\ A \subset C\ \ekv\ A \subset B \cap C$ \newline
c) \ $A \subset C\ \wedge\ B \subset C\ \ekv\ A \cup B \subset C$ \newline
d) \ $A \subset B\ \ekv\ A \cap C \subset B \cap C\ \forall C$ \newline
e) \ $\,\emptyset \subset A\ \forall A$

\item Muotoile ja todista Tehtävän \ref{H-I-3: tautologioita} väittämien b)--d)
joukko-opilliset vastineet.

\item
Olkoon $\,A=\{(p,q) \mid p,q\in\Z\}$ (kokonaislukuparien joukko) ja 
$Q=\{(p,q) \in A \mid q \neq 0\}$. Määritellään $Q$:ssa relaatio 
$(p_1,q_1) \sim (p_2,q_2)\ \ekv\ p_1q_2=p_2q_1\,$. Näytä, että kyseessä on $Q$:n 
ekvivalenssirelaatio. Miten tulos liittyy rationaalilukuihin?

\item
Mitä ekvivalenssirelaation ominaisuuksia on seuraavilla, annetuissa joukoissa $A$ määritellyillä
relaatioilla\,? \newline
a) \ $A=\N:\,\ x R y\ \ekv\ x+y$ on parillinen. \newline
b) \ $A=\N:\,\ x R y\ \ekv\ x+y$ on pariton. \newline
c) \ $A=\Q:\,\ x R y\ \ekv\ \abs{x-y} \le 10^{-100}$. \newline 
d) \ $A=\{\,\text{suomalaiset}\,\}:\,\ x R y\ \ekv x$ ja $y$ ovat toisilleen sukua suoraan 
ylenevässä tai alenevassa polvessa.

\item (*) \index{alkuluku}
Luonnollinen luku $k\in\N$ on luvun $n\in\N$ \kor{tekijä}, jos $n=k \cdot m$ jollakin $m\in\N$.
Luku $n$ on \kor{alkuluku}, jos $n$:llä ei ole muita tekijöitä kuin $k=1$ ja $k=n$. Näytä, että
alkuluvuilla ei ole loppua, ts.\ ei ole olemassa suurinta alkulukua.

\end{enumerate}  % Logiikka ja joukko-oppi

\section{Jonon käsite} \label{jono}
\alku

\index{jono}%
\kor{Jono} (engl.\ sequence, ruots.\ följd) on olio muotoa
\[
\{\,a_1, a_2, a_3,\,\ldots\,\}.
\]
Joukosta jonon erottaa ennen muuta j\pain{är}j\pain{est}y\pain{s}: Jonon jäsenet, joita 
\index{termi (jonon)}%
kutsutaan jonon \kor{termeiksi} (vrt. joukon alkio) on 'pantu jonoon' eli järjestykseen. 
Matemaattisessa jonossa järjestys tarkoittaa vastaavuutta luonnollisten lukujen joukon ja jonon
välillä:
\begin{align*}
&1 \map a_1 \\
&2 \map a_2 \\
&\vdots
\end{align*}
Tässä '$\map$' tarkoittaa jälleen 'liittämistä': Jokaiseen $n \in \N$ liitetään yksikäsitteinen 
jonon termi ($n$:s termi) $a_n$, jolloin sanotaan, että $n$ on ko.\ termin järjestysnumero eli 
\index{indeksi!b@jonomerkinnän}%
\kor{indeksi}. Indeksijoukko on siis koko $\N$, mikä tarkoittaa, että jono on 
p\pain{äätt}y\pain{mätön}\footnote[2]{'Päättyvästä jonosta' muotoa
\[
(\,a_1, a_2,\,\ldots, a_n\,), \quad n \in \N
\]
käytetään tässä tekstissä yleisnimitystä (äärellinen) \kor{järjestetty joukko} (tai '$n$ alkion
järjestetty joukko'), tapauksissa $n=2,3$ nimiä \kor{pari} ja \kor{kolmikko}. Jonomerkinnästä
poiketen käytetään tässä yhteydessä pääsääntöisesti kaarisulkeita. --- Kirjallisuudessa 
kaarisulkeilla merkitään joskus myös päättymättömiä jonoja, samoin merkintää $<a_n>$ näkee 
jonoista käytettävän. \index{jzy@järjestetty joukko|av} \index{pari|av} \index{kolmikko|av}}.

Matemaattisen jonoon, niinkuin joukkoonkin, voi periaatteessa sijoittaa mitä tahansa. Jatkossa
tarkastelun kohteena ovat ennen muuta
\index{lukujono}%
\kor{lukujonot}, joille käytetään lyhennysmerkintöjä
\begin{align*}
&\{\,a_n,\ n = 1,2,\,\ldots\,\}, \\
&\{a_n\}_{n=1}^{\infty}, \\
&\{a_n\}.
\end{align*}
Jos jonon termit määräytyvät tunnettuna, indeksistä riippuvana lausekkeena, voidaan ko.\
lauseke kirjoittaa $a_n$:n paikalle.
\begin{Exa}
\begin{align*}
%&\{\,1,2,3,4,\ \ldots\} = \{n\}_{n=1}^{\infty}, \{k\}_{k=1}^{\infty} \\
&\{\,1,-1,1,-1,\ \ldots\ \} = \{\,(-1)^{n+1},\,\ n=1,2,\ \ldots\ \} \\
&\left\{\,\frac{1}{9}, \frac{1}{16}, \frac{1}{25},\ \ldots\ \right\} 
                        = \left\{\,\frac{1}{(k+2)^2}\,,\ \ k = 1,2,\ \ldots \ \right\} \\
&\{\,1,1,2,6,24,120,720,5040,40320,\ \ldots\ \} = \{\,n!\,\}_{n=0}^{\infty} \loppu
\end{align*}
\end{Exa}
Viimeinen esimerkki on myös esimerkki yleisemmästä jonosta muotoa
\[
\{\,a_m, a_{m+1}, a_{m+2},\ \ldots\ \} = \{a_n\}_{n=m}^{\infty}, \quad m \in \Z.
\]
Indeksin vaihdolla tämä voidaan palauttaa normaalimuotoon:
\[
\{a_n\}_{n=m}^{\infty} = \{a_{m+k-1}\}_{k=1}^{\infty}.
\]
\begin{Exa} \label{rekursio} \index{lukujono!a@palautuva (rekursiivinen)}
\index{palautuva lukujono} \index{rekursiivinen lukujono}
Lukujono
\[
a_0 = \tfrac{1}{2}, \quad a_n = 1 - a_{n-1}^2, \quad n = 1,2, \ldots
\]
on esimerkki \kor{palautuvasta} eli \kor{rekursiivisesta} lukujonosta, joka 'määrittelee 
itsensä'. Tässä $a_1 = 3/4,\ a_2 = 7/16,\ a_3 = 207/256,\ \ldots$ \loppu 
\end{Exa} 

\subsection{Sarja}
\index{sarja|vahv} \index{lukujono!b@sarja|vahv}%

Jos $\{\,a_k,\ k = 1,2,\ \ldots\ \}$ on lukujono ja $\{s_n\}$ toinen lukujono, joka
määritellään
\[
s_n = a_1 + \ldots + a_n = \sum_{k=1}^{n} a_k\,,
\]
niin sanotaan että $\{s_n\}$ on \kor{sarja} (engl. series). Sarjan tavanomainen lyhennetty 
merkintätapa on
\[
\sum_{k=1}^{\infty} a_k \quad \Bigl(\ = \ \ \{\ \sum_{k=1}^{n} a_k\,,\ \ n = 1,2,\ \ldots\ \}\ 
\Bigr).
\]
\index{termi (sarjan)} \index{osasumma (sarjan)}%
Lukuja $a_k$ sanotaan \kor{sarjan termeiksi} ja lukuja $s_n$ \kor{sarjan osasummiksi}. Sarja
tulkitaan siis lukujonoksi, joka muodostuu sarjan osasummista.\footnote[2]{Sarjan tulkinta tässä
tekstissä hieman 'oikoo' sarjan formaalia määritelmää, joka kuuluu: Sarja on jonojen $\seq{a_k}$
(sarjan termit) ja $\seq{s_n}$ (osasummat) muodostama j\pain{ono}p\pain{ari}.}
\begin{Exa}
\[
\sum_{k=0}^{\infty} (-1)^k\ =\ \{\ \sum_{k=0}^{n} (-1)^k\,,\ \ n 
                            = 0,1,\ \ldots\ \}\ = \ \{\,1,0,1,0,\ \ldots\ \} \quad \loppu
\] 
\end{Exa}
'Kaikkien sarjojen äiti' on (perusmuotoinen)
\index{geometrinen sarja} \index{sarja!a@geometrinen}%
\kor{geometrinen sarja}, jonka termit ovat $a_k = q^k,\ k = 0,1,\ \ldots$, ts.\ sarja on muotoa 
$\sum_{k=0}^\infty q^k = \{1,1+q,1+q+q^2,\ \ldots\,\}$\footnote[3]{Mukavuussyistä sovittakoon,
että geometrisessa sarjassa $\sum_{k=0}^\infty q^k$ ensimmäinen termi on $q^0 = 1$ myös kun 
$q=0$.}. Osasummille saadaan tässä tapauksessa laskukaava
(ks.\ Harj.teht.\,\ref{kunta}:\,\ref{H-I-2: kuntakaavat}\,c)
\[
\boxed{\ s_n = \sum_{k=0}^n q^k\ =\ \dfrac{q^{n+1}-1}{q-1}, \quad q \neq 1.\ }
\]

\subsection{Induktio}
\index{induktio(periaate)|vahv}%

Luonnollisiin lukuihin ja jonon käsitteeseen liittyy läheisesti myös matemaattinen 
todistusperiaate nimeltä \kor{induktio}. Olkoon $P(n),\ n \in \N$, predikaatti. Tällöin 
proposition
\[
\mathcal{P}: \quad P(n)\ \ \forall n \in \N
\]
voi tulkita viittaavan väittämäjonoon $\{\,P(1),P(2), \ldots\,\}$. Induktion ideana on muuntaa
tämä jono palautuvaksi, jolloin jono 'todistaa itsensä' samalla tavoin kuin palautuva lukujono
'laskee itsensä' (vrt. Esimerkki \ref{rekursio} edellä). Idea realisoidaan tarkastelemalla
kahta propositiota:
\begin{align*}
&\mathcal{P}_1\ : \quad P(1), \\
&\mathcal{P}_2\ : \quad P(n) \impl P(n+1)\ \ \forall n \in \N.
\end{align*}
Jos $\mathcal{P}_1$ ja $\mathcal{P}_2$ ovat molemmat tosia, niin $\mathcal{P}_1$ käynnistää 
propositioon $\mathcal{P}_2$ perustuvan 'todistusautomaatin', joten implikaatio
\[
\mathcal{P}_1 \wedge \mathcal{P}_2 \quad \impl \quad \mathcal{P},
\]
joka lausuu nk.\ \kor{induktioperiaatteen}, tuntuu ilmeisen 
todelta.\footnote[2]{Induktioperiaate seuraa Peanon aksioomasta (P5) asettamalla
$S = \{n \in \N \mid P(n)\ \text{tosi}\,\}$, ks.\ alaviite Luvussa \ref{ratluvut}.} 
Induktioperiaatteen mukaan siis väittämän $\mathcal{P}$ todistamiseksi riittää osoittaa, että
$\mathcal{P}_1$ ja $\mathcal{P}_2$ ovat molemmat tosia. Proposition $\mathcal{P}_2$ 
\index{induktioaskel, -oletus}%
toteennäyttämistä sanotaan \kor{induktioaskeleeksi} ja todistuksen ko.\ osan lähtöoletusta 
'$n \in \N$ ja $P(n)$ tosi' \kor{induktio-oletukseksi}.

Näytetään induktion avulla oikeaksi seuraava tuttu tulos:
\begin{Prop} \label{binomikaava} (\vahv{Binomikaava}) \index{binomikaava, -kerroin|emph}
Rationaalilukujen kunnassa tai sen laajennuksissa pätee
\[
(x+y)^n\ =\ \sum_{k=0}^n \binom{n}{k}\, x^{n-k}y^k, \quad n \in \N,
\]
missä \kor{binomikertoimet} määritellään
\[
\binom{n}{k}\ =\ \dfrac{n!}{k!(n-k)!}.
\]
\end{Prop}
\tod Väittämä on muotoa $P(n)\ \forall n \in \N$. Tässä $P(1)$ on ilmeisen tosi, joten riittää
suorittaa induktioaskel. Oletetaan siis, että $P(n)$ on tosi, ts.\ että binomikaava pätee 
tietyllä (mutta mielivaltaisesti valitulla) $n$:n arvolla. Tällöin on kunnan aksioomien ja 
oletuksen perusteella
\begin{align*}
(x+y)^{n+1}\ =\ (x+y)(x+y)^n\ 
            &=\ (x+y) \sum_{k=0}^n \binom{n}{k}\, x^{n-k}y^k \\
            &=\ \sum_{k=0}^n \binom{n}{k}\, x^{n-k+1}y^k\ 
                                +\ \sum_{k=0}^n \binom{n}{k}\, x^{n-k}y^{k+1} \\
            &=\ \sum_{k=0}^n \binom{n}{k}\, x^{n+1-k}y^k\ 
                                +\ \sum_{l=1}^{n+1} \binom{n}{l-1}\, x^{n+1-l}y^l.
\end{align*}
Tässä on jälkimmäisessä summassa tehty indeksin vaihto $l=k+1$. Kun jälleen kirjoitetaan $l=k$,
voidaan molemmat summat yhdistää tulokseksi
\[
(x+y)^{n+1}\ =\ \sum_{k=0}^{n+1} c_k\ x^{n+1-k}y^k,
\]
missä
\[
c_0 = \binom{n}{0} = 1 = \binom{n+1}{0}, \quad \quad 
c_{n+1} = \binom{n}{n} = 1 = \binom{n+1}{n+1},
\]
ja indeksin arvoilla $k = 1 \ldots n$
\begin{align*}
c_k\ =\ \binom{n}{k} + \binom{n}{k-1}\ 
    &=\ \dfrac{n!}{k!(n-k)!} + \dfrac{n!}{(k-1)!(n-k+1)!} \\
    &=\ \dfrac{n!}{(k-1)!(n-k)!}\ \Bigl(\dfrac{1}{k} + \dfrac{1}{n-k+1} \Bigr) \\
    &=\ \dfrac{n!}{(k-1)!(n-k)!}\ \cdot\ \dfrac{n+1}{k(n-k+1)} \\
    &=\ \dfrac{(n+1)!}{k!(n-k+1)!}\ =\ \binom{n+1}{k}.
\end{align*} 
Näin ollen $P(n+1)$ on tosi ja induktioaskel siis suoritettu. Todistuksessa tarvittiin yleisten
kuntaoperaatioiden lisäksi luonnollisten lukujen välisiä laskuoperaatioita
(mukaan lukien jakolasku), joten todistus on pätevä rationaalilukujen kunnassa tai sen
laajennuksissa. \loppu

Kuten Proposition \ref{binomikaava} todistuksesta käy ilmi, binomikertoimet ovat laskettavissa
palautuvasti kaavasta
\[
\binom{n+1}{0}\ =\ \binom{n+1}{n+1}\ =\ 1, \quad \quad 
\binom{n+1}{k}\ =\ \binom{n}{k} + \binom{n}{k-1}\,, \quad k = 1 \ldots n.
\]
Kaava havainnollistuu 
\index{Pascalin kolmio}%
\kor{Pascalin kolmiossa}, jonka avulla kertoimet voidaan määrittää 
kätevästi pienillä $n$:n arvoilla:
\begin{align}
                                        &\ 1                                     \notag \\
                                   1\,\ &\ \quad\ 1                              \tag{$n=1$} \\
                             1\ \ \quad &\ 2\,\qquad 1                           \tag{$n=2$} \\
                         1\ \qquad 3\,\ &\ \quad\ 3\,\qquad 1                    \tag{$n=3$} \\
                 1 \qquad\ 4\ \ \quad\, &\ 6\,\qquad 4\,\qquad 1                 \tag{$n=4$} \\
                  1\ \qquad 5\qquad 10\ &\ \quad 10 \qquad 5\,\qquad 1           \tag{$n=5$} \\
       1\ \qquad 6\,\qquad 15\ \ \quad  &20\ \ \quad 15 \qquad 6\,\qquad 1       \tag{$n=6$} \\
    1\,\qquad 7\,\qquad 21\ \ \quad 35\ &\,\ \quad 35 \ \ \quad 21 \qquad 7 \qquad 1  
                                                                                 \tag{$n=7$} \\
1 \qquad 8 \qquad 28 \qquad 56\ \ \quad &70\ \ \quad 56\ \ \quad 28 \qquad 8\,\qquad 1
                                                                                 \tag{$n=8$}
\end{align}
Pascalin kolmion avulla binomikertoimet määräytyvät pelkällä luonnollisten lukujen 
yhteenlaskulla.

Toisena induktion sovelluksena todistetaan usein käytetty epäyhtälö.
\begin{Prop} (\vahv{Bernoullin epäyhtälö}) \label{Bernoulli} \index{Bernoullin epäyhtälö|emph}
Rationaalilukujen kunnassa tai sen järjestetyissä laajennuksissa pätee
\[
\boxed{\kehys\quad (1+x)^n\ >\ 1+nx \quad 
     \text{kun}\ \  n \in \N,\ n \ge 2\ \ \text{ja}\ \ x \ge -1,\ x \neq 0. \quad}
\] 
\end{Prop}
\tod Jos $x>0$, niin epäyhtälö seuraa helpoiten binomikaavasta:
\[
(1+x)^n\ =\ 1 + \binom{n}{1}\,x\ +\ [\ldots]\ =\ 1 + nx + [\ldots],
\]
missä $[\ldots] > 0$, kun $x>0$ ja $n \ge 2$. Yleispätevämpi todistus perustuu induktioon,
eikä sekään ole pitkä: Ensinnäkin
\[
(1+x)^2\ =\ 1+2x+x^2\ > 1+2x, \quad \text{kun}\ x \neq 0,
\]
joten väittämä on tosi, kun $n=2$. Toisaalta jos väite on tosi millä tahansa $n$:n arvolla 
(induktio-oletus, $n \ge 2$), niin
\begin{align*}
(1+x)^{n+1}\ &=\   (1+x)(1+x)^n      \\
             &\ge\ (1+x)(1+nx)       \\ 
             &=\   1 + (n+1)x + nx^2 \\
             &>\   1 + (n+1)x,
\end{align*}
kun $x \ge -1 \impl 1+x \ge 0$ ja $x \neq 0 \impl x^2 > 0$. Induktioperiaatteen mukaan väittämä
on näin todistettu. \loppu

\subsection{Numeroituvuus, mahtavuus}
\index{numeroituvuus (joukon)|vahv} \index{mahtavuus (joukon)|vahv}%

Sanotaan, että joukot $A$ ja $B$ ovat \kor{yhtä mahtavat} (engl. of the same cardinality), jos 
joukkojen välillä on olemassa \pain{kääntäen} y\pain{ksikäsitteinen} \pain{vastaavuus} siten, 
että jokaista $A$:n alkiota vastaa yksikäsitteinen $B$:n alkio ja kääntäen. Vastaavuutta 
merkitään jatkossa kaksoisnuolella '$\vast$':
\[
A \vast B \qquad \text{(kääntäen yksikäsitteinen vastaavuus)}. \\
\]
Jos erityisesti $\,A \vast \{\,1,2,\ \ldots\ n\,\}\,$ jollakin $n\in\N$, niin sanotaan, että 
$A$ on \kor{äärellinen} (engl.\ finite) joukko. Tällöin $A$:ssa on täsmälleen $n$ alkiota, ja 
$A$ ja $B$ ovat yhtä mahtavat täsmälleen, kun myös $B$:ssä on $n$ alkiota. Äärellisen joukon
tapauksessa vastaavuutta $A \vast \{\,1,2,\ \ldots\ n \}$ sanotaan $A$:n alkioiden 
\kor{numeroinniksi} (indeksoinniksi). Numerointi tekee $A$:sta j\pain{är}j\pain{estet}y\pain{n}
joukon. 
\begin{Exa}
\begin{align*}
A &= \{\,1,2,3,1000\,\} \\
B &= \{\,\text{vuohi V, hevonen H, opiskelija O, professori P}\,\} \\
C &= \{\,\text{linnunradan atomit}\,\}
\end{align*}
Kaikki kolme ovat äärellisiä joukkoja. Joukot $A$ ja $B$ ovat yhtä mahtavat. \loppu 
\end{Exa}
\index{zyzy@äärellinen, ääretön joukko}%
Jos joukko ei ole äärellinen, niin se on \kor{ääretön} (engl. infinite).
\begin{Exa} \label{numer} Joukot
\begin{align*}
&A = \{\,1,4,9,16,25,36,49,\ \ldots\ \}, \\
&B = \{\,1,100,10000,1000000,\ \ldots\ \}
\end{align*}
ovat molemmat äärettömiä. Ne ovat myös yhtä mahtavat, sillä vastaavuuksista
\begin{align*}
n \in \N \quad &\vast \quad n^2 \in A \\
n \in \N \quad &\vast \quad 10^{2n-2} \in B
\end{align*}
nähdään, että $A \vast \N \vast B$. \loppu \end{Exa}
Esimerkin perusteella pienemmältä tai suuremmalta 'tuntumiseen' ei voi luottaa vertailtaessa 
äärettömien joukkojen mahtavuuksia. Sanonnat kuten 'yhtä monta' tai 'sama määrä' on myös
selkeintä rajata äärellisten joukkojen vertailuun.

Esimerkin \ref{numer} joukkojen $A,B$ alkiot voidaan numeroida esitettyjen vastaavuuksien 
perusteella. Yleisesti sanotaan, että joukko $A$ on \kor{numeroituva} 
(tai 'numeroituvasti ääretön', engl.\ (d)enumerable tai countably infinite), jos $A \vast \N$.
Numeroituvuus siis tarkoittaa, että joukon alkiot voidaan järjestää jonoksi.
\begin{Exa} Kokonaislukujoukon $\Z = \{\,0,\pm 1, \pm 2, \ldots\, \}$ eräs jonomuoto on
\[
\{\,0,1,-1,2,-2,\ \ldots\ \}\ \ =\ \ \{\,a_n,\ n = 1,2,\ \ldots\,\},
\]
joten \Z\ on numeroituva. Esimerkiksi luku $-777$ on jonon $1555$:s termi. \loppu 
\end{Exa}
\begin{Exa} Rationaalilukujoukko $\Q$ on esitettävissä muodossa
\[
\Q\ =\ A_1\ \cup\ A_2\ \cup\ \cdots\ = \ \bigcup_{m=1}^{\infty} A_m,
\]
missä
\begin{align*}
A_1\,\ &=\ \{\,0\,\}, \\
A_m\   &=\ \bigl\{\,x = p/q \in \Q\,\mid\ \abs{p} + \abs{q} 
                      = m\ \ja\ x \not\in A_k\ \ \text{kun}\ \ k<m\,\bigr\}, \ \ m = 2,3, \ldots
\end{align*}
Osajoukot $A_m$ ovat määritelmän mukaisesti keskenään pistevieraita 
(\mbox{$A_m \cap A_k = \emptyset$} kun $k<m$), joten jokainen $x \in \Q$ on enintään yhden 
osajoukon alkio. Toisaalta määritelmästä seuraa myös, että jos $x \in \Q$, niin $x \in A_m$, 
missä $m\in\N$ on pienin luku, jolle pätee $m = \abs{p} + \abs{q}$ ja $x = p/q$. Näin ollen 
jokainen $x\in\Q$ on täsmälleen yhden osajoukon $A_m$ alkio. Joukot $A_m$ ovat myös äärellisiä,
joten jokainen niistä voidaan numeroida erikseen. Jos nyt $A_m$:n alkioiden lukumäärä $= N_m$,
niin mielivaltaiselle $x\in\Q$ saadaan yksikäsitteinen järjestysnumero $n$ säännöllä
\[
n = \begin{cases} \begin{aligned}
    1,\  \qquad\qquad\qquad\qquad\quad &\text{jos $x=0$}, \\
    N_1 + \cdots + N_{m-1} + k, \quad  &\text{jos $x$ on $A_m$:n $k$:s alkio},\ \ m \ge 2.
                  \end{aligned}
    \end{cases}
\]
Kaikki rationaaliluvut on näin järjestetty jonoksi, ja voidaan siis todeta, että $\Q$ on 
numeroituva. \loppu 
\end{Exa}

Jos joukko on ääretön mutta ei numeroituva, sanotaan että se on 
\index{ylinumeroituva joukko}%
\kor{ylinumeroituva} (engl.\ uncountable). Näinkin 'mahtavia' joukkoja --- myös lukujoukkoja ---
on olemassa, kuten tullaan näkemään.

\Harj
\begin{enumerate}

\item
Näytä induktiolla todeksi summakaavat ($n\in\N$) \vspace{1mm}\newline
a) \ $\sum_{k=1}^n k = \frac{1}{2} n(n+1) \qquad\ $
b) \ $\sum_{k=1}^n k^2 = \frac{1}{6} n(n+1)(2n+1)$ \vspace{1mm}\newline 
c) \ $\sum_{k=1}^n k^3 = \frac {1}{4} n^2(n+1)^2 \qquad$
d) \ $\sum_{k=1}^n k^4 = \frac{1}{30} n(n+1)(2n+1)(3n^2+3n-1)$

\item
Näytä induktiolla seuraavat summakaavat päteviksi jokaisella $n\in\N$\,:
\begin{align*}
&\text{a)} \quad 1 \cdot 3 + 3 \cdot 5 + \ldots + (2n-1)(2n+1) = \frac{n}{3}(4n^2+6n-1) \\
&\text{b)} \quad \sum_{k=1}^n k^2 2^k = (n^2-2n+3)2^{n+1}-6 \\
&\text{c)} \quad \sum_{k=1}^n kq^{k-1} 
                          = \frac{1-(n+1)q^n+nq^{n+1}}{(1-q)^2}\,, \quad q\in\Q,\ q \neq 1
\end{align*}

\item \label{jono-H4}
Määritellään palautuvat lukujonot
\begin{align*}
&\text{a)} \quad a_0=1, \quad a_{n+1} = qa_n + 1, \quad n=0,1\,\ldots \\
&\text{b)} \quad a_0=2, \quad a_{n+1} = \frac{a_n}{2} + \frac{1}{a_n}\,, \quad n=0,1\,\ldots \\
&\text{c)} \quad a_1=3,\ a_2=6,\ a_{n+1} = \frac{na_n+a_{n-1}+3}{n}\,, \quad n=2,3\,\ldots \\
&\text{d)} \quad a_0\in\Q,\ a_0 \not\in \{-1/n \mid n\in\N\}, \quad 
                            a_{n+1} = \frac{a_n}{1+a_n}\,, \quad n=0,1\,\ldots
\end{align*}
Näytä induktiolla, että \ a) $\seq{a_n}$ on geometrinen sarja, \ 
b) $1 \le a_n \le 2\ \forall n$, \newline
c) $a_n < 4n\ \forall n$, \ d) $a_n=a_0/(1+na_0)\ \forall n$.

\item
Laske seuraavien summalausekkeiden arvot ($n\in\N$)\,:
\[
\text{a)}\ \ \sum_{k=0}^n \binom{n}{k} \qquad 
\text{b)}\ \ \sum_{k=0}^n (-1)^k \binom{n}{k} \qquad
\text{c)}\ \ \sum_{k=0}^n 2^k\binom{n}{k} \qquad
\text{d)}\ \ \sum_{k=0}^n 3^{-k}\binom{n}{k}
\]

\item (*)
Olkoon $x\in\Q,\ x>0$. Näytä, että jos jollakin $a\in\Q$ ja $n\in\N,\ n \ge 2$ pätee 
$1 < x^n \le a$, niin $1 < x < 1+(a-1)/n$. \ \kor{Vihje}: Kirjoita $x=1+y$.

\item (*) a) Näytä, että jos $x\in\Q,\ x> 0$ ja $n\in\N,\ n\ge 3,$ niin
$(1+x)^n > 1+nx+\frac 12 n(n-1)x^2.$ \ b) Millaisia vielä parempia arvioita saadaan, jos 
$n\ge k,\ k=4,5, \ldots$\,? \ c) Näytä, että on olemassa $n\in\N$ siten, että pätee
\[
\frac{(1+10^{-100})^n}{n^{100}} > 10^{100}.
\]

\item (*)
Näytä, että jos joukot $A_1,\,A_2,\,A_3, \ldots$ ovat keskenään pistevieraita ja numeroituvia,
niin myös joukko
\[
A = A_1 \cup A_2 \cup \ldots = \bigcup_{n=1}^\infty A_n
\]
on numeroituva. Totea väitteen pätevyys suoraan (määrittämällä $A$), kun 
$A_n=\{\,x\in\Q \mid n-1 \le \abs{x} < n\,\}$.

\end{enumerate}    % Jonon käsite
\section{Äärettömät desimaaliluvut} \label{desimaaliluvut}
\alku
\index{zyzy@ääretön desimaaliluku|vahv}

Matematiikan arkipäivässä hyvin yleisesti kohdattava lukujono on
\kor{ääretön desimaaliluku}\footnote[2]{Vaihtoehtoinen nimitys on \kor{päättymätön}
desimaaliluku. \index{pzyzy@päättymätön desimaaliluku|av}} (engl.\ infinite decimal). Tämä on
ilmiasultaan merkkijono muotoa
\[
e d_{-m} d_{-m+1} \ldots d_0 . d_1 d_2 d_3 \ldots\ =\ x_0.d_1 d_2 d_3 \ldots\,,
\]
missä
\index{desimaali, -luku, -piste} \index{kokonaisluvut!a@kokonaislukuosa (desim.-luvun)}%
\begin{itemize}
\item[-] $\quad$ '$e$' on \kor{etumerkki}, joko $e = +$ tai $e = -\ $ (\ 'tyhjä' = +\,)
\item[-] $\quad$ '$.$' on \kor{desimaalipiste} 
                           (tai 'desimaalipilkku', $d_0$:n ja $d_1$:n välissä)
\item[-] $\quad \ \ m\in\N\cup\{0\}$
\item[-] $\quad \ \ x_0\ =\ e d_{-m} d_{-m+1} \ldots d_0\,\in\,\Z\ =$\ \ desimaaliluvun 
                    \kor{kokonaislukuosa}
\item[-] $\quad \ \ d_n,\,\ n=1,2,\ldots\ =$\ \ desimaaliluvun \kor{desimaalit}
                    ($d_n\,\in\,\{0, \ldots, 9\}$)
\end{itemize}
\begin{Exa}
\begin{align*}
141&.42135623730950488 \ldots \qquad (\,e = +,\ m = 2,\ x_0 = 141\,) \\
- 0&.00314159265358979 \ldots \qquad (\,e = -,\ m = 0,\ x_0 =0\,) \qquad \loppu
\end{align*}
\end{Exa}

Ääretön desimaaliluku on siis pääosin numerojono, jossa numeroiden lisäksi on erillinen 
etumerkki ja desimaalipiste, jonka sijainti numerojonossa määrittää kokonaislukuosan $x_0$. 
Tällaisen merkkijonon \pain{tulkitaan} \pain{tarkoittavan} rationaalilukujonoa
\[
\{\,x_0, x_1, \ldots\,\} = \{\,x_n\,\}_{n=0}^{\infty}\,,
\]
missä $x_0$ on kokonaislukuosa ja
\[
x_n\ =\ x_0 \pm \bigl(\, d_1 \cdot 10^{-1} + \cdots + d_n \cdot 10^{-n}\,\bigr)
     =\ x_0 \pm \sum_{k=0}^n d_k \cdot 10^{-k}, \quad n\in\N,
\]
missä summalausekkeen etumerkki valitaan $e$:n mukaan. Tulkinnan mukaisesti ääretön
desimaaliluku on siis itse asiassa \pain{sar}j\pain{a}, jonka ensimmäinen termi on $x_0$ ja
muut $\pm d_k \cdot 10^{-k}$, $k=1,2,\ldots\,$ Sarjan osasummat ovat indeksistä $n=1$
lähtien nk.\ 
\index{zyzy@äärellinen desimaaliluku}%
\kor{äärellisiä} (tai päättyviä, ks.\ alaviite) \kor{desimaalilukuja}, jotka jatkossa merkitään
\[
x_n\ =\ x_0. d_1 \ldots d_n \quad (n\in\N).
\]
Tällä tarkoitetaan siis lukua $\,\pm(|x_0| + \sum_{k=0}^n d_k \cdot 10^{-k})$, missä etumerkki
on erikseen asetettu (sama kuin $x_0$:ssa, jos $x_0 \neq 0$). Kirjoittamalla luku muotoon
\[
x_n\ =\ \pm 10^{-n} \left(|x_0| \cdot 10^n + \sum_{k=0}^n d_k \cdot 10^{n-k}\right)
\]
nähdään, että $x_n=p/10^{n}$, missä $p = e d_{-m} \ldots d_0 d_1 \ldots d_n \in \Z$ on 
kokonaisluku. Toisaalta jos lähtökohtana on luku $x=p/10^n,\ p\in\Z$, niin esittämällä $p$ 
kymmenjärjestelmässä nähdään, että $x$ on saatettavissa muotoon $x = x_0.d_1 \ldots d_n$.
Siis ko.\ muotoa olevat äärelliset desimaaliluvut muodostavat täsmälleen $\Q$:n osajoukon 
\[
\Q_n = \{\,p/10^n \mid p \in \Z\,\}.
\]
Kun tämän mukaisesti asetetaan $\Q_0=\Z$, niin pätee
\[
\Z = \Q_0 \subset \Q_1 \subset \Q_2 \subset \ldots \subset \Q.
\]
Lukujoukossa $\Q_n$ luvut ovat tasavälein siten, että peräkkäisten lukujen $x,y$ etäisyys on
$|x-y|=10^{-n}$. Kuvassa on lukujoukkoja $\Q_0,\Q_1,\Q_2$ 
havainnollistettu geometrisesti.\footnote[2]{Lukujen 'geometrisointia' tarkastellaan myöhemmin 
Luvussa \ref{geomluvut}. Tässä vaiheessa pidetään geometrisen havainnollistamisen perusideoita 
tunnettuina.}
\begin{figure}[H]
\setlength{\unitlength}{1cm}
\begin{center}
\begin{picture}(12,3)(0,0)
\path(0,0)(12,0)
\path(0,1)(12,1)
\path(0,2)(12,2)
\multiput(1,0)(5,0){3}{\line(0,1){0.2}}
\multiput(1,1)(5,0){3}{\line(0,1){0.2}}
\multiput(0,1)(0.5,0){25}{\line(0,1){0.1}}
\multiput(1,2)(5,0){3}{\line(0,1){0.2}}
\multiput(0,2)(0.5,0){25}{\line(0,1){0.1}}
\multiput(0,2)(0.05,0){241}{\line(0,1){0.05}}
\put(12.5,0){$\Q_0$}
\put(12.5,1){$\Q_1$}
\put(12.5,2){$\Q_2$}
\put(5.9,-0.5){$0$} \put(0.6,-0.5){$-1$} \put(10.9,-0.5){$1$}
\end{picture}
\end{center}
\end{figure}
Jatkossa äärettömiin desimaalilukuihin viitataan joko tavanomaisilla lukusymboleilla
$x,y,z$ jne., tai vaihtoehtoisesti symboleilla $\x,\y,\z$ jne., kun halutaan tehdä ero 
'oikeisiin' lukuihin, erityisesti rationaalilukuihin. Äärettömien desimaalilukujen joukkoa
merkitään symbolilla $\DD\,$:
\[
\DD = \{\text{äärettömät desimaaliluvut}\}.
\]
Yksittäisen desimaaliluvun tapauksessa pidetään merkkijonoesitystä ko.\ desimaaliluvun 
'nimenä', jolloin merkinnässä
\[
\x\ =\ x_0. d_1 d_2 \ldots\ =\ \{\,x_n\,\}_{n=0}^{\infty} 
\]
ensimmäinen '$=$' nimeää ja toinen tulkitsee (antaa merkityssisällön).
\jatko  \begin{Exa} (jatko) Tehdyin merkintäsopimuksin voidaan kirjoittaa
\begin{align*}
x\ &=\ 141.421356 \ldots \\ 
   &=\ \{\,141,141.4,141.42, \ldots\,\} \\
   &=\ \{\,141,\ \dfrac{1414}{10}\,,\ \dfrac{14142}{100}\,,\ \ldots\,\} \\
   &=\ \{\,x_0,\,x_1,\,x_2,\,\ldots\,\} \\[3mm]
y\ &=\ -0.00314159 \ldots \\
   &=\ \{\,0,\,0.0,\,0.00,\,-0.003,\,-0.0031,\, \ldots\,\} \\
   &=\ \{\,0,\ 0,\ 0,\ -\dfrac{3}{1000}\,,\ -\dfrac{31}{10000}\,,\ \ldots\,\} \\
   &=\ \{\,y_0,\,y_1,\,y_2,\,\ldots\,\} \quad \loppu
\end{align*}
\end{Exa}

Jokainen ääretön desimaaliluku $x\in\DD$ on siis merkitykseltään lukujono 
$\seq{x_n}_{n=0}^\infty\,$, missä $x_n\in\Q_n\ \forall n$. Määritelmänsä mukaisesti lukujono 
$\seq{x_n}$ on määrättävissä asettamalla ensin $x_0\in\Z$ ja sitten $x_1,x_2, \ldots\,$ 
palautuvasti muodossa
\[
x_n\ =\ \begin{cases}
         x_{n-1} + d_n \cdot 10^{-n}, \quad \text{jos $e\ =\ +$}\,, \\
         x_{n-1} - d_n \cdot 10^{-n}, \quad \text{jos $e\ =\,\ $ --}\ ,
        \end{cases}
\]
missä $e$ on $x$:n etumerkki ja $d_n\in\{0, \ldots ,9\}\ \forall n\in\N$. Tämän mukaan on
joko $x_n \ge x_{n-1}\ \forall n\in\N\,$ tai $\,x_n \le x_{n-1}\ \forall n\in\N$, riippuen 
etumerkistä $e$. Tällaiset 'yksitoikkoiset' jonot ovat huomattavan tärkeä lukujonojen
erikoistapaus, joten asetetaan niitä varten
\begin{Def} \label{monotoninen jono} \index{lukujono!c@monotoninen|emph}
\index{monotoninen!a@lukujono|emph}
\index{aidosti kasvava, vähenevä, monotoninen!a@lukujono|emph}
Lukujono $\{\,a_n,\ n=m,m+1, \ldots\,\}$ on
\begin{itemize}
\item[-] \kor{kasvava},\ \ \ \ jos $\ a_{n+1} \ge a_n\ \ \forall n \ge m$
\item[-] \kor{vähenevä},\ \    jos $\ a_{n+1} \le a_n\ \ \forall n \ge m$
\end{itemize}
Lukujono on \kor{aidosti} kasvava, jos $\,a_{n+1} > a_n\ \forall n \ge m$ ja aidosti vähenevä,
jos $\,a_{n+1} < a_n\ \forall n \ge m$. Jos lukujono on (aidosti) kasvava tai vähenevä, se on 
(aidosti) \kor{monotoninen}.
\end{Def}
Ääretön desimaaliluku on siis lukujonona monotoninen: kasvava etumerkin ollessa $e=+$ ja 
vähenevä, kun etumerkki on $e=-$. Seuraavan tuloksen mukaan tämä kasvu tai väheneminen kuitenkin
'hiipuu' nopeasti $n$:n kasvaessa.
\begin{Prop} \label{desim} Jos $x = \{\,x_n\,\}_{n=0}^{\infty} \in \DD$, niin jokaisella
$k\in\N\cup\{0\}$ ja jokaisella $n\in\N$, $n>k$ pätee $\abs{x_n-x_k}<10^{-k}$, tarkemmin
\[
\abs{x_n - x_k}\ \le\ 10^{-k}-10^{-n}.
\]
\end{Prop}
\tod Kun $n > k \ge 0$, on
\begin{align*}
\abs{x_n - x_k}\ =\ \sum_{j=k+1}^n d_j \cdot 10^{-j}\
                 &\le\ \sum_{j=k+1}^n 9 \cdot 10^{-j} \\
                 &=\ 9 \cdot 10^{-k-1} \sum_{i=0}^{n-k-1} \Bigl(\frac{1}{10}\Bigr)^i \\
                 &=\ 10^{-k} \cdot \dfrac{9}{10} \cdot 
                                    \dfrac{1 - (\frac{1}{10})^{n-k}}{1-\frac{1}{10}} \\
                 &=\ 10^{-k} - 10^{-k}\left(\frac{1}{10}\right)^{n-k} \\[2mm] 
                 &=\ 10^{-k}-10^{-n}. \loppu
\end{align*}
Kun Proposition \ref{desim} tuloksen ohella huomioidaan lukujonon $\seq{x_n}$ monotonisuus, niin
nähdään, että jokaiselle äärettömälle desimaaliluvulle $\x=\seq{x_n}$ pätee
\[
0 \le k < n \qimpl \begin{cases} 
                   \,x_k \le x_n < x_k + 10^{-k}, \quad &\text{jos etumerkki $e\ =\ +$}\,, \\
                   \,x_k - 10^{-k} < x_n \le x_k, \quad &\text{jos etumerkki $e\ =\,\ $ --}\,.
                   \end{cases}
\]
Tämän mukaan termien $x_n$ liikkumavara supistuu nopeasti indeksin $n$ kasvaessa.

\subsection{Jakokulma-algoritmi}
\index{jakokulma-algoritmi|vahv}%

Edellä on todettu, että äärettömät desimaaliluvut ovat lukujonoina $\seq{x_n}$ sekä monotonisia
että 'hiipuvia'. Voi myös käydä niin, että jono $\{x_n\}$ ei ainoastaan 'hiivu', vaan 
\kor{lähestyy rationaalilukua} $x \in \Q$ siinä mielessä, että $\abs{x_n-x}$ tulee yhä 
pienemmäksi $n$:n kasvaessa. Milloin tällainen 'lähestyminen' on yleensä mahdollista, 
selvitetään myöhemmissä luvuissa. Tässä vaiheessa asetetaan kysymys toisin päin: Jos on annettu
$x\in\Q$, niin miten löydetään ääretön desimaaliluku $\x=\seq{x_n}$ siten, että $\abs{x_n-x}$ 
tulee yhä pienemmäksi $n$:n kasvaessa\,? Vastauksen antaa tunnettu 
\kor{jakokulma-algoritmi}\footnote[2]{\kor{Algoritmi} on täsmällinen ja toteutuskelpoinen
(ohjelmointikelpoinen) toimintaohje jonkin laskennallisen päämäärän saavuttamiseksi. Jos 
päämääränä on lukujono $\{\,a_n,\ n=1,2, \ldots\,\}$, on algoritmia seuraamalla saatava 
äärellisellä määrällä laskutoimituksia vastaus kysymykseen $a_n=\,$?, olipa indeksi $n \in \N$
mikä hyvänsä annettu luku. \index{algoritmi|av}}. Algoritmissa konstruoidaan desimaalivastine
luvulle $|x|$ ja varustetaan tämä $x$:n mukaisella etumerkillä, joten voidaan rajoittua
tapaukseen $x \ge 0$. Valitaan tällöin $x_n\in\Q_n,\ n=0,1,2,\ldots$ siten, että jokaisella
$n$ toteutuu ehto
\begin{equation} \label{jakoehto 1}
x_n \le x < x_n + 10^{-n}.
\end{equation}
Jos $x$ ei ole äärellinen desimaaliluku, ts.\ $x\not\in\Q_n\ \forall n$, niin ehdon
\eqref{jakoehto 1} mukaisesti on $x_n < x < x_n+10^{-n}\,\ \forall n$. Jos $x\in\Q_k$
jollakin $k$, niin $x_n\in\Q_n$ myös kun $n>k$, jolloin ehdosta \eqref{jakoehto 1}
seuraa, että $x_n=x$ jokaisella $n \ge k$. Molemmissa tapauksissa $x_n+10^{-n}$ on $x$:ää
lähinnä oleva, $x$:ää aidosti suurempi luku joukossa $\Q_n$. On ilmeistä, että näin
konstruoitu lukujono $\seq{x_n}$ on yksikäsitteinen, toteuttaa
\begin{equation}  \label{jakoehto 2}
\abs{x_n - x}\ <\ 10^{-n}, \quad n=0,1, \ldots
\end{equation}
ja määräytyy yksikäsitteisesti myös tästä ehdosta $x$:n etumerkistä riippumatta.

Jos $x \ge 0$, niin ehdon \eqref{jakoehto 1} toteuttavan lukujonon konstruoiminen 
jakokulma-algoritmilla käy seuraavasti: Olkoon $x = r/s$, missä $r\in\N\cup\{0\}$ ja 
$s \in \N$. Tällöin voidaan ensinnäkin kirjoittaa
\[
\dfrac{r}{s} = x_0 + \dfrac{r_1}{s}\,,
\]
missä $x_0 \in \N \cup \{0\}$ ja $r_1 \in \{\,0, \ldots, s-1\,\}$. Luvut $x_0$ ja $r_1$ ovat
ilmeisen yksikäsitteiset. Koska $0\,\le\,r_1/s\,<\,1\ \impl\ 0\,\le\,10\,r_1/s\,<\,10$, on 
edelleen löydettävissä yksikäsitteiset $d_1 \in \{\,0, \ldots, 9\,\}$ ja 
$r_2 \in \{\,0, \ldots, s-1\,\}$ siten, että
\[
\dfrac{10\,r_1}{s} = d_1 + \dfrac{r_2}{s}\,.
\]
Jatkamalla näin saadaan yleiseksi algoritmiksi
\[
\boxed{\quad \dfrac{10\,r_k}{s} = d_k + \dfrac{r_{k+1}}{s}, \quad 
                     k = 1,2, \ldots \quad \quad \text{(jakokulma-algoritmi)} \quad}
\]
Tässä $d_k \in \{\,0, \ldots, 9\,\}$ ja luvut $r_{k+1} \in \{\,0, \ldots, s-1\,\}$ ovat 
\index{jakojäännös}%
nk.\ \kor{jakojäännöksiä}. Jos algoritmi katkaistaan indeksiin $k=n$, on luvulle $x$ saatu
esitysmuoto
\begin{align*}
x\ =\ x_0 + \dfrac{r_1}{s}\ 
  &=\ x_0 + d_1 \cdot 10^{-1} + 10^{-1} \cdot \dfrac{r_2}{s} \\ 
  &\ \ \vdots \\
  &=\ (\,x_0 + d_1 \cdot 10^{-1} + \ldots + d_n \cdot 10^{-n}\,) 
                                          + 10^{-n} \cdot \dfrac{r_{n+1}}{s} \\
  &=\ x_n + 10^{-n} \cdot \dfrac{r_{n+1}}{s}\ .
\end{align*}
Tässä on $\ 0\,\le\,10^{-n}\,r_{n+1}/s\,<\,10^{-n}$, joten ehto \eqref{jakoehto 1} toteutuu
algoritmin määrittämälle lukujonolle $\{\,x_0,\,x_0.d_1,\,x_0.d_1d_2,\,\ldots\,\} = \seq{x_n}$.

Jos $\x\in\DD$ on jakokulma-algoritmin tuottama desimaalilukuvastine rationaaliluvulle $x$, on
tapana muitta mutkitta kirjoittaa $x = \x$, eli \pain{samastaa} $x$ ja $\x$. Jatkossa
noudatetaan tätä sopimusta. Tämän mukaisesti siis jokaiselle rationaaliluvulle luotu 'kopio'
äärettömien desimaalilukujen joukossa, eli pätee $\DD \supset \Q$.
\begin{Exa} Tehdyn sopimuksen mukaan
\begin{align*}
50/3    &= 16.66666666666666666666666666666666666666 \ldots\,\in \DD, \\
311/125 &= 2.488000000000000000000000000000000000000 \ldots\,\in \DD, \\
245/17  &= 14.41176470588235294117647058823529411764 \ldots\,\in \DD. \quad\loppu 
\end{align*} \end{Exa}
Esimerkissä luku $311/125$ on äärellinen desimaaliluku. Tällaisessa tapauksessa jakojäännös
$r_k = 0$ jollakin $k$ (tässä kun $k=4$), jolloin algoritmi antaa $r_n = d_n = 0$ kun $n \ge k$.
\index{jaksollinen desimaaliluku}%
Muissakin esimerkeissä algoritmin antama desimaaliluku on \kor{jaksollinen}, eli jollakin
$k,m \in \N$ muotoa
\[
\x\ =\ x_0.d_1 d_2 \ldots d_k d_{k+1} \ldots d_{k+m} d_{k+1} \ldots d_{k+m} \ldots,
\]
jolloin jaksollisuus alkaa indeksistä $k+1$ ja jakson pituus $=m$. Jakokulma-algoritmi antaa 
itse asiassa \pain{aina} jaksollisen tuloksen, sillä koska jakojäännöksellä $r_n$ on vain $s-1$
erilaista $0$:sta poikkeavaa arvoa, niin jollakin $n=k+1\in \{\,1, \ldots,s\,\}$ ja 
$m \in \{\,1, \ldots, s-1\,\}$ on oltava $r_{n+m} = r_{n}$. Tällöin algoritmi antaa 
$r_{n+m} = r_n$ ja $d_{n+m} = d_n$ myös kun $n > k+1$, jolloin tulos on ym.\ jaksollista muotoa.
Jaksollisuus alkaa siis viimeistään indeksistä $n=s$ ja jakson pituus on enintään $m=s-1$. 

Jatkossa merkitään jaksollisten desimaalilukujen joukkoa symbolilla $\DD_p\,$:
\[
\DD_p = \{\text{jaksolliset desimaaliluvut}\}.
\]
\index{jaksoton desimaaliluku}%
Jos $x\in\DD$ ja $x\not\in\DD_p$, niin sanotaan, että $x$ on \kor{jaksoton}.

\subsection{Äärettömät binaariluvut}
\index{zyzy@ääretön desimaaliluku!48@ääretön binaariluku|vahv}%

\kor{Äärettömillä binaariluvuilla} tarkoitetaan merkki/lukujonoja muotoa
\[
x\ =\ eb_m b_{m+1} \ldots b_0 . b_1 b_2 \ldots\ =\ \{x_n\}_{n=0}^{\infty},
\]
\index{bitti}%
missä $b_n$:t ovat binaarijärjestelmän numeroita eli \kor{bittejä}, $b_n \in \{\,0,1\,\}$, ja
\[
x_n\ =\ x_0 \pm \sum_{k=1}^n b_k \cdot 2^{-k},
\]
missä $x_0 = e b_m b_{m+1} \ldots b_0$ on kokonaislukuosa (binaarimuodossa) ja summalausekkeen
etumerkki on sama kuin $x$:ssä. Jos lukujärjestelmän kantalukuna on $8$ tai $16$, puhutaan 
\index{oktaaliluku} \index{heksadesimaaliluku}%
vastaavasti (äärettömistä) \kor{oktaali}- ja \kor{heksadesimaaliluvuista}. Tietokoneiden ja 
laskinten käyttäminä lukujärjestelminä nämä mainitut ovat tavallisia. --- Kaikki mitä edellä
on sanottu äärettömistä desimaaliluvuista, on helposti muunnetavissa mainittuja (ja muitakin)
lukujärjestelmiä koskevaksi.

\subsection{Desimaaliluvun skaalaus}
\index{skaalaus (desimaaliluvun)|vahv}

Toistaiseksi äärettömille desimaaliluvuille ei määritellä muita laskuoperaatioita kuin
\kor{merkin vaihto} ja \kor{skaalaus} = kertominen luvuilla $10^m,\ m\in\Z$. Skaalaus
luvulla $10^m$ tarkoittaa yksinkertaisesti desimaalipisteen paikan siirtämistä joko $m$
desimaalia eteenpäin ($m>0$) tai $|m|$ numeroa taaksepäin ($m<0$). Jälkimmäisessä tapauksessa
lisätään kokonaislukuosaan tarvittaessa riittävä määrä 'etunollia'. Luvulla $10^0=1$ 
skaalattaessa jää desimaaliluku ennalleen. Skaalaukselle pätevät vaihdanta- ja
liitäntälait 
\[
10^m \cdot (10^n \cdot x) \,=\, 10^{m+n} \cdot x 
                          \,=\, 10^n \cdot (10^m \cdot x), \quad m,n\in\Z.
\]
Kun merkin vaihto tulkitaan samaksi kuin kertominen luvulla $-1$ (kuten kunta-algebrassa),
niin merkin vaihto ja skaalaus yhdistämällä tulee määritellyksi skaalaus luvuilla
$\pm 10^m,\ m\in\Z$. 
\jatko\jatko \begin{Exa} (jatko)
\begin{align*}
 10^{-4} \cdot 141.421356 \ldots\  &=\ 0.0141421356 \ldots \\
 -10^4 \cdot (-0.00314159 \ldots)\ &=\ 31.4159 \ldots
\end{align*}
\end{Exa} \seur

\subsection{Liukuluvut}
\index{liukuluku|vahv}%

\kor{Liukuluvuiksi} (engl.\ floating point number, ruots.\ flyttal) sanotaan 
kymmenjärjestelmässä äärellisiä desimaalilukuja muotoa
\[
\pm 10^{\beta} \cdot 0 . d_1 \ldots d_n,
\]
missä $\beta\in\Z$ ja $n\in\N$ on \pain{kiinteä}. Numeerisessa laskennassa liukulukuja
käytetään äärettömien desimaalilukujen \kor{approksimaatioina}, eli likimääräisinä
vastineina. Approksimointi tapahtuu kirjoittamalla desimaaliluku ensin skaalaustekijän
avulla muotoon $x=\pm 10^\beta \cdot 0.d_1d_2..$, missä $\beta\in\Z$ ja $d_1 \neq 0$
(mahdollista aina kun $x \neq 0$) ja katkaisemalla näin saatu luku desimaaliin $d_n$.
Sanotaan, että näin saadut numerot $d_1 \ldots d_n$ ovat desimaaliluvun $n$ ensimmäistä 
\index{merkitsevät numerot}%
\kor{merkitsevää numeroa} (engl.\ significant digits), ja että ko.\ liukuluku
on desimaaliluvun
\index{katkaistu desimaaliluku}%
\kor{katkaistu} (engl.\ truncated) muoto. Katkaisun vaihtoehtona 
\index{pyzz@pyöristys (desimaaliluvun)}%
approksimaatiossa voidaan käyttää \kor{pyöristystä} (engl. rounding), yleensä lähimpään
samaa tyyppiä olevaan liukulukuun (nk.\ normaalipyöristys). Tieto desimaalipisteen paikasta
suhteessa ensimmäiseen merkitsevään numeroon sisältyy lukuun $\beta \in \Z$. Kun ääretön
desimaaliluku ilmoitetaan $n$ \kor{merkitsevän numeron tarkkuudella}, tarkoitetaan luvun
(katkaisutua tai pyöristettyä) liukulukuapproksimaatiota.
\begin{Exa} Yhdeksän ($n=9$) merkitsevän numeron tarkkuudella on
\begin{align*}
0.0001414213562373095   \ldots \quad \approx \quad\, 10^{-3} &\cdot 0.141421356 \\
-172/3\ = \ -57.333333333 \ldots \quad \approx \quad   -10^2 &\cdot 0.573333333 
\end{align*} 
Tässä katkaisu ja normaalipyöristys antavat saman tuloksen. \loppu
\end{Exa}

Tietokoneet ja laskimet operoivat numeerisissa laskuissa aina liukuluvuilla, esim.\ 
binaarijärjestelmään perustuvilla. Tällöin sekä merkitsevien numeroiden määrä että $\beta$:n
sisältämien numeroiden määrä on rajoitettu koneelle ominaisesti. Mahdollisia
\index{koneluku}%
\kor{konelukuja} (engl.\ machine numbers) on siten aina äärellinen määrä.
\begin{Exa} Koneluvut ovat muotoa $\ \pm 2^{\beta} \cdot 0 .\,b_1 \ldots b_{30}$, missä 
$\beta = \pm \beta_1 \ldots \beta_{10}$, $b_k, \beta_k \in \{\,0,1\,\}$. Yhden luvun 
tallettamiseen tarvitaan $42$ bittiä (numerot ja etumerkit), ja erilaisia lukuja on
\[
2^{30} \cdot (2^{11} - 1) + 1\ =\ 2197949513729\ \ \text{kpl.} \quad \quad \loppu
\]
\end{Exa}

Tietokoneen/laskimen suorittamia peruslaskuoperaatioita (yhteen-, vähennys-, kerto- ja 
jakolasku) koneluvuilla sanotaan \kor{liukulukuoperaatioiksi} (engl.\ floating point operation,
lyh.\ flop). Nämä poikkeavat hiukan tavanomaisista laskuoperaatioista, koska tulos on yleensä
(etenkin kerto- ja jakolaskuissa) katkaistava tai pyöristettävä koneluvuksi. Laskutoimitusten
\index{pyzz@pyöristys (desimaaliluvun)!pyöristysvirhe}%
likimääräisyydestä aiheutuvat \kor{pyöristysvirheet} (engl.\ roundoff errors) voivat vahvistua
tai kasautua peräkkäisissä laskutoimituksissa. Yksittäisessäkin laskutoimituksessa voi tapahtua
kohtalokas
\index{merkitsevät numerot!merkitsevien numeroiden kato}%
\kor{merkitsevien numeroiden kato}, kuten seuraava yksinkertainen esimerkki osoittaa.
\begin{Exa} Olkoon $a$ ja $b$ annettuja lukuja. Halutaan laskea
\[
x=\frac{1}{a}\left(1-\frac{1}{1+b}\right).
\]
Mikä arvo saadaan $x$:lle, jos $a=10^{-20},\ b=2 \cdot 10^{-20}$, ja koneluvut ovat muotoa 
$\pm 10^{\beta} \cdot 0 . d_1 \ldots d_{15}$ (kymmenjärjestelmä), missä $-99 \le \beta \le 99$\,?
\end{Exa}
\ratk Suoraan annettuun lausekkeeseen perustuva algoritmi on:
\[
y=1+b,\ \ z=1/y,\ \ u=1-z,\ \ x=u/a.
\]
Koska $y$:n tarkka arvo on $\,y = 1.00000000000000000002$,
niin kone sijoittaa muistipaikkaan '$y$' tämän luvun katkaistun (tai pyöristetyn)
liukulukumuodon $y = 10^1 \cdot 0.100000000000000 = 1$. Laskun muut vaiheet ovat tämän jälkeen
virheettömiä, ja lopputulokseksi saadaan $x=0$. \loppu

Esimerkissä vaarattoman näköinen pyöristys johtaa suureen virheeseen lopputuloksessa. Ongelma 
poistuu, jos laskut suoritetaan 
\index{kaksoistarkkuus}%
\kor{kaksoistarkkuudella} (engl.\ double precision), eli käyttämällä liukulukuesitystä, jossa
talletetaan $15$:n sijasta $30$ merkitsevää numeroa. Tällöin saadaan tulos $x=2$, joka on 
varsin lähellä tarkkaa arvoa $x=1.999999999999999999960000000000000000000799999\ldots$
Kaksoistarkkuuteen (tai useampikertaiseen tarkkuuteen) siirtyminen on yleinen menettely
pyöristysvirheiden uhatessa. Toinen vaihtoehto on algoritmin parempi suunnittelu.
\jatko \begin{Exa} (jatko) Laskettava lauseke yksinkertaistuu kunta-algebran keinoin muotoon
\[
x=\frac{b}{a(1+b)}\,.
\]
Tässä approksimaatio $1+b \approx 1$ ei aiheuta merkittävää virhettä. \loppu
\end{Exa}

\Harj
\begin{enumerate}

\item
a) Näytä, että jos luvuilla $p,q\in\N$ ei ole yhteisiä tekijöitä, niin $x=p/q\in\Q$ on 
äärellinen desimaaliluku täsmälleen kun on olemassa $\,m,n\in\N\cup\{0\}\,$ siten, että 
$q=2^m \cdot 5^n$. \ b) Ilmoita luku $x=(2/5)^{10}$ äärellisenä desimaalilukuna. \ 
c) Millaisia rationaalilukuja ovat äärellisten desimaalilukujen vastineet
lukujärjestelmässä, jonka kantaluku on $60$\,? \ d) Luvulla $1111111/5400$ on
$60$-järjestelmässä äärellistä desimaalilukua vastaava esitysmuoto. Määritä tämä
käyttämällä luvuille $n=0 \ldots 59$ symboleja $(n)$ $60$-järjestelmässä.

\item \label{H-I-5: monotonisten jonojen yhdistely}
Näytä, että jos lukujonot $\seq{a_n}$ ja $\seq{b_n}$ ($n\in\N$) ovat kasvavia, niin \
\mbox{a) myös} lukujono $\seq{a_n+b_n}$ on kasvava, \ b) jos lisäksi $a_n \ge 0\ \forall n$ ja
$b_n \ge 0\ \forall n$, niin lukujono $\seq{a_n b_n}$ on kasvava. \ c) Näytä esimerkillä,
että jos b)-kohdassa lisäehto $b_n \ge 0\ \forall n$ poistetaan, niin lukujono $\seq{a_n b_n}$
voi olla aidosti vähenevä.

\item
Desimaaliluvuista $x=\seq{x_n}\in\DD$ ja $y=\seq{y_n}\in\DD$ tiedetään, että
$x=23.4569 \ldots$ ja $y=23.4567 \ldots$ Kuinka suuri luku $x_n-y_n$ on tämän tiedon
perusteella vähintään ja kuinka suuri enintään, kun $n=10$\,?

\item
Muotoile ja todista Proposition \ref{desim} vastine äärettömille binaariluvuille.

\item
Laske jakokulma-algoritmin avulla seuraavien rationaalilukujen vastineet jaksollisina 
desimaalilukuina:
\ a) $1/11$, \ b) $31/13$, \ c) $19/17$.

\item
Jakokulma-algoritmin mukainen luvun $x\in\Q$ vastine äärettömänä desimaalilukuna olkoon
jaksollinen siten, että jakson pituus $=m$ ja jaksollisuus alkaa desimaalista $d_k$. Anna 
esimerkki luvusta $x=r/s,\ r,s\in\N$, jolle pätee \ 
a) $\,m=1,\ k=5$, \,\ b) $\,m=s-1,\ k=s$, \,\ c) $\,m=s-1,\ k=1$.

\item
Kymmenjärjestelmän luku $7/11$ samastuu $2$-kantaisessa lukujärjestelmässä äärettömään 
binaarilukuun ja $3$-kantaisessa järjestelmässä äärettömään triaarilukuun. Määritä nämä 
jakokulma-algoritmilla. \kor{Huomaa}, että jakojäännöstä ei algoritmissa kerrota kymmenellä vaan
kantaluvulla, siis $2$:lla tai $3$:lla. Jatka laskua, kunnes jaksollisuus ilmenee\,!

\item
Halutaan laskea $y=(x+a)^2-(x+b)^2$ kun $x=1/7,\ a=0.1457 \cdot 10^{-6}$ ja 
$b=0.6973 \cdot 10^{-6}$. \ a) Määritä $y$ neljän merkitsevän numeron tarkkuudella muokkaamalla
$y$:n lauseke ensin sellaiseen muotoon, että merkitsevien numeroiden katoa ei tapahdu. \ b) Jos
käytetään alkuperäistä lauseketta, niin millä tarkkuudella laskut on suoritettava, jotta
päästään mainittuun tarkkuuteen lopputuloksessa\,?

\item
Jos $x=7 \cdot 10^{-8}$, niin monenko merkitsevän numeron tarkkuudella on
\[
\text{a)} \quad \frac{1}{8+3x} \approx 0.125 \qquad \text{b)} \quad (1+x)^3 \approx 1+3x\ ?
\]

\item
Muokkaa seuraavat lausekkeet niin, että alttius merkitsevien numeroiden kadolle poistuu. Laske
sitten lausekkeiden arvot kymmenen merkitsevän numeron tarkkuudella.
\begin{align*}
&\text{a)} \quad 10^{20}\left(\frac{1}{1+x}-\frac{1}{1+2x}\right), \quad 
                                                 x = \frac{6}{7} \cdot 10^{-20} \\
&\text{b)} \quad 10^{40}\left(\frac{1}{1+x}-\frac{1}{1-2x}+3x\right), \quad 
                                                 x = \frac{2}{7} \cdot 10^{-20} \\
&\text{c)} \quad 10^{60}\left(\frac{1+x}{1-x}-\frac{1-x}{1+x}-4x\right), \quad
                                                 x = \frac{5}{7} \cdot 10^{-20} \\
&\text{d)} \quad 10^{80}\left(\frac{1+x}{1-x}+\frac{1-x}{1+x}+\frac{2}{1+2x^2}-4\right), \quad
                                                 x = \frac{3}{7} \cdot 10^{-20} \\
&\text{e)} \quad 10^{120}\left(\frac{1+x}{1-x}+\frac{1-x}{1+x}
                                              -\frac{2}{1-2x^2+2x^4}\right), \quad
                                                 x = \frac{4}{7} \cdot 10^{-20}
\end{align*}

\item (*) \index{zzb@\nim!Mummon nelilaskin} 
(Mummon nelilaskin) Nelilaskin pystyy näyttämään luvun, jossa on kahdeksan numeroa, etumerkki
ja mahdollinen desimaalipiste numeroiden välissä, esim.\ $-0.0000999$. Montako erilaista
rationaalilukua laskin kykenee näyttämään tarkasti?

\item (*) \label{H-I-4.8}
Olkoon $x=\seq{x_n}\in\DD$ ja $y=\seq{y_n}\in\DD$. Näytä, että jos $x_k < y_k$ jollakin
$k\in\N\cup\{0\}$, niin a) $x_n < y_n$ jokaisella $n \ge k$, \ b) $x_n \le y_n$ jokaisella $n$.

\item (*)
Ikuisesti laskeva tietokone tuottaa bittijonon $\{b_n\}_{n=1}^\infty$ seuraavalla algoritmilla:

1. Asetetaan $A=0$, $B=1$, $n=0$.\newline
2. Asetetaan 
\begin{align*}
&n\leftarrow n+1 \text{ (uusi arvo=vanha arvo}+1\text{)} \\
&C=\frac{1}{2}(A+B)\quad\text{ja} \\
&b_n=1\,\ \text{ja}\,\ A=C, \quad \text{jos}\ \ 3 \cdot C \le 1 \\
&b_n=0\,\ \text{ja}\,\ B=C, \quad \text{jos}\ \ 3 \cdot C > 1
\end{align*}
Ilmoitetaan $b_n$ ja palataan kohtaan 2.

a) Millaiset luvut $b_n,$ $n=1\ldots 4,$ kone ilmoittaa? \newline
b) Jos laskun lopputulos (iäisen odottelun jälkeen) tulkitaan äärettömänä binaarilukuna 
$\x=0.b_1b_2\ldots$, niin minkä rationaaliluvun $x$ binaariesitys on kyseessä\,? Perustele\,!

\end{enumerate}  % Äärettömät desimaaliluvut
\section{Lukujonon raja-arvo} \label{jonon raja-arvo}
\alku

Tässä ja seuraavissa luvuissa tarkastelun kohteena ovat j\pain{är}j\pain{estet}y\pain{n}
\pain{kunnan} lukujonot $\seq{a_n}$. Oletetaan siis, että $a_n\in\K\ \forall n$, missä 
$(\K,+,\cdot,<)$ on järjestetty kunta; tämä voi olla joko rationaalilukujen kunta tai 
jokin tämän kuntalaajennus, jossa myös järjestysrelaatio on määritelty. Esimerkiksi voi
olla $\K=\J$, vrt.\ Luku \ref{kunta}.

Lukujonon $\{a_n\}$ sanotaan
\index{lukujono!d@suppeneva}%
\kor{suppenevan} eli \kor{konvergoivan} (engl.\ converge)
kohti lukua $a$, jos '$n$:n kasvaessa $a_n$ tulee yhä lähemmäksi $a$:ta'. Täsmällisepi
määritelmä seuraa hieman tuonnempana; tässä vaiheessa riittäköön toteamus, että
suppeneminen tarkoittaa esimerkiksi seuraavaa:
\begin{align*}
\abs{a_n - a}\ &<\ 10^{-1} \quad \text{indeksistä $n=N_1$ alkaen}, \\
\abs{a_n - a}\ &<\ 10^{-2} \quad \text{indeksistä $n=N_2$ alkaen}, \\
               &\ \vdots \\
\abs{a_n - a}\ &<\ 10^{-k} \quad \text{indeksistä $n=N_k$ alkaen}, \\
               &\ \vdots
\end{align*}
Tässä jokainen $N_k$ \pain{on} \pain{äärellinen}, eli $N_k \in \N$ jokaisella $k \in \N$.
Suppenemista on havainnollistettu alla olevissa kuvissa graafisesti\footnote[2]{Lukujen 
graafisen (geometrisen) havainnollistamisen menetelmiä pidetään jälleen tuttuina. Asiaan 
palataan myöhemmin Luvussa II.}.

\begin{figure}[H]
\setlength{\unitlength}{1cm}
\begin{center}
\begin{picture}(12,5)(0,-1)
\path(0,1)(12,1)
\multiput(1,1)(10,0){2}{\line(0,1){0.15}}
\multiput(5.5,1)(1,0){2}{\line(0,1){0.15}}
\put(6,1){\line(0,1){0.15}}
\put(1,3){$\overbrace{\hspace{10cm}}^{\displaystyle{a_n, \ n>N_1}}$}
\put(5,1.5){$\overbrace{\hspace{1cm}}^{\displaystyle{a_n, \ n>N_2}}$}
\put(5.9,0.5){$a$}
\put(0.5,0.5){$a-10^{-1}$} \put(10.5,0.5){$a+10^{-1}$}
\put(4,-1){$a-10^{-2}$} \put(5,-0.5){\vector(1,3){0.5}}
\put(6.65,-1){$a+10^{-2}$} \put(7,-0.5){\vector(-1,3){0.5}}
\end{picture}
\end{center}
\end{figure}

Jos lukujono suppenee kohti $a$:ta, sanotaan lukua $a$ 

jonon \kor{raja-arvo}ksi (engl.\ limit). Myös termiä
\index{limes}%
\kor{limes} käytetään (lat.\ limes = raja). Merkintätapoja ovat
\[
\lim_{n \kohti \infty} a_n\ =\ a \qquad \text{tai} \qquad \lim_n a_n\ =\ a
\]
tai kuten jatkossa useammin:
\[
a_n \kohti a \quad (\text{kun}\ \ n \kohti \infty).
\]
Merkinnöissä '$\lim$' luetaan 'limes' ja symboli '\kohti' luetaan 'suppenee kohti',
'menee kohti' tai 'lähestyy'. 

\begin{figure}[H]
\setlength{\unitlength}{1cm}
\begin{center}
\begin{picture}(14,14)(-2,-0.5)
\put(0,0){\vector(1,0){12}} \put(11.8,-0.5){$n$}
\put(0,0){\vector(0,1){13.5}}
\put(0,2){\line(1,0){12}}
\Thicklines
\put(0,7){\line(1,0){12}}
\thinlines
\put(0,12){\line(1,0){12}}
\put(0,6.5){\line(1,0){12}}
\put(0,7.5){\line(1,0){12}}
\multiput(0.5,0)(0.5,0){23}{\line(0,1){0.1}}
\put(0.44,1){$\scriptstyle{\bullet}$}
\put(0.94,5){$\scriptstyle{\bullet}$}
\put(1.44,9){$\scriptstyle{\bullet}$}
\put(1.84,13){$\scriptstyle{\bullet}$}
\put(2.44,10){$\scriptstyle{\bullet}$}
\put(2.94,7){$\scriptstyle{\bullet}$}
\put(3.44,4){$\scriptstyle{\bullet}$}
\put(3.94,6){$\scriptstyle{\bullet}$}
\put(4.44,8){$\scriptstyle{\bullet}$}
\put(4.94,7){$\scriptstyle{\bullet}$}
\put(5.44,6){$\scriptstyle{\bullet}$}
\put(5.94,6.7){$\scriptstyle{\bullet}$}
\put(6.44,7.2){$\scriptstyle{\bullet}$}
\put(6.94,6.7){$\scriptstyle{\bullet}$}
\put(7.44,7.1){$\scriptstyle{\bullet}$}
\put(7.94,6.75){$\scriptstyle{\bullet}$}
\put(8.44,7.05){$\scriptstyle{\bullet}$}
\put(8.94,7.05){$\scriptstyle{\bullet}$}
\put(9.44,6.8){$\scriptstyle{\bullet}$}
\put(9.94,6.8){$\scriptstyle{\bullet}$}
\put(10.44,7){$\scriptstyle{\bullet}$}
\put(10.94,6.85){$\scriptstyle{\bullet}$}
\put(11.44,7){$\scriptstyle{\bullet}$}
\put(11.94,6.9){$\scriptstyle{\bullet}$}
\dashline{0.2}(2.5,12)(2.5,0)
\dashline{0.2}(6,7)(6,0)
\path(2.5,0)(2.5,-0.5) \put(2.5,-0.5){\vector(1,0){0.5}} \put(3.2,-0.6){$n>N_1$}
\path(6,0)(6,-0.5) \put(6,-0.5){\vector(1,0){0.5}} \put(6.7,-0.6){$n>N_2$}
\put(0.4,-0.5){$1$} \put(0.9,-0.5){$2$} \put(1.4,-0.5){$3$} 
\multiput(0,2)(0,10){2}{\line(-1,0){0.1}}
\multiput(0,6.5)(0,1){2}{\line(-1,0){0.1}}
\put(0,7){\line(-1,0){0.1}} \put(-0.5,6.9){$a$}
\put(-2,8.1){$a+10^{-2}$}
\put(-2,5.5){$a-10^{-2}$}
\put(-1.7,11.9){$a+10^{-1}$}
\put(-1.7,1.9){$a-10^{-1}$}
\put(-0.6,8){\vector(1,-1){0.5}}
\put(-0.6,6){\vector(1,1){0.5}}
\end{picture}
\end{center}
\end{figure}

Suppeneminen on lukujonolle varsin voimakas vaatimus, ja onkin helppo esittää esimerkkejä 
jonoista, jotka eivät suppene kohti mitään lukua. Tällainen on vaikkapa
\index{rajatta kasvava (lukujono)}%
\kor{rajatta kasvava} lukujono $\{a_n\} = \{n\}$. Tässä tapuksessa on tapana kirjoittaa
\[
a_n \kohti \infty\ \ \text{kun}\ \ n \kohti \infty \qquad 
                     \text{tai jopa:} \quad\lim_n a_n = \infty.
\]
Luvallisia luku- ja puhetapoja ovat '$a_n$ lähestyy ääretöntä' tai 'raja-arvo on ääretön'. 
Näiden sanontojen mukaisesti merkintä '$\lim_{n \kohti \infty}$' luetaan yleensä 
'limes $n$ lähestyy ääretöntä', tarkoittaen siis raja-arvoa, kun $n$ kasvaa rajatta.
Vastaavasti jos lukujono on
\index{rajatta vähenevä (lukujono)}%
\kor{rajatta vähenevä}, voidaan kirjoittaa 
$\,a_n \kohti - \infty\,$ tai $\,\lim_n a_n = - \infty$, ja lukea 'raja-arvo on miinus 
ääretön'. Näistä merkintä- ja puhetavoista huolimatta rajatta kasvava tai rajatta vähenevä
lukujono ei ole suppeneva.
\begin{Exa} \label{jonoja} Etsi mahdollinen raja-arvo rationaalilukujonolle
$\seq{a_n}_{n=1}^\infty$, kun \vspace{1mm}\newline
a) \ $a_n = n^2 \qquad\qquad\,$ 
b) \ $a_n = 100-n \qquad\qquad$ 
c) \ $a_n = (-1)^n$ \vspace{1mm}\newline
d) \ $a_n = (n+1)/n \quad$
e) \ $a_n = (10^{100}+ 2n)/(10^{100} + n)$
\end{Exa}
\ratk a) Kyseessä on rajatta kasvava jono: $a_n \kohti \infty$. Raja-arvoa ei siis ole.

b) Tämä jono on rajatta vähenevä: $a_n \kohti - \infty$. Raja-arvoa ei ole.

c) Jonon termit saavat vuorotellen arvoja $+1$ ja $-1$, mutta eivät 'lähesty' kumpaakaan näistä
tai mitään muutakaan lukua. Päätellään, että tälläkään jonolla ei ole raja-arvoa.

d) Tässä on ensimmäinen suppeneva jono: Kirjoittamalla
\[
a_n = 1+\frac{1}{n}
\]
nähdään, että $n$:n kasvaessa termit tulevat yhä lähemmäksi lukua $1$, joten päätellään, että 
$\lim_n a_n = 1$.

e) Tämä jono on myös suppeneva, mutta oikea raja-arvo paljastuu vasta hyvin suurilla indeksin
arvoilla: Jos esimerkiksi tutkitaan jonon termejä indeksin arvoilla 
$n = N, N+1,\ldots,N+10^{10}$, niin saadaan seuraavat tulokset:
\[
N \le n \le N + 10^{10} \quad \impl \quad \begin{cases}
                                          a_n \approx 1, \quad \ \ \text{kun}\ N = 1,        \\
                                          a_n \approx 1, \quad \ \ \text{kun}\ N = 10^{50},  \\
                                          a_n \approx 3/2,\ \      \text{kun}\ N = 10^{100}, \\
                                          a_n \approx 2, \quad \ \ \text{kun}\ N = 10^{120}.
                                          \end{cases}
\]
Vasta viimeinen testi kertoo totuuden: $\lim_n a_n = 2$. \loppu

Esimerkki \ref{jonoja}e) paljastaa, että lukujonon raja-arvoa ei välttämättä heti 'näe' tai 
löydä kokeilemallakaan. Tämäntyyppinen epävarmuus voidaan poistaa vain täsmällisemmällä 
raja-arvon määritelmällä.

\begin{Def} \label{jonon raja} (\vahv{Lukujonon raja-arvo})
\index{lukujonon raja-arvo|emph} \index{lukujono!d@suppeneva|emph}
\index{raja-arvo!a@lukujonon|emph} \index{suppeneminen!a@lukujonon|emph} Lukujono $\{a_n\}$
suppenee kohti raja-arvoa $a$, jos jokaisella $\eps > 0$\footnote[2]{Kreikkalainen kirjain
$\eps$ (luetaan  'epsilon') on matematiikassa hyvin vakiintunut pienen positiivisen luvun
symboli. Toinen usein käytetty symboli on myös kreikkalaiseen kirjaimistoon kuuluva $\delta$
(luetaan 'delta').} on olemassa $N\in\N$ siten, että pätee
\[
\abs{a_n - a}\ < \eps, \quad \text{kun}\ n > N.
\]
\end{Def}
Määritelmän mahdollisimman 'pakattu' muoto on 
\[
\lim_n a_n = a 
    \qekv \forall \eps > 0\ \exists N \in \N\ (\,\abs{a_n-a} < \eps\ \,\forall n > N\,).
\]
Jos $a$:n tilalla on $\infty$, on määritelmässä ehdon $\abs{a_n - a} < \eps$ tilalla oltava ehto
$a_n > M$, missä $M$ on mielivaltaisen suuren luvun symboli. Määritelmän tiivis muoto on tällöin
\[
\lim_n a_n = \infty \qekv \forall M\ \exists N \in \N\ (\,a_n > M\ \,\forall n > N\,).
\]

Määritelmässä \ref{jonon raja} $N$ riippuu yleensä (jokseenkin aina!) $\eps$:sta. Riippuvuuden
voimakkuudella ei ole väliä, kunhan jokaisella p\pain{ositiivisella} $\eps$ on löydettävissä
\pain{äärellinen} indeksi $N$ (eli luku $N\in\N$) siten, että asetettu ehto on voimassa. Ehto
puolestaan merkitsee, että indeksistä $N$ eteenpäin jonon \pain{kaikki} termit pysyvät enintään
$\eps$:n päässä $a$:sta. Kun ehtoa kiristetään antamalla $\eps$:lle yhä pienempiä mutta 
positiivisia arvoja, niin $N$ yleensä kasvaa, mutta pysyy aina äärellisenä. Vastaavasti jos
$\lim_n a_n = \infty$, niin indeksistä $N$ eteen päin jonon kaikki termit ovat lukua $M$
suurempia. Luvun $M$ kasvaessa kasvaa yleensä myös $N$, mutta pysyy äärellisenä jokaisella
$M$:n (äärellisellä) arvolla.

\begin{Lause} \label{raja-arvon yksikäsitteisyys} Jos lukujono $\{a_n\}$ on suppeneva, niin 
raja-arvo $a = \lim_n a_n$ on yksikäsitteinen.
\end{Lause}
\tod Jos $a_n \kohti a$ ja $b \neq a$, niin kolmioepäyhtälön (Lause \ref{kolmioepäyhtälö}) ja 
suppenemisen määritelmän perusteella jokaisella $\eps>0$ on olemassa $N\in\N$ siten, että
pätee
\begin{align*}
\abs{a_n - b}\ &=\ \abs{\,(a_n - a) + (a-b)\,} \\
               &\ge\ \abs{a-b} - \abs{a_n - a} \\
               &>\ \abs{a-b} - \eps, \quad \text{kun}\ n > N.
\end{align*}
Kun tässä valitaan $\eps = \tfrac{1}{2}\abs{a-b}$ (mahdollista, koska 
$a \neq b\ \impl\ \abs{a-b} > 0\,$), niin kyseisellä $\eps$:n arvolla pätee
\[
\abs{a_n - b} > \eps, \quad \text{kun}\ n > N,
\]
joten suppenemisen määritelmän mukaan $a_n \not\kohti b$. On näytetty toteen lauseen
väittämä muodossa
\[
a_n \kohti a\ \ \ja\ \ b \neq a \quad \impl \quad a_n \not\kohti b. \quad \loppu
\]
\jatko \begin{Exa}(jatko) c) $a_n = (-1)^n$. \ Jos valitaan mikä tahansa luku $a$, niin 
$\forall n \in \N$ pätee
\[
\max \{\abs{a_n - a},\ \abs{a_{n+1} - a}\}\ =\ \max \{\abs{1-a},\ \abs{1+a}\}\ \ge\ 1.
\]
Näin ollen jos valitaan $\eps < 1$, niin suppenevuden määritelmän ehto
\[
\abs{a_n - a}\ < \eps, \quad \text{kun}\ n > N
\]
ei ole voimassa millään indeksin $N$ arvolla. Siis Määritelmän \ref{jonon raja} mukaista 
raja-arvoa ei ole olemassa.

d) $a_n = (n+1)/n$. \ Kun $a=1$, pätee
\[
\abs{a_n - a} = \frac{1}{n} < \eps, \quad \text{kun}\ \ n > \frac{1}{\eps}\,.
\]
Näin ollen jos valitaan esimerkiksi $N=$ (lukua $1/\eps$ lähinnä suurempi kokonaisluku),
niin raja-arvon määritelmässä asetettu ehto toteutuu ko.\ $N$:n arvolla. Siis
$a_n \kohti 1$.

e) $a_n = (10^{100}+2n)/(10^{100}+n)$. \ Kun valitaan $a=2$, niin
\[
a_n - a = \dfrac{10^{100}+2n}{10^{100}+n} - 2 
        = - \dfrac{10^{100}}{10^{100}+n} \qimpl \abs{a_n - a} < \frac{10^{100}}{n}\,.
\]
Näin ollen
\[
\abs{a_n - a} < \eps, \quad \text{kun}\ \ n \ge 10^{100} \cdot {\eps}^{-1}.
\]
Siis jos valitaan $N$ esimerkiksi siten, että
\[
10^{100} \cdot {\eps}^{-1}\ \le\ N\ <\ 10^{100} \cdot {\eps}^{-1} + 1,
\]
niin suppenevuuden määritelmässä asetettu vaatimus $\abs{a_n - a} < \eps$ on voimassa, kun 
$n > N$. Vaikka $N$ on hyvin suuri (jo kun $\eps=1$), on $N$ kuitenkin äärellinen jokaisella 
$\eps > 0$, joten määritelmän mukaan $\,a_n \kohti 2$. \loppu 
\end{Exa}

Seuraavaa raja-arvotulosta tarvitaan käytännössä (ja teoriassakin) usein.
\begin{Prop} \label{potenssilimes} $\D \
\lim_{n \kohti \infty} q^n = \begin{cases}     
                                 0,      \quad\,\ \text{jos}\ -1 < q < 1,  \\
                                 \infty, \quad    \text{jos}\ \    q\,> 1.
                             \end{cases} $
\end{Prop}
\tod Jos $q=0$, niin $q^n = 0\ \forall n \in \N$, jolloin $q^n \kohti 0$ raja-arvon määritelmän
mukaan. Jos $0<q<1$, niin $q = 1/(1+x)$, missä $x>0$. Tällöin on Bernoullin epäyhtälön 
(Propositio \ref{Bernoulli}) perusteella
\[
q^n = \frac{1}{(1+x)^n} < \frac{1}{1+nx} < \frac{1}{nx}\,, \quad n \ge 2.
\]
Tässä on jokaisella $\eps>0$
\[
\frac{1}{nx} < \eps, \quad \text{kun}\ \ n > \frac{1}{x \eps}\,,
\]
joten valitsemalla $N\in\N$ siten, että $N \ge (x \eps)^{-1}$ (mahdollista aina kun $x>0$
ja $\eps > 0$), seuraa
\[
0 < q^n < \eps, \quad \text{kun}\ n > N.
\]
Raja-arvon määritelmän perusteella $q^n \kohti 0$. Jos lopulta $-1 < q < 0$, niin
$q^n = (-1)^n (-q)^n$, missä $0 < -q < 1$, joten valisemalla $N$ samalla tavoin kuin
edellä seuraa
\[
-\eps < q^n < \eps, \quad \text{kun}\ n > N.
\]
Jälleen $q^n \kohti 0$ raja-arvon määritelmän perusteella, joten ensimmäinen väittämä on 
todistettu. Tapausta $q>1$ koskeva toinen väittämä todistetaan vastaavalla tavalla,
kirjoittamalla ensin $q = 1+x$ ja käyttämällä Bernoullin epäyhtälöä. \loppu

Lukujonon suppenemisen määritelmästä todettakoon vielä, että siinä asetettua ehtoa 
'jokaisella $\eps>0$' on mahdollista lieventää määritelmän muuttumatta. Esimerkiksi riittää
valita $\eps$ jonosta $\{10^{-k},\ k=0,1,2,\ldots\,\}$, jolloin määritelmäksi tulee
(vrt.\ johdattelu luvun alussa): Jokaisella $k=0,1,2,\ldots$ on olemassa indeksi $N_k\in\N$
siten, että $\abs{a_n-a} < 10^{-k}$ kun $n>N_k$, eli:
\[
\lim_n a_n = a 
\qekv \forall k\in\N\cup\{0\}\ \exists N_k\in\N\ (\,\abs{a_n-a} < 10^{-k}\ \,\forall n > N_k\,).
\]
Tässä voi $10^{-k}$:n tilalla olla yhtä hyvin $q^k,\ 0<q<1$, tai vielä yleisemmin, $\eps$ on
mahdollista poimia mistä tahansa aidosti vähenevästä (ks.\ Määritelmä \ref{monotoninen jono})
lukujonosta $\seq{b_k}$, kunhan on ensin varmistettu Määritelmään \ref{jonon raja} vedoten, että
$\lim_k b_k=0$. Tulos on täsmällisemmin muotoiltuna seuraava. Todistus jätetään
harjoitustehtäväksi (Harj.teht.\,\ref{H-I-6: jonon raja}a).
\begin{Prop} \label{jonon raja 2} Olkoon $\{b_k,\ k=1,2,\ldots\}$ aidosti vähenevä lukujono,
jonka raja-arvo on $\lim_k b_k = 0$. Tällöin lukujonon $\seq{a_n}$ raja-arvo $=a$ täsmälleen
kun jokaisella $k\in\N$ on olemassa indeksi $N_k\in\N$ siten, että $\abs{a_n-a}<b_k$ kun $n>N_k$.
\end{Prop}

\subsection{Sarjan summa}
\index{sarjan summa|vahv}

Jos sarja $\,\sum_{n=m}^{\infty} a_n$ suppenee (osasummiensa lukujonona, vrt.\ Luku \ref{jono}),
niin raja-arvoa sanotaan \kor{sarjan summaksi}. Raja-arvo merkitään
\[
s\ =\ \lim_{n \kohti \infty}\,\sum_{k=m}^n a_k\ =\ \sum_{k=m}^{\infty} a_k.\footnote[2]{Jos 
sarja on suppeneva, niin asiayhteydestä on selvitettävä, tarkoittaako 
$\,\sum_{n=m}^{\infty} a_n\,$ itse sarjaa vai sen summaa. Esimerkiksi merkintä 
$\sum_{k=m}^\infty a_k = \sum_{k=m}^\infty b_k$ voi tarkoittaa joko, että sarjat ovat samat 
lukujonoina (jolloin ne ovat samat myös termeittäin: $a_k=b_k\ \forall k$), tai ainoastaan, että
sarjojen summat ovat samat.}
\]  
\begin{Exa} \label{geometrinen sarja} Tutki perusmuotoisen geometrisen sarjan suppenemista. \end{Exa}
\ratk Perusmuotoinen geometrinen sarja on lukujono
\[
\{s_n\}\ =\ \{\, \sum_{k=0}^n q^k\,\}\ 
         =\ \begin{cases}
            \{\,n+1,\ n = 0,1, \ldots\,\}, \qquad\qquad\qquad \ \ \text{jos}\ q=1, \\
            \{\,(q^{n+1}-1)/(q-1),\ n = 0,1, \ldots\,\}, \quad \text{jos}\ q \neq 1.
            \end{cases}
\]
Jos $q=1$, niin sarja ei suppene ($s_n \kohti \infty$). Jos $q = -1$, niin 
$\,\{s_n\} = \{\,1,0,1,0 \ldots\,\}$, joten sarja ei suppene tässäkään tapauksessa. Jos 
$\abs{q} > 1$, niin arvioidaan kolmioepäyhtälön avulla
\[
\abs{s_n}\ =\ \dfrac{1}{\abs{q-1}}\,\abs{q^{n+1}-1}\ 
           \ge\ \dfrac{1}{\abs{q-1}}\,\left(\abs{q}^{n+1} - 1\right).
\]
Proposition \ref{potenssilimes} avulla päätellään tästä, että $\abs{s_n} \kohti \infty$, joten
tässäkään tapauksessa sarja ei suppene. Jäljelle jääneessä tapauksessa $-1 < q < 1$ valitaan 
raja-arvokandidaatiksi $\ s = 1/(1-q)$, jolloin
\[
\abs{s_n - s}\ =\ \dfrac{\abs{q}^{n+1}}{\abs{1-q}}\,.
\]
Olkoon $\eps>0$, jolloin on myös $\abs{1-q}\,\eps>0$. Koska $\abs{q}^{n+1} \kohti 0$
(Propositio \ref{potenssilimes}) ja $\abs{1-q}\,\eps>0$, niin lukujonon raja-arvon määritelmän
perusteella on olemassa indeksi $N\in\N$ siten, että
\[
\abs{q}^{n+1} < \abs{1-q}\,\eps \qimpl \abs{s_n-s}<\eps, \quad \text{kun}\ n>N.
\]
Tässä $\eps>0$ oli mielivaltainen ja $N\in\N$, joten lukujonon raja-arvon määritelmän 
perusteella $\ \lim_n s_n = s$. Siis geometrinen sarja suppenee täsmälleen kun $\abs{q} < 1$,
ja raja-arvo on tällöin
\[
\lim_n s_n\ =\ \lim_n\,(\,\sum_{k=0}^n q^k\,)\ =\ \frac{1}{1-q}\,. \quad \loppu
\]

Sarjan summamerkintää käyttäen esimerkin tulos on
\[
\boxed{\quad \sum_{k=0}^{\infty} q^k\ =\ \frac{1}{1-q}\,, \quad |q|<1. \quad}
\]

\subsection{Rationaalilukujono, jonka raja-arvo ei ole rationaalinen}

Rationaalilukujonon raja-arvon ei tarvitse olla rationaaliluku, sillä Määritelmän
\ref{jonon raja} mukaisesti riittää, että raja-arvo on löydettävissä lukujoukosta
$\K\supset\Q$, jossa laskuoperaatiot ja järjestysrelaatio ovat määriteltyjä niin,
että $(\K,+,\cdot,<)$ on rationaalilukujen kunnan laajennus. Seuraavassa on esimerkki 
rationaalilukujonosta, joka suppenee, kunhan lukujoukkoon $\K\supset\Q$ sisältyy 
myös luku $a=\sqrt{2}$ (vrt.\ Luku \ref{kunta}, Esimerkki \ref{muuan kunta}).
\begin{Exa} \label{sqrt 2} Tarkastellaan palautuvaa rationaalilukujonoa
\[
a_0 = 2, \quad a_{n+1} = \frac{a_n}{2} + \frac{1}{a_n}\,, \quad n = 0,1,\ldots
\]
Palautuskaavasta seuraa, että $a_n>0\ \forall n$ (induktio!) ja että
\[
a_{n+1}^2 -2 = \frac{1}{4 a_n^2}\left(a_n^2 - 2\right)^2, \quad n = 0,1,\ldots
\]
Tästä nähdään (induktio!), että on oltava $a_n^2-2>0\ \forall n$, jolloin voidaan edelleen 
päätellä:
\[
a_{n+1}^2-2\,=\,\frac{a_n^2-2}{4 a_n^2}\,(a_n^2-2)\,
             <\,\frac{a_n^2}{4a_n^2}\,(a_n^2-2)
             =\,\frac{1}{4}\,(a_n^2-2), \quad n=0,1, \ldots
\]
Koska $a_0^2-2 = 2$, niin seuraa (induktio!)
\[
0\,<\,a_n^2-2\,<\,2\cdot 4^{-n}, \quad n=0,1, \ldots
\]
Propositioiden \ref{potenssilimes} ja \ref{jonon raja 2} perusteella seuraa tästä, että
$a_n^2 \kohti 2$.

On päätelty, että $a_n>0\ \forall n$ ja lisäksi, että lukujono $\seq{a_n^2}$ on suppeneva ja
$a_n^2 \kohti 2$. Näytetään nyt, että Määritelmän \ref{jonon raja} mukaisesti on oltava 
$\,\lim_n a_n=a=\sqrt{2}$. Tätä silmällä pitäen kirjoitetaan ensin
\[
a_n^2 - 2 = a_n^2 - a^2 = (a_n + a)(a_n - a).
\]
Koska oli $a_n>0$ ja $a_n^2>2=a^2$ jokaisella $n$, niin $a_n>a\ \forall n$ 
(ks.\ Luku \ref{kunta}). On myös $a>1$, joten seuraa
\begin{align*}
a_n^2-2\,&=\,(a_n+a)(a_n-a)\,>\,2a\,(a_n-a)\,>\,2(a_n-a) \\[1mm]
         &\impl\quad a_n-a\,<\,2^{-1}(a_n^2-2)\,<\,4^{-n}, \quad n=0,1, \ldots
\end{align*}
Siis $\,0<a_n-a<4^{-n}\ \forall n$, joten $\,a_n \kohti a\,$ (vrt.\ päättely edellä).
Lukujonolle $\seq{a_n}$ on näin löydetty ei-rationaalinen raja-arvo $a=\sqrt{2}$. \loppu
\end{Exa}

Esimerkin lukujonolla $\seq{a_n}$ ei voi olla rationaalista raja-arvoa, koska 
$\sqrt{2}\not\in\Q$ ja Lauseen \ref{raja-arvon yksikäsitteisyys} mukaan raja-arvo on
yksikäsitteinen. Rationaalilukujonojen joukossa on siis sellaisia, jotka 
'näyttävät suppenevan', mutta joilla ei kuitenkaan ole raja-arvoa rationaalilukujen
joukossa. Milloin rationaalilukujonolle on yleisemmin löydettävissä raja-arvo 
lukujoukkoa (rationaalilukujen kuntaa) laajentamalla ja milloin ei, on keskeinen kysymys
jatkossa.

\Harj
\begin{enumerate}

\item
Seuraavissa esimerkeissä lukujono $\seq{a_n}$ joko suppenee kohti lukua $a$ tai kasvaa
rajatta. Määritä pienin $N\in\N$ siten, että pätee joko $\abs{a_n-a}<1/100$ tai $a_n>100$,
kun $n>N$\,:
\begin{align*}
&\text{a)}\ \ a_n = \frac{4n-1}{2n+1} \qquad 
 \text{b)}\ \ a_n = \frac{4n+100}{2n+100} \qquad
 \text{c)}\ \ a_n = \frac{3n-10000}{2n+10000} \\
&\text{d)}\ \ a_n = \frac{n^2}{2n+1} \qquad 
 \text{e)}\ \ a_n = \frac{n^2}{3n+100} \qquad\,
 \text{f)}\ \ a_n = \frac{n^2}{4n+10000}
\end{align*}

\item
Olkoon
\[
a_n = \begin{cases} 1/k, &\text{kun}\ n=10^k,\ k\in\N, \\
                    0,   &\text{muulloin ($n\in\N$)},
      \end{cases} \quad
b_n = \begin{cases} 10^{-100}, &\text{kun}\ n=10^k,\ k\in\N, \\
                    0,         &\text{muulloin ($n\in\N$)}.
      \end{cases}
\]
Näytä, että \ a) $a_n \kohti 0$, \ b) \ $\seq{b_n}$ ei suppene kohti mitään lukua.

\item \label{H-I-6: lukujonopari}
Todista, että lukujonoille pätee: \vspace{1mm}\newline
a) \ $\lim_n a_n=a\,\ \impl\,\ \lim_n|a_n|=|a|$. \vspace{1mm}\newline
b) \ $\lim_n a_n=a\ \ja\ a_n \ge b_n \ge a\ \forall n\,\ \impl\,\ \lim_n b_n=a$.
\vspace{1mm}\newline
c) \ $a_n \kohti a\ \ja\ b_n \kohti b\,\ 
                    \ekv\,\ c_n=\max\{\abs{a_n-a},\abs{b_n-b}\} \kohti 0$. \vspace{1mm}\newline
d) \ $a_n \kohti a\ \ja\ b_n \kohti b\,\ 
                    \ekv\,\ (a_n-a)^2+(b_n-b)^2 \kohti 0$. \vspace{1mm}\newline
e) \ $a_n \kohti a\ \ja\ b_n \kohti b\ \ja\ c_n \kohti c\,\
                    \ekv\,\ (a_n-a)^2+(b_n-b)^2+(c_n-c)^2 \kohti 0$.

\item
Lukujonoista $\seq{a_n}$ ja $\seq{b_n}$ tiedetään, että $a_n \kohti a$ ja $b_n \kohti a$ ja
lukujonosta $\seq{c_n}$, että jokaisella $n$ on joko $c_n=a_n$ tai $c_n=b_n$ (ei tiedetä, kumpi).
Näytä, että $c_n \kohti a$.

\item
Olkoon $q\in\Q,\ q>0$. Laske $\displaystyle{\ \lim_n\,q^{-n}\sum_{k=0}^n 2^k}\,$ eri $q$:n 
arvoilla.

\item
Määritä pienimmät Proposition \ref{jonon raja 2} mukaiset luvut $N_k,\ k=1,2,3$, kun
$b_k=10^{-k}$ ja $a_n=0.9^n,\ n\in\N$.

\item (*) \label{H-I-6: jonon raja}
a) Todista Propositio \ref{jonon raja 2}. \ b) Muotoile ja todista Proposition
\ref{jonon raja 2} vastine koskemaan lukujonoa $\seq{a_n}$, joka kasvaa rajatta.

\item (*)
Todista: \ $a_n>0\ \forall n\ \ja\ \lim_n a_n = 0\ \impl\ \exists \max_n\{a_n\}$. Näytä myös 
vastaesimerkillä, että $\,\lim_n a_n = 0\ \not\impl\ \exists \max_n\{a_n\}$.

\item (*)
Olkoon $k\in\N$, $\Q_k = \{\,\text{äärelliset desimaaliluvut muotoa}\ x_0.d_1 \ldots d_k\,\}$,
ja $a_n\in\Q_k\ \forall n\in\N$. Näytä, että jos $\,\lim_n a_n=a$, niin $a\in\Q_k$ ja jollakin
$N\in\N$ pätee: $a_n=a$ kun $n>N$. 

\item (*)
Olkoon $\{\,a_k,\ k=1,2, \ldots\,\}$ rationaalilukujono ja $\{\,N_k,\ k=1,2, \ldots\,\}$ aidosti
kasvava indeksijono (luonnollisten lukujen jono) siten, että $N_1=1$. Määritellään jono
$\seq{b_n}$ asettamalla
\[
b_n=a_k, \quad \text{kun}\ N_k \le n < N_{k+1}, \quad k=1,2, \ldots
\]
Todista: \ $\lim_k a_k=a\ \impl\ \lim_n b_n = a$.

\item (*) \label{H-I-6: sqrt-kunta}
Tarkastellaan kunnassa $(\J,+,\cdot,<)$ (ks.\ Luku \ref{kunta}) palautuvasti määriteltyä
lukujonoa
\[
a_0 = x, \quad a_{n+1} = \frac{1}{2}\left(a_n + \frac{x}{a_n}\right), \quad n=0,1, \ldots
\]
missä $x\in\J,\ x>0$. Näytä: \\ \\
a) $\displaystyle{\quad 
      a_{n+1}^2-x = \frac{1}{4a_n^2}\left(a_n^2-x\right)^2, \quad n=0,1, \ldots}$ \\ \\
b) $\quad a_n\ge\sqrt{x}, \quad n=1,2, \ldots$ \\ \\ 
c) $\quad \lim_n a_n^2 = x.$ \\ \\
d) $\quad \lim_n a_n = \sqrt{x}.$

\end{enumerate}
  % Lukujonon raja-arvo
\section{Suppenevien lukujonojen ominaisuuksia} \label{suppenevat lukujonot}
\alku
\sectionmark{Suppenevat lukujonot}

Tässä luvussa tarkastellaan suppenevien lukujonojen ominaisuuksia yleiseltä kannalta 
--- 'luku' (mat.) ymmärretään samoin kuin edellisessä luvussa.
\begin{Def} \label{rajoitettu jono} \index{lukujono!e@rajoitettu|emph}
\index{rajoitettu!a@lukujono|emph}
Lukujono $\{a_n\}$ on \kor{rajoitettu} (engl.\ bounded),  jos on olemassa luku $C$ siten, että
pätee $\abs{a_n} \le C\ \ \forall n$. 
\end{Def}
\begin{Lause} \label{suppeneva jono on rajoitettu} Suppeneva lukujono on rajoitettu. 
\end{Lause}
\tod Olkoon jono yleistä muotoa $\{a_n\}_{n=m}^{\infty},\ m \in \Z$. Jos pätee $a_n \kohti a$,
niin suppenemisen määritelmän mukaan $\exists N \in \N$ siten, että
\[
\abs{a_n - a} < 1, \quad \text{kun}\ n > N
\]
(Määritelmässä \ref{jonon raja} valittu $\eps = 1$). Tällöin on kolmioepäyhtälön mukaan
\begin{align*}
\abs{a_n}\ &=\ \abs{\,(a_n - a) + a\,} \\
           &\le\ \abs{a_n - a} + \abs{a}\ <\ 1 + \abs{a}, \quad \text{kun}\ n > N,
\end{align*}
joten
\[
\abs{a_n}\ \le\ \max\{\,\abs{a_m}, \ldots, \abs{a_N}, \abs{a}+1\,\}\ 
           =\ C \quad \forall n \ge m. \loppu
\]

Seuraavat yleiset laskusäännöt ovat huomattavan käyttökelposia sekä raja-arvoja
laskettaessa että teoreettisemmissa tarkasteluissa.
\begin{Lause} \label{raja-arvojen yhdistelysäännöt} (\vahv{Raja-arvojen yhdistelysäännöt})
\index{lukujonon raja-arvo!64@raja-arvojen yhdistely|emph}
Olkoon $\seq{a_n}$ ja $\seq{b_n}$ lukujonoja ja $a,b,\lambda$ lukuja. Tällöin jos 
$\lim_n a_n = a$ ja $\lim_n b_n = b$, niin pätee:
\begin{itemize}
\item[(1)] $\quad \lim_n (a_n + b_n)\ =\ a+b$.
\item[(2)] $\quad \lim_n (\lambda a_n)\ =\ \lambda a$.
\item[(3)] $\quad \lim_n (a_n b_n)\ =\ ab$.
\item[(4)] $\quad \lim_n (a_n / b_n)\ =\ a/b \quad\,\ $ lisäehdoilla: \
                  $b_n \neq 0\ \forall n$ ja $b \neq 0$.
\end{itemize} \end{Lause}
Todistetaan ainoastaan väittämät (3) ja (4)
(muut ovat helpompia: Harj.teht.\,\ref{H-I-7: todistuksia}).

\pain{Väittämä} (3) \ Kirjoitetaan ensin
\begin{align*}
a_n b_n - ab\ &=\ (a_n b_n - a b_n) + (a b_n - ab) \\
              &=\ b_n(a_n - a) + a(b_n - b).
\end{align*} 
Koska $\{b_n\}$ on suppeneva jono, niin Lauseen \ref{suppeneva jono on rajoitettu} mukaan on 
jollakin $C$
\[
\abs{b_n} \le C\ \ \forall n,
\]
jolloin kolmioepäyhtälön perusteella
\[
\abs{a_n b_n - ab}\ \le\ C \abs{a_n - a} + \abs{a} \abs{b_n - b}.
\]
Jatkossa oletetaan, että $C > 0$ ja $\abs{a} > 0$ (jos $C=0$ tai $a=0$, niin päättely
yksinkertaistuu). Valitaan mielivaltainen $\eps > 0$, jolloin
\[
\eps_1 = \frac{\eps}{2C} > 0 \quad \text{ja} \quad \eps_2 = \frac{\eps}{2\abs{a}} > 0.
\]
Koska lukujonot $\{a_n\}$ ja $\{b_n\}$ suppenevat, on tällöin löydettävissä indeksit 
\mbox{$N_1 \in \N$} ja $N_2 \in \N$, siten että 
pätee
\begin{align*}
\abs{a_n - a}  &< \eps_1\,, \quad \text{kun}\ n > N_1, \\
\abs{b_n\,- b} &< \eps_2\,, \quad \text{kun}\ n > N_2.
\end{align*}
Tässä molemmat epäyhtälöt ovat voimassa indeksistä $N = \max\{N_1,N_2\}\in\N$ eteenpäin,
joten pätee
\[
\abs{a_n b_n - ab}\ <\ C \eps_1 + \abs{a} \eps_2\ 
                    =\ \frac{\eps}{2} + \frac{\eps}{2}\ =\ \eps, \quad \text{kun}\ n > N. 
\]
Koska tässä $\eps > 0$ oli mielivaltainen ja $N\in\N$ jokaisella $\eps>0$, niin lukujonon
raja-arvon määritelmän mukaan $a_n b_n \kohti ab$. \loppu

\pain{Väittämä} (4) \ Tässä aloitetaan kirjoittamalla
\begin{align*}
\frac{a_n}{b_n} - \frac{a}{b}\ &=\ \frac{a_n b - a b_n}{b_n b}\
                                =\ \frac{(a_n b - ab) + (ab - ab_n)}{b_n b} \\
                               &=\ \frac{a_n - a}{b_n} - \frac{a(b_n - b)}{b_n b}\,.
\end{align*}
Koska $b_n \kohti b \neq 0$, niin $\exists N_0 \in \N$ siten, että
\[
\abs{b_n - b} < \tfrac{1}{2} \abs{b}, \quad \text{kun}\ n > N_0
\]
(raja-arvon määritelmässä asetettu $\eps = \frac{1}{2} \abs{b},\ N=N_0$). Tällöin on
kolmioepäyhtälön nojalla
\[
\abs{b_n}\ \ge\ \abs{b} - \abs{b_n - b}\ >\ \tfrac{1}{2} \abs{b} \quad \forall n >N_0,
\]
joten
\begin{align*}
\abs{\,\frac{a_n}{b_n} - \frac{a}{b}\,}\ 
         &\le\ \frac{\abs{a_n - a}}{\abs{b_n}} + \abs{\,\frac{a}{b_n b}\,}\,\abs{b_n - b} \\
         &<\ \frac{2}{\abs{b}}\,\abs{a_n - a} + \frac{2 \abs{a}}{b^2}\,\abs{b_n - b} \quad 
                                                                         \forall n > N_0.
\end{align*}
Tässä on oikea puoli $< \eps\ (\eps > 0)$, kun
\begin{align*}
\frac{2}{\abs{b}}\,\abs{a_n - a}\ <\ \frac{\eps}{2} \quad 
        &\text{ja} \quad \frac{2 \abs{a}}{b^2}\,\abs{b_n - b}\ < \frac{\eps}{2}\,, \\
\intertext{eli kun}
\abs{a_n - a} < \frac{\abs{b}}{4}\,\eps = \eps_1 \quad 
        &\text{ja} \quad \abs{b_n - b} < \frac{b^2}{4 \abs{a}}\,\eps = \eps_2\,,
\end{align*}
olettaen että tässä $a \neq 0$. (Jos $a=0$, päättely yksinkertaistuu.) Koska $a_n \kohti a$ ja
$b_n \kohti b$ ja koska $\eps_1, \eps_2 > 0$, niin ensimmäinen ehto toteutuu jostakin indeksistä
$N_1$ ja toinen jostakin indeksistä $N_2$ eteenpäin, jolloin
\[
\abs{\,\frac{a_n}{b_n}-\frac{a}{b}\,}\ <\ \eps, \quad \text{kun}\ n>N = \max\{N_0,N_1,N_2\}\in\N.
\]
Tässä oli $\eps > 0$ mielivaltainen, joten määritelmän mukaan $a_n/b_n \kohti a/b$. \loppu

\begin{Exa} Laske $\ \lim_n a_n\,$, kun
\[
a_n = \frac{n^2+2n+2}{3n^2+n}\,.
\] \end{Exa}

\ratk Kirjoittamalla $a_n$ muotoon
\[
a_n = \frac{1 + 2 \cdot \tfrac{1}{n} + 2 \cdot \tfrac{1}{n} \cdot \tfrac{1}{n}}{3 + \tfrac{1}{n}}
\]
voidaan suoraan soveltaa Lauseen \ref{raja-arvojen yhdistelysäännöt} sääntöjä (1)--(4)\,:
\begin{align*}
\lim_n a_n\ &=\ \frac{\lim_n(1 + 2 \cdot \tfrac{1}{n} 
              + 2 \cdot \tfrac{1}{n} \cdot \tfrac{1}{n})}{\lim_n(3 + \tfrac{1}{n})} \\
            &=\ \frac{\lim_n(1) + 2\,\lim_n(\tfrac{1}{n}) 
              + 2\,\lim_n(\tfrac{1}{n}) \cdot \lim_n(\tfrac{1}{n})}{3 + \lim_n(\tfrac{1}{n})} \\
            &=\ \frac{1 + 2 \cdot 0 + 2 \cdot 0 \cdot 0}{3 + 0} = \frac{1}{3}\,.
\end{align*}
Tässä tarvittiin siis ainoastaan yhtä suoraan määritelmästä todistettavaa raja-arvotulosta
$\ \lim_n (\tfrac{1}{n}) = 0\ $ (vrt.\ Esimerkki \ref{jonoja}d edellisessä luvussa). \loppu

\begin{Exa} Yleinen geometrinen sarja on lukujono 
\[
\sum_{k=0}^{\infty} aq^k\ =\ \{\,\sum_{k=0}^n aq^k,\ \ n=0,1, \ldots\,\}. 
\]
Kun perusmuotoisen sarjan summa tiedetään (edellisen luvun Esimerkki \ref{geometrinen sarja}),
niin Lauseen \ref{raja-arvojen yhdistelysäännöt} säännön (2) mukaan yleisen geometrisen
sarjan summa on
\[
\sum_{k=0}^{\infty} aq^k\ =\ a\,\sum_{k=0}^{\infty} q^k\ = \frac{a}{1-q}, \quad |q|<1.
\]
Jos $\abs{q} \ge 1$, niin yleinen geometrinen sarja suppenee vain kun $a=0$. \loppu 
\end{Exa}
\begin{Exa} Edellisen luvun Esimerkissä \ref{sqrt 2} tarkasteltiin palautuvaa lukujonoa
\[
a_0 = 2, \quad a_{n+1} = \frac{a_n}{2} + \frac{1}{a_n}\,, \quad n = 0,1,\ldots
\] 
Jos oletetaan, että $a_n>0\ \forall n$ ja $a_n \kohti a>0$ (vrt.\ edellisen luvun tarkastelut),
niin Lauseen \ref{raja-arvojen yhdistelysäännöt} mukaan on oltava
\[
a = \lim_n a_{n+1} = \lim_n \left(\frac{a_n}{2} + \frac{1}{a_n}\right) 
                   = \frac{a}{2} + \frac{1}{a}\,.
\]
Siis
\[
a = \frac{a}{2} + \frac{1}{a} \qimpl a^2 = 2.
\]
Koska $a>0$, niin seuraa $a=\sqrt{2}$.  \loppu
\end{Exa}
\begin{Exa} Jos palautuva lukujono
\[
a_{n+1} = qa_n - 1, \quad n=0,1,\ldots
\]
suppenee (jollakin $a_0$) kohti raja-arvoa $a$, niin tästä oletuksesta seuraa:
\begin{align*}
q=1     &\qimpl a=a-1  \,\ \qimpl 0=1. \\
q\neq 1 &\qimpl a=qa-1 \qimpl a=1/(q-1).
\end{align*}
Siis päätellään, että jono \pain{ei} suppene (millään $a_0$), jos $q=1$, ja että muissa
tapauksissa ainoa mahdollinen raja-arvo on $a=1/(q-1)$. Jonoa tarkemmin tutkimalla selviää, että 
\[
a_n = a_0 q^n - \sum_{k=0}^{n-1} q^k, \quad n=1,2,\ldots,
\]
joten jono suppenee alkuarvosta $a_0$ riippumatta täsmälleen kun $\abs{q}<1$, ja tällöin siis
$a_n \kohti 1/(q-1)$. --- Huomattakoon, että päättely tapauksessa $q=1$ on itse asiassa epäsuora
todistus väittämälle: $\seq{a_n}$ ei suppene. (Vrt.\ Luku \ref{logiikka}.) \loppu
\end{Exa}

Seuraavassa vielä kolme väittämää, jotka perustuvat suoraan lukujonon raja-arvon 
määritelmään. Väittämistä keskimmäinen tunnetaan '\kor{voileipälauseena}'.
\begin{Lause} \label{jonotuloksia} \index{lukujonon raja-arvo!68@'voileipälause'|emph}
Lukujonoille pätee
\begin{itemize}
\item[V1.] Jos $\,a_n \le b\ \forall n\ (a_n \ge b\ \forall n)$ ja $a_n \kohti a$,
           niin $a \le b\ (a \ge b)$.
\item[V2.] Jos $a_n \kohti c\ \ja\ b_n \kohti c\ \ja\ a_n \le c_n \le b_n\ \forall n$,
           niin $c_n \kohti c$. 
\item[V3.] Jos $a_n \kohti 0$ ja jono $\seq{b_n}$ on rajoitettu, niin $a_n b_n \kohti 0$.
\end{itemize}
\end{Lause}
\tod \ V1. \ Oletetaan, että $a_n \le b\ \forall n$, ja tehdään vastaoletus: 
$\lim_n a_n = a > b$. Merkitään $\eps = a-b$, jolloin vastaoletuksen perusteella on $\eps > 0$.
Koska $a_n \le b\ \forall n$ (oletus) ja $b = a-\eps,\ \eps > 0$ (vastaoletuksen seuraamus),
niin seuraa $a_n \le a - \eps\ \forall n$ ja siis $\abs{a_n - a} \ge \eps > 0\ \forall n$. 
Suppenevuuden määritelmästä seuraa tällöin, että $a_n \not\kohti a$. Oletuksen mukaan oli 
kuitenkin $a_n \kohti a$, joten on päädytty loogiseen ristiriitaan: $a_n \kohti a$ ja 
$a_n \not\kohti a$. Vastaoletus $a>b$ on näin ollen väärä, eli on oltava $a \le b$. Väittämän 
ensimmäinen osa on siis todistettu. Toinen osa seuraa, kun jo todistettua väittämää
sovelletaan jonoon $\seq{-a_n}$.

V2. \ Koska $a_n \le c_n \le b_n\ \forall n$, niin
\begin{align*}
a_n-c\,\le\,c_n-c\,\le\,b_n-c\ \ \ja\ \ c-b_n\,&\le\,c-c_n\,\le\,c-a_n \\
                   \impl \quad \abs{c-c_n}\,&\le\,\max\{\abs{a_n-c},\abs{b_n-c}\} \,\ \forall n.
\end{align*}
Koska $a_n \kohti c$ ja $b_n \kohti c$, niin jokaisella $\eps>0$ on olemassa indeksit 
$N_1,N_2\in\N$ siten, että $\abs{a_n-c}<\eps$ kun $n>N_1$ ja $\abs{b_n-c}<\eps$ kun $n>N_2$.
Tällöin
\[
\abs{c-c_n} \le \max\{\abs{a_n-c},\abs{b_n-c}\} < \eps, \quad \text{kun}\ n>N=\max\{N_1,N_2\}.
\]
Määritelmän mukaan $c_n \kohti c$.

V3. \ Harjoitustehtävä (Tehtävä \ref{H-I-7: jonotulos 3}a). \loppu

\begin{Exa} Väittämän (vrt.\ Lauseen \ref{jonotuloksia} väittämämä V1) 
\[
a_n < b\,\ \forall n\,\ \ja\,\ a_n \kohti a \qimpl a<b
\]
osoittaa vääräksi vastaesimerkki $\,a_n=1-1/n,\ n\in\N$, $\,b=1$. \loppu
\end{Exa}
\begin{Exa} Kunnassa $(\J,+,\cdot,<)$ (ks.\ Luku \ref{kunta}) määritelty palautuva lukujono
\[
a_0=0, \quad a_{n+1}=\sqrt{a_n + 1}, \quad n=0,1,\ldots
\]
ilmoitetaan suppenevaksi. Mikä on raja-arvo?
\end{Exa}
\ratk Palautuskaavasta seuraa, että $a_n \ge 0\ \forall n$ ja että
\[
a_{n+1}^2 = a_n + 1, \quad n=0,1,\ldots
\]
Kun merkitään $\,\lim_n a_n=a$, niin Lauseen \ref{raja-arvojen yhdistelysäännöt} perusteella
seuraa
\[
\lim_n a_{n+1}^2 = a^2 = \lim_n(a_n + 1) = a+1.
\]
Siis $a^2-a-1=0\ \impl\ a=\tfrac{1}{2}(1\pm\sqrt{5})$. Koska $a_n \ge 0\ \forall n$, niin
on oltava $a \ge 0$ (Lause \ref{jonotuloksia}, väittämä V1). Siis 
$\,\lim_n a_n =\tfrac{1}{2}(1+\sqrt{5})$. \loppu

\subsection{Jaksolliset desimaaliluvut}
\index{jaksollinen desimaaliluku|vahv}
\index{zyzy@ääretön desimaaliluku!69@jaksollinen desimaaliluku|vahv}%

Luvussa \ref{desimaaliluvut} todettiin, että jokaiseen rationaalilukuun $x \in \Q$ voidaan
liittää ääretön, jaksollinen desimaaliluku $\x = \{x_n\} \in \DD_p$ kriteerillä
\[
\abs{x - x_n}\ <\ 10^{-n} \quad \forall n.
\]
Tämä merkitsee (vrt.\ Propositio \ref{jonon raja 2}), että
\[
\lim_n x_n = x.
\]
Jatkossa on luontevaa hieman väljentää Luvussa \ref{desimaaliluvut} sovittua samastusrelaatiota
rationaalilukujen ja äärettömien desimaalilukujen välillä. Asetetaan

\begin{Def} \label{samastus QD} Jos $x\in\Q\,$ ja $\x = \{x_n\} \in \DD$, niin
\[
x=\x \qekv \lim_n x_n = x.
\] 
\end{Def}

Määritelmän \ref{samastus QD} kriteerillä siis jokainen rationaaliluku samastuu edelleen ainakin
yhteen jaksolliseen desimaalilukuun, nimittäin jakokulma-algoritmin antamaan. Entä toisinpäin: 
Jos lähtökohtana on jaksollinen desimaaliluku, niin onko tämä aina samastettavissa johonkin 
rationaalilukuun Määritelmän \ref{samastus QD} mukaisesti? Asian tutkimiseksi oletetaan, että
desimaaliluku $\x$ on muotoa
\[
\x\ =\ x_0. d_1 \ldots d_k d_{k+1} \ldots d_{k+m} d_{k+1} \ldots d_{k+m} \ldots,
\]
eli jaksollisuus alkaa (viimeistään) desimaalista n:o $k+1$ ja jakson pituus on $m$.
Tarkastellaan jonoa
$\{x_k, x_{k+m}, x_{k+2m}, \ldots\,\} = \{x_{k+lm},\ l=0,1,\ldots\,\} = \seq{y_l}$. Tälle pätee
\[
y_{l+1}-y_l\ =\ \sum_{i=k+1}^{k+m} d_i \cdot 10^{-lm-i}\ 
                         =\ 10^{-lm}\,(y_1 - y_0), \quad l = 0,1 \ldots
\]
joten
\begin{align*}
y_{l+1}\ &=\ y_0\ +\ (y_1-y_0)\ +\ \ldots\ +\ (y_{l+1}-y_l) \\
         &=\ y_0\ +\ (y_1 -y_0)\,\sum_{j=0}^l 10^{-jm}, \quad l = 0,1 \ldots
\end{align*}
Näin ollen jono $\seq{y_l}$ suppenee geometrisena sarjana:
\[
\lim_{l \kohti \infty} y_l\ =\ y_0\ +\ \frac{y_1-y_0}{1 - 10^{-m}}\
                            =\ x_k\ +\ \frac{x_{k+m} - x_k}{1 - 10^{-m}}\ =\ x \in \Q.
\]
Tästä ja jonon $\{x_n\}$ monotonisuudesta (vrt.\ Luku \ref{desimaaliluvut}) seuraa, että myös
koko jono $\{x_n\}$ suppenee. Nimittäin jos esim.\ $\,x > 0$, niin jono $\seq{x_n}$ on kasvava,
jolloin jokaisella $n>k$ pätee
\[
x_{k+lm}=y_l\ \le\ x_n\ \le\ y_{l+1}=x_{k+(l+1)m},
\]
missä $l\in\N\cup\{0\}$ on valittu ($n$:stä riippuen) siten, että
\[
k + lm\ <\ n\ \le\ k + (l+1)m.
\]
Kun tässä $n\kohti\infty$, niin $l\kohti\infty$, jolloin $y_l \kohti x$ ja $y_{l+1} \kohti x$
ja näin ollen $x_n \kohti x$ (vrt.\ Lause \ref{jonotuloksia}, väittämä V2).
Samastussopimuksen \ref{samastus QD} mukaisesti on siis todistettu (ks.\ myös Lause 
\ref{raja-arvon yksikäsitteisyys})
\begin{Lause} \label{samastuslause QD} Jokainen jaksollinen desimaaliluku samastuu
Määritelmän \ref{samastus QD} mukaisesti yksikäsitteiseen rationaalilukuun. 
\end{Lause}
\begin{Exa}
\[
5.5027027027\ldots\ =\ 5.5 + 10^{-1}\cdot\frac{0.027}{1-10^{-3}}\ 
                    =\ \frac{55}{10}+\frac{27}{9990}\ =\ \frac{1018}{185}\,. \loppu
\]
\end{Exa}
\begin{Exa} \label{samastuksia} Mitkä desimaaliluvut samastuvat lukuihin $0$, $10$ ja $ 3/125$ ? 
\end{Exa}
\ratk Vaihtoehtoja on kaikissa tapauksissa kaksi:
\begin{align*}
0\     &=\ +0.0000000 \ldots\ =\ -0.0000000 \ldots \\
10\    &=\ +10.000000 \ldots\ =\ +9.9999999 \ldots \\
3/125\ &=\ +0.0080000 \ldots\ =\ +0.0079999 \ldots \qquad \loppu
\end{align*}

\subsection{Desimaalilukujen samastus}

Esimerkin \ref{samastuksia} mukaan on mahdollista, että kaksi merkkijonoina erilaista
(eli ei-identtistä) desimaalilukua $\x,\y\in\DD$ samastuu samaan rationaalilukuun $x$. Tällöin
on luonnollista samastaa $\x$ ja $\y$ myös keskenään, eli kirjoittaa $\x=\y$. Jos $\x=\seq{x_n}$
ja $\y=\seq{y_n}$, niin Määritelmän \ref{samastus QD} mukaan mainitussa tilanteessa on
$\,\lim_n x_n = \lim_n y_n = x$, jolloin Lauseen \ref{raja-arvojen yhdistelysäännöt} mukaan on
$\lim_n (x_n-y_n) = 0$. Otetaan tämä raja-arvoehto yleiseksi samastuskriteeriksi verrattaessa 
äärettömiä desimaalilukuja suoraan keskenään:
\begin{Def} \label{samastus DD} \index{samastus '$=$'!e@desimaalilukujen|emph} Kaksi
desimaalilukua $\x = \seq{x_n} \in \DD$ ja $\y = \seq{y_n} \in \DD$ samastetaan kriteerillä
\[
\x = \y \qekv \lim_n\,(x_n - y_n) = 0.
\] 
\end{Def}

Kahden 'eri näköisen' desimaaliluvun samastuminen on siis mahdollista ainakin, jos molemmat
samastuvat äärelliseen desimaalilukuun. --- Tämä osoittautuu ainoaksi erikoistapaukseksi:
\begin{Lause} \label{samastuslause DD} Jos desimaaliluvut $\x=\seq{x_n}\in\DD$ ja
$\y=\seq{y_n}\in\DD$ samastuvat keskenään Määritelmän \ref{samastus DD} mukaisesti, niin joko
$\x$ ja $\y$ ovat merkkijonoina identtiset tai $\x=x=\y$, missä $x$ on äärellinen desimaaliluku. 
\end{Lause}
\tod Jos $\seq{x_n}=\seq{0}$, niin myös $\seq{y_n}=\seq{0}$, sillä muuten
$\abs{x_n-y_n} = \abs{y_n} \not\kohti 0$, koska jono $\seq{\abs{y_n}}$ on kasvava. Tällöin on
siis $\x,\y=\pm 0.000 \ldots\,$ ja $\x=0=\y$. Jatkossa oletetetaan, että 
$\seq{x_n} \neq \seq{0}$ ja $\seq{y_n} \neq \seq{0}$. Tällöin on $\x$:n ja $\y$:n etumerkkien
oltava samat, muuten $\abs{x_n-y_n}=\abs{x_n}+\abs{y_n} \not\kohti 0$. Koska edelleen $\x$ ja
$\y$ voidaan kumpikin skaalata tekijällä $\pm 10^m,\ m \in \Z$ samastuskriteerin muuttumatta,
niin voidaan olettaa, että $\,0 \le x_n < 1\,$ ja $\,0 \le y_n <1\,$ jokaisella $n$, jolloin
$\x$ ja $\y$ ovat jollakin $k\in\N$ muotoa
\begin{align*}
\x\ &=\ 0.d_1 \ldots d_{k-1} d_k \ldots\ldots d_m \ldots\ldots, \\
\y\ &=\ 0.d_1 \ldots d_{k-1} \tilde{d}_k \ldots\ldots \tilde{d}_m \ldots\ldots,
\end{align*}
missä $\tilde{d}_k \neq d_k$. Tällöin on $\,x_n=y_n\,$ kun $n<k$ ja 
$\,x_k-y_k=(d_k-\tilde{d}_k)\,10^{-k}$. Yleisyyttä rajoittamatta voidaan edelleen olettaa, että 
$\tilde{d}_k < d_k$ ($d_k \neq 0$). Tällöin nähdään, että erotus $\,x_n-y_n$ saa pienimmän
mahdollisen arvonsa jokaisella $n>k$, kun $d_n=0,\,\tilde{d}_n=9\ \forall n>k$, eli kun
$\x$ ja $\y$ ovat muotoa
\begin{align*}
\x\ &=\ 0.d_1 \ldots d_{k-1} d_k 0 0 0 \ldots \\
\y\ &=\ 0.d_1 \ldots d_{k-1} \tilde{d}_k 9 9 9 \ldots
\end{align*} 
Tässä tapauksessa on (vrt.\ Propositio \ref{desim})
\begin{align*} \label{I-7: välitulos}
x_n-y_n &= \left(x_k + 0 \cdot 10^{-k-1} + \ldots + 0 \cdot 10^{-n}\right)
          -\left(y_k + 9 \cdot 10^{-k-1} + \ldots + 9 \cdot 10^{-n}\right) \\
        &= (x_k-y_k) - \left(10^{-k}-10^{-n}\right) \\
        &= (d_k-\tilde{d}_k-1)\,10^{-k} + 10^{-n}, \quad n>k.
\end{align*}
Siis erotuksen $x_n-y_n$ pienin mahdollinen arvo on $10^{-n}$ kun $n \ge k$, ja tämä
saavutetaan, kun $d_k=\tilde{d}_k+1$ ja $d_n=0$ ja $\tilde{d}_n=9$ jokaisella $n>k$. Tässä
erikoistapauksessa on siis $x_n=x_k\ \forall n \ge k$ ja $\lim_n y_n=x_k$, jolloin 
$\,\x=x_k=\y\,$ Määritelmän \ref{samastus QD} mukaisesti.

Muissa kuin em.\ erikoistapauksessa on oltava joko $d_k-\tilde{d}_k \ge 2$ (tapaus 1),
$d_m \ge 1$ jollakin $m>k$ (tapaus 2) tai $\tilde{d}_m \le 8$ jollakin $m>k$ (tapaus 3). 
Tapauksessa 1 seuraa em.\ tuloksesta, että $x_n-y_n > 10^{-k}=\eps>0\ \forall n>k$, jolloin 
$x_n-y_n \not\kohti 0$. Tapauksissa 2 ja 3 on oltava $x_m-y_m \ge 2 \cdot 10^{-m}$
(koska erotuksen pienin arvo oli $10^{-m}$). Tällöin koska jono $\seq{x_n}$ on kasvava
ja koska $y_n < y_m + 10^{-m}\ \forall n>m$ (Propositio \ref{desim}), niin seuraa, että
$x_n-y_n > 10^{-m}=\eps>0$, kun $n>m$. Siis myös tässä tapauksessa $x_n-y_n \not\kohti 0$.
On päätelty, että jos $\x$ ja $\y$ eivät ole merkkijonoina identtiset, niin oletus
$\lim_n(x_n-y_n)=0$ toteutuu vain edellä kuvatussa erikoistapauksessa, jossa $\x$ ja $\y$ 
samastuvat äärelliseen desimaalilukuun. \loppu

Yhteenvetona rationaalilukujen ja äärettömien desimaalilukujen samastuksesta voidaan todeta:
\begin{Kor} \label{rat ja desim samastus} Jokainen rationaaliluku $\,x\,$ samastuu vähintään
yhteen ja enintään kahteen äärettömään desimaalilukuun, jotka ovat jaksollisia.
Samastusvaihtoehtoja on kaksi täsmälleen kun $x$ on äärellinen desimaaliluku. \end{Kor}
\begin{Exa} Mihin äärettömiin binaarilukuihin samastuu kymmenjärjestelmän luku \ 
a) $x = 3/4$, \ b) $y=4/3$\,? 
\end{Exa}
\ratk a) Koska $\,3/4 = 2^{-2}(1 \cdot 2 + 1) = 1 \cdot 2^{-1} + 1 \cdot 2^{-2}$, niin
\[
x\ =\ 0.11000000 \ldots \ =\ 0.10111111 \ldots
\]
Vaihtoehtoja on kaksi, koska $x$ on äärellinen binaariluku.
 
b) Koska
\begin{align*}
\frac{4}{3}\ =\ \frac{1}{1-\frac{1}{4}}\
             =\ 1+\frac{1}{4}+\frac{1}{4^2}+\ldots\ 
             =\ 1 \cdot 2^0 + 0 \cdot 2^{-1} +1 \cdot 2^{-2} + 0 \cdot 2^{-3} + \ldots\,,
\end{align*}
niin $\,y=1.010101\ldots\,$ Tässä on vain yksi vaihtoehto. \loppu

\Harj
\begin{enumerate}

\item \label{H-I-7: todistuksia}
Todista Lauseen \ref{raja-arvojen yhdistelysäännöt} väittämät (1) ja (2).

\item
Määritä raja-arvojen yhdistelysääntöjä hyväksi käyttäen
\begin{align*}
&\text{a)}\ \ \lim_n \frac{2n^3-100n^2-5000n}{(n+200)^3} \qquad\quad
 \text{b)}\ \ \lim_n \frac{1}{n^3}\left[(n+3)^4-(n-2)^4\right] \\
&\text{c)}\ \ \lim_n \left(\frac{1+2+ \ldots +n}{n+2}-\frac{n}{2}\right) \qquad 
 \text{d)}\ \ \lim_n \left[\frac{(n+2)!-n!}{(n+1)!+(n-1)!}-n\right]
\end{align*}

\item 
Laske lukujonojen (sikäli kuin \kor{luku}jonoja)
\[
a_n=\frac{x^n}{1+x^n}\,, \quad b_n=\frac{x^n}{1+x^{2n}}\,, \quad n=1,2, \ldots\,, 
\]
raja-arvot kaikilla $x$:n rationaalisilla arvoilla.

\item \label{H-I-7: sqrt-kunta}
Määritä seuraavien palautuvien lukujonojen mahdolliset raja-arvot olettaen, että
jonot ovat suppenevia:
\begin{align*}
&\text{a)}\ \ a_{n+1}=-\frac{1}{3}\,a_n+\frac{2n^2}{n^2+1} \qquad\qquad\
 \text{b)}\ \ a_{n+1} = \frac{na_n-8n+3}{2n+5} \\
&\text{c)}\ \ a_{n+2} = \frac{2n}{3n+4} - a_{n+1} - \frac{1}{4}\,a_n \qquad
 \text{d)}\ \ a_{n+1} = \frac{1}{2+a_n} \\[2mm]
&\text{e)}\ \ a_{n+1} = \sqrt{a_n+2} \qquad\qquad\qquad\quad\,\ \
 \text{f)}\ \ a_{n+1} = \sqrt[3]{a_n+6} \\[2mm]
&\text{g)}\ \ a_{n+1} = \frac{3}{4-a_n^2} \qquad\qquad\qquad\qquad\
 \text{h)}\ \ a_{n+1} = \frac{1}{\sqrt{4-a_n^2}}
\end{align*}

\item 
Olkoon $a_n+a \neq 0\ \forall n$ ja $\,\lim_n (a_n+a)^{-1} = b \neq 0$. Näytä, että jono 
$\seq{a_n}$ suppenee ja määritä $\lim_n a_n$.

\item \label{H-I-7: jonotulos 3}
a) Todista Lauseen \ref{jonotuloksia} väittämä V3. \vspace{1mm}\newline
b) Olkoon $\seq{a_n}$ rajoitettu jono. Todista, että 
   $\displaystyle{\,\lim_n \frac{2n+a_n}{n}=2}$.

\item 
a) Mitkä seuraavista luvuista ovat eri suuria ja mitkä samoja\,? \newline
$1.285714285714285714..\,$, $\frac{11}{9}\,$, $\frac{1287}{1250}\,$, $\frac{9}{7}\,$, 
$1.029599999..\,$, $1.02949999..\,$, \newline
$1.029600000..\,$, $a=$ yksi kokonaista kaksi yhdeksäsosaa. \vspace{2mm}\newline
b) Mitkä seuraavista desimaaliluvuista ovat kaksittain verrattaessa varmasti eri suuret ja mitkä
mahdollisesti samat\,? \newline
$\x=2.41789..,\ \ \y=2.41799..,\ \ \breve{a}=2.41788..,\ \ \breve{b}=2.41798.., \ \ 
 \breve{c}=2.4179..$

\item
Määritä rationaaliluku $x$ siten, että $x=\x$, kun \newline
a) \ $\x=-0.46127127127 \ldots\,$, \ b) \ $\x=2015.201520152015 \ldots$

\item
Eräässä lukujärjestelmässä, jossa kirjoitetaan $1+1=2$, muodostavat äärettömien desimalilukujen
vastineet joukon $\mathbb L$. Tässä joukossa on luvulla $\x=0.222\ldots \in \mathbb L$ toinenkin
esitysmuoto. Mikä on ko.\ lukujärjestelmän kantaluku, ja mihin rationaalilukuun $\x$ samastuu?

\item 
Olkoon $x,y\in\Q$ ja $\x,\y\in\DD$. Lähtien Määritelmistä \ref{samastus QD} ja \ref{samastus DD}
ja käyttäen ainoastaan raja-arvojen yhdistelysääntöjä 
(Lause \ref{raja-arvojen yhdistelysäännöt}) näytä, että pätee \
a) \ $\x=x\ \ja\ x=\y\ \,\impl\,\ \x=\y$, \ \ b) \  $\x=\y\ \ja\ \y=y \,\impl\,\ \x=y$.

\item (*)
Todista: \ 
$\displaystyle{\lim_n a_n = a\,\ \impl\,\ \lim_n \frac{a_1+a_2+ \ldots + a_n}{n} = a}$.

\item (*)
Määritellään palautuva lukujono
\[
a_1\in\Q, \quad a_{n+1}=\frac{a_n}{1+10^{-n}a_n}\,, \quad n=1,2, \ldots
\]
Näytä, että $\seq{a_n}$ suppenee ja
\[
\lim_n a_n = \frac{9a_1}{a_1+9}\,, 
\]
paitsi eräillä poikkeuksillisilla $a_1$:n arvoilla --- millä\,? \newline
\kor{Vihje}: Tutki jonoa $\seq{b_n}=\seq{a_n^{-1}}$\,.

\item (*)
\ a) Määritä palautuvan lukujonon
\[
a_{n+1}=(2-a_n)^2, \quad n=0,1, \ldots
\]
mahdolliset raja-arvot $\,a=\lim_n a_n$.\, b) Tutkimalla jonoa $\seq{b_n}=\seq{a_n-a}$ päättele,
että $\,\lim_n a_n = a$ vain kun $a_n=a$ jollakin $n$.\, c) Päättele, että jos $a_0\in\Q$, niin
$\seq{a_n}$ suppenee täsmälleen, kun $a_0$ on jokin luvuista $0,1,2,3,4$.

\end{enumerate}  % Suppenevat lukujonot
\section{Monotoniset ja rajoitetut lukujonot}
\label{monotoniset jonot}
\alku
\index{lukujono!c@monotoninen|vahv} \index{lukujono!e@rajoitettu|vahv}
\index{monotoninen!a@lukujono|vahv} \index{rajoitettu!a@lukujono|vahv}

Tässä luvussa tarkastelun kohteena ovat lukujonot, jotka ovat sekä monotonisia (kasvavia tai 
väheneviä, Määritelmä \ref{monotoninen jono}) että rajoitettuja (Määritelmä 
\ref{rajoitettu jono}). Tällaiset lukujonot ovat käytännössä yleisiä, vaikka ne muodostavatkin
vain pienen 'vähemmistön' kaikista lukujonoista. Tämän luvun päätulos on, että monotoniselle
ja rajoitetulle jonolle löydetään aina raja-arvo --- kunhan raja-arvo ja suppeneminen sitä kohti
määritellään sopivasti.
\begin{Exa} \label{desimaaliluku on rajoitettu jono} Jokainen ääretön desimaaliluku 
$\x=\seq{x_n}\in\DD$ on lukujonona monotoninen, kuten aiemmin on todettu 
(ks.\ Luku \ref{desimaaliluvut}). Jono $\seq{x_n}$ on myös rajoitettu, sillä 
$\abs{x_n} \le C = \abs{x_0}+1\ \forall n$ (Propsitio \ref{desim}). \loppu
\end{Exa}
\begin{Exa} \label{kaksi sarjaa} Näytä, että sarja \ a) $\sum_{k=0}^\infty 1/k!\,$, \ 
b) $\sum_{k=1}^\infty 1/k^2\,$ on lukujonona monotoninen ja rajoitettu. \end{Exa}
\ratk Kummassakin tapauksessa kyseeessä on
\index{sarja!b@positiiviterminen}%
\kor{positiiviterminen} sarja eli sarja muotoa $\sum_k a_k$, missä $a_k>0\ \forall k$.
Positiiviterminen sarja on (osasummiensa) lukujonona aidostikin kasvava, joten riittää
osoittaa, että jonot
\begin{align*}
&\text{a)}\,\ \seq{s_n}\ =\ \left\{\,\sum_{k=0}^n \frac{1}{k!}\,,\ \ n=0,1,2,\,\ldots\,\right\}\
                         =\ \left\{\,1,2,\frac{5}{2}\,,\,\ldots\,\right\}, \\
&\text{b)}\,\ \seq{s_n}\ =\ \left\{\,\sum_{k=1}^n \frac{1}{k^2}\,,\ \ n=1,2,3,\,\ldots\,\right\}\
                         =\ \left\{\,1,\frac{5}{4},\frac{23}{18}\,,\,\ldots\,\right\}
\end{align*}
ovat rajoitettuja. \ a) Kun $n \ge 4$, voidaan arvioida
\begin{align*}
s_n\  =\ \sum_{k=0}^n \frac{1}{k!}\ 
     &=\ 1 + 1 + \frac{1}{2!}\left(1+\frac{1}{3}+\frac{1}{3\cdot 4}+\,\ldots\,
                                                +\frac{1}{3\cdot 4 \cdots n}\,\right) \\
     &\le\ 2 + \frac{1}{2}\left(1+\frac{1}{3}+\frac{1}{3^2}+\,\ldots\,
                                               +\frac{1}{3^{n-2}}\,\right) \\[1mm]
     &<\ 2 + \frac{1}{2}\cdot\frac{1}{1-\frac{1}{3}}\ =\ 2.75.
\end{align*}
Siis $0 < s_n < 2.75\ \forall n$, joten $\seq{s_n}$ on rajoitettu.

b) Tässä tarvitaan hiukan enemmän kekseliäisyyttä. Esimerkiksi: Koska
\[
\frac{1}{k^2}\ <\ \frac{1}{(k-1)k}\ =\ \frac{1}{k-1} - \frac{1}{k}, \quad k \ge 2,
\]
niin osasummia voidaan arvioida teleskooppisummien 
(vrt.\ Harj.teht. \ref{ratluvut}:\ref{H-I-1: teleskooppisumma}) avulla seuraavasti:
\begin{align*}
s_n\ &=\ 1 + \sum_{k=2}^n \frac{1}{k^2}\ 
      <\ 1 + \sum_{k=2}^n \left( \frac{1}{k-1} - \frac{1}{k} \right) \\
     &=\ 1 + \left(1 - \frac{1}{2}\right) 
           + \left(\frac{1}{2} - \frac{1}{3}\right) + \ldots
           + \left(\frac{1}{n-1} - \frac{1}{n}\right)         \\
     &=\ 1 + 1 + \left(- \frac{1}{2} + \frac{1}{2}\right) + \ldots 
      + \left(- \frac{1}{n-1} + \frac{1}{n-1}\right) - \frac{1}{n} \\
     &=\ 2 - \frac{1}{n}\ <\ 2, \quad n \ge 3.
\end{align*}
Siis $0 < s_n < 2\ \forall n$, joten $\seq{s_n}$ on rajoitettu. \loppu

\subsection{Suppeneminen kohti desimaalilukua}
\index{lukujono!d@suppeneva|vahv}

Esimerkin \ref{kaksi sarjaa} sarjoilla ei kummallakaan ole raja-arvoa toistaiseksi 
tunnettujen lukujen joukossa, kuten tullaan näkemään. Silti nämä sarjat 
'näyttävät suppenevan', kun sarjojen (rationaaliset) osasummat muunnetaan äärettömiksi 
desimaaliluvuiksi. 
\jatko \begin{Exa} (jatko) a) Osasummat $\,s_1 \ldots s_{12}\,$ ovat
\[
\begin{aligned}
s_1         &= 2 \quad\               =\ 2.0000000000 \ldots \\
s_2         &= \tfrac{5}{2} \quad\    =\ 2.5000000000 \ldots \\
s_3         &= \tfrac{8}{3} \quad\    =\ 2.6666666666 \ldots \\
s_4         &= \tfrac{65}{24} \ \ \   =\ 2.7083333333 \ldots \\
s_5         &= \tfrac{163}{60} \ \    =\ 2.7166666666 \ldots \\
s_6         &= \tfrac{1957}{720} \    =\ 2.7180555555 \ldots \\
\end{aligned} \qquad\quad
\begin{aligned}
s_7\        &= \tfrac{685}{252} \qquad\        =\ 2.7182539682 \ldots \\
s_8\        &= \tfrac{109601}{40320} \quad\    =\ 2.7182787698 \ldots \\
s_9\        &= \tfrac{98641}{36288} \quad\ \   =\ 2.7182815255 \ldots \\
s_{10}      &= \tfrac{9864101}{3628800} \quad  =\ 2.7182818011 \ldots \\
s_{11}      &= \tfrac{13563139}{4989600} \ \   =\ 2.7182818261 \ldots \\
s_{12}      &= \tfrac{260412269}{95800320}\    =\ 2.7182818282 \ldots
\end{aligned}
\]

b) Osasummat indeksin arvoilla $\,n=10^k,\ k=1 \ldots 12\,$ ovat
\[
\begin{aligned}
s_{10}       &=\ 1.5497677311 \ldots \\
s_{100}      &=\ 1.6349839001 \ldots \\
s_{1000}     &=\ 1.6439345666 \ldots \\
s_{10000}    &=\ 1.6448340718 \ldots \\
s_{100000}   &=\ 1.6449240668 \ldots \\
s_{1000000}  &=\ 1.6449330668 \ldots
\end{aligned} \qquad\quad
\begin{aligned}
s_{10000000}      &=\ 1.6449339668 \ldots \\
s_{100000000}     &=\ 1.6449340568 \ldots \\
s_{1000000000}    &=\ 1.6449340658 \ldots \\
s_{10000000000}   &=\ 1.6449340667 \ldots \\
s_{100000000000}  &=\ 1.6449340668 \ldots \\
s_{1000000000000} &=\ 1.6449340668 \ldots \loppu
\end{aligned}
\]
\end{Exa}
Osasummien perusteella näyttää siltä, että raja-arvo $s=\lim_n s_n$ (eli sarjan summa) on
esimerkkitapauksessa mahdollista tulkita äärettömäksi desimaaliluvuksi:
\[
\text{a)}\ \ \sum_{k=0}^\infty \frac{1}{k!}\, = 2.71828182 \ldots \qquad
\text{b)}\ \ \sum_{k=1}^\infty \frac{1}{k^2}  = 1.64493406 \ldots
\]
Mutta mitä täsmällisemmin tarkoittaa, että ääretön desimaaliluku on lukujonon raja-arvo?
--- Alkuperäiseen raja-arvon määritelmään (Määritelmä \ref{jonon raja}) ei voida suoraan
vedota, koska määritelmän ehdossa $\abs{a_n-a}<\eps$ tarvittavaa vähennyslaskua ja lukujen
vertailua ei ole (ainakaan toistaiseksi) määritelty, kun $a\in\DD$. Ratkaistaan ongelma
asettamalla erillinen määritelmä.
\begin{Def} \label{jonon raja - desim} \index{lukujonon raja-arvo|emph}
\index{raja-arvo!a@lukujonon|emph} \index{suppeneminen!a@lukujonon|emph}
\index{hajaantuminen!a@lukujonon|emph} Lukujono $\seq{a_n}$ \kor{suppenee kohti} ääretöntä
desimaalilukua $\x=\seq{x_n}\in\DD$ täsmälleen kun
\[
\lim_n (a_n-x_n) = 0.
\]
Sanotaan tällöin, että $\x$ on jonon $\seq{a_n}$ \kor{raja-arvo}, ja merkitään 
$\,\lim_n a_n = \x$ tai $a_n \kohti \x$. Jos lukujonolla on raja-arvo $\x\in\DD$, niin
sanotaan, että lukujono \kor{suppenee}, muulloin lukujono \kor{hajaantuu} eli
\kor{divergoi}.
\end{Def}
Huomattakoon, että ehdossa $\lim_n(a_n-x_n)=0$ on kyse lukujonon $\seq{a_n-x_n}$ 
tavanomaisesta (rationaalisesta) raja-arvosta Määritelmän \ref{jonon raja} mukaisesti. 
Jos myös jonolla $\seq{a_n}$ on rationaalinen raja-arvo, niin Määritelmät
\ref{jonon raja - desim} ja \ref{jonon raja}  ovat sopusoinnussa, sillä jos 
$a_n \kohti x\in\Q$ (Määritelmä \ref{jonon raja}) ja $x=\x=\seq{x_n}\in\DD$
(Määritelmä \ref{samastus QD}), niin $a_n-x_n = (a_n-x)+(x-x_n) \kohti 0+0=0$
(Lause \ref{raja-arvojen yhdistelysäännöt}), eli $a_n \kohti \x$ Määritelmän 
\ref{jonon raja - desim} mukaisesti. Myös on voimassa (vrt.\ Lauseet 
\ref{suppeneva jono on rajoitettu} ja \ref{raja-arvon yksikäsitteisyys})
\begin{Lause} \label{raja-arvon yksikäsitteisyys - desim} Jos lukujono $\seq{a_n}$ suppenee
Määritelmän \ref{jonon raja - desim} mukaisesti, niin \newline
(i) $\seq{a_n}$ on rajoitettu lukujono, ja (ii) raja-arvo $\lim_n a_n$ on yksikäsitteinen.
\end{Lause}
\tod (i) Kirjoittamalla $\,a_n=(a_n-x_n)+x_n\,$ ja soveltamalla kolmioepäyhtälöä päätellään
\[
\abs{a_n} \,\le\, \abs{a_n-x_n}+\abs{x_n} \,\le\, C_1+C_2=C\ \forall n,
\]
sillä $\seq{a_n-x_n}$ on rajoitettu suppenevana jonona (Lause 
\ref{suppeneva jono on rajoitettu}) ja myös $\seq{x_n}$ on rajoitettu
(ks.\ Esimerkki \ref{desimaaliluku on rajoitettu jono} edellä). Siis $\seq{a_n}$ on
rajoitettu jono. 

(ii) Oletetaan, että $a_n \kohti \x = \seq{x_n}\in\DD$ ja $a_n \kohti \y = \seq{y_n}\in\DD$.
Tällöin $a_n-x_n \kohti 0$ ja $a_n-y_n \kohti 0$ (Määritelmä \ref{jonon raja - desim}),
jolloin Lauseen \ref{raja-arvojen yhdistelysäännöt} perusteella
\[
x_n-y_n\ =\ (a_n-y_n)-(a_n-x_n) \,\kohti\, 0.
\]
Siis $\,\lim_n(x_n-y_n)=0$, mikä Määritelmän \ref{samastus DD} mukaan tarkoittaa: $\,\x=\y$.
\loppu
\begin{Exa} Määritelmän \ref{jonon raja - desim} mukaisesti jokainen ääretön desimaaliluku
$\x=\seq{x_n}\,$ 'suppenee itseensä' lukujonona $\seq{a_n}=\seq{x_n}$ (!). \loppu
\end{Exa}

Jatkossa keskeinen kysymys on: Jos lukujonolle $\seq{a_n}$ ei ole osoitettavissa mitään
ilmeistä (esim.\ rationaalista) raja-arvoa, niin millaisilla ehdoilla voidaan olla varmoja,
että Määritelmän \ref{jonon raja - desim} mukainen raja-arvo äärettömänä desimaalilukuna on
olemassa? Täsmällinen vastaus tähän kysymykseen saadaan jäljempänä Luvussa \ref{Cauchyn jonot};
tässä yhteydessä tyydytään osittaiseen vastaukseen. Ensinnäkin, Lauseen
\ref{raja-arvon yksikäsitteisyys - desim} mukaan \pain{välttämätön} ehto raja-arvon
olemassaololle on, että $\seq{a_n}$ on rajoitettu. Seuraavan lauseen mukaan \pain{riittävä}
ehto on, että $\seq{a_n}$ sekä monotoninen että rajoitettu. 
\begin{*Lause}\footnote[2]{Tässä ja jatkossa merkintä (*) lauseen yhteydessä tarkoittaa,
että lauseen todistus on tavallista vaativampi looginen konstruktio.} 
\label{monotoninen ja rajoitettu jono} (\vahv{Monotoninen ja rajoitettu lukujono suppenee})
Jokaisella monotonisella ja rajoitetulla lukujonoilla on Määritelmän
\ref{jonon raja - desim} mukainen raja-arvo äärettömänä desimaalilukuna.\footnote[3]{Lause
\ref{monotoninen ja rajoitettu jono} on esimerkki olemassaolo- eli \kor{eksistenssilauseesta},
jossa jokin (tässä raja-arvo) väitetään olemassa olevaksi. --- Paino lauseessa onkin sanalla
'on'. \index{eksistenssilause|av}}
\end{*Lause}
\tod Määritelmästä \ref{jonon raja - desim} nähdään, että jos $a_n \kohti \x$, niin
$-a_n \kohti -\x$. Voidaan sen vuoksi rajoittua tapauksiin, joissa joko (i) $\seq{a_n}$ on
kasvava ja jollakin $N\in\N$ pätee $a_n \ge 0\ \forall n \ge N$ tai (ii) $\seq{a_n}$ on
vähenevä ja $a_n \ge 0\ \forall n\in\N$, sillä muissa tapauksissa jono $\seq{-a_n}$ on tyyppiä
(i) tai (ii). Olkoon ensin jono tyyppiä (i), ja määritellään jonoon liittyen predikaatti
(vrt.\ Luku \ref{logiikka})
\[
Q(x): \quad a_n < x \ \ \forall n \quad (x \in \Q).
\]
Koska $\seq{a_n}$ on rajoitettu, on löydettävissä $M \in \N$ siten, että $\,a_n < M\ \forall n$.
Oletuksien (i) perusteella on silloin $\,0 \le a_n < M\ \forall n \ge N$. Tällöin voidaan valita
yksikäsitteisesti $x_0 \in \{0,\ldots,M-1\}$ siten, että $Q(x_0)$ on epätosi ja $Q(x_0 + 1)$ on 
tosi. Koska $Q(x_0)$ on epätosi, on $a_n \ge x_0$ jollakin $n$. Olkoon $n=N_0$ pienin tällainen 
indeksi. Tällöin koska jono $\seq{a_n}$ on kasvava ja koska $Q(x_0+1)$ on tosi, on 
$\,x_0 \le a_n < x_0 + 1\ \forall n \ge N_0$.

Valitaan seuraavaksi $x_1 = x_0 + d_1 \cdot 10^{-1},\ d_1 \in \{0,1, \ldots, 9\}$ siten, että
$Q(x_1)$ on epätosi ja $Q(x_1 + 10^{-1})$ on tosi. Koska $Q(x_1)$ on epätosi, on $a_n \ge x_1$
jollakin $n$. Olkoon $n=N_1$ pienin tällainen indeksi ($N_1 \ge N_0$). Tällöin koska jono 
$\seq{a_n}$ on kasvava ja koska $Q(x_1 + 10^{-1})$ on tosi, on 
$\,x_1 \le a_n < x_1 + 10^{-1}\ \forall n \ge N_1$.

Jatkamalla samalla tavoin saadaan konstruoiduksi kasvava jono $\seq{x_k}$ ja kasvava indeksijono
$\{N_0,N_1, \ldots\,\}$ siten, että jokaisella $k$ pätee
\begin{equation}  \label{suppenemisehto - desim}
x_k \le a_n < x_k + 10^{-k}, \quad \text{kun}\ n \ge N_k. \tag{$\star$}
\end{equation}
Tapauksessa (ii) tullaan samaan tulokseen, kun predikaatti $Q(x)$ määritellään
\[
Q(x): \quad a_n < x \ \ \text{jollakin}\ n \quad (x \in \Q).
\]

Kummassakin tapauksessa (i) ja (ii) on siis konstruoitu ääretön desimaaliluku 
$\x=\{x_k,\ k=0,1,\ldots\,\}$ ja indeksijono $\{N_k,\ k=0,1,\ldots\,\}$ siten, että pätee
\[
\abs{a_n-x_k} < 10^{-k}, \quad \text{kun}\ n \ge N_k.
\]
Kun huomioidaan, että jonolle $\seq{x_k}$ pätee (Propositio \ref{desim})
\[
\abs{x_n-x_k} < 10^{-k}, \quad \text{kun}\ n \ge k,
\]
niin kirjoittamalla $a_n-x_n = (a_n-x_k)+(x_k-x_n)$ ja soveltamalla kolmioepäyhtälöä seuraa
kahden viimeksi kirjoitetun epäyhtälön perusteella
\[
\abs{a_n-x_n} \,\le\, \abs{a_n-x_k}+\abs{x_n-x_k} \,<\, 2 \cdot 10^{-k}, \quad
                \text{kun}\ n \ge \max\{k,N_k\} = N_k'.
\]
Proposition \ref{jonon raja 2} mukaan tämä ehto on sama kuin ehto: $\,\lim_n(a_n-x_n)=0$. Siis
$\lim_n a_n = \x = \seq{x_n}$ Määritelmän \ref{jonon raja - desim} mukaisesti. \loppu

Kun lukujono $\seq{a_n}$ tunnetaan, niin Lauseen \ref{monotoninen ja rajoitettu jono}
todistuskonstruktiota voidaan seurata algoritmina, joka määrittää raja-arvon
alkaen kokonaislukuosasta ja edeten desimaali kerrallaan.
\jatko\jatko \begin{Exa} (jatko) Raja-arvo $\lim_n s_n = \seq{x_k}\in\DD$ määrätään
valitsemalla luvut $x_k\in\Q_k,\ k=0,1,\ldots\,$ siten, että $s_n \ge x_k$ jollakin $n$
ja $s_n < x_k + 10^{-k}\ \forall n$, jolloin ehto \eqref{suppenemisehto - desim} toteutuu 
jollakin $N_k\in\N$ (jokaisella $k$), kun $a_n=s_n$.
 
a) Koska $s_n < 3\ \forall n$ ja $s_1=2\ \impl\ s_n \ge 2\ \forall n \ge 1$, niin $x_0=2$ ja
pienin $N_0$:n arvo, jolla ehto \eqref{suppenemisehto - desim} toteutuu kun $k=0$, on $N_0=1$.
Vastaavasti koska $s_3 < 2.7$, $s_4 > 2.7$ ja ilmeisesti $s_n < 2.8\ \forall n$ (tämä on 
erikseen varmistettava osasummia arvioimalla, ks.\ Harj.teht.\,\ref{H-I-8: kaksi sarjaa}),
niin $x_1=2.7$ ja indeksin $N_1$ pienin arvo on $N_1=4$. Samalla tavoin nähdään lasketuista
osasummista, että $x_2=2.71,\,x_3=2.718,\,\ldots,\,x_8=2.71828182$ ja että indeksien 
$N_k,\ k=2 \ldots 8$ pienimmät arvot ehdossa \eqref{suppenemisehto - desim} ovat $N_2=5$,
$N_3=6$, $N_4=7$, $N_5=N_6=9$, $N_7=10$ ja $N_8=11$.

b) Osasummista
\[
\begin{aligned}
s_{21}  &=\ 1.59843081 \ldots \\
s_{22}  &=\ 1.60049693 \ldots \\[2mm]
s_{202} &=\ 1.63999580 \ldots \\
s_{203} &=\ 1.64002007 \ldots
\end{aligned} \qquad\quad
\begin{aligned}
s_{1070}  &=\ 1.64399992 \ldots \\
s_{1071}  &=\ 1.64400079 \ldots \\[2mm]  
s_{29353} &=\ 1.64489999 \ldots \\
s_{29354} &=\ 1.64490000 \ldots
\end{aligned}
\]
on pääteltävissä (ja varmistettavissa, ks.\ Harj.teht.\,\ref{H-I-8: kaksi sarjaa}), että
pienimmät indeksin $N_k$ arvot, joilla ehdot \eqref{suppenemisehto - desim} toteutuvat 
kun $k=1 \ldots 4\,$, ovat $N_1=22$, $N_2=203$, $N_3=1071$ ja $N_4=29354$. \loppu
\end{Exa} \seur
\begin{Exa} \label{sqrt 2 algoritmina} Luvun \ref{jonon raja-arvo}
Esimerkissä \ref{sqrt 2} näytettiin, että palautuvalle rationaalilukujonolle
\[
a_0 = 2, \quad a_{n+1} = \frac{a_n}{2} + \frac{1}{a_n}\,, \quad n = 0,1,\ldots
\]
pätee $a_n > 0\ \forall n$ ja $a_n^2 > 2\ \forall n$. Palautuskaavasta ja mainituista
tuloksista seuraa myös, että
\[
a_{n+1}-a_n\ =\ \frac{a_n}{2}+\frac{1}{a_n}-a_n\ =\ \frac{1}{2a_n}\,(2-a_n^2)\ <\ 0,
\]
joten $\seq{a_n}$ on aidosti vähenevä ja myös rajoitettu: $\,0 < a_n \le 2\ \forall n$. Siis
$\seq{a_n}$ suppenee Määritelmän \ref{jonon raja - desim} mukaisesti kohti ääretöntä
desimaalilukua. Suppeneminen on itse asiassa hyvin nopeaa:
\begin{align*}
a_0\ &=\ 2.000 \ldots \\
a_1\ &=\ 1.5000 \ldots \\
a_2\ &=\ 1.41666 \ldots \\
a_3\ &=\ 1.41421568 \ldots \\
a_4\ &=\ 1.41421356237468 \ldots \\
a_5\ &=\ 1.41421356237309504880168962 \ldots \\
a_6\ &=\ 1.4142135623730950488016887242096980785696 \ldots \\
a_7\ &=\ 1.4142135623730950488016887242096980785696 \ldots \loppu
\end{align*}
\end{Exa}
Esimerkin lukujonosta tiedetään, että se suppenee myös Määritelmän \ref{jonon raja}
mukaisesti kohti lukua $\sqrt{2}$ (ks.\ Luku \ref{jonon raja-arvo}). Edellä todettiin myös,
että Määritelmien \ref{jonon raja} ja \ref{jonon raja - desim} mukaiset raja-arvot ovat samat,
jos raja-arvo on rationaalinen. On odotettavissa, että näin on laita yleisemminkin, jos
molemmat raja-arvot ovat olemassa. Näin olettaen (asia varmistuu seuraavassa luvussa) saa
abstrakti luku $\sqrt{2}$ konkreettisen 'ilmiasun' äärettömänä desimaalilukuna:
\[
\sqrt{2} =\ 1.4142135623730950488016887242096980785696 \ldots
\]
Algoritmina (haluttaessa laskea $\sqrt{2}\,$:n desimaaleja) esimerkin lukujono $\seq{a_n}$ on
nopeimpia tunnettuja.\footnote[2]{Esimerkin algoritmi tunnettiin jo muinaisessa Babyloniassa 
3000 vuotta sitten. Yalen yliopistossa säilytettävässä savitaulussa n:o 7289 on algoritmilla 
laskettu $\sqrt{2}\,$:n approksimaatio $x_3$ lähtien alkuarvosta $x_1$. Laskut on suoritettu
$60$-kantaisessa lukujärjestelmässä neljän merkitsevän numeron tarkkuudella, jolloin 
tulokseksi on saatu
\[
x_3^*\ =\  1 + 24 \cdot 60^{-1} + 51 \cdot 60^{-2} + 10 \cdot 60^{-3}\ =\ 1.41421296\,..
\]
Katkaisuvirheestä johtuen tuloksen virhe (noin $-6 \cdot 10^{-7}$) sattuu olemaan jopa 
pienempi kuin $x_3$:n tarkan arvon virhe (noin $+21 \cdot 10^{-7}$).} 

\pagebreak

\subsection{Neperin luku}
\index{Neperin luku|vahv}%

\begin{Prop} \label{Neperin jonot} Lukujonot 
\[
\seq{a_n}\ =\ \left\{\left(1 + \frac{1}{n}\right)^n,\ n = 1,2, \ldots\,\right\}, \quad 
\seq{b_n}\ =\ \left\{\left(1 - \frac{1}{n}\right)^n,\ n = 1,2, \ldots\,\right\} 
\]
ovat kasvavia ja rajoitettuja. Lisäksi $\ \lim_n a_n b_n = 1$. \end{Prop}
\tod Soveltamalla Bernoullin epäyhtälöä (Propositio \ref{Bernoulli}) saadaan
\begin{align*}
\frac{a_{n+1}}{a_n}\ &=\ \frac{\bigl(1 + \frac{1}{n+1}\bigr)^{n+1}}{(1 + \frac{1}{n})^n}\
                      =\ \biggl(1 + \frac{1}{n}\biggr) 
                         \Biggl( \frac{1 + \frac{1}{n+1}}{1 + \frac{1}{n}} \Biggr)^{n+1} \\
                     &=\ \biggl(1 + \frac{1}{n}\biggr) 
                         \Biggl[ 1 - \frac{1}{(n+1)^2} \Biggr]^{n+1}\ 
                      >\ \biggl(1 + \frac{1}{n}\biggr)\biggl(1 - \frac{1}{n+1}\biggr)\ =\ 1.
\end{align*}
Siis $a_{n+1} > a_n\ \ \forall n$, eli $\seq{a_n}$ on (aidosti) kasvava jono. Vastaavalla 
päättelyllä todetaan (Harj.teht. \ref{H-I-8: Neperin jono}), että myös jono $\seq{b_n}$ on 
aidosti kasvava. Tällöin koska jokaisella $n \ge 2$ pätee
\[
1 - \frac{1}{n^2}\ <\ 1 \quad \ekv \quad 1 + \frac{1}{n}\ <\ \frac{1}{1-\frac{1}{n}}\,,
\]
päätellään
\[
a_n\ <\ \frac{1}{b_n}\ \le\ \frac{1}{b_2}\ =\ 4 \quad \text{kun}\ n \ge 2.
\]
Siis $1 < a_n < 4$ ja $0 \le b_n < 1$ kaikilla $n$, eli jonot $\seq{a_n}$ ja $\seq{b_n}$ ovat
paitsi kasvavia myös rajoitettuja. Bernoullin epäyhtälöstä seuraa lisäksi
\[
1\ >\ a_n b_n\ =\ \bigl(1 - \frac{1}{n^2}\bigr)^n\ >\ 1 - \frac{1}{n} \quad (n \ge 2),
\]
joten $0 < 1 - a_n b_n < 1/n\ \ \forall n \ge 2$. Tästä ja lukujonon raja-arvon määritelmästä
seuraa väittämän viimeinen osa. \loppu  

Koska Proposition \ref{Neperin jonot} jonot ovat kasvavia ja rajoitettuja, niin niille
voidaan laskea Määritelmän \ref{jonon raja - desim} mukainen raja-arvo äärettömänä
desimaalilukuna. Jonon $\seq{a_n}$ tapauksessa raja-arvo on nk.\ \kor{Neperin luku}, jonka
vakiintunut symboli on $\,e\,$:
\[
\lim_n (1+\tfrac{1}{n})^n = e =\ 2.71828182845905 \ldots
\]
Luku $e$ on jaksoton, eli se ei samastu mihinkään rationaalilukuun. 
\begin{Exa} Määritä pienimmät indeksit $N_k,\ k=0\ldots 3\,$ Lauseen
\ref{monotoninen ja rajoitettu jono} todistuskonstruktiossa, kun $a_n=(1+1/n)^n$.
\end{Exa}
\ratk Todistuskonstruktion vertailuluvut ovat
\[
x_0=2, \quad x_1=2.7, \quad x_2=2.71, \quad x_3=2.718, \quad \ldots
\]
Laskemalla jonon $\seq{a_n}$ alkupään termejä desimaalilukuina todetaan
\[
\begin{aligned}
a_1    &=\ 2.00000000 \ldots \\ 
a_2    &=\ 2.25000000 \ldots \\[2mm] 
a_{73} &=\ 2.69989423 \ldots \\ 
a_{74} &=\ 2.70013967 \ldots
\end{aligned} \qquad\quad
\begin{aligned}
a_{163}   &=\ 2.70999015 \ldots \\
a_{164}   &=\ 2.71004043.. \\[2mm]
a_{4821}  &=\ 2.71799996 \ldots \\
a_{4822}  &=\ 2.71800001 \ldots
\end{aligned}
\]
Siis $\,N_0=1,\ N_1=74,\ N_2=164,\ N_3=4822$. \loppu

Palataan vielä Esimerkin \ref{kaksi sarjaa} positiivitermiseen sarjaan 
$\,\sum_{k=0}^\infty 1/k!\,$. Tämän sarjan edellä lasketuista osasummista
voi arvella, että sarja itse asiassa suppenee kohti Neperin lukua (ja vieläpä nopeasti).
Tämän arvelun varmistamiseksi riittää näyttää, että jos $\,s_n=\sum_{k=0}^n 1/k!$ ja 
$\,a_n=(1+1/n)^n$, niin $\,\lim_n(s_n-a_n)=0$, vrt.\ Neperin luvun määritelmä edellä.
Todistetaan tämä väittämä, joka samalla antaa nopean algoritmin Neperin luvun desimaalien 
laskemiseksi.
\begin{Prop} \label{e:n sarja}
$\quad \displaystyle{
\lim_n \left[\sum_{k=0}^n \frac{1}{k!}-\left(1+\frac{1}{n}\right)^n\right] = 0.}$
\end{Prop}
\tod Binomikaavan mukaan
\begin{align*}
\left(1+\frac{1}{n}\right)^n 
  &= 1+n\cdot\frac{1}{n}+\frac{n(n-1)}{1\cdot 2}\left(\frac{1}{n}\right)^2
                        +\frac{n(n-1)(n-2)}{1\cdot 2\cdot 3}\left(\frac{1}{n}\right)^3 + \cdots
                        + \left(\frac{1}{n}\right)^n \\
  &=1+1+\frac{1}{2!}\left(1-\frac{1}{n}\right)
                +\frac{1}{3!}\left(1-\frac{1}{n}\right)\left(1-\frac{2}{n}\right) + \ldots \\
  &\qquad\qquad\qquad\quad\quad\ \ \ldots  
       + \frac{1}{n!}\left(1-\frac{1}{n}\right)
                     \left(1-\frac{2}{n}\right)\cdots\left(1-\frac{n-1}{n}\right) \\
  &=\sum_{k=0}^n a_k^{(n)}\,\frac{1}{k!}\,,
\end{align*}
missä
\[
a_k^{(n)} = \begin{cases} 
            \,1,                                           \quad &\text{kun}\ k=0,1, \\
            \,\prod_{j=1}^{k-1}\left(1-\frac{j}{n}\right), \quad &\text{kun}\ k=2 \ldots n.
            \end{cases} \]
Nähdään, että jokaisella $1 \le m \le n$ pätee 
\begin{align*}
1\            &\ge\ a_k^{(n)}\ \ge\ a_m^{(n)}\ 
                           \ge\ \left(1-\frac{m-1}{n}\right)^{m-1}, \quad k = 0 \ldots m \\[2mm]
\impl\quad 0\ &\le\ 1-a_k^{(n)}\ \le\ 1-\left(1-\frac{m-1}{n}\right)^{m-1}, \quad k= 0 \ldots m.
\end{align*}
Tämän sekä arvion (ks.\ Harj.teht.\,\ref{H-I-8: kaksi sarjaa})
\[
\sum_{k=0}^n \frac{1}{k!}\ <\ \sum_{k=0}^m \frac{1}{k!} \,+\, \frac{2}{(m+1)!}\,, \quad 
                           0 \le m < n
\]
perusteella voidaan arvioida
\begin{align*}
0\ \le\ \sum_{k=0}^n \frac{1}{k!}\,-\,\left(1+\frac{1}{n}\right)^n\
       &=\ \sum_{k=0}^n \left(1-a_k^{(n)}\right)\frac{1}{k!} \\
       &=\ \sum_{k=0}^m \left(1-a_k^{(n)}\right)\frac{1}{k!}\ 
                          +\ \sum_{k=m+1}^n \left(1-a_k^{(n)}\right)\frac{1}{k!} \\
       &<\ \left[1-\left(1-\frac{m-1}{n}\right)^{m-1}\right]\,\sum_{k=0}^m \frac{1}{k!}\ 
                          +\ \sum_{k=m+1}^n \frac{1}{k!} \\
       &<\ \left[1-\left(1-\frac{m-1}{n}\right)^{m-1}\right] \cdot 3 \,+\, \frac{2}{(m+1)!}\,,
\end{align*}
missä $0 \le m < n$. Olkoon nyt $\eps>0$ ja valitaan $m\in\N$ siten, että
\[
\frac{2}{(m+1)!}\ <\ \frac{\eps}{2}
\]
(mahdollista, koska $\,2/(m+1)! \kohti 0\,$ kun $m\kohti\infty$). Kun $m$ on näin kiinnitetty,
valitaan edelleen $N\in\N,\ N \ge m\,$ siten, että 
\[
1-\left(1-\frac{m-1}{n}\right)^{m-1}\ <\ \frac{\eps}{6}\,, \quad \text{kun}\ n>N
\]
(mahdollista, koska $\left(1-\frac{m-1}{n}\right)^{m-1} \kohti 1\ 
\impl\ 1-\left(1-\frac{m-1}{n}\right)^{m-1} \kohti 0\,$, kun $n\kohti\infty$). Tällöin seuraa
\[
0\ \le\ \sum_{k=0}^n \frac{1}{k!}\,-\,\left(1+\frac{1}{n}\right)^n\ <\ \eps, 
                                                        \quad \text{kun}\ n>N.
\]
Koska tässä $\eps>0$ oli mielivaltainen ja $N\in\N$ ($N$ riippuu vain $\eps$:sta), niin 
lukujonon raja-arvon määritelmän perusteella seuraa väite. \loppu

\Harj
\begin{enumerate}


\item \label{H-I-8: kaksi sarjaa}
Laskemalla summia \ a) $s_n=\sum_{k=0}^n 1/k!\,$, \ b) $s_n=\sum_{k=1}^n 1/k^2\,$ todetaan:
\newline
a) $s_n=2.718281826..$ kun $n=11$, \ b) $s_n=1.644934065..$ kun $n=10^9$.
Vedetään tuloksista johtopäätös: $\,\lim_n s_n=s\in\DD$ katkaistuna kahdeksaan merkitsevään
numeroon on \ a) $s=2.71828182..$ \ b) $s=1.64493406..$ \newline 
Varmista johtopäätös näyttämällä ensin: Jos \ a) $m\in\N\cup\{0\}$, \ b) $m\in\N$, niin
jokaisella $n>m$ pätee
\[
\text{a)}\ \ s_n < s_m + \frac{m+2}{m+1}\cdot\frac{1}{(m+1)!}\,, \qquad
\text{b)}\ \ s_n < s_m + \frac{1}{m}\,.
\]

\item
Näytä, että positiivitermisen sarjan
\[
\sum_{k=0}^\infty \frac{10^{-k}}{k+1}
\]
osasummien jono on rajoitettu. Laske sarjan summa $x\in\DD$ viiden merkitsevän numeron 
tarkkuudella ja varmista tulos, ts.\ todista lasketut numerot oikeiksi.

\item
Millaisen desimaalimuodon Lauseen \ref{monotoninen ja rajoitettu jono} todistuskonstruktio
antaa raja-arvoille $\,a=\lim_n (1-1/n)$, $\,b=\lim_n (1+1/n)$, $c=\lim_n (-1+1/n)$ ja
$d=\lim_n (-1-1/n)\,$?

\item \label{H-I-8: Neperin jono}
Näytä, että Proposition \ref{Neperin jonot} lukujono $\seq{b_n}$ on aidosti kasvava.

\item
Näytä, että palautuva lukujono
\[
a_0 = \frac{1}{10}\,, \quad \displaystyle{a_{n+1} = \frac{a_n}{1+a_n^6}\,, \quad n=0,1, \ldots}
\]
on vähenevä ja rajoitettu. Jonolla on myös rationaalinen raja-arvo --- mikä\,?

\item
Näytä, että palautuva rationaalilukujono
\[
a_0\in\Q, \quad a_{n+1}=\frac{a_n}{1+10^{-n}a_n^2}\,,\quad n=0,1, \ldots
\]
on monotoninen ja rajoitettu. Laske raja-arvo $\,\lim_n a_n\, = a\in\DD$ kuuden 
merkitsevän numeron tarkkuudella ja perustele tarkkuus, kun \ a) $a_0=1$, \ b) $a_0=3$. 

\pagebreak

\item
Tarkastellaan palautuvaa rationaalilukujonoa
\[
a_0=1, \quad a_{n+1} = 1-\frac{1}{a_n+3}\,, \quad n=0,1,\ldots
\]
a) Näytä induktiolla: $\,\tfrac{5}{7} < a_n \le \tfrac{3}{4}$, kun $n=1,2,\ldots$ \newline
b) Näytä: Jonolle $\seq{b_n}=\seq{a_{n+1}-a_n}$ pätee
\[
b_{n+1} = \frac{b_n}{(a_n+3)(a_{n+1}+3)}\,, \quad n=0,1,\ldots
\]
c) Näytä induktiolla: $\seq{a_n}$ on aidosti vähenevä lukujono. \newline
d) Laske raja-arvo $\lim_n a_n$ Määritelmän \ref{jonon raja} mukaan. \newline
e) Päättele: Määritelmän \ref{jonon raja - desim} mukainen raja-arvo katkaistuna kuuteen
merkitsevään numeroon on $\lim_n a_n = 0.732050..$ (voit tukeutua laskimeen\,!). 

\item (*)
Eräs algoritmi tunnetun luvun $\pi\ =\ 3.1415926535897932384626433832..$ laskemiseksi
perustuu kaavaan
\[
\pi=\sum_{k=0}^\infty \frac{1}{16^k}
    \left(\frac{4}{8k+1}-\frac{2}{8k+4}-\frac{1}{8k+5}-\frac{1}{8k+6}\right).
\]
Totea sarja positiivitermiseksi ja arvioi, montako $\pi$:n oikeaa desimaalia on 
poimittavissa sarjan osasumman $s_n$ desimaalilukumuodosta, kun a) $n=10$, \ b) $n=15$, \ 
c) $n=20$. (Kaava on keksitty v. 1995.)

\item (*) \label{H-I-8: sqrt-kunta}
Näytä, että seuraavat lukujonot (kunnassa $(\J,+\cdot,<)$, ks.\ Luku \ref{kunta})
ovat monotonisia ja rajoitettuja (\kor{vihje}: induktio!). Laske lukujonoille myös raja-arvot
(\kor{vihje}: juuriluvun määritelmä ja Lause \ref{raja-arvojen yhdistelysäännöt}). \\[2mm]
a) $\ \ a_0=\sqrt{3}\,, \quad   a_{n+1}=\sqrt{3+a_n}\,,      \quad n=0,1, \ldots$ \\[2mm]
b) $\ \ a_0=\sqrt{3}\,, \quad   a_{n+1}=\sqrt{2a_n}\,,       \quad n=0,1, \ldots$ \\[2mm]
c) $\ \ a_0=\sqrt{5}\,, \quad\, a_{n+1}=\sqrt{2a_n}\,,       \quad n=0,1, \ldots$ \\[2mm]
d) $\ \ a_0=3,          \qquad  a_{n+1}=\sqrt[4]{6+a_n^2}\,, \quad n=0,1, \ldots$  

\item (*)
Näytä, että palautuva lukujono
\[
a_{n+1} = \frac{a_n}{5-a_n}\,, \quad n=0,1, \ldots
\]
on rajoitettu ja eräästä indeksistä alkaen monotoninen, sikäli kuin pätee
\[
a_0 \neq \frac{4}{1-5^{-n}} \quad \forall n\in\N.
\]
Määritä myös tällä ehdolla ($a_0$:sta riippuva) raja-arvo $\,\lim_n a_n$. \newline
\kor{Vihje}: Tutki jonoa $\seq{b_n}=\seq{a_n^{-1}}$. 

\end{enumerate} 
  % Monotoniset ja rajoitetut lukujonot
\section{Reaaliluvut} \label{reaaliluvut}
\alku

Aloitetaan määritelmästä.
\begin{Def} \label{reaaliluvut desimaalilukuina} \index{reaaliluvut!a@desimaalilukuina|emph}
(\vahv{Reaaliluvut}) \kor{Reaalilukujen} joukko $\R$ koostuu äärettömistä desimaaliluvuista,
ts.\ $\R=\DD$.
\end{Def}
Määritelmä \ref{reaaliluvut desimaalilukuina} reaaliluvuille ei ole ainoa mahdollinen, sillä on
ilmeistä, että esimerkiksi äärettömät binaariluvut voitaisiin yhtä hyvin ottaa määrittelyn
lähtökohdaksi. Monia muitakin lähestymistapoja on, kuten havaitaan tuonnempana. Määritelmä 
\ref{reaaliluvut desimaalilukuina} antaa joka tapauksessa reaaliluvuille erään mahdollisen 
tulkinnan, joka jatkossa otetaan lähtökohdaksi. 

Yhdessä aiemmin sovitun samastusrelaation (Määritelmä \ref{samastus DD}) kanssa Määritelmä 
\ref{reaaliluvut desimaalilukuina} kertoo vasta, miltä reaaliluvut 'näyttävät' ja miten niitä 
erotellaan. Jotta reaaliluvuilla päästäisiin myös laskemaan, on määriteltävä lukujen väliset
laskuoperaatiot. 

\subsection{$\R$:n laskuoperaatiot}

Viitaten Määritelmään \ref{jonon raja - desim} asetetaan
\begin{Def} \label{reaalilukujen laskutoimitukset}
\index{laskuoperaatiot!b@reaalilukujen|emph}
\index{reaalilux@reaalilukujen!a@laskuoperaatiot|emph}
(\vahv{Reaalilukujen laskutoimitukset}) Jos $\x = \seq{x_n}$ ja $\y = \seq{y_n}$ ovat
reaalilukuja, niin määritellään
\begin{align*}
\text{yhteenlasku:} \quad\quad       \x+\y &= \lim_n\,(x_n+y_n) = \z\in\R. \\
\text{kertolasku:} \ \quad\quad\quad  \x\y &= \lim_n\,x_n y_n = \z\in\R.\\
\text{jakolasku:} \,\ \ \quad\quad   \x/\y &= \lim_n\,x_n / y_n = \z\in\R.
\end{align*} \end{Def}
Tässä on jakolaskussa oletettava tavalliseen tapaan $\y \neq 0$, eli $y_n \not\kohti 0$
(Määritelmä \ref{samastus QD}). Ehto merkitsee, että jollakin $m$ on  $y_m \neq 0$, jolloin
tästä indeksistä alkaen on $\abs{y_n} \ge \abs{y_m} > 0,\ n \ge m$ (koska $\seq{\abs{y_n}}$ on
kasvava jono). Jakolaskun määritelmä on siis muodollisesti kunnossa, kunhan $\y \neq 0$ ja
jonossa $\seq{x_n/y_n}$ rajoitutaan indekseihin $n \ge m$. Yhteenlaskun määritelmästä nähdään,
että luvun $\x = \seq{x_n}$ vastaluku on $-\x = \seq{-x_n}$, eli operaatio $\x \map -\x$ vastaa
desimaaliluvun etumerkin vaihtoa, kuten jo Luvussa \ref{desimaaliluvut} sovittiin. Vähennyslasku
määritellään tämän jälkeen normaaliin tapaan, eli $\x-\y=\x+(-\y)=\lim_n(x_n-y_n)\in\R$.

Määritelmä \ref{reaalilukujen laskutoimitukset} on käytännön läheinen sikäli, että se
perustuu suoraan lukujen $\breve{x}$ ja $\breve{y}$ esitysmuotoihin annetussa
lukujärjestelmässä, tässä kymmenjärjestelmässä. Lukujonojen $\seq{x_n+y_n}$, $\seq{x_ny_n}$ ja
$\seq{x_n/y_n}$ termien laskemisen voi ajatella vastaavan likimääräisiä laskuoperaatioita
liukuluvuilla (vrt.\ Luku \ref{desimaaliluvut}). Määritelmään sisältyy kuitenkin toistaiseksi
ratkaisematon teoreettinen ongelma: Lukujonojen $\seq{x_n+y_n}$, $\seq{x_ny_n}$ ja
$\seq{x_n/y_n}$ oletetetaan suppenevan kohti reaalilukua (eli ääretöntä desimaalilukua), mutta
toistaiseksi tämä on varmistettu vain monotonisen ja rajoitetun lukujonon osalta
(Lause \ref{monotoninen ja rajoitettu jono}). Määritelmän
\ref{reaalilukujen laskutoimitukset} lukujonot ovat kyllä rajoitettuja (mainituin lisäehdoin
koskien jonoa $\seq{x_n/y_n}$) mutta eivät välttämättä monotonisia. Määritelmän tueksi
tarvitaankin seuraava lause.
\begin{Lause} \label{monotoninen ja rajoitettu jono - yleistys} Jos $\seq{a_n}$ ja $\seq{b_n}$
ovat monotonisia ja rajoitettuja lukujonoja, niin lukujonot $\seq{a_n+b_n}$ ja $\seq{a_nb_n}$
suppenevat Määritelmän \ref{jonon raja - desim} mukaisesti kohti reaalilukua. Jos edelleen
$|b_n| \ge b>0\ \forall n$, niin myös jono $\seq{a_n/b_n}$ suppenee kohti reaalilukua.
\end{Lause}
\tod \ \underline{Yhteenlasku}. Jos $\seq{a_n}$ ja $\seq{b_n}$ ovat molemmat kasvavia tai
molemmat väheneviä, niin $\seq{a_n+b_n}$ on vastaavasti kasvava/vähenevä
(Harj.teht.\,\ref{desimaaliluvut}:\ref{H-I-5: monotonisten jonojen yhdistely}), jolloin
$\lim_n(a_n+b_n)\in\R$ on olemassa Lauseen \ref{monotoninen ja rajoitettu jono} perusteella.
Olkoon toinen jonoista kasvava ja toinen vähenevä, esim.\ $\seq{a_n}$ kasvava. Valitaan
$b\in\Q$ siten, että $b_n+b \ge 0\ \forall n$ (mahdollista, koska $\seq{b_n}$ on rajoitettu) ja
kirjoitetaan
\[
a_n+b_n = (a_n-b)+(b_n+b).
\]
Tässä $\seq{b_n+b}$ on rajoitettu ja vähenevä jono, joten Lauseen
\ref{monotoninen ja rajoitettu jono} mukaan se suppenee:
$\lim_n(b_n+b)=\breve{x}=\seq{x_n}\in\R$. Koska $b_n+b \ge 0\ \forall n$, on $\breve{x}$:n
etumerkki $+$ (mahdollisesti $\breve{x}=0$), joten lukujono $\seq{x_n}$ on kasvava.
Koska myös $\seq{a_n-b}$ on kasvava jono, niin samoin on $\seq{a_n-b+x_n}$, joten
Lauseen \ref{monotoninen ja rajoitettu jono} mukaan 
$\lim_n(a_n-b+x_n)=\breve{y}=\seq{y_n}\in\R$. Näin määrätty $\breve{y}$ on lukujonon
$\seq{a_n+b_n}$ raja-arvo Määritelmän \ref{jonon raja - desim} mukaisesti, sillä ko.\
määritelmän ja Lauseen \ref{raja-arvojen yhdistelysäännöt} perusteella
\[
(a_n+b_n)-y_n = [(a_n-b+x_n)-y_n]+(b_n+b-x_n) \kohti 0+0 = 0.
\]
\underline{Kertolasku}. Olkoon $\lim_na_n=\breve{x}=\seq{x_n}\in\R$ ja
$\lim_nb_n=\breve{y}=\seq{y_n}\in\R$. Jos $\breve{x}$:llä ja $\breve{y}$:llä on sama etumerkki,
niin jono $\seq{x_ny_n}=\seq{|x_n||y_n|}$ on kasvava (koska $\seq{|x_n|}$ ja $\seq{|y_n|}$, ovat
kasvavia, vrt.\ Harj.teht.\,\ref{desimaaliluvut}:\ref{H-I-5: monotonisten jonojen yhdistely}),
muussa tapauksessa $\seq{x_ny_n}=\seq{-|x_n||y_n|}$ on vähenevä jono. Lauseen
\ref{monotoninen ja rajoitettu jono} mukaan on siis olemassa raja-arvo
$\lim_nx_ny_n=\breve{z}=\seq{z_n}\in\R$. Tämä on myös lukujonon $\seq{a_nb_n}$ raja-arvo
Määritelmän \ref{jonon raja - desim} mukaisesti, sillä ko.\ määritelmän ja Lauseen
\ref{jonotuloksia} (V3) mukaan
\[
a_nb_n-z_n = (a_n-x_n)b_n+(b_n-y_n)x_n+(x_ny_n-z_n) \kohti 0+0+0=0.
\]
\underline{Jakolasku}. Koska $\seq{b_n}$ on monotoninen lukujono, niin tehdyin lisäoletuksin
on jostakin indeksistä $n=m$ alkaen oltava joko $b_n \ge b$ tai $b_n \le -b$. Tällöin
lukujono $\seq{1/b_n}$ on monotoninen indeksistä $n=m$ alkaen. Kirjoitetaan tästä indeksistä
eteenpäin $a_n/b_n=a_n(1/b_n)$ ja vedotaan jo käsiteltyyn kertolaskuun. \loppu

Koska Määritelmän \ref{reaalilukujen laskutoimitukset} lukujonot $\seq{x_n}$ ja $\seq{y_n}$ ovat
monotonisia ja rajoitettuja, niin Lause \ref{monotoninen ja rajoitettu jono - yleistys} takaa
määritelmän toimivuuden. Havainnollistettakoon lauseen todistuksessa käytettyä päättelyä vielä
laskuesimerkillä.
\begin{Exa} \label{laskuesimerkki desimaaliluvuilla pi ja e} Olkoon
\begin{align*}
\breve{\pi} &= 3.141592653\ldots = \seq{3,3.1,3.14,3.141\ldots} = \seq{p_n}, \\
\breve{e}   &= 2.718281828\ldots = \seq{2,2.7,2.71,2.718\ldots} = \seq{e_n}.
\end{align*}
Määritä a) $\breve{\pi}+\breve{e}$ ja $\breve{\pi}\breve{e}$, b) $\breve{\pi}-\breve{e}$,
c) $\breve{\pi}/\breve{e}$ Määritelmän \ref{reaalilukujen laskutoimitukset} mukaisesti
viiden desimaalin tarkkuudella käyttämällä Lauseen \ref{monotoninen ja rajoitettu jono}
todistusta algoritmina.
\end{Exa}
\ratk a) Lukujonot
\[
\seq{p_n+e_n}=\seq{5,5.8,5.85,5.859\ldots}, \quad 
\seq{p_ne_n}=\seq{6,8.37,8.5094,8.537238\ldots}
\]
ovat molemmat kasvavia ja rajoitettuja, joten ne suppenevat kohti reaalilukua
(Lause \ref{monotoninen ja rajoitettu jono}). Lauseen \ref{monotoninen ja rajoitettu jono} 
todistuskonstruktiota algoritmina käyttäen saadaan
\[
\breve{\pi}+\breve{e} = 5.85987\,..\,, \quad \breve{\pi}\breve{e}  = 8.53974\,..\,.
\]
b) Lukujono $\seq{p_n-e_n}=\seq{0,0.4,0.43,0.423,\ldots}$ ei ole monotoninen, joten Lause
\ref{monotoninen ja rajoitettu jono} ei sovellu suoraan. Kirjoitetaan sen vuoksi ensin
\[
p_n-e_n = (p_n-3) + (3-e_n).
\]
Tässä $\seq{3-e_n}$ on rajoitettu ja vähenevä lukujono. Sen raja-arvoksi saadaan Lauseen
\ref{monotoninen ja rajoitettu jono} todistuskonstruktiolla
\[
\lim_n(3-e_n) = 3-\breve{e} = 0.28171\,..
\]
Kun tämä välitulos tulkitaan lukujonoksi $\seq{0,0.2,0.28,0.281,\ldots}=\seq{x_n}$, niin
$\seq{p_n-3}$ ja $\seq{x_n}$ ovat molemmat kasvavia lukujonoja. Siis myös $\seq{(p_n-3)+x_n}$
on kasvava, ja voidaan laskea (algoritmi sama kuin edellä)
\[
\breve{\pi}-\breve{e} = \lim_n[(p_n-3)+x_n] = 0.42331\,.. 
\]
c) Lukujonon $\seq{p_n/e_n}$ ei ole monotoninen, joten Lause
\ref{monotoninen ja rajoitettu jono} ei sovellu tähänkään suoraan. Lasketaan ensin
välituloksena vähenevän lukujonon $\seq{1/e_n}$ raja-arvo:
\[
\lim_n(1/e_n) = 1/\breve{e} = 0.36787\,..
\]
Kun välitulos tulkitaan lukujonoksi $\seq{x_n}=\seq{0,0.3,0.36,0.367,\ldots}$, niin
$\seq{p_nx_n}$ on kasvava lukujono. Tähän Lauseen \ref{monotoninen ja rajoitettu jono}
todistuskonstruktio soveltuu ja antaa
\[
\breve{\pi}/\breve{e} = \breve{\pi}(1/\breve{e}) = \lim_n p_nx_n = 1.15572\,.. \loppu
\]

\subsection{$\R$:n järjestysrelaatio}

Myös järjestysrelaation määritelmä reaaliluvuille on suoraviivainen. 
\begin{Def} \label{reaalilukujen järjestys} 
\index{jzy@järjestysrelaatio!c@$\R$:n|emph}
\index{reaalilux@reaalilukujen!b@järjestysrelaatio|emph}
(\vahv{$\R$:n järjestysrelaatio}) Jos $\x=\seq{x_n}\in\R,\ \y=\seq{y_n}\in\R$, niin
$\,\x<\y\,$ täsmälleen kun $\,x_n<y_n\,$ jollakin $n\,$ ja $\,\x\neq\y$.
\end{Def}
Määritelmän \ref{reaalilukujen järjestys} mukaisesti voidaan reaalilukujen
$\x=\seq{x_n}$ ja $\y=\seq{y_n}$ suuruusjärjestys ratkaista laskennallisesti vertailemalla
lukuja $x_n,y_n,\ n=0,1,\ldots$ Vertailua jatketaan, kunnes tavataan ensimmäinen indeksi, 
jolla $x_n \neq y_n$. (Jollei tällaista indeksiä tavata, on $\x=\y$.) Jos 
$\abs{x_n-y_n} \ge 2 \cdot 10^{-n}$, on suuruusjärjestys ratkennut. Muussa tapauksessa,
eli jos $x_n-y_n = \pm 10^{-n}$, on vielä mahdollisuus, että $\x$ ja $\y$ samastuvat
kumpikin lukuun $x_n$ tai $y_n$, jolloin on $\x=\y$ (ks.\ Luvun \ref{suppenevat lukujonot}
tarkastelut liittyen Lauseeseen \ref{samastuslause DD}).

Määritelmistä \ref{reaaliluvut desimaalilukuina}--\ref{reaalilukujen järjestys} on
johdettavissa seuraava merkittävä tulos. Todistus esitetään luvun lopussa
(harjoitustehtävillä tuettuna).
\begin{*Lause} \label{R on kunta} (\vahv{$(\R,+,\cdot,<)$ on järjestetty kunta})
Reaalilukujen joukko varustettuna Määritelmän \ref{reaalilukujen laskutoimitukset}
mukaisilla laskuoperaatioilla ja Määritelmän \ref{reaalilukujen järjestys} mukaisella
järjestysrelaatiolla on järjestetty kunta. 
\end{*Lause}

Määritelmästä \ref{reaalilukujen järjestys} seuraa välittömästi, että jos
$\x = \seq{x_n} \in \R$, niin
\begin{align*}
\x = 0 \quad &\ekv \quad x_n = 0\ \ \forall n \\
\x < 0 \quad &\ekv \quad x_n < 0\ \ \text{jollakin}\ n \quad (\text{etumerkki}\ e = - ) \\
\x > 0 \quad &\ekv \quad x_n > 0\ \ \text{jollakin}\ n \quad (\text{etumerkki}\ e = + )
\end{align*}
Tämä vertailu siis ratkeaa pelkästään $\x$:n etumerkin perusteella, ellei ole $\x=0$. $\R$:n
järjestysrelaation määrittely voidaan vaihtoehtoisesti perustaa tähän vertailuun yhdistettynä
vähennyslaskuun (vrt.\ $\Q$:n järjestysrelaatio, Luku \ref{ratluvut})\,:
\[
\x<\y \,\ \ekv \,\ \x-\y<0.
\]
On ilmeistä, että järjestysrelaation aksioomista ainakin (J1) (vrt.\ Luku \ref{ratluvut})
on voimassa kummalla tahansa määritelmällä. 

\subsection{Reaaliluvut abstrakteina lukuina}
\index{reaaliluvut!b@abstrakteina lukuina|vahv}

Kun reaaliluvuille on määritelty sekä laskuoperaatiot että järjestysrelaatio, voidaan 
reaalilukujen olemus jonoina haluttaesssa 'unohtaa' ja käsitellä lukuja vain abstrakteina 
lukuina, joilla voi laskea ja joita voi vertailla. Tämän ajattelutavan mukaisesti pidetään
jatkossa myös reaalilukuja 'oikeina' lukuina, joita ei symbolisissa merkinnöissä enää
erotella rationaaliluvuista. 
\begin{Exa} \label{kertausesimerkkejä} Luvun \ref{monotoniset jonot} tulosten ja 
Määritelmien \ref{reaaliluvut desimaalilukuina}-- \ref{reaalilukujen laskutoimitukset}
perusteella voidaan nyt kirjoittaa
\begin{align*}
\lim_n \left(1 + \dfrac{1}{n}\right)^n\ =\ \sum_{k=0}^\infty \frac{1}{k!}\    
                                           &= \ 2.71828182845905\,..\ = \ e\in\R, \\ 
\lim_n \left(1 - \dfrac{1}{n}\right)^n\    &= \ 0.36787944117144\,..  = \ \frac{1}{e}\,.
\end{align*}
Ottamalla käyttöön toinen hyvin tunnettu reaaliluku $\,\pi\ =\ 3.1415926535897\,..$ voidaan
myös osoittaa, että (vrt.\ Luku \ref{monotoniset jonot})
\[
\sum_{k=1}^\infty \dfrac{1}{k^2}\ = \ 1.64493406684822\,..\ = \ \frac{\pi^2}{6}\,.
\]
\end{Exa}
Reaalilukujen tultua määritellyksi järjestettynä kuntana tulee myös lukujonon raja-arvon 
alkuperäisestä määritelmästä (Määritelmä \ref{jonon raja}) pätevä, kun jonon termit ja/tai 
raja-arvo ovat reaalilukuja. Koska lukujonon suppeneminen kohti reaalista raja-arvoa on
aiemmin määritelty erikseen (Määritelmä \ref{jonon raja - desim}), tarvitaan ympyrän
sulkemiseksi seuraava tulos (todistus luvun lopussa).
\begin{*Lause} \label{suppeneminen kohti reaalilukua} Määritelmät \ref{jonon raja} ja 
\ref{jonon raja - desim} lukujonon suppenemiselle kohti reaalilukua ovat yhtäpitävät. 
\end{*Lause}

\subsection{Kymmenjako- ja puolituskonstruktiot}
\index{kymmenjakokonstruktio|vahv} \index{puolituskonstruktio|vahv}%

Määritelmän \ref{reaaliluvut desimaalilukuina} mukaan reaalilukua voi laskennallisesti 
'vain lähestyä, ei saavuttaa', paitsi jos luku on rationaalinen. Algoritmisesti voi 
'lähestyminen' tapahtua esim.\ suoraan Määritelmään \ref{reaaliluvut desimaalilukuina} 
perustuen, jolloin lasketaan ensin luvun kokonaislukuosa ja sen jälkeen desimaaleja yksi
kerrallaan. Tarkastellaan nyt hieman yleisemmin tätä laskemisen ongelmaa.

Olkoon reaaliluku $a$ määritelty yksikäsitteisesti, esimerkiksi asettamalla luvulle jokin 
algebrallinen ehto, määrittelemällä luku jonkin tunnetun jonon raja-arvona, tai jollakin
muulla tavalla. Halutaan konstruoida $\,a\,$ Määritelmän \ref{reaaliluvut desimaalilukuina}
mukaisena desimaalilukuna, eli muodossa $a = x_0.d_1 d_2\ldots$  Oletetaan jatkossa, että $a$:n
etumerkki on $+$ (muuten konstruoidaan luku $-a\,$). Tällöin $a$:n kokonaislukosa ja desimaalit
luvuissa $x_n=x_0.d_1 \ldots d_n$ määrytyvät seuraamalla toimintaohjetta:
\[
\text{Etsi $\,x_n\in\Q_n\,$ siten, että pätee}\ \ x_n \le a\ \ 
                            \text{ja}\ \ x_n + 10^{-n} > a, \quad n = 0,1, \ldots
\]
Jos $a$ on rationaaliluku, niin toimintaohje on sama kuin jakokulma-algoritmissa (vrt.\ Luku 
\ref{desimaaliluvut}). Yleisemmän reaaliluvun $a>0$ tapauksessa toimintaohjetta voi pitää
ajatuskonstruktiona, joka \pain{määrittelee} luvun $a$. Annetaan tälle nimi
\kor{kymmenjako}konstruktio. --- Kymmenjakokonstruktiota on jo aiemmin käytetty Lauseen
\ref{monotoninen ja rajoitettu jono} todistuksessa. Reaaliluvun määritelmänä 
konstruktiota voi pitää järjestysrelaatioon nojaavana Määritelmän
\ref{reaaliluvut desimaalilukuina} toisintona tai täsmennyksenä.

Kymmenjakokonstruktion on jo aiemmin todettu toimivan myös algoritmina (vrt.\ edellisen
luvun esimerkit ja harjoitustehtävät). Yleisemmin algoritmi toimii, jos oletetaan, että
konstruktioon sisältyvät vertailut ovat laskennallisesti toteutettavissa. Näin on ainakin,
jos oletetaan laskettava luku $a \in \R$ sellaiseksi, että minkä tahansa \pain{rationaaliluvun}
$x \in \Q$ kohdalla voidaan selvittää \pain{äärellisellä} määrällä \pain{rationaalisia}
operaatioita (laskuoperaatioita ja vertailuja), mikä vaihtoehdoista $a < x$, $a = x$, $a > x$
on voimassa. Tällöin saadaan ensin $a$:n etumerkki selville vertaamalla lukuun $x=0$, minkä
jälkeen $a$:n tai $-a$:n kokonaislukuosa ja desimaalit $d_n$ voidaan laskea ym.\ toimintaohjetta
seuraamalla, periaatteessa mihin tahansa haluttuun indeksiin asti. Tällainen
\kor{kymmenjakoalgoritmi} on siis jakokulma-algoritmin yleistys.
\begin{Exa} \label{neliöjuuri 2} Konstruoi reaaliluku $a=\sqrt{2}$ kymmenjakoalgoritmilla.
\end{Exa}
\ratk Luvun $a$ ja positiivisen rationaaliluvun $x$ vertailu palutuu rationaaliseksi
ekvivalenssilla
\[
a<x\ \ekv\ 2<x^2.
\]
Koska $1^2<2$ ja $2^2>2$, on ensinnäkin oltava $\ a = 1.d_1d_2d_3\,..\ $ Edelleen koska
$\,1.4^2 = 1.96 < 2$, $\,1.5^2 = 2.25 > 2$, $\,1.41^2 = 1.9881 < 2$, $\,1.42^2 = 2.0164 > 2$,
$\,1.414^2 = 1.999396 < 2$ ja $1.415^2 = 2.002225 > 2$, on $d_1 = 4$, $d_2=1$ ja $d_3=4$.
Jatkamalla tällä tavoin on tuloksena (jaksoton) desimaaliluku
$a = 1.4142135623730950488016887\,..$\footnote[2]{Ennen laskimia on neliöjuurien käsinlaskua
esimerkin tapaan harjoiteltu kouluissakin. -- Nykyisin laskimet ja tietokoneet laskevat
neliöjuuria kymmenjakoa paljon tehokkaammilla, palautuviin lukujonoihin perustuvilla
algoritmeilla, vrt.\ edellisen luvun Esimerkki \ref{sqrt 2 algoritmina}.} \loppu
\begin{Exa} Laske luku $a = \sqrt{\sqrt{2}+1}\ $ $20$ merkitsevän numeron tarkkuudella käyttäen
kymmenjakoalgoritmia. 
\end{Exa}
\ratk Verrattaessa lukua $a$ rationaalilukuun $x \ge 1$ pätee
\[
a\,<\,x \qekv \sqrt{2} + 1\,<\,x^2 \qekv \sqrt{2}\,<\,x^2 - 1 \qekv 2\,<\,(x^2 - 1)^2,
\]
joten vertailu palautuu rationaalioperaatioksi. Algoritmia seuraten saadaan
\[
a\ =\ 1.5537739740300373073\,.. \quad \loppu
\]
\index{symbolinen laskenta}%
Esimerkeissä vertailun $a<x$ palauttaminen rationaaliseksi perustuu \kor{symboliseen} 
(ei-numeeriseen) \kor{laskentaan}, tässä algebraan, joka nojaa viime kädessä reaalilukujen
kunta-aksioomiin, järjestysrelaation aksioomiin ja neliöjuuren symboliseen määritelmään
$(\sqrt{a})^2 = a$. Kuten näissä esimerkeissä, symbolisen laskennan tehtävänä on yleensä
yksinkertaistaa tai selkeyttää laskentatehtävää ennen varsinaisia numeerisia laskuja.
 
Kymmenjaon vastine voidaan luonnollisesti konstruoida myös muihin lukujärjestelmiin perustuvana. 
Binaarijärjestelmän tapauksessa käytetään nimitystä \kor{puolituskonstruktio} (tai 
puolitusmenetelmä, engl.\ bisection). Puolituskonstruktiossa reaaliluku $a \ge 0$ konstruoidaan
binaarimuodossa $a = x_0.b_1b_2b_3\,..$, missä kokonaislukuosa $x_0$ ja bitit $b_n \in \{0,1\}$
valitaan siten, että luvut $x_0$ ja $\,x_n = x_0 + \sum_{k=1}^n b_k \cdot 2^{-k},\ n \in \N,$
toteuttavat  
\[
x_n \le\,a \quad \text{ja} \quad x_n + 2^{-n} >\,a, \quad n = 0,1, \ldots
\]
Yksinkertaisen logiikkansa vuoksi puolitusmenetelmää käytetään matemaattisten todistusten 
ajatuskonstruktioissa hyvin usein. (Puolituskonstruktio olisi ollut vaihtoehto myös Lauseen
\ref{monotoninen ja rajoitettu jono} todistuksessa.) Ym.\ lisäoletusten voimassa ollessa voi
konstruktiota käyttää myös toimivana algoritmina.

\subsection{Reaalilukujen jonot}
\index{lukujono|vahv}

Reaalilukujen jono-olemuksen 'unohtaminen' on erityisen suositeltavaa silloin, kun tutkitaan 
\kor{reaalilukujonoja} ja niiden suppenemista. Järjestysrelaatioon sekä (vähäisessä määrin)
kunta-algebraan vetoava Määritelmä \ref{jonon raja} on sellaisenaan pätevä myös reaalilukujonon
suppenemisen määritelmänä. Jos reaalilukujonolla on tämän määritelmän mukainen raja-arvo
(reaalilukuna), niin sanotaan, että jono
\index{suppeneminen!a@lukujonon} \index{hajaantuminen!a@lukujonon}%
\kor{suppenee}, muussa tapauksessa \kor{hajaantuu} (vrt.\ Määritelmä \ref{jonon raja - desim}). 
\begin{Exa} \label{reaalinen geometrinen sarja} Määritä perusmuotoisen geometrisen sarjan summa,
kun $q = 1/\sqrt{2}$. 
\end{Exa}
\ratk Luvussa \ref{jonon raja-arvo} johdettu geometrisen sarjan summakaava on pätevä, kun 
$q \in \R$ ja $\abs{q}<1$. Tässä on $0<q<1$, joten
\[
\sum_{n=0}^{\infty} \bigl( \frac{1}{\sqrt{2}} \bigr)^n\ 
                      =\ \frac{1}{1-\frac{1}{\sqrt{2}}}\ =\ \frac{\sqrt{2}}{\sqrt{2}-1}\ 
                      =\ 2 + \sqrt{2}\ =\ 3.4142135623730950488 \ldots  \loppu
\]
Esimerkin lasku on jälleen esimerkki myös symbolisesta laskennasta, jossa lukua $\sqrt{2}$ 
käsitellään abstraktina lukuna pelkästään kunnan $(\R,+,\cdot)$ yleisiä ominaisuuksia 
(kunta-aksioomia) ja symbolista määritelmää $\ (\sqrt{2})^2 = 2\ $ hyväksi käyttäen.

Todettakoon, että kaikki Luvuissa \ref{jonon raja-arvo}--\ref{monotoniset jonot} esitetyt
lukujonoja koskevat määritelmät ja väittämät ovat päteviä reaalilukujen kunnassa
$(\R,+,\cdot,<)$ --- syystä, että nämä perustuvat vain järjestetyn kunnan algebraan ja
oletukseen, että kunta sisältää rationaaliluvut. 

Esimerkkinä reaalilukujonon suppenemistarkastelusta todistettakoon
\begin{Prop} \label{juurilemma} $\quad \displaystyle{\boxed{\kehys\quad 
\lim_n \sqrt[n]{a}=1 \quad \forall a\in\R,\ a>0. \quad}}$ 
\end{Prop}
\tod Merkitään $b_n=\sqrt[n]{a}$. Tapauksessa $a=1$ on väittämä ilmeisen tosi ja tapauksessa 
$0<a<1$ pätee: $\,b_n=1/(1/a)^{1/n} \kohti 1$, jos $(1/a)^{1/n} \kohti 1$ 
(Lause \ref{raja-arvojen yhdistelysäännöt} reaalilukujonoille). Riittää siis tarkastella
tapausta $a>1$. Tällöin on oltava $b_n>1\ \forall n$ (koska $b_n^n=a>1$), jolloin seuraa
\[
\left(\frac{b_{n+1}}{b_n}\right)^n =\, \frac{1}{b_{n+1}}\frac{b_{n+1}^{n+1}}{b_n^n}
\,=\, \frac{1}{b_{n+1}}\frac{a}{a} \,=\, \frac{1}{b_{n+1}} \,<\, 1. 
\]  
Siis $b_{n+1}/b_n<1\ \forall n$, joten $\seq{b_n}$ on aidosti vähenevä lukujono. Koska 
$\seq{b_n}$ on myös ilmeisen rajoitettu lukujono, niin $b_n \kohti b\in\R$
(Lause \ref{monotoninen ja rajoitettu jono}), ja koska $\,b_n>1\ \forall n$, niin $\,b \ge 1$
(Propositio \ref{jonotuloksia} (V1) reaalilukujonoille). Tässä vaihtoehto $b>1$ johtaisi
loogiseen ristiriitaan: $\,a=b_n^n \ge b^n\kohti\infty$. Siis $\,\lim_n b_n=1$. \loppu 

\subsection{*Lauseiden \ref{R on kunta} ja \ref{suppeneminen kohti reaalilukua} todistukset}

\vahv{Lause \ref{R on kunta}}. \ Kunta-aksioomista todistetaan esimerkkinä ainoastaan 
aksioomien (K4) (kertolaskun liitäntälaki) ja (K9) (käänteisalkio) voimassaolo, muut
jätetään harjoitustehtäväksi (Harj.teht.\,\ref{H-I-9: R:n kunta-aksioomat}). 

\fbox{K4} \ Olkoon $\x = \seq{x_n},\ \y = \seq{y_n},\ \z = \seq{z_n}$ reaalilukuja ja olkoon
edelleen $\x\y = \breve{a} = \seq{a_n} \in \R\ $ ja $\ \y\z = \breve{b} = \seq{b_n} \in \R$. 
Tällöin aksiooman (K4) sisältö on Määritelmien \ref{reaalilukujen laskutoimitukset}
ja \ref{jonon raja - desim} perusteella
\[
\breve{a}\z\ =\ \x\breve{b} \quad \ekv \quad \lim_n\,(a_n z_n - x_n b_n)\ =\ 0.
\]
Kirjoitetaan tässä
\[
a_n z_n - x_n b_n\ =\ (a_n - x_n y_n) z_n + x_n (y_n z_n - b_n).
\]
Koska tässä $a_n - x_n y_n \kohti 0\ $ ja $\ y_n z_n - b_n \kohti 0$ (Määritelmät 
\ref{reaalilukujen laskutoimitukset} ja \ref{jonon raja - desim}) ja jonot $\seq{x_n}$
ja $\seq{z_n}$ ovat rajoitettuja, niin väite $a_n z_n - x_n b_n \kohti 0$ seuraa Lauseen
\ref{raja-arvojen yhdistelysäännöt} ja Proposition \ref{jonotuloksia} (V3) perusteella.

\fbox{K9} \ Luvun $\x = \seq{x_n} \in \R$ käänteisluvun $\x^{-1}$ konstruoimiseksi olkoon 
\mbox{$\x \neq 0$}, jolloin jollakin $m \in \N$ on oltava
$\,\abs{x_n} \ge \abs{x_m} > 0\ \ \forall n \ge m$.
Tällöin jono \seq{\,x_m^{-1}, x_{m+1}^{-1}, \ldots\,}\ on monotoninen ja rajoitettu, jolloin
on olemassa reaaliluku $\,\y = \seq{y_n}\,$ siten, että $\,x_n^{-1}-y_n \kohti 0$ 
(Lause \ref{monotoninen ja rajoitettu jono}). Proposition \ref{jonotuloksia} (V3) ja
Määritelmän \ref{reaalilukujen laskutoimitukset} perusteella päätellään tällöin
\[
x_n^{-1} - y_n \kohti 0 \quad \ekv \quad 1 - x_n y_n \kohti 0 \quad \ekv \quad \x\y = 1.
\]
Siis $\y = \x^{-1}$ ja aksiooma (K9) on näin ollen voimassa. \loppu

Järjestyrelaation aksiooman (J1) voimassaolo on jo todettu. Aksioomien (J2) ja (J4)
todentaminen jätetään harjoitustehtäväksi
(Harj.teht.\,\ref{H-I-9: R:n järjestysominaisuuksia}cd). Todistetaan siis ainoastaan aksiooman
(J3) voimassaolo.
\begin{Prop} \label{R:n aksiooma (J3)} Reaaliluvuille pätee: Jos $\x<\y$, niin
$\x+\z<\y+\z\ \forall \z\in\R$.
\end{Prop}
\tod Olkoon $\x=\seq{x_n}\in\R$, $\y=\seq{y_n}\in\R$, $\z=\seq{z_n}\in\R$ ja merkitään
$\x+\z=\breve{a}=\seq{a_n}\in\R$ ja $\y+\z=\breve{b}=\seq{b_n}\in\R$, jolloin
$x_n+z_n-a_n \kohti 0$ ja $y_n+z_n-b_n \kohti 0$, kun $n\kohti\infty$
(Määritelmä \ref{reaalilukujen laskutoimitukset}). Väitetään: Jos $\x<\y$, niin
$\breve{a}<\breve{b}$. Tehdään vastaoletus: $\breve{a}=\breve{b}$ tai 
$\breve{a}>\breve{b}$. Ensimmäisessä vaihtoehdossa on $\lim_n(a_n-b_n)=0$
(Määritelmä \ref{samastus DD}), jolloin oletusten ja Lauseen 
\ref{raja-arvojen yhdistelysäännöt} perusteella seuraa
\[
x_n-y_n \,=\, (x_n+z_n-a_n)-(y_n+z_n-b_n)+(a_n-b_n) \,\kohti\, 0.
\]
Siis $\x=\y$ (Määritelmä \ref{samastus DD}), mutta tämä on looginen ristiriita,
koska (J1) on voimassa ja oletettiin $\x<\y$. Toisessa vaihtoehdossa
($\breve{a}>\breve{b}$) on $a_n \ge b_n\ \forall n$
(Harj.teht.\,\ref{H-I-9: R:n järjestysominaisuuksia}a) ja samoin
$x_n \le y_n\ \forall n$ (koska $\x<\y$), joten seuraa
\begin{align*}
c_n \,&=\, (x_n+z_n-a_n)-(y_n+z_n-b_n) \\ 
      &=\, (x_n-y_n)+(b_n-a_n) \,\le\, x_n-y_n \,\le\, 0\,\ \forall n.
\end{align*}
Koska tässä $c_n \kohti 0$, niin $x_n-y_n \kohti 0$ (Propositio \ref{jonotuloksia} (V2)),
joten on jälleen päädytty loogiseen ristiriitaan: $\x<\y$ ja $\x=\y$. Vaihtoehdot
$\breve{a}=\breve{b}$ ja $\breve{a}>\breve{b}$ on näin muodoin pois suljettu, joten on
oltava $\breve{a}<\breve{b}$. \loppu

\vahv{Lause \ref{suppeneminen kohti reaalilukua}}. \ Perustetaan todistus seuraavaan
väittämään, joka seuraa helposti Määritelmästä \ref{reaalilukujen järjestys}
(Harj.teht.\,\ref{H-I-9: R:n järjestysominaisuuksia}b).
\begin{Lem} \label{R:n järjestyslemma} Jos $\,\x=\seq{x_n}\in\R$, niin 
$\,\abs{\x-x_n} \le 10^{-n}\ \forall n$.
\end{Lem}
Oletetaan, että $a_n \kohti \x,\ \x=\seq{x_n}\in\R$ Määritelmän \ref{jonon raja} 
mukaisesti ja olkoon $\eps > 0$. Tällöin on olemassa indeksit $N_1,N_2 \in \N$ siten, että 
$\abs{a_n-\x} < \eps/2\,$ kun $n > N_1$ ja $10^{-n} < \eps/2\,$ kun $n > N_2$, jolloin 
(järjestetyn kunnan) kolmioepäyhtälön ja Lemman \ref{R:n järjestyslemma} perusteella
\[
\abs{a_n-x_n} \le \abs{a_n-\x} + \abs{\x-x_n} < \abs{a_n-\x} + 10^{-n} 
              < \eps, \quad \text{kun}\ n > \max\,\{N_1,N_2\} = N.
\]
Tässä $\eps > 0$ oli mielivaltainen, joten Määritelmän \ref{jonon raja} mukaan 
$a_n-x_n \kohti 0$, eli $\,a_n \kohti \x$ Määritelmän \ref{jonon raja - desim} mukaisesti.
Tämä todistaa väittämän ensimmäisen osan. Toinen osa todistetaan samankaltaisella 
päättelyllä. \loppu

\Harj
\begin{enumerate}

\item \label{H-I-9: R:n kunta-aksioomat} 
Näytä Määritelmään \ref{reaalilukujen laskutoimitukset} perustuen, että reaaliluvuille
ovat voimassa: \newline
a) yhteen- ja kertolaskun vaihdantalait (K1,K2) \newline 
b) yhteenlaskun liitäntälaki (K3) \newline
c) yhteen- ja kertolaskun osittelulaki (K5)

\item \label{H-I-9: R:n järjestysominaisuuksia}
Olkoon $\x=\seq{x_n}\in\R$, $\y=\seq{y_n}\in\R$ ja $z=\seq{z_n}\in\R$. Näytä, että Määritelmien
\ref{reaalilukujen järjestys} ja \ref{reaalilukujen laskutoimitukset} perusteella pätee \newline 
a) \ $\x<\y\ \ekv\ \x \neq \y\ \ja\ x_n \le y_n\ \forall n$ \hspace{5mm}
b) \ $\abs{\x-x_n} \le 10^{-n}\ \forall n$ \newline
c) \ $\x<\y\ \ja\ \y<\z\ \impl\ \x<\z$ \hspace{15mm}\,
d) \ $\x>0\ \ja\ \y>0\ \impl\ \x\y>0$

\item
Tutki, kuinka suuri virhe tehdään, kun laskettaessa \ a) $e+\pi$, \ b) $e^2/\pi$, \ 
c) $\pi^2-e^2$ \ c) $e^5\pi^6$ katkaistaan $e$ ja $\pi$ ensin $9$ merkitsevään numeroon
ja laskuoperaatioissa tulos samoin $9$ merkitsevään numeroon.

\item
Olkoon lukujärjestelmän kantaluku $=k$. Käyttäen kymmenjakoalgoritmia vastaavaa 
$k$-jakoalgoritmia konstruoi neljän merkitsevän numeron tarkkuudella \ a) $\sqrt[3]{11}$
binaarijärjestelmässä, \ b) $\sqrt{7}$ $3$-kantaisessa järjestelmässä.

\item (*)
Tiedetään, että $\,\sum_{k=1}^\infty k^{-2}=\pi^2/6$. \,Näytä tämän tiedon perusteella:
\vspace{1mm}\newline
$\D
\text{a)}\,\ \sum_{k=0}^\infty \frac{1}{(2k+1)^2}=\frac{\pi^2}{8} \qquad
\text{b)}\,\ \sum_{k=0}^\infty \frac{(-1)^k}{(k+1)^2}=\frac{\pi^2}{12}$

%\item (*)
%Näytä, että $\,\lim_n \sqrt[n]{n^k}=1\ \forall k\in\N$.

\item (*)
Olkoon $a_k \kohti a\ (k \kohti \infty)$, missä on desimaalilukumerkinnöin 
\[
a_k=x_0^{(k)}.\,d_1^{(k)}d_2^{(k)}\,\ldots =\seq{x_n^{(k)}} \in\R, \quad a
                                           =x_0.d_1d_2\ldots = \seq{x_n} \in\R.
\]
Todista seuraavat väittämät: \newline
a) \ Jos $a$ ei ole äärellinen desimaaliluku, niin jokaisella $n\in\N$ on olemassa indeksi 
$N_n\in\N$ siten, että \ $a_k = x_0.d_1 \ldots d_n \ldots\ \forall k>N_n$. \newline
b) \ a-kohdan väittämä ei ole tosi jokaisella $a\in\R$ (vastaesimerkki\,!). \newline
c) \ Jos $a$ on äärellinen desimaaliluku, niin on olemassa $m\in\N$ ja jokaisella $n \ge m$ 
indeksi $N_n\in\N$ siten, että $d_n^{(k)}\in\{0,9\}\ \forall k>N_n$. \newline
d) \ Jokaisella $a\in\R$ pätee: $\,\lim_k x_k^{(k)}=a$.

\end{enumerate}  % Reaaliluvut
\section{Cauchyn jonot} \label{Cauchyn jonot}
\alku

Tässä luvussa 'lukujono' tarkoittaa reaalilukujen jonoa. Tarkastelun kohteena on lukujonojen
(myös rationaalilukujonojen) teorian edelleen avoin kysymys, joka kuuluu:
Täsmälleen millaisilla, jonoa itseään koskevilla ehdoilla lukujono $\seq{a_n}$ suppenee, ts.\
raja-arvo $\lim_na_n=a$ on olemassa Määritelmän \ref{jonon raja} mukaisesti reaalilukuna?
Toistaiseksi tunnetut suppenemisen ehdot Lauseissa \ref{monotoninen ja rajoitettu jono} ja
\ref{monotoninen ja rajoitettu jono - yleistys} ovat riittäviä, eivät välttämättömiä.
Etsittäessä suppenemiskysymyksen täydellista ratkaisua, ts.\ sekä välttämättömiä että riittäviä
ehtoja suppnemiselle, ratkaisun avaimeksi osoittautuu
\index{Cauchyn!a@jono|emph} \index{lukujono!f@Cauchyn jono|emph}
\begin{Def} \label{Cauchyn jono} (\vahv{Cauchyn}\footnote[2]{Ranskalainen matemaatikko 
\hist{Augustin Louis Cauchy} (1789-1857) kuuluu matemaatikoiden tähtikaartiin kautta
aikojen. Cauchy täsmensi merkittävästi matematiikan käsitteistöä ja loi pohjaa uusille
tutkimussuunnille. Hänen laaja tuotantonsa ulottui myös fysiikkaan, mm.\ kiinteän aineen
kimmoteoriaan. \index{Cauchy, A. L.|av}} 
\vahv{jono}) Lukujono $\{a_n\}$ on \kor{Cauchy} eli \kor{Cauchyn jono}, jos jokaisella
$\eps > 0$ on olemassa $N\in\N$ siten, että pätee
\[
\abs{a_n - a_m}\ < \eps \quad \text{kun}\ \,n,m > N.
\]
\end{Def}

\begin{Exa} Näytä, että sarjan $\,\sum_{k=0}^\infty (-1)^k/(k+1)^2\,$ osasummien jono
\[
\seq{s_n} = \left\{1,\,1-\frac{1}{2^2},\,1-\frac{1}{2^2}+\frac{1}{3^2},\ldots\right\}
          = \left\{1,\,\frac{3}{4},\,\frac{7}{9},\ldots\right\}
\]
on Cauchyn jono. \end{Exa}
\ratk Jos $\,0 \le m < n$, niin kolmioepäyhtälöä soveltaen voidaan arvioida \newline
(ks.\ Harj.teht.\,\ref{monotoniset jonot}:\ref{H-I-8: kaksi sarjaa})
\[
\abs{s_n-s_m} \,=\, \left|\sum_{k=m+1}^n \frac{(-1)^k}{(k+1)^2}\right|
              \,\le\, \sum_{k=m+1}^n \frac{1}{(k+1)^2} \,<\, \frac{1}{m+1}\,.
\]
Näin ollen jos $\eps>0$ ja valitaan $N\in\N$ siten, että $1/(N+1)<\eps$ (mahdollista
jokaisella $\eps>0$), niin
\[
\abs{s_n-s_m} \,<\, \max\left\{\frac{1}{n+1},\,\frac{1}{m+1}\right\} \,<\, \eps,
                    \quad \text{kun}\ n,m >N.
\]
Siis $\seq{s_n}$ on Cauchy. \loppu
\begin{Exa} Jos $a \neq 0$, niin lukujono $\seq{a_n}=\seq{(-1)^n a}$ ei ole Cauchy.
Nimittäin jos $0 < \eps \le 2\abs{a}$, niin Määritelmän \ref{Cauchyn jono} ehto ei
toteudu millään $N\in\N$, koska $\abs{a_n - a_m}=2\abs{a} \ge \eps\,$ aina kun $n-m$
on pariton. \loppu \end{Exa}
 
Cauchyn jonoilla on samankaltaisia ominaisuuksia kuin suppenevilla jonoilla. Näytetään
ensinnäkin:
\begin{Lause} \label{suppeneva jono on Cauchy} Suppeneva jono on Cauchy. \end{Lause}
\tod Jos $\lim_na_n=a\in\R$ Määritelmän \ref{jonon raja} mukaisesti, niin jokaisella $\eps>0$
on olemassa $N\in\N$ siten, että pätee $\abs{a_n - a} < \eps/2\,$ kun $n>N$. Tällöin on
kolmioepäyhtälön nojalla
\begin{align*}
\abs{a_n - a_m}\ &=\ \abs{(a_n - a) + (a - a_m)} \\
                 &\le\ \abs{a_n-a}+\abs{a_m-a}\ <\ \frac{\eps}{2} + \frac{\eps}{2} = \eps, 
                                                   \quad \text{kun}\ n,m > N. \loppu 
\end{align*}%
Suppenevia lukujonoja koskevilla Lauseilla \ref{suppeneva jono on rajoitettu} ja 
\ref{raja-arvojen yhdistelysäännöt} on seuraavat vastineet Cauchyn jonoille:
\begin{Lause} \label{Cauchyn jono on rajoitettu} Cauchyn jono on rajoitettu.
\end{Lause}
\begin{Lause} \label{Cauchyn jonojen yhdistelysäännöt} 
\index{Cauchyn!b@jonojen yhdistely|emph} (\vahv{Cauchyn jonojen yhdistelysäännöt})
Jos lukujonot $\{a_n\}$ ja $\{b_n\}$ ovat Cauchyn jonoja ja $\lambda\in\R$, niin myös lukujonot
$\{a_n + b_n\}$, $\{\lambda a_n\}$ ja $\{a_n b_n\}$ ovat Cauchyn jonoja. Jos lisäksi 
$\abs{b_n} \ge \delta > 0\ \ \forall n$, niin myös $\{a_n/b_n\}$ on Cauchyn jono.
\end{Lause}
Näiden lauseiden todistukset (jotka sivuutetaan, ks.\ Harj.teht.\,\ref{H-I-9: todistuksia})
ovat hyvin samanlaisia kuin mainittujen vastinlauseiden. Esimerkiksi Lauseen 
\ref{Cauchyn jono on rajoitettu} väittämään päädytään, kun vertailukohdaksi otetaan raja-arvon
sijasta jonon termi $a_m$, missä indeksi $m$ valitaan siten, että pätee 
$\abs{a_n - a_m} < 1$ kun $n>m$, vrt.\ Määritelmä \ref{Cauchyn jono} ja Lauseen 
\ref{suppeneva jono on rajoitettu} todistus.

Päätulos jatkossa on
\begin{*Lause} \label{Cauchyn kriteeri} \index{Cauchyn!c@suppenemiskriteeri|emph}
\index{suppeneminen!a@lukujonon|emph} (\vahv{Cauchyn suppenemiskriteeri})
Reaalilukujono $\seq{a_n}$ suppenee kohti reaalilukua täsmälleen kun $\seq{a_n}$ on Cauchy. 
\end{*Lause}
\jatko\jatko \begin{Exa} (jatko) Lauseen \ref{Cauchyn kriteeri} perusteella esimerkin
sarja suppenee. Osasummia riitttävän pitkälle laskemalla selviää, että sarjan summa katkaistuna
$10$ merkitsevään numeroon on
\[
\sum_{k=0}^\infty \frac{(-1)^k}{(k+1)^2} \,=\, 0.8224670334\ldots \loppu
\]
\end{Exa} \seur
Koska jo tiedetään, että jokainen suppeneva jono on Cauchy (Lause 
\ref{suppeneva jono on Cauchy}), niin Lauseen \ref{Cauchyn kriteeri} todistamiseksi riittää
näyttää todeksi implikaatio
\begin{equation} \label{Cauchyn jonon suppenevuus}
\seq{a_n}\ \text{on Cauchy}\ \ \impl\ \ \seq{a_n}\ \text{suppenee}. \tag{$\star$}
\end{equation}
Todistus on useampivaiheinen. Ensin on tutkittava lukujonon raja-arvon määritelmää
tarkemmin \kor{osajonon} käsitteen pohjalta. Seuraavassa nämä tarkastelut ja niihin perustuva
väittämän \eqref{Cauchyn jonon suppenevuus} todistus erotetaan omaksi 
osaluvukseen\footnote[2]{Merkintä (*) luvun tai osaluvun otsikossa kertoo 
(tässä ja jatkossa), että kyse on näkökulmaa laajentavasta --- usein myös muuta tekstiä 
haastavammasta --- tekstin osasta.}.

\subsection{*Osajonot}

\begin{Def} \index{osajono|emph} \index{lukujono!g@osajono|emph} 
Lukujono $\{b_k\}_{k=1}^{\infty}$ on jonon $\{a_n\}_{n=m}^{\infty}$ \kor{osajono}
(engl.\ subsequence), jos on olemassa indeksit $\ m \le n_1 < n_2 < \ldots\ $ siten, että
\[
b_k = a_{n_k}, \quad k = 1,2, \ldots
\]  \end{Def}

\begin{Exa} Jos $\x = \{x_n\}_{n=p}^{\infty}$ on jaksollinen desimaaliluku, niin jollakin
$m \in \N$ ja riittävän suurella $k \in \N$ osajono
\[
\{x_k, x_{k+m}, x_{k+2m}, \ldots \}\ =\ \{x_{k+(l-1)m}\}_{l=1}^{\infty}
\]
on geometrinen sarja, vrt.\ Luku \ref{jonon raja-arvo}. Tässä siis osajonon indeksinä on $l$ ja
$n_l = k + (l-1)m$. \loppu 
\end{Exa}
\begin{Exa} Lukujono $\{\,a_n = (-1)^n,\ \ n=0,1,2, \ldots\,\}$ ei suppene, mutta sillä on
(kohti rationaalilukua) suppenevia osajonoja, esim.\ $\{\,a_0, a_2, a_4, \ldots\,\}$. \loppu 
\end{Exa}
\begin{Exa} Rajatta kasvavan lukujonon $\seq{n} = \{1,2,3,\ldots\}$ jokainen osajonokin on
rajatta kasvava. \loppu
\end{Exa}
Osajonon käsitettä valaisee hieman seuraava 'lämmittelylause'.
\begin{Lause} \label{suppenevat osajonot} $\lim_na_n=a\in\R$ täsmälleen kun $\lim_k b_k = a\,$
jokaiselle jonon $\{a_n\}$ osajonolle $\{b_k\}$.
\end{Lause}
\tod Väittämän osa \ \fbox{$\impl$} \ on ilmeinen, sillä Määritelmän \ref{jonon raja} 
perusteella osajono suppenee vähintään yhtä nopeasti kuin itse jono. Tästä huolimatta osa 
\fbox{$\limp$} on vieläkin ilmeisempi, sillä osajonoksi kelpaa myös itse jono $\{a_n\}$. \loppu

Seuraava tulos, jossa osajonon käsite on keskeinen, on avain jatkon kannalta. Tuloksessa 
tulkitaan proposition $a_n \kohti a$ negaatio ($a_n \not\kohti a$) Määritelmään \ref{jonon raja}
perustuen. --- Huomattakoon, että kyse ei ole negaatiosta 'ei suppene' (kohti mitään lukua),
koska $a_n \not\kohti a$ on tosi myös, kun $a_n \kohti b \neq a$.
\begin{*Lause} \label{negaatioperiaate} Jos $\seq{a_n}$ on lukujono ja $a\in\R$, niin
$\,a_n \not\kohti a\,$ täsmälleen kun $\exists\,\eps > 0\,$ ja osajono 
$\,\{\,a_{n_1},a_{n_2}, \ldots\,\}\,$ siten, että 
$\,\abs{a_{n_k}-a} \ge \eps\ \ \forall k \in \N$.
\end{*Lause}
\tod Lähdetään raja-arvon määritelmästä (Määritelmä \ref{jonon raja})
\begin{align*}
a_n \kohti a \quad                 &\ekv \quad \forall \eps > 0\ \,\text{pätee}\ [\,\ldots\,]
\intertext{ja ryhdytään negaation purkuun: Ensinnäkin (vrt.\ Luku \ref{logiikka})} 
a_n \not\kohti a \quad             &\ekv \quad \text{jollakin}\ \eps > 0\ \,
                                               \text{ei päde}\ [\,\ldots\,]
\intertext{Tässä on $\,\ [\,\ldots\,]\,=\,[\ \exists N\ \in \N\ $ siten, että 
$\ \{\,\ldots\,\}\ ]$, \ joten}
\text{ei päde}\ [\,\ldots\,] \quad &\ekv \quad \not\exists N \in \N\ \ 
                                               \text{siten, että}\ \{\,\ldots\,\} \\[3mm]
                                   &\ekv \quad ei\,\{\,\ldots\,\}\ \ \forall N \in \N.
\intertext{Tässä on $\,\ \{\,\ldots\,\}\,=\,\{\,\abs{a_n-a} < \eps\ \ \forall n>N\,\}$, \ 
joten}
ei\,\{\,\ldots\,\} \quad           &\ekv \quad \abs{a_n-a} \ge \eps\ \ \text{jollakin}\ n>N.
\end{align*}
Siis on päätelty:
\[
a_n \not\kohti a \quad \ekv \quad \text{jollakin}\ \eps>0\ 
    \text{pätee:}\quad \forall\ N \in \N\ (\,\abs{a_n-a} \ge \eps\ \text{jollakin}\ n>N\,)\,.
\]
Kyseisellä $\eps$ voidaan nyt  menetellä seuraavasti:
\begin{itemize}
\item[1.] Valitaan $\,N=1\ \,\,$   ja $\,n_1>N\,$ siten, että $\,\abs{a_{n_1}-a} \ge \eps$.
\item[2.] Valitaan $\,N=n_1\,$     ja $\,n_2>N\,$ siten, että $\,\abs{a_{n_2}-a} \ge \eps$.
\item[$\ \vdots$]
\end{itemize}
Konstruktiota voidaan jatkaa loputtomasti, joten saadaan osajono
$\{\,a_{n_1},a_{n_2}, \ldots\,\}$ (ks.\ kuva alla). Tällä on vaadittu ominaisuus, joten on
todistettu väittämän osa \ \fbox{$\impl$}\,. Osa \ \fbox{$\limp$} \ on ilmeinen raja-arvon
määritelmän perusteella. \loppu

\begin{figure}[H]
\import{kuvat/}{kuvaI-1.pstex_t}
\end{figure}

Kun Lausetta \ref{negaatioperiaate} sovelletaan Cauchyn jonoon, saadaan johdetuksi 
mielenkiintoinen tulos.
\begin{*Lause} \label{Cauchyn jonon vaihtoehdot} Jos lukujono $\{a_n\}$ on Cauchyn jono, niin 
jokaisella $x\in\R$ on voimassa täsmälleen yksi seuraavista vaihtoehdoista:
\begin{itemize}
\item[1.] $\,a_n \kohti x$.
\item[2.] $\exists\,\eps>0$ ja $N \in \N$ siten, että $\ a_n\ < x - \eps\ $ aina kun $\ n>N$.
\item[3.] $\exists\,\eps>0$ ja $N \in \N$ siten, että $\ a_n\ > x + \eps\ $ aina kun $\ n>N$.
\end{itemize} 
\end{*Lause}
\tod Ensinnäkin on joko $a_n \kohti x$ (vaihtoehto 1) tai $a_n \not\kohti x$.
Jälkimmäisessä vaihtoehdossa on Lauseen \ref{negaatioperiaate} mukaan löydettävissä luku 
$\eps>0$ ja osajono $\{\,b_k = a_{n_k},\ k = 1, 2, \ldots\,\}$ siten, että 
$\,\abs{b_k - x} \ge 2 \eps\ \,\forall k$. (Tässä on luvun $\eps$ tilalle kirjoitettu $2 \eps$
mukavuussyistä.) Koska $\{a_n\}$ on Cauchyn jono, on myös olemassa lukua $\eps$ vastaava
indeksi $N$ siten, että pätee $\abs{a_n - a_m} < \eps$ kun $n,m > N$. Valitaan nyt osajonossa
$\{b_k\}$ indeksi $k$ siten, että $n_k > N$, ja kirjoitetaan
\[
a_n\ =\ b_k + (a_n - b_k).
\]
Tässä on $b_k = a_m,\ m = n_k > N$, joten $\abs{a_n - b_k} < \eps$, kun $n > N$. Myös oli
$\abs{b_k - x} \ge 2 \eps$, joten on oltava joko $b_k \ge x + 2 \eps$ tai $b_k \le x - 2 \eps$.
Jos  $b_k \ge x + 2 \eps$ (kuten kuvassa alla), niin arvioidaan
\[
a_n\ \ge x + 2 \eps - \abs{a_n - a_k}\ >\ x + 2 \eps - \eps\ =\ x + \eps, \quad \text{kun}\ n>N.
\]
Jos  $b_k \le x - 2 \eps$, niin arvioidaan
\[
a_n\ \le x - 2 \eps + \abs{a_n - a_k}\ <\ x - 2 \eps + \eps\ =\ x -\eps, \quad \text{kun}\ n>N.
\]
Lause on näin todistettu. \loppu
\vspace{5mm}
\begin{figure}[H]
\import{kuvat/}{kuvaI-2.pstex_t}
\end{figure}

Lauseen \ref{Cauchyn jonon vaihtoehdot} mukaan Cauchyn jonon on 'valittava puolensa' annetun
luvun --- esimerkiksi rationaaliluvun --- suhteen, ellei ko.\ luku ole jonon raja-arvo.
Käyttäen tätä tulosta 'työhevosena' voidaan nyt todistaa väittämä
\eqref{Cauchyn jonon suppenevuus} ja siis myös Lause \ref{Cauchyn kriteeri}. Todistus on
muuten olennaisesti sama kuin Lauseen \ref{monotoninen ja rajoitettu jono} todistus, paitsi
että monotonisuuden sijasta vedotaan Lauseeseen \ref{Cauchyn jonon vaihtoehdot}. 
\begin{*Lause} \label{Cauchyn jono suppenee} Cauchyn jono suppenee. \end{*Lause}
\tod Olkoon $\seq{a_n}$ Cauchyn jono. Jos $a_n \kohti x\in\Q$, niin lauseen väittämä on tosi
, joten voidaan olettaa, että $\seq{a_n}$ ei suppene kohti rationaalilukua. Tällöin on
ensinnäkin jokaisen kokonaisluvun $x \in \Z$ kohdalla voimassa Lauseen
\ref{Cauchyn jonon vaihtoehdot} vaihtoehdoista joko 2 tai 3. Koska Cauchyn jono on rajoitettu
(Lause \ref{Cauchyn jono on rajoitettu}), niin silloin on löydettävissä yksikäsitteinen
$x_0 \in \Z$ siten, että $a_n > x_0 + \eps_1$ kun $n > N_1$ ja $a_n < x_0 + 1 - \eps_2$ kun
$n > N_2$ (tässä $\eps_1,\eps_2 > 0$ ja $N_1,N_2 \in \N$), jolloin 
\[
x_0\ <\ a_n\ <\ x_0 + 1, \quad \text{kun}\ n>N = \max \{N_1,N_2\}.
\]
Oletetaan jatkossa, että $x_0 \ge 0$ (muussa tapauksessa tutkitaan jonoa $\seq{-a_n}$).
Jatkamalla päättelyä vastaavasti on löydettävissä yksikäsitteiset
$d_k \in \{0,\ldots,9\}$, $ k = 1,2,\ldots$ ja vastaavat äärelliset desimaaliluvut 
$x_k = x_0.d_1 \ldots d_k$ ja indeksit $N_k \in \N$ siten, että pätee
\[
x_k\ <\ a_n\ <\ x_k + 10^{-k}, \quad \text{kun}\ n>N_k, \quad k=0,1,2,\ldots
\]
Tämän mukaan on $\lim_n (a_n-x_n)=0$ eli $\lim_n a_n =\x=\seq{x_n}\in\R$ Määritelmien
\ref{jonon raja - desim} ja \ref{reaaliluvut desimaalilukuina} mukaisesti. Lauseen
\ref{suppeneminen kohti reaalilukua} perusteella $\lim_na_n=\x$ myös Määritelmän
\ref{jonon raja} mukaisesti. \loppu

\subsection{Reaaliluvut Cauchyn jonoina}
\index{reaalilux@reaalilukujen!a@laskuoperaatiot|vahv}
\index{laskuoperaatiot!b@reaalilukujen|vahv}
\index{reaalilux@reaalilukujen!b@järjestysrelaatio|vahv}
\index{jzy@järjestysrelaatio!c@$\R$:n|vahv}

Määritelmä \ref{jonon raja - desim} tulkitsee äärettömän desimaaliluvun 'suppenevan itseensä'
lukujonona. Tällainen näkökulma voidaan ottaa yleisemminkin rationaalilukujen Cauchyn jonoihin,
jolloin saadaan seuraava vaihtoehtoinen määritelmä reaaliluvuille (vrt.\ Määritelmät
\ref{reaaliluvut desimaalilukuina} ja \ref{samastus DD}).
\begin{Def} \label{reaaliluvut Cauchyn jonoina} \index{reaaliluvut!c@Cauchyn jonoina|emph}
(\vahv{Reaaliluvut Cauchyn jonoina}) Reaaliluvut ovat rationaalilukujen Cauchyn jonoja. Kaksi
reaalilukua $a=\seq{a_n}$ ja $b=\seq{b_n}$ ovat samat täsmälleen kun $\lim_n\,(a_n - b_n) = 0$. 
\end{Def}
Määritelmän mukaisesti reaaliluvut ovat rationaalilukujen Cauchyn jonojen muodostamia 
\pain{ekvivalenssiluokkia} (tai samastusluokkia, vrt.\ Luku \ref{logiikka}), joiden sisällä 
eri jonot samastetaan määritelmän kriteerillä. Jokaiseen samastusluokkaan kuuluu äärettömän
monta erilaista jonoa; esim.\ jonot $\seq{a_n}$ ja $\seq{a_n + n^{-k}}$ kuuluvat samaan
luokkaaan jokaisella $k\in\N$. Ellei kyseessä ole äärellinen desimaaliluku, on samaan 
samastusluokkaan kuuluvista Cauchyn jonoista täsmälleen yksi Määritelmän
\ref{reaaliluvut desimaalilukuina} mukainen reaaliluvun 'edustaja', ts.\ ääretön desimaaliluku.

Määritelmään \ref{reaaliluvut Cauchyn jonoina} perustuen on laskuoperaatioiden määrittely 
reaaliluvuille varsin yksinkertaista: Jos $a=\seq{a_n}$ ja $b=\seq{b_n}$ ovat reaalilukuja 
(eli $\seq{a_n}$ ja $\seq{b_n}$ ovat rationaalilukujen Cauchyn jonoja), niin myös jonot
$\seq{a_n + b_n}$ ja $\seq{a_n b_n}$ ovat Cauchyn jonoja 
(Lause \ref{Cauchyn jonojen yhdistelysäännöt}) ja edustavat sellaisenaan lukuja $a+b$ ja $ab$.
Vastaavasti jos $b \neq 0$ (eli $b_n \not\kohti 0$), niin jostakin indeksistä $n=m$ alkaen on
$\abs{b_n} \ge \delta>0$ (Lause \ref{Cauchyn jonon vaihtoehdot}), jolloin $\seq{a_n/b_n}$ on
ko.\ indeksistä eteenpäin Cauchy (Lause \ref{Cauchyn jonojen yhdistelysäännöt}) ja edustaa
lukua $a/b$.

Myös järjestysrelaation määrittely käy Cauchyn jonojen avulla helposti, sillä jos
$a=\seq{a_n}$, niin vaihtoehdot $a=0$, $a>0$ ja $a<0$ määräytyvät asettamalla $x=0$ Lauseessa
\ref{Cauchyn jonon vaihtoehdot}\,: $\,a=0$ vaihtoehdossa 1, $a<0$ vaihtoehdossa 2, ja $a>0$
vaihtoehdossa 3. Yleisemmin lukujen $a$ ja $b$ suuruusjärjestys ratkaistaan vertaamalla lukua
$a-b=\seq{a_n-b_n}$ lukuun $0$ mainitulla tavalla.

Jos Määritelmä \ref{reaaliluvut Cauchyn jonoina} tulkitaan laskennallisesti, niin yksittäistä
Cauchyn jonoa voi pitää algoritmina, joka tuottaa reaaliluvulle jonon rationaalisia 
approksimaatioita. Algoritmi on yleisesti sitä parempi, mitä nopeammin jono suppenee 
(tai 'suippenee') suhteessa tarvittavien (rationaalisten) laskuoperaatioiden määrään. Cauchyn
jonoille voidaan siis asettaa myös laatukriteereitä!
\begin{Exa} Eräs rationaalilukujen Cauchyn jonojen ekvivalenssiluokka on $\sqrt{2}$. Tähän
luokkaan kuuluva ääretön desimaaliluku, eli lukujono $\{1,1.4,1.414,\ldots\}$ vastaa
algoritmina edellisen luvun Esimerkin \ref{neliöjuuri 2} kymmenjakoalgoritmia. Toinen samaan
ekvivalenssiluokkaan kuuluva lukujono on Luvun \ref{monotoniset jonot} Esimerkissä
\ref{sqrt 2 algoritmina} tarkasteltu. Tämä on algoritmina kymmenjakoa selvästi tehokkaampi.
\loppu \end{Exa}

\subsection{*Bolzanon--Weierstrassin lause}

Asiayhteyden vuoksi esitettäköön vielä osajonon käsitteeseen liittyvä lause, joka kuuluu
modernin matemaattisen analyysin peruskiviin.

\begin{*Lause} \label{B-W} \index{Bolzanon--Weierstrassin lause|emph}
(\vahv{Bolzano-Weierstrass}\footnote[2]{Bolzanon-Weierstrassin lauseella on monia muotoja.
Perusidean esitti ensimmäisenä t\v{s}ekkiläinen matemaatikko, filosofi ja pappi
\hist{Bernard Bolzano} (1781-1848) v.\ 1817, mutta tulos jäi vähälle huomiolle. Saksalainen
\hist{Karl Weierstrass} (1815-1897), jota on pidetty 'modernin matemaattisen analyysin isänä',
todisti lauseen myöhemmin tuntematta Bolzanon työtä. \index{Bolzano, B.|av} 
\index{Weierstrass, K.|av}}) \ Jokaisella rajoitetulla reaalilukujonolla on suppeneva osajono. 
\end{*Lause}
\tod Todistus on perusidealtaan puolituskonstruktio, jossa looginen päättely nojaa
kaksipaikkaiseen predikaattiin
\[
Q(x,y): \quad x \le a_n \le y \quad 
\text{äärettömän monella indeksin $n$ arvolla.}
\]
Koska $\{a_n\}$ on rajoitettu jono, on löydettävissä luvut $a$ ja $b=a+L$ siten, että $Q(a,b)$
on tosi. Valitaan jokin indeksi $n_1$ siten, että $a \le a_n \le b$ kun $n=n_1$, ja tutkitaan
seuraavaksi, onko $Q(a,(a+b)/2)$ tosi. Jos on, asetetaan $b$:n uudeksi arvoksi $(a+b)/2$. Jos
ei, on proposition $Q((a+b)/2,b)$ oltava tosi (koska $Q(a,b)$ oli tosi). Tässä tapauksessa
asetetaan $a$:n uudeksi arvoksi $(a+b)/2$. Toisen luvuista $a,b$ tultua näin uudelleen
määritellyksi tiedetään, että $Q(a,b)$ on jälleen tosi. Etsitään nyt indeksi $n_2 > n_1$ siten,
että $a \le a_n \le b$ kun $n=n_2$ (mahdollista, koska $Q(a,b)$ oli tosi). Etenemällä samalla
tavoin saadaan konstruoiduksi alkuperäisen jonon osajono
$\{\,b_k = a_{n_k},\ k = 1,2,\ldots\,\}$, jolla on konstruktion perusteella ominaisuus
$\abs{b_k - b_l}\ \le\ 2^{-N} L \quad \text{kun}\ k,l > N$. Näin ollen jos jokaisella $\eps>0$
valitaan $N\in\N$ siten, että $\,2^{-N}L<\eps$, niin 
$\abs{b_k-b_l}<\eps$ kun $\,k,l>N$. Määritelmän \ref{Cauchyn jono} mukaan $\{b_k\}$ on Cauchyn
jono ja näin ollen suppee (Lause \ref{Cauchyn jono suppenee}). \loppu

\subsection{*Epäkonstruktiivinen päättely}
\index{epzy@epäkonstruktiivinen päättely|vahv}

Lauseen \ref{B-W} todistukseen kätkeytyy filosofinen ongelma,
joka esiintyy matemaattisessa ajattelussa yleisemminkin: Jos predikaatissa $Q(x,y)$
ajatellaan $x$ ja $y$ kiinnitetyiksi (sidotuiksi), niin proposition $P = Q(x,y)$ totuusarvon
selville saamiseksi on käytävä kirjaimellisesti läpi j\pain{okainen} jonon $\seq{a_n}$ termi,
ts.\ on suoritettava \pain{äärettömän} \pain{monta} lukujen vartailua. Tämä ei tietenkään ole 
käytännössä mahdollista. Matemaattisessa analyysissä katsotaan kuitenkin myös tällainen nk.\ 
\kor{epäkonstruktiivinen} päättely luvalliseksi. (Viime kädessä 'luvan' antavat joukko-opin 
perusaksioomat.) Epäkonstruktiiviselle todistukselle on tunnus\-omaista, että todistusta ei voi
seurata 'laskemalla', ellei käytettävissä ole jotakin lisätietoa, ts.\ ellei tehdä 
lisäoletuksia. Esimerkiksi aiemmin esitetty Lauseen \ref{monotoninen ja rajoitettu jono}
todistus on myös epäkonstruktiivinen, sillä tämäkin sisältää kuvitelman kaikkien jonon termien
vertaamisesta kiinteään lukuun. Epäkonstruktiivisia ovat itse asiassa myös Lauseiden 
\ref{negaatioperiaate} ja \ref{Cauchyn jonon vaihtoehdot} todistukset, sillä 
'olemassa olevaksi' väitetty luku $\eps$ konstruoidaan todistuksissa vain loogisena 
välttämättömyytenä. Mitään laskennallisesti toimivaa ideaa luvun määrittämiseksi ei anneta ---
eikä tehtyjen yleisten oletusten puitteissa olisi mahdollistakaan antaa.

Kaikissa mainituissa todistuksissa filosofiset ongelmat poistuvat (tai ainakin vähenevät), jos
oletetaan tarkasteltava lukujono sellaiseksi, että todistuksen sisältämät loogiset askeleet
voidaan toteuttaa (yksi kerrallaan) \pain{äärellisellä} \pain{määrällä} (viime kädessä)
\pain{rationaalisia} laskuoperaatioita tai vertailuja. Tällaista lukujonoa voi kutsua
vaikkapa \kor{ennustettava}ksi. Esimerkiksi Lukujen \ref{jono} --- \ref{reaaliluvut}
esimerkkijonot, tai yleisemmin kaikki 'ennustettavasti määritellyt' yksittäiset lukujonot, ovat
tällaisia. Lisäoletuksen ollessa voimassa voidaan todistuksissa esitettyjä ajatuskonstruktioita
seurata myös laskennallisesti. Lauseen \ref{negaatioperiaate} tapauksessa tämä tapahtuu esim.\ 
hakemalla käypää $\eps$:n arvoa jonosta $\,\{\,\eps_n = 10^{-n},\ n = 1,2, \ldots\,\}$.
Lisäoletuksen perusteella voidaan millä tahansa kiinteällä $n \in \N$ ratkaista äärellisellä
laskutyöllä, onko $\eps = \eps_n$ käypä valinta. Ellei ole, asetetaan $n \leftarrow n+1$, ja 
jatketaan laskemista. Ennustettavuusoletuksen perusteella laskun tiedetään päättyvän johonkin
indeksiin, vaikkei etukäteen tiedetä, mihin. 


\Harj
\begin{enumerate}

\item \label{H-I-9: todistuksia}
a) Todista Lause \ref{Cauchyn jono on rajoitettu}. \newline
b) Todista Lauseen \ref{Cauchyn jonojen yhdistelysäännöt} väittämät koskien lukujonoja
$\seq{a_n+b_n}$ ja $\seq{\lambda a_n}$. \newline
c) Todista Lauseen \ref{Cauchyn jonojen yhdistelysäännöt} väittämä koskien lukujonoa
$\seq{a_n b_n}$. \newline
d) Todista: Jos $\seq{a_n}$ on Cauchy ja $\abs{a_n} \ge \delta>0\ \forall n$, niin 
$\seq{a_n^{-1}}$ on Cauchy. \newline
e) Todista Lauseen \ref{Cauchyn jonojen yhdistelysäännöt} väittämä koskien lukujonoa
$\seq{a_n/b_n}$.

\item
Olkoon $\{a_k,\ k=0,1,\ldots\}$ lukujono ja $b_n=\sum_{k=0}^n 2^{-k}a_k,\ n=0,1,\ldots\,$
Näytä, että jos $\seq{a_k}$ on rajoitettu, niin $\seq{b_n}$ on Cauchy.

\item
Olkoon $\seq{a_n}$ rationaalilukujono. Näytä todeksi tai epätodeksi: \newline
a) \ $a_n^3 \kohti 4\,\ \impl\,\ \seq{a_n}$ on Cauchy, $\quad$
b) \ $a_n^4 \kohti 3\,\ \impl\,\ \seq{a_n}$ on Cauchy.

\item
Jos $\seq{b_k}$ on jonon $\seq{a_n,\ n=1,2, \ldots}$ suppeneva osajono, niin mitkä ovat
mahdolliset osajonon raja-arvot seuraavissa tapauksissa?
\[
\text{a)}\ \ a_n = \frac{(n+1)^n}{(-n)^n} \qquad 
\text{b)}\ \ a_n = \frac{n^3+(-1)^n(n-1)^3}{(n+2)^3+(-1)^n n^3}
\]

\item
a) Olkoon luvun $1/7\,$ $n$:s desimaali $=a_n$ ja luvun $1/17\,$ $n$:s desimaali $=b_n\,$.
Määritä lukujonon $\seq{a_n+b_n}$ suppenevien osajonojen mahdolliset raja-arvot. \
b) Olkoon $d_n =$ tunnetun jaksottoman desimaaliluvun $\pi=3.14159628..\,$ $n$:s desimaali.
Esitä kymmenen lukujonoa, joista vähintään kaksi on jonon $\seq{d_n}$ suppenevia osajonoja.
Perustele!

\item \label{H-I-9: suppenevat osajonot}
a) Näytä Lause \ref{suppenevat osajonot} päteväksi myös, kun $a=\infty$, \ b) Näytä, että
lukujono ei ole rajoitettu täsmälleen kun jonolla on osajono, joka on aidosti monotoninen ja
joko rajatta kasvava tai rajatta vähenevä.

\item (*)
Todista: \ Jos $\seq{a_n}$ ja $\seq{b_n}$ ovat rajoitettuja lukujonoja, niin on olemassa 
aidosti kasvava indeksijono $\seq{n_k}$ siten, että jonot $\seq{a_{n_k}\,,\ k=1,2, \ldots}$
ja $\seq{b_{n_k}\,,\ k=1,2, \ldots}$ ovat molemmat suppenevia.

\item (*) Käyttäen Cauchyn jonoihin perustuvaa reaaliluvun määritelmää näytä, että
$(\R,+,\cdot,<)$ on järjestetty kunta.

\item (*) \index{satunnaisluku(jono)}
\kor{Satunnaisluku}jono on rationaalilukujono $\{a_n,\ n=1,2,\ldots\}$, jolle pätee 
$0 \le a_n \le 1\ \forall n$ ja jolla on ominaisuus: Jos $0 \le x < y \le 1$ ja $N_n =$ niiden
indeksien $k\in\N$ lukumäärä, joille pätee $k \le n$ ja $x \le a_k \le y$ ($n\in\N$), niin
$\lim_n N_n/n = y-x$. Bolzanon-Weierstrassin lauseen mukaan jonolla $\seq{a_n}$ on suppeneva
osajono. Näytä, että pätee vahvempi tulos: Jos $0 \le x \le 1$, niin löytyy osajono, jonka
raja-arvo $=x$. 

\item (*)
Näytä, että jokaisella reaalilukujonolla on monotoninen osajono.

\end{enumerate} % Cauchyn jonot
\section{Reaalilukujen ominaisuuksia}  \label{reaalilukujen ominaisuuksia}
\alku

Tässä luvussa esitellään eräitä reaalilukujoukkoihin liittyviä matemaattisen analyysin 
peruskäsitteitä. Samalla tuodaan esiin myös reaalilukujen yleisiä ominaisuuksia,
vaihtoehtoisia määritelmiä ja luokittelutapoja.

\subsection{Supremum ja infimum}

\begin{Def} \label{supremum} \index{supremum|emph} \index{pienin yläraja|emph} 
Luku $a \in \R$ on joukon $A \subset \R$ \kor{pienin yläraja} eli
\kor{supremum}, jos pätee
\begin{itemize}
\item[(1)] $\quad x \le a\ \ \forall x \in A$.
\item[(2)] $\quad b<a\,\ \impl\,\ \exists x \in A\ (x>b)$.
\end{itemize}
Merkitään $a = \sup A$. 
\end{Def}
\index{ylzy@yläraja (joukon)}%
Määritelmässä ehto (1) merkitsee, että $a$ on joukon $A$ \kor{yläraja}, ja ehto (2) kertoo, että
jos $b<a$, niin $b$ ei ole yläraja. Ehdot merkitsevät siis yhdessä, että $a$ on ylärajoista 
pienin mahdollinen. Jotta tällainen pienin yläraja voisi olla olemassa, on joukon $A$ oltava
\index{ylhäältä rajoitettu}%
ainakin \kor{ylhäältä rajoitettu} (ehto (1) voimassa jollakin $a \in \R$) ja ei-tyhjä 
(ehto (2) voimassa jollakin $a \in \R$). Sikäli kuin $\sup A$ on olemassa, on määritelmästä 
varsin ilmeistä, että se on yksikäsitteinen. Jos $a=\sup A$ ja $a \in A$, niin sanotaan, että 
\index{maksimi (joukon)}%
$a$ on $A$:n \kor{maksimi} (= suurin luku).

Jos Määritelmässä \ref{supremum} vaihdetaan epäyhtälöiden suunnat, tulee määritellyksi joukon 
\index{infimum} \index{suurin alaraja}%
$A$ \kor{suurin alaraja} eli \kor{infimum}, jota merkitään $\inf A$. Infimum voi olla olemassa
\index{alhaalta rajoitettu}%
vain, jos $A$ on ei-tyhjä ja \kor{alhaalta rajoitettu}, eli jos jollakin $a \in \R$ pätee 
\index{minimi (joukon)}%
$\,x \ge a\ \ \forall x \in A$. Jos $a=\inf A$ ja $a \in A$, niin $a$ on $A$:n \kor{minimi} 
(= pienin luku).
\begin{Exa} Jos $A = \{x \in \R \mid x \le 1\}$, niin Määritelmän \ref{supremum} mukaan 
$\ \sup A = 1$. Koska $1 \in A$, niin kyseessä on myös $A$:n maksimi. Koska $A$ ei ole alhaalta
rajoitettu, ei lukua $\inf A$ ole olemassa. 
\end{Exa}
\begin{Exa} \label{sup-esimerkki} Reaalilukujoukolla $A = \{\,x \in \R \mid x^2 < 2\,\}$ ja 
rationaalilukujoukolla $B = \{\,x \in \Q \mid x^2 \le 2\,\}$ on kummallakin sama pienin yläraja
ja suurin alaraja: $\ \sup A = \sup B = \sqrt{2}$ ja $\ \inf A = \inf B = -\sqrt{2}$. Maksimia
tai minimiä ei kummallakaan joukolla ole. \loppu 
\end{Exa}
Esimerkin \ref{sup-esimerkki} mukaan rajoitetun rationaalilukujoukon pienin yläraja tai suurin
alaraja ei välttämättä ole rationaalinen. Vastaava ilmiö on entuudestaan tuttu lukujonoista:
Rationaalilukujonon raja-arvo ei välttämättä ole rationaalinen. Lukujonoista tiedetään myös,
että kasvavalle ja rajoitetulle rationaalilukujonolle on aina konstruoitavissa raja-arvo 
reaalilukuna (eli äärettömänä desimaalilukuna, ks.\ Lauseen \ref{monotoninen ja rajoitettu jono}
todistuskonstruktio). Vastaavalla tavalla on jokaiselle ei-tyhjälle ja ylhäältä rajoitetulle
reaalilukujoukolle konstruoitavissa pienin yläraja reaalilukuna.
\begin{*Lause} (\vahv{Supremum--lause}) \label{supremum-lause} Jokaisella ei-tyhjällä, ylhäältä
rajoitetulla reaalilukujoukolla on supremum. 
\end{*Lause}
\tod Tarkastellaan oletukset täyttävää joukkoa $A \subset \R$. Koska $A$ on ei-tyhjä, on
olemasssa $c_1 \in \R$ siten, että $x>c_1$ jollakin $x \in A$, ts.\ $c_1$ ei ole $A$:n yläraja.
Koska $A$ on ylhäältä rajoitettu, niin $A$:lla on yläraja $c_2 = c_1 + L \in \R$. Yleisyyttä 
rajoittamatta voidaan edelleen olettaa, että $c_1$ ja $c_2$ ovat rationaalilukuja, tai jopa 
kokonaislukuja. Konstruoidaan nyt kasvava rationaalilukujono $\seq{a_n}$ ja vähenevä 
rationaalilukujono $\seq{b_n}$ puolituskonstruktiolla seuraavasti:
\begin{itemize}
\item[1.] Asetetaan $a_0=c_1$, $b_0=c_2$ ja $n=1$.
\item[2.] Asetetaan $c = (c_1 + c_2)/2$. Jos $c$ on $A$:n yläraja, niin asetetaan $c_2=c$,
          muussa tapauksessa asetetaan $c_1=c$. 
\item[3.] Asetetaan $a_n=c_1,\ b_n=c_2$.
\item[4.] Lisätään indeksin $n$ arvoa yhdellä\ ($n \leftarrow n+1$) ja siirrytään kohtaan 2.
\end{itemize}
Konstruktion mukaisesti jono $\seq{a_n}$ on kasvava ja rajoitettu ja jono $\seq{b_n}$ on
vähenevä ja rajoitettu, ja lisäksi $b_n-a_n=2^{-n}L$. Siis $\seq{a_n}$ ja $\seq{b_n}$
suppenevat kohti yhteistä raja arvoa: $a_n \kohti a$ ja $b_n \kohti a$ jollakin $a\in\R$.
Konstruktion perusteella jokainen $b_n$ on $A$:n yläraja ja mikään $a_n$ ei ole $A$:n yläraja,
joten ilmeinen kandidaatti $A$:n pienimmäksi ylärajaksi on yhteinen raja-arvo $a$. Ensinnäkin
tämä on $A$:n yläraja, sillä jos $x \in A$, niin $x \le b_n\ \forall n$ (koska jokainen $b_n$ 
oli yläraja), jolloin seuraa $\,x \le \lim_n b_n=a$ (Propositio \ref{jonotuloksia} (V1)).
Siis pätee $x \in A\ \impl\ x \le a$, eli $a$ on $A$:n yläraja.

Vielä on näytettävä, että $a$ on ylärajoista pienin, joten olkoon $b<a$. Tällöin koska 
$a_n \kohti a$ ja $a>b$, niin jollakin $n$ on $a_n>b$. Silloin $b$ ei voi olla $A$:n yläraja, 
koska edes $a_n$ ei ole. Siis mikään lukua $a$ pienempi luku ei ole $A$:n yläraja, eli 
$a=\sup A$. \loppu

Jos $A$ on ei-tyhjä, alhaalta rajoitettu reaalilukujoukko, niin on ilmeistä, että 
$B = \{-x \mid x \in A\}$ on ylhäältä rajoitettu ja että $\inf A = - \sup B$. Tästä ja 
Lauseesta \ref{supremum-lause} seuraa välittömästi, että jokaisella ei-tyhjällä, alhaalta 
rajoitetulla reaalilukujoukolla on infimum.

\subsection{Aksiomaattiset reaaliluvut}
\index{reaaliluvut!d@aksiomaattisina lukuina|vahv}

Edellisessä luvussa esitettyjä, desimaalilukuihin tai Cauchyn jonoihin perustuvia reaaliluvun 
\index{konstruktiivinen (määrittely)}%
määritelmiä sanotaan \kor{konstruktiivisiksi}. Konstruktiivisille määritelmille on 
tunnusomaista, että mitään 'ulkopuolista maailmaa' suhteessa rationaalilukuihin ei oleteta. 
Päätetään vain kutsua tietyn tyyppisiä rationaalilukujen jonoja reaaliluvuiksi, ja määritellään
näiden lukujen väliset laskuoperaatiot sekä samastus- ja järjestysrelaatiot palauttamalla kaikki
operaatiot viime kädessä rationaalisiksi. Konstruktiivisen määrittelyn hyvä puoli on, että se 
muistuttaa reaalilukujen todellista, laskennallista 'määrittelyä' laskinten ja tietokoneiden 
avulla. Toisaalta kun reaaliluvuilla halutaan laskea symbolisesti, on reaaliluvut usein 
helpompaa mieltää suoraan abstrakteina lukuina kuin konkreettisempien lukujen jonoina 
(vrt.\ Esimerkki \ref{reaalinen geometrinen sarja} Luvussa \ref{reaaliluvut}). Abstrakti
\index{aksiomaattinen (määrittely)}%
ajattelu viedään pisimmälle reaalilukujen \kor{aksiomaattisessa} määrittelyssä, jossa
yksinkertaisesti \pain{sovitaan}, että muitakin kuin rationaalilukuja on olemassa. Menettely on
tällöin samankaltainen kuin Luvun \ref{kunta} Esimerkissä \ref{muuan kunta}, jossa luvun
$\sqrt{2}\ $  'olemassaoloon' suhtauduttiin sopimuskysymyksenä. Kuten konstruktiivisia
määritelmiä, myös aksiomaattisia lähestymistapoja reaalilukuihin on monia samanarvoisia.
Määritelmien yhteisenä lähtökohtana on oletus, että reaaliluvut muodostavat järjestetyn kunnan,
joka on rationaalilukujen kunnan laajennus. Nämä oletukset jättävät mahdollisuuden, että
$\R = \Q$, joten tarvitaan vielä aksiooma, joka erottaa joukot toisistaan. Varsin usein
käytetty menettely on tehdä joko Lauseesta \ref{supremum-lause} tai Lauseesta 
\ref{Cauchyn kriteeri} aksiooma, eli asettaa jompi kumpi seuraavista.
\begin{itemize}
\item[(A)] (\vahv{Supremum-aksiooma}) Jokaisella ylhäältä rajoitetulla reaalilukujoukolla on 
           pienin yläraja.\footnote[2]{Supremum-aksioomaan tai supremum-lauseeseen liittyen
           reaaliluvut voidaan määritellä myös ylhäältä rajoitettuina rationaalilukujen
           j\pain{oukkoina}. Tämän nk.\ \kor{Dedekindin leikkauksiin} nojaavan konstruktiivisen
           määritelmän esitti saksalainen matemaatikko \hist{Richard Dedekind} (1831-1916).
           Dedekindin määritelmä on ollut aikanaan suosittu opetuksessakin.
           \index{Dedekind, R.|av} \index{Dedekindin leikkaus|av}
           \index{reaaliluvut!da@Dedekindin leikkauksina|av}} 
\item[(B)] (\vahv{Täydellisyysaksiooma}) Jokainen reaalilukujen Cauchyn jono suppenee kohti 
           reaalilukua.
\end{itemize}
Reaalilukujen em.\ konstruktiivisessa määrittelyssä nämä ovat siis tosia väittämiä. Aksioomina
ne ovat myös keskenään vaihdannaiset: Jos oletetaan (A), niin (B) on tosi väittämä, ja
päinvastoin.

Kuten konstruktiivisessa, myös reaalilukujen aksiomaattisessa määrittelyssä lähtökohtana ovat 
siis aluksi rationaaliluvut. Perustuen aksioomaan (A) tai (B), rationaalilukujen joukko 
\kor{täydennetään} liittämällä joukkoon joko ylhäältä rajoitettujen rationaalilukujoukkojen 
pienimmät ylärajat tai rationaalisten Cauchyn jonojen raja-arvot, jotka siis oletetun aksiooman
mukaan ovat olemassa reaalilukuina. Koska täydentäminen on taka-ajatuksena myös
\index{tzy@täydennyskonstruktio}%
konstruktiivisessa määrittelyssä, niin tätä on tapana kutsua \kor{täydennyskonstruktioksi}.
Rationaaliluvut voi siis täydentää reaaliluvuiksi joko konstruktiivisesti 'rakentamalla' tai
aksiomaattisesti 'olettamalla'.

\subsection{$\R$ on ylinumeroituva}
\index{ylinumeroituva joukko|vahv}%

Jos $a,b \in \R$ ja $a<b$, niin on helposti nähtävissä, että lukujen $a,b$ väliin voidaan aina
sijoittaa rationaaliluku, ts.\ löytyy $x \in \Q$ jolle pätee $a<x<b$. Samantien lukuja löytyy
ääretön (numeroituva) määrä, eikä konstruktiossa itse asiassa edes tarvita koko joukkoa $\Q$,
vaan esim.\ äärelliset desimaaliluvut (tai äärelliset binaariluvut) riittävät
(Harj.teht.\,\ref{H-I-11: rationaaliluvut välillä}). Tuloksen perusteella sanotaan, että $\Q$
\index{tihzy@tiheä (osajoukko)}%
(tai mainitun tyyppinen $\Q$:n osajoukko) on $\R$:ssä \kor{tiheä} (engl.\ dense). Tästä 
tuloksesta huolimatta on $\R$ joukkoa $\Q$ mahtavampi, sillä pätee (vrt. Luku \ref{jono})
\begin{Lause} \label{R on ylinumeroituva} $\R$ on ylinumeroituva joukko. 
\end{Lause}
\tod Olkoon \seq{x^{(n)}} jono reaalilukuja, joille pätee $0 < x^{(n)} < 1\ \ \forall n$. 
Kirjoitetaan jonon termit äärettöminä desimaalilukuina\,:
\begin{align*}
x^{(1)}\ &= 0.d_1^{(1)}d_2^{(1)}d_3^{(1)} \ldots \\
x^{(2)}\ &= 0.d_1^{(2)}d_2^{(2)}d_3^{(2)} \ldots \\
x^{(3)}\ &= 0.d_1^{(3)}d_2^{(3)}d_3^{(3)} \ldots \\
          &\ \vdots
\end{align*}
Määritellään sitten ääretön desimaaliluku $y = 0.d_1d_2 \ldots\ $ valitsemalla 
\[
d_n \in \{1,\ldots 8\},\ \ d_n \neq d_n^{(n)}, \quad n = 1,2, \ldots
\] 
Tällöin $0<y<1$ ja $y$ ei ole äärellinen desimaaliluku, joten $y$ ei samastu mihinkään muuhun,
merkkijonona vähänkään erilaiseen desimaalilukuun. Konstruktiosta seuraa tällöin, että 
$y \neq x^{(n)}\ \forall n$. Päätellään, että mikään lukujoukon 
$A = \{\,x \in \R \mid 0<x<1\,\}$ numerointiyritys ei kata kaikkia joukon $A$ lukuja, joten 
edes $A$ ei ole numeroituva. \loppu

Olkoon $a,b \in \R$, $a<b$ ja $B = \{\,x \in \R \mid a<x<b\,\}$. Koska
\[
B = \{\,x = a + (b-a)t \mid t \in \R,\ 0<t<1\,\},
\]
niin Lauseen \ref{R on ylinumeroituva} todistuksesta voidaan päätellä, että $B$ on 
ylinumeroituva. Toisin sanoen: kahden eri suuren reaaliluvun väliin mahtuu aina ylinumeroituva
määrä reaalilukuja.

\subsection{Avoin ja suljettu väli}

\index{vzy@väli}%
Joukko $A\subset\R$ on \kor{väli} (engl.\ interval), jos $A$:ssa on enemmän kuin yksi alkio ja
pätee $\,x,y \in A\ \ja\ x<z<y\ \impl\ z \in A$. Väli voi olla joko
\index{zyzy@äärellinen, ääretön väli}%
\kor{äärellinen}, eli joukkona ylhäältä ja alhaalta rajoitettu, tai \kor{ääretön}
(ylhäältä ja/tai alhaalta rajoittamaton). Äärellisen välin päätyypit ovat
\index{avoin väli} \index{suljettu väli}%
\kor{avoin} ja \kor{suljettu} väli, joiden merkinnät ja määritelmät ovat:
\begin{align*}
\text{Avoin väli:}    \quad &(a,b)\     =\ \{\,x \in \R \mid a<x<b\,\}. \\
\text{Suljettu väli:} \quad &[\,a,b\,]\ =\ \{\,x \in \R \mid a \le x \le b\,\}.
\end{align*}
\index{pzyzy@päätepiste (välin)}%
Tässä $a=\inf A$ ja ja $b=\sup A$ ovat välin \kor{päätepisteet} ($a,b\in\R,\ a<b$). Avoimen
välin vaihtoehtoinen merkintätapa on $]a,b[$. Väli voi myös olla muotoa $(a,b]$ tai $[a,b)$ 
(vaihtoehtoiset merkinnät $]a,b]$ ja $[a,b[\,$), jolloin sanotaan, että väli on
\index{puoliavoin väli}%
\kor{puoliavoin} --- määritelmät ovat ilmeiset. Muita kuin välin päätepisteitä sanotaan välin
\index{siszy@sisäpiste}%
\kor{sisäpisteiksi}. Avoimen välin kaikki pisteet ovat siis sisäpisteitä. Suljettu väli on
aina äärellinen. Muun tyyppiset välit voivat olla myös äärettömiä, jolloin päätepisteen
puuttuminen merkitään symbolilla $\pm\infty$\,:
\begin{align*}
(a,\infty)\  &=\ \{\,x \in \R \mid x>a\,\}, \\ 
(-\infty,b]\ &=\ \{\,x \in \R \mid x \le b\,\}, \\
(-\infty,\infty)\ &=\ \R.
\intertext{Mainittakoon tässä yhteydessä myös yleisesti käytetyt merkinnät}
(0,\infty)\       &=\ \R_+ \quad \text{(positiiviset reaaliluvut, 'R plus')}\,, \\
(-\infty,0)\      &=\ \R_- \quad \text{(negatiiviset reaaliluvut, 'R miinus')}\,.
\end{align*}

\pain{Sul}j\pain{ettu}j\pain{en} välien $[a_n, b_n],\ n =1,2, \ldots$ muodostamaa j\pain{onoa}
sanotaan \kor{sisäkkäiseksi} (engl.\ nested = 'pesitetty'), jos 
$[a_n, b_n] \supset [a_{n+1}, b_{n+1}]\ \ \forall n$. Tällaiseen jonoon liittyen reaaliluvuilla 
on seuraava hauska ominaisuus:
\begin{itemize}
\item[(C)] Jos $\{\,[a_n, b_n],\ n = 1,2, \ldots\,\}$ on jono sisäkkäisiä suljettuja välejä ja
           pätee $b_n - a_n \kohti 0$, niin on olemassa yksikäsitteinen reaaliluku $x$ siten, 
           että $x \in [a_n, b_n]\ \ \forall n \in \N$.
\end{itemize}
Reaalilukujen konstruktiivisen määrittelyn perusteella (C) on tosi väittämä, sillä $x$ määräytyy
jonojen $\seq{a_n}$ (kasvava ja rajoitettu) ja $\seq{b_n}$ (vähenevä ja rajoitettu) yhteisenä 
raja-arvona. Reaalilukujen aksiomaattisessa määrittelyssä tämä tulos voidaan myös ottaa 
perusaksioomaksi (olettaen $a_n,b_n\in\Q$), jolloin edellä mainituista aksioomista (A),\,(B) 
tulee tosia väittämiä.

Jos $x \in \R$ ja $\delta\in\R,\ \delta > 0$, niin $x$:n 
\index{ympzy@($\delta$-)ympäristö}%
$\delta$-\kor{ympäristö} määritellään \pain{avoimena} välinä
\[
U_{\delta}(x)\ =\ (x-\delta,x+\delta).
\]
Termit 'päätepiste' ja 'ympäristö' viittaavat ajatukseen, että reaaliluku on myös miellettävissä
'pisteeksi' jossakin 'paikassa'. Tämä ajatus ei ole peräisin algebrasta vaan \kor{geometriasta},
toisesta matemaattisen ajattelun suuresta päähaarasta. Geometrian perusteita käsitellään 
lähemmin Luvussa II.

\subsection{Algebralliset ja transkendenttiset luvut}
\index{algebrallinen luku|vahv} \index{transkendenttinen luku|vahv}%

Määritelmänsä perusteella reaaliluvut jakautuvat luonnostaan rationaalisiin ja ei-rationaalisiin
\index{irrationaalinen luku}%
eli \kor{irrationaalisiin} lukuihin. Toinen, myös yleisesti käytetty luokittelutapa on jakaa 
reaaliluvut \kor{algebrallisiin} ja ei-algebrallisiin eli \kor{transkendenttisiin} lukuihin. 
Lukua $x \in \R$ sanotaan algebralliseksi, jos se toteuttaa yhtälön muotoa
\[
\sum_{k=0}^n a_k x^k\ =\ 0,
\]
missä $n \in \N$ ja $a_k \in \Z,\ k = 0,1,\ldots,n$. Luku on siis algebrallinen, jos se on 
jonkin \pain{kokonaislukukertoimisen} p\pain{ol}y\pain{nomin} \pain{nollakohta}.
\begin{Exa} Juuriluvut ja niiden erikoistapauksena rationaaliluvut toteuttavat yhtälön muotoa 
$p x^m - q = 0$, missä $m \in \N$ ja $p,q \in \Z$, joten tällaiset luvut ovat algebrallisia. 
\loppu \end{Exa}
Jos algebrallisten lukujen joukkoa merkitään symbolilla $\A$, niin esimerkin perusteella on siis
\[ 
\Q\ \subset\ \A\ \subset\ \R. 
\]
Kuten $\Q$, myös $\A$ on osoitettavissa numeroituvaksi joukoksi, joten reaalilukujen 'enemmistö'
on transkendenttisia.\footnote[2]{Matematiikan lajia, joka tutkii lukujen, kuten reaalilukujen
tai luonnollisten lukujen, ominaisuuksia mm.\ erilaisten luokittelujen näkökulmasta, sanotaan 
\kor{lukuteoriaksi}. Lukuteorian tunnettuja tuloksia on esimerkiksi, että Neperin luku $e$ on 
transkendenttinen (\hist{Charles Hermite}, 1873), samoin luku $\pi$ 
(\hist{Ferdinand von Lindemann}, 1882). Molempien lukujen irrationaalisuus osoitettiin jo 
1700-luvulla. Sen sijaan niinkään yksinkertaisia väittämiä kuin
\begin{itemize}
\item[P1:] Luku $\,e + \pi\,$ ei ole rationaaliluku.
\item[P2:] Luku $\,e \pi\,$ ei ole rationaaliluku.
\end{itemize}
ei ole vielä tätä kirjoitettaessa (2015) pystytty osoittamaan tosiksi (saati epätosiksi). 
--- Lukuteoria on  tunnettu monista helposti muotoiltavissa olevista, mutta usein vaikeasti 
ratkaistavista matemaattisista pähkinöistään.}

\pagebreak

\Harj
\begin{enumerate}

\item
Määritä seuraavien joukkojen supremum, infimum, maksimi ja minimi, sikäli kuin olemassa\,: 
\newline
a) \ $\{x\in\R \mid (x+1)(x-2)(x+4)>0\}\quad\ $ 
b) \ $\{x\in\R \mid \abs{x}+\abs{x+2}<5\}$ \newline
c) \ $\{x\in\R \mid \abs{x}\abs{x+2}<5\}\qquad\qquad\qquad\ \ $ 
d) \ $\{n/(n+1) \mid n\in\N\}$ \newline 
e) \ $\{(2-3n)/(5+n) \mid n\in\N\}\qquad\qquad\quad\ $
f) \ $\{n^3 2^{-n} \mid n\in\N\}$

\item \label{H-I-11: sup ja inf}
a) Millaisille ei-tyhjille joukoille $A\subset\R$ pätee $\,\inf A = \sup A$\,? \newline
c) Olkoon $A\subset\R$ ylhäältä rajoitettu joukko ja $a=\sup A$. Näytä, että jokaisella
$\eps>0$ on olemassa $x \in A$ siten, että $x > a-\eps$. \newline
c) Näytä, että jos $A,B\subset\R$ ovat molemmat ylhäältä (alhaalta) rajoitettuja ja
$A \subset B$, niin $\sup A \le \sup B$ ($\inf A \ge \inf B$).

\item
Olkoon \ a) $A=\{x\in\R \mid x^3<200\}$, \ b) $A=\{x\in\R \mid x^3<-400\}$. Seuraa Lauseen
\ref{supremum-lause} todistuskonstruktiota algoritmina indeksiin $n=3$ asti valitsemalla 
$c_0$:n ja $c_1$:n alkuarvot siten, että algoritmin tuottamille lukujonoille pätee: \ 
a) $\seq{a_n}=a$, \ b) $\seq{b_n}=a$, missä $\,a=\sup A\,$ äärettömänä binaarilukuna.

\item \label{H-I-11: rationaaliluvut välillä}
Olkoon $a,b\in\R$ ja $a<b$. Näytä, että on olemassa äärettömän monta eri suurta äärellistä 
desimaalilukua $x$, joille pätee $a<x<b$.

\item
Määritä kaikki välit, joilla on ominaisuus: Välin päätepisteet --- sikäli kuin niitä on --- 
sisältyvät joukkoon $\{-1,2\}$.

\item
Anna seuraaville joukoille yksinkertaisempi määritelmä väleinä, eli muodossa $(a,b)$, $[a,b]$,
$(a,b]$ tai $[a,b)$.
\begin{align*}
&\text{a)}\ \ \bigcup_{n=1}^\infty \left(\frac{1}{3n}\,,\ \frac{n}{n+1}\right) \qquad
 \text{b)}\ \ \bigcap_{n=1}^\infty \left(\frac{1}{3n}\,,\ \frac{n+1}{n}\right) \\
&\text{c)}\ \ \bigcap_{n=1}^\infty \left[\,\frac{1}{3n}\,,\ \frac{n}{n+1}\,\right] \qquad
 \text{d)}\ \ \bigcup_{n=1}^\infty \left[\,\frac{1}{3n}\,,\ \frac{n+1}{n}\,\right]
\end{align*}

\item
Näytä: \ a) Jos $x$ on irrationaalinen, niin samoin on $y=(3+x)/(x-2)$. \
b) Jos $x \neq 0$ on algebrallinen luku, niin samoin on $x^{-1}$. \ c) Jos $x>0$ on
algebrallinen luku, niin samoin on $\sqrt[m]{x}\,\ \forall m\in\N,\ m \ge 2$. \
d) Jokainen rationaalikertoimisen polynomin nollakohta on algebrallinen luku.

\item (*)
Olkoon $\{\,[a_n,b_n],\ n=1,2,\ldots\,\}$ jono suljettuja välejä. Näytä, että 
$A=\bigcap_{n=1}^\infty [a_n,b_n]$ on joko tyhjä joukko, sisältää täsmälleen yhden alkion, tai
on suljettu väli.

\end{enumerate} % Reaalilukujen ominaisuuksia
\section{Klassinen sarjaoppi. Potenssisarja} \label{potenssisarja}
\sectionmark{Klassinen sarjaoppi}
\alku
\index{sarjaoppi (klassinen)|vahv}

Kerrattakoon aiemmista luvuista, että merkinnällä
\[
\sum_{k=1}^\infty a_k\ =\ a_1 + a_2 + a_3 + \ldots
\]
tarkoitetaan joko sarjaa, eli lukujonoa 
$\,\{a_1,\,a_1 + a_2,\,\ldots\} = \{s_1,s_2,\,\ldots\}$, tai sikäli kuin $\seq{s_n}$ suppenee
(kohti reaalilukua), myös \kor{sarjan summaa} eli raja-arvoa $\,s = \lim_n s_n$. Kerrattakoon
myös, että lukuja $a_k$ sanotaan sarjan \kor{termeiksi} ja lukujonon $\seq{s_n}$ termejä sarjan
\kor{osasummiksi}. Sarjoilla on paljon käyttöä sovelluksissa ja myös matemaattisessa
ajattelussa. Tässä luvussa tarkastellaan sarjojen suppenevuusteoriaa aiempaa yleisemmältä
kannalta sekä tutkitaan eräitä sarjojen tavallisia erikoistyyppejä. Sarjan termit $a_k$
oletetaan jatkossa reaaliluvuiksi.

\subsection{Positiiviterminen sarja}
\index{sarja!b@positiiviterminen|vahv}%

Suppenemisteorian kannalta 'ystävällismielisin', sarja on \kor{positiiviterminen}
sarja $\sum_ka_k$, jonka termit ovat positiiviset, tai yleisemmin ei-negatiiviset:
$a_k \ge 0\ \forall k$. Esimerkkejä on tarkasteltu jo Luvussa \ref{monotoniset jonot}.
Positiivitermisen sarjan osasummien jono on kasvava, joten sarja suppenee täsmälleen kun
ko.\ lukujono on rajoitettu (Lause \ref{monotoninen ja rajoitettu jono}).
Suppenemistarkastelua voidaan usein helpottaa edelleen vertaamalla sarjan termejä
yksinkertaisempaan sarjaan, joka jo tiedetään suppenevaksi tai hajaantuvaksi. Seuraavat
vertailuperiaatteet ovat hyödyllisiä.
\begin{Lause} \label{sarjojen vertailu} 
\index{majoranttiperiaate|emph} \index{minoranttiperiaate|emph}
Jos $0 \le a_k \le b_k\ \forall k$, niin pätee
\begin{align*}
\text{\vahv{Majoranttiperiaate:}}\quad &\sum_k b_k\ \text{suppenee} 
                                     \qimpl \sum_k a_k\ \text{suppenee} \\
\text{\vahv{Minoranttiperiaate:}}\quad &\sum_k a_k\ \text{hajaantuu} 
                                     \qimpl \sum_k b_k\ \text{hajaantuu}
\end{align*}
\end{Lause}
\tod Jos sarjojen $\sum_ka_k$ ja $\sum_kb_k$ osasummien jonot ovat vastaavasti $\seq{s_n}$ ja
$\seq{t_n}$, niin oletuksen perusteella on $s_n \le t_n\ \forall n$. Majoranttiperiaate seuraa
tällöin päättelyllä
\[
\seq{t_n}\ \text{suppenee}\ \impl\ \seq{t_n}\ \text{rajoitettu}\ 
                            \impl\ \seq{s_n}\ \text{rajoitettu}\
                            \impl\ \seq{s_n}\ \text{suppenee}.
\]
Minoranttiperiaate seuraa vastaavalla päättelyllä -- tai vain toteamalla vertailuperiaatteet
loogisesti yhtäpitäviksi (vrt.\ Luku \ref{logiikka}). \loppu
\begin{Exa} Suppeneeko vai hajaantuuko sarja
\[
\text{a)}\,\ \sum_{k=1}^\infty \frac{(k+3)^2}{k^4}\,, \quad 
\text{b)}\,\ \sum_{k=0}^\infty\,\frac{2k+50}{k^2 + 1}\ ?
\] 
\end{Exa}
\ratk a) Koska
\[
0 < a_k = \frac{(k+3)^2}{k^4} 
        = \frac{1}{k^2}+\frac{6}{k^3}+\frac{9}{k^4} \le \frac{16}{k^2}, \quad k \ge 1,
\]
ja koska vertailusarja $\sum_{k=1}^\infty 16k^{-2}=16\sum_{k=1}^\infty k^{-2}$ tiedetään 
suppenevaksi (Luku \ref{monotoniset jonot}, Esimerkki \ref{kaksi sarjaa}), niin
majoranttiperiaatteen nojalla sarja suppenee.

b) Kun kirjoitetaan
\[
a_k\ =\ \dfrac{2k+50}{k^2 + 1}\ =\ \dfrac{1}{k} \cdot \dfrac{2 + 50\,k^{-1}}{1 + k^{-2}}\ 
                                =\ \dfrac{1}{k} \cdot c_k\,,
\]
niin nähdään, että $c_k \kohti 2$, joten jostakin indeksistä $k=m$ alkaen (itse asiassa kun 
$k \ge 50$) on $1 \le c_k \le 3$. Tällöin on
\[
\dfrac{1}{k}\ \le\ a_k\ \le\ \dfrac{3}{k}, \quad k=m,\,m+1,\,\ldots\,,
\]
joten päätellään majorantti- ja minoranttiperiaatteiden avulla, että tarkasteltava sarja 
suppenee/hajaantuu täsmälleen kun sarja $\sum_{k=m}^\infty\, 1/k$ suppenee/hajaantuu. Tämä 
vertailusarja osoittautuu hajaantuvaksi (Lause \ref{harmoninen sarja} jäljempänä), joten 
tarkasteltava sarja siis hajaantuu myös. \loppu

\subsection{Harmoninen, aliharmoninen ja yliharmoninen sarja}

Positiivitermistä sarjaa
\[
\sum_{k=1}^\infty \dfrac{1}{k^\alpha}\ 
                      =\ 1 + \dfrac{1}{2^\alpha} + \dfrac{1}{3^\alpha} + \ldots\,,
\]
\index{sarja!c@harmoninen} \index{sarja!d@ali-, yliharmoninen}
\index{harmoninen sarja} \index{aliharmoninen sarja} \index{yliharmoninen sarja}%
missä $\alpha$ on rationaaliluku\footnote[2]{Toistaiseksi $x^\alpha\ (x \in \R,\ x>0)$ on 
määritelty vain, kun $\alpha \in \Q$.}, sanotaan \kor{harmoniseksi}, jos $\alpha = 1$, 
\kor{aliharmoniseksi}, jos $\alpha < 1$, ja \kor{yliharmoniseksi}, jos $\alpha > 1$. Majorantti-
ja minoranttiperiaatteita käytettäessä tämä sarjatyyppi on usein hyvä vertailukohta. 
\begin{Lause} \label{harmoninen sarja} Harmoninen ja aliharmoninen sarja hajaantuvat. 
Yliharmoninen sarja suppenee. 
\end{Lause}
\tod Tarkastellaan ensin harmonista sarjaa
\[
\sum_{k=1}^\infty \dfrac{1}{k}\ =\ 1 + \dfrac{1}{2} + \dfrac{1}{3} + \ldots\,.
\]
Arvioidaan osasummia $s_4$, $s_8$, $s_{16}$ jne.\ seuraavasti:
\begin{align*}
s_4     &=\ 1 + \dfrac{1}{2} + \left( \dfrac{1}{3} + \dfrac{1}{4} \right)\ 
         >\ \dfrac{3}{2} + 2 \cdot \dfrac{1}{4}\ =\ 2, \\
s_8     &=\ s_4 + \left( \dfrac{1}{5} + \ldots + \dfrac{1}{8} \right)\ 
         >\ 2 + 4 \cdot \dfrac{1}{8}\ =\ \dfrac{5}{2}, \\
        &\ \vdots
\end{align*}
Päätellään, että jos $n = 2^m,\ m \in \N,\ m \ge 2$, niin $s_n > 1 + m/2$, joten harmonisen
sarjan osasummien osajono $\{s_4, s_8, s_{16}, \ldots\}$ on hajaantuva. Siis sarja hajaantuu:
$s_n \kohti \infty$. Tällöin myös aliharmoninen sarja hajaantuu minoranttiperiaatteen nojalla,
sillä lukujen $k^\alpha,\ \alpha \in \Q$ määritelmästä (ks.\ Luku \ref{kunta}) seuraa helposti, 
että $k > k^\alpha\ \ekv\ k^{-\alpha} > k^{-1}$, kun $\alpha < 1$ ja $k \ge 2$.

Yliharmonisen sarjan tapauksessa merkitään $\alpha = 1 + \delta,\ \delta > 0$ ja arvioidaan 
osasummia $s_3, s_7$, jne.\ seuraavasti:
\begin{align*}
s_3\ &=\ 1 + \left( \dfrac{1}{2^\alpha} + \dfrac{1}{3^\alpha} \right)\ 
                            <\ 1 + 2 \cdot \dfrac{1}{2^\alpha}\ =\ 1 + \dfrac{1}{2^\delta}, \\
s_7\ &=\ s_3 + \left( \dfrac{1}{4^\alpha} + \ldots \dfrac{1}{7^\alpha} \right)\ 
                            <\ 1 + \dfrac{1}{2^\delta} + 4 \cdot \dfrac{1}{4^\alpha}\
                            =\ 1 + \dfrac{1}{2^\delta} + \left(\dfrac{1}{2^\delta}\right)^2, \\
     &\ \vdots
\end{align*}
(Tässä on käytetty arviota $(k/j)^\alpha > 1\ \ekv\ k^{-\alpha} < j^{-\alpha}$, kun $k/j>1$ ja
$\alpha > 0$.) Päätellään, että jos $n = 2^m - 1,\ m \in \N,\ m \ge 2$, niin
\[
s_n\ <\ 1 + \dfrac{1}{2^\delta} + \ldots + \left(\dfrac{1}{2^\delta}\right)^{m-1} 
     <\ \ \sum_{k=0}^\infty \left(\dfrac{1}{2^\delta}\right)^k\ 
     =\ \dfrac{2^\delta}{2^\delta - 1}\,.
\]                                                               
Siis sarjan osasummien jono on rajoitettu, joten sarja suppenee. \loppu

Huomautettakoon, että tapauksessa $\,\alpha = 2\,$ Lauseen \ref{harmoninen sarja} väittämä on
jo aiemmin todistettu muilla keinoin (ks.\ Luku \ref{monotoniset jonot}).
\begin{Exa} Yliharmoninen sarja $\sum_{k=1}^\infty k^{-5/4}$ suppenee, mutta suppeneminen on niin
hidasta, että pelkästään osasummia laskemalla ei sarjan summaa saada selville tarkasti.
Paremmilla algoritmeilla
(ks.\ Harj.teht.\,\ref{numeerinen integrointi}:\ref{H-int-9: hidas sarja}) tulos on
\[
s\ =\ \sum_{k=1}^\infty \dfrac{1}{k^{5/4}}\ 
   =\ 1 + \dfrac{1}{2^{5/4}} + \dfrac{1}{3^{5/4}} + \ldots\ =\ 4.5951118258\,.. \loppu
\]
\end{Exa}
\begin{Exa} Hajaantuvan aliharmonisen sarjan $\sum_{k=1}^\infty 1/\sqrt{k}$ osasummille
pätee suurilla $n$:n arvoilla arvio (ks.\ Harj.teht.\,\ref{H-I-12: teleskooppiarvio})
\[
s_n = \sum_{k=1}^n \frac{1}{\sqrt{k}} \,\approx\, 2\sqrt{n}. \loppu
\]
\end{Exa}

\subsection{Vuorotteleva sarja}
\index{sarja!e@vuorotteleva (alternoiva)|vahv}
\index{vuorotteleva sarja|vahv}
\index{alternoiva sarja|vahv}

Sarjaa sanotaan \kor{vuorotteleva}ksi eli \kor{alternoiva}ksi (engl.\ alternating), jos sen 
termien etumerkki vaihtuu aina indeksistä seuraavaan siirryttäessä. Vuorotteleva sarja on siis
muotoa $\sum_k (-1)^k a_k$, missä jonon $\seq{a_k}$ termit ovat samanmerkkiset.
\begin{Exa} \label{muuan vuorotteleva sarja} Sarja
\[
\sum_{k=0}^\infty (-1)^k \dfrac{1}{\sqrt{k+1}}\ 
=\ \left\{1\,,\ 1-\frac{1}{\sqrt{2}}\,,\ 1-\frac{1}{\sqrt{2}}
                                          +\frac{1}{\sqrt{3}}\,,\,\ldots\,\right\}
\]
on vuorotteleva. \loppu \end{Exa}
Esimerkkisarjaan soveltuu seuraava yleisempi väittämä.
\begin{Lause} \label{alternoiva sarja} Jos vuorottelevalle sarjalle 
$\,\sum_{k=0}^\infty (-1)^k a_k\,$ pätee \ (i) $\,a_k>0\ \forall k$, (ii) jono $\seq{a_k}$
on vähenevä, ts.\ $\,a_0\ \ge\ a_1\ \ge\ a_2\ \ge\ \ldots\,$ ja \ (iii) $\,\lim_k a_k = 0$,
niin sarja suppenee. Lisäksi approksimoitaessa sarjan summaa $s$ osasummilla $s_n$ on voimassa
arvio
\[ 
\abs{s - s_n}\ \le\ a_{n+1}\,. 
\] 
\end{Lause}
\tod Tarkastellaan sarjan osasummia $\,s_n = \sum_{k=0}^n (-1)^k a_k\,$ erikseen parillisilla
ja parittomilla indeksin arvoilla. Koska
\begin{align*}
s_{2n}    &=\ (a_0-a_1) + \ldots + (a_{2n-2}-a_{2n-1}) + a_{2n} \\    
          &=\ a_0 - (a_1 - a_2) - \ldots - (a_{2n-1} - a_{2n}), \\[2mm]
s_{2n+1}\ &=\ (a_0 - a_1) + \ldots + (a_{2n} - a_{2n+1}) \\
          &=\ a_0 - (a_1 - a_2) - \ldots - (a_{2n-1}-a_{2n}) - a_{2n+1}.
\end{align*}
niin oletuksista (i)--(ii) seuraa, että $\,0 \le s_n \le a_0\,$ sekä parillisilla että
parittomilla $n$:n arvoilla. Siis $\seq{s_n}$ on rajoitettu jono. Samoin nähdään, että jono 
$\seq{s_{2n}}$ on vähenevä ja $\seq{s_{2n+1}}$ on kasvava, joten molemmat suppenevat. Koska
edelleen $\,s_{2n} - s_{2n+1} = a_{2n+1} \kohti 0$ oletuksen (iii) mukaan, niin jonoilla
$\seq{s_{2n}}$ ja $\seq{s_{2n+1}}$ on yhteinen raja-arvo. Tästä on helposti pääteltävissä,
että koko jono $\seq{s_n}$ suppenee.

Väittämän toisen osan todistamiseksi olkoon ensin $n$ pariton. Tällöin
\begin{align*}
s - s_n\ &=\ a_{n+1} - (a_{n+2} - a_{n+3}) -\ \ldots \\
         &=\ (a_{n+1} - a_{n+2}) + (a_{n+3} - a_{n+4}) +\ \ldots,
\end{align*}
joten oletusten perusteella $\,0 \le s - s_n \le a_{n+1}$. Jos $n$ on parillinen, niin 
päätellään vastaavasti, että $\,- a_{n+1} \le s - s_n \le 0$. Siis molemmissa tapauksissa pätee
väitetty arvio. \loppu

\jatko \begin{Exa} (jatko) Esimerkkisarja on Lauseen \ref{alternoiva sarja} perusteella
suppeneva. Summaksi saadaan (numeerisin keinoin) $\,s=0.6048986434\,..\,$ Tässä
tapauksessa arvio $\abs{s-s_n} \le a_{n+1}$ tarkentuu suurilla $n$:n arvoilla muotoon
$\abs{s-s_n} \approx \tfrac{1}{2}a_{n+1}$, vrt.\ taulukko alla. (\,Tarkemmin on
osoitettavissa: $\,\lim_n\abs{s-s_n}/a_{n+1}=\tfrac{1}{2}.\,$)
\begin{align*}
s - s_{100}        &=\ -0.049628\,.. \qquad\quad a_{101}\ \ =\ 0.099503\,.. \\
s - s_{101}        &=\ +0.049386\,.. \qquad\quad a_{102}\ \ =\ 0.099014\,.. \\[2mm]
s - s_{1000}       &=\ -0.015799\,.. \qquad\quad a_{1001}\  =\ 0.031606\,.. \\
s - s_{1001}       &=\ +0.015791\,.. \qquad\quad a_{1002}\  =\ 0.031591\,.. \loppu
\end{align*}
\end{Exa}

\subsection{Cauchyn kriteeri sarjoille}

Mikäli sarja ei ole positiiviterminen tai vuorotteleva, on suppenemisteoriassa yleensä
turvattava Lauseeseen \ref{Cauchyn kriteeri}. Tämän mukaan sarja $\sum_{k=1}^\infty a_k$
suppenee  (kohti reaalilukua) täsmälleen, kun sen osasummien $s_n = \sum_{k=1}^n a_k$
muodostama jono on Cauchyn jono. Koska $s_m - s_n = \sum_{k=n+1}^m a_k$, kun $m>n$, niin
Cauchyn jonon määritelmän perusteella pätee siis
\begin{Lause} (\vahv{Cauchyn kriteeri sarjoille}) \label{Cauchyn sarjakriteeri}
\index{Cauchyn!d@kriteeri sarjoille|emph} Sarja $\sum_{k=1}^n a_k$ suppenee täsmälleen, kun
jokaisella $\eps > 0$ on olemassa $N \in \N$ siten, että pätee
\[ 
\abs{\sum_{k=n+1}^m a_k\,}\ <\ \eps, \quad \text{kun}\ m>n>N. 
\] 
\end{Lause}
Kun kriteerissä valitaan $m=n+1$, niin päädytään seuraaviin ilmeisiin (keskenään loogisesti
yhtäpitäviin) johtopäätöksiin:
\begin{Kor} \label{Cauchyn korollaari 1} Jos sarja $\,\sum_k a_k\,$ suppenee,
niin $\,a_k \kohti 0$. 
\end{Kor}
\begin{Kor} \label{Cauchyn korollaari 2} Jos $\,a_k \not\kohti 0$, niin sarja 
$\,\sum_k a_k\,$ hajaantuu. 
\end{Kor}
Jos sarjan termien etumerkit vaihtelevat, niin sarjan suppenemiskysymyksen voi yrittää
ratkaista yksinkertaisesti vertaamalla positiivitermiseen sarjaan, josta etumerkkien vaihtelu
on poistettu. Onnistuessaan tämä yritys perustuu seuraavaan lauseeseen. Ensin määritelmä:
\begin{Def} \index{itseinen suppeneminen|emph} Sarja $\sum_k a_k$ suppenee \kor{itseisesti}
(engl.\ absolutely), jos sarja $\sum_k \abs{a_k}$ suppenee.
\end{Def}
\begin{Lause} Jos sarja suppenee itseisesti, niin se suppenee. 
\end{Lause}
\tod Kolmioepäyhtälön nojalla on
\[ 
\abs{\sum_{k=n+1}^m a_k\,}\ \le\ \sum_{k=n+1}^m \abs{a_k} \quad (m>n). 
\]
Koska sarja $\sum_k \abs{a_k}$ suppenee, on tässä Lauseen \ref{Cauchyn sarjakriteeri} mukaan
jokaisella $\eps > 0$ löydettävissä $N \in \N$ siten, että oikea puoli $< \eps$ kun $n>N$, 
jolloin saman kriteerin perusteella myös sarja $\sum_k a_k$ suppenee. \loppu
\begin{Exa} Jos $\seq{b_k}$ on lukujono, jolle pätee $\,b_k \in [-1,1]\ \forall k$, niin sarja
\linebreak $\ \sum_{k=1}^\infty b_k/k^2$ suppenee itseisesti (majoranttiperiaate ja Lause 
\ref{harmoninen sarja}). Sarja siis suppenee, valittiinpa luvut $b_k$ väliltä $[-1,1]$ miten
tahansa. \loppu
\end{Exa}
\begin{Exa} Sarja $\,\sum_{k=1}^\infty (-1)^k/\sqrt{k}\,$ suppenee
(Lause \ref{alternoiva sarja}) mutta ei itseisesti (Lause \ref{harmoninen sarja}). \loppu
\end{Exa}

\subsection{Potenssisarja} 
\index{potenssisarja|vahv} \index{sarja!f@potenssisarja|vahv}%

Sarjaa muotoa
\[
\sum_{k=0}^\infty a_k x^k\ =\ a_0 + a_1\,x + a_2\,x^2 + \ldots
\]
sanotaan \kor{potenssisarja}ksi (engl.\ power series). Luvut $a_k$ ovat nimeltään 
\index{kerroin (potenssisarjan)}%
\kor{potenssisarjan kertoimet}, ja symbolia $x$, joka myös edustaa reaalilukua, sanotaan sarjan
\kor{muuttuja}ksi. Termi 'muuttuja' kertoo, että $x$:n lukuarvo voi vaihdella. Kyseessä ei siis
ole pelkästään lukujono, vaan pikemminkin joukko lukujonoja, missä kertoimet $a_k$ ajatellaan 
kiinnitetyiksi ja muuttujalle $x$ sallitaan erilaisia (reaali)arvoja.\footnote[2]{Tapauksessa
$x=0$ potenssisarja tulkitaan lukujonoksi $\{a_0,a_0,\ldots\}$, eli sarjamerkinnässä sovitaan,
että $0^0=1$.}

Kuten tullaan havaitsemaan, potenssisarja on matematiikassa hyvin keskeinen käsite ja työkalu.
Tässä tutkitaan toistaiseksi vain sarjan suppenemiskysymystä, jonka luonteva muotoilu on: Millä
$x$:n arvoilla sarja suppenee? Koska suppenevan sarjan tapauksessa summa yleensä riippuu $x$:n
arvosta, ilmaistaan tämä kirjoittamalla sarjan summaksi $s(x)$, luetaan '$s$ $x$'.
\begin{Exa} Perusmuotoinen geometrinen sarja $\sum_{k=0}^\infty x^k$ on esimerkki 
potenssisarjasta. Sarja suppenee täsmälleen kun $x \in (-1,1)$, ja sarjan summa on tällöin 
$s(x) = 1/(1-x)$. \loppu 
\end{Exa}
Esimerkin perusteella potenssisarjaa voi pitää geometrisen sarjan yleistyksenä. Osoittautuukin,
että suppenevuustarkasteluissa geometrinen sarja on hyvä vertailukohta. Tällaiseen vertailuun 
perustuu esimerkiksi seuraava, potenssisarjojen suppenemisteorian keskeisin tulos.
\begin{Lause} \vahv{(Potenssisarjan suppeneminen)} \label{suppenemissäde}
\index{potenssisarja!a@suppenemissäde|emph} \index{suppenemissäde|emph} Potenssisarjalle
$\sum_{k=0}^\infty a_k x^k$ on voimassa jokin seuraavista vaihtoehdoista:
\begin{itemize}
\item[(a)] Sarja suppenee vain kun $x=0$.
\item[(b)] $\exists \rho\in\R_+$ siten, että sarja suppenee, myös itseisesti, kun 
           $x\in (-\rho,\rho)$ ja hajaantuu aina kun $\abs{x}>\rho$. Tällöin $\rho$ on sarjan
           \kor{suppenemissäde}.
\item[(c)] Sarja suppenee itseisesti $\forall x\in\R$.
\end{itemize}
\end{Lause}
\tod Riittää osoittaa, että (b) tai (c) toteutuu silloin kun (a) ei. Oletetaan siis, että sarja
suppenee jollakin $x_0 \in \R$, $x_0 \neq 0$. Tällöin on oltava (ks.\ Korollaari
\ref{Cauchyn korollaari 1} ja Lause \ref{suppeneva jono on rajoitettu}\,)
\[
a_nx_0^n\kohti 0 \ \impl \ \abs{a_nx_0^n}\leq C\quad \forall n \quad (C\in\R_+).
\]
Tästä seuraa, että jos $\abs{x}<\abs{x_0}$, niin
\[
\abs{a_nx^n}\leq C\abs{x/x_0}^n,
\]
jolloin majoranttiperiaatteen nojalla sarja suppenee itseisesti geometrisen sarjan 
($q=\abs{x/x_0}<1$) tavoin. Siis on todettu: Jos sarja suppenee kun $x=x_0\neq 0$, niin se 
suppenee itseisesti $\forall x\in (-\abs{x_0},\abs{x_0})$. Tästä seuraa edelleen, että jos sarja
hajaantuu kun $x=x_1$, niin se hajaantuu aina kun $\abs{x}>\abs{x_1}$, sillä muutoin se jo 
todetun perusteella olisi sekä suppeneva että hajaantuva, kun $x=x_1$. On siis päätelty: Jos (a)
ei ole voimassa, niin joko (c) sarja suppenee jokaisella $x \in \R$, jolloin se suppenee 
jokaisella $x \in \R$ myös itseisesti, tai (b) sarja suppenee kun $\abs{x} < \rho$ ja hajaantuu
kun $\abs{x} > \rho$, missä $\rho$ määritellään (vrt.\ Lause \ref{supremum-lause})
\[
\rho=\sup\,\left\{x \in \R \mid \text{sarja $\sum_{k=0}^\infty a_k x^k$ suppenee}\right\}.
\]
Lause on näin todistettu. \loppu

Lauseen vaihtoehdoissa (a), (c) voidaan sopia merkintätavoista
\[
\text{(a)} \,\ \rho=0, \qquad \text{(c)}\,\ \rho=\infty.
\]
Sikäli kuin potenssisarjan suppenemissäde pystytään määräämään, ratkaisee siis
Lause \ref{suppenemissäde} sarjan suppenemiskysymyksen täydellisesti, lukuun ottamatta 
muuttujan arvoja  $x = \rho$ ja $x = -\rho$, kun $\rho \in \R_+\,$. Nämä on selvitettävä 
tapauskohtaisesti, vrt.\ Esimerkki \ref{suppenemisvälejä} jäljempänä. Keskeisimmässä 
laskennallisessa ongelmassa, eli suppenemissäteen määräämisessä, auttaa usein seuraava tulos:
\begin{Lause} \vahv{(Potenssisarjan suppenemissäde)} \label{suppenemissäteen laskukaava}
\index{potenssisarja!a@suppenemissäde|emph} \index{suppenemissäde|emph} Jos potenssisarjassa
\newline
$\sum_{k=0}^\infty a_k x^k$ on $a_k \neq 0\ \forall k\,$ ja on olemassa raja-arvo
\[
\rho=\lim_{k\kohti\infty} \left|\frac{a_k}{a_{k+1}}\right|,
\]
missä $\rho\in\R_+$, $\rho=0$ tai $\rho=\infty$, niin sarjan suppenemissäde $=\rho$.
\end{Lause}
\tod Olkoon ensin $\rho>0$ ja $\abs{x} < \rho$. Valitaan $q$ siten, että $\abs{x} < q < \rho$ 
(esim.\ $q = (\abs{x}+\rho)/2$). Tällöin koska $\abs{a_k}/\abs{a_{k+1}} \kohti \rho$ ja 
$q < \rho$, niin jostakin indeksistä $k=m$ alkaen on $\abs{a_k}/\abs{a_{k+1}} \ge q$ 
(lukujonon suppenemisen määritelmässä valittu $\eps = \rho-q$ ja $m>N$). Tällöin voidaan
päätellä
\[
\abs{a_{k+1}} \le q^{-1}\,\abs{a_k},\quad k = m,m+1, \ldots 
               \qimpl \abs{a_k} \le q^{m-k}\abs{a_m}, \quad k \ge m,
\]
jolloin
\[
\abs{a_k x^k} \le q^{m-k}\abs{a_m}\abs{x}^k = q^m\,\abs{a_m}\left(\frac{\abs{x}}{q}\right)^k 
                                            = C\left(\frac{\abs{x}}{q}\right)^k, \quad k \ge m.
\]
Tässä on $\abs{x}/q < 1$, joten päätellään majoranttiperiaatteen nojalla, että sarja 
$\sum_{k=0}^\infty a_k x^k$ suppenee itseisesti, indeksistä $k=m$ eteenpäin geometrisen sarjan
tavoin. Siis jos $\rho \in \R_+$, niin sarja suppenee itseisesti kun $\abs{x} < \rho$. Jos 
oletettu raja-arvo on $\rho = \infty$, niin ym.\ päättelyssä voidaan $\rho \in \R_+$ valita 
miten tahansa, joten tässä tapauksessa myös suppenemissäde $= \infty$.

Jos $\rho\in\R_+$ ja $\abs{x} > \rho$, niin nähdään samalla tavoin kuin yllä, että indeksin $m$
ollessa riittävän suuri pätee jokaisella $k \ge m$
\[
\abs{a_k x^k} \ge q^{m-k}\abs{a_m}\abs{x}^k = q^m\,\abs{a_m}\left(\frac{\abs{x}}{q}\right)^k 
                                            = C\left(\frac{\abs{x}}{q}\right)^k, \quad k \ge m.
\]
missä nyt $\rho < q < \abs{x}$. Näin ollen $\abs{a_k x^k}\kohti \infty$ kun $k \kohti \infty$, 
jolloin sarja $\sum_k a_k x^k$ hajaantuu (Korollaari \ref{Cauchyn korollaari 2}). Siis jos
$\rho \in \R_+$, niin sarja suppenee kun $\abs{x} < \rho$ ja hajaantuu kun $\abs{x} > \rho$,
eli sarjan suppenemissäde $= \rho$. Jos $\rho = 0$, niin sarja todetaan samalla päättelyllä
hajaantuvaksi aina kun $x \neq 0$, eli tässä tapauksessa suppenemissäde $= 0$. Lause on näin
todistettu. \loppu
\begin{Exa} Määritä potenssisarjan suppenemissäde, kun kertoimet ovat
\[
\text{a) } a_k=k! \quad \text{b) } a_k=(-1)^k (k+1)^{-1}\quad 
\text{c) } a_k=k^2\cdot 3^{-k} \quad \text{d) } a_k=1/k!
\]
\end{Exa}
\ratk Lauseen \ref{suppenemissäteen laskukaava} perusteella
\begin{align*}
\text{a) } \rho &= \lim_k 1/(k+1) = 0 \qquad\quad\,\   
\text{b) } \rho = \lim_k\,(k+2)/(k+1) = 1 \\
\text{c) } \rho &= \lim_k\,3k^2/(k+1)^2 = 3 \qquad     
\text{d) } \rho = \lim_k\,(k+1) = \infty \qquad\loppu
\end{align*}
Jos potenssisarja suppenee, kun $x \in A$, ja hajaantuu, kun $x \not\in A$, niin joukkoa 
$A\ (A\subset\R)$ sanotaan ko.\ sarjan
\index{potenssisarja!ab@suppenemisväli} \index{suppenemisväli}%
\kor{suppenemisväli}ksi. Lauseen \ref{suppenemissäde}
mukaisesti $A$ on jokin vaihtoehdoista $(-\rho,\rho)$, $[-\rho,\rho)$, $(-\rho,\rho]$, 
$[-\rho,\rho]$, missä $\rho=$ suppenemissäde. 
\begin{Exa} \label{suppenemisvälejä} Määritä potenssisarjan  $\sum_{k=0}^\infty a_k x^k$ 
suppenemisväli, kun 
\[
\text{a)}\ a_k = 1 \quad \text{b)}\ a_k = 1/(k+1) \quad 
\text{c)}\ a_k = (-1)^k/(k+1) \quad \text{d)}\ a_k = 1/(k+1)^2
\]
\end{Exa}
\ratk Lauseen \ref{suppenemissäteen laskukaava} laskukaava antaa suppenemissäteeksi $\rho = 1$
kaikissa tapauksissa, joten sarjat suppenevat, kun $\abs{x} < 1$, ja hajaantuvat, kun 
$\abs{x} > 1$. Tapauksissa $x = \pm 1$ esimerkkisarjat edustavat jo entuudestaan tuttuja 
geometrisen, harmonisen, alternoivan ja yliharmonisen sarjan tyyppejä. Tulokset yhdistämällä 
saadaan vastaukseksi
\[
\text{a)}\ (-1,1), \quad \text{b)}\ [-1,1), \quad \text{c)}\ (-1,1], 
\quad \text{d)}\ [-1,1]. \loppu
\]
\begin{Exa} Määritä suppenemisväli potenssisarjalle
\[
\sum_{k=0}^\infty (-1)^k (k^2+1) \dfrac{x^{2k}}{2^k}\ =\ 1 - x^2 + \dfrac{5}{4} x^4 - \ldots
\]
\end{Exa}
\ratk Lauseen \ref{suppenemissäteen laskukaava} laskukaava suppenemissäteelle ei sovellu 
suoraan, koska sarja on muotoa $\sum_{k=0}^\infty a_k x^k$, missä $a_k = 0$ parittomilla 
indeksin arvoilla. Tehtävä ratkeaa kuitenkin yksinkertaisesti vaihtamalla muuttujaksi $t=x^2$,
jolloin sarja saa muodon $\sum_{k=0}^\infty b_k t^k$. Lause \ref{suppenemissäteen laskukaava}
soveltuu tähän sarjaan, sillä
\[
\lim_k \left|\dfrac{b_k}{b_{k+1}}\right|\ 
               =\ \lim_k \dfrac{(k^2 +1)\,2^{-k}}{[(k+1)^2 + 1]\,2^{-k-1}}\ =\ 2.
\]
Siis sarja suppenee, kun $\abs{t} = x^2 < 2$, ja hajaantuu, kun $\abs{t} = x^2 > 2$. 
Suppenemissäde muuttujan $x$ suhteen on näin ollen $\rho = \sqrt{2}$. Muuttujan arvoilla 
$x=\pm\sqrt{2}$ sarja nähdään hajaantuvaksi, joten suppenemisväli on $(-\sqrt{2},\sqrt{2})$.
\loppu
\begin{Exa} Millä arvoilla $x \in \R_+$ suppenee sarja 
\[ 
\sum_{k=0}^\infty \dfrac{x^{2k-1/2}}{k!}\ 
          =\ \dfrac{1}{\sqrt{x}} + x\sqrt{x} + \dfrac{x^3\sqrt{x}}{2} + \ldots \ \ ?
\]
\end{Exa}
\ratk Sikäli kuin sarja suppenee, niin sekä osasummista että raja-arvosta $s(x)$ voidaan
erottaa skaalaustekijä $1/\sqrt{x}$ (vrt.\ Lause \ref{raja-arvojen yhdistelysäännöt}),
jolloin
\[
s(x)\ =\ \dfrac{1}{\sqrt{x}} \sum_{k=0}^\infty \dfrac{(x^2)^k}{k!}\,.
\]
Tässä $\sum_{k=0}^\infty x^{2k}/k!\,$ on tavanomainen potenssisarja, joka suppenee 
$\forall x\in\R$, joten kysytty sarja suppenee jokaisella $x \in \R_+$. \loppu

Päätetään potenssisarjojen ensiesittely tulokseen, jolla on myöhempää käyttöä.
\begin{Lause} \label{potenssisarjan skaalaus} Potenssisarjoilla $\sum_{k=1}^\infty k^m a_k x^k$
on sama suppenemissäde $\forall m\in\Z$. 
\end{Lause}
\tod Olkoon sarjan $\sum_ka_kx^k$ ($m=0$) suppenemissäde joko $\rho\in\R_+$ tai $\rho=\infty$
ja olkoon $0<|x|<\rho$. Valitaan $q\in(0,1)$ siten, että $|x|/q<\rho$ 
(esim.\ $q=2|x|/(|x|+\rho)$, jos $\rho\in\R_+$, tai $q=1/2$, jos $\rho=\infty$), ja
kirjoitetaan
\[
k^m|a_k||x|^k = (k^mq^k)|a_k|\left(\frac{|x|}{q}\right)^k, \quad k \ge 1,\ m\in\Z.
\]
Tässä lukujono $\seq{k^mq^k,\ k=1,2,\ldots}$ on rajoitettu (koska se on suppeneva, ks.\
Harj.teht.\,\ref{H-I-12: raja-arvotulos}), joten jollakin $C\in\R_+$ pätee
\[
k^m|a_k||x|^k \le C|a_k|\left(\frac{|x|}{q}\right)^k, \quad k=1,2,\ldots
\]
Oletuksen mukaan sarja $\sum_k|a_k|(|x|/q)^k$ suppenee (koska oli $|x|/q<\rho$), joten
majoranttiperiaatteen nojalla myös sarja $\sum_kk^m|a_k||x|^k$ suppenee. On päätelty, että jos
sarjan $\sum_ka_kx^k$ supenemissäde on $\rho \neq 0$, niin myös sarja $\sum_kk^ma_kx^k$
suppenee itseisesti aina kun $|x|<\rho$ ja jokaisella $m\in\Z$.

Kirjoittamalla $k^ma_k=b_k,\ a_k=k^{-m}b_k$ nähdään, että em.\ päättely on tarkasteltujen
sarjojen $\sum_ka_kx^k$ ja $\sum_kk^ma_kx^k\ (m\in\Z)$ suhteen symmetrinen, ts.\ jos kumman
tahansa suppenemissäde on $\rho \neq 0$, niin toinenkin suppenee itseisesti kun $|x|<\rho$.
Tästä seuraa, että suppenemissäteiden on oltava kaikissa tapauksissa (myös kun $\rho=0$)
samat, sillä muu johtaisi loogiseen ristiriitaan tehtyjen päätelmien kanssa. \loppu  

\Harj
\begin{enumerate}

\item
Tiedetään, että $a_1=3$, $a_2=5$, $a_3=7$, $a_4=-6$ ja $\sum_{k=1}^\infty a_k=4$. Laske \newline
$\sum_{k=1}^\infty(a_k+a_{k+1}+a_{k+2}+a_{k+3})$.

\item
Luokittele seuraavat sarjat suppeneviksi tai hajaantuviksi. Jos sarjassa on parameri $\alpha$,
niin oleta $\alpha\in\Q$ ja luokittele sarja eri $\alpha$:n arvoilla.
\begin{align*}
&\text{a)}\ \ \sum_{k=1}^\infty \frac{1}{k^2-k+3} \qquad\quad\
 \text{b)}\ \ \sum_{k=1}^\infty \frac{k-1}{k^2+1} \qquad\quad\ \
 \text{c)}\ \ \sum_{k=1}^\infty \frac{100+k}{9901+k-k^3} \\
&\text{d)}\ \ \sum_{k=1}^\infty \frac{\sqrt{k+1}-\sqrt{k}}{k} \qquad 
 \text{e)}\ \ \sum_{k=0}^\infty \frac{1}{\sqrt{k^2+100}} \qquad
 \text{f)}\ \ \sum_{k=1}^\infty \left(\sqrt{\frac{k+1}{k}}-1\right) \\
&\text{g)}\ \ \sum_{k=1}^\infty \frac{\sqrt{k+1}\sqrt[3]{k+2}}{k^2+k+1} \quad\
 \text{h)}\ \ \sum_{k=1}^\infty \frac{1+k!}{(1+k)!} \qquad\quad\
 \text{i)}\ \ \sum_{k=1}^\infty \frac{2+k!}{(2+k)!} \\
&\text{j)}\ \sum_{k=0}^\infty (-1)^k\left(\sqrt{k+7}-\sqrt{k+1}\right) \quad\
 \text{k)}\ \sum_{k=1}^\infty \frac{3(-1)^k+1}{k} \quad\
 \text{l)}\ \sum_{k=1}^\infty \frac{(-1)^k}{\sqrt[100]{k}} \\
&\text{m)}\ \ \sum_{k=1}^\infty \left(\frac{1}{k}-\frac{1}{k+\abs{\alpha}}\right) \qquad
 \text{n)}\ \ \sum_{k=1}^\infty \frac{1}{k^\alpha+k^{2\alpha}} \qquad 
 \text{o)}\ \ \sum_{k=1}^\infty \frac{\sqrt{k^\alpha+1}-1}{k} \\
&\text{p)}\ \ \sum_{k=1}^\infty (\sqrt{k^{2\alpha}+1}-k^\alpha) \qquad
 \text{q)}\ \ \sum_{k=1}^\infty \frac{k^\alpha+1}{k^{2\alpha}+1} \qquad
 \text{r)}\ \ \sum_{k=1}^\infty \frac{(-1)^k k^{4\alpha+3}+1}{k^{\alpha+2}}
\end{align*}

\item
Arvioi, montako termiä on sarjoista otettava seuraavissa laskukaavoissa, jotta $\pi$
saadaan oikein $10$ desimaalin tarkkuudella:
\[
\pi\ =\ \sum_{k=0}^\infty \frac{4(-1)^k}{2k+1}\ 
     =\ 2\sqrt{3}\sum_{k=0}^\infty \frac{(-1)^k}{3^k(2k+1)}\
     =\ 2\sqrt{3}\sqrt{\sum_{k=0}^\infty \frac{(-1)^k}{(k+1)^2}}
\]

\item
a) Näytä vuorotteleva sarja $\sum_{k=0}^\infty (-1)^k/\sqrt{k+1}$ suppenevaksi
majoranttiperiaatteen avulla ryhmittelemällä sarjan peräkkäiset termit yhteen niin, että
sarjasta tulee positiiviterminen. \vspace{1mm}\newline
b) Tarkastellaan sarjaa $\sum_{k=1}^\infty a_k$, missä
\[
a_{2p-1}=\frac{1}{\sqrt{p+1}-1}\,, \quad a_{2p}=-\frac{1}{\sqrt{p+1}+1}\,, \quad p=1,2,\ldots
\]
Mitkä Lauseen \ref{alternoiva sarja} oletuksista ovat voimassa tälle sarjalle\,? Ratkaise
sarjan suppenemiskysymys  ryhmittelemällä sarjan peräkkäiset termit parittain yhteen, eli
tutkimalla sarjaa $\sum_{p=1}^\infty b_p = \sum_{p=1}^\infty (a_{2p-1}+a_{2p})$.

\item
Osoita oikeaksi tai vääräksi: \newline
a) \ $\sum_k a_k$ suppenee $\,\ \impl\,$ $\seq{a_k}$ on rajoitettu jono \newline
b) \ $\sum_k a_k$ suppenee ja $\sum_k b_k$ suppenee $\ \,\impl\,$ $\sum_k(a_k+b_k)$ 
     suppenee \newline
c) \ $\sum_k a_k$ hajaantuu ja $\sum_k b_k$ hajaantuu $\,\ \impl\,$ $\sum_k(a_k+b_k)$ 
     hajaantuu \newline
d) \ $\sum_k a_k$ suppenee ja $\sum_k b_k$ hajaantuu $\,\ \impl\,$ $\sum_k(a_k+b_k)$ 
     hajaantuu \newline
e) \ $\sum_k a_k$ suppenee ja $\seq{b_k}$ rajoitettu $\,\ \impl\,$ $\sum_k a_k b_k$
     suppenee \newline
f) \ $\sum_k \abs{a_k}$ suppenee ja $\seq{b_k}$ rajoitettu $\,\ \impl\,$
     $\sum_k \abs{a_k b_k}$ suppenee \newline
g) \ $\sum_k a_k$ suppenee $\,\ \impl\,$ $\sum_k a_k^2$ suppenee \newline
h) \ $\sum_k a_k^2$ suppenee $\,\ \impl\,$ $\sum_k a_k$ suppenee \newline
i) \ $\,\sum_k a_k$ suppenee itseisesti $\,\ \impl\,$ $\sum_k a_k^4$ suppenee

\item \index{suhdetesti (sarjaopin)}
Klassisen sarjaopin \kor{suhdetesti} positiivitermisille sarjoille on väittämä: Jos on 
olemassa raja-arvo
\[
\lim_k \frac{a_{k+1}}{a_k} = L\in\R,
\]
niin sarja $\sum_k a_k$ suppenee, jos $L<1$, ja hajaantuu, jos $L>1$. Todista!

\item \index{juuritesti (sarjaopin)}
Klassisen sarjaopin \kor{juuritesti} positiivitermisille sarjoille on väittämä: Jos on
olemassa raja-arvo
\[
\lim_k \sqrt[k]{a_k} = L\in\R,
\]
niin sarja $\sum_k a_k$ suppenee, jos $L<1$, ja hajaantuu, jos $L>1$. Todista!

\item \label{H-I-12: raja-arvotulos}
Todista potenssisarjojen teorian (Lauseet \ref{suppenemissäde} ja 
\ref{suppenemissäteen laskukaava}) ja Korollaarin \ref{Cauchyn korollaari 1} avulla
raja-arvotulos
\[
\lim_k k^m q^k = 0, \quad -1 < q < 1,\ m\in\N.
\]

\item
Olkoon potenssisarjojen $\sum_k a_k x^k$ ja $\sum_k b_k x^k$ suppenemissäteet $\rho_1$ ja
$\rho_2$. Mitä varmaa voidaan sanoa sarjan $\sum_k(a_k+b_k)\,x^k$ suppenemissäteestä, jos \ 
a) $\rho_1<\rho_2$, \ b) $\rho_2=\rho_1\in\R_+$, \ c) $\rho_1=\rho_2=\infty$, \
d) $\rho_1=\rho_2=0$\,?

\item
a) Potenssisarjan kertoimista tiedetään, että $k \le a_k \le k^2\ \forall k\in\N$. Mikä on
sarjan suppenemisväli? \newline
b) Potenssisarjan kertoimet ovat $a_k=1$, kun $k=m!\,$ ja $m\in\N$, muulloin on $a_k=0$. Mikä
on sarjan summa $20$ merkitsevän numeron tarkkuudella, kun $x=1/10$, ja mikä on sarjan
suppenemisväli? 

\item
Määritä seuraavien potenssisarjojen suppenemisvälit:
\begin{align*}
&\text{a)}\ \  \sum_{k=0}^\infty \frac{k^2}{k+1}x^k \quad\
 \text{b)}\ \ \sum_{k=0}^\infty (k-2)2^{k+1}x^k \quad\
 \text{c)}\ \  \sum_{k=1}^\infty (-1)^k (\frac{x}{k})^k \\
&\text{d)}\ \ \sum_{k=1}^\infty \frac{1}{\sqrt{k}} x^k \quad\ 
 \text{e)}\ \ \sum_{k=1}^\infty \frac{(-1)^k}{\sqrt{k}} 3^k x^k \quad\
 \text{f)} \ \ \sum_{k=1}^\infty \frac{\sqrt{k+1}}{k^2} 2^k x^k \\
&\text{g)}\ \ \sum_{k=1}^\infty \frac{2^k}{k} x^{2k} \quad\ 
 \text{h)}\ \ \sum_{k=1}^\infty \frac{1}{k^3 3^k} x^{3k} \quad \
 \text{i)}\ \ \sum_{k=1}^\infty \frac{1}{k^{1/3} 3^k} x^{3k}
\end{align*}

\item
Totea seuraavat sarjat joko suppeneviksi tai hajaantuviksi potenssisarjojen teoriaan vedoten:
\[
\text{a)}\,\ \sum_{k=1}^\infty (-1)^k k^{-10} \pi^k e^{-k} \quad\ 
\text{b)}\,\ \sum_{k=1}^\infty \frac{k!}{(3k)^k} \quad\
\text{c)}\,\ \sum_{k=1}^\infty \frac{(2k)^k}{k!}
\]

\item
Määritä muuttujan vaihdolla seuraaville sarjoille joukko $A\subset\R$ siten, että sarja 
suppenee täsmälleen kun $x \in A$:
\[
\text{a)}\ \ \sum_{k=1}^\infty \frac{1}{k} (2x+1)^k \quad
\text{b)}\ \ \sum_{k=1}^\infty (x^2-3x+1)^k \quad 
\text{c)}\ \ \sum_{k=1}^\infty \frac{(-1)^k}{k}(x^2-4x+3)^k
\]

\item
Todista Lause \ref{potenssisarjan skaalaus} Lauseen \ref{suppenemissäteen laskukaava} avulla
siinä tapuksessa, että raja-arvo $\lim_k |a_k/a_{k+1}|$ on olemassa.

\item
Näytä, että sarjoilla $\sum_{k=0}^\infty a_k x^k$, $\sum_{k=0}^\infty (k-2)^2 a_k x^k$ ja 
$\sum_{k=1}^\infty k^{-2}(k+2)^3 a_k x^k$ on kaikilla sama suppenemissäde.

\item (*) \label{H-I-12: teleskooppiarvio}
Näytä, että sarjan $\sum_{k=1}^\infty 1/\sqrt{k}$ osasummille $s_n$ pätee
\[
2\sqrt{n+1}-2 \,<\, s_n \,\le\, 2\sqrt{n}-1, \quad n\in\N.
\]
\kor{Vihje}: Vertaa teleskooppisummiin!

\item (*)
Todista: Jos positiivitermisen sarjan $\sum_k a_k$ termien jono $\seq{a_k}$ on vähenevä ja sarja
suppenee, niin $\,\lim_k ka_k=0$.

\item (*)
Esitä jokin rationaalikertoiminen potenssisarja, jonka suppenemisväli on 
$(-\sqrt[3]{e},\sqrt[3]{e}\,]$ ($e=$ Neperin luku).

\end{enumerate} % Klassinen sarjaoppi

\chapter{Vektorit ja analyyttinen geometria}

Ihmisten luoman matemaattisen kulttuurin peruspilareita on matemaattinen malli nimeltä 
\kor{euklidinen taso}. Termi viittaa kreikkalaisen 
\index{Eukleides}%
\hist{Eukleideen} n. 300 eKr kirjoittamaan teokseen Stoikheia eli 'Alkeet'. Eukleideen
pyrkimyksenä oli asettaa selkeät aksiomaattiset perusteet silloiselle matematiikalle,
erityisesti \kor{geometrian} nimellä tunnetulle matematiikan suuntaukselle. Geometria on ---
toisin kuin edellä tarkasteltu lukujen algebra --- näköaistimuksiin hyvin voimakkaasti vetoava
matematiikan laji. Algebra ja geometria edustavat nykyisessäkin matematiikassa kahta
matemaattisen ajattelun suurta päähaaraa.\footnote[2]{Algebran ja geometrian välinen rajanveto
matematiikassa ei ole aina helppoa. Ongelmaa kuvastaa esim.\ moderni matematiikan laji nimeltä
\kor{algebrallinen geometria}.}

Eukleideen teos on kaikkien aikojen menestynein matematiikan oppikirja, jonka vaikutus
varhaisena aksiomaattisen ajattelun esikuvana on ulottunut matematiikan ulkopuolellekin.
Käytännön kannalta 'tyhjästä' teos ei tietenkään syntynyt, vaan sitä edelsi itse asiassa
vuosituhantinen laskentamenetelmien perinne sekä Egyptissä että Babyloniassa. Esimerkiksi
Egyptissä geometrisia menetelmiä olivat kehittäneet (jos historioitsijoihin on uskominen) sekä
maanmittarit että papit.

Tässä luvussa lähdetään euklidisen geometrian perusteista tasossa ja avaruudessa ja paneudutaan
tarkastelemaan geometrian pohjalta syntyvää \kor{vektorin} käsitettä ja vektoreilla laskemista 
eli \kor{vektorialgebraa}. Vektoreiden avulla geometriset ongelmat on mahdollista 
'algebralisoida' eli muotoilla ja ratkaista vektorialgebran ongelmina. Tällaista --- 
historiallisesti melko myöhäsyntyistä --- lähestymistapaa geometriaan kutsutaan 
\kor{analyyttiseksi} geometriaksi. % Vektorit ja analyyttinen geometria
\section{Euklidinen taso. Geometriset luvut} \label{geomluvut}
\alku
\sectionmark{Euklidinen taso}

Oletetaan lähtökohtaisesti tutuiksi mm.\ sellaiset geometrian käsitteet kuin
\begin{itemize}
\item  \kor{piste}, \kor{suora}, \kor{jana} 
\item  \kor{kulma}, \kor{kolmio}, \kor{$n$-kulmio}, \kor{monikulmio}, \kor{yhdenmuotoisuus}
\item  \kor{suunta} (\kor{puolisuora}, säde), \kor{yhdensuuntaisuus}, \kor{suunnikas}
\item  \kor{suora kulma}, \kor{Pythagoraan lause}, \kor{suorakulmio}
\item  \kor{ympyrä}, janan \kor{pituusmitta} ('mittatikku')
\end{itemize}
Näiden taustalla ovat euklidisessa geometriassa lähtökohtana oletetut perusaksioomat, jotka 
tunnetaan \kor{Hilbertin aksioomina}\footnote[2]{Saksalainen \hist{David Hilbert} (1862-1943)
oli modernin matematiikan suunnannäyttäjä ja kaikkien aikojen huomattavimpia matemaatikkoja. 
Hilbert tunnetaan mm.\ hänen v. 1900 asettamistaan ratkaisemattomista matematiikan ongelmista, 
jotka sittemmin on nimetty \kor{Hilbertin ongelmiksi}. Euklidisen geometrian perusaksioomat ovat
peräisin Hilbertin pienestä klassikkoteoksesta ''Grundlagen der Geometrie'' vuodelta 1899. Teos
oli ensimmäinen kattava ja systemaattinen esitys geometrian --- myös eräiden epäeuklidisten 
geometrioiden ---  aksiomaattisista perusteista. \index{Hilbert, D.|av}}.

\index{euklidinen!a@taso (pisteavaruus) \Ekaksi} \index{pisteavaruus}%
\kor{Euklidisen tason} perusolioita ovat p\pain{isteet}, joista muiden olioiden
(kuten jana, suora ym.) ajatellaan koostuvan. Euklidista tasoa sanotaankin
\kor{pisteavaruudeksi}. Sen symbolinen merkki on \Ekaksi, luetaan 'E kaksi'.
\vspace{5mm}
\begin{figure}[htb]
\begin{center}
\epsfig{file=kuvat/kuvaII-1.eps}
\end{center}
%\caption{Euklidisen tason olioita}
\label{fig:oliot}
\end{figure}

Sellaiset geometriset operaatiot kuin yhdensuuntaisten suorien piirtäminen, suoran kulman
konstruoiminen, tai annetun geometrisen olion (pistejoukon) \kor{siirto} ja \kor{kierto}
\index{euklidinen liike}%
eli nk.\ \kor{euklidinen liike} (esim.\ janan siirtely pituus säilyttäen) tapahtuvat
euklidisessa geometriassa ympyräviivojen (käytännössä harpin) avulla. Myös suora on 
konstruoitavissa pelkästään harppia käyttäen, sillä suora on sellaisten pisteiden joukko
eli \kor{ura}, jotka ovat yhtä etäällä kahdesta annetusta pisteestä. --- Viivoitinta 
tarvitaan geometriassa siis vain pisteiden 'massatuotantovälineenä'. 

\subsection{Kulma}
\index{kulma|vahv}%

Kulman määrittelee geometriassa kaksi suuntaa. Konkreettisemmin kulma on ajateltavissa
pistejoukoksi, jonka muodostavat kaksi samasta pisteestä lähtevää puolisuoraa, tai
vaihtoehtoisesti näiden puolisuorien rajaamaksi euklidisen tason \kor{sektoriksi}.
Jälkimmäisessä tulkinnassa (ei edellisessä!) on tehtävä ero \kor{sisäkulman} ja 
\kor{ulkokulman} välillä. Jos $A$ ja $B$ ovat puolisuorilla olevia pisteitä ja $O$ on 
puolisuorien yhteinen piste eli kulman \kor{kärki} ($A \neq O$, $B \neq O$), niin kulma
merkitään $\kulma AOB$. Kuvioon on merkitty sisäkulma (sektoritulkinta!) ympyräkaarella.  
\begin{figure}[H]
\setlength{\unitlength}{1cm}
\begin{center}
\begin{picture}(6,3)(-2,0)
%\Thicklines
\put(0,0){\line(3,2){4}} \put(0,0){\line(-3,3){3}}
\put(-1.55,1.45){\line(1,1){0.1}} 
\put(2.7,1.8){\line(-2,3){0.04}} \put(2.7,1.8){\line(2,-3){0.04}}
\put(-0.1,-0.5){$O$} \put(2.75,1.4){$A$} \put(-1.85,1.05){$B$}
\put(0,0){\arc{1}{-2.356}{-0.61}}
\end{picture}
\end{center}
\end{figure}
Kulmien vertailussa ja laskuoperaatioissa (myös kulman mittauksessa jäljempänä) tulkitaan
kulma yleensä sektoriksi. Sanotaan tällöin, että kaksi kulmaa ovat \kor{yhtä suuret}, jos
ne voidaan muuntaa toisikseen euklidisella liikkeellä.\footnote[2]{Kulmien yhtäsuuruus on
kulmien joukossa määritelty ekvivalenssirelaatio, vrt.\ Luku \ref{logiikka}.} 
Yhtäsuuruus on siis tarkistettavissa geometrisesti. Myös kahden kulman \kor{summa} ja 
\kor{erotus} voidaan määrätä geometrisesti siirtämällä toista kulmaa euklidisella liikkeellä
niin, että kulmat joutuvat 'vierekkäin' (summa) tai 'päällekkäin' (erotus). Geometrisella
yhteenlaskulla on esimerkiksi todistettavissa tunnettu kolmion kulmia koskeva väittämä: 
Kulmien summa = \kor{oikokulma} (Harj.teht.\,\ref{H-II-1: kulmat}b).

Jos kahdessa kolmiossa (vastin)kulmat ovat yhtä suuret, niin kolmiot ovat
\index{kulma!a@yhdenmuotoisuus} \index{yhdenmuotoisuus}%
\kor{yhdenmuotoiset}. Geometrian perusväittämiin kuuluu, että yhdenmuotoisissa kolmioissa
vastinsivujen pituuksien suhteet ovat samat. Väittämä näyttelee keskeistä roolia monissa
geometrisissa todistuksissa (ks.\ esim.\ Harj.teht.\,\ref{H-II-1: yhdenmuotoisuus}). 

\subsection{Geometriset luvut}
\index{geometrinen luku|vahv}%
\index{laskuoperaatiot!ba@geometristen lukujen|vahv}

Kulmien mittalukuja ei euklidisen geometrian perusoperaatioissa tarvita. Sen sijaan janan 
pituuden mittaus harpin ja mittajanan avulla on perusgeometriaa. Operaatio tuo geometriaan 
reaaliluvut (tai ainakin osan niistä) ja myös lukuihin liittyvät laskutoimitukset.
Lähtökohtana pidetään lukuja $0$ ja $1$:
\[
0\ \vast\ \text{piste}, \quad 1\ \vast\ \text{mittajana}.
\]
Kun tarkastellaan kiinteää suoraa, jolta valitaan referenssipiste $O$ luvun $0$ vastineeksi, 
niin kahden (janan pituutena) annetun luvun summa ja erotus voidaan määrätä geometrisesti:
\begin{figure}[htb]
\begin{center}
\epsfig{file=kuvat/kuvaII-2.eps}
\end{center}
%\caption{Kahden janan pituuksien summa ja erotus}
%\label{fig:summajaerotus}
\end{figure}

Kuvassa summalla ja erotuksella on geometriset vastineet janoina ja suoran pisteinä:
\begin{align*}
\qquad &x+y\ \vast\ OQ\ \vast Q,  \\
\qquad &x-y\ \vast\ OR\ \vast R.
\end{align*}
Kun janan loppupiste on $O$:sta vasemmalle (suuntahan oli määritelty!), voidaan $OR$ tulkita 
negatiiviseksi luvuksi $x-y$. Näin on siis lukujärjestelmässä määritelty sekä yhteenlasku että
vähennyslasku. Kun sääntöjä $+,-$ sovelletaan peräkkäin lähtien luvusta 1 (= mittajana
sijoitettuna suoralle) nähdään, että minkä tahansa kokonaisluvun $m\in{\Z}$ geometrinen vastine
saadaan äärellisellä määrällä geometrisia operaatioita. Siis {\Z} on konstruoitu geometrisesti.
Myös millä tahansa kokonaisluvulla jakaminen onnistuu 'viipaloimalla' luku yhdensuuntaisilla 
suorilla, joita voidaan \pain{tasossa} tuottaa harpilla ja viivoittimella (ks.\ kuvio alla).
\begin{figure}[htb]
\begin{center}
\epsfig{file=kuvat/kuvaII-3.eps}
\end{center}
\end{figure}

\pagebreak

Yleisemmin jos lähtökohtana on kaksi mittalukua $x,y$, niin niiden välinen kerto- ja jakolasku
voidaan suoraan geometrisoida yhdenmuotoisten kolmioiden avulla käyttäen suhteita 
\begin{align*}
&\frac{z}{x}\ =\ \frac{y}{1} \quad (z=xy), \\
&\frac{z}{1}\ =\ \frac{x}{y} \quad (z=x/y).
\end{align*}
Kuvassa (alla) on
\begin{alignat*}{3}
OA &\ \vast\ 1,\quad & OB &\ \vast\ y,\quad & BC&\ \vast\ OD\ \vast\ x, \\
OE &\ \vast\ x\cdot y,\quad & AF &\ \vast\ x/y. &&
\end{alignat*} 
\begin{figure}[H]
\setlength{\unitlength}{0.67cm}
\begin{center}
\begin{picture}(12,9)(-3.5,-6)
\put(-2,0){\line(1,0){11}}
\put(-0.1,-0.1){$\bullet$} \put(-0.5,0.5){$O$}
\put(2.9,-0.1){$\bullet$} \put(3,-0.6){$A$}
\put(4.9,-0.1){$\bullet$} \put(4.9,-0.6){$B$}
\put(7.9,1.9){$\bullet$} \put(7.9,2.4){$C$}
\put(4.7,1.1){$\bullet$} \put(4.3,1.4){$F$}
\put(-1.84,-3.26){$\bullet$} \put(-1.5,-3.4){$D$}
\put(-3.02,-5.4){$\bullet$} \put(-2.6,-5.5){$E$}
\put(-3,-4){\line(3,2){11}}
\put(-3.5,-5.67){\line(3,2){12.5}}
\put(-2,-0.5){\line(4,1){11}}
\path(0,0)(-3.5,-6.34)
\put(0,-0.2){$\underbrace{\hspace{2cm}}_1$}
\put(0,0.4){$\overbrace{\hspace{3.33cm}}^y$}
\put(6.8,0.8){$x$}
\put(7.2,1.1){\vector(3,2){1.15}}
\put(6.55,0.7){\vector(-3,-2){1}}
\put(-1.3,-1.6){$x$}
\put(-1.1,-1.3){\line(10,18){0.6}}
\put(-0.5,-0.25){\vector(2,3){0.13}}
\put(-1.3,-1.8){\line(-10,-19){0.6}}
\put(-1.9,-2.92){\vector(-2,-3){0.05}}
\end{picture}
\end{center}
\end{figure}
Jatkossa nimetään luvut, jotka ovat konstruoitavissa (janan pituutena) äärellisellä määrällä
geometrisia operaatioita, \kor{geometrisiksi luvuiksi}\footnote[2]{Englanninkielinen termi on 
'constructible numbers'. Suomennos ei ole vakiintunut.}, ja käytetään tälle lukujoukolle 
symbolia $\G$\,:
\begin{align*}
\G \ = \ \text{\{geometriset luvut\}}. 
\end{align*}
Tiedetään jo, että $\G$ sisältää kaikki rationaaliluvut: $\G\supset\Q$. Lisäksi tiedetään, että
$\G$ on peruslaskutoimitusten suhteen suljettu:
\[
x,y \in \G \qimpl x+y,\ x-y,\ xy,\ x/y \in \G \quad (y\neq0).
\]
Tämä merkitsee, että $(\G,+,\cdot)$ on \pain{kunta}.

Lukujoukon $\G$ muodostamisessa eivät geometrian mahdollisuudet lopu aivan 
peruslaskutoimituksiin, sillä Pythagoraan lauseen mukaan myös operaatiot
\begin{align*}
&x,y\ \map\ \sqrt{x^{2}+y^{2}}, \quad x,y \in \G, \\
&x,y\ \map\ \sqrt{x^{2}-y^{2}}, \quad x,y \in \G,\ x \ge y
\end{align*}
ovat geometrisesti toteutettavissa. Erityisesti on siis vuosituhansia päänvaivaa aiheuttanut
luku $\sqrt{2}$ geometristen lukujen joukossa:
\begin{figure}[H]
\setlength{\unitlength}{1cm}
\begin{center}
\begin{picture}(2.5,2.5)(-0.5,0)
\path(0,2)(0,0)(2,0)(0,2)
\put(0.8,-0.5){$1$} \put(-0.3,0.9){$1$} \put(0.9,1.1){$\sqrt{2}$}
\path(0.15,0)(0.15,0.15)(0,0.15)
\end{picture}
\end{center}
\end{figure}
Itse asiassa neliöjuurioperaatio $\,x\ \map\ \sqrt{x}\,$ on geometrisesti mahdollinen, olipa
mittaluku $x \in \G$ (ajatellaan $x \geq 0$) mikä tahansa:
\begin{figure}[H]
\setlength{\unitlength}{1cm}
\begin{center}
\begin{picture}(7,4.5)(0,-0.5)
\path(3,4)(0,0)(3,0)(3,4)(8.33,0)(3,0)
%\put(3,0){\arc{0.6}{-3.14}{-1.6} \linethickness{0.05cm} \put(-0.12,0.12){\picsquare}}
%\put(3,4){\arc{0.6}{0.8}{2.2} \linethickness{0.1cm} \put(0,-0.15){\picsquare}}
\path(2.85,0)(2.85,0.15)(3,0.15) %\path(2.82,3.76)(3,3.625)(3.18,3.865)
\path(2.91,3.88)(3,3.812)(3.09,3.932)
\put(0,-0.2){$\underbrace{\hspace{3cm}}_{1}$} \put(3,-0.2){$\underbrace{\hspace{5.33cm}}_{x}$}
\put(2.8,1.9){$\left. \begin{array}{c} \vspace{3.4cm} \end{array} \right\} 
                                                                    \scriptstyle{y=\sqrt{x}}$}
\end{picture}
\end{center}
\end{figure}
Tähän geometrian mahdollisuudet kuitenkin loppuvat, joten $\G$ koostuu kaikkiaan luvuista, jotka
on saatu mittaluvusta $1$ lähtien äärellisellä määrällä viittä eri operaatiota: 
$+$, $-$, $\cdot\,$, $/$ ja $\sqrt{ }\,$. Kun $\G$ vielä varustetaan geometrisella 
järjestysrelaatiolla (janojen pituuksien vertailu harpin avulla), niin tuloksena on pelkästään
geometrisiin operaatioihin perustuva järjestetty kunta $(\G,+,\cdot,<)$. Tämä on 
rationaalilukujen kunnan pienin mahdollinen laajennus, joka täyttää ehdon 
$x\in\G\,\ja\,x>0\ \impl\ \sqrt{x}\in\G$ (vrt.\ lukujoukon $\J$ konstruktio Luvussa
\ref{kunta}). On osoitettavissa, että kaikki geometriset luvut ovat algebrallisia (!), joten
pätee
\[ 
\Q\ \subset\ \G\ \subset\ \A\ \subset\ \R.
\]
Kun geometriset luvut sijoitetaan em.\ tavalla kiinteälle suoralle, voidaan näiden lukujen 
'väliin' ajatella sijoitelluksi myös muut reaaliluvut. Näin ajatellen jokainen reaaliluku saa 
geometrisen vastineen
\index{lukusuora}
\kor{lukusuoran} pisteenä.

\begin{figure}[H]
\setlength{\unitlength}{1cm}
\begin{center}
\begin{picture}(14,2)(-2,0)
\put(-1,0){\line(1,0){13}}
\put(0,0){\line(0,1){0.2}}
\put(5,0){\line(0,1){0.2}}
\put(10,0){\line(0,1){0.2}}
\put(20,0){\line(0,1){0.1}}
\put(2.07,0){\line(0,1){0.1}}
\put(8.6,0){\line(0,1){0.1}}
\put(10.7,0){\line(0,1){0.1}}
\put(2.07,1){\vector(0,-1){0.6}} \put(8.6,1){\vector(0,-1){0.6}}
\put(10.7,1){\vector(0,-1){0.6}}
\put(-0.1,-0.5){$1$} \put(4.9,-0.5){$2$} \put(9.9,-0.5){$3$}
\put(1.8,1.5){$\sqrt{2}$} \put(8.5,1.5){$e$} \put(10.6,1.5){$\pi$} 
%\multiput(0,2)(0,10){2}{\line(-1,0){0.1}}
%\multiput(0,6.5)(0,1){2}{\line(-1,0){0.1}}
%\put(-0.6,8){\vector(1,-1){0.5}}
%\put(-0.6,6){\vector(1,1){0.5}}
\end{picture}
\end{center}
\end{figure}

\subsection{Kulman mittaluku}
\index{kulma!b@kulman mittaluku|vahv}%

Kulmien käsittelyssä geometrian mahdollisuudet ovat melko rajalliset. Kulmien yhteen- ja
vähennyslasku onnistuu geometrisesti, kuten todettiin, ja kulman voi helposti myös 
puolittaa eli jakaa kahteen yhtä suureen osaan geometrian keinoin. Sen sijaan jo kulman
jako kolmeen yhtä suureen osaan (äärellisellä) geometrisella konstruktiolla on mahdotonta,
ellei kyse ole erikoisesta (kuten suorasta) kulmasta. Geometrian keinot loppuvat myös, kun
kulmalle yritetään määrittää sen suuruutta kuvaava mittaluku. Geometrisesti pystytään 
konstruoimaan vain lukujono, jonka raja-arvo mittaluku on, eli mittaluku saadaan selville
vain päättymättömällä konstruktiolla. 
\begin{multicols}{2} \raggedcolumns
Tarkastellaan kulmaa $\kulma AOB$, missä pisteet $A$ ja $B$ on sijoitettu $O$-keskisen
\index{yksikköympyrä}%
\kor{yksikköympyrän} kehälle, ts.\ janat $OA$ ja $OB$ ovat valitun mittajanan pituiset
(vastaten lukua $1$ lukusuoralla).
\begin{figure}[H]
\setlength{\unitlength}{1cm}
\begin{center}
\begin{picture}(4.5,3.5)(-0.5,0.3)
\put(2,2){\bigcircle{4}}
\put(2,2){\line(3,1){1.9}}
\put(2,2){\line(1,3){0.63}}
\put(1.9,1.5){$O$} \put(4,2.55){$A$} \put(2.55,4){$B$}
\end{picture}
\end{center}
\end{figure}
\end{multicols}
Kuvion tilanteessa asetetaan kulman $\kulma AOB$ mittaluvuksi yksikköympyrän
\kor{kaarenpituus} pisteiden $A$ ja $B$ välillä. Koska vaihtoehtoja on kaksi, on
päätettävä, mitataanko sisä- vai ulkokulmaa. Seuraavaksi on ratkaistava vakavampi ongelma:
Miten määritellä ympyräkaaren pituus? Seuraavassa ongelman ratkaisussa yhdistellään
geometrian ja algebran keinoja.

Koska kulman $\kulma OAB$ puolitus onnistuu geometrisesti, niin saman tien kulma on
jaettavissa $n=2^k$ yhtä suureen osaan jokaisella $k\in\N$. Olkoon $k\in\N$ annettu
ja merkitään jaossa syntyviä osakulmia $\kulma P_{i-1} O P_i,\ i=1 \ldots n=2^k$,
missä $P_0=A$, $P_n=B$ ja muutkin pisteet $P_i$ ovat samalla $O$-keskisellä
ympyräkaarella. Yhdistetään nyt peräkkäiset pisteet $P_i$ janoilla \kor{murtoviivaksi}
$P_0 \ldots P_n$ ja asetetaan $\,a_k=$ ko.\ murtoviivan pituus, ts.\ $\,a_k=$ janojen 
$P_{i-1}P_i,\ i=1 \ldots n\,$ pituuksien summa. On ilmeistä, että lukujonon $\seq{a_k}$
jokainen termi on määrättävissä geometrisesti lähtien  luvusta $a_0=$ janan $AB$ pituus,
jolloin on $a_k/a_0\in\G\ \forall k$ (mahdollisesti $a_0\not\in\G$). Kuviossa on
konstruoitu murtoviiva $P_0 \ldots P_n$ indekseillä $k=0,1,2$, kun mitattavana on
suora kulma.
\begin{figure}[H]
\setlength{\unitlength}{1cm}
\begin{picture}(6,4.5)(-0.3,-0.5)
\put(0,0){\line(1,0){4}}
\put(0,0){\line(0,1){4}}
\put(0,0){\arc{6}{-1.57}{0}}
\path(3,0)(0,3)
\put(3.1,0.15){$P_0$} \put(0.1,3.15){$P_1$}
\put(1,-0.8){$k=0$}
\put(5,0){\line(1,0){4}}
\put(5,0){\line(0,1){4}}
\put(5,0){\arc{6}{-1.57}{0}}
\path(8,0)(7.121,2.121)(5,3)
\path(5,0)(7.121,2.121)
\put(8.1,0.15){$P_0$} \put(5.1,3.15){$P_2$} \put(7.22,2.27){$P_1$}
\put(6,-0.8){$k=1$}
\put(10,0){\line(1,0){4}}
\put(10,0){\line(0,1){4}}
\put(10,0){\arc{6}{-1.57}{0}}
\path(13,0)(12.772,1.148)(12.121,2.121)(11.148,2.772)(10,3)
\path(10,0)(12.772,1.148) \path(10,0)(12.121,2.121) \path(10,0)(11.148,2.772)
\put(13.1,0.15){$P_0$} \put(10.1,3.15){$P_4$} \put(12.22,2.27){$P_2$}
\put(12.87,1.3){$P_1$} \put(11.25,2.92){$P_3$}
\put(11,-0.8){$k=2$}
%\dashline{0.2}(0.3,3.4)(3.3,1.4)
%\put(1.56,2.8){$P$}
%\path(2.93,0.6)(3.3,0.6)
%\path(2.08,3.1)(2.8,3.1)
\end{picture}
\end{figure}
Pythagoraan lauseeseen vedoten on helppo näyttää, että lukujono $\seq{a_k}$ on aidosti
kasvava (Harj.teht.\,\ref{H-II-1: suoran kulman mitta}a), ja geometrisin perustein on myös
osoitettavissa, että $\seq{a_k}$ on rajoitettu lukujono 
(Harj.teht.\,\ref{H-II-1: suoran kulman mitta}b). Tämän jälkeen on turvattava algebraan: 
Koska $\seq{a_k}$ siis on kasvava ja rajoitettu lukujono, niin on olemassa raja-arvo 
$\lim_k a_k=a\in\R$ (Lause \ref{monotoninen ja rajoitettu jono}, 
Määritelmä \ref{reaaliluvut desimaalilukuina}). Määritellään kaaren $AB$ pituus, eli kulman 
$\kulma AOB$ mittaluku, kyseisenä raja-arvona $a$. 

Mittayksikköä kulman kaarenpituusmittauksessa sanotaan \kor{radiaaniksi}. Tunnetusti
luku $\,\pi=3.14159..\,$ määritellään yksikköympyrän puolikkaan kaarenpituutena, eli
\[
\boxed{\kehys\quad \pi \,=\, \text{oikokulman mittaluku}. \quad}
\]
Koska $\pi$ on transkendenttinen luku (ks.\ Luku \ref{reaalilukujen ominaisuuksia}), niin
myös suoran kulman mittaluku ($=\pi/2$) on transkendenttinen, samoin esimerkiksi
tasasivuisen kolmion kulman mittaluku ($=\pi/3$).

Kulmia on perinteisesti mitattu myös \kor{asteina} ($\aste$), jolloin oikokulman mitaksi
on sovittu $180\aste$. Yleisemmin mittaluvut asteina ja radiaaneina saadaan skaalaamalla
toisistaan:
\[
\boxed{\kehys\quad \text{kulma asteina}\ 
               =\ \frac{180}{\pi}\,\cdot\,\text{(kulma radiaaneina)}. \quad}
\]

\Harj
\begin{enumerate}

\item \label{H-II-1: kulmat}
a) Suoran leikatessa kaksi yhdensuuntaista suoraa syntyy kahdeksan kulmaa. Näytä, että
näistä jokainen on yhtä suuri kolmen muun kanssa. \newline
b) Todista: Kolmion kulmien summa = oikokulma.

\item \label{H-II-1: yhdenmuotoisuus} \index{keskijana (kolmion)}
a) Jana jakaa suorakulmaisen kolmion kahteen osaan siten, että syntyy kolme yhdenmuotoista 
kolmiota. Lähtien tästä ajatuksesta ja kolmioiden yhdenmuotoisuusopista todista Pythagoraan 
lause! \newline
b) Kolmion \kor{keskijana} yhdistää kolmion kärjen vastakkaisen sivun keskipisteeseen.
Lähtien kolmioiden yhdenmuotoisuusopista todista: Kolmion $ABC$ keskijana $AD$ puolittaa
jokaisen janan $BC$ suuntaisen janan, jonka päätepisteet ovat sivuilla $AB$ ja $AC$.

\item \index{kultainen leikkaus}
Esitä geometrinen konstruktio (aseina harppi ja viivoitin), jonka tulos on \newline
a) \ ympyrä, joka kulkee annetun kolmen pisteen kautta, \newline
b) \ annetun kolmion sisään piirretty (suurin mahdollinen) ympyrä, \newline
c) \ suora, joka kulkee annetun pisteen kautta ja sivuaa annettua ympyrää,
d) \ annetun janan $AB$ \kor{kultainen leikkaus} eli janalla oleva piste $C$ siten, että
jos $a,b,c$ ovat janojen $AB$, $AC$ ja $CB$ pituudet, niin $\,a/b=b/c$.

\item
Seuraavista luvuista kaksi ei ole geometrisia. Mitkä kolme ovat\,? \vspace{1mm}\newline
a) \ $\sqrt[1024]{17}$ $\quad$ b) \ $\sqrt[3]{512}$ $\quad$ c) \ $\sqrt[6]{343}$ $\quad$
d) \ $\sqrt[3]{7\sqrt{2}+5}$ $\quad$ e) \ $\sqrt[3]{5\sqrt{2}+7}$

\item (*) \label{H-II-1: suoran kulman mitta}
Suoran kulman mittaluku radiaaneina määritellään raja-arvona $\,a=\lim_k a_k$, missä $4a_k=$ sen 
säännöllisen $2^{k+2}$-kulmion piiri, jonka kärjet ovat yksikköympyrällä. \vspace{1mm}\newline
a) Näytä, että lukujono $\seq{a_k}_{k=0}^\infty$ on aidosti kasvava. \newline
b) Näytä, että $a_k < 2\ \forall k$. \newline
c) Näytä, että $\seq{a_k}$ on laskettavissa palautuvana lukujonona
\[
a_0=\sqrt{2}, \quad 
a_{k+1}=\frac{2a_k}{\sqrt{2+\sqrt{4-(2^{-k}a_k)^2}}}\,, \quad k=0,1,\ldots
\]
Laske $a_k,\ k=1 \ldots 5$ ja vertaa lukuun $\tfrac{\pi}{2}$.

\end{enumerate} % Euklidinen taso
\section{Tason vektorit} \label{tasonvektorit}
\alku
\index{vektoria@vektori (geometrinen)!a@tason \Ekaksi|vahv}%

\kor{Vektori} on matemaattisena käsitteenä hieman kaksijakoinen. Se voi olla lähtökohtaisesti 
geometrinen olio, jolloin se sovelluksissa ajatellaan usein fysikaalisena, 'maailmassa 
vaikuttavana'. Tällaisia 'fysiikkavektoreita' ovat mm.
\begin{itemize}
\item  paikka, nopeus, kiihtyvyys, kulmanopeus
\item  voima, momentti
\item  X-kentän voimakkuus, X=gravitaatio, sähkö, magneetti...
\end{itemize}
Toisaalta vektori voidaan tulkita myös algebrallisesti, jolloin päädytään geometris-fysikaalista
vektoria yleisempään (myös abstraktimpaan) vektorikäsitteeseen. Jäljempänä nähdään ensimmäisiä
esimerkkejä myös tällaisista 'matematiikkavektoreista' (myöhemmin esimerkkejä tulee paljon 
lisää). Jatkossa lähtökohta vektorin käsitteeseen on kuitenkin aluksi geometrinen.

Euklidisessa tasossa \Ekaksi\, vektori määritellään geometrisesti \kor{suuntajanana}, ts. 
janana, jolla on suunta. Ajatus on tällöin, että jana sisältyy sen toisesta päätepisteestä 
lähtevään puolisuoraan. Puolisuoran ilmaisema suunta kuvataan janan päähän merkityllä
nuolenkärjellä:
\begin{figure}[H]
\setlength{\unitlength}{1cm}
\begin{center}
\begin{picture}(13,6)(0,-0.5)
%\Thicklines
\put(1,3){\line(2,1){5}} \put(0,0){\line(4,1){4}}
\put(9,3){\vector(2,1){3}} \put(12,1){\vector(-4,-1){3}}
\put(0.95,3.1){\line(1,-2){0.1}} \put(0.9,2.5){$A$}
\put(3.95,4.6){\line(1,-2){0.1}} \put(3.9,4.0){$B$}
\put(3.975,1.1){\line(1,-4){0.05}} \put(3.925,0.5){$A$}
\put(0.975,0.35){\line(1,-4){0.05}} \put(0.925,-0.25){$B$}
\put(8.9,2.5){$A$} \put(11.9,4){$B$}\put(11.925,0.5){$A$} \put(8.925,-0.25){$B$}
\put(6.5,4){$\hookrightarrow$}
\put(6.5,0){$\hookrightarrow$}
\end{picture}
\end{center}
\end{figure}
Vektorilla tarkoitetaan täsmällisemmin vain suuntajanan sisältävää \pain{osatietoa} 
\[
\{\text{(janan) pituus, suunta}\}.
\] 
Tiettyyn suuntajanaan liittyvä vektori merkitään $\overrightarrow{AB}$, yleisemmin 
käytetään vektorimerkintöjä $\vec a, \vec b, \vec v$ jne.

Huomattakoon, että vektori $\overrightarrow{AB}$ siis \pain{ei} 'tiedä', missä piste
$A$ sijaitsee euklidisessa tasossa. Näin ollen 'vektoria saa siirtää', kunhan 'ei kierrä
eikä venytä'. (Fysikaalinen vastine: Voiman vaikutus on sama vaikutuspisteestä
riippumatta.) 

Jos tunnetaan vektorit $\vec a=\overrightarrow{AB}$ ja $\vec b=\overrightarrow{BC}$,
niin tunnetaan myös $\vec c=\overrightarrow{AC}$\,:
\begin{figure}[H]
\setlength{\unitlength}{1cm}
\begin{center}
\begin{picture}(6,5)
%\Thicklines
\put(0,0){\vector(2,3){2}} \put(2,3){\vector(4,1){4}} \put(0,0){\vector(3,2){6}}
\put(-0.1,-0.1){$\bullet$} \put(1.9,2.9){$\bullet$} \put(5.9,3.9){$\bullet$}
\put(-0.2,0.3){$A$} \put(1.9,3.2){$B$} \put(5.9,4.2){$C$}
\put(0.9,1.8){$\vec a$} \put(4.8,3.8){$\vec b$} \put(4,2.8){$\vec c$}
\end{picture}
\end{center}
\end{figure}
\index{laskuoperaatiot!c@tason vektoreiden|(}%
Sanotaan, että $\vec c$ on vektoreiden $\vec a, \vec b$ \kor{summavektori} ja merkitään
\[
\vec c=\vec a+\vec b.
\]
Summa määrätään siis geometrisesti kolmiodiagrammilla. Näin määritellylle vektorien 
yhteenlaskulle pätevät tavanomaiset vaihdanta- ja liitäntälait:
\begin{itemize}
\item[(V1)] $\quad \vec a+\vec b = \vec b+\vec a$
\item[(V2)] $\quad (\vec a+\vec b\,)+\vec c = \vec a+(\vec b+\vec c\,)$
\end{itemize}
Tässä ei ole kyse aksioomista (vrt.\ vastaavat kunta-aksioomat, Luku \ref{kunta}) vaan
väittämistä, joiden todistus on geometrinen:
\begin{figure}[H]
\setlength{\unitlength}{1cm}
\begin{center}
\begin{picture}(14,5)
%\Thicklines
\put(0,0){\vector(2,3){2}} \put(2,3){\vector(4,1){4}}
\dashline{0.2}(0,0)(4,1) \dashline{0.2}(4,1)(6,4)
\put(3.9,0.975){\vector(4,1){0.1}} \put(5.9,3.85){\vector(2,3){0.1}}
\put(1.2,2.2){$\vec a$} \put(5.2,4){$\vec b$} \put(3.5,1.1){$\vec b$} \put(5.8,3.3){$\vec a$}

\put(6,0){\vector(2,3){2}} \put(8,3){\vector(4,1){4}}
\dashline{0.2}(6,0)(12,4) \put(11.9,3.933){\vector(3,2){0.1}} \put(12,4){\vector(1,-1){2}}
\put(6,0){\vector(4,1){8}}
\dashline{0.2}(8,3)(14,2) \put(13.9,2.0167){\vector(4,-1){0.1}}
\put(7.2,2.2){$\vec a$} \put(11.2,4){$\vec b$} \put(13.6,2.5){$\vec c$} 
\put(12,1.1){$\vec a+\vec b+\vec c$}
\put(12.9,2.3){$\scriptstyle{\vec b+\vec c}$} \put(11.5,3.3){$\scriptstyle{\vec a+\vec b}$}
\end{picture}
\end{center}
\end{figure}
Vektorin $\vec a = \overrightarrow{AB}$
\index{itseisarvo}%
\kor{itseisarvo} (arkisemmin 'pituus') on
\[
\abs{\vec a}=|\overrightarrow{AB}|=\abs{AB}=\text{janan }AB\text{ pituus}.
\]
Tälle pätee \kor{kolmioepäyhtälö} (vrt.\ Lause \ref{kolmioepäyhtälö})
\index{kolmioepäyhtälö!b@tason vektoreiden}%
\[
\abs{\vec a+\vec b} \le \abs{\vec a}+\abs{\vec b}.
\]
Jos $\lambda \in \R$, niin $\lambda \vec a$ tarkoittaa
\index{skaalaus (vektorin)}%
\kor{skaalattua} vektoria, jolle pätee:
\begin{itemize}
\item[(1)] $\quad |\lambda \vec a|=|\lambda||\vec a|$
\item[(2)] $\quad \lambda \vec a \uparrow\uparrow \vec a,
                                 \text{ jos } \lambda > 0, \text{ ja }
                  \lambda \vec a \uparrow\downarrow \vec a, \text{ jos } \lambda<0$
\end{itemize}
Tässä $\uparrow\uparrow$ tarkoittaa yhdensuuntaisuutta ja $\uparrow\downarrow$ 
vastakkaissuuntaisuutta. Vektorin skaalausta $\vec a \map \lambda \vec a \ (\lambda \in \R)$
sanotaan vektorialgebrassa yleisemmin \kor{skalaarilla kertomiseksi}. Skaalaajia eli
reaalilukuja kutsutaan siis tässä laskuoperaatiossa
\index{skalaari}%
\kor{skalaareiksi}. Skalaarilla kertomisen ja yhteenlaskun
määritelmistä on pääteltävissä (Harj.teht.\,\ref{H-II-1: perusteluja}a), että seuraavat
'luonnonmukaiset' säännöt ovat voimassa (vrt.\ kunta-aksioomat Luvussa \ref{kunta})\,:
\begin{itemize}
\item[(V3)] $\quad (\lambda + \mu)\vec a = \lambda \vec a + \mu \vec a$
\item[(V4)] $\quad \lambda(\mu \vec a\,) = (\lambda \mu)\vec a$
\item[(V5)] $\quad \lambda(\vec a + \vec b\,)= \lambda \vec a + \lambda \vec b$
\item[(V6)] $\quad 1 \vec a =  \vec a$
\end{itemize}
Operaation $\vec a \kohti \lambda \vec a$ määritelmän mukaisesti on $|0 \vec a|=0$.
Tällaista vektoria sanotaan
\index{nollavektori}%
\kor{nollavektoriksi} ja merkitään $\,0 \vec a = \vec 0$.
Nollavektorin suunta on epämääräinen  (ei määritelty). Nollavektori on vektorien
yhteenlaskun nolla-alkio, ts.\ pätee
\begin{itemize}
\item[(V7)] $\quad \vec a + \vec 0 = \vec a\ \forall \vec a$ 
\end{itemize}
Jokaisella vektorilla $\vec a$ on myös
\index{vastavektori}%
\kor{vastavektori} $-\vec a$, jolle pätee
\begin{itemize} 
\item[(V8)] $\quad \vec a + (-\vec a\,) = \vec 0$
\end{itemize}
Nimittäin kun säännössä (V3) valitaan $\lambda=1, \mu=-1$ ja käytetään nollavektorin
määritelmää ja sääntöä (V6), niin seuraa
\[
-\vec a = (-1)\,\vec a.
\]
Vastavektorin avulla tulee määritellyksi myös vektorien vähennyslasku:
\[
\vec a - \vec b = \vec a + (-\vec b\,).
\]
\begin{figure}[H]
\setlength{\unitlength}{1cm}
\begin{center}
\begin{picture}(6,2.3)(2,0.5)
%\Thicklines
\put(4,0){\vector(2,3){2}} \put(4,0){\vector(4,1){4}} \put(8,1){\vector(-1,1){2}}
\dashline{0.2}(4,0)(2,2) \put(2.1,1.9){\vector(-1,1){0.1}} 
\dashline{0.2}(6,3)(2,2) \put(2.1,2.025){\vector(-4,-1){0.1}}
\put(5.8,2.3){$\vec a$} \put(7.6,0.4){$\vec b$} \put(6.5,2.7){$\vec a-\vec b$} 
\put(2.7,1.4){$\vec a+(-\vec b\,)$}
\end{picture}
\end{center}
\end{figure}
Säännöt (V1)--(V8), yhdessä nollavektorin ja vastavektorin määritelmien kanssa, muodostavat
(tason vektoreiden)) \kor{vektorialgebran} laskulait.
\index{laskuoperaatiot!c@tason vektoreiden|)}
\begin{Exa} \label{kolmion keskipiste}
Laskulaeista (V1)--(V8) seuraava identiteetti
\[
\vec a + \frac{2}{3} \left( \frac{1}{2}\vec b - \vec a \right)\ 
             =\ \vec b + \frac{2}{3} \left( \frac{1}{2}\vec a - \vec b \right)\
             =\ \frac{2}{3} \left[ \vec b + \frac{1}{2}\left( \vec a - \vec b \right) \right]\
             =\ \frac{1}{3} \left( \vec a + \vec b \right)
\]
voidaan lukea: Kolmion keskijanat leikkaavat toisensa samassa pisteessä, joka jakaa keskijanat
suhteessa $2:1$ (vrt.\ kuvio). Leikkaupistettä sanotaan 
\index{keskizza@keskiö (kolmion)}%
kolmion \kor{keskiöksi} (keskipisteeksi). \loppu
\begin{figure}[H]
\setlength{\unitlength}{1cm}
\begin{center}
\begin{picture}(6.5,4.6)(0,0)
%\Thicklines
\put(0,0){\vector(1,2){2}} \put(0,0){\vector(1,0){6}}
\path(2,4)(6,0)(1,2)
\drawline(2,4)(3,0) \drawline(0,0)(4,2) \put(0,0){\vector(2,1){2.67}}
\put(1.4,3.5){$\vec a$} \put(5.5,-0.5){$\vec b$}
\put(0,-0.5){$A$} \put(1.9,4.2){$B$} \put(6.2,-0.1){$C$}
\put(1.2,1.2){$\scriptstyle{\frac{1}{3}(\vec a+\vec b\,)}$}
\end{picture}
\end{center}
\end{figure}
\end{Exa}

\subsection{Vektoriavaruus}
\index{vektoric@vektoriavaruus|vahv}%

Jos tason vektorien joukkoa merkitään symbolilla $V$, niin edellä olevilla määritelmillä on 
syntynyt algebra, joka merkittäköön $(V,\R)$. Laskusääntöjen (V1)--(V8) (yleisemmin:
aksioomien) ollessa voimassa sanotaan, että kyseessä on (lineaarinen) \kor{vektoriavaruus}.
Käsitteeseen siis sisältyvät
\index{laskuoperaatiot!ca@ yleisen vektoriavaruuden}
\begin{itemize}
\item  vektorien muodostama joukko $V$ (jossa operoidaan)
\item  vektorien yhteenlasku (+)
\item  nk. \kor{skalaarien} muodostama \kor{kertojakunta}, tässä = $\R$
\item  skalaarin ja vektorin kertolasku
\end{itemize}
\index{skalaari} \index{kertojakunta}%
Lyhennetty sanonta '$V$ on vektoriavaruus' tarkoittaa koko tätä algebraa, jolloin 
kertojakunta on juuri $\R$, tai muuten asiayhteydestä selvä. Sanonnat \kor{reaalikertoiminen}
vektoriavaruus tai yleisemmin \kor{$\K$-kertoiminen} vektoriavaruus kiinnittävät kertojakunnan
tarkemmin. Kertojakunnan ei siis tarvitse olla $\R$, vaan se voisi olla esim. $\K=\Q$ tai 
$\K=\G=\{\text{geometriset luvut}\}$. Lineaarinen vektoriavaruus on hyvin keskeinen käsite
matematiikassa, ja sillä on keskeinen rooli myös monissa matematiikan sovelluksissa.

\subsection{Kanta ja koordinaatisto}
\index{kanta|vahv} \index{koordinaatisto|vahv}%

Olkoon $\vec v \in V$ mielivaltainen tason vektori ja $\vec a, \vec b \in V$ kaksi vektoria, 
joille pätee
\[
\vec a, \vec b \neq \vec 0 ,\quad \vec a \neq \lambda \vec b \quad \forall \ \lambda \in \R.
\] 
Tällöin voidaan geometrisella konstruktiolla (ks.\ kuvio) löytää yksikäsitteiset $x,y \in \R$
siten, että
\[
\vec v = x \vec a + y \vec b.
\]
\begin{figure}[H]
\setlength{\unitlength}{1cm}
\begin{center}
\begin{picture}(6,4.5)(0,-0.5)
%\Thicklines
\put(0,0){\vector(1,1){4}} \put(0,0){\vector(1,0){2}} \put(0,0){\vector(3,1){6}}
\put(0,0){\vector(1,0){4}} \put(0,0){\vector(1,1){2}}
\dashline{0.2}(6,2)(2,2) \dashline{0.2}(6,2)(4,0)
\put(3.1,3.6){$\vec b$} \put(0.9,1.6){$y\vec b$} \put(1.5,-0.5){$\vec a$} 
\put(3.3,-0.5){$x\vec a$} \put(5.5,2.2){$\vec v$}
\end{picture}
\end{center}
\end{figure}
Sanotaan tällä perusteella, että $\vec v$ on $\vec a$:n ja $\vec b$:n
\index{lineaariyhdistely}%
\kor{lineaariyhdistely} (lineaarinen yhdistely, \kor{lineaarikombinaatio}) ja että
$x \vec a$ ja $y \vec b$ ovat $\vec v$:n
\index{komponentti (vektorin)}%
(vektori)\kor{komponentit} $\vec a$:n ja $\vec b$:n suuntaan. 

Kun siis kaksi em.\ ehdot täyttävää vektoria on valittu, on koko $V$ esitettävissä muodossa
\[
V=\{x \vec a + y \vec b \mid x,y \in \R\}.
\]
Sanotaan, että vektoripari $\{ \vec a, \vec b \}$ on $V$:n \kor{kanta}\footnote[2]{Kanta on
vektoreiden j\pain{är}j\pain{estett}y joukko, joten merkintä $(\vec a,\vec b)$ olisi
loogisempi. Erinäisten sekaannusten välttämiseksi käytetään tässä yhteydessä kuitenkin
yleisemmin aaltosulkeita.} ja että $x,y$ ovat vektorin $\vec v = x \vec a + y \vec b \in V$
\index{koordinaatti (vektorin)}%
\kor{koordinaatit} kannassa $\{\vec a,\vec b\,\}$. Jos koordinaatit esitetään lukuparina
$(x,y)$ (pari = kahden alkion järjestetty joukko), niin on synnytetty yhteys $V$:n ja
tällaisten lukuparien joukon välille. Merkitään jälkimmäistä joukkoa symbolilla $\Rkaksi$,
äännetään 'R kaksi':
\[
\Rkaksi = \{ (x,y) \mid x \in \R\,\ja\,y \in \R \}.\index{karteesinen tulo|av}%
\footnote[3]{Joukko $\R^2$ voidaan myös  merkitä $\R\times\R$, jolloin käytetään
joukko-opillista \kor{karteesisen tulon} merkintää
\[
A \times B = \{ (x,y) \mid x \in A\,\ja\,y \in B \}.
\] 
Tämä luetaan '$A$:n ja $B$:n karteesinen tulo', tai vain '$A$ risti $B$'. Joukon
$A \times B$ alkiot ovat siis pareja. Näiden välinen samastusrelaatio on
\[
(x_1,y_1) = (x_2,y_2) \qekv  x_1=x_2\ \wedge\ y_1=y_2.
\] }
\]

Em. sopimuksilla on itse asiassa luotu \kor{kääntäen yksikäsitteinen} (=molempiin suuntiin 
yksikäsitteinen)
\index{vastaavuus ($\ensuremath  {\leftrightarrow }$)}%
\kor{vastaavuus} $V$:n ja $\Rkaksi$:n välille. Merkitään tätä: 
\[
V \vast \Rkaksi.
\]
Vastaavuuteen liittyen nimetään $\Rkaksi$ kantaan $\{\vec a, \vec b\,\}$ liittyväksi $V$:n
\index{koordinaattiavaruus}% 
\kor{koordinaatti-avaruudeksi}. Nimitys jo viittaa siihen, että myös $\Rkaksi$ on tulkittavissa
\index{vektorib@vektori (algebrallinen)!a@$\R^2$:n}%
lineaariseksi vektoriavaruudeksi. Nimittäin kaikki vektoriavaruuden laskulait ovat voimassa,
kun yhteenlasku ja skalaarilla kertominen $\Rkaksi$:ssa määritellään
\index{laskuoperaatiot!cb@vektoriavaruuden $(\Rkaksi,\R)$}
\begin{align*}
(x_1,y_1)+(x_2,y_2)& = (x_1+x_2,y_1+y_2), \\
\lambda(x,y) &= (\lambda x, \lambda y).
\end{align*}
Nämä operaatiot myös vastaavat $V$:n laskutoimituksia, sillä
\begin{align*}
\vec v_1=x_1 \vec a + y_1 \vec b, \ \vec v_2= x_2 \vec a + y_2 \vec b \ 
                               &\impl \ \vec v_1 + \vec v_2 = (x_1+x_2)\vec a 
                                                                        + (y_1+y_2)\vec b, \\
\vec v = x \vec a +y \vec b \  &\impl \ \lambda \vec v = (\lambda x)\vec a +(\lambda y)\vec b.
\end{align*}

Vektoriavaruus $(\Rkaksi,\R)$, mainituin laskuoperaatioin, on esimerkki (lajissaan ensimmäinen)
algebrallisten vektoreiden muodostamasta vektoriavaruudesta. Tämä avaruus on
siis geometrisista tulkinnoista vapaa. Toisaalta tulkitsemalla $(\Rkaksi,\R)$ tason vektorien
koordinaattiavaruudeksi (valitussa kannassa) saadaan vektorien laskuoperaatiot muunnetuksi 
algebralliseen muotoon. Esimerkiksi yhteenlaskussa tällainen laskukaavio on
\[
\begin{array}{cccccc}
&\vec v_1   &  &\vec v_2   &     &\vec v_1+\vec v_2 \\
&\downarrow &  &\downarrow &     &\uparrow \\ 
&(x_1,y_1)  &  &(x_2,y_2)  &\map &(x_1+x_2,y_1+y_2)
\end{array}
\]
Kaavion etuna on, että itse laskuoperaatiossa ei tarvita mitään tietoa vektorien 'ulkonäöstä'.
Monia geometrian tuloksia voidaan tällä tavoin johtaa algebran keinoin (vrt.\ Esimerkki 
\ref{kolmion keskipiste} edellä).

Koordinaattiavaruus $\Rkaksi$ voidaan yhtä hyvin liittää myös siihen alkuperäiseen 
pisteavaruuteen $\Ekaksi$, josta koko vektoriajatus oli lähtöisin. Nimittäin jos vektoreita 
ajatellaan $\Ekaksi$:n suuntajanoina, 'joita voi siirrellä', niin voidaan yhtä hyvin ajatella,
että jokainen vektori vastaa yksikäsitteistä suuntajanaa, jonka lähtöpiste on kiinteä piste $O$.
Näin on synnytetty $\Ekaksi$:n ja $V$:n kääntäen yksikäsitteinen vastaavuus:
\[
P \in \Ekaksi \ \leftrightarrow \ \overrightarrow{OP} \in V.
\]
Kun nyt tulkitaan myös edellä valitut kantavektorit $\vec a, \vec b$ pisteestä $O$ alkaviksi
suuntajanoiksi, niin euklidiseen tasoon on luotu \kor{koordinaatisto}\footnote[2]{Ajatuksen 
koordinaatiston avulla tapahtuvasta geometrian aritmetisoimisesta toi matematiikkaan 
ranskalainen filosofi-matemaatikko \hist{Ren\'e Descartes} (1596-1650) tutkielmassaan 
''La g\'eom\'etrie'', joka ilmestyi laajemman filosofisen teoksen liitteenä vuonna 1637. 
Tutkielma merkitsi \kor{analyyttisen geometrian} alkua, ja enteili yleisemminkin geometrian 
'algebralisoitumista' --- trendiä, joka erityisesti tietokoneiden aikakaudella on entisestään
vahvistunut. \index{Descartes, R.|av}}, jota merkitään $\{O,\vec a,\vec b\}$. Pistettä $O$
sanotaan koordinaatiston 
\index{origo}%
\kor{origoksi}.

\begin{figure}[H]
\setlength{\unitlength}{1cm}
\begin{center}
\begin{picture}(6,4)(-2,-0.2)
%\Thicklines
\put(0,0){\vector(3,2){4}} \put(0,0){\vector(-2,3){2}}
\put(-0.1,-0.5){$O$} \put(-1.7,2.8){$\vec a$} \put(3.5,2.6){$\vec b$}
\end{picture}
\end{center}
\end{figure}
\index{koordinaatti (pisteen)}%
Jos $P \in \Ekaksi$ ja $\Vect{OP} = x \vec a + y \vec b$, niin sanotaan, että \kor{pisteen}
$P$ \kor{koordinaatit} valitussa koordinaatistossa $\{O,\vec a,\vec b\,\}$ ovat 
$(x,y)$. Koordinaatiston ollessa kiinitetty voidaan koordinaattiparia $(x,y)$ haluttaessa 
pitää $P$:n 'nimenä', jolloin on lupa kirjoittaa muitta mutkitta
\[
P=(x,y).
\]
Näin on luotu kääntäen yksikäsitteinen vastaavuus $\Ekaksi \ \leftrightarrow \ \Rkaksi$.
\begin{Exa} Olkoon $T$ tason (aito) kolmio, jonka kärkikipisteet ovat $0,A,B$. Merkitään 
$\vec a=\Vect{OA}$, $\vec b=\Vect{OB}$. Tällöin $T$:n kärkipisteet, $T$:n sivujen 
keskipisteet ja $T$:n keskiö koordinaattistossa $\{O,\vec a,\vec b\,\}$ ovat
(vrt.\ Esimerkki \ref{kolmion keskipiste})
\begin{align*}
&\text{kärkipisteet:} \quad O=(0,0), \quad A=(1,0), \quad B=(0,1), \\
&\text{sivujen keskipisteet:} \quad O'=(1/2,1/2), \quad A'=(0,1/2), \quad B'=(1/2,0), \\
&\text{keskiö:} \quad C=(1/3,1/3). \quad\loppu
\end{align*}
\end{Exa}
Koordinaatistoon tarvitaan siis ensinnäkin referenssipiste $O$. Tästä nk. origosta valitaan
kaksi suuntaa, jotka määrääväät vektoreiden $\vec a, \vec b$ suunnat. Vielä päätetään, että 
mitattaessa etäisyyttä $O$:sta on pituusyksikkö $\alpha$ mentäessä $\vec a$:n suuntaan ja 
$\beta$ mentäessä $\vec b$:n suuntaan. Tässä $\alpha,\beta \in \R$ ja $\alpha,\beta>0$. Kun nyt
valitaan $\vec a$ ja $\vec b$ siten, että $|\vec a|=\alpha$ ja $|\vec b|=\beta$, ovat $\vec a$
ja $\vec b$ yksikäsitteisesti määrätyt ja koordinaatisto siis valmis. Origon kautta kulkevia
suoria
\begin{align*}
S_1=\{ P \in \Ekaksi \ | \ \overrightarrow{OP} = \lambda \vec a, \ \lambda \in \R \}, \\
S_2=\{ P \in \Ekaksi \ | \ \overrightarrow{OP} = \mu \vec b, \ \mu \in \R \},
\end{align*}
\index{suuntavektori} \index{koordinaattiakseli}%
joiden \kor{suuntavektoreina} ovat $\vec a, \vec b$, sanotaan \kor{koordinaattiakseleiksi}.
\begin{figure}[H]
\setlength{\unitlength}{1cm}
\begin{center}
\begin{picture}(10,6)(-3,-2)
\Thicklines
\put(0,0){\vector(3,2){4}} \put(0,0){\vector(-1,1){2}}
\thinlines
\put(0.3,-0.1){$O=(0,0)$} \put(-2.1,1.4){$\vec a$} \put(3.5,2.7){$\vec b$}
\put(-3,-2){\line(3,2){8}} \put(2,-2){\line(-1,1){5}}
\put(-1.8,2){$(1,0)$} \put(4,2.2){$(0,1)$}
\put(2.2,-2){$S_1$} \put(-2.5,-2){$S_2$}
\put(-0.07,-0.07){\piste} \put(3.93,2.6){\piste} \put(-2.07,1.93){\piste}
\end{picture}
\end{center}
\end{figure}

\subsection{Lineaarinen riippumattomuus}
\index{lineaarinen riippumattomuus|vahv}

Koordinaatiston kantavektoreista $\vec a,\vec b$ edellä tehdyt oletukset
($\vec a,\vec b \neq \vec 0$ ja $\vec a,\vec b$ eivät saman- tai vastakkaissuuntaiset) 
voidaan asettaa lyhyemmin ehtona
\begin{equation} \label{lin riippumattomat}
x \vec a + y \vec b = \vec 0 \qimpl x=y=0. \tag{$\star$}
\end{equation}
Jos tason vektoreilla on ominaisuus \eqref{lin riippumattomat}, niin sanotaan, että 
$\{\vec a,\vec b\,\}$ on \kor{lineaarisesti} \kor{riippumaton} (vektori)\kor{systeemi}
(= joukko) tai että $\vec a$ ja $\vec b$ ovat lineaarisesti riippumattomat. Tason
vektoriavaruuden $V$ kannaksi kelpaa siis mikä tahansa lineaarisesti riippumaton
vektorisysteemi $\{\vec a,\vec b\,\}$.

Jos vektorit $\vec a,\vec b$ eivät ole lineaarisesti riippumattomat, niin ne ovat
\index{lineaarinen riippuvuus}%
\kor{lineaarisesti riippuvat}. Ehdon  \eqref{lin riippumattomat} mukaisesti näin on, jos
\[
x\vec a + y\vec b=\vec 0 \quad \text{jollakin}\ (x,y) \neq (0,0).
\]
Tämä ehto puolestaan toteutuu täsmälleen kun joko
(i) $\vec a = \vec 0 \,\impl\, x\vec a+0\vec b=\vec 0$ $\forall x\in\R$,
(ii) $\vec b = \vec 0 \,\impl\, 0\vec a+y\vec b=\vec 0\ \forall y\in\R$, tai
(iii) $\vec a$ ja $\vec b$ ovat yhdensuuntaiset (= saman- tai vastakkaissuuntaiset, ol.\
$\vec a \neq \vec 0$ ja $\vec b \neq \vec 0\,$), jolloin 
$\exists\lambda\in\R,\ \lambda \neq 0$ siten, että $\vec a + \lambda\vec b = \vec 0$. 
\begin{Exa} Tason vektoreista $\vec a, \vec b$ tiedetään, että $\vec a-\vec b$ ja 
$\vec a+2\vec b$ ovat lineaarisesti riippuvat. Voidaanko päätellä, että myös $\vec a$ ja 
$\vec b$ ovat lineaarisesti riippuvat\,? \end{Exa}
\ratk Annetun tiedon mukaan on jollakin $(x,y) \neq (0,0)$
\[
x(\vec a-\vec b)+y(\vec a+2\vec b\,)=\vec 0.
\]
Vektorialgebran säännöillä tämä saadaan muotoon
\[
\vec 0 = (x+y)\vec a+(-x+2y)\vec b = x'\vec a+y'\vec b.
\]
Koska
\[
\begin{cases} \,x' = 0 \\ \,y' = 0 \end{cases} \qekv \begin{cases} \,\ x+y = 0 \\ -x+2y = 0 
\end{cases} \qekv \begin{cases} \,x =0 \\ \,y = 0 \end{cases}
\]
ja tiedetään, että $(x,y) \neq (0,0)$, niin päätellään, että $(x',y') \neq (0,0)$. Koska siis
jollakin $(x',y') \neq (0,0)$ on $x'\vec a+y'\vec b=\vec 0$, niin vastaus on: Voidaan\,! \loppu

\subsection{Koordinaatiston vaihto}
\index{koordinaatisto!a@koordinaatiston vaihto|vahv}%

Kun tason geometrisia tehtäviä ratkotaan vektorialgebran keinoin, on tehtävään sopivan 
koordinaatiston valinta yleensä ratkaisemisen ensimmäinen askel (vrt.\ Esimerkki 
\ref{kolmion keskipiste} edellä). Jos valittu koordinaatisto osoittautuu ratkaisun kuluessa
epämukavaksi, voidaan suorittaa \kor{koordinaatiston vaihto}. Koordinaatiston vaihto koostuu
\index{origon siirto} \index{kanta!b@kannan vaihto}%
\kor{origon siirrosta} ja vektoriavaruuden \kor{kannan vaihdosta}, tai pelkästään jommasta
kummasta. Kuvassa pisteen $P$ koordinaatit ovat $(x,y)$ koordinaatistossa $\{O,\vec a,\vec b\,\}$
ja $(x',y')$ koordinaatistossa $\{O',\vec c,\vec d\,\}$.
\begin{figure}[H]
\setlength{\unitlength}{1cm}
\begin{center}
\begin{picture}(10,6)(-3,-2)
\Thicklines
\put(-2,-2){\vector(1,1){3}} \put(-2,-2){\vector(-1,1){2}}
\put(4.5,2){\vector(-1,-4){0.5}} \put(4.5,2){\vector(1,0){3}}
\thinlines
\put(-2,-2){\vector(1,2){2.5}} \put(4.5,2){\vector(-4,1){4}}
\put(-1.9,-2.3){$O$} \put(-4.1,-0.6){$\vec a$} \put(1,0.4){$\vec b$}
\put(7.3,2.2){$\vec c$} \put(4.3,0){$\vec d$} \put(4.5,2.2){$O'$}
\put(0.5,3.2){$P=(x,y)=(x',y')$}
\end{picture}
\end{center}
\end{figure}
Vektorien yhteenlaskudiagrammin mukaan on
\[
\Vect{OP}=\Vect{OO'}+\Vect{O'P}\,\ \ekv\,\ x\vec a+y\vec b=\Vect{OO'}+(x'\vec c+y'\vec d\,).
\]
Olkoon tässä $\Vect{OO'}=\alpha\,\vec a+\beta\,\vec b$, eli $O'=(\alpha,\beta)$
koordinaatistossa $\{O,\vec a,\vec b\,\}$, ja
\[
\vec c=\lambda_1\vec a+\mu_1\vec b, \quad \vec d=\lambda_2\vec a+\mu_2\vec d,
\]
eli vektoreiden $\vec c,\vec d\,$ koordinaatit kannassa $\{\vec a,\vec b\}$ ovat
$(\lambda_1,\mu_1)$ ja $(\lambda_2,\mu_2)$. Tällöin em.\ yhtälö sievenee vektorialgebran
säännöillä muotoon
\[
(x-\lambda_1 x'-\lambda_2 y'-\alpha)\,\vec a+(y-\mu_1 x'-\mu_2 y'-\beta)\,\vec b\,=\,\vec 0.
\]
Koska vektorit $\vec a$ ja $\vec b$ ovat lineaarisesti riippumattomat, niin seuraa
\[
\begin{cases} \,x-\lambda_1 x'-\lambda_2 y'-\alpha=0\\ \,y-\mu_1 x'-\mu_2 y'-\beta =0 \end{cases}
\ \ekv \quad
\begin{cases} \,\lambda_1 x'+\lambda_2 y'=x-\alpha \\ \,\mu_1 x'+\mu_2 y'=y-\beta \end{cases}
\]
Ratkaisemalla tästä $(x',y')$ koordinaattien $(x,y)$ avulla --- yhtälöryhmä ratkeaa 
aina kun $\vec c\,$ ja $\vec d$ ovat lineaarisesti riippumattomat --- saadaan selville 
\index{koordinaattimuunnos}%
\kor{koordinaattimuunnoksen} $(x,y) \ext (x',y')$ laskukaavat. Yhtälöryhmästä nähdään myös 
suoraan, miten käänteinen muunnos $(x',y') \ext (x,y)$ on laskettava. 
\begin{Exa} Olkoon
\[
\Vect{OO'}=\vec a - \vec b, \quad \vec c=2\vec a+\vec b, \quad \vec d=\vec a-2\vec b.
\]
Em.\ laskun kulku on tällöin
\begin{align*}
&x\vec a+y\vec b\ =\ (\vec a-\vec b\,)+x'(2\vec a+\vec b\,)+y'(\vec a-2\vec b\,) \\
&\qekv (x-2x'-y'-1)\,\vec a + (y-x'+2y'+1)\,\vec b = \vec 0 \\
&\qekv \begin{cases} \,2x'+y'=x-1 \\ \,x'-2y'=y+1 \end{cases}
\end{align*}
Ratkaisemalla saadaan koordinaattimuunnoksen laskukaavoiksi
\[
\begin{cases} \,x=2x'+y'+1\\ \,y=x'-2y'-1 \end{cases}\ \ekv \quad
\begin{cases} \,x'=\frac{1}{5}(2x+y-1) \\ \,y'=\frac{1}{5}(x-2y-3) \end{cases} \loppu
\]
\end{Exa}

\subsection{Dimensio. Aliavaruus}
\index{dimensio|vahv} \index{aliavaruus|vahv}

Koska tason vektoriavaruuden kannassa on oltava kaksi lineaarisesti riippumatonta vektoria, niin
sanotaan, että $V$ on $2$-\kor{ulotteinen} vektoriavaruus tai että $V$:n \kor{dimensio}
(ulotteisuus) on $2$. Merkitään
\[
\text{dim } V=2.
\]
Myös koodinaattiavaruus $\Rkaksi$ on vektoriavaruutena $2$-ulotteinen. Nimittäin jos merkitään
\[
\ma=(1,0), \quad \mb=(0,1),
\]
niin $\Rkaksi$:n vektorialgebran mukaan on $(x,y)=x\ma + y\mb$, eli jokainen
$\mv v = (x,y) \in \Rkaksi$ voidaan esittää yksikäsitteisesti $\ma$:n ja $\mb$:n
lineaariyhdistelynä. Siis $\{\ma,\mb\}$ on $\Rkaksi$:n kanta, ja koska kannassa on kaksi
vektoria, niin $\text{dim}(\Rkaksi)=2$.

Tason tai $\Rkaksi$:n vektoreista voidaan muodostaa myös $1$-\kor{ulotteisia} vektoriavaruuksia.
Tason vektoreiden tapauksessa nämä ovat kaikki ilmaistavissa jonkin vektorin
$\vec a \in V, \  \vec a \neq \vec 0$ avulla muodossa
\[
W=\{\vec v = x\vec a,\ x\in\R\}.
\]
Koska ilmeisesti pätee
\begin{align*}
\vec v_1, \vec v_2 \in W &\qimpl \vec v_1 + \vec v_2 \in W, \\
\vec v \in W             &\qimpl \lambda\vec v \in W \ \ \forall\,\lambda \in \R,
\end{align*}
on $W$ itsekin vektoriavaruus. Sen kantaan tarvitaan vain yksi vektori, esim $\vec a$, joten 
$\text{dim } W=1$. Koska myös $W \subset V$, sanotaan, että $W$ on $V$:n (aito) \kor{aliavaruus}
(engl.\ subspace) ja että $\vec a$
\index{virittää (aliavaruus)}%
\kor{virittää} (engl.\ span) $W$:n. Aliavaruuden $W$ geometrinen vastine on origon kautta
kulkeva suora $S\subset\Ekaksi$. Tämä on
\index{pisteavaruus}%
\kor{$1$-ulotteinen pisteavaruus}.
\begin{figure}[H]
\setlength{\unitlength}{1cm}
\begin{center}
\begin{picture}(10,3)
\put(0,0){\line(4,1){10}}
\put(3.9,0.9){$\bullet$} \put(3.9,0.5){$O$}
\Thicklines \put(4,1){\vector(4,1){4}} \thinlines
\put(7.9,1.5){$\vec a$}
\curve(1,0.25,1.3,0.1,1.6,0.1) \put(1.8,0){$S\leftrightarrow W$}
\end{picture}
\end{center}
\end{figure}
Vastaavuus
\[
P \in S \ \leftrightarrow \ \Vect{OP} = x \vec a \in W \ \leftrightarrow \ x \in \R
\]
synnyttää kääntäen yksikäsitteisen vastaavuuden $\R$:n ja pisteavaruuden $S$ välille. Lukujen 
geometrisointi lukusuoran pisteiksi (vrt.\ Luku \ref{geomluvut}) perustui juuri tähän 
vastaavuuteen.
    
\Harj
\begin{enumerate}

\item \label{H-II-1: perusteluja}
a) Perustele vektorialgebran säännöt (V4)--(V6) geometrisesti. \vspace{1mm}\newline
b) Vektoreille pätee kolmioepäyhtälö
$\abs{\vec a+\vec b} \le \abs{\vec a}+\abs{\vec b}$. Päättele geometriaan enempää
turvautumatta, että pätee myös $\,\abs{\vec a}-\abs{\vec b} \le \abs{\vec a+\vec b}$. 
%\vspace{1mm}\newline

\item
Tason vektoreista $\vec a,\vec b$ oletetaan, että $\vec a,\vec b\neq\vec 0$ ja että
$\vec a$ ja $\vec b$ eivät ole yhdensuuntaiset. Millä vektorin $\vec c$ arvoilla voidaan
vektoreita $\vec a+\vec b$, $\vec a+\vec c$ ja $\vec b+2\vec c\,$ 'siirtelemällä' muodostaa
kolmio?

\item
Todista Harjoitustehtävän \ref{geomluvut}:\ref{H-II-1: yhdenmuotoisuus}b väittämä
vektorialgebran avulla.

\item
Nelikulmiossa $ABCD$ on $\Vect{AB} = \vec{a},\ \Vect{AD} = \vec{b}$ ja
$\Vect{BC} = (1/2)(\vec{a} + \vec{b}\,)$. Laske vektorien $\vec{a}$ ja $\vec{b}$
avulla vektori $\Vect{AE}$, missä $E$ on nelikulmion lävistäjien leikkauspiste.

\item
Suunnikkaassa ABCD kärki $A$ yhdistetään sivun $CD$ keskipisteeseen $P$ ja kärki $B$ sivun
$AD$ keskipisteeseen $R$. Yhdysjanat leikatkoot pisteessä $X$. Lausu vektori $\Vect{AX}$
vektoreiden $\vec u=\Vect{AB}$ ja $\vec v=\Vect{AD}$ avulla.

\item
Kolmiossa $ABC$ merkitään $\vec a=\Vect{AB},\ \vec b=\Vect{AC}$. \ a) Päättele geometrisesti,
että vektori $\vec c=\abs{\vec b}\vec a+\abs{\vec a}\vec b$ puolittaa kulman $BAC$. \
b) Piste $D$ on janalla $BC$ ja jana $AD$ puolittaa kulman $BAC$. Todista vektorilaskulla
\kor{kulmanpuolittajalause}: Janojen $BD$ ja $DC$ pituuksien suhde
$=\abs{\vec a}/\abs{\vec b}$.

\item
Olkoot pisteet $M$ ja $N$ kolmioiden $ABC$ ja $DEF$ keskiöt. Näytä, että \newline
$\Vect{AD}+\Vect{BE}+\Vect{CF}=3\Vect{MN}$.

\item
Kolmion $ABC$ sivut $BC$, $CA$ ja $AB$ jakautuvat pisteissä $A^*$, $B^*$ ja $C^*$ suhteessa
$m:n$. Todista, että kolmioiden $ABC$ ja $A^*B^*C^*$ keskiöt yhtyvät.

\item
Olkoon $\{\vec a,\vec b\,\}$ tason vektoriavaruuden kanta ja olkoon $\vec u=2\vec a+3\vec b$,
$\vec v=-3\vec a+2\vec b$ ja $\vec w=-\vec a-2\vec b$. Määritä vektorin $2\vec u-\vec v+\vec w$
koordinaatit kannassa $\{\vec a,\vec b\,\}$. Piirrä kuva!

\item
Vektorit $\vec a,\vec b$ muodostavat tason vektoriavaruuden $V$ kannan. Näytä, että myös 
vektorit $\vec u=\vec a+\vec b$, $\vec v=\vec a-2\vec b$ muodostavat kannan. Laske 
vektorin $\vec w=\vec a-\vec b$ koor\-di\-naa\-tit tässä kannassa.

\item
Kolmiossa $ABC$ piste $O$ puolittaa sivun $AB$, piste $E_1$ jakaa sivun $BC$ suhteessa $1:2$
ja piste $E_2$ sivun $CA$ suhteessa $1:3$. Määritä kolmion kärkipisteiden koordinaatit siinä
koordinaatistossa, jonka origo on piste $O$ ja kantavektorit ovat $\Vect{OE}_1$ ja
$\Vect{OE}_2\,$.

\item 
Jos $\vec a,\vec b$ ovat lineaarisesti riippumattomat tason vektorit, niin millä $t$:n arvoilla
($t\in\R$) vektorit $\vec a-t\vec b$ ja $t\vec a-2\vec b$ ovat myös lineaarisesti
riippumattomat\,?

\item
Pisteen $P$ koordinaatit ovat $(x,y)$ koordinaatistossa $\{O,\vec a,\vec b\,\}$ ja $(x',y')$ 
koordinaatistossa $\{O',\vec c,\vec d\,\}$. Johda koordinaattimuunnoksen laskukaavat molempiin
suuntiin, kun \newline
a) \ $O'=O$, $\,\vec c=\vec a+\vec b\,$ ja $\,\vec d=\vec a-\vec b$. \newline
b) \ $\Vect{O'O}=\vec c+2\vec d$, $\,\vec c=2\vec a+3\vec b\,$ ja $\,\vec d=-\vec a+2\vec b$. 

\item
Olkoon $\vec a=\Vect{OA}$ ja $\vec b=\Vect{OB}$. Johda kordinaattimuunnoksen laskukaavat 
koordinaatistosta $\{O,\vec a,\vec b\,\}$ koordinaatistoon $\{O',\vec c,\vec d\,\}$,
kun $O'=$ kolmion $OAB$ keskiö ja $\vec c=\Vect{O'A},\ \Vect d=\Vect{O'B}$. Mitkä ovat kolmion
kärkipisteiden ja sivujen keskipisteiden koordinaatit jälkimmäisessä koordinaatistossa?

\end{enumerate} % Tason vektorit
\section{Skalaaritulo} \label{skalaaritulo}
\alku
\index{laskuoperaatiot!c@tason vektoreiden|vahv}

Olkoot $\vec a \neq \vec 0$ ja $\vec b \neq \vec 0$ kaksi tason vektoria. Näiden kanssa
samansuuntaiset
\index{yksikkövektori}%
\kor{yksikkövektorit} (yksikön pituiset vektorit) ovat
\[
\vec a^{\,0} = \frac{1}{\abs{\vec a}}\,\vec a=\frac{\vec a}{\left|\vec a\right|}\,, \quad\ 
\vec b^{\,0} = \frac{1}{\abs{\vec b}}\,\vec b=\frac{\vec b}{|\vec b|}\,.
\]
Liitetään jatkossa vektoreihin $\vec a,\vec b$ luku $\kos(\vec a, \vec b) \in \R$, joka 
riippuu vain $\vec a$:n ja $\vec b$:n suunnista, eli vain yksikkövektoreista 
$\vec a^{\,0},\vec b^{\,0}$. Liittäminen tapahtuu geometrisella konstruktiolla seuraavasti:
Olkoon $\vec a^{\,0}=\Vect{OA}$, $\vec b^{\,0} = \Vect{OB}$ ja valitaan pisteiden $O$ ja $A$
kautta kulkevalta suoralta piste $B'$, joka on lähinnä pistettä $B$. Tällöin kulma
$\kulma OB'B$ (tai $\kulma AB'B$, jos $B'=O$) on suora.
\begin{figure}[H]
\setlength{\unitlength}{1cm}
\begin{center}
\begin{picture}(14,3.5)(0,0)
%\Thicklines
\put(0,0){\vector(1,1){2.1}} \put(0,0){\vector(1,0){3}} \dashline{0.2}(2.1,2.1)(2.1,0) 
\put(5,0){\vector(1,0){3}} \put(5,0){\vector(0,1){3}}
\put(11,0){\vector(1,0){3}} \put(11,0){\vector(-1,3){0.95}} \dashline{0.2}(10.05,2.85)(10.05,0) 
\dashline{0.2}(9.5,0)(11,0)
\put(2.8,-0.5){$A$} \put(7.8,-0.5){$A$} \put(13.8,-0.5){$A$}
\put(-0.1,-0.5){$O$} \put(4.9,-0.5){$O=B'$} \put(10.9,-0.5){$O$}
\put(2,-0.5){$B'$} \put(10,-0.5){$B'$}
\put(2.7,0.2){$\vec a^{\,0}$} \put(7.7,0.2){$\vec a^{\,0}$} \put(13.7,0.2){$\vec a^{\,0}$}
\put(1.6,2){$\vec b^{\,0}$} \put(5.2,2.8){$\vec b^{\,0}$} \put(10.25,2.65){$\vec b^{\,0}$}
\put(2.3,2.1){$B$} \put(4.5,2.8){$B$} \put(9.55,2.65){$B$}
\path(2.1,0.15)(1.95,0.15)(1.95,0) \path(5,0.15)(5.15,0.15)(5.15,0) 
\path(10.05,0.15)(10.2,0.15)(10.2,0)
\end{picture}
\end{center}
\end{figure}
Kuvioon liittyen asetetaan:
\[
\kos(\vec a,\vec b) = \kos(\vec a^{\,0}, \vec b^{\,0}) = \left\{ \begin{array}{rl} 
 | \Vect{OB'} |, & \text{jos } \  \Vect{OB'} \ \uparrow \uparrow \ \vec a^{\,0}, \\
              0, & \text{jos } \  \Vect{OB'} = \vec 0, \\
-| \Vect{OB'} |, & \text{jos } \  \Vect{OB'} \ \uparrow \downarrow \ \vec a^{\,0}. 
\end{array} \right.
\]
Luku $\kos(\vec a,\vec b)$ määräytyy siis geometrisesti janan pituutena tai sen vastalukuna,
kun tunnetaan vektorit $\vec a$ ja $\vec b$. Konstruktio on ilmeisen symmetrinen näiden
vektorien suhteen, ts.\ $\kos(\vec a, \vec b)=\kos(\vec b, \vec a)$. Ilmeistä on myös, että
$\kos(\vec a, \vec b)$ ei muutu, jos konstruktioon liittyvää geometrista kuviota
muutetaan euklidisella liikkeellä. Tästä voidaan päätellä, että $\kos(\vec a,\vec b)$
riippuu vain vektoreiden $\vec a,\vec b$ suuntien määräämän \pain{sisäkulman}
\pain{mittaluvusta}. Konstruktiosta nähdään, että sama pätee myös kääntäen: Luku 
$\kos(\vec a,\vec b)$ määrää minitun mittaluvun yksikäsitteisesti.

Jatkossa vektorien $\vec a,\vec b$ suuntien määräämää kulmaa merkitään symbolilla
$\kulma(\vec a,\vec b)$.\footnote[2]{Merkinnässä $\kulma(\vec a,\vec b)$ ei ole tarpeen
tehdä eroa sisä- ja ulkokulman välillä, ts.\ kulmaa ei tarvitse tulkita sektoriksi.
Vrt.\ Luku \ref{geomluvut}.} 
Lukua $\kos(\vec a,\vec b)$ sanotaan tästä lähtien \kor{kulman} $\kulma(\vec a,\vec b)$
\kor{kosiniksi} ja merkitään
\index{kulma!c@kulman kosini}%
\[
\boxed{\kehys\quad \kos(\vec a,\vec b) = \cos\kulma(\vec a,\vec b). \quad}
\]
Koska luku $\cos\kulma(\vec a, \vec b)$ määrää kulmaan $\kulma(\vec a,\vec b)$ liittyen
sisäkulman mittaluvun, niin se määrää myös kulman geometrisen 'ulkonäön'. Erityisesti pätee
\[
\begin{array}{lcll}
\cos\kulma(\vec a, \vec b) = 1 \quad  & \Leftrightarrow \quad & \vec a 
                                  \uparrow \uparrow \vec b \quad   & \text{(nollakulma),} \\
\cos\kulma(\vec a, \vec b) = -1 \quad & \Leftrightarrow \quad & \vec a 
                                  \uparrow \downarrow \vec b \quad & \text{(oikokulma),} \\
\cos\kulma(\vec a, \vec b) = 0 \quad  & \Leftrightarrow \quad & \vec a \perp \vec b \quad 
                                                                   & \text{(suora kulma).}
\end{array}
\]
Tässä on käytetty merkintää $\vec a \perp \vec b$ ilmaisemaan, että vektorit $\vec a$ ja 
\index{ortogonaalisuus!a@vektoreiden} \index{kohtisuoruus (vektoreiden)}%
$\vec b$ ovat \kor{kohtisuorat} eli \kor{ortogonaaliset} ($\kulma(\vec a,\vec b)$ 
= suora kulma). Luvun $\cos\kulma(\vec a,\vec b)$ määritelmän ja Pythagoraan lauseen
perusteella on kaikissa tapauksissa
\begin{equation} \label{skalaari1}
-1 \le \cos\kulma(\vec a, \vec b) \leq 1.
\end{equation}
Määritelmästä nähdään myös, että pätee
\begin{equation} \label{skalaari2}
\cos\kulma(\lambda\vec a,\vec b)=\cos\kulma(\vec a,\lambda\vec b)
                                =\begin{cases}
                                  \ \ \cos\kulma(\vec a,\vec b), \ \ \text{jos}\ \lambda>0, \\
                                     -\cos\kulma(\vec a,\vec b), \ \ \text{jos}\ \lambda<0.
                                 \end{cases}
\end{equation}
\begin{Def} (\vahv{Skalaaritulo}) \label{vektorien skalaaritulo}
\index{skalaaritulo!a@tason vektoreiden|emph} \index{pistetulo|emph}
Vektorien $\vec a, \vec b$ \kor{skalaaritulo} eli \kor{pistetulo} on reaaliluku, joka merkitään
$\vec a \cdot \vec b$, \,luetaan '$a$ piste $b$', ja määritellään
\[
\vec a\cdot\vec b \
                  = \begin{cases} 
                    \,\abs{\vec a} \abs{\vec b}\cos\kulma(\vec a, \vec b), 
                         &\text{jos}\,\ \vec a\neq\vec 0\,\,\ \text{ja}\ \ \vec b\neq\vec 0, \\
                    \,0, &\text{jos}\,\ \vec a = \vec 0\,\ \text{tai}\,\ \vec b = \vec 0.
                    \end{cases}
\]
\end{Def}
Määritelmästä seuraa ensinnäkin
\begin{equation} \label{skalaari3}
\boxed{\kehys\quad \vec a \cdot \vec a = \abs{\vec a}^{2}. \quad}
\end{equation}
Toiseksi on voimassa symmetrialaki
\begin{equation} \label{skalaari4}
\boxed{\kehys\quad \vec a \cdot \vec b = \vec b \cdot \vec a. \quad}
\end{equation}
Kolmanneksi seuraa määritelmästä ja ominaisuudesta \eqref{skalaari2} skalaarilla kertomisen
ja pistetulon välinen osittelulaki
\begin{equation} \label{skalaari5}
\boxed{\kehys\quad (\lambda \vec a) \cdot \vec b 
  = \vec a \cdot (\lambda \vec b) = \lambda ( \vec a \cdot \vec b), \quad \lambda\in\R. \quad}
\end{equation}
Viimeksi mainitun lain perusteella sulkeiden pois jättäminen merkinnässä 
$\lambda \vec a\cdot\vec b$ ei aiheuta sekaannusta. 

Hieman vähemmän ilmeinen skalaaritulon määritelmän seuraamus on pistetulon ja vektorien 
yhteenlaskun välinen osittelulaki
\begin{equation} \label {skalaari6}
\boxed{\kehys\quad \vec a \cdot (\vec b + \vec c\,) 
                        = \vec a \cdot \vec b + \vec a \cdot \vec c. \quad}
\end{equation}
Tämän todistamiseksi tarkastellaan alla olevia kuvioita. \vspace{5mm}\newline
Kuvio 1:
\begin{figure}[H]
\setlength{\unitlength}{1cm}
\begin{center}
\begin{picture}(10,5.2)(-1,-0.5)
%\Thicklines
\put(-1,0){\line(1,0){9.5}} \put(8.7,-0.13){$S$}
\put(0,0){\vector(3,2){6}}
\dashline{0.2}(6,4)(6,0)
\put(0,0){\vector(1,0){3}} \put(0,0){\vector(1,0){6}}
\put(-0.5,-0.5){$O$} \put(6.2,-0.5){$B'$} \put(6.2,3.9){$B$}
\put(2.7,0.2){$\vec a$} \put(5.5,3.9){$\vec b$}
\dashline{0.2}(2,1.33)(2,0)
\multiput(2,0)(4,0){2}{\path(0,0.15)(0.15,0.15)(0.15,0)}
\put(0,-0.1){$\underbrace{\hspace{2cm}}_{\displaystyle{\cos\kulma(\vec a,\vec b)}}$}
\put(0.8,0.9){$1$}
\put(1,1.1){\vector(3,2){0.9}}
\put(0.7,0.9){\vector(-3,-2){0.8}}
\end{picture}
\end{center}
\end{figure}
Kuvio 2: \vspace{2mm}\newline
\begin{figure}[H]
\setlength{\unitlength}{1cm}
\begin{center}
\begin{picture}(14,6)(-0.1,-1.7)
\path(0,0)(5,0) \path(6,0)(13,0)
\put(0,0){\vector(4,1){4}} \put(0,0){\vector(3,4){3}} \put(3,4){\vector(1,-3){1}} 
\put(0,0){\vector(1,0){2}}
\put(-0.1,-0.5){$O$} \put(2.9,-0.5){$B'$} \put(3.9,-0.5){$C'$} \put(5.1,-0.1){$S$}
\dashline{0.2}(3,0)(3,4) \dashline{0.2}(4,0)(4,1)
\put(1.7,-0.5){$\vec a$} \put(2.5,3.7){$\vec b$} \put(4,1.2){$\vec c$} 
\put(3.05,0.35){$\vec b+\vec c$}
\put(2.9,4.2){$B$} \put(4.1,0.9){$C$}
\put(6,0){\vector(1,1){4}} \put(6,0){\vector(3,1){6}} \put(12,2){\vector(-1,1){2}}
\dashline{0.2}(10,0)(10,4) \dashline{0.2}(12,0)(12,2)
\put(5.9,-0.5){$O$} \put(9.9,-0.5){$C'$} \put(11.9,-0.5){$B'$} \put(13.1,-0.1){$S$}
\put(7.7,-0.5){$\vec a$} \put(8.8,3.7){$\vec b+\vec c$} \put(10.5,3.7){$\vec c$} 
\put(11.4,1.9){$\vec b$}
\put(12.1,1.9){$B$} \put(9.9,4.1){$C$}
\put(0.5,-2){$\displaystyle{
\begin{array}{ll}
\abs{\vec a}\abs{\Vect{OC'}} = \abs{\vec a}\abs{\Vect{OB'}} + \abs{\vec a}\abs{\Vect{B'C'}} 
\qquad & \qquad \abs{\vec a}\abs{\Vect{OC'}} = \abs{\vec a}\abs{\Vect{OB'}} 
                                             - \abs{\vec a}\abs{\Vect{B'C'}} \\
         \impl \ \vec a \cdot (\vec b + \vec c\,) = \vec a \cdot \vec b + \vec a \cdot \vec c 
\qquad & \qquad \impl \ \vec a \cdot (\vec b + \vec c\,) 
         = \vec a \cdot \vec b + \vec a \cdot \vec c \hspace{1cm}
\end{array}}$}
\end{picture}
\end{center}
\end{figure}

Kuviossa 1 on $\abs{\Vect{OB'}} = \abs{\cos\kulma(\vec a, \vec b)} \abs{\vec b}$
kolmioiden yhdenmuotoisuuden perusteella, jolloin skalaaritulon määritelmästä seuraa
\[
\vec a \cdot \vec b = \left\{ \begin{array}{rcl}
\abs{\vec a}\abs{\Vect{OB'}},  & \text{jos} & \Vect{OB'} \ \uparrow\uparrow \ \vec a, \\ 
-\abs{\vec a}\abs{\Vect{OB'}}, & \text{jos} & \Vect{OB'} \ \uparrow\downarrow \ \vec a.
\end{array} \right.
\]
Perustuen tähän geometriseen tulkintaan on Kuviossa 2 johdettu osittelulaki \eqref{skalaari6}
tapauksessa $\cos\kulma(\vec a, \vec b)>0$ ja $\cos\kulma(\vec a,\vec b+\vec c)>0$. Muissakin
tapauksissa ($\cos\kulma(\vec a, \vec b) \le 0$ ja/tai $\cos\kulma(\vec a,\vec b+\vec c) \le 0$)
voidaan päättellä vastaavalla tavalla, että osittelulaki \eqref{skalaari6} on voimassa.

Skalaaritulon ominaisuuksia \eqref{skalaari4}, \eqref{skalaari5}, \eqref{skalaari6} 
yhdistelemällä saadaan nk.\
\index{bilineaarisuus}%
\kor{bilineaarisuusominaisuudet}
\begin{equation} \label{skalaari7}
\boxed{\begin{array}{ll}
\ykehys \quad (\lambda \vec a + \mu \vec b) \cdot \vec c 
                    &= \ \lambda\,\vec a \cdot \vec c + \mu\,\vec b \cdot \vec c, \\ 
\akehys \quad\vec a \cdot (\lambda \vec b + \mu \vec c\,)  
                    &= \ \lambda\,\vec a \cdot \vec b + \mu\,\vec a \cdot \vec c,
                                                    \quad \lambda,\mu \in \R. \quad
\end{array}}
\end{equation} 

Skalaaritulo, joka edellä on määritelty geometristen ideoiden pohjalta, on itse asiassa hyvin 
yleinen vektoriavaruuksiin liittyvä käsite. Yleisemmissä yhteyksissä ei skalaaritulolle oleteta
\index{symmetrisyys!b@skalaaritulon}%
lähtökohtaisesti muita ominaisuuksia kuin (i) \kor{symmetrisyys}, eli symmetrialakia 
\eqref{skalaari4} vastaava ominaisuus, (ii) \kor{bilineaarisuus}, eli laskusääntöjen 
\eqref{skalaari7} vastineet, ja
\index{positiividefiniittisyys!a@skalaaritulon}%
(iii) \kor{positiividefiniittisyys}, jonka muotoilu tason vektoreille on
\[
\vec a \cdot \vec a \geq 0 \,\ \forall \vec a \quad \text{ja} \quad
             \vec a \cdot \vec a = 0  \,\impl\, \vec a = \vec 0.
\]
Tämä on ilmeinen em.\ geometrisen määritelmän perusteella (ominaisuus \eqref{skalaari3}).
Määritelmästä ja ominaisuudesta \eqref{skalaari1} seuraa myös, että tason vektoreiden 
skalaaritulolle pätee epäyhtälö
\begin{equation} \label{Cauchy-Schwarz}
\boxed{\kehys\quad\abs{\vec a \cdot \vec b} \leq \abs{\vec a\,} \abs{\vec b\,}. \quad}
\end{equation}
Tällä epäyhtälöllä on yleisempiä --- vähemmän ilmeisiä --- algerallisia ulottuvuuksia, joita
tarkastellaan lähemmin seuraavassa luvussa.
\begin{Exa}
Tason vektoreista $\vec a,\vec b$ ja $\vec u$ tiedetään
\[
\abs{\vec a}=1, \quad \abs{\vec b} = 3, \quad \vec a\cdot\vec b = 2, \quad 
\vec a\cdot\vec u = -2, \quad \vec b\cdot\vec u = 1.
\]
Määritä $\vec u$:n koordinaatit kannassa $\{\vec a,\vec b\}$. \end{Exa}
\ratk Jos merkitään $\vec u = x\vec a + y\vec b$, niin kertomalla skalaarisesti $\vec a$:lla ja
$\vec b$:lla ja käyttämällä bilineaarisuussääntöjä \eqref{skalaari7} ja sääntöjä 
\eqref{skalaari3},\eqref{skalaari4} saadaan
\[
x\vec a + y\vec b = \vec u 
                      \qimpl \begin{cases}
                             \,\abs{\vec a}^2 x + (\vec a\cdot\vec b) y = \vec a\cdot\vec u \\
                             \,(\vec a\cdot\vec b) x + \abs{\vec b}^2 y = \vec b\cdot\vec u
                             \end{cases}
\]
Siis
\[
\begin{cases} \,x + 2y = -2 \\ \,2x + 9y = 1 \end{cases} \impl\quad 
\begin{cases} \,x=-4 \\ \,y=1 \end{cases} \loppu
\]
\jatko \begin{Exa} (jatko) Jaa esimerkin vektori $\vec u$ kahteen komponenttiin siten,
että toinen komponentti on $\vec a$:n suuntainen ja toinen on kohtisuora a) $\vec a$:ta, 
b) $\vec b$:ta  vastaan. 
\end{Exa}
\ratk Halutaan laskea $\vec u_1,\vec u_2$ siten, että $\vec u = \vec u_1 + \vec u_2$, 
$\,\vec u_1 = t\vec a,\ t\in\R$ ja
\[
\text{a)}\,\ \vec u_2\cdot\vec a = (\vec u - t\vec a)\cdot\vec a = 0, \qquad 
\text{b)}\,\ \vec u_2\cdot\vec b = (\vec u - t\vec a)\cdot\vec b = 0.
\]
Esimerkin tiedoin saadaan
\[
\text{a)}\,\ t = \frac{\vec a\cdot\vec u}{\abs{\vec a}^2} = -2, \qquad
\text{b)}\,\ t = \frac{\vec b\cdot\vec u}{\vec a\cdot\vec b} = \frac{1}{2}\,.
\]
Siis $\vec u = \vec u_1 + \vec u_2$, missä
\begin{align*}
&\text{a)}\,\ \vec u_1 = -2\vec a,\quad \vec u_2 
                       = \vec u - \vec u_1 = (-4\vec a + \vec b) + 2\vec a 
                       = -2\vec a + \vec b, \\
&\text{b)}\,\ \vec u_1 = \tfrac{1}{2}\vec a,\quad\ \ \vec u_2 
                       = \vec u - \vec u_1 = (-4\vec a + \vec b) - \tfrac{1}{2}\vec a
                       = -\tfrac{9}{2}\vec a + \vec b. \loppu
\end{align*}
Esimerkin a)-kohdassa on kyse perustehtävästä, jossa annettu vektori $\vec u\in V$
halutaan jakaa kahteen komponenttiin $\vec u_1,\,\vec u_2$ siten, että $\vec u_1$ on
annetun vektorin $\vec a$ suuntainen ja $\vec u_2 \perp \vec u_1$. Tällöin sanotaan,
\index{ortogonaaliprojektio}%
että $\vec u_1$ on $\vec u$:n \kor{ortogonaaliprojektio} $\vec a$:n \kor{suuntaan} eli
$\vec a$:n virittämään $V$:n yksiulotteiseen aliavaruuteen
\[
W=\{\vec v = \lambda \vec a \mid \lambda \in \R\}.
\]
Euklidisessa tasossa aliavaruutta $W$ vastaa origon kautta kulkeva suora.
\begin{figure}[H]
\setlength{\unitlength}{1cm}
\begin{center}
\begin{picture}(9,5)
\path(0,0)(8,2)
\put(2,0.5){\vector(4,1){4}}
\put(2,0.5){\vector(1,1){3.4}}
\put(6,1.5){\vector(-1,4){0.6}}
\put(1.9,0.1){$O$}
\put(5.7,0.9){$\vec u_1$} \put(4.9,3.7){$\vec u$} \put(5.6,3.7){$\vec u_2$}
%\put(6,1.5){\arc{0.6}{-1.8}{-0.3}} \put(6.02,1.58){$\scriptscriptstyle{\bullet}$}
\path(6.16,1.54)(6.12,1.7)(5.96,1.66)
\put(7,2.2){$W\leftrightarrow S\subset E^2$} \curve(6.9,2.3,6.7,2,6.5,1.63)
\end{picture}
\end{center}
\end{figure}

\subsection{Ortonormeerattu kanta}
\index{kanta!a@ortonormeerattu|vahv}

Luvussa \ref{tasonvektorit} konstruoitiin avaruudelle $V$ kanta käyttäen kahta lineaarisesti 
riippumatonta vektoria. Skalaaritulon tultua määritellyksi saadaan käytännön laskut 
huomattavasti yksinkertaistumaan valitsemalla kantavektorit niin, että ne ovat ortogonaaliset,
ts.\ kantavektorien välinen skalaaritulo $=0$. Jos vielä kantavektorit
\index{normeeraus (vektorin)}  \index{ortogonaalisuus!b@kannan}% 
\kor{normeerataan} yksikkövektoreiksi, niin sanotaan, että näin saatu kanta on
\kor{ortonormeerattu} (muussa tapauksessa vain \kor{ortogonaalinen}). Ortonormeerattu kanta
$\{\vec i, \vec j\}$ on siis sellainen, että pätee
\[
\left\{ \begin{array}{ll}
\vec i \cdot \vec j = 0 & \text{(ortogonaalisuusehto),} \\
\abs{\vec i} = \abs{\vec j} = 1 & \text{(normeerausehto).}
\end{array} \right.
\]
Nämä ehdot toteuttava $\{\vec i, \vec j\}$ on lineaarisesti riippumaton, sillä
\[
x\vec i+y\vec j = \vec 0 \qimpl \begin{cases} 
                                \,x=\vec i\cdot(x\vec i+y\vec j)=\vec i\cdot\vec 0=0, \\ 
                                \,y=\vec j\cdot(x\vec i+y\vec j)=\vec j\cdot\vec 0=0.
                                \end{cases}
\]
Vastaavaa euklidisen tason koordinaatistoa sanotaan
\index{koordinaatisto!b@karteesinen}%
\kor{karteesiseksi}\footnote[2]{Termi on muotoutunut ranskalaisen \hist{Ren\'e Descartes}in
latinankielisestä nimestä \newline
Cartesius, vrt.\ alaviite edellisessä luvussa.} koordinaa\-tis\-toksi.
\begin{figure}[H]
\setlength{\unitlength}{1cm}
\begin{center}
\begin{picture}(7,4)(-1,-1)
\put(-1,0){\vector(1,0){5}} \put(3.8,-0.5){$x$}
\put(0,-1){\vector(0,1){4}} \put(0.2,2.8){$y$}
\put(-0.5,-0.5){$O$}
\put(0,0){\vector(1,0){1}} \put(0.8,-0.6){$\vec i$}
\put(0,0){\vector(0,1){1}} \put(-0.3,0.7){$\vec j$}
\put(2.9,1.4){$\bullet$}
\dashline{0.2}(0,1.5)(3,1.5) \dashline{0.2}(3,1.5)(3,0)
\put(2.9,-0.4){$x$} \put(-0.4,1.4){$y$}
\put(2.9,1.8){$P=(x,y)$}
\end{picture}
\end{center}
\end{figure}
Ortonormeeratussa kannassa $\{\vec i,\vec j\}$ annettujen vektorien skalaaritulo on helposti
laskettavissa: Jos
\[
\vec v_1 = x_1 \vec i + y_1 \vec j, \quad \vec v_2 = x_2 \vec i + y_2 \vec j,
\]
niin skalaaritulon bilineaarisuuden ja symmetrisyyden perusteella
\begin{align*}
\vec v_1 \cdot \vec v_2 &= (x_1 \vec i + y_1 \vec j) \cdot (x_2 \vec i + y_2 \vec j) \\
                        &= x_1x_2\,\vec i \cdot \vec i + (x_1y_2 + x_2y_1)\,\vec i \cdot \vec j 
                                                       + y_1y_2\,\vec j \cdot \vec j.
\end{align*}
Koska tässä on $\vec i\cdot\vec i=\vec j\cdot\vec j=1$ ja $\vec i\cdot\vec j=0$, niin 
laskukaavaksi tulee
\begin{equation} \label{skalaari9}
\boxed{\kehys\quad \vec v_1\cdot\vec v_2 = x_1x_2 + y_1y_2. \quad}
\end{equation}
Tämän mukaan skalaaritulo määräytyy ortonormeeratussa kannassa laskukaaviolla 
(vrt.\ vektorien yhteenlaskun vastaava kaavio edellisessä luvussa)
\[
\begin{array}{cccccc}
&\vec v_1   &  &\vec v_2   &     & \\
&\downarrow &  &\downarrow &     & \\ 
&(x_1,y_1)  &  &(x_2,y_2)  &\map &(x_1x_2+y_1y_2)\,=\,\vec v_1\cdot\vec v_2
\end{array}
\]
Myös vektorin itseisarvon laskeminen käy ortonormeeratussa kannassa helposti, sillä
laskukaavojen \eqref{skalaari3} ja \eqref{skalaari9} mukaan
\begin{equation} \label{skalaari10}
\boxed{\kehys\quad \abs{\vec v\,}^2= \vec v\cdot\vec v=x^2+y^2, \quad 
                                                \vec v=x\vec i+y\vec j. \quad }
\end{equation}
\begin{Exa} Laske $\cos\kulma(\vec a,\vec b)$, kun $\vec a=4\vec i+3\vec j$ ja
$\vec b=2\vec-3\vec j$.
\end{Exa}
\ratk Määritelmän \ref{vektorien skalaaritulo} ja kaavojen
\eqref{skalaari9}--\eqref{skalaari10} perusteella
\[
\cos\kulma(\vec a,\vec b)=\frac{\vec a\cdot\vec b}{\abs{\vec a\,}\abs{\vec b\,}}
                   =\frac{4\cdot 2+3\cdot(-3)}{\sqrt{4^2+3^2}\sqrt{2^2+3^2}}
                   =-\frac{1}{5\sqrt{13}}\,. \loppu
\]

\Harj
\begin{enumerate}

\item
Laske \newline
a) \ $\abs{4\vec a-5\vec b}$, kun $\abs{\vec a}=1$, $\abs{\vec b}=2$ ja 
     $\vec a\cdot\vec b=-\frac{1}{3}\abs{\vec a}\abs{\vec b}$ \newline
b) \ $\vec a\cdot\vec b$, kun $\abs{\vec a+3\vec b}=16$ ja $\abs{\vec a-3\vec b}=2\sqrt{58}$

\item
a) Kolmiossa $ABC$ ovat sivujen $AB$, $AC$ ja $BC$ pituudet $a$, $b$ ja $c$. Näytä skalaaritulon
avulla, että pätee $\,c^2 = a^2+b^2-2ab\cos\kulma BAC$. \newline
b) Nelikulmiossa $ABCD$ on $\cos\kulma BAD=\gamma$ ja sivujen $AB$, $AD$, $BC$ ja $CD$ pituudet
ovat $a$, $b$, $c$ ja $d$. Laske $\cos\kulma BCD$.

\item
Kun $\vec a$ ja $\vec b$ ovat tason vektoreita ja $|\vec a |=|\vec b |$, saa
skalaaritulo $(\vec a+\vec b)\cdot(\vec a-\vec b)$ yksinkertaisen muodon. Millaisen? Mitä 
tulos tarkoittaa geometrisesti, jos $\,\vec a=\Vect{OA}\,$ ja $\,\vec b=\Vect{OB}$, missä \ 
a) $\,OA\,$ ja $\,OB\,$ ovat suunnikkaan kaksi sivua, \ b) $\,O$ on ympyrän keskipiste ja
$A$, $B$ sen kehän pisteitä? Kuva! 

\item
Suorakulmaisessa kolmiossa $ABC$ on suoran kulman kärjestä lähtevä korkeusjana $AD$. Sivujen
$AB$ ja $AC$ pituudet ovat $5$ ja $12$. Laske skalaaritulot $\Vect{AB}\cdot\Vect{DC}$,
$\Vect{BD}\cdot\Vect{CA}$ ja $\Vect{AC}\cdot\Vect{CD}$.

\item
Tason vektoreista $\vec a$ ja $\vec b$ tiedetään, että $|\vec a|=|\vec b|=1$ ja
$\vec a \cdot \vec b =\frac{1}{7}$. Lisäksi tiedetään vektorista $\vec v$,
että $\vec a \cdot \vec v =3$ ja $\vec b \cdot \vec v =-1$. Määritä $\vec v$:n
koordinaatit kannassa $\{\vec a ,\vec b \}$.

\item
Tason koordinaatistossa $\{O,\vec a,\vec b\}$ ovat pisteen $P$ koordinaatit $(2,1)$ ja pisteen
$O'$ koordinaatit $(-1,3)$. Määritä pisteen $P$ koordinaatit koordinaatistossa
$\{O',\vec a+\vec b,\vec a-2\vec b\}$, kun tiedetään, että $\abs{\vec a}=\abs{\vec b}=1$ ja
$\vec a\cdot\vec b=-\frac{1}{3}\,$.

\item
Laske $\abs{\vec a}$, $\abs{\vec b}$ ja $\cos\kulma(\vec a,\vec b)$, kun \newline
a) \ $\vec a=\vec i-\vec j,\,\ \vec b=\vec i+2\vec j\qquad\qquad\quad$ \newline
b) \ $\vec a=-2\vec i+3\vec j,\,\  \vec b=3\vec i-\vec j$ \newline
c) \ $\vec a=-68\vec i+51\vec j,\,\ \vec b=3\vec i+4\vec j\qquad$ \newline
d) \ $\vec a=76\vec i-57\vec j,\,\ \vec b=92\vec i+69\vec j$ \newline
e) \ $\vec a=(3\sqrt{2}+5)\vec i+(5\sqrt{2}-3)\vec j,\ \ $
     $\vec b=(5\sqrt{2}+3)\vec i+(3\sqrt{2}-5)\vec j$

\item
Tason vektoreista $\vec a,\vec b \in V$ tiedetään, että $\abs{\vec a}=1$, $\abs{\vec b}=3$ ja
$\vec a\cdot\vec b=-2$. Millä kertoimien $\lambda,\mu$ arvoilla 
$\{\vec a,\lambda\vec a+\mu\vec b\}$ on $V$:n ortonormeerattu kanta\,?

\item (*)
Näytä skalaarituloa käyttäen, että kolmion korkeusjanat ovat kolmella saman pisteen kautta
kulkevalla suoralla, ts.\ korkeusjanat tai niiden jatkeet leikkaavat samassa pisteessä.

\item (*)
Joukko $S\subset\Ekaksi$ koostuu pisteistä $P$, joiden karteesiset koordinaatit toteuttavat
ehdon
\[
(x-1)^2-(y+2)^2=1.
\]
Olkoot $(\xi,\eta)$ pisteen $P$ koordinaatit toisessa ortonormeeratussa koordinaatistossa,
jonka origo on pisteessä $(x,y)=(1,-2)$ ja kantavektorit
\[
\vec e_1=\frac{1}{\sqrt{2}}(\vec i+\vec j),\ \ \vec e_2=\frac{1}{\sqrt{2}}(\vec i-\vec j).
\]
Jos pisteen $P=(x,y)$ koordinaatit tässä koordinaatistossa ovat $(x',y')$, niin millä
koordinaateille $x',y'$ asetettavalla ehdolla on $P=(x',y') \in S\,$? Piirrä kuva, jossa
näkyvät molemmat koordinaatistot, ja hahmottele kuvaan joukko $S$.

\end{enumerate}

 % Skalaaritulo
\section{*Abstrakti skalaaritulo ja normi} \label{abstrakti skalaaritulo}
\sectionmark{*Abstrakti skalaaritulo}
\alku

Oletetaan lähtökohdaksi annettu reaalikertoiminen vektoriavaruus $(U,\R)$ ja siinä määritelty
skalaaritulo eli \kor{sisätulo}, jota merkitään symbolilla $\scp{\cdot}{\cdot}$. $U$:n alkioita
(vektoreita) merkitään $\mpu, \mpv$, jne. --- Mitään hypoteeseja näiden 'ulkonäöstä' ei tehdä.
Skalaaritulosta oletetaan, että $\scp{\mpu}{\mpv} \in \R$ on määritelty kaikille pareille 
$(\mpu, \mpv),\ \mpu,\mpv \in U$ (eli $(\mpu,\mpv) \in U \times U$, vrt.\ alaviite
Luvussa \ref{tasonvektorit}) ja että seuraavat lainalaisuudet ovat voimassa:
\index{skalaaritulo!d@abstrakti}
\begin{enumerate}
\item \kor{Symmetrisyys} \index{symmetrisyys!b@skalaaritulon}
\begin{itemize}
\item[] $\scp{\mpu}{\mpv}=\scp{\mpv}{\mpu} \quad \forall\,\mpu,\mpv \in U.$
\end{itemize}
\item \kor{Bilineaarisuus} \index{bilineaarisuus}
\begin{itemize}
\item[(a)] $\scp{\alpha \mpu + \beta \mpv}{\mw} 
                             = \alpha \scp{\mpu}{\mw} + \beta \scp{\mpv}{\mw}$,
\item[(b)] $\scp{\mpu}{\alpha \mpv + \beta \mw} 
                             = \alpha \scp{\mpu}{\mpv} + \beta \scp{\mpu}{\mw}$
\item[]    $\qquad\qquad\forall\,\mpu,\mpv,\mw \in U,\ \alpha,\beta\in\R.$
\end{itemize}
\item \kor{Positiividefiniittisyys} \index{positiividefiniittisyys!a@skalaaritulon}
\begin{itemize}
\item[(a)] $\scp{\mpu}{\mpu} \ge 0 \quad \forall\,\mpu \in U,$
\item[(b)] $\scp{\mpu}{\mpu} = 0 \ \ekv \ \mpu = \mathbf{0}.$
\end{itemize}
\end{enumerate}
Tässä $U$:n nollavektoria (vektorien yhteenlaskun nolla-alkiota) on merkitty $\mathbf{0}$:lla.

Oletukset 1--3 ovat yleiset, reaalikertoimisen vektoriavaruuden 
\kor{skalaaritulon aksioomat}.\footnote[2]{Bilineaarisuusvaatimus (b) on aksioomana tarpeeton,
koska se seuraa ehdosta (a) sekä symmetriasta. Positiividefiniittisyysaksiooman osa (b) riittää
myös asettaa muodossa '$\impl$', sillä '$\Leftarrow$' seuraa bilineaarisuudesta.} Jos 
$\scp{\cdot}{\cdot}$ on aksioomat 1--3 täyttävä $U$:n skalaaritulo, niin sanotaan, että ko.\ 
skalaarituloon liittyvä vektorin $\mpu \in U$ \kor{normi} on 
\[ 
\abs{\mpu} = \scp{\mpu}{\mpu}^{1/2}.
\]
\begin{Exa} Olkoon $U=(\Rkaksi,\R)$ tason vektoriavaruuden $(V,\R)$ ortonormeerattua kantaa
$\{\vec i,\vec j\}$ vastaava koordinaattiavaruus. Tällöin jos $\mpu=(x_1\,,y_1) \in U$ ja 
$\mpv=(x_2\,,y_2) \in U$, ja vastaavat tason vektorit ovat $\vec u=x_1\vec i+y_1\vec j$ ja
$\vec v=x_2\vec i+y_2\vec j$, niin edellisen luvun päätelmien perusteella aksioomat 1--3 ovat 
voimassa, kun $U$:n skalaaritulo määritellään
\[
\scp{\mpu}{\mpv}=\vec u\cdot\vec v=x_1 x_2 + y_1 y_2. \loppu
\]
\end{Exa}
Asetetaan esimerkkiin viitaten
\begin{Def} \label{R2:n euklidinen skalaaritulo ja normi}
\index{euklidinen!c@skalaaritulo|emph} \index{euklidinen!b@normi|emph}
\index{skalaaritulo!c@$\R^n$:n euklidinen|emph} \index{normi!euklidinen|emph}
Jos $\mpu=(x_1\,,y_1)\in\Rkaksi$ ja $\mpv=(x_2\,,y_2)\in\Rkaksi$, niin sanotaan, että
skalaaritulo
\[
\scp{\mpu}{\mpv}=x_1 x_2 + y_1 y_2
\]
on $\Rkaksi$:n (vektoriavaruuden $(\Rkaksi,\R)$) \kor{euklidinen skalaaritulo}. Normi
\[
\abs{\mpu} = \scp{\mpu}{\mpu}^{1/2} = (x^2+y^2)^{1/2}, \quad \mpu=(x,y)\in\Rkaksi
\]
on $\Rkaksi$:n \kor{euklidinen normi}.
\end{Def}
\jatko \begin{Exa} (jatko) Tason vektoreiden skalaaritulolle pätevä epäyhtälö
$\abs{\vec u\cdot\vec v\,} \le \abs{\vec u\,}\abs{\vec v\,}$ voidaan kirjoittaa $\Rkaksi$:n
euklidisen skalaaritulon ja normin avulla muodossa
\[
\abs{\scp{\mpu}{\mpv}}\le\abs{\mpu}\abs{\mpv} \quad \forall\,\mpu,\mpv\in\Rkaksi. \loppu
\]
\end{Exa}
Esimerkin epäyhtälö tarkoittaa auki kirjoitettunaa väittämää
\[
\abs{x_1 x_2 + y_1 y_2} \le \sqrt{x_1^2+y_1^2}\,\sqrt{x_2^2+y_2^2}\ \quad 
                                           \forall\,x_1,\,x_2,\,y_1,\,y_2\in\R.
\]
Tämä on siis tosi geometrisin perustein --- mutta väittämänä tämä ei näytä lainkaan geometriasta
riippuvalta. Seuraava lause osoittaa, että kyseessä ei olekaan geometrinen vaan 
\pain{al}g\pain{ebrallinen} väittämä: Epäyhtälö on erikoistapaus yleisemmästä, kaikille 
skalaarituloille pätevästä \kor {Cauchyn} tai \kor{Cauchyn--Schwarzin epäyhtälöstä}. Kyseessä on
kolmioepäyhtälön ohella matemaattisen analyysin yleisimmin käytetty epäyhtälö.
\begin{Lause} \vahv{(Cauchy--Schwarz)} \label{schwarzR}
\index{Cauchyn!f@--Schwarzin epäyhtälö|emph} Jokaiselle aksioomat 1--3 toteuttavalle,
reaalikertoimisen vektoriavaruuden $U$ skalaaritulolle pätee epäyhtälö
\[
\abs{\scp{\mpu}{\mpv}} \le \abs{\mpu} \abs{\mpv}, \quad 
\]                            
missä $\abs{\mpu}=\sqrt{\scp{\mpu}{\mpu}}$.
\end{Lause}
\tod Jos $\mpu=\mathbf{0}$, niin $\scp{\mpu}{\mpv}=\scp{\mpu}{\mpu}=0$ bilineaarisuusehtojen
nojalla, joten tässä tapauksessa väittämä on tosi muodossa $0 \leq 0$. Oletetaan siis, että 
$\mpu \neq \mathbf{0}$, jolloin aksiooman 3 mukaan on $\scp{\mpu}{\mpu} > 0$. Tällöin saman 
aksiooman mukaan 
\[
\scp{\beta \mpu + \mpv}{\beta \mpu + \mpv} \geq 0 \quad \forall \beta \in \R.
\]
Aksioomien 1--2 perusteella tämä voidaan kirjoittaa yhtäpitävästi muotoon
\[
\beta^2\scp{\mpu}{\mpu} + 2 \beta \scp{\mpu}{\mpv} 
                        + \scp{\mpv}{\mpv} \geq 0 \quad \forall \beta \in \R.
\]
Valitsemalla tässä
\[
\beta=-\frac{\scp{\mpu}{\mpv}}{\scp{\mpu}{\mpu}}
\]
seuraa
\begin{align*}
-\frac{\scp{\mpu}{\mpv}}{\scp{\mpu}{\mpu}}^2+\scp{\mpv}{\mpv} \geq 0
            &\qekv \scp{\mpu}{\mpv}^2 \ \leq \ \scp{\mpu}{\mpu}\scp{\mpv}{\mpv} \\
            &\qekv \abs{\scp{\mpu}{\mpv}} \ \leq \ \abs{\mpu} \abs{\mpv}. \loppu
\end{align*}

\subsection{Normi}
\index{normi|vahv}

Edellä kutsuttiin jo vektorin $\mpu$ \pain{normiksi} skalaaritulon avulla määriteltyä
reaalilukua $\abs{\mpu}=\scp{\mpu}{\mpu}^{1/2}$. Yleisemmin \kor{vekoriavaruuden normi} 
ei edellytä skalaaritulon määrittelyä, vaan kyse on yleisemmästä tavasta määrätä vektorille
'pituus'. Yleisempää vektorin $\mpu$ normia merkitään symbolilla $\norm{\mpu}$. Vektorin
'normittamisessa' on kyse laskuoperaatiosta (tai mittausoperaatiosta) 
$\mpu \in U \map \norm{\mpu}\in\R$, jonka on täytettävä seuraavat \kor{normin aksioomat}:
\begin{enumerate}
\item \kor{Positiividefiniittisyys} \index{positiividefiniittisyys!b@normin}
\begin{itemize}
\item[(a)] $\norm{\mpu} \ge 0 \quad \forall\,\mpu \in U,$ 
\item[(b)] $\norm{\mpu}=0\,\impl\,\mpu=\mv{0}.$
\end{itemize}
\item \kor{Skaalautuvuus} \index{skaalautuvuus (normin)}
\begin{itemize} 
\item[]$\norm{\alpha\mpu}=\abs{\alpha}\norm{\mpu} \quad \forall\,\mpu\,\in U,\ \alpha\in\R.$
\end{itemize}
\item \kor{Kolmioepäyhtälö} \index{kolmioepäyhtälö!c@normiavaruuden}
\begin{itemize}
\item[] $\norm{\mpu+\mpv} \le \norm{\mpu}+\norm{\mpv} \quad \forall\,\mpu,\mpv \in U.$
\end{itemize}
\end{enumerate}
Valitsemalla $\alpha=0$ aksioomassa 2 nähdään, että pätee $\mpu=\mv{0}\,\impl\,\norm{\mpu}=0$.
Aksiooman 1b perusteella tämä imlikaatio pätee myös kääntäen, joten normille on voimassa
\[
\norm{\mpu}=0\,\ekv\,\mpu=\mv{0}.
\]
\begin{Exa}
Skalaaritulon avulla määritellylle normille $\norm{\mpu}=\abs{\mpu}=\scp{\mpu}{\mpu}^{1/2}$
aksioomien 1--2 voimassaolo on ilmeistä skalaaritulon positiividefiniittisyyden ja 
bilineaarisuuden perusteella (skalaaritulon aksioomat 2 ja 3). Myös kolmioepäyhtälö on voimassa 
(Harj.teht.\,\ref{H-II-4: kolmioey}a), joten kyseessä on todella normi. \loppu
\end{Exa}
\begin{Exa} \label{R2:n max-normi} \index{maksiminormia@maksiminormi ($\R^2$:n)}
\index{normi!xa@maksiminormi ($\R^2$:n)}
Avaruuden $\Rkaksi$ nk.\ \kor{maksiminormi} määritellään
\[
\norm{\mpu}=\norm{(x,y)}=\max\{\abs{x},\abs{y}\}.
\]
Aksioomien 1--2 toteutuminen tälle normille on helposti todettavissa. Aksiooman 3 toteen
näyttämiseksi olkoon $\mpu=(x_1\,,y_1)\in\Rkaksi$ ja $\mpv=(x_2\,,y_2)\in\Rkaksi$, jolloin
\[
\norm{\mpu+\mpv}=\max\{\abs{x_1+x_2},\,\abs{y_1+y_2}\}.
\]
Jos tässä on $\abs{x_1+x_2}\le\abs{y_1+y_2}$, niin reaalilukujen kunnassa pätevän
kolmioepäyhtälön (Lause \ref{kolmioepäyhtälö}) perusteella voidaan päätellä:
\begin{align*}
\norm{\mpu+\mpv}\,=\,\abs{y_1+y_2} 
                 &\le\,\abs{y_1}+\abs{y_2} \\
                 &\le\,\max\{\abs{x_1},\,\abs{y_1}\}+\max\{\abs{x_2},\,\abs{y_2}\}
                \,=\,\norm{\mpu}+\norm{\mpv}.
\end{align*}
Jos $\abs{x_1+x_2}>\abs{y_1+y_2}$, niin päätellään vastaavasti
\begin{align*}
\norm{\mpu+\mpv}\,=\,\abs{x_1+x_2} 
                 &\le\,\abs{x_1}+\abs{x_2} \\
                 &\le\,\max\{\abs{x_1},\,\abs{y_1}\}+\max\{\abs{x_2},\,\abs{y_2}\}
                \,=\,\norm{\mpu}+\norm{\mpv}.
\end{align*}
Siis kolmioepäyhtälö on voimassa. \loppu
\end{Exa}

Reaalikertoimista vektoriavaruutta, jossa on määritelty normi, sanotaan 
(reaalikertoimiseksi)
\index{normiavaruus} \index{siszy@sisätuloavaruus}%
\kor{normiavaruudeksi}. Jos avaruudessa on määritelty myös sisätulo ja normi on siitä johdettu,
sanotaan avaruutta \kor{sisätuloavaruudeksi}.
\begin{Exa} Vektoriavaruus $U=(\Rkaksi,\R)$ varustettuna euklidisella skalaaritulolla
(Määritelmä \ref{R2:n euklidinen skalaaritulo ja normi}) on sisätuloavaruus nimeltä 
\kor{euklidinen avaruus} $\Rkaksi$. Maksiminormilla (Esimerkki \ref{R2:n max-normi}) 
varustettuna $U$ on vain normiavaruus, sillä maksiminormi ei ole johdettavissa mistään
sisätulosta (Harj.teht.\,\ref{H-II-4: kolmioey}b).
\end{Exa}
\begin{Exa} Hyvin yksinkertainen esimerkki sisätuloavaruudesta on yksiulotteinen avaruus 
$(U,\R,\cdot)$, missä $U=\R$. Tässä avaruudessa tulkitaan $\mpu + \mpv$ reaalilukujen 
yhteenlaskuksi ja $\alpha\mpu$ ja $\scp{\mpu}{\mpv}$ molemmat reaalilukujen kertolaskuksi.
Tällöin normi $\abs{\mpu}=\mpu$:n itseisarvo, ja Cauchyn--Schwarzin epäyhtälö toteutuu yhtälönä
$\abs{\mpu\mpv}=\abs{\mpu}\abs{\mpv}$. \loppu 
\end{Exa}

\Harj
\begin{enumerate}

\item
Näytä geometriaan vetoamatta, että $\Rkaksi$:n euklidiselle skalaaritulolle ovat voimassa
skalaaritulon aksioomat.

\item
Tutki, onko kyseessä $\Rkaksi$:n skalaaritulo, kun $\mpu=(x_1,y_1),\ \mpv = (x_2, y_2)$ ja
määritellään \newline
a) \ $\scp{\mpu}{\mpv}=x_1 y_1 + x_2 y_2$ \newline
b) \ $\scp{\mpu}{\mpv}=2x_1 x_2 + 5y_1 y_2$ \newline
c) \ $\scp{\mpu}{\mpv}=x_1 x_2 - y_1 y_2$ \newline
d) \ $\scp{\mpu}{\mpv}=(x_1+y_1)(x_2+y_2)$ \newline
e) \ $\scp{\mpu}{\mpv}=(x_1+y_1)(x_2+y_2)+y_1 y_2$ \newline
Minkä muodon saa (myönteisessä tapauksessa) Cauchyn--Schwarzin epäyhtälö\,?

\item
Olkoon $\{\vec a,\vec b\}$ tason vektoriavaruuden $U=(V,\R)$ kanta. Näytä, että jos vektorin
$\vec v \in V$ koordinaatit ko.\ kannassa ovat $(x,y)$, niin 
$\,\norm{\vec v\,} = \abs{x}+\abs{y}\,$ määrittelee $U$:n normin.

\item
Olkoon $\mpu=(x,y)\in\Rkaksi$. Mitkä seuraavista ovat vektoriavaruuden $U=(\Rkaksi,\R)$
normeja? \newline
a) \ $\norm{\mpu}=\abs{x+y}$ \newline
b) \ $\norm{\mpu}=\abs{x}+\abs{y}$ \newline
b) \ $\norm{\mpu}=2\abs{x}-\abs{y}$ \newline
c) \ $\norm{\mpu}=0.01\abs{x}+0.003\abs{y}$ \newline
d) \ $\norm{\mpu}=x+\abs{y}$ \newline
e) \ $\norm{\mpu}=\abs{x}+y^2$ \newline
f) \ $\norm{\mpu}=\sqrt{2x^2+3y^2}$ \newline
g) \ $\norm{\mpu}=\sqrt{\abs{xy}}$

\item \label{H-II-4: kolmioey} \index{kolmioepäyhtälö!d@sisätuloavaruuden}
\index{suunnikasyhtälö}
a) Todista sisätuloavaruuden normille pätevä kolmioepäyhtälö muodossa
$\norm{\mpu+\mpv}^2 \le (\norm{\mpu}+\norm{\mpv})^2$. \ b) Näytä, että sisätuloavaruuden
normille pätee \kor{suunnikasyhtälö} 
$\norm{\mpu+\mpv}^2+\norm{\mpu-\mpv}^2=2\norm{\mpu}^2+2\norm{\mpv}^2$. Päättele, että
$\Rkaksi$:n maksiminormi $\norm{\mpu}=\norm{(x,y)}=\max\{\abs{x},\abs{y}\}$ ei ole johdettavissa
mistään sisätulosta.

\item
Tutki, millainen geometrinen muodonmuutos tapahtuu $\Rkaksi$:n yksikköympyrässä
$S=\{\mpu=(x,y)\in\Rkaksi \mid \abs{\mpu}=1\}$, kun pisteen etäisyys origosta mitataan 
euklidisen normin sijasta seuraavilla normeilla:
\[
\text{a)}\ \ \norm{\mpu}=\max\{\abs{x},\abs{y}\} \qquad
\text{b)}\ \ \norm{\mpu}=\abs{x}+\abs{y} \qquad
\text{c)}\ \ \norm{\mpu}=\sqrt{4x^2+y^2}
\]

\item (*)
Olkoon $a,b,c,d\in\R$ ja määritellään
\[
\scp{\mpu}{\mpv}\,=\,ax_1x_2+bx_1y_2+cx_2y_1+dy_1y_2\,,
\]
missä $\mpu=(x_1,y_1)\in\Rkaksi$ ja $\mpv=(x_2,y_2)\in\Rkaksi$. Täsmälleen millä luvuille
$a,b,c,d$ asetettavilla ehdoilla $\scp{\cdot}{\cdot}$ on vektoriavaruuden $U=(\Rkaksi,\R)$ 
skalaaritulo?

\end{enumerate} % *Abstrakti skalaaritulo
\section{Funktion käsite. Trigonometriset funktiot} \label{trigonometriset funktiot}
\sectionmark{Trigonometriset funktiot}
\alku

Tässä luvussa esitellään matematiikassa keskeinen käsite \kor{funktio}, sen johdannainen
\kor{reaalifunktio} sekä ensimmäisinä reaalifunktioina \kor{trigonometriset funktiot}.
Osoittautuu, että riittävän yleisesti ja abstraktisti ymmärrettynä  funktioita on esiintynyt
jo aiemmin erinäisillä 'salanimillä'. Reaalifunktioiden osalta ei muita kuin trigonometrisia
funktioita toistaiseksi käsitellä; yleisempi reaalifunktioiden teoria esitetään jäljempänä
Luvuissa IV-VI.

\subsection{Funktion käsite}
\index{funktio A!a@joukko-opillinen|vahv}

\kor{Funktio} l. \kor{kuvaus} (engl. function, map, mapping) ymmärretään matematiikassa
tavallisimmin kolmikkona muotoa
\[
\{\text{joukko} \ A, \ \text{sääntö} \ f, \ \text{joukko} \ B\}.
\]
Tällä tarkoitetaan, että on jokin sääntö $f$, jonka mukaan määräytyy y\pain{ksikäsitteinen}
$y \in B$ j\pain{okaisella} $x \in A$. Yhteys $x$:n $y$:n välillä merkitään $y=f(x)$ ja
lausutaan '$y$ on $f$ $x$'. Voidaan myös käyttää \pain{liittämisnuolta} ja merkitä 
$x \map f(x) (=y)$. Sanotaan, että $A$ on $f$:n
\index{mzyzy@määrittelyjoukko} \index{maalijoukko}%
\kor{määrittelyjoukko} (tai lähtöjoukko,
engl. domain) ja $B$ \kor{maalijoukko} ja merkitään $f:A \kohti B$. Funktion määrittelyjoukon
alkioihin viitataan yleisnimellä \kor{muuttuja} (engl.\ variable). Jos $y=f(x)$, niin sanotaan,
että $y$ on $f$:n \kor{arvo} (engl.\ value) \kor{$x$:ssä} (tai muuttujan arvolla $x$);
lyhyemmin lausutaan '$f$ $x$:ssä' tai usein myös '$f$ pisteessä $x$'. Maalijoukon osajoukko
\[
\{y \in B \ | \ y=f(x) \ \text{jollakin} \ x \in A\}
\]
\index{arvojoukko} \index{kuva (arvojoukko)}%
on nimeltään $f$:n \kor{arvojoukko} tai $A$:n \kor{kuva} $f$:ssä (engl.\ range, image). Tätä 
merkitään lyhyesti $f(A)$. 
\begin{figure}[H]
\begin{center}
\import{kuvat/}{kuvaII-4.pstex_t}
\end{center}
\end{figure}
\begin{Exa}
Jos $x$ ja $y$ ovat reaalilukuja ja kirjoitetaan $f(x,y)=x^2y-y^3$, tarkoitetaan (ellei 
määrittelyjoukosta toisin sovita) funktiota
\[
\{ \Rkaksi, \ f(x,y)=x^2y-y^3, \ \R \},
\]
joka siis liittää mihin tahansa lukupariin $(x,y)\in\R^2$ laskusäännön ilmaiseman reaaliluvun.
Funktion arvojen määräämisessä tarvitaan tässä tapauksessa ainoastaan reaalilukujen 
kerto- ja vähennyslaskuoperaatioita. Esim.
\begin{align*}
&f(3,-2)=3^2 \cdot (-2) - (-2)^3 = -10, \\
&f(2,\sqrt{2})=2^2\cdot\sqrt{2}-(\sqrt{2})^3=0, \\
&f(\pi,e)=\pi^2 e - e^3=6.74282937.. \loppu
\end{align*}
\end{Exa}

\subsection{Funktio joukko-opissa}
\index{funktio A!a@joukko-opillinen|vahv}

Hyvin yleisesti ja abstraktisti ajatellen voidaan funktiota pitää 'säännön' sijasta enemmänkin
'luettelona', joka liittää kuhunkin alkioon $x \in A$ yksikäsitteisen ($x$:stä riippuvan)
alkion $y \in B$. Näin ajatellen funktiosta tulee puhtaasti joukko-opillinen käsite: Funktion
määrittelee joukon $A \times B = \{(x,y) \mid x \in A\, \ja\, y \in B\}$ (= $A$:n ja $B$:n
karteesinen tulo, vrt.\ alaviite Luvussa \ref{tasonvektorit}) mikä tahansa osajoukko
$F \subset A \times B$, joka toteuttaa ehdon
\[
\forall x \in A\ [\,(x,y) \in F\,\ \text{täsmälleen yhdellä}\,\ y \in B\,].
\]
Nimittäin tällaista joukkoa vastaa $A$:ssa määritelty funktio $f$, kun tulkitaan
\[
y=f(x)\ \ekv\ (x,y) \in F.
\]
Funktion käsitteen kannalta ei siis ole lopulta lainkaan merkitystä sillä, onko funktio
tunnettu 'lausekkeena' (laskusääntönä) vai pelkkänä 'datana'.\footnote[2]{Käytännössä
'funktiodata' voi perustua esim.\ suoriin mittauksiin tai se voi olla tietokonemalleihin
perustuva laskettu ennuste, kuten sääennustuksissa.} 
\begin{Exa} \pain{Reaaliluku}j\pain{ono} $\{a_1, a_2, \ldots\}$ on tulkittavissa funktioksi
tyyppiä $f: \ \N\kohti\R$. Nimittäin jono on ajateltavissa 'luettelona' $1 \map a_1$, 
$2 \map a_2$, $\ldots$ eli joukkona
\[
F= \{ (1,a_1), (2,a_2),\ldots \} \subset \N \times \Q. \loppu
\]
\end{Exa}
Kaikki tähän asti esiintyneet \pain{laskuo}p\pain{eraatiot} ovat itse asiassa funktioita. 
Funktion voi tällöin mieltää 'laskukoneeksi', joka antaa laskuoperaation lopputuloksen
annetuilla lähtötiedoilla.
\begin{Exa}
Reaalilukujen yhteenlasku ja kertolasku ovat funktioita tyyppiä $f:\ \Rkaksi \map \R$ 
laskusäännöillä
\[
\text{yhteenlasku:} \quad f(x,y)=x+y, \qquad \text{kertolasku:} \quad f(x,y)=xy.
\]
Jakolasku on funktio $f: \{(x,y) \in \Rkaksi \ | \ y \neq 0 \} \kohti \R$ säännöllä
$f(x,y)=x/y$. \loppu
\end{Exa}
\begin{Exa} \label{skalaaritulo funktiona} Edellisessä luvussa liitettiin euklidisen tason
kulmaan $\kulma(\vec a,\vec b)$ geometrisin keinoin määräytyvä reaaliluku, jota merkittiin 
$\cos\kulma(\vec a,\vec b)$. Koska ko.\ luku on yksikäsitteinen, niin kyseessä on funktio 
$\,\cos:\ A \kohti \R$, missä $A=$ kaikkien kulmien joukko. Funktion arvojoukko on 
$[-1,1]\subset\R$. Määrittelyjoukossa voidaan kulmat haluttaessa tulkita tason vektoripareina
$(\vec a,\vec b)$, missä $\vec a,\vec b \neq \vec 0$, tai yksikkövektorien pareina 
($\abs{\vec a}=\abs{\vec b}=1$).  \loppu
\end{Exa}

\subsection{Reaalifunktio}
\index{funktio A!b@reaalifunktio|vahv}
\index{reaalifunktio|vahv}

'Funktioiden äiti' on \kor{reaalifunktio} eli reaalimuuttujan reaaliarvoinen funktio muotoa
$f: A \kohti \R$, missä $A \subset \R$. Myös termiä 'yhden muuttujan funktio' käytetään usein 
tässä rajatussa merkityksessä. Reaalimuuttujan funktio voidaan kätevästi havainnollistaa
\index{kuvaaja}%
\kor{kuvaajan} (engl.\ graph) avulla. Funktion $f: A \kohti \R$ kuvaaja joukossa
$B \subset A\ (A\subset\R)$ on euklidisen avaruuden $\Ekaksi$ pistejoukko
\[ 
G = \{P = (x,y) \mid y = f(x)\,\ja\, x \in B\},
\]
missä $P$:n koordinaatit $(x,y)$ viittaavat karteesiseen koordinaatistoon (ellei toisin sovita).
Huomattakoon, että jos $f$:n määrittelyjoukko rajataan joukoksi $B$, niin funktion
joukko-opillinen määritelmä on
\[ 
F = \{(x,y)\in\Rkaksi \mid y = f(x)\,\ja\, x \in B\}.
\]
Funktion tulkinta 'käyränä' (eli kuvaajana $G$) siis yksinkertaisesti geometrisoi funktion
abstraktin joukko-opillisen määritelmän.

Kun reaalilukujono tulkitaan funktioksi $f:\ \N\kohti\R$, niin $f$:n kuvaaja on $\Ekaksi$:n
pistejono. Tätä lukujonon kuvaustapaa on edellä jo käytetty Luvuissa \ref{jonon raja-arvo} ja
\ref{Cauchyn jonot}.
\begin{Exa} Alla on kuvattu funktio, jonka laskusääntö on $f(x)=x^2/(x+1)$,
joukoissa $B=[0,4]$ ja $B=[0,4]\cap\N$. Jälkimmäisessä tapauksessa kuvaaja esittää lukujonon
$\seq{a_n}=\{\,n^2/(n+1),\ n=1,2,\ldots\,\}$ alkupäätä. \loppu
\begin{figure}[H]
\setlength{\unitlength}{1cm}
\begin{center}
\begin{picture}(12.5,7.5)(0,-2)
\multiput(0,0)(7,0){2}{
\put(0,0){\vector(1,0){6}} \put(1,-1){\vector(0,1){6}}
\put(1.9,-0.5){$1$} \put(2.9,-0.5){$2$} \put(3.9,-0.5){$3$} \put(4.9,-0.5){$4$} 
\put(5.8,-0.5){$x$} \put(1.2,4.8){$y$}
\multiput(2,0)(1,0){4}{\line(0,-1){0.1}}}
\put(3.6,1.5){$G:\ y=f(x)$}
\put(9,0.5){$\scriptstyle{\bullet}$} \put(10,1.33){$\scriptstyle{\bullet}$}
\put(11,2.25){$\scriptstyle{\bullet}$} \put(12,3.2){$\scriptstyle{\bullet}$}
\put(2.5,-2){$f(x)=\dfrac{x^2}{x+1}$} \put(9.5,-2){$a_n=\dfrac{n^2}{n+1}$}
\put(1,0){
\curve( 
0, 0,
0.5, 0.17,
1, 0.5,
1.5, 0.9,
2, 1.33,
3, 2.25,
4, 3.2)}
\end{picture}
\end{center}
\end{figure}
\end{Exa}
Jos reaalifunktioon liittyy helposti ilmaistavissa oleva laskusääntö, niin funktio
ilmoitetaan tavallisimmin pelkkänä laskusääntönä. Tällöin oletetaan (ellei toisin sovita), että
määrittelyjoukko on suurin $\R$:n osa\-joukko, jossa laskusääntöä voi soveltaa. Yksittäistä
\index{funktioevaluaatio}%
laskusäännön käyttöä, eli laskuoperaatiota $x \map f(x)$ sanotaan \kor{funktioevaluaatioksi}.
Usein tämä onnistuu vain numeerisesti (eli likimäärin),  jolloin käytännössä pystytään vain
laskemaan äärellinen määrä termejä lukujonosta, jonka raja-arvo $=f(x)$.
\begin{Exa} Reaalifunktion $f(x)=x^2/(x^2-2)$ määrittelyjoukko on \newline
$A\ =\ \{x\in\R \mid x^2 \neq 2\}\ 
    =\ (-\infty,-\sqrt{2}) \cup (-\sqrt{2},\sqrt{2}) \cup (\sqrt{2},\infty)$. \loppu
\end{Exa}
\begin{Exa} \pain{Potenssisar}j\pain{a} $\sum_k a_k x^k$ voidaan tulkita reaalifunktioksi, jonka
arvo $x$:ssä on sarjan summa ja määrittelyjoukko on sarjan suppenemisväli. Esimerkiksi funktion
\[
f(x)=\sum_{k=1}^\infty \frac{x^k}{k}
\]
määrittelyjoukko on väli $[-1,1)$, vrt.\ Luku \ref{potenssisarja}. Ellei ole $x=0$, on
funktioevaluaatio $x \map f(x)$ tässä tapauksessa suoritettava numeerisesti, esim.\ 
laskemalla sarjan osasummia. \loppu
\end{Exa}

\pagebreak

\subsection{Funktiot kosini ja sini}
\index{funktio C!a@$\sin$, $\cos$|vahv}
\index{trigonometriset funktiot|vahv}

Tarkastellaan yksikköympyrää, jonka keskipiste on karteesisen koordinaatiston origossa $O$ ja
jolta on valittu referenssipisteeeksi $A=(1,0)$. Pisteestä $A$ lähtien voidaan yksikköympyrää
kiertää joko \kor{positiiviseen kiertosuuntaan} eli vastapäivään tai
\kor{negatiiviseen kiertosuuntaan} eli myötäpäivään. Edellisessä tapauksessa asetetaan
\index{kulma!d@kiertokulma} \index{kiertokulma}%
\kor{kiertokulman} mitaksi kaarenpituus (= kehää pitkin kuljettu matka), jälkimmäisessä sen
vastaluku. Tällä tavoin kiertokulman mitta voi saada minkä tahansa reaaliarvon, ja jokaista
mittalukua $\alpha$ vastaa yksikäsitteinen kehän piste
\[
P(\alpha)=(x(\alpha),y(\alpha)), \quad \alpha \in \R.
\]
Tässä siis $P$ ei tarkoita yksittäistä pistettä, vaan yksikäsitteistä riippuvuutta 
$\alpha \map P(\alpha)$ eli 'pistearvoista' reaalimuuttujan funktiota. Vastaavasti $x$ ja $y$
tarkoittavat edellä funktioita tyyppiä
\[
x: \ \R \kohti [-1,1], \qquad y: \ \R \kohti [-1,1].
\]
\begin{figure}[H]
\begin{center}
\import{kuvat/}{kuvaII-8.pstex_t}
\end{center}
\end{figure}
Kulman kosinin aiemman määritelmän (ks.\ Luku \ref{skalaaritulo}) mukaisesti on
\[
\cos\kulma AOP(\alpha) = x(\alpha).
\]
Määritellään nyt reaalifunktio cos eli \kor{kosini} vaihtamalla tässä
'kulmamuuttujan' tilalle kiertokulman mittaluku $\alpha$\,:
\[
\cos\alpha = x(\alpha), \quad \alpha\in\R.
\]
Määritellään vastaavasti reaalifunktio sin eli \kor{sini} asettamalla
\[
\sin\alpha = y(\alpha), \quad \alpha \in \R. 
\]
Reaalifunktiot kosini ja sini (määrittelyjoukkona $\R$) ovat \kor{trigonometristen funktioiden}
perustyypit. Yhdenmuotoisten kolmioiden periaatteella nähdään, että määritelmät vastaavat 
tuttuja trigonometrian merkintöjä, kun $0 < \alpha < \pi/2$.
\begin{figure}[H]
\begin{center}
\import{kuvat/}{kuvaII-9.pstex_t}
\end{center}
\end{figure}
Muutamia määritelmistä suoraan seuraavia, tai symmetrian avulla helposti perusteltavia, kosinin
ja sinin ominaisuuksia on lueteltu alla. Nämä ovat voimassa $\forall \alpha \in \R$.
\begin{align}
&\cos(\alpha + 2 \pi) = \cos\alpha, \ \ \sin(\alpha + 2 \pi) = \sin\alpha. \label{trig1} \\
&\cos(-\alpha) = \cos\alpha, \ \ \sin(-\alpha) = -\sin\alpha. \label{trig2} \\
&\cos(\alpha + \pi) = -\cos\alpha, \ \ \sin(\alpha + \pi) = -\sin\alpha. \label{trig3} \\ 
&\sin\alpha = \cos(\pi/2 - \alpha), \ \ \cos\alpha = \sin(\pi/2 - \alpha). \label{trig4} \\
&\cos^2 \alpha + \sin^2 \alpha = 1. \label{trig5}
\end{align}
Näistä ominaisuus \eqref{trig1} (joka seuraa myös \eqref{trig3}:sta) kertoo, että cos ja sin
\index{jaksollinen funktio} \index{funktio B!c@jaksollinen}%
ovat \kor{jaksollisia} (periodisia) funktioita, jaksona $2\pi$ = yksikköympyrän kehän pituus. 
\index{parillinen, pariton!b@funktio} \index{funktio B!b@parillinen, pariton}%
Ominaisuus \eqref{trig2} kertoo, että kosini on \kor{parillinen} (symmetrinen) ja sini 
\kor{pariton} (antisymmetrinen) funktio. Ominaisuus \eqref{trig5} on Pythagoraan lause.
\begin{figure}[H]
\setlength{\unitlength}{1cm}
\begin{picture}(14,4)(-2,-2)
\put(-2,0){\vector(1,0){14}} \put(11.8,-0.4){$x$}
\put(0,-2){\vector(0,1){4}} \put(0.2,1.8){$y$}
%\linethickness{0.5mm}
\multiput(3.14,0)(3.14,0){3}{\drawline(0,-0.1)(0,0.1)}
\put(0.1,-0.4){$0$} \put(3.05,-0.4){$\pi$} \put(6.10,-0.4){$2\pi$} \put(9.20,-0.4){$3\pi$}
\curve(
   -1.5708,    0.0000,
   -1.0708,    0.4794,
   -0.5708,    0.8415,
   -0.0708,    0.9975,
    0.4292,    0.9093,
    0.9292,    0.5985,
    1.4292,    0.1411,
    1.9292,   -0.3508,
    2.4292,   -0.7568,
    2.9292,   -0.9775,
    3.4292,   -0.9589,
    3.9292,   -0.7055,
    4.4292,   -0.2794,
    4.9292,    0.2151,
    5.4292,    0.6570,
    5.9292,    0.9380,
    6.4292,    0.9894,
    6.9292,    0.7985,
    7.4292,    0.4121,
    7.9292,   -0.0752,
    8.4292,   -0.5440,
    8.9292,   -0.8797,
    9.4292,   -1.0000,
    9.9292,   -0.8755,
   10.4292,   -0.5366,
   10.9292,   -0.0663)
\curvedashes[1mm]{0,1,2}
\curve(
   -1.5708,   -1.0000,
   -1.0708,   -0.8776,
   -0.5708,   -0.5403,
   -0.0708,   -0.0707,
    0.4292,    0.4161,
    0.9292,    0.8011,
    1.4292,    0.9900,
    1.9292,    0.9365,
    2.4292,    0.6536,
    2.9292,    0.2108,
    3.4292,   -0.2837,
    3.9292,   -0.7087,
    4.4292,   -0.9602,
    4.9292,   -0.9766,
    5.4292,   -0.7539,
    5.9292,   -0.3466,
    6.4292,    0.1455,
    6.9292,    0.6020,
    7.4292,    0.9111,
    7.9292,    0.9972,
    8.4292,    0.8391,
    8.9292,    0.4755,
    9.4292,   -0.0044,
    9.9292,   -0.4833,
   10.4292,   -0.8439,
   10.9292,   -0.9978)
\put(1,1.2){$y=\sin x$}
\put(5.5,1.2){$y=\cos x$}
\end{picture} 
\end{figure}
Trigonometrisissa funktioevaluaatioissa $x\map\cos x$ ja $x\map\sin x$ on, poikkeuksellisia 
$x$:n arvoja lukuunottamaatta, turvauduttava laskimiin. Laskinta tarvitaan myös, kun halutaan
saada selville, missä kulmassa trigonometrinen funktio saa annetun arvon, ts.\ halutaan 
ratkaista $\alpha$ yhtälöstä $\,\cos\alpha=y\,$ tai $\,\sin\alpha=y\,$, kun tunnetaan
$y\in[-1,1]$. Laskimella saadaan yleensä yhden ratkaisun likiarvo (esim.\ komennolla 
$\cos^{-1} y$ tai $\arccos y$). Muut ratkaisut voidaan tämän jälkeen päätellä laskimeen 
enempää turvautumatta, sillä kosinin ja sinin määritelmien perusteella pätee
\[ \boxed{ \begin{aligned} \quad\ykehys
\cos\alpha=\cos\beta &\qekv \beta=\alpha + n \cdot 2\pi\ 
                            \tai\ \beta=-\alpha + n \cdot 2\pi, \qquad n\in\Z, \\
\sin\alpha=\,\sin\beta &\qekv \beta=\alpha + n \cdot 2\pi\ 
                              \tai\ \beta=\pi-\alpha + n \cdot 2\pi, \quad n\in\Z. \quad\akehys
\end{aligned} } \]
\begin{Exa} Millä kulman $\alpha$ asteluvuilla välillä $[0^{\circ},360^{\circ}]$ on
$\cos 2\alpha=-2/3$ (yhden desimaalin tarkkuus)?
\end{Exa}
\ratk Laskin antaa yhtälölle $\cos x=-2/3$ ratkaisun $x=2.300523.. \vastaa 131.8^{\circ}$,
joten mahdolliset $\alpha$:n arvot ovat
\[ \begin{cases}
\,2\alpha = 131.8^{\circ} + n \cdot 360^{\circ} 
               &\qimpl \alpha = \ 65.9^{\circ} + n \cdot 180^{\circ}, \quad n\in\Z, \\
\,2\alpha = -131.8^{\circ} + m \cdot 360^{\circ} 
               &\qimpl \alpha =  -65.9^{\circ} + m \cdot 180^{\circ}, \quad m\in\Z.
\end{cases} \]
Kysytyt $\alpha$:n arvot suuruusjärjestyksessä ($n=0,\ m=1,\ n=1,\ m=2$)\,:
\[
65.9^{\circ},\ 114.1^{\circ},\ 245.9^{\circ},\ 294.1^{\circ}. \loppu
\]

Perusominaisuuksien \eqref{trig1}--\eqref{trig5} ohella tarvitaan monesti nk.\
\kor{yhteenlaskukaavoja}
\begin{equation} \label{trig6} \boxed{ \begin{aligned}
\ykehys\quad &\cos(\alpha + \beta) 
                    = \cos\alpha\cos\beta - \sin\alpha\sin\beta, \quad  \\
     \akehys &\sin(\alpha + \beta) = \,\sin\alpha\cos\beta + \cos\alpha\sin\beta. 
\end{aligned} } \end{equation}
Nämä ovat voimassa $\forall \alpha, \beta \in \R$. Kaavat on helpointa perustella 
vektorilaskulla, ks.\ kuvio.
\begin{multicols}{2}
\begin{figure}[H]
\setlength{\unitlength}{1cm}
\begin{center}
\begin{picture}(4.5,4.5)(-1.5,0)
\put(2,2){\bigcircle{4}}
\put(2,2){\vector(3,1){1.9}}
\put(2,2){\vector(1,3){0.63}}
\put(2,2){\vector(1,0){2}} \put(2,2){\vector(0,1){2}}
\put(4,2.55){$\vec a$} \put(2.55,4){$\vec b$} \put(4.2,1.9){$\vec i$} \put(1.9,4.2){$\vec j$}
\put(2,2){\arc{2}{-0.32}{0}} \put(3.2,2.1){$\alpha$}
\put(2,2){\arc{1}{-1.25}{0}} \put(2.4,2.4){\begin{turn}{40}$\beta$\end{turn}}
\end{picture}
\end{center}
\end{figure}
\begin{align*}
\vec a &= \cos \alpha \, \vec i + \sin \alpha \, \vec j \\[2mm]
\vec b &= \cos \beta \, \vec i + \sin \beta \, \vec j
\end{align*}
\end{multicols}
Kuvion ja skalaaritulon määritelmän perusteella on
\[
\vec a \cdot \vec b = \cos\alpha\cos\beta + \sin\alpha\sin\beta = \cos(\beta-\alpha).
\]
Vaihtamalla $\alpha \rightarrow -\alpha$ ja käyttämällä symmetriaominaisuuksia \eqref{trig2}
seuraa kaavoista \eqref{trig6} ensimmäinen. Toinen seuraa tämän jälkeen ensimmäisestä sekä
kaavoista \eqref{trig4},\,\eqref{trig2}:
\begin{align*}
\sin(\alpha + \beta) &= \cos\left(\tfrac{\pi}{2} - \alpha - \beta\right) \\[1mm]
                     &= \cos\left(\tfrac{\pi}{2} - \alpha\right) \cos(-\beta) 
                            - \sin\left(\tfrac{\pi}{2} - \alpha\right) \sin(-\beta) \\[1mm]
                     &= \sin \alpha \cos \beta + \cos \alpha \sin \beta.
\end{align*}
Valitsemalla em.\ kaavoissa $\beta = \alpha$ saadaaan monesti kysytyt
\kor{kaksinkertaisen kulman kaavat}:
\begin{equation} \label{trig7} \boxed{ \begin{aligned}
\ykehys\quad \cos 2\alpha &= \cos^2\alpha-\sin^2\alpha \quad \\ 
                          &= 2\cos^2 \alpha - 1 \\
                          &= 1-2\sin^2\alpha, \quad \\
     \akehys \sin 2\alpha &= 2\sin \alpha \cos \alpha.
\end{aligned} } \end{equation}
\begin{Exa} \label{sin kolme alpha} Kaavojen \eqref{trig6},\,\eqref{trig7} ja \eqref{trig5}
perusteella
\begin{align*}
\sin 3\alpha &= \sin (\alpha + 2\alpha) \\
&= \sin \alpha \cos 2\alpha + \cos \alpha \sin 2\alpha \\
&= \sin \alpha(1-2\sin^2 \alpha) + 2\cos^2 \alpha \sin \alpha \\
&= \sin \alpha - 2\sin^3 \alpha + 2(1-\sin^2 \alpha) \sin \alpha \\
&= 3\sin\alpha - 4\sin^3\alpha. \loppu
\end{align*}
\end{Exa}
\begin{Exa} \label{kulman kolmijako}
Laske $\,x = \sin\frac{\pi}{18} = \sin 10\aste$.
\end{Exa}
\ratk Edellisen esimerkin perusteella $x$ toteuttaa kolmannen asteen yhtälön
\[
3x - 4x^3 = \sin 30\aste = \tfrac{1}{2}.
\]
Numeerisin keinoin löydetään ratkaisu $x=0.173648177669303 \ldots$\footnote[2]{Esimerkin luku
$x$ on algebrallinen mutta ei geometrinen, ts.\ $x \not\in\G$ (vrt.\ Luku \ref{geomluvut}). 
Kulmaa $\frac{\pi}{18} = 10\aste$ ei siten ole mahdollista konstruoida geometrisesti.
Yleisestikin gometrisesti konstruoitavissa olevat kulmat ovat melko harva joukko. Esimerkiksi
kulma $3\aste$ voidaan konstruoida (ks.\ Harj.teht. \ref{H-II-5: geometrinen kulma}), sen
sijaan kulmia $2\aste$ ja $1\aste$ ei voida.} \loppu

\subsection{Tangentti ja kotangentti}
\index{funktio C!b@$\tan$, $\cot$|vahv}
\index{trigonometriset funktiot|vahv}

Muista trigonometrisista funktioista tärkeimmät ovat \kor{tangentti} ($\tan$) ja
\kor{kotangentti} ($\cot$), jotka määritellään
\begin{align*}
\tan\alpha = \frac{\sin\alpha}{\cos\alpha} \qquad 
                     &(\alpha \in \R, \ \alpha \neq (n + \tfrac{1}{2})\pi,\ n \in \Z), \\[3mm]
\cot\alpha = \frac{\cos\alpha}{\sin\alpha} \qquad 
                     &(\alpha \in \R, \ \alpha \neq n \pi, \ n \in \Z).
\end{align*}
\begin{figure}[H]
\setlength{\unitlength}{1cm}
\begin{picture}(14,4.5)(-2.5,-2.5)
\put(-2,0){\vector(1,0){13}} \put(10.8,-0.4){$x$}
\put(0,-2){\vector(0,1){4}} \put(0.2,1.8){$y$}
\multiput(3.14,0)(3.14,0){3}{\drawline(0,-0.1)(0,0.1)}
\put(0.1,-0.4){$0$} \put(3.05,-0.4){$\pi$} \put(6.10,-0.4){$2\pi$} \put(9.20,-0.4){$3\pi$}
\multiput(0,0)(3.14,0){4}{
\curve(
   -1.1,-2,     
   -1.0708,   -1.8305,
   -0.9708,   -1.4617,
   -0.8708,   -1.1872,
   -0.7708,   -0.9712,
   -0.6708,   -0.7936,
   -0.5708,   -0.6421,
   -0.4708,   -0.5090,
   -0.3708,   -0.3888,
   -0.2708,   -0.2776,
   -0.1708,   -0.1725,
   -0.0708,   -0.0709,
    0.0292,    0.0292,
    0.1292,    0.1299,
    0.2292,    0.2333,
    0.3292,    0.3416,
    0.4292,    0.4577,
    0.5292,    0.5848,
    0.6292,    0.7279,
    0.7292,    0.8935,
    0.8292,    1.0917,
    0.9292,    1.3386,
    1.0292,    1.6622,
        1.1,2)}
\curvedashes[1mm]{0,1,2}
\multiput(0,0)(3.14,0){3}{
\curve(
    0.5000,   1.8305,
    0.6000,    1.4617,
    0.7000,    1.1872,
    0.8000,    0.9712,
    0.9000,    0.7936,
    1.0000,    0.6421,
    1.1000,   0.5090,
    1.2000,   0.3888,
    1.3000,    0.2776,
    1.4000,   0.1725,
    1.5000,    0.0709,
    1.6000,  -0.0292,
    1.7000,  -0.1299,
    1.8000,   -0.2333,
    1.9000,   -0.3416,
    2.0000,   -0.4577,
    2.1000,   -0.5848,
    2.2000,   -0.7279,
    2.3000,   -0.8935,
    2.4000,   -1.0917,
    2.5000,   -1.3386,
    2.6000,   -1.6622,
    2.7000,   -2.1154)}
\put(-2.2,-0.4){$y=\tan x$}
\put(1.3,0.5){$y=\cot x$}
\end{picture}
\end{figure}
Tangentin määritelmästä voidaan päätellä, että yhtälöllä $\tan\alpha=y$ on ratkaisu jokaisella
$y\in\R$ ja että ratkaisu on yksikäsitteinen esim.\ lisäehdolla $\alpha \in (-\pi/2,\,\pi/2)$ 
tai $\alpha \in [0,\pi)$. Poikkeuksellisia $y$:n arvoja lukuunottamatta tarvitaan tällaisen
ratkaisun likiarvon hakemiseen laskin (komento $\tan^{-1} y$ tai $\arctan y$). Muut yhtälön 
ratkaisut saadaan perusratkaisusta lisäämällä $\pi$:n monikertoja, sillä tangentilla on
määritelmänsä perusteella ominaisuus
\[
\boxed{\quad\kehys \tan\alpha=\tan\beta \qekv \beta=\alpha + n \cdot \pi, \quad n\in\Z. \quad}
\]

Suoraan kaavasta \eqref{trig5} seuraa tangenttiin liittyvä, usein käyttöön tuleva kaava
\begin{equation} \label{trig8}
\boxed{\quad \frac{\ygehys 1}{\rule[-2mm]{0mm}{2mm} \cos^2 \alpha} = 1 + \tan^2 \alpha. \quad }
\end{equation}
Tangentin avulla saadaan myös joskus tarvittavia \kor{puolen kulman kaavoja}: Kun lähdetään
kaavoista \eqref{trig7} ja käytetään kaavaa \eqref{trig8}, saadaan
\begin{align*}
\sin \alpha &\,=\,2 \sin \frac{\alpha}{2} \cos \frac{\alpha}{2}
             \,=\, 2 \tan \frac{\alpha}{2} \cos^2 \frac{\alpha}{2}
             \,= \frac{2\tan \frac{\alpha}{2}}{1 + \tan^2 \frac{\alpha}{2}}\,, \\
\cos \alpha &\,=\,2\cos^2 \frac{\alpha}{2} - 1
             \,=\,\frac{2}{1+\tan^2 \frac{\alpha}{2}} - 1 
             \,=\,\frac{1 - \tan^2 \frac{\alpha}{2}}{1 + \tan^2 \frac{\alpha}{2}}\,.
\end{align*}
Näin saatiin kaavat \\
\begin{equation} \label{trig9}
\boxed{\quad\left.\begin{array}{ll}
\sin\alpha = \frac{\ykehys \D 2t}{\D 1+t^2} \\ \\
\cos\alpha = \frac{\D 1-t^2}{\D 1+t^2} \\ \\
\tan\alpha = \frac{\D 2t}{\akehys \D 1-t^2}
\end{array}\right\} \ t= \tan \frac{\alpha}{2}\,. \quad}
\end{equation} \\
Näiden mukaan trigonometristen funktioiden arvojen laskemiseen riittää suorittaa ainoastaan yksi
'aidosti trigonometrinen' laskuoperaatio $\alpha \map \tan \frac{\alpha}{2} = t$, minkä jälkeen
tarvitaan vain reaalilukujen kunnan laskuoperaatioita.\footnote[2]{Trigonometristen funktioiden
laskukaavoista \eqref{trig1}--\eqref{trig9} on syytä huomauttaa, että kaikki näissä kaavoissa
esiintyvät laskuoperaatiot, kuten $\alpha\map\sin\alpha$, $\alpha\map\pi/2-\alpha$,
$(\alpha,\beta)\map\alpha+\beta$ tai $\alpha\map\alpha/2$, voidaan toteuttaa geometrisesti, jos
lähtökohtana olevat kulmat tunnetaan geometrisina olioina. Näin ymmärrettynä kaavat 
\eqref{trig1}--\eqref{trig9} ovat siis päteviä, vaikkei kulman mittaa lainkaan määriteltäisi.}

Trigonometrisista funktioista maininnan arvoisia ovat vielä kosinin ja sinin johdannaisfunktiot
\index{funktio C!c@$\sec$, $\csc$}
\kor{sekantti} ($\sec$) ja \kor{kosekantti} ($\csc$):
\begin{align*}
\sec \alpha = \frac{1}{\cos \alpha} \qquad 
                     &(\alpha \in \R, \ \alpha \neq (n + \tfrac{1}{2}) \pi, \ n \in \Z), \\[3mm]
\csc \alpha = \frac{1}{\sin \alpha} \qquad 
                     &(\alpha \in \R, \ \alpha \neq n \pi, \ n \in \Z).
\end{align*}


\subsection{Sovellusesimerkki: Harmoninen värähtely}
\index{zza@\sov!Harmoninen värähtely|vahv}

Sinin ja kosinin yhdistelmäfunktio $f(x) = A \sin x + B \cos x\ (A,B\in\R)$ on sovelluksissa 
yleinen. Tyypillisesti kyse on 
\index{harmoninen värähtely}%
\pain{harmonisesta} \pain{värähtel}y\pain{stä}, jossa jokin 
fysikaalinen suure $y$ (esim.\ sähköjännite) vaihtelee \pain{aika}muuttujan $t$ mukaan siten, 
että $y(t) = A \sin \omega t + B \sin \omega t$. Suuretta $\omega$  (yksikkö $= 1/$s) sanotaan 
värähtelyn \pain{kulmataa}j\pain{uudeksi}. Mainittuun funktioon $f(x)$ päädytään, kun otetaan 
käyttöön (dimensioton) muuttuja $x=\omega t$.

Tarkasteltava funktio saadaan selvempään muotoon, kun merkitään ensin
\[
R = \sqrt{A^2 + B^2}.
\]
Tällöin on $(A/R)^2+(B/R)^2=1$, eli piste $P=(A/R,B/R)$ on yksikköympyrällä
(ol.\ $(A,B)\neq(0,0)$). Näin ollen jollakin $\alpha\in\R$ pätee
\[
\frac{A}{R} = \cos\alpha, \quad \frac{B}{R} = \sin\alpha \qimpl \tan\alpha = \frac{B}{A}\,.
\]
Näistä viimeinen ehto määrää $\alpha$:n $\,\pi$:n monikertaa vaille 
yksikäsitteisesti. Huomioimalla myös $\sin\alpha$:n tai $\cos\alpha$:n merkki
(kahdesta muusta ehdosta) nähdään, että $\alpha$ on $2\pi$:n monikertaa vaille yksikäsitteinen.
Yhteenlaskukaavan \eqref{trig7} perusteella $f(x)$ voidaan nyt esittää muodossa
\begin{align*}
f(x) &= R(\cos\alpha\sin x + \sin\alpha\cos x) \\
     &= R\sin(x+\alpha) \\
     &= R\sin(x-\beta), \quad \beta=-\alpha.
\end{align*}
Sanotaan tällöin, että $R$ on (esim.\ värähtelyn)
\index{amplitudi} \index{vaihekulma}%
\kor{amplitudi} ja $\alpha$ 
(tai $\beta=-\alpha$) on \kor{vaihekulma} (vaihesiirtymä). Funktion $f$ kuvaaja saadaan siis 
skaalaamalla funktion $\sin x$ kuvaaja amplitudilla $R$ ja siirtämällä skaalattu kuvaaja 
vaihekulman verran $x$-akselin suunnassa.
\begin{figure}[H]
\setlength{\unitlength}{1cm}
\begin{center}
\begin{picture}(6.5,3)(0,-1)
\put(0,0){\vector(1,0){6}} \put(5.8,-0.5){$x$}
\put(1.7,-1){\vector(0,1){3}} \put(1.9,1.8){$y$}
\put(3,0){\line(0,-1){0.1}}   \put(3.1,-0.4){$\beta$}
\put(1.7,1){\line(1,0){0.1}}  \put(2,0.8){R}
\put(3,0){
\curve(
-2.17,-0.22,
   -2, 0.28,
-1.83, 0.70,
-1.67, 0.96,
 -1.5, 0.98,
-1.33, 0.76,
-1.17, 0.36,
   -1,-0.14,
-0.83, -0.6,
-0.67,-0.90,
 -0.5, -1.0,
-0.33,-0.84,
-0.17,-0.48,
    0,    0,
 0.17, 0.48,
 0.33, 0.84,
  0.5,  1.0,
 0.67, 0.90,
 0.83,  0.6,
    1, 0.14,
 1.17,-0.36,
 1.33,-0.76,
  1.5,-0.98,
 1.67,-0.96,
 1.83,-0.70,
    2,-0.28,
 2.17, 0.22)}
\end{picture}
\end{center}
\end{figure}
\begin{Exa} Kirjoita seuraavat funktiot perusmuotoon $f(x)=R\sin(x-\beta)$:
\[
\text{a)}\,\ f(x) = \sin x - \sqrt{3}\,\cos x \qquad \text{b)}\,\ f(x)=-3\sin x - 4\cos x
\]
\end{Exa}
\ratk \ a) \ Tässä on $R=2$, jolloin $\alpha$ ratkeaa ehdoista
\[
\cos\alpha = \frac{1}{2}\,, \quad \sin\alpha = -\frac{\sqrt{3}}{2} \qimpl \alpha 
                                             = -\frac{\pi}{3} + n\cdot 2\pi, \quad n\in\Z.
\]
Siis $f(x) = 2\sin(x-\tfrac{\pi}{3})$.

b) \ Tässä on $R=5$ ja $\tan\alpha=4/3$. Laskin antaa $\alpha=0.927295..\,$ eli 
$\alpha \approx 53.1\aste$, mutta ratkaisu ei ole käypä, koska on $\cos\alpha>0$. Valitaan 
$\alpha=-\pi+0.927295..$ $=-2.214297..\,$ eli $\alpha \approx 53.1\aste-180\aste=-126.9\aste$,
jolloin on edelleen $\tan\alpha=4/3$ ja $\cos\alpha<0$. Siis $f(x)=5\sin(x-\beta)$, missä 
$\beta=2.214297..\,\Vastaa\, 126.9\aste$. \loppu

\Harj
\begin{enumerate}

\item
Psykiatrin vastaanottoa voi kuvata matemaattisesti kolmikkona (psykiatri, potilas, diagnoosi).
Pohdi seuraavissa tapauksissa (kussakin erikseen), millaisilla oletuksilla kyseessä on
funktio: \newline
a) \ psykiatri : potilaat $\kohti$ diagnoosit \newline
b) \ potilas  :  psykiatrit $\kohti$ diagnoosit \newline
c) \ psykiatri : diagnoosit $\kohti$ potilaat \newline
d) \ potilas : diagnoosit $\kohti$ psykiatrit

\item
Funktiot $f$ ja $g$ määritellään laskusäännöillä $f(x,y)=x^5+2x^4y-2x^2y^3$ ja
$g(x,y,z)=(x+y+z)/(1+x^2+y^2+z^2)$, missä $x,y,z\in\R$. Laske $f(2,-3)$, $f(\sqrt{2},\sqrt{2})$,
$g(0,0,0)$ ja $g(1,-2,3)$. 

\item 
Mitkä ovat seuraavien funktioiden määrittelyjoukot ($x,y\in\R$)? 
\[
\text{a)}\,\ f(x)=x^{0} \quad\ \text{b)}\,\ f(x)=\frac{x^2}{x} \quad\ 
\text{c)}\,\ f(x,y)=\frac{x+y}{x^2-y^2}
\]
Voiko nämä funktiot ilmaista määrittelyjoukossaan jollakin yksinkertaisemmalla laskusäännöllä?

\item
Mitä yhteistä ja mitä eroa on seuraavilla reaalifunktioilla? \newline
a) \ $\D{f(x)=\frac{x^4}{x^2}\ \ \text{ja}\ \ g(x)=x^2 \quad\ }$
b) \ $\D{f(x)=\sum_{k=1}^\infty \left(\frac{x}{2}\right)^k\ \text{ja}\ \
               g(x)=\frac{x}{2-x}}$

\item
Mitkä seuraavista joukon $\Rkaksi=\R\times\R$ osajoukoista $F$ ovat funktioita? \newline
a) \ $F=\{(1,2),(2,1),(3,3),(\pi,e),(e,\pi)\}$ \newline
b) \ $F=\{(1,2),(2,3),(3,3),(3,2),(2,1)\}$ \newline
c) \ $F=\{(n,n^2) \mid n\in\Z\}$ \newline
d) \ $F=\{(n^2,n) \mid n\in\Z\}$ \newline
e) \ $F=\{(n^2,n) \mid n\in\N\}$

\item
Tulkitse funktioina (määrittely- ja arvojoukko!) tason vektorien laskuoperaatiot: yhteenlasku,
skalaarilla kertominen ja pistetulo.
 
\item
a) Onko olemassa funktio $f$, jonka määrittelyjoukko $=\R$ ja $\forall x\in\R$ pätee
$f(x)=\sum_{k=0}^\infty x^k$\,? Jos vastaus on myönteinen, niin määrittele $f$! \newline 
b) Yhden reaalimuuttujan sisältävän predikaatin (Luku \ref{logiikka}) voi tulkita funktioksi
ja jopa reaalifunktioksi. Miten?

\item
Olkoon $\sin\alpha=7/25$ ja $\cot\beta=-5/12$. Laske lausekkeen $\sin(\alpha-\beta)$ mahdolliset
arvot.

\item
Nelikulmion sivujen pituudet ovat $1$, $2$, $3$ ja $4$, ja yhden kulman mitta asteina on
$160\aste$. Laske kaikkien nämä ehdot täyttävien nelikulmioiden kolmen muun kulman mitat
$0.1$ asteen tarkkuudella. Piirrä kuviot! 

\item
Ratkaise seuraavat trigonometriset yhtälöt: \newline
a) \ $\sin 2x=\cos 7x\quad\,$ b) \ $\tan 2x=3\tan x\quad\,$ 
c) \ $4\sin^2 x=\tan x$ \newline
d) \ $\abs{\sin x+\abs{\sin x}}=\cos x+\abs{\cos x}\quad\,$ e) \ $\cos 2x=\sin x + \cos x$ 

\item
Ratkaise trigonometrinen epäyhtälö, eli määritä joukko $A\subset\R$ tai $A\subset\Rkaksi$
siten, että epäyhtälö toteutuu täsmälleen kun $x \in A$ tai $(x,y) \in A$\,: \newline
a) \ $\sin\abs{x}<\abs{\sin x}\quad$ b) \ $\abs{\sin 2x}\ge\abs{\sin 3x}\quad$
c) \ $\sin 4x>\cot x-\tan x$ \newline
d) \ $2\sin(x-y^2)>1 \quad$ e) \ $\sin(x-y)+\cos x>0$

\item
a) Johda tangentin yhteenlaskukaava $\ \displaystyle{\,\tan(\alpha + \beta)
   =\frac{\tan\alpha + \tan\beta}{1-\tan\alpha\tan\beta}\,}.$ \vspace{1mm}\newline
b) Tunnetaan $t=\cos\alpha$. Lausu $t$:n avulla $\cos n\alpha$, kun $n=2,3,4,5$. \newline
c) Johda käänteiset yhteenlaskukaavat, joissa $\,\cos x \cos y$, $\,\sin x\sin y\,$ ja \newline
$\,\cos x\sin y\,$ lausutaan $\,\cos{(x\pm y)}$:n ja $\,\sin{(x\pm y)}$:n avulla.

\item \label{H-II-5: trigtuloksia}
a) Näytä, että
\[
\sin\frac{\alpha}{2}=\pm\sqrt{\tfrac{1}{2}(1-\cos\alpha)}, \quad
\cos\frac{\alpha}{2}=\pm\sqrt{\tfrac{1}{2}(1+\cos\alpha)}.
\]
Millä väleillä on molemmissa kaavoissa voimassa etumerkki $+$\,? \newline
b) Johda kaavat
\[
\tan \frac{\alpha}{2}=\frac{\sin \alpha}{1+\cos \alpha}\,, \quad
  \cot \frac{\alpha}{2}=\frac{\sin \alpha}{1-\cos \alpha}\,.
\]

\item
Määritä amplitudi ja vaihekulma: \newline
a) \ $f(x)=3\cos x-4\sin x \quad\ \ \ $ b) \ $f(x)=-4\sin x+\cos x$ \newline
c) \ $f(x)=76\cos x+57\sin x \quad$     d) \ $f(x)\,=\,\sin 2x(\sec x-2\csc x)$

\item
Vaihtovirran kolmen eri vaiheen jännitteet ovat
\[
V_i(t)=V_0\sin(\omega t+\varphi_i),\ \ i=1,2,3, \quad \varphi_2=\varphi_1 + \frac{2\pi}{3}\,,\
                                                      \varphi_3=\varphi_1 + \frac{4\pi}{3}\,.
\]
Määritä vaiheiden 1 ja 2 välisen jännitteen $V_1-V_2$ amplitudi ja vaihekulma. Mikä on 
kaikkien vaiheiden jännitteiden summa?

\item
Saata funktio
\[
f(x) = \sin x + 2\sin(x+\frac{2\pi}{3})+3\sin(x+\frac{4\pi}{3})
\]
muotoon $\,f(x)=R\sin(x+\alpha)$.

\item (*) 
Funktioista puhuttaessa lausutaan $f(x)$ usein '$f$ pisteessä $x$'. Olkoon nyt $x\in\R$ annettu
'piste' ja määritellään 'pistefunktio' $x$ seuraavasti:
\[
x(f)=f(x)\text{ }\forall f\in A,
\]
missä $A=\{$reaalifunktiot, jotka on määritelty pisteessä $x\}$. Onko tällainen 
funktion määrittely todella mahdollinen ja jos on, miten $x(f)$ pitäisi lausua?

\item (*) \label{H-II-5: geometrinen kulma}
Tasakylkinen kolmio, jonka huippukulma $=36\aste$, voidaan jakaa kahteen kolmioon siten, että
molemmat osakolmiot ovat tasakylkisiä. Käyttäen tätä ideaa lähtökohtana laske $x=\sin 3\aste$
tarkasti juurilukujen avulla. Päättele, että kulma $3\aste$ on konstruoitavissa geometrisesti.

\item (*) \label{H-II-5: sinin raja-arvo}
Todista: $\ {\D \sin\frac{\pi}{2^{n+1}} 
               \,\le\, \left(\frac{1}{\sqrt{2}}\right)^n, \quad n=0,1,2, \ldots}$ 

\item (*) \label{H-II-5: minmax}
Määritä funktion $f(x)=\cos^2 x+4\sin x\cos x+3\sin^2 x$ pienin ja suurin arvo sekä minimi-
ja maksimikohdat saattamalla funktio muotoon $f(x)=R\sin(2x+\alpha)+C$.

\end{enumerate} % Trigonometriset funktiot
\section{Avaruuden vektorit. Ristitulo} \label{ristitulo}
\sectionmark{Avaruuden vektorit}
\alku
\index{vektoria@vektori (geometrinen)!b@avaruuden \Ekolme|vahv}


Avaruusgeometrisissa tarkasteluissa klassisena lähtökohtana on taaskin
\index{euklidinen!ab@pisteavaruus \Ekolme} \index{pisteavaruus}%
\kor{euklidinen pisteavaruus}, tällä kertaa nimeltään $\Ekolme$, 'E kolme'. Seuraavat
avaruudelliset käsitteet oletetaan jatkossa (myös myöhemmissä luvuissa) tunnetuiksi:
\begin{itemize}
\item piste, (avaruus)jana, \kor{avaruuskolmio}
\item \kor{avaruussuora}, \kor{avaruustaso}, avaruuden puolisuora = \kor{avaruussuunta}
\item \kor{tetraedri}, \kor{suuntaissärmiö}, \kor{monitahokas}
\item \kor{suorakulmainen särmiö}
\item suorakulmaisen särmiön, suuntaissärmiön ja tetraedrin \kor{tilavuus}
\end{itemize}
\index{laskuoperaatiot!cc@avaruusvektoreiden|(}
\kor{Avaruusvektorin} määrittelyn lähtökohtana on euklidisen avaruusgeometrian perusoletus
(tai perusaksioomien seuraus), että geometria jokaisella $\Ekolme$:n avaruustasolla on sama kuin
$\Ekaksi$:ssa. Tällöin voidaan ensinnäkin mitata avaruusjanan pituus (avaruustasolla, joka 
sisältää janan). Kun pituuteen liitetään avaruussuunta, tulee määritellyksi avaruusvektori 
suuntajanana. Vektori on siis jälleen yhdistetty tieto pituudesta (= vektorin itseisarvo) ja
suunnasta.

Vektorin kertominen skalaarilla (reaaliluvulla) määritellään kuten tasossa, ts.\ skaalataan
vektorin pituus ja joko säilytetään suunta tai vaihdetaan se vastakkaiseksi, riippuen kertojan
etumerkistä. Vektorien yhteenlaskun määrittelemiseksi olkoon $O$ $\Ekolme$:n referenssipiste = 
\index{origo}%
\kor{origo} ja $\vec a=\Vect{OA}$ ja $\vec b=\Vect{OB}$ kaksi avaruuden vektoria. Tällöin on
olemassa avaruustaso $T$, joka kulkee pisteiden $O,A,B$ kautta (täsmälleen yksi, jos pisteet 
eivät ole samalla avaruussuoralla). Kun siirrytään tasolle $T$, voidaan $\vec a+\vec b$ 
määritellä $T$:n kolmiodiagrammilla, eli avaruuskolmion avulla. Myös avaruusvektoreiden
\index{skalaaritulo!b@avaruusvektoreiden}%
\kor{skalaaritulo} määritellään kuten tasossa:
\[
\vec a \cdot \vec b = \abs{\vec a} \abs{\vec b} \cos \kulma(\vec a, \vec b).
\]
Tässä geometrinen konstruktio $\kulma(\vec a, \vec b)\map\cos\kulma(\vec a,\vec b)$ toimii
tasolla $T$ oletuksen mukaan kuten $\Ekaksi$:ssa, vrt.\ Luku \ref{skalaaritulo}. Skalaaritulolle
ovat myös voimassa samat lait kuin tasossa. Näistä kuitenkin osittelulaki
\[
\vec a \cdot (\vec b + \vec c\,) = \vec a \cdot \vec b + \vec a \cdot \vec c
\]
kaipaa lisäperusteluja, sillä kolmea avaruuden vektoria ei yleisesti voi sijoittaa samaan
tasoon.
\begin{figure}[H]
\begin{center}
\import{kuvat/}{kuvaII-10.pstex_t}
\end{center}
\end{figure}
Kuviossa $C$ ei ole yleisesti samassa tasossa kuin pisteet $O,A,B$. Suorat $l$ ja $l'$ kuitenkin
ovat yhdensuuntaiset (suuntavektori = $\vec a$), jolloin ne voidaan leikata kohtisuorasti
pisteiden $B$ ja $C$ kautta kulkevilla avaruustasoilla (yhdensuuntaiset katkoviivat kuvassa).
Leikkauspisteistä ja pisteistä $O,A,B$ muodostettavat avaruuskolmiot $OB'B$ ja $OC'C$
(ks.\ kuvio) ovat tällöin suorakulmaiset. Kun nyt skalaaritulon tasogeometrista määritelmää 
sovelletaan tasoilla, joihin ko.\ kolmiot sisältyvät, ja huomioidaan, että avaruuden 
suorakulmiossa $BB'C'C''$ sivujanat $BC''$ ja $B'C'$ ovat yhtä pitkät, niin seuraa 
(vrt.\ Luku \ref{skalaaritulo})
\begin{align*}
\vec a \cdot \vec b + \vec a \cdot \vec c 
       &= \abs{\vec a}\left(\abs{\overrightarrow{OB'}} + \abs{\overrightarrow{BC''}}\right) \\
       &= \abs{\vec a}\left(\abs{\overrightarrow{OB'}} + \abs{\overrightarrow{B'C'}}\right)
          =\abs{\vec a}\abs{\overrightarrow{OC'}} = \vec a \cdot (\vec b + \vec c).
\end{align*}
Siis osittelulaki on pätevä.
\index{laskuoperaatiot!cc@avaruusvektoreiden|)}

Yhteenlaskun, skalaarilla kertomisen ja skalaaritulon tultua määritellyksi joukossa 
$V=\{\text{avaruuden vektorit}\}$ on $(V,\R)$ jälleen vektoriavaruus ja skalaaritulolla 
(sisätulolla) $\vec u,\vec v \map \vec u\cdot\vec v$ varustettuna sisätuloavaruus, 
vrt.\ Luvut \ref{skalaaritulo}--\ref{abstrakti skalaaritulo}. 
Jos $\vec a$, $\vec b$ ja $\vec c$ ovat kolme
avaruuden vektoria, jotka poikkeavat $\vec 0$:sta eivätkä ole saman avaruustason suuntaiset,
niin voidaan päätellä avaruusgeometrisesti (vrt.\ vastaava tasogeometrinen päättely Luvussa
\ref{tasonvektorit}), että jokainen $\vec v \in V$ voidaan esittää yksikäsitteisesti muodossa
$\vec v=x\vec a+y\vec b+z\vec c$, missä $x,y,z\in\R$. Mainitut ehdot vektoreille 
$\vec a,\vec b,\vec c$ voidaan pelkistää ehdoksi
\index{lineaarinen riippumattomuus}%
\[
\boxed{\kehys\quad x\vec a+y\vec b+z\vec c=\vec 0 \qimpl x=y=z=0. \quad}
\]
Sanotaan tällöin, että vektorit $\vec a,\vec b,\vec c$ ovat \kor{lineaarisesti riippumattomat},
ja että vektorisysteemi $\{\vec a,\vec b,\vec c\}$ on $V$:n
\index{kanta}%
\kor{kanta}. Siis avaruusvektoreista muodostuva vektoriavaruus on $3$-ulotteinen: dim $V=3$.

Kun jokainen avaruuden vektori $\vec v \in V$ lausutaan annettujen kantavektoreiden 
lineaariyhdistelynä muodossa $\vec v = x\vec a+y\vec b+z\vec c$, niin syntyy kääntäen 
yksikäsitteinen vastaavuus $V \leftrightarrow \Rkolme$, missä $\Rkolme$ ('R kolme') on 
reaalikukukolmikkojen joukko:
\[
\Rkolme\ =\ \{ (x,y,z) \ | \ x \in \R, \ y \in \R, \ z \in \R \}\ =\ \R \times \R \times \R.
\]
\index{vektorib@vektori (algebrallinen)!b@$\R^3$:n}
\index{laskuoperaatiot!cd@vektoriavaruuden $(\R^3,\R)$}%
Myös $\Rkolme$ on vektoriavaruus, jossa vektorien laskuoperaatiot määritellään 
(vrt.\ $\Rkaksi$:n operaatiot)
\begin{align*}
(x_1,y_1,z_1)+(x_2,y_2,z_2)\ &=\ (x_1+x_2,\,y_1+y_2,\,z_1+z_2), \\
             \lambda(x,y,z)\ &=\ (\lambda x,\lambda y,\lambda z) \quad (\lambda\in\R).
\end{align*}

Avaruusvektoreiden muodostaman vektoriavaruuden
\index{kanta!a@ortonormeerattu}%
\kor{ortonormeerattu kanta} on kolmen vektorin 
systeemi $\{\vec i, \vec j, \vec k\}$, joka toteuttaa
\[
\abs{\vec i} = \abs{\vec j} = \abs{\vec k} = 1, \quad 
\vec i \cdot \vec j = \vec j \cdot \vec k = \vec i \cdot \vec k = 0.
\]
Lisäksi oletetaan yleensä, että $\{\vec i, \vec j, \vec k\}$ on nk.
\index{oikeakätinen (vektori)systeemi}%
\kor{oikeakätinen 
systeemi}.\footnote[2]{Oikeakätinen vektorisysteemi muuttuu \kor{vasenkätiseksi}
(ja vastaavasti vasenkätinen oikeakätiseksi), jos yhden vektorin suunta vaihdetaan
vastakkaiseksi. \index{vasenkätinen vektorisysteemi|av}} Tällä tarkoitetaan oikeaan käteen
liittyvää (funktio)vastaavuutta
\[
(\,\text{peukalo},\,\text{etusormi},\,\text{keskisormi}\,)\ \vast\ (\vec i,\vec j,\vec k)
\]
tai vaihtoehtoisesti
\[
(\,\text{peukalo},\,\text{etusormi},\,\text{keskisormi}\,)\ \vast\ (\vec k,\vec i,\vec j).
\] 
\begin{figure}[H]
\begin{center}
\import{kuvat/}{kuvaII-11.pstex_t}
\end{center}
\end{figure}
Em.\ oletuksin sanotaan $\Ekolme$:n koordinaatistoa $\{O,\vec i,\vec j,\vec k\}$
\index{koordinaatisto!b@karteesinen}% 
\kor{karteesiseksi}. Suoria ja tasoja, joilla koordinaateista kaksi (suora) tai yksi (taso)
\index{koordinaattiakseli} \index{koordinaattiakseli!b@--taso}%
saa arvon $0$, sanotaan \kor{koordinaattiakseleiksi} ja \kor{koordinaattitasoiksi}.
Nämä nimetään ko.\ suorilla tai tasoilla muuttuvien koordinaattien mukaan, esim.\
$x$-akseli, $xy$-taso.

Avaruuden vektoreiden skalaarituloa vastaa $\Rkolme$:n
\index{euklidinen!c@skalaaritulo} \index{skalaaritulo!c@$\R^n$:n euklidinen}%
\kor{euklidinen skalaaritulo}
(vrt.\ Määritelmä \ref{R2:n euklidinen skalaaritulo ja normi})
\[
\mpu_1 = (x_1,y_1,z_1), \ \mpu_2 = (x_2,y_2,z_2)\,: \quad
                                    \mpu_1 \cdot \mpu_2 = x_1x_2 + y_1y_2 + z_1z_2.
\]
Tälle ovat voimassa skalaaritulon aksioomat (ks.\ Luku \ref{abstrakti skalaaritulo}), joten
pätee myös Cauchyn--Schwarzin epäyhtälö (Lause \ref{schwarzR})
\[
\abs{\mpu_1\cdot\mpu_2} \le \abs{\mpu_1}\abs{\mpu_2},
\]
eli
\[
\abs{x_1x_2 + y_1y_2 + z_1z_2} \leq (x_1^2+y_1^2+z_1^2)^{1/2}(x_2^2+y_2^2+z_2^2)^{1/2},
\]
missä
\[
\abs{\mpu}=(x^2+y^2+z^2)^{1/2}, \quad \mpu=(x,y,z)\in\Rkolme
\]
on $\Rkolme$:n
\index{normi!euklidinen} \index{euklidinen!b@normi}%
\kor{euklidinen normi}.
\begin{Exa} Avaruuskolmion kärjet ovat $A=(-1,1,2)$, $B=(4,-2,3)$ ja $C=(-3,3,3)$. Laske kolmion
kulmien mitat asteina (yhden desimaalin tarkkuus).
\end{Exa}
\ratk Jos $\vec a=\Vect{AB}$ ja $\vec b=\Vect{AC}$, niin skalaaritulon määritelmän mukaan
\[
\cos\kulma BAC = \frac{\vec a\cdot\vec b}{\abs{\vec a}\abs{\vec b}}\,.
\]
Tässä on $\vec a=5\vec i-3\vec j+\vec k$ ja $\vec b=-2\vec i+2\vec j+\vec k$, joten saadaan
\[
\cos\kulma BAC\ 
  =\ \frac{5 \cdot (-2) + (-3) \cdot 2 +1 \cdot 1}{(5^2+3^2+1^2)^{1/2}(2^2+2^2+1^2)^{1/2}}\
  =\ -\sqrt{\frac{5}{7}}\,.
\]
Vastaavalla tavalla laskien saadaan
\[
\cos\kulma ABC\ =\ \sqrt{\frac{250}{259}}\,, \qquad
\cos\kulma ACB\ =\ \sqrt{\frac{32}{37}}\,.
\]
Laskimen avulla saadaan vastaukseksi: 
\[
\kulma BAC \approx 147.7\aste, \quad \kulma ABC \approx 10.7\aste, \quad
\kulma ACB \approx 21.6\aste. \loppu
\]

\subsection{Vektorien ristitulo}
\index{laskuoperaatiot!cc@avaruusvektoreiden|vahv}

Avaruuden vektoreille on määritelty skalaaritulon ohella toinen kertolaskun luonteinen
operaatio, jota sanotaan \kor{ristituloksi} tai \kor{vektorituloksi} (engl. cross product,
vector product). Jos $V=\{\text{avaruuden vektorit}\}$, niin ristitulo on kuvaus (funktio)
tyyppiä  $V \times V \rightarrow V$, ts. tulos on vektori (tästä nimitys vektoritulo).
Ristitulo merkitään $\vec a \times \vec b$, luetaan '$a$ risti $b$'.
\begin{Def} (\vahv{Ristitulo}) \label{ristitulon määritelmä}
\index{ristitulo (vektoritulo)|emph} \index{vektoritulo (ristitulo)|emph}
Avaruusvektorien $\,\vec a, \vec b\,$ \kor{ristitulo} eli \kor{vektoritulo} on vektori 
$\vec a \times \vec b$, joka toteuttaa
\begin{enumerate}
\item $\abs{\vec a \times \vec b} = \abs{\vec a}\abs{\vec b}\sin{\kulma(\vec a, \vec b)}$
\item $\vec a \times \vec b \ \perp \ \vec a \,\ \ja \,\ \vec a \times \vec b \ \perp \ \vec b$
\item $\{\vec a,\ \vec b,\ a \times \vec b\} \ \text{on oikeakätinen systeemi}$.
\end{enumerate}
\end{Def}
Säännössä (1) esiintyvä kulman sini tulkitaan ei-negatiiviseksi, ts.\
\[
\sin\kulma(\vec a,\vec b) = \sin\alpha,
\]
missä $\alpha$ on \pain{sisäkulman} mittaluku ($0\le\alpha\le\pi$). Sääntö (2) jättää
ristitulon suunnalle kaksi vaihtoehtoa, joista valinta suoritetaan säännön (3) ilmaisemalla
oikean käden säännöllä
\[
(\,\text{peukalo},\,\text{etusormi},\,\text{keskisormi}\,) 
                       \quad \map \quad (\vec a,\ \vec b,\ \vec a \times \vec b).
\]
\begin{figure}[H]
\begin{center}
\import{kuvat/}{kuvaII-14.pstex_t}
\end{center}
\end{figure}
Määritelmästä \ref{ristitulon määritelmä} seuraa vektoritulolle 'vino' vaihdantalaki
\begin{equation} \label{cross1}
\boxed{\kehys\quad (\vec b \times \vec a) = - (\vec a \times \vec b). \quad}
\end{equation}
Erityisesti on
\[
\vec a \times \vec a = \vec 0
\]
ja yleisemmin
\[
\vec a \times \vec b = \vec 0 \qekv \vec a \parallel \vec b \ \tai \ \vec a = \vec 0 \ 
                                                              \tai \ \vec b = \vec 0.
\]
(Tässä $\vec a \parallel \vec b$ tarkoittaa: $\vec a \uparrow \uparrow \vec b$ tai 
$\vec a \uparrow \downarrow \vec b$.) Ristitulosta (jollei muuten) voidaan siis päätellä, ovatko
kaksi nollavektorista poikkeavaa vektoria yhdensuuntaiset.

Vektoritulo ei ole liitännäinen:
\[
(\vec a \times \vec b) \times \vec c\ \neq\ \vec a \times (\vec b \times \vec c).
\]
Esim.\ jos $\vec a=\vec i$, $\vec b=\vec i$ ja $\vec c=\vec j$, on vasen puoli $=\vec 0$,
mutta oikea puoli $=-\vec j$.

Jos $\lambda \in \R$ ja $\lambda \geq 0$, niin ristitulon määritelmästä seuraa
\[
(\lambda\vec a) \times \vec b \,=\, \vec a \times (\lambda\vec b) 
                              \,=\, \lambda(\vec a \times \vec b).
\] 
Suuntaussäännöstä seuraa myös että jos $\vec a$:n tai $\vec b$:n suunta vaihdetaan, niin vaihtuu
myös $(\vec a \times \vec b)$:n suunta, ts.
\[
(-\vec a) \times \vec b \,=\, \vec a \times (-\vec b) \,=\, - (\vec a \times \vec b).
\]
Yhdistämällä nämä tulokset todetaan, että ristitulon ja skalaarilla kertomisen voi yhdistää 
normaaleilla osittelulaeilla:
\begin{equation} \label{cross2}
\boxed{\kehys\quad (\lambda \vec a) \times \vec b \,=\, \vec a \times (\lambda \vec b) 
                     \,=\, \lambda (\vec a \times \vec b), \quad \lambda \in \R. \quad}
\end{equation}
Ristitulon ja vektorien yhteenlaskun välillä toimivat myös normaalit osittelulait:
\begin{equation} \label{cross3} \boxed{ \begin{aligned}
\ykehys\quad \vec a \times (\vec b + \vec c) 
                           &= \vec a \times \vec b + \vec a \times \vec c, \\
\akehys (\vec a + \vec b) \times \vec c      
                           &= \vec a \times \vec c + \vec b \times \vec c. \quad
\end{aligned} } \end{equation}
Nämä lait eivät kuitenkaan ole määritelmän perusteella ilmeisiä, vaan tarvitaan hieman 
geometrista erittelyä: Olkoon $T$ taso, jonka normaalivektori $=\vec a$. Tällöin 
$\vec a \times \vec b$ voidaan ymmärtää kolmivaiheisena geometrisena operaationa:
\begin{itemize}
\item[1.] Projisioidaan vektori $\vec b$ tasolle $T$ (\kor{ortogonaaliprojektio}).
          Tulos $\vec b_T$. \index{ortogonaaliprojektio}
\item[2.] Suoritetaan vektorin $\vec b_T$ \kor{kierto} tasolla $T$ suunnasta $\vec a$ katsottuna
          vastapäivään kulman $\pi/2$ verran (vektorin pituus säilyy).
          \index{kierto!a@geom.\ kuvaus}
\item[3.] Skaalataan vaiheen 2 tulos kertomalla luvulla $\abs{\vec a}$.
\end{itemize}
Vaiheita 1--2 on havainnollistettu kahdella kuvalla alla. Ensimmäisessä kuvassa tarkkailija on
suunnassa $\vec a \times \vec b$ pisteestä $P$, toisessa suunnassa $\vec a$. Kuviin on merkitty
myös toisen kuvan tarkkailusuunta.
\begin{figure}[H]
\setlength{\unitlength}{1cm}
\begin{center}
\begin{picture}(12,8.2)(0,1)
\path(0,3)(5,3)
\put(3.9,2.9){$\bullet$}
\put(4,3){\arc{1}{-2.35}{-1.59}}
\put(4,3){\vector(-1,1){3}}
\put(4,3){\vector(0,1){4}}
\put(4,3){\vector(-1,0){3}}
\put(10,3){\vector(-1,1){2}}
\put(10,3){\vector(1,1){2}}
%\put(10,3){\arc{1.5}{-2.3561}{-0.7854}}
\put(10.5,3.5){\line(-1,1){0.15}} \put(10.07,3.71){\line(5,-1){0.2}}
\put(9.5,3.5){\line(1,1){0.15}} \put(9.93,3.71){\line(-5,-1){0.2}}
\put(9.5,3.5){\line(0,1){0.12}} \put(9.5,3.5){\line(1,0){0.12}} 
\put(4.2,6.7){$\vec a$}
\put(1.2,3.2){$\vec b_T$}
\put(1.2,6){$\vec b$}
\put(3.6,3.6){$\alpha$}
\put(4.1,2.5){$P$}
\put(5.2,2.9){$T$}
\dashline{0.2}(1,6)(1,3) \put(1,3){\line(1,2){0.1}} \put(1,3){\line(-1,2){0.1}}
\put(9.9,2.9){$\bullet$}
\put(10.1,2.5){$P$}
\put(11.8,4.3){$\vec b_T$}
\put(8.4,4.8){$\vec a\times\vec b$}
%\put(10,3){\arc{1}{-2.35}{-0.8}}
%\put(9.76,3.41){\vector(-3,-2){0.1}}
\path(10.15,3.15)(10,3.3)(9.85,3.15)
\put(3.87,7.5){$\Downarrow$}
\put(2.7,8.2){\pain{Kuvakulma (b)}}
\put(3.87,-2.5){\begin{turn}{45}
\put(3.87,7.5){$\Downarrow$}
\put(2.7,8.2){\pain{Kuvakulma (a)}}
\end{turn}}
\put(0,1){\parbox{5cm}
         {\begin{center}(a) Tarkkailija suunnassa $\vec a\times\vec b$\end{center}}}
\put(6,1.2){\parbox{8cm}{\begin{center}(b) Tarkkailija suunnassa $\vec a$\end{center}}}
\end{picture}
\end{center}
\end{figure}
Ensimmäisessä kuvassa näkyy projektio-operaatio $\vec b \map \vec b_T$. Kierto-operaation
(vaihe 2) tulos osoittaa katsojan suuntaan. Toisessa kuvassa ei näy vektoria $\vec a$, ja 
$\vec b$:stä nähdään vain sen projektio $\vec b_T$. Kierto-operaatio näkyy tässä kuvassa.

Olkoon nyt $\,\vec b, \vec c\,$ avaruusvektoreita. Tällöin yhteenlaskudiagrammi
$\,\vec b + \vec c = \vec d\,$ nähdään alla olevan kuvion mukaisena em.\ 
tarkkailutilanteessa (b)\,:
\begin{figure}[H]
\setlength{\unitlength}{1cm}
\begin{center}
\begin{picture}(6,3)(0,0.5)
\put(0,0){\vector(4,1){6}} 
\put(0,0){\vector(1,1){3}}
\put(3,3){\vector(2,-1){3}}
\put(2.1,2.6){$\vec b_T$}
\put(5.5,1.9){$\vec c_T$}
\put(5.5,0.8){$\vec d_T$}
\end{picture}
\end{center}
\end{figure}
Kuvassa nähdään vain oikean yhteenlaskudiagrammin (kolmio avaruudessa) projektio tasolle $T$. 
Kuva kuitenkin kertoo, että projisoitu diagrammi on tason vektoreiden yhteenlaskudiagrammi, ts.
$\,\vec d_T = \vec b_T + \vec c_T\,$ eli
\[
\text{\kor{projektio}}(\vec b + \vec c) 
                    = \text{\kor{projektio}}(\vec b) + \text{\kor{projektio}}(\vec c).
\]
Toisaalta kierto vaiheessa 2 merkitsee vain koko projisoidun yhteenlaskudiagrammin kiertoa,
jolloin
\[
\text{\kor{kierto}}(\vec b_T + \vec c_T) 
                 = \text{\kor{kierto}}(\vec b_T) + \text{\kor{kierto}}(\vec c_T).
\]
Yhdistämällä nämä kaksi tulosta seuraa:
\begin{align*}
\text{\kor{kierto}}\,[\text{\kor{projektio}}(\vec b + \vec c\,)]
&= \text{\kor{kierto}}\,[\text{\kor{projektio}}(\vec b\,) + \text{\kor{projektio}}(\vec c\,)] \\
&= \text{\kor{kierto}}\,[\text{\kor{projektio}}(\vec b\,)] 
                                   + \text{\kor{kierto}}\,[\text{\kor{projektio}}(\vec c\,)].
\end{align*}
Kun tässä molemmat puolet kerrotaan luvulla $\abs{\vec a}$ ja todetaan, että
\[
\abs{\vec a}\, \text{\kor{kierto}}\,[\text{\kor{projektio}}(\Vect{\rule{0mm}{1.5mm}\ldots})] 
                                     \,=\, \vec a \times (\Vect{\rule{0mm}{1.5mm}\ldots}),
\]
niin on todistettu osittelulain \eqref{cross3} edellinen osa. Jälkimmäinen osa seuraa tästä
sekä vaihdantalaista \eqref{cross1}\,:
\[
(\vec a + \vec b) \times \vec c \,=\, -\vec c \times(\vec a + \vec b)
                                \,=\, -\vec c \times \vec a - \vec c \times \vec b
                                \,=\, \vec a \times \vec c + \vec b \times \vec c.
\]

\subsection{Ristitulon determinanttikaava}

Ristitulo lasketaan käytännössä ortonormeeratussa kannassa $\{\vec i,\vec j,\vec k\}$. Koska
\begin{align*}
&\vec i \times \vec i = \vec j \times \vec j = \vec k \times \vec k = \vec 0, \\
&\vec i \times \vec j = \vec k, \quad \vec j \times \vec k 
                      = \vec i, \quad \vec k \times \vec i = \vec j, \\
&\vec j \times \vec i = -\vec k, \quad \vec k \times \vec j 
                      = -\vec i, \quad \vec i \times \vec k = -\vec j,
\end{align*}
niin sääntöjen \eqref{cross2}--\eqref{cross3} perusteella
\begin{align*}
\vec a \times \vec b &= (x_1\vec i + y_1\vec j + z_1\vec k) 
                                            \times (x_2\vec i + y_2\vec j + z_2\vec k) \\
                     &= (y_1z_2-y_2z_1)\vec i - (x_1z_2-x_2z_1)\vec j + (x_1y_2-x_2y_1)\vec k.
\end{align*}
Ilmaistaan tämä taulukkomuotoisella muistisäännöllä
\[
\vec a \times \vec b = \left| \begin{array}{ccc}
\vec i & \vec j & \vec k \\
x_1 & y_1 & z_1 \\
x_2 & y_2 & z_2 
\end{array} \right|.
\]
Tässä siis toinen ja kolmas vaakarivi muodostuvat $\vec a$:n ja $\vec b$:n koordinaateista.
Kyse on nk.\ 
\index{determinantti|(}%
\kor{kolmirivisestä determinantista}, joka purkautuu ensin säännöllä
\[
\vec a \times \vec b = \left|\begin{array}{cc} 
y_1 & z_1 \\
y_2 & z_2
\end{array} \right| \vec i -
\left|\begin{array}{cc} 
x_1 & z_1 \\
x_2 & z_2
\end{array} \right| \vec j +
\left|\begin{array}{cc} 
x_1 & y_1 \\
x_2 & y_2
\end{array} \right| \vec k,
\]
missä uudet taulukko-oliot ovat \index{determinantti|)}%
\kor{kaksirivisiä determinantteja}.\footnote[2]{Determinanttioppi on käsinlaskussa
ennen paljon käytetty, oma erikoinen matematiikan lajinsa. Tähän yhteyteen lainataan 
determinanttiopista vain ristitulon käsittelyssä tarvittavat muistisäännöt.} Nämä on saatu 
alkuperäisestä kolmirivisestä poistamalla ne rivit ja sarakkeet, joiden yhtymäkohdassa on 
kertoimena oleva $\vec i$, $\vec j$ tai $\vec k$ --- siis ensimmäinen rivi sekä sarake, 
johon ko.\ vektori sisältyy --- huom.\ myös merkin vaihto kaavan keskimmäisessä termissä!
Kaksiriviset determinantit puretaan lopulta säännöllä
\[
\left|\begin{array}{cc} 
a_1 & b_1 \\
a_2 & b_2
\end{array} \right| = a_1b_2 - a_2b_1.
\]
\begin{Exa} Jos $\vec a = 3\vec i -6\vec j + 9\vec k$, $\vec b = 2\vec i + 5\vec j - 3\vec k$, 
$\vec c = -\vec i + 2\vec j - 3\vec k$, niin
\begin{align*}
\vec a \times \vec b &= \left| \begin{array}{rrr} \vec i & \vec j & \vec k \\ 3 & -6 & 9 \\ 
                                                          2 & 5 & -3 \end{array} \right| \\
                     &= [(-6)\cdot(-3)-5\cdot 9]\vec i - [3\cdot(-3)-2\cdot 9]\vec j 
                                                       + [3\cdot 5 -2\cdot(-6)]\vec k \\
                     &= -27\vec i -27\vec j + 27\vec k, \\
\vec a \times \vec c &= \left|\begin{array}{rrr} \vec i & \vec j & \vec k \\ 3&-6&9 \\
                                                -1&2&-3 \end{array}\right| = \vec 0. 
\end{align*}
Jälkimmäinen tulos merkitsee, että vektorit $\vec a$ ja $\vec c$ ovat yhdensuuntaiset (itse 
asiassa vastakkaissuuntaiset). \loppu
\end{Exa}

\subsection{Skalaarikolmitulo $\vec a\times\vec b\cdot\vec c$}
\index{skalaarikolmitulo|vahv}
\index{laskuoperaatiot!cc@avaruusvektoreiden|vahv}

Kolmen avaruusvektorin \kor{skalaarikolmitulo} määritellään
\[
\vec a \times \vec b \cdot \vec c \,=\, (\vec a \times \vec b\,) \cdot \vec c\,.
\]
Tässä on $\vec a \times \vec b$ laskettava ensin, joten sulkeita ei tarvitse merkitä. 
Karteesisessa koordinaatistossa komponenttimuodossa laskettuna on 
\[
\vec a \times \vec b \cdot \vec c = \vec c \cdot (\vec a \times \vec b\,) = 
\left| \begin{array}{ccc}
x_3 & y_3 & z_3 \\
x_1 & y_1 & z_1 \\
x_2 & y_2 & z_2 
\end{array} \right|,
\]
missä $\,\vec a \leftrightarrow (x_1,y_1,z_1)$, $\,\vec b \leftrightarrow (x_2,y_2,z_2)$ ja 
$\,\vec c \leftrightarrow (x_3,y_3,z_3)$. Determinantin määritelmästä seuraa, että rivit voidaan
'kierrättää' siten, että ensimmäinen rivi joutuu pohjimmaiseksi determinantin arvon muuttumatta:
\[
\left| \begin{array}{ccc}
x_3 & y_3 & z_3 \\
x_1 & y_1 & z_1 \\
x_2 & y_2 & z_2 
\end{array} \right| =
\left| \begin{array}{ccc}
x_1 & y_1 & z_1 \\
x_2 & y_2 & z_2 \\
x_3 & y_3 & z_3 
\end{array} \right|.
\]
Näin ollen pätee 'pisteen ja ristin vaihtosääntö'
\[
\boxed{\kehys\quad \vec a \times \vec b \cdot \vec c = \vec a \cdot \vec b \times \vec c. \quad}
\]

\subsection{Kolmion pinta-ala. Tetraedrin tilavuus}
\index{pinta-ala!a@avaruuskolmion|vahv}
\index{tilavuus!a@tetraedrin|vahv}

Ristitulolla ja skalaarikolmitulolla on fysikaalisten sovellutusten ohella myös hauskoja
geometrisia käyttökohteita. Esimerkiksi avaruuskolmion (myös tasokolmion) p\pain{inta-ala}
voidaan määrätä kätevästi ristitulon avulla, jos tunnetaan kolmion kärkipisteet karteesisessa
koordinaatistossa. Esimerkki valaiskoon asiaa.
\begin{Exa} \label{avaruuskolmion ala}
Avaruuskolmion kärjet ovat pisteissä $A=(1,0,0)$, $B=(1,1,-1)$ ja $C=(0,2,2)$. Mikä on kolmion
pinta-ala?
\end{Exa}
\ratk Tunnetun kolmion pinta-alakaavan ja \newline
vektoritulon määritelmän perusteella:
\begin{multicols}{2} \raggedcolumns
\begin{align*}
\text{ala} &= \frac{1}{2} \cdot \text{kanta} \cdot \text{korkeus} \\
           &= \frac{1}{2}\abs{\Vect{AB}}\abs{\Vect{AC}}\sin\kulma BAC \\
           &= \frac{1}{2}\abs{\Vect{AB}\times\Vect{AC}}.
\end{align*}
\begin{figure}[H]
\setlength{\unitlength}{1cm}
\begin{center}
\begin{picture}(4.5,2.5)(0,0.5)
\put(0,1){\vector(4,-1){4}}
\put(0,1){\vector(3,2){3}}
\put(3,3){\vector(1,-3){1}}
\put(-0.2,0.5){$A$} \put(4.1,-0.2){$B$} \put(3.2,3){$C$}
\end{picture}
\end{center}
\end{figure}
\end{multicols}
Tässä on $\,\Vect{AB}=\vec j - \vec k, \ \Vect{AC}=-\vec i +2\vec j + 2 \vec k$,
joten
\begin{align*}
\Vect{AB}\times\Vect{AC} = 
\left| \begin{array}{ccc}
\vec i & \vec j & \vec k \\
0 & 1 & -1 \\
-1 & 2 & 2 
\end{array} \right|
&\,=\, \left|\begin{array}{rr} 
1 & -1 \\
2 & 2
\end{array} \right|\,\vec i\ -\
\left|\begin{array}{rr} 
0 & -1 \\
-1 & 2
\end{array} \right|\,\vec j\ +\
\left|\begin{array}{rr} 
0 & 1 \\
-1 & 2
\end{array} \right|\,\vec k \\
&\,=\, (1 \cdot 2 + 1 \cdot 2)\,\vec i - (0 \cdot 2 - 1 \cdot 1)\,\vec j 
                                   +(0 \cdot 2 + 1 \cdot 1)\,\vec k \\
&\,=\, 4\vec i + \vec j + \vec k.
\end{align*}
Siis pinta-ala $=\tfrac{1}{2}\sqrt{4^2 + 1^2 + 1^2}= \frac{3}{2}\sqrt{2}$. \loppu
%Sivutuotteena saatiin pisteitten $A,B,C$ kautta kulkevan tason normaalivektori 
%$\vec n = 4\vec i + \vec j + \vec k$.

Kolmion kolmiulotteinen vastine on tetraedri. Tämän \pain{tilavuus} saadaan kätevästi
lasketuksi skalaarikolmitulon avulla, kun tunnetaan kärkipisteet karteesisessa 
koordinaatistossa. Esimerkki valaiskoon jälleen asiaa.
\begin{Exa}
Tetraedrin $K$ kärkinä ovat pisteet $O=(0,0,0)$, $A=(1,0,0)$, $B=(1,1,-1)$ ja $C=(0,2,2)$.
Määritä tetraedrin tilavuus $V$.
\end{Exa}
\ratk Tetraedrin tilavuuden tunnettu laskukaava on
\[
V = \frac{1}{3}\,\cdot\, \text{pohjan ala}\,\cdot\,\text{korkeus}.
\]
Jos tässä pohjaksi tulkitaan kolmio $ABC$, niin
\[
V = \frac{1}{6}\abs{\Vect{AO}\cdot\Vect{AB}\times\Vect{AC}},
\]
sillä
\[
\text{pohjan ala} = \frac{1}{2}\abs{\Vect{AB} \times \Vect{AC}}, 
\quad \text{korkeus} = \abs{\Vect{AO'}} = \abs{\Vect{AO}}\abs{\cos\kulma OAO'},
\]
missä $\Vect{AO'}$ on vektorin $\Vect{AO}$ kohtisuora projektio vektorin 
$\Vect{AB} \times \Vect{AC}$ suuntaan, ks.\ kuvio.
\begin{figure}[H]
\setlength{\unitlength}{1cm}
\begin{center}
\begin{picture}(8,4)(-0.1,-3)
\put(0,0){\line(1,-1){3}} \put(0,0){\line(2,-1){7}} \put(0,0){\line(4,-1){8}}
\dashline{0.2}(3,-3)(8,-2)
\path(3,-3)(7,-3.5)(8,-2)
\put(3,-3){\vector(0,1){4}}
\dashline{0.2}(0,0)(3,0)
\put(-0.1,0.2){$O$} \put(2.9,-3.5){$A$} \put(7.9,-1.8){$B$} \put(6.9,-4){$C$}
\put(3.2,-0.1){$O'$}
\put(3.2,0.8){$-\Vect{AB}\times\Vect{AC}$}
\path(2.8,0)(2.8,-0.2)(3,-0.2) \path(3,-2.6)(3.4,-2.5)(3.4,-2.9) 
\path(3,-2.8)(3.2,-2.825)(3.2,-3)
\end{picture}
\end{center}
\end{figure}
Koska
\[
\Vect{AO}\cdot\Vect{AB}\times\Vect{AC} =
\left| \begin{array}{rrr}
-1 & 0 &  0 \\
 0 & 1 & -1 \\
-1 & 2 &  2 
\end{array} \right|
=
(-1) \cdot \left| \begin{array}{rr}
1 & -1 \\
2 & 2
\end{array} \right| = -4,
\]
niin $\,V=\dfrac{1}{6}\cdot\abs{-4}=\dfrac{2}{3}$ \loppu

\subsection{Suunnikkaan ala. Suuntaissärmiön tilavuus}
\index{pinta-ala!b@avaruussuunnikkaan|vahv}
\index{tilavuus!b@suuntaissärmiön|vahv}

Suunnikas on tason nelikulmio, jonka vastakkaiset sivut ovat yhdensuuntaiset. 
Avaruudessa suunnikkaan kolmiulotteinen vastine on suuntaissärmiö eli $6$-tahokas, jonka
sivut ovat suunnikkaita (avaruustasoilla). Tämän erikoistapaus on suorakulmainen särmiö
(sivut suorakulmioita). Jos janat $OA$, $OB$ ja $OC$ ovat suuntaissärmiön kolme särmää,
niin sanotaan, että vektorit
\[
\vec a=\Vect{OA}, \quad \vec b=\Vect{OB}, \quad \vec c=\Vect{OC}
\]
\index{virittää (suunnikas, särmiö)}%
\kor{virittävät} ko.\ särmiön. Vastaavasti vektorit $\,\vec a,\vec b$, samoin vektorit
$\,\vec a,\vec c\,$ ja $\,\vec b,\vec c$, virittävät suunnikkaan (avaruustasolla).
Suuntaissärmiön tilavuuden laskukaava on: $\,V=$ pohjan ala $\cdot$ korkeus. Tässä pohja on
suunnikas, jonka pinta-ala saadaan ristitulon avulla (ala = kanta $\cdot$ korkeus), joten 
särmiön tilavuus saadaan skalaarikolmitulon avulla kuten tetraedrin tapauksessa. Laskukaavat,
kun virittäjävektorit tunnetaan:
\[
\boxed{ \begin{aligned}
\rule{0mm}{5mm}\quad &\text{Sunnikkaan pinta-ala} \qquad =\ \abs{\vec a \times \vec b}. \\
\akehys\quad &\text{Suuntaissärmiön tilavuus}\   
                   =\ \abs{\vec a \times \vec b \cdot \vec c\,}. \quad
        \end{aligned} }
\]

\subsection{Vektorikolmitulo $\vec a \times (\vec b \times \vec c\,)$}
\index{vektorikolmitulo|vahv}
\index{laskuoperaatiot!cc@avaruusvektoreiden|vahv}

Jos $T$ on taso, jonka suuntavektoreina ovat $\vec b, \vec c$ (ol. $\vec b \nparallel \vec c$),
niin vektori $\,\vec b \times \vec c\,$ on tason normaalivektori. Näin ollen 
$\vec a  \times(\vec b \times \vec c\,)$ on jälleen tason suuntainen, ts.
\[
\vec a  \times(\vec b \times \vec c\,) = \lambda \vec b + \mu \vec c
\]
joillakin $\lambda, \mu \in \R$. Toisaalta tämä vektori on myös $\vec a$:ta vasaan kohtisuora,
eli
\[
\vec a \cdot (\lambda \vec b + \mu \vec c\,) 
                     = \lambda (\vec a \cdot \vec b\,) + \mu(\vec a \cdot \vec c\,) = 0.
\]
Tämä toteutuu täsmälleen, kun $\lambda = \gamma\,\vec a \cdot \vec c$ ja 
$\mu = - \gamma\,\vec a \cdot \vec b$ jollakin $\gamma \in \R$, joten on päätelty, että
\[
\vec a \times (\vec b \times \vec c\,) 
             = \gamma\,[\,(\vec a \cdot \vec c\,)\vec b - (\vec a \cdot \vec b\,)\vec c\,\,].
\]
Tapauksessa $\,\vec a = \vec b = \vec i$, $\vec c = \vec j\,$ tämä toteutuu, kun $\gamma=1$,
joten päätellään, että vektorikolmitulolle pätee purkukaava\footnote[2]{Purkusääntöä tarkemmin
perusteltaessa olisi myös näytettävä, että  $\gamma$ ei riipu vektoreista $\vec a$, $\vec b$,
$\vec c$. Perustelut sivuutetaan tässä, sillä purkusääntö on näytettävissä oikeaksi myös
suoraan vektorien komponenttimuodosta.}
\[
\boxed{\kehys\quad \vec a \times (\vec b \times \vec c\,)
                    =(\vec a \cdot \vec c\,)\vec b - (\vec a \cdot \vec b\,)\vec c. \quad}
\]

\Harj
\begin{enumerate}

\item
Vektorin $\vec u$ koordinaatit kannassa $\{\vec i,\vec j,\vec k\}$ ovat $(x,y,z)$. Mitkä
ovat $\vec u\,$:n koordinaatit kannassa $\{\vec i+\vec j,\,\vec j+\vec k,\,\vec i+\vec k\}\,$?

\item
Laske seuraavien avaruusvektorien muodostamat kulmat $\kulma(\vec a,\vec b)$ asteina: \newline
a) \ $\vec a=\vec i,\ \vec b=\vec i+\vec j+\vec k \quad$
b) \ $\vec a=\vec i+\vec j,\ \vec b=\vec j+\vec k$

\item
a) Avaruusvektoreista $\vec a,\vec b,\vec c$ tiedetään, että $\abs{\vec a}=\abs{\vec b}=1$, 
$\abs{\vec c}=2$, $\vec a\perp\vec b$ ja $\kulma(\vec a,\vec c)=\kulma(\vec b,\vec c)=60\aste$.
Mikä on vektorin $\vec u=\vec a+\vec b+\vec c$ pituus\,? \vspace{1mm}\newline
b) Määritä yksikkövektorit, jotka muodostavat yhtä suuret kulmat vektoreiden $\vec k$, 
$\vec j+\vec k$ ja $\vec i+\vec j+\vec k$ kanssa. \vspace{1mm}\newline
c) Janasta $OP$ ($O=$ origo) tiedetään, että janan pituus on $5$ ja että jana muodostaa 
positiivisen $x$-akselin kanssa kulman $\alpha=32\aste$ ja positiivisen $y$-akselin kanssa
kulman $\beta=73\aste$. Laske $P$:n koordinaatit.

\item
a) Säännöllisen tetraedrin kärjestä lähtevät särmävektorit ovat $\vec a$, $\vec b$ ja $\vec c$.
Laske $\cos\kulma(\vec a,\vec b+\vec c)$ ja
$\cos\kulma(\vec a+\vec b+2\vec c,\,2\vec a-\vec b)$. \vspace{1mm}\newline
b) Pöydällä oleva A4-kokoinen paperiarkki on suorakulmio $ABCD$, jonka sivun pituuksien suhde
on $|AB|/|BC|=\sqrt{2}$. Arkki taitetaan pitkin lävistäjää $AC$ siten, että kolmio-osa
$ABC$ jää pöydälle ja osa $ACD$ kääntyy kolmioksi $ACD'$ pöydän pintaa vastaan vastaan
kohtisuoralle tasolle. Laske $\cos\kulma D'AB$.

\item
a) Määritä kaikki avaruusvektorit $\vec u$, jotka täyttävät ehdot (1) $\abs{\vec u}=1$, \newline
(2) $\vec u=\lambda(\vec i+\vec j)+\mu(\vec j+\vec k)$ jollakin $(\lambda,\mu)\in\Rkaksi$
ja (3) $\vec u \perp \vec i-2\vec j-\vec k$. \vspace{1mm}\newline
b) Jaa vektori $\vec u=\vec i+2\vec j-3\vec k$ kolmeen komponenttiin, joista yksi on vektorin
$\vec a=2\vec i+\vec j+\vec k$ suuntainen, toinen vektorin $\vec b=2\vec i+3\vec j-3\vec k$
suuntainen ja kolmas kohtisuorassa vektoreita $\vec a$ ja $\vec b$ vastaan. Mikä on luvun
$\sin\kulma(\vec u,\vec v\,)$ pienin arvo, kun $\vec v=\lambda\vec a + \mu\vec b$, 
$(\lambda,\mu)\in\Rkaksi$ $(\vec v\neq\vec 0\,)$\,? \vspace{1mm}\newline
c) Muodosta ortonormeerattu, oikeakätinen avaruusvektorien kanta $\{\vec a,\vec b,\vec c\,\}$ 
siten, että $\vec a$ on vektorin $\vec i+\vec j+\vec k$ suuntainen ja $\vec b$ on $xy$-tason 
suuntainen. Määräytyykö kanta yksikäsitteisesti näistä ehdoista?

\item \label{H-I-7:oktaedri}
Sijoita säännöllinen oktaedri (= kahdeksansivuinen monitahokas, jonka sivutahkot ovat 
tasasivuisia kolmioita) mukavaan asentoon koordinatistoon siten, että yksi kärki on origossa.
Laske tästä kärjestä alkavien särmien vektoriesitykset ja näiden avulla sivutahkojen 
normaalivektorit. Laske edelleen näiden avulla vierekkäisten sivutahkojen välinen 
\index{diedrikulma}%
\kor{diedrikulma}, eli kulma, jonka sivutahkot muodostavat yhteisen särmän suunnasta
katsottuna.

\item
Osoita: \ $\vec a+\vec b+\vec c=\vec 0\ \impl\ 
                       \vec a\times\vec b=\vec b\times\vec c=\vec c\times\vec a\,$.

\item
a) Pisteet $(-1,-2,4)$, $(5,-1,0)$, $(2,-3,6)$ ja $(1,-1,1)$ ovat sekä tetraedrin että
suuntiassärmiön kärkiä. Laske kummankin tilavuus ja pinnan ala. \vspace{1mm}\newline
 b) Tason kolmion kaksi kärkeä ovat pisteissä $(1,-2)$ ja $(3,3)$. Millaisessa $\Ekaksi$:n
pistejoukossa on kolmannen kärjen täsmälleen oltava, jotta kolmion pinta-ala $=10$\,? Kuva!

\item
Päättele, että avaruusvektorit $\vec a,\,\vec b,\,\vec c$ ovat lineaarisesti riippuvat
täsmälleen kun vektoreiden virittämän suuntaissärmiön tilavuus $=0$. Tutki tällä kriteerillä
ovatko seuraavat vektorisysteemit lineaarisesti riippumattomat: \vspace{1mm}\newline
a) \ $\{\vec i-\vec j+\vec k,\,3\vec i+2\vec j+\vec k,\,\vec i+\vec j-5\vec k\} \quad$
b) \ $\{\vec i-2\vec k,\,2\vec i-\vec j+3\vec k,\,5\vec i-2\vec j+4\vec k\}$

\item \label{H-II-8: vektorihajotelma}
Olkoon $\vec n$ avaruuden yksikkövektori. Halutaan esittää avaruusvektori $\vec F$ muodossa
$\vec F= \vec F_1 + \vec F_2$, missä $\vec F_1 || \vec n$ ja $\vec F_2 \perp \vec n$.
Näytä, että $\vec F_1 = (\vec F \cdot \vec n)\vec n$ ja 
$\vec F_2 = -\vec n\times(\vec n\times\vec F)$. 

\item
Todista: \newline
a) \ $\abs{\vec a\times\vec b}^2 = \abs{\vec a}^2\abs{\vec b}^2
                                   -(\vec a\cdot\vec b\,)^2$. \newline
b) \ $\abs{\vec a\times(\vec a\times \vec b\,)}^2
     =\abs{\vec a}^2\bigl[\abs{\vec a}^2\abs{\vec b}^2-(\vec a\cdot\vec b\,)^2\bigr]$. \newline
c) \ $(\vec a\times\vec b\,)\times(\vec b\times\vec c\,)\cdot(\vec c\times\vec a\,)\,
        =\,(\vec a\times\vec b\cdot\vec c\,)^2$.
                                              
\item
a) Vektorit $\vec a,\,\vec b,\,\vec c$ ovat lineaarisesti riippumattomat. Sievennä
lauseke \newline
$[(\vec a\times\vec b\,)\times(\vec b\times\vec c\,)]\times(\vec c\times\vec a\,)$. Millä 
ehdolla lauseke on $=\vec 0$\,? \vspace{1mm}\newline
b) Määritä vektorin $\,(\vec a\times(\vec a\times(\vec a\times(\vec a\times
(\vec a\times(\vec a\times\vec b\,))))))\,$ pituus, kun $\abs{\vec a}=3$, $\abs{\vec b}=1$ ja
$\vec a\cdot\vec b=-2$. \vspace{1mm}\newline
c) Määritä vektorit $\vec u$, jotka toteuttavat $\,\vec u\cdot\vec j=0\,$ ja
$\,\vec u\times(\vec k\times\vec u)=\vec k$.

\item (*) a) Näytä, että tetraedrin keskijanat, eli kärjen ja vastakkaisen sivun keskiön 
yhdysjanat, leikkaavat toisensa samassa pisteessä (= tetraedrin keskiö). Missä suhteessa 
keskijanat jakautuvat leikkauspisteessä? \newline
c) Anna kaksi esimerkkiä tetraedrista, jonka kaikilla neljällä korkeusjanalla on yhteinen piste.

\item (*)
Pisteet $A=(1,1,4)$, $B=(1,-1,3)$, $C=(-1,-1,2)$ ja $D=(0,-2,2)$ ovat eräällä avaruustasolla
$T$. Tutki, millainen ko.\ tason kuvio syntyy, kun pisteet yhdistetään mainitussa
järjestyksessä janoilla suljetuksi murtoviivaksi. Onko kyseessä nelikulmio? Valitse haluamasi 
kuvakulma (koordinaatisto) tasossa $T$ ja piirrä kuva!

\item (*)
Pisteet $(-3,-1,4)$, $(0,-1,-2)$, $(2,5,1)$, $(3,2,7)$ ja $(5,1,-2)$ ovat $5$-tahokkaan $K$
kärkipisteet. Laske $K$:n tilavuus ja pinnan ala.
%Kaikki paitsi kolmas piste ovat tasolla $2x-5y+z=3$ eli kyseessä on pyramidi.

\end{enumerate} % Avaruuden vektorit
\section{Suorien, tasojen ja pintojen geometriaa} \label{suorat ja tasot}
\sectionmark{Suorat, tasot ja pinnat}
\alku

Kun geometrian tehtäviä ratkotaan algebran keinoin käyttäen hyväksi (yleensä karteesista) 
koordinaatistoa, puhutaan
\index{analyyttinen geometria}%
\kor{analyyttisestä geometriasta}. Vektorit, skalaari- ja 
ristituloineen, tarjoavat moniin tehtäviin kätevän apuneuvon.

Jos $O$ on Euklidisen avaruuden ($\Ekaksi$ tai $\Ekolme$) origo ja $P$ on ko.\ avaruuden 
piste, sanotaan vektoria $\overrightarrow{OP}$ tästedes pisteen $P$
\index{paikkavektori}%
\kor{paikkavektoriksi} ja merkitään
\[
\vec r = \overrightarrow{OP} \quad \text{(paikkavektori)}.
\]
Käytetään myös vastaavuusmerkintää
\[ 
\vec r \vastaa P \ \ekv \ \vec r = \overrightarrow{OP}.
\]
\subsection{Suora}
\index{suora|vahv}

Jos $P_0=(x_0,y_0) \vastaa \vec r_0 = x_0\vec i + y_0\vec j$ on euklidisen tason piste ja 
$\vec v=\alpha\vec i + \beta\vec j$ on tason vektori, $\vec v \neq \vec 0$, niin pistejoukko
\index{parametri(sointi)!a@suoran|(}%
\[
S=\{P \in \Ekaksi \ | \ P \vastaa \vec r = \vec r_0 + t\vec v \quad \text{jollakin} \ t \in \R\}
\]
on suora, jonka
\index{suuntavektori}%
\kor{suuntavektori} $=\vec v$ \ ja joka kulkee pisteen $P_0$ kautta.
\begin{figure}[H]
\setlength{\unitlength}{1cm}
\begin{center}
\begin{picture}(9.5,5.5)(-1,-0.5)
\put(0,0){\vector(0,1){2}} \put(0,0){\vector(2,1){8}}
\Thicklines
\put(0,2){\vector(4,1){3}}
\thinlines
\path(-1,1.75)(9,4.25)
\put(-0.1,1.9){$\bullet$} \put(-0.1,-0.1){$\bullet$} \put(7.9,3.9){$\bullet$}
\put(-0.1,-0.5){$O$} \put(0.2,1.6){$\vec r_0$} \put(-0.1,2.2){$P_0$} \put(2.7,2.9){$\vec v$} 
\put(8.2,3.5){$P$}
\put(7.5,3.2){$\vec r$}
\end{picture}
\end{center}
\end{figure}
Sanotaan, että yhtälö
\begin{equation} \label{suora par}
\vec r = \vec r_0 + t\vec v
\end{equation}
on suoran $S$ \kor{parametrimuotoinen yhtälö} tai \kor{parametrisointi} (parametrisaatio) ja
että $t$ on \kor{parametri}. Yleisemmin näillä termeillä tarkoitetaan, että 
\index{parametri(sointi)!a@suoran|)}%
tarkasteltavan pistejoukon (tässä suoran) jokainen piste saadaan annetusta esitysmuodosta
jollakin parametrin arvolla. Tässä tapauksessa jokaista suoran pistettä vastaa yksikäsitteinen 
parametrin $t$ arvo ja kääntäen, ts.\ parametrisointi \eqref{suora par} luo kääntäen 
yksikäsitteisen vastaavuuden $S\ \vast\ \R$.

Kun kirjoitetaan $\vec r = x \vec i + y \vec j$, niin yhtälölle \eqref{suora par} saadaan 
\kor{koordinaattimuoto}
\begin{equation} \label{suora komp taso}
\left\{ \begin{array}{ll}
x=x_0 + \alpha t, \\
y=y_0 + \beta t.
\end{array} \right.
\end{equation}
Eliminoimalla $t$ seuraa tästä $x$:n ja $y$:n välinen riippuvuus
\begin{equation} \label{suora perus}
ax+by+c=0,
\end{equation}
missä $a=\beta$, $b=-\alpha$ ja $c=-\beta x_0+\alpha y_0$. Tämä on suoran yhtälön
\index{perusmuoto!a@tason suoran yhtälön}%
\kor{perusmuoto} (ei-parametrinen muoto). Perusmuotoisen yhtälön \eqref{suora perus}
voi ilmaista myös vektoreiden avulla, sillä kun asetetaan
\[
\vec n = a \vec i + b \vec j \ \ (= \beta\vec i-\alpha\vec j\,),
\]
niin yhtälö on sama kuin
\begin{equation} \label{suora norm}
\vec n \cdot (\vec r - \vec r_0) = 0.
\end{equation}
Tässä $\vec n$ on $S$:n
\index{normaali(vektori)!a@suoran, käyrän}%
\kor{normaalivektori}, ts.\ $S$:n suuntavektoria $\vec v$ vastaan
kohtisuora ($\vec 0$:sta poikkeava) vektori. Normaalivektorin avulla suoran yhtälö voidaan
siis kirjoittaa suoraan ehtona $\vec r - \vec r_0 \perp \vec n$, kuten geometrisestikin
on helppo päätellä.
%\begin{figure}[H]
%\begin{center}
%\import{kuvat/}{kuvaII-17.pstex_t}
%\end{center}
%\end{figure}
\begin{figure}[H]
\setlength{\unitlength}{1cm}
\begin{center}
\begin{picture}(10,6)(-1,-0.5)
\put(0,0){\vector(1,2){2}} \put(0,0){\vector(2,1){6}} \put(2,4){\vector(1,4){0.3}}
\put(-1,4.75){\line(4,-1){10}} \put(2,4){\vector(4,-1){4}}
\path(1.84,4.04)(1.88,4.2)(2.04,4.16)
\put(5.8,3.2){$\vec r-\vec r_0$}
\put(-0.1,-0.5){$O$} \put(9.3,2.1){$S$}
\put(2,3.55){$\vec r_0$} \put(2.5,5){$\vec n$} \put(5.95,2.55){$\vec r$} 
\end{picture}
\end{center}
\end{figure}
\begin{Exa}
Suoran yhtälöstä $3x-2y+5=0$ nähdään heti, että suoran normaalivektori on
$\vec n=3\vec i -2\vec j$. Siis suoran eräs suuntavektori on $\vec v = 2\vec i + 3\vec j$. 
Koska esimerkiksi $P_0=(1,4)$ on suoralla, saadaan suoralle parametrisoitu esitys
\eqref{suora par}, missä
\[
\vec r_0 = \vec i + 4\vec j, \quad \vec v = 2\vec i + 3 \vec j.
\]
Parametrisoinnin koordinaattimuoto \eqref{suora komp taso} on tässä tapauksessa
\[
\left\{ \begin{array}{ll}
x=1+2t, \\
y=4+3t.
\end{array}\right. 
\]
Mainittu piste $P_0$ vastaa siis tässä parametrin arvoa $t=0$. \loppu
\end{Exa}
Parametrisointi \eqref{suora par} toimii myös avaruussuoralle. Jos
$\vec r_0=x_0\vec i + y_0 \vec j + z_0 \vec k$ ja 
$\vec v = \alpha \vec i + \beta \vec j + \gamma \vec k$, niin koordinaattimuotoiset
suoran yhtälöt ovat
\begin{equation} \label{suora komp avaruus}
\left\{ \begin{array}{ll}
x = x_0 + \alpha t, \\ 
y = y_0 + \beta t, \\
z = z_0 + \gamma t.
\end{array} \right.
\end{equation}
Eliminoimalla $t$ jää jäljelle kaksi ei-parametrista yhtälöä.
\begin{Exa}
Suora kulkee pisteiden $A=(1,1,1)$ ja $B=(1,-2,3)$ kautta. Mikä on suoran yhtälö?
\end{Exa}
\ratk Eräs suoran suuntavektori on
\[
\vec v = \overrightarrow{AB} = -3\vec j + 2\vec k
\]
ja siis eräs parametriesitys on muotoa \eqref{suora par}, missä esimerkiksi
\[
\vec r_0 = \overrightarrow{OA} = \vec i + \vec j + \vec k.
\]
Koordinaattimuodossa
\[
\left\{ \begin{array}{ll}
x=1, \\
y=1-3t, \\
z=1+2t.
\end{array} \right.
\]
Eliminoimalla $t$ saadaan saadaan suoran ei-parametrisiksi yhtälöiksi
\[
\left\{ \begin{array}{ll}
x &= 1, \\
2y+3z &=5.
\end{array} \right.
\]
Nämä voidaan kirjoittaa myös muotoon
\[
\left\{ \begin{array}{ll}
\vec n_1 \cdot (\vec r - \vec r_0)=0, \\
\vec n_2 \cdot (\vec r - \vec r_0)=0,
\end{array} \right.
\]
missä $\vec n_1 = \vec i$ ja $\vec n_2=2\vec j + 3\vec k$ ovat suoran normaalivektoreita.
Tämä vastaa tason suoran yhtälöä \eqref{suora norm}. \loppu
%\begin{figure}[H]
%\begin{center}
%\import{kuvat/}{kuvaII-18.pstex_t}
%\end{center}
%\end{figure}
\begin{figure}[H]
\setlength{\unitlength}{1cm}
\begin{center}
\begin{picture}(10,5.5)(-1,0)
\put(0,0){\vector(1,2){2}} \put(0,0){\vector(2,1){6}} 
\put(2,4){\vector(1,4){0.3}} \put(2,4){\vector(-2,-1){1}}
\put(-2,5){\line(4,-1){11}} \put(2,4){\vector(4,-1){4}}
\path(1.84,4.04)(1.88,4.2)(2.04,4.16)
\path(1.84,4.04)(1.72,3.98)(1.88,3.94)
\Thicklines 
\put(1,4.25){\vector(-4,1){2}} 
\thinlines
\put(0.93,4.18){$\scriptstyle{\bullet}$} \put(-1.07,4.68){$\scriptstyle{\bullet}$}
\put(1,4.4){$A$} \put(-1,4.9){$B$} \put(-0.95,4.25){$\vec v$} 
\put(5.8,3.2){$\vec r-\vec r_0$}
\put(-0.1,-0.5){$O$} \put(9.3,2.1){$S$}
\put(2,3.55){$\vec r_0$} \put(5.95,2.55){$\vec r$} 
\put(2.5,5){$\vec n_1$} \put(0.9,3.1){$\vec n_2$}
\end{picture}
\end{center}
\end{figure}
\begin{Exa}
Hävittäjä A nousee kentältä, joka on pisteessä $(5,7,0)$ (yksikkö=km), suuntaan 
$-\vec i - 2\vec j + \vec k$ ja hävittäjä B nousee toiselta kentältä, joka on pisteessä 
$(1,-9,0)$, suuntaan $-\vec i + \vec j + 2\vec k$. Kuinka lähelle toisiaan koneet voivat joutua
(pahin mahdollinen skenaario)\,?
\end{Exa}
\ratk Tässä on kyse nk. kahden suoran ongelmasta, jossa on etsittävä suorien lyhin etäisyys $d$.
Suorat ovat
\begin{align*}
S_1: \ \vec r &= 5\vec i + 7\vec j + t(-\vec i - 2\vec j + \vec k) =\vec r_1+t\vec v_1, \\
S_2: \ \vec r &= \vec i - 9\vec j + s(-\vec i + \vec j + 2\vec k)\ =\vec r_2+s\vec v_2.
\end{align*}
Lyhin etäisyys on $d=\abs{\overrightarrow{PQ}}$, missä $P \in S_1$, $Q \in S_2$ ja 
vektori $\Vect{PQ}$ on molempia suoria vastaan kohtisuora, ts. 
\[
\Vect{PQ}\cdot\vec v_1=\Vect{PQ}\cdot\vec v_2=0.
\]
Kun tässä $P \vastaa \vec r(t)$ ja $Q \vastaa \vec r(s)$ esitetään parametrisointien
mukaisesti, niin
\begin{align*}
\Vect{PQ} &= \vec r_2-\vec r_1+s\vec v_2-t\vec v_1 \\
                    &= (-4+t-s)\vec i + (-16+2t+s)\vec j + (-t+2s)\vec k,
\end{align*}
joten saadaan yhtälöryhmä
\begin{align*} 
&\begin{cases}
\,-1\cdot(-4+t-s)-2\cdot(-16+2t+s)+1\cdot(-t+2s) = 0 \\
\,-1\cdot(-4+t-s)+1\cdot(-16+2t+s)+2\cdot(-t+2s) = 0
\end{cases} \\
&\quad\qekv \begin{cases}
             \,6t-s = 36 \\ \,t-6s = -12
            \end{cases}
\end{align*}
Ratkaisemalla $t,s$ saadaan
\[
t=\frac{228}{35}\,, \quad s=\frac{108}{35}\,.
\]
Näillä arvoilla on
\[
\overrightarrow{PQ} = \frac{1}{35}(-20\vec i+4\vec j-12\vec k).
\]
Koska komponentin $\vec k$ kerroin on negatiivinen, lentää kone A kohtaamistilanteessa 
korkeammalla. Koneiden etäisyys on siis pahimman skenaarion mukaan
\[
d=\abs{\overrightarrow{PQ}} = 4/ \sqrt{35} \approx 0.676 = 676 \text{ m},
\]
ja pisteet $P$ ja $Q$ saadaan erikseenkin suorien parametriesityksistä:
\begin{align*}
S_1:\ \vec r\left(t=\frac{228}{35}\right) &\vastaa P \approx (-1.514,-6.029,6.514), \\
S_2:\ \vec r\left(s=\frac{108}{35}\right) &\vastaa Q \approx (-2.086,-5.914,6.171). \loppu
\end{align*}

\subsection{Taso}
\index{taso|vahv}

Tarkastellaan avaruuden tasoa $T$, joka kulkee pisteiden $P_0=(x_0,y_0,z_0)$,
$P_1=(x_1,y_1,z_1)$ ja $P_2=(x_2,y_2,z_2)$ kautta. Oletetaan, että pisteet eivät ole
samalla suoralla, jolloin vektorit
\begin{align*}
\vec v_1 &= \overrightarrow{P_0P_1} = \alpha_1 \vec i + \beta_1 \vec j + \gamma_1 \vec k, \\
\vec v_2 &= \overrightarrow{P_0P_2} = \alpha_2 \vec i + \beta_2 \vec j + \gamma_2 \vec k
\end{align*}
ovat lineaarisesti riippumattomat. Näiden, tason $T$
\index{suuntavektori}%
\kor{suuntavektoreiden} avulla saadaan
tasolle parametrisoitu esitysmuoto
\begin{equation} \label{taso par}
\vec r = \vec r_0 + t_1 \vec v_1 + t_2 \vec v_2, \quad t_1, t_2 \in \R.
\end{equation}
\index{parametri(sointi)!b@tason}%
Tässä siis reaalisia parametreja on kaksi. Jos parametri halutaan nähdä 'yhtenä', niin
parametriksi on tulkittava $\mathbf{t}=(t_1,t_2)\in\Rkaksi$. Parametrisointi \eqref{taso par}
luokin kääntäen yksikäsitteisen vastaavuuden $T\vast\Rkaksi$, ja $\{P_0,\vec v_1,\vec v_2\}$
voidaan tulkita $T$:n koordinaatistoksi, vrt.\ Luku \ref{tasonvektorit}. 

Yhtälön \eqref{taso par} koordinaattimuoto on
\[
\left\{ \begin{array}{lll}
x &= x_0 + \alpha_1 t_1 + \alpha_2 t_2, \\
y &= y_0 + \beta_1 t_1 + \beta_2 t_2,   \\
z &= z_0 + \gamma_1 t_1 + \gamma_2 t_2.
\end{array} \right.
\]
Parametrien eliminointi käy kuitenkin helpommin vektorimuodosta \eqref{taso par}, kun otetaan 
käyttöön vektori
\[
\vec n = \vec v_1 \times \vec v_2.
\]
Tämä on molempia vectoreita $\vec v_1, \vec v_2$ vastaan kohtisuora, eli kyseessä on
tason
\index{normaali(vektori)!b@tason, hypertason}%
\kor{normaalivektori}. Kertomalla yhtälö \eqref{taso par} skalaarisesti vektorilla
$\vec n$ ja huomioimalla, että $\vec n \cdot \vec v_1 = \vec n \cdot \vec v_2 = 0$, seuraa
tasolle yhtälö
\begin{equation} \label{taso norm}
\vec n \cdot (\vec r - \vec r_0)=0,
\end{equation}
joka siis on aivan samaa muotoa kuin tason suoran yhtälö \eqref{suora norm}. Jos
\[
\vec n = a\vec i + b \vec j + c\vec k, \quad d=-\vec n \cdot \vec r_0\,,
\]
niin \eqref{taso norm} on edelleen sama kuin
\begin{equation} \label{taso perus}
ax+by+cz+d=0.
\end{equation}
\index{perusmuoto!b@avaruustason yhtälön}%
Tämä on tason yhtälön \kor{perusmuoto}. Perusmuodosta nähdään siis suoraan, mikä on tason
normaalivektori.\footnote[2]{Tason \kor{normaali} on jokainen avaruussuora, jonka suuntavektori
= tason normaalivektori. Termiä käytetään usein myös normaalivektorista puhuttaessa. Esim:
'Yksikkövektori $\vec n$ on tason normaali.'}
\begin{Exa}
Taso $T_1$ kulkee pisteiden $A=(1,1,2)$, $B=(0,3,-1)$ ja $C=(-2,-2,-3)$ kautta. Taso $T_2$ 
kulkee pisteen $(-1,-1,-1)$ kautta ja sen normaalivektori on 
$\vec n_2 = \vec i - \vec j - \vec k$. Määrää tasojen yhtälöt perusmuodossa \eqref{taso perus}
ja tasojen leikkaussuoran parametriesitys muodoissa \eqref{suora par} ja
\eqref{suora komp avaruus}.
\end{Exa}
\ratk Tason $T_2$ yhtälö on
\[
\vec n_2 \cdot (\vec r + \vec i + \vec j + \vec k) = 0,
\]
eli
\[
T_2: \quad x-y-z-1=0.
\]
Tason $T_1$ normaalivektori on
\[
\vec n_1 = \Vect{AB} \times \Vect{AC} =
\left| \begin{array}{ccc}
\vec i & \vec j & \vec k \\
-1 & 2 & -3 \\
-3 & -3 & -5
\end{array} \right| =
-19\vec i + 4\vec j + 9\vec k,
\]
joten valitsemalla esimerkiksi $\vec r_0 \vastaa A$ saadaan tason $T_1$ yhtälöksi
\[
T_1: \quad \vec n_1 \cdot (\vec r - \vec i - \vec j - 2\vec k) = 0 \qekv -19x+4y+9z-3=0.
\]
Leikkaussuoran $S$ suuntavektori on
\[
\vec v = \vec n_1 \times \vec n_2 =
\left| \begin{array}{ccc}
\vec i & \vec j & \vec k \\
-19 & 4 & 9 \\
1 & -1 & -1 
\end{array} \right| =
5 \vec i - 10 \vec j + 15 \vec k.
\]
Suuntavektoriksi kelpaa myös $\frac{1}{5}\vec v = \vec i - 2 \vec j + 3 \vec k$. Vielä tarvitaan
yksi suoran piste $P_0 \vastaa \vec r_0$. Tämä löydetään esim. tasojen $T_1, T_2$ ja (sopivasti
valitun) kolmannen tason $T_3$ leikkauspisteenä. Valitaan kolmanneksi tasoksi $yz$-taso
\[
T_3: \ x=0,
\]
jolloin $P_0=(x_0,y_0,z_0)$ ratkeaa (jos ratkeaa) yhtälöryhmästä
\[
\left\{ \begin{array}{ll}
-19x_0+4y_0+9z_0 &= \ 3 \\
x_0-y_0-z_0 &= \ 1 \\
x_0 &= \ 0
\end{array} \right.
\]
Ratkaisu on $P_0=(0,-\frac{12}{5},\frac{7}{5})$, ja suoran $S$ parametriesitys näin muodoin 
\[
\vec r = \frac{1}{5}(-12 \vec j + 7 \vec k) + t(\vec i - 2 \vec j + 3 \vec k),
\]
eli koordinaattimuodossa
\[
S: \quad \left\{ \begin{array}{ll}
x &= \ t, \\
y &= \ -\frac{12}{5}-2t, \\
z &= \ \frac{7}{5} + 3t.
\end{array} \right. \loppu
\]
\begin{Exa} \label{pisteen etäisyys tasosta} Määritä annetun pisteen $P=(x_0,y_0,z_0)$ etäisyys
tasosta \newline $T:\ ax+by+cz+d=0$.
\end{Exa}
\ratk Jos $Q$ on pistettä $P$ lähinnä oleva piste tasolla $T$, niin ilmeisesti jollakin $t\in\R$
on $\overrightarrow{PQ}=t\vec n$, missä $\vec n=a\vec i+b\vec j+c\vec k$ on tason 
normaalivektori. Siis $Q=(x_0+ta,y_0+tb,z_0+tc)$ jollakin $t$. Koska $Q \in T$, niin $T$:n
yhtälön perusteella
\[
0\,=\,a(x_0+ta)+b(y_0+tb)+c(y_0+tc)+d\,=\,ax_0+by_0+cz_0+d+t\abs{\vec n}^2.
\]
Ratkaisemalla tästä $t$ saadaan etäisyyden laskukaavaksi
\[
h\,=\,\abs{t}{\abs{\vec n}}\,=\,\frac{\abs{ax_0+by_0+cz_0+d}}{\sqrt{a^2+b^2+c^2}}\,.
\]
Tason pisteen $(x_0,y_0)$ etäisyys suorasta $S:\ ax+by+c=0$ saadaan samasta kaavasta 
asettamalla ensin $c=0$ ja sitten $d$:n tilalle $c$. \loppu

\subsection{Ympyrä ja pallo}
\index{ympyrä|vahv} \index{pallo(pinta)|vahv}

Ympyrä on (harpin olomuodossa) euklidisen tason alkuperäisiä olioita. Jos ympyrän keskipiste
on $P_0\vastaa\vec r_0$ ja säde $R$, niin \kor{ympyrän} (ympyräviivan) \kor{yhtälö} on
\[
\abs{\vec r-\vec r_0} = R.
\]
Tämä on myös \kor{pallon} (pallopinnan)\footnote[2]{Ympyrällä ja pallolla saatetaan tarkoittaa 
myös euklidisen avaruuden 'täyteistä' joukkoa
\[
K = \{P\vastaa\vec r\, \mid \abs{\vec r-\vec r_0} \le R\}.
\]
Tämän täsmällisempi nimitys tasossa on \kor{kiekko} (engl.\ disc; ympyrä = circle), avaruudessa
\kor{kuula} (engl.\ ball; pallopinta = sphere). \index{kiekko|av} \index{kuula|av}} yhtälö.
Neliöitynä ja koordinaattien avulla kirjoitettuna yhtälöt ovat
\begin{align*}
\text{Ympyrä:} \quad &(x-x_0)^2+(y-y_0)^2=R^2. \\
\text{Pallo:}  \quad &(x-x_0)^2+(y-y_0)^2+(z-z_0)^2=R^2.
\end{align*}
\index{tangentti (käyrän)}%
\kor{Ympyrän tangentti} on suora, jolla on ympyrän kanssa täsmälleen yksi yhteinen piste. 
Sanotaan, että tangentti \kor{sivuaa} ympyrää ko.\ pisteessä. Jos sivuamispiste on annettu
piste $Q$, niin tangentti konstruoidaan geometrisesti piirtämällä ensin suora $S_1$ pisteen
$Q$ ja ympyrän keskipisteen kautta, jolloin tangentti $S_2$ on tätä vastaan kohtisuora. 
Sanotaan, että suora $S_1$ on
\index{normaali(vektori)!a@suoran, käyrän}%
\kor{ympyrän normaali} pisteessä $Q$ tai että suora $S_1$
leikkaa ympyrän \kor{kohtisuorasti}. Yleisemminkin voidaan ympyräviivan ja suoran välinen 
\index{leikkauskulma}%
\kor{leikkauskulma} määritellä ko.\ suoran ja leikkauspisteeseen asetetun tangentin
suuntavektoreiden välisenä kulmana.

Vastaavalla tavalla kuin ympyrän tangentti määritellään
\index{tangenttitaso}%
\kor{pallon tangenttitaso}. Pallon keskipisteen kautta kulkeva suora on
\index{normaali(vektori)!c@pinnan}%
\kor{pallon normaali}, eli se on kohtisuora tangenttitasoa
vastaan pallon ja suoran leikkauspisteessä. Jos pallolla ja tasolla $T$ on enemmän kuin yksi 
yhteinen piste, on yhteisten pisteiden joukko
\index{avaruusympyrä}%
\kor{avaruusympyrä} (ympyrä tasolla $T$). Jos $T$
kulkee pallon keskipisteen kautta, sanotaan leikkausviivaa pallon
\index{isoympyrä}%
\kor{isoympyräksi}.
\kor{Avaruusympyrän tangentti} on ympyrän kanssa samassa tasossa oleva, ympyrää sivuava
avaruussuora.
\begin{Exa} Pallon keskipiste on $(x_0,y_0,z_0)$ ja säde $=R$. Millä ehdolla taso 
$T:\ ax+by+cz+d=0$ sivuaa palloa\,?
\end{Exa}
\ratk Taso $T$ sivuaa palloa täsmälleen, kun pallon keskipisteen etäisyys tasosta $=R$.
Esimerkin \ref{pisteen etäisyys tasosta} perusteella ehto on
\[
\abs{ax_0+by_0+cz_0+d\,} =R\sqrt{a^2+b^2+c^2}. \loppu
\]
\begin{Exa} Yhtälöryhmä
\[ \begin{cases}
\,x^2+y^2+y^2=R^2 \\ \,x+y-z=1
\end{cases} \]
määrittelee origokeskisen pallon ja tason leikkausviivan $S$. Koska tason etäisyys origosta on
$h=1/\sqrt{3}$ (Esimerkki \ref{pisteen etäisyys tasosta}), niin päätellään, että $S$ on 
avaruusympyrä jos $R>1/\sqrt{3}$, piste jos $R=1/\sqrt{3}$, ja tyhjä joukko jos $R<1/\sqrt{3}$.
Ensiksi mainitussa tapauksessa $S$:n keskipiste on origon kautta kulkevalla suoralla, jonka 
suuntavektori on tason normaalivektori $\vec n=\vec i+\vec j-\vec k$. \loppu
\end{Exa}

\subsection{Lieriö ja kartio}
\index{lieriö|vahv} \index{kartio|vahv}

Olkoon $S$ avaruussuora, joka kulkee pisteen $P_0 \vastaa \vec r_0$ kautta ja jonka suuntavektori
on yksikkövektori $\vec e$. Tällöin pisteen $P \vastaa \vec r$ etäisyys suorasta $S$ voidaan
laskea kaavalla
\[
d = \abs{\vec r-\vec r_0}\abs{\sin\kulma(\vec r-\vec r_0,\vec e\,)} 
  = \abs{(\vec r-\vec r_0)\times\vec e\,}.
\]
Pinta, jolla tämä etäisyys on vakio $=R$ ($R>0$) on nimeltään \kor{lieriö}, tarkemmin
\kor{ympyrälieriö}. Suora $S$ on lieriön \kor{akseli} ja $R=$ lieriön \kor{säde}. Asettamalla
kaavassa $d=R$ ja neliöimällä saadaan lieriön yhtälölle muoto
\begin{equation} \label{lieriön y}
\abs{(\vec r-\vec r_0)\times\vec e\,}^2 = R^2.
\end{equation}
\begin{figure}[H]
\setlength{\unitlength}{1cm}
\begin{picture}(14,5)(-4,-1.5)
\put(0,-1.118){\line(2,1){6}} \put(-0.894,0.671){\line(2,1){6}}
\put(0,0){\vector(2,1){2}} \put(0,0){\vector(4,1){4.472}}
\dashline{0.1}(2.2,1.1)(4.6,2.3) \put(3,1.8){$S$}
\put(4.5,0.8){$\vec r-\vec r_0$}
\put(0,-0.5){$P_0$} \put(-0.07,-0.07){$\scriptstyle{\bullet}$}
\put(1.7,1.1){$\vec e$}
\put(0,0){\vector(-1,1){0.61}} \put(-0.4,0.5){$R$}
\renewcommand{\xscale}{.447} \renewcommand{\yscale}{.894}
\renewcommand{\xscaley}{-.447} \renewcommand{\yscalex}{.223}
\multiput(0,0)(4.6,2.3){2}{
\scaleput(0,0){
\curve(
 0.0,    -1.0,
-0.342, -0.940,
-0.643, -0.766,
-0.866, -0.5,
-0.940, -0.342,
-1.0,    0.0,
-0.940,  0.342,
-0.866,  0.5,
-0.643,  0.766,
-0.342,  0.940,
 0.0,    1.0)}
\curvedashes[1mm]{0,1,2}
\scaleput(0,0){
\curve(
0.0,   -1.0,
0.342, -0.940,
0.643, -0.766,
0.866, -0.5,
0.940, -0.342,
1.0,    0.0,
0.940,  0.342,
0.866,  0.5,
0.643,  0.766,
0.342,  0.940,
0.0,    1.0)}}
\end{picture}
\end{figure}
\begin{Exa} Olkoon lieriön akseli origon kautta kulkeva, yksikkövektorin
$\vec e=\frac{1}{3}(\vec i+2\vec j-2\vec k)$ suuntainen suora ja säde $R=2$. Tällöin on
$\vec r_0=\vec 0$ ja 
\[
\vec r\times\vec e\,=\,\frac{1}{3}\left|\begin{array}{rrr}
                                        \vec i&\ \vec j&\vec k\\x&\ y&z\\1&\ 2&-2
                                        \end{array}\right|
                  \,=\,\frac{1}{3}\left[(-2y-2z)\vec i+(2x+z)\vec j+(2x-y)\vec k\right],
\]
joten lieriön yhtälö on
\begin{align*}
&\qquad \frac{1}{9}[(2y+2z)^2+(2x+z)^2+(2x-y)^2]\,=\,4 \\[1mm]
&\ekv\quad 8x^2+5y^2+5z^2-4xy+4xz+8yz=36. \loppu
\end{align*}
\end{Exa}

Jos yhtälössä \eqref{lieriön y} korvataan ristitulo pistetulolla ja kirjoitetaan $R$:n tilalle
$\gamma\abs{\vec r -\vec r_0}$, missä $\gamma\in(0,1)$, niin yhtälö saa muodon
\begin{equation} \label{kartion y}
\abs{(\vec r-\vec r_0)\cdot\vec e\,}^2 = \gamma^2\abs{\vec r -\vec r_0}^2 \qekv
\abs{\cos\kulma(\vec r-\vec r_0,\vec e\,)}=\gamma.
\end{equation}
Tämä yhtälö määrittelee \kor{kartion}, tarkemmin \kor{ympyräkartion}. Piste 
$P_0\vastaa \vec r_0$ on kartion \kor{kärki}. Kartio koostuu kahdesta
\index{puolikartio}%
\kor{puolikartiosta}, joiden yhtälöt ovat
\[
\cos\kulma(\vec r-\vec r_0,\vec e\,)=\pm\gamma.
\]
\begin{figure}[H]
\setlength{\unitlength}{1cm}
\begin{picture}(14,4)(-4,-0.5)
\put(0,0){\vector(2,1){2}} \put(0,0){\vector(4,1){3.5}}
\put(3.5,0.4){$\vec r-\vec r_0$}
\put(0,-0.5){$P_0$} \put(-0.07,-0.07){$\scriptstyle{\bullet}$} 
\put(1.7,1.1){$\vec e$}
\path(-1,-0.813)(4,3.26) 
\put(0,0){\line(-4,-1){1.3}} \put(3.5,0.875){\line(4,1){1.5}}
\renewcommand{\xscale}{.447} \renewcommand{\yscale}{.894}
\renewcommand{\xscaley}{-.447} \renewcommand{\yscalex}{.223}
\put(4.025,2.012){
\scaleput(0,0){
\curve(
 0.0,    -1.0,
-0.342, -0.940,
-0.643, -0.766,
-0.866, -0.5,
-0.940, -0.342,
-1.0,    0.0,
-0.940,  0.342,
-0.866,  0.5,
-0.643,  0.766,
-0.342,  0.940,
 0.0,    1.0)}
\curvedashes[1mm]{0,1,2}
\scaleput(0,0){
\curve(
0.0,   -1.0,
0.342, -0.940,
0.643, -0.766,
0.866, -0.5,
0.940, -0.342,
1.0,    0.0,
0.940,  0.342,
0.866,  0.5,
0.643,  0.766,
0.342,  0.940,
0.0,    1.0)}}
\end{picture}
\end{figure}
Koordinaattien $x,y,z$ avulla esitettynä yhtälöt \eqref{lieriön y} ja \eqref{kartion y}
voidaan kumpikin kirjoittaa muotoon
\[
Ax^2+By^2+Cz^2+Dxy+Exz+Fyz+Gx+Hy+Iz+J=0,
\]
missä $A,B,$ jne.\ ovat reaalilukuja (vrt.\ esimerkki edellä). Tämä merkitsee, että lieriö ja 
kartio, myös pallo, ovat nk.\ 
\index{toisen asteen pinta}%
\kor{toisen asteen pintoja}. Toisen asteen pinnan ja avaruustason 
(esim.\ koordinaattitason) leikkausviiva on ko.\ tason
\index{toisen asteen käyrä}%
\kor{toisen asteen käyrä}. Tällainen on
esimerkiksi ympyrä. Yleinen toisen asteen käyrän yhtälö $xy$-tasolla on muotoa
\[
Ax^2+By^2+Cxy+Dx+Ey+F=0.
\]
Toisen asteen käyrien ja pintojen yleisempi luokittelu on matemaattinen ongelma, johon palataan
myöhemmin toisessa asiayhteydessä.

\pagebreak
\Harj
\begin{enumerate}

\item
a) Näytä, että yhtälöt esittävät samaa suoraa:
\[
\begin{cases} \,x=3+2t, \\ \,y=-11-6t \end{cases} \quad \text{ja} \quad
\begin{cases} \,x=-1-t, \\ \,y=1+3t \end{cases}
\]
b) Määritä suorien leikkauspiste:
\[
\begin{cases} \,x=3+2t, \\ \,y=-1-3t \end{cases} \quad \text{ja} \quad
\begin{cases} \,x=-1-t, \\ \,y=2(1+t) \end{cases}
\]
c) Määritä suoran $\,x=3+t,\ y=-2-t,\ z=4-2t\,$ ja koordinaattitasojen
leikkauspisteet.

\item
Määritä $\alpha$ siten, että vektori $\,\vec i+2\vec j+\alpha\vec k$ sekä avaruussuorat
\[
2(x-1)=1-y=2z-3 \quad \text{ja} \quad \begin{cases} x=17,\\y=7+3t,\\z=2t \end{cases}
\]
ovat saman tason suuntaiset.

\item
Määritä pisteet $P_1 \in S_1$ ja $P_2 \in S_2$ suorilla $\,S_1:\ x=-y=z$ ja 
$S_2:\ x+y-1=0,\ z=0$ siten, että vektori $\Vect{P_1P_2}$ on yhdensuuntainen vektorin
$2\vec i-\vec j-\vec k$ kanssa.

\item
Suora $S$ kulkee pisteen $(-1,1,3)$ ja sen janan keskipisteen kautta, jonka $xy$- ja $xz$-tasot
leikkaavat suorasta $\,x-1=2(y+1)=z+3$. Määritä $S$:n suuntavektori.

\item
Määritä $\alpha$ siten, että suorat $\,2(x-1)=y+1=2\alpha(z-1)$ ja $x+1=y-1=z$ leikkaavat.
Mikä on suorien lyhin etäisyys, jos $\alpha=1$\,?

\item
Määritä suorien
\[
\vec r=2\vec i+5\vec k+t(\vec i-2\vec j+2\vec k) \quad \text{ja} \quad
\begin{cases} 3x-2y=12\\x+2z=6 \end{cases}
\]
lyhin etäisyys ja lähinnä toisiaan olevat pisteet.

\item
Määritä sen suoran parametrimuotoinen yhtälö, joka leikkaa kohtisuorasti suorat
\[
\begin{cases} x=1+t, \\ y=1-2t, \\ z=t \end{cases} \quad \text{ja} \quad
\begin{cases} x=-1, \\ y=2s, \\ z=1-2s. \end{cases}
\]

\item
Laske pisteen $(2,3,-1)$ etäisyys suorasta
\[
\text{a)}\,\ \vec r=\vec i+2\vec j+3\vec k+t(\vec i-2\vec j+2\vec k) \qquad
\text{b)}\,\ \begin{cases} 3x-y+z=0\\x-2y+8=0 \end{cases}
\]

\item
a) Esitä tason $T$ yhtälö perusmuodossa $\,ax+by+cz+d=0$, kun parametrimuotoiset
yhtälöt ovat
\[ 
T:\ \begin{cases}
     x=2+2t_1+t_2, \\ y=-1+3t_1+2t_2, \\ z=3-t_1+t_2.
     \end{cases}
\]
b) Taso kulkee pisteen $P=(1,-13,-5)$ kautta ja sen suuntavektorit ovat
$\vec v_1=\vec i-\vec j$ ja $\vec v_2=\vec i+\vec j+\vec k$. Onko piste $Q=(3,-1,2)$
tasossa?

\item
Määritä tason yhtälö (perusmuoto!), kun tiedetään, että taso sisältää suoran
$x=3+t,\ y=1-2t,\ z=-2+t$ ja a) kulkee pisteen (0,2,1) kautta, b) on vektorin
$3\vec i+\vec j-2\vec k$ suuntainen

\item
Määritä seuraavien tasojen yhteiset pisteet: \newline
a) \ $x+y-z+2=0,\ 2x+y+2z-4=0,\ x-y+3z-2=0$ \newline
b) \ $x+y+z-6=0,\ x+2y-z-2=0,\  x+4y-5z+5=0$ \newline
c) \ $x+2y-2z-1=0,\ x-y+z-2=0,\ x+5y-5z=0$

\item Taso sisältää suoran $S_1:\ \vec r=(1+t)\vec i+(1+2t)\vec j+(1+3t)\vec k$ ja on suoran
$S_2:\ \vec r=(1+t)\vec i+(-1+t)\vec j+\vec k$ suuntainen. Johda tason yhtälö perusmuodossa.
Johda samoin sen tason yhtälö, joka sisältää suoran $S_2$ ja on $S_1$:n suuntainen.

\item
Johda sen tason yhtälö, joka puolittaa pisteen $P_0=(x_0,y_0,z_0)$ ja tason $T:\ ax+by+cz+d=0$
väliset janat.

\item
Määritä pisteen $(3,4,-2)$ kohtisuora projektio tasolla, jonka normaalivektori on
$\vec i-2\vec j+\vec k$ ja joka kulkee pisteen $(1,1,1)$ kautta.

\item
Määritä pisteen $(3,2,-4)$ peilikuvapiste tason $\,x+y-2z+5=0$ suhteen.

\item 
Määritä origon suurin mahdollinen etäisyys pisteiden $(0,1,0)$ ja $(2,2,-1)$ kautta
kulkevasta tasosta. Mikä taso antaa maksimietäisyyden?

\item
Määritä tasojen $\,x+y+z=3$ ja $3x-2y-z=1$ leikkaussuoran kautta kulkevat tasot, jotka
puolittavat tasojen välisen kulman.

\item
Kheopsin pyramidin (alkuperäinen) korkeus on $147$ m ja neliön muotoisen 
poh\-jan sivun pituus $230$ m. Sijoita pyramidi koordinaatistoon niin, että se
tuntee olonsa mahdollisimman mukavaksi, ja määritä tässä koordinaatistossa pyramidin
sivutasojen yhtälöt, mittayksikkönä $100$ m. Laske myös vierekkäisten sivujen välinen 
diedrikulma (vrt.\ Harj.teht \ref{ristitulo}:\ref{H-I-7:oktaedri}).

\item
Vektorin $2\vec i-\vec j+3\vec k$ suuntaan kulkeva valonsäde heijastuu tasosta $T$ pisteessä
$(1,2,-1)$. Heijastunut säde kulkee pisteen $(2,5,-3)$ kautta. Mikä on tason yhtälö?

\item
Positiivisen $z$-akselin suunnasta tuleva valonsäde osuu pisteessä $(1,2,3)$ tasolla
$3x+2y+z=10$ olevaan peiliin. Määritä heijastuneen säteen suunta. Missä pisteessä heijastunut
säde leikkaa (jos leikkaa) $xy$-tason?

\item
Painovoima vaikuttaa negatiivisen $z$-akselin suuntaan. Pisara putoaa pisteestä $(1,1,3)$ 
tasolle $\,3x-4y+12z=12$ ja alkaa valua tasoa pitkin alaspäin. Missä pisteessä pisara kohtaa
$xy$-tason?

\item
Tasangolta $z=0$ kohoaa vuorenrinne pitkin tasoa $x+2y+4z=0$. Rinteen pisteestä $P$, joka on 
korkeudella $h=10$, lähtee liikkeelle pistemäinen lumivyöry. Se etenee suoraviivaisesti 
rinnettä alas painovoimalakien mukaisesti. Tasangolle saavuttuaan se jatkaa suoraviivaista
liikettään vaakasuoraan, kunnes osuu mökkiin, joka on pisteessä $Q=(10,40,0)$.
Mikä oli piste $P$?

\item \index{zzb@\nim!Haukka ja kaksi kanaa} 
(Haukka ja kaksi kanaa) Universaalikoordinaatistossa maan pinta on 
taso $x+2y-3z=0$. Maan pinnalla käyskentelee kaksi pistemäistä kanaa $A$ ja 
$B$. Kanoja vaanii pistemäinen haukka, joka lentää maan pinnan suuntaisella
tasolla korkeudella $h=5$. Pistemäinen aurinko loistaa suunnassa
$-4\vec i +\vec k $. Hetkellä $H$ tapahtuu seuraavaa: Kanalta $A$ pääsee säikähtynyt 
'kot' sen huomatessa haukan varjon päällään. Kotkotuksen kuulee kana $B$,
joka vilkaisee samassa taivaalle ja näkee haukan suunnassa $\vec i -\vec j +\vec k $.
Määritä vektori $\Vect{AB}$ kyseisellä hetkellä $H$.

\item
Missä kulmassa tason suora $y=4x$ leikkaa ympyräviivan \newline
$x^2+y^2-2x-4y+4=0$\,?

\item
Avaruuskolmion kärjet ovat $(0,0,0)$, $(3,2,1)$ ja $(2,-1,3)$. Laske kolmion sisään piirretyn
(eli kaikkia sivuja sivuavan) ympyrän keskipiste ja säde.

\item
Avaruuden $E^3$ kolmen pisteen paikkavektorit ovat $\vec a$, $\vec b$ ja $\vec c$. Esitä
menettely, jolla voidaan määrittää pisteiden kautta kulkevan ympyrän keskipisteen paikkavektori.
Sovella menettelyä, kun pisteet ovat $(1,2,3)$, $(2,-5,3)$ ja $(-1,3,-6)$. Määritä myös
ympyrän säde ja ympyrän tason normaalivektori.

\item
Määritä avaruusympyrän
\[ \begin{cases}
\,x^2+y^2+z^2=49 \\ \,x+2y-z=10
\end{cases} \]
tangentti pisteessä $(-2,3,-6)$.

\item
Lieriön säde on $R=2$ ja akseli on pisteen $P_0=(1,1,1)$ kautta kulkeva suora, jonka 
suuntavektori on $-\vec i+\vec j+\vec k$. Määritä lieriön yhtälö toisen asteen pinnan yhtälön
perusmuodossa. Mitkä ovat lieriön ja suoran $x=y=z$ leikkauspisteet, ja mikä lieriön piste on
lähinnä origoa?

\item
Lieriöllä ja kartiolla on yhteisenä akselina suora $S:\ x=2y=-2z$ ja kumpikin kulkee pisteen
$(1,0,0)$ kautta. Kartion kärkenä on piste $(-2,-1,1)$. Laske lieriön säde ja kartion
(sivuprofiilin) aukeamiskulma sekä saata kummankin pinnan yhtälöt toisen asteen pinnan
yhtälön perusmuotoon. Määritä edelleen molempien pintojen ja $xy$-tason leikkauskäyrien
yhtälöt ja hahmottele näiden käyrien muoto.

\item (*)
Kuution, jonka sivun pituus $=4$, yksi kärki on origossa ja kolme muuta positiivisilla
koordinaattiakseleilla. Kuutiota katsotaan kaukaa vektorin $\vec i+2\vec j+3\vec k$
osoittamasta suunnasta kuvakulmassa, jossa $z$-akseli näkyy pystysuorana. Laske, millaisena
kuutio näkyy tästä kuvakulmasta. Kuva!

\item (*)
Millä tavoin saadaan selville avaruustasot, jotka sivuavat kolmea annettua palloa? Montako
tällaista tasoa on, jos pallot eivät leikkaa tai sivua toisiaan eikä mikään palloista ole 
toisen sisällä?

\item (*)
Näytä, että yhtälö $K:\ xy+yz+xz=0$ määrittelee kartion, ja määritä se $K$:n piste, joka on
lähinnä pistettä $(-1,2,3)$.

\item (*)
Kartion kärki on origossa, symmetria-akseli on vektorin $\vec i-2\vec j+2\vec k$ suuntainen
ja $y$-akseli on kartiopinnalla. Taso $T$ kulkee pisteen $(1,1,1)$ kautta ja sivuaa kartiota
pitkin avaruussuoraa. Määritä $T$:n yhtälö (kaksi ratkaisua!).

\end{enumerate} % Suorat, tasot ja pinnat
\section{Käyräviivaiset koordinaatistot} \label{koordinaatistot}
\alku
\index{koordinaatisto!c@käyräviivainen|vahv}
\index{kzyyrzy@käyräviivaiset koordinaatistot|vahv}

Euklidinen taso ja avaruus ovat matemaattisina käsitteinä aineettomia tai \kor{iso\-trooppisia} 
siinä mielessä, että mikään piste tai suunta ei niissä ole erikoisasemassa. Erityisesti voidaan
karteesisen koordinaatiston origo, ja myös koordinaattiakselien suunnat 
ortogonaalisuusehtojen puitteissa, valita mielivaltaisesti, ilman että avaruuden geometriset 
lait siitä muuttuisivat. Sovellettaessa geometrista ajattelua fysikaaliseen maailmaan törmätään
kuitenkin siihen realiteettiin, että maailmassa yleensä on materiaa, joka ei jakaudu tasaisesti.
Voi esimerkiksi esiintyä eri materiaalien välisiä rajapintoja. Fysikaalinen ongelma voi olla
ei-isotrooppinen myös siinä vaikuttavien ulkoisten voimien (kuten gravitaation) vuoksi. 
Tällaisissakin tilanteissa ongelmassa voi kuitenkin olla \kor{symmetrioita}, millä tarkoitetaan,
että vaikka koko avaruus ei ole isotrooppinen, niin fysikaalisen ongelman geometria kuitenkin on
johonkin suuntaan tai joihinkin suuntiin liikuttaessa sama. Esimerkiksi 'pannukakkumaailmassa',
jossa puoliavaruus
\[
\mathit{E}_-^{\mathit{3}} = \{P=(x,y,z) \in \Ekolme \ | \ z \leq 0 \}
\]
on homogeenisen materian täyttämä ja muu osa avaruudesta on tyhjää (tai homogeenista toista 
ainetta), geometria säilyy samana liikuttaessa tasoilla, joilla $z=$ vakio. Toisin sanoen: 
geometria on koordinaateista $x,y$ riippumaton. Tässä on kyseessä \kor{tasosymmetrinen} tilanne.
Hieman realistisemmassa (vaikka edelleen maakeskisessä) 'pallomaailmassa', jossa materian 
täyttämää palloa ympäröi homogeeninen avaruus, geometria on puolestaan \kor{pallosymmetrinen}.
Lopulta 'tekniikan maailmassa', jossa avaruuden homogeenisuuden rikkoo yksinäinen, 
poikkipinnaltaan pyöreä kappale (maitohinkki, putki, koaksiaalikaapeli ym.) tilanne on 
\kor{lieriösymmetrinen}.
\begin{figure}[H]
\begin{center}
\import{kuvat/}{kuvaII-19.pstex_t}
\end{center}
\end{figure}
Lieriösymmetrian vastine $\Ekaksi$:ssa on \kor{ympyräsymmetria} (pyörähdyssymmetria).
\begin{figure}[H]
\begin{center}
\import{kuvat/}{kuvaII-20.pstex_t}
\end{center}
\end{figure}
Em.\ tasosymmetrisessä tilanteessa on karteesinen koordinaatisto probleeman 
geometriaan sopiva, kunhan yksi kantavektoreista (esim.\ $\vec k$) valitaan symmetriatason
normaaliksi. Muissa esimerkkitilanteissa sen sijaan tulee kyseeseen siirtyminen sellaiseen 
\kor{käyräviivaiseen} (engl. curvilinear) koordinaatistoon, jossa jokin koordinaateista on vakio 
materiaalirajapinnalla. Koordinaatistoa sanotaan käyräviivaiseksi, jos
\index{koordinaattiakseli!a@--viiva}%
\kor{koordinaattiviivat},
eli $\Ekaksi$:n tai $\Ekolme$:n pistejoukot, joilla vain yksi koordinaateista on muuttuva, eivät
kaikki ole suoria. Seuraavassa esitellään fysiikan kolme tavallisinta käyräviivaista 
koordinaatistoa, jotka liittyvät em. esimerkkitilanteisiin. Nämä ovat:
\index{polaarikoordinaatisto|see{napak.}}%
\index{sylinterikoordinaatisto|see{lieriök.}}%
\begin{itemize}
\item  \kor{napa}- eli \kor{polaarikoordinaatisto} ($\Ekaksi$)
\item  \kor{lieriö}- eli \kor{sylinterikoordinaatisto} ($\Ekolme$)
\item  \kor{pallokoordinaatisto} ($\Ekolme$)
\end{itemize}
Tällaisia koordinaatistoja sovellettaessa (niinkuin yleensäkin matematiikkaa sovellettaessa) 
tehdään yleensä käytännön ongelmatilannetta koskevia yksinkertaistavia olettamuksia eli 
idealisaatioita -- joskus voimakkaita.

\subsection{Napakoordinaatisto}
\index{napakoordinaatisto|vahv}

Napakoordinaatistossa euklidisen tason pisteen $P$ koordinaatit ilmoitetaan origon ($O$) ja 
$P$:n välisellä etäisyydellä $r$ ja nk. 
\index{napakulma}%
\kor{napakulmalla} $\varphi$, joka mittaa vektorin $\overrightarrow{OP}$ ja positiivisen
$x$-akselin (karteesinen koordinaatisto) välistä kulmaa, positiivisena $x$-akselista lähtien
vastapäivään ja negatiivisena myötäpäivään. Kaikki tason pisteet saadaan, kun napakulma
valitaan esim.\ väliltä $[0,2\pi)$ tai $(-\pi,\pi]$. Koska origossa napakulma $\varphi$ ei ole
määritelty, niin sovitaan, että $O=(0,\varphi)\ \forall\varphi$.
\begin{figure}[H]
\begin{center}
\import{kuvat/}{kuvaII-21.pstex_t}
\end{center}
\end{figure}
Muunnoskaavat napakoordinaatistosta karteesiseen ovat
\[
\boxed{\kehys\quad x=r\cos{\varphi}, \quad y=r\sin{\varphi} \quad}
\]
ja päinvastaiseen suuntaan esimerkiksi
\[
\boxed{\kehys\quad r=\sqrt{x^2 + y^2}\,, \quad \sin{\varphi}=y/\sqrt{x^2 + y^2}. \quad}
\]
Tässä on $\varphi$:n määräämiseksi osattava 'kääntää' trigonometrinen sini. Kun huomioidaan myös
$x$:n ja $y$:n etumerkit, niin kulma $\varphi$ määräytyy yksikäsitteisesti esim.\ välillä 
$[0,2\pi)$, kun $r>0$. 
\begin{Exa}
Millainen pistejoukko on
\[
S=\{P=(r,\varphi) \in \Ekaksi \mid r=2\cos\varphi, \ 
                   \varphi\in[-\tfrac{\pi}{2}\,,\tfrac{\pi}{2}]\}\,?
\]
\end{Exa}
\ratk Jos $r > 0$, niin
\begin{align*}
r=2\cos{\varphi} \ &\ekv \ r^2=2r\cos{\varphi} \\
&\ekv x^2+y^2=2x \\
&\ekv (x-1)^2+y^2=1.
\end{align*}
Myös origo on käyrässä mukana $(r=0, \ \varphi=\pm\frac{\pi}{2})$, joten kyseessä on ympyrä,
jonka säde $=1$ ja keskipiste $=(1,0)$. \loppu
\begin{figure}[H]
\begin{center}
\import{kuvat/}{kuvaII-22.pstex_t}
\end{center}
\end{figure}

\begin{multicols}{2} \raggedcolumns
Napakoordinaatistossa koordinaattiviivat, joilla $r=\text{vakio}$, ovat origokeskisiä ympyröitä
ja viivat, joilla $\varphi=\text{vakio}$, ovat origosta lähteviä puolisuoria. Koska nämä 
leikkaavat toisensa kohtisuorasti (ks.\ kuvio), niin voidaan sanoa, että polaarikoordinaatisto
on \kor{suorakulmainen}, vaikkakaan ei suoraviivainen koordinaatisto.
\begin{figure}[H]
\setlength{\unitlength}{1cm}
\begin{center}
\begin{picture}(6,4.5)(-1,-0.5)
\put(-1,0){\line(1,0){5}}
\put(0,-1){\line(0,1){5}}
\put(0,0){\arc{6}{-1.7}{0.15}}
\put(0,0){\line(2,3){2.5}}
\dashline{0.2}(0.3,3.4)(3.3,1.4)
\put(1.59,2.43){$\scriptstyle{\bullet}$}
\put(1.56,2.8){$P$}
\put(3.5,0.5){$r=\text{vakio}$}
\path(2.93,0.6)(3.3,0.6)
\put(3,3){$\varphi=\text{vakio}$}
\path(2.08,3.1)(2.8,3.1)
\end{picture}
\end{center}
\end{figure}
\end{multicols}
Kun napakoordinaatistossa tarkastellaan vektoreita, ilmaistaan nämä yleensä tarkkailupisteen
$P$ mukana 'pyörivässä' kannassa $\{\vec e_r,\vec e_\varphi\}$, missä kantavektorit ovat 
koordinaattiviivojen yksikkötangenttivektorit $P$:ssä. Suunnat valitaan siten, että
vektorit osoittavat ko.\ koordinaattiviivalla muuttuvan koordinaatin kasvun suuntaan:
\begin{figure}[H]
\setlength{\unitlength}{1cm}
\begin{center}
\begin{picture}(8,4)(0,-0.4)
\put(0,0){\vector(1,0){8}} \put(7.8,-0.4){$x$} \put(-0.07,-0.07){$\scriptstyle{\bullet}$}
\put(0,0){\line(3,1){6}} \put(5.93,1.93){$\scriptstyle{\bullet}$}
\put(6,2){\vector(-1,3){0.33}} \put(6,2){\vector(3,1){1}}
%\put(6,2){\arc{0.6}{-1.9}{-0.35}} \put(6.05,2.1){$\scriptstyle{\bullet}$}
\path(6.15,2.05)(6.1,2.2)(5.95,2.15) 
\put (-0.1,-0.5){$O$} \put(5.95,1.6){$P$}
\put(0,0){\arc{2}{-0.33}{0}} \put(0.95,0.28){\vector(-1,4){0.01}} \put(1.2,0.15){$\varphi$}
\put(5.8,3){$\vec e_\varphi$} \put(7,2){$\vec e_r$}
\end{picture}
\end{center}
\end{figure}
Jos pisteen $P$ napakoordinaatit ovat $(r,\varphi)$, niin kantavektorit ovat (vrt.\ kuva)
\[
\boxed{\kehys\quad \vec e_r=\cos\varphi\vec i+\sin\varphi\vec j, \quad
                   \vec e_\varphi=-\sin\varphi\vec i+\cos\varphi\vec j. \quad}
\]
Jos $P$ ajatellaan kiinnitetyksi, voidaan mikä tahansa tason vektori esittää kannassa 
$\{\vec e_r, \vec e_\varphi\}$, eli $\{P,\vec e_r,\vec e_\varphi\}$ on tason koordinaatisto. 
Toisaalta jos $P$:tä muutetaan siten, että koordinaatti $\varphi$ muuttuu, niin muuttuvat myös 
$\vec e_r$ ja $\vec e_\varphi$. Koordinaatisto $\{P, \vec e_r, \vec e_\varphi\}$ poikkeaa siis 
tässä mielessä vastaavasta suoraviivaisesta koordinaatistosta $\{P, \vec i, \vec j\}$, jossa 
kantavektorit $\vec i,\vec j$ pysyvät vakioina referenssipisteen $P$ muuttuessa. 

Tyypillinen fysikaalinen sovellustilanne pisteen $P$ mukana 'pyörivälle' koordinaatistolle 
$\{P,\vec e_r,\vec e_\varphi\}$ syntyy, kun $P$ edustaa \pain{keskeisvoimakentässä} liikkuvaa
kappaletta (esim.\ planeetta, avaruusalus). 
\begin{Exa} \index{zza@\sov!Ceres ja Dawn} \vahv{Ceres ja Dawn}. Karteesisen koordinaatiston
origossa on kääpiöplaneetta, ja planeetasta suuntaan $3\vec i+4\vec j$ olevassa pisteessä $P$
on avaruusluotain, jonka massa on $m=1200$ kg. Luotain kiertää planeettaa tasossa $z=0$
olevalla ympyräradalla vastapäivään positiivisen $z$-akselin suunnasta katsottuna.
Tarkasteluhetkellä luotaimen ionimoottori käynnistetään, jolloin luotaimeen vaikuttaa planeetan
vetovoiman $\vec G=-78\vec e_r$ lisäksi moottorin työntövoima $\vec T=450(-\vec i + \vec j)$
(yksiköt N=kg\,m/s$^2$). Mihin suuntaan luotaimen liikerata kaartuu, ja mikä on luotaimen
kiihtyvyys liikeradan suuntaan\,?
\end{Exa}
\ratk Pisteessä $P$ on $\vec e_r = \frac{1}{5}(3\vec i + 4\vec j)$, ja luotaimen liikesuunta on 
$\,\vec e_\varphi = \frac{1}{5}(-4\vec i + 3\vec j)$. Koska
\[ \begin{cases}
(\vec T + \vec G) \cdot \vec e_r\,    = \frac{450}{5}[(-1) \cdot 3 + 1 \cdot 4] - 78 = 12, \\
(\vec T + \vec G) \cdot \vec e_\varphi = \frac{450}{5}[(-1)\cdot(-4)+1 \cdot 3] + 0  = 630,
\end{cases} \] 
niin luotaimeen vaikuttava kokonaisvoima koordinaatistossa $\{P,\vec e_r,\vec e_\varphi\}$ on
\[
\vec F = \vec T + \vec G = 12 \vec e_r + 630 \vec e_\varphi 
                         = F_r\vec e_r + F_\varphi\vec e_\varphi.
\]
Koska $F_r>0$, niin rata kaartuu oikealle (poispäin planeetasta). Ratakiihtyvyys määräytyy
Newtonin liikelaista: $\,a_\varphi = F_\varphi/m \approx$ \underline{\underline{$0.53$ m/s$^2$}}.
\loppu

\subsection{Lieriökoordinaatisto}
\index{lieriökoordinaatisto|vahv}

\begin{multicols}{2} \raggedcolumns
Lieriökoordinaatisto on napakoordinaatiston vastine kolmessa dimensiossa. Karteesisista 
koordinaateista $x,y,z$ jätetään $z$ ennalleen ja muunnetaan $(x,y) \map (r,\varphi)$ samalla 
tavoin kuin napakoordinaatistoon siirryttäessä. Lieriökoordinaatiston 'pyöriviä'
kantavektoreita merkitään $\vec e_r, \vec e_\varphi, \vec e_z$. Tässä $\vec e_r,\vec e_\varphi$
ovat samat kuin polaarikoordinaatistossa ja $\,\vec e_z = \vec k\,$ kuten karteesisessa 
koordinaatistossa. Kanta $\{\vec e_r,\vec e_\varphi,\vec e_z\}$ on ortonormeerattu oikeakätinen
systeemi.
\begin{figure}[H]
\setlength{\unitlength}{1cm}
\begin{center}
\begin{picture}(5,6)(-2,-2)
\put(0,0){\vector(1,0){3}} \put(2.8,-0.4){$y$}
\put(0,0){\vector(0,1){4}} \put(0.2,3.8){$z$}
\put(0,0){\vector(-3,-2){2}} \put(-2.3,-1.7){$x$}
\path(0,0)(1.5,-1)(1.5,2)
\dashline{0.1}(1.5,2)(0,3)
\put(1.43,1.93){$\scriptstyle{\bullet}$} 
\put(1.5,2){\vector(4,-3){0.7}} \put(2,1){$\vec e_r$}
\put(1.5,2){\vector(0,1){1}} \put(1.6,3.1){$\vec e_z$}
\put(1.5,2){\vector(2,1){0.8}} \put(2.4,2.3){$\vec e_\varphi$}
\put(1.67,2){\begin{turn}{-90} 
   $\overbrace{\hspace{3cm}}^{\displaystyle{\text{\begin{turn}{90}$z$\end{turn}}}}$ \end{turn}}
\put(-0.4,-0.1){\begin{turn}{-33.7} 
   $\underbrace{\hspace{1.8cm}}_{\displaystyle{\text{\begin{turn}{33.7}$r$\end{turn}}}}$
\end{turn}}
\put(0,0){\arc{0.8}{0.588}{2.56}} 
\put(0.33,-0.23){\vector(1,1){0.01}} \put(-0.15,-0.7){$\varphi$}
%\put(1.5,-1){\arc{0.6}{3.75}{4.7}} \put(1.35,-0.9){$\scriptscriptstyle{\bullet}$}
\path(1.35,-.9)(1.35,-.75)(1.5,-.85)
\end{picture}
\end{center}
\end{figure}
\end{multicols}
\begin{Exa}
\index{ruuviviiva}%
\kor{Ruuviviiva} on avaruuden $\Ekolme$ pistejoukko
\[
S=\{P=(r,\varphi,z) \ | \ r=R, \ z=a\varphi\} \quad (a \neq 0). \loppu
\]
\end{Exa}
Lieriökoordinaatiston koordinaattiviivat ovat $\Ekolme$:n pistejoukkoja, joissa kaksi 
koordinaateista $r,\varphi,z$ saa vakioarvon. Näille saadaan seuraavat geometriset luonnehdinnat
(vrt.\ kuvio edellä):
\begin{align*}
&S_r\       (\varphi,z\ \text{vakioita}): \quad \text{$z$-akselilta lähtevä, 
                                                      $xy$-tason suuntainen puolisuora}, \\
&S_\varphi\ (r,z \text{vakioita})\,\ :  \quad \text{ympyrä $xy$-tason suuntaisella tasolla}, \\
&S_z\       (r,\varphi\ \text{vakioita})\, : \quad \text{$z$-akselin suuntainen suora}.
\end{align*}
\index{koordinaattiakseli!c@--pinta}%
\kor{Koordinaattipinnoiksi} sanotaan sellaisia $\Ekolme$:n pistejoukkoja, joilla yksi
koordinaateista saa vakioarvon. Lieriökoordinaatistossa näiden geometriset luonnehdinnat ovat
\begin{align*}
&r=\ \text{vakio}:       \quad\, \text{lieriö, akselina $z$-akseli}, \\
&\varphi=\ \text{vakio}: \quad   \text{$z$-akseliin rajoittuva puolitaso}, \\
&z=\ \text{vakio}:       \quad   \text{$xy$-tason suuntainen taso}.
\end{align*}

\subsection{Pallokoordinaatisto}
\index{pallokoordinaatisto}

Pisteen $P=(x,y,z)$ \kor{pallokoordinaatit} ovat (ks.\ kuvio)
\begin{itemize}
\item[-] $P\,$:n etäisyys origosta: $\ r=\sqrt{x^2+y^2+z^2}$
\item[-] paikkavektorin $\vec r = \overrightarrow{OP}$ ja positiivisen $z$-akselin 
         (vektorin $\vec k$) muodostama kulma $\theta$, $\ 0 \leq \theta \leq \pi$
\item[-] kulma $\varphi\,,\ 0 \le \varphi < 2\pi$, jonka muodostavat positiivinen $x$-akseli
         (vektori $\vec i$) ja vektorin $\vec r = \overrightarrow{OP}$ ortogonaaliprojektio 
         $xy$-tasolle
\end{itemize}
Koordinaatteja $\theta, \varphi$ sanotaan
\index{pallonpintakoordinaatit}%
\kor{pallonpintakoordinaateiksi}. 
\begin{figure}[H]
\setlength{\unitlength}{1cm}
\begin{center}
\begin{picture}(8,6)(-2,-2)
\put(0,0){\vector(1,0){3}} \put(2.8,-0.4){$y$}
\put(0,0){\vector(0,1){4}} \put(0.2,3.8){$z$}
\put(0,0){\vector(-3,-2){2}} \put(-2.3,-1.7){$x$}
\dashline{0.1}(0,0)(1.5,-1)
\dashline{0.1}(1.5,-1)(1.5,2)
\dashline{0.1}(1.5,2)(0,3)
\put(1.43,1.93){$\scriptstyle{\bullet}$} 
\put(1.5,2){\vector(1,-1){0.65}} \put(2.2,1){$\vec e_\theta$}
\put(1.5,2){\vector(3,4){0.67}} \put(1.6,3.1){$\vec e_r$}
\put(1.5,2){\vector(2,1){0.8}} \put(2.4,2.3){$\vec e_\varphi$}
\put(-1.7,3){\begin{turn}{-90} 
$\underbrace{\hspace{3cm}}_{\displaystyle{\text{\begin{turn}{90}
                                               $r\cos\theta$\end{turn}}}}$ \end{turn}}
\put(-0.7,0){\begin{turn}{-33.7} 
$\underbrace{\hspace{1.8cm}}_{\displaystyle{\text{\begin{turn}{33.7}$r\sin\theta$\end{turn}}}}$
\end{turn}}
\put(0,0){\arc{0.8}{0.588}{2.56}} 
\put(0.33,-0.23){\vector(1,1){0.01}} \put(-0.15,-0.7){$\varphi$}
\path(0,0)(1.5,2)
\put(1.7,2){\begin{rotate}{-127} 
$\overbrace{\hspace{2.5cm}}^{\displaystyle{\text{\begin{turn}{127}$r$\end{turn}}}}$ \end{rotate}}
\put(0,0){\arc{0.8}{-1.57}{-0.93}}
\put(0.1,0.5){$\theta$}
%\put(0,3){\arc{0.6}{0.6}{1.57}} \put(0.03,2.75){$\scriptscriptstyle{\bullet}$}
%\put(1.5,-1){\arc{0.6}{3.75}{4.7}} \put(1.35,-0.9){$\scriptscriptstyle{\bullet}$}
\path(0,2.85)(0.15,2.75)(0.15,2.9)
\path(1.35,-.9)(1.35,-.75)(1.5,-.85)
\put(1.4,2.2){$P$}
\end{picture}
\end{center}
\end{figure}
Muunnoskaavat ovat
\[
\boxed{\quad
\begin{array}{ll}
\ykehys x &= \ r \sin{\theta} \cos{\varphi}, \quad \\
        y &= \ r \sin{\theta} \sin{\varphi}, \\
        z &= \ r \cos{\theta}. \akehys
\end{array}
}
\]

Pallokoordinaatiston koordinaattiviivat ovat seuraavaa tyyppiä:
\begin{align*}
&S_r\ (\theta,\varphi\ \text{vakioita}): \quad \text{origosta lähtevä puolisuora}, \\
&S_\theta\ (r,\varphi\ \text{vakioita}):  \quad \text{origokeskinen, $z$-akseliin rajoittuva
                                                                              puoliympyrä}, \\
&S_\varphi\ (r,\theta\ \text{vakioita}): \quad \text{ympyrä $xy$-tason suuntaisella tasolla}.
\end{align*}
Koordinaattipinnat voidaan luokitella seuraavasti:
\begin{align*}
&r=\ \text{vakio}:       \quad\, \text{origokeskinen pallo}, \\
&\theta=\ \text{vakio}:  \quad   \text{puolikartio, kärkenä origo}, \\
&\varphi=\ \text{vakio}: \quad   \text{$z$-akseliin rajoittuva puolitaso}.
\end{align*}

Myös pallokoordinaatistossa koordinaattiviivat leikkaavat toisensa kohtisuorasti, ts.\
koordinaattiviivojen tangentit (ks.\ edellinen luku) ovat leikkauspisteessä parittain
kohtisuorat. Koordinaattiviivojen yksikkötangenttivektoreista (tangenttien suuntavektoreista)
muodostettua avaruuden vektorien kantaa merkitään $\{\vec e_r, \vec e_\theta, \vec e_\varphi\}$
ja määritellään (vrt. kuvio edellä)
\[
\boxed{
\begin{array}{ll}
\ykehys\quad \vec e_r &= \ \sin{\theta} \cos{\varphi} \vec i + \sin{\theta}\sin{\varphi} \vec j 
                                                           + \cos{\theta} \vec k, \quad \\[4pt]
\quad   \vec e_\theta &= \ \cos{\theta} \cos{\varphi} \vec i + \cos{\theta}\sin{\varphi} \vec j 
                                                             - \sin{\theta} \vec k, \\[4pt]
\quad  \vec e_\varphi &= \ -\sin{\varphi} \vec i + \cos{\varphi} \vec j. \akehys
\end{array}
}
\]
Tässä $\vec e_\varphi$ on sama kuin polaari- ja lieriökoordinaatistoissa ja $\vec e_r$:n 
lauseke seuraa välittömästi edellä esitetyistä koordinaattien muunnoskaavoista:
\begin{align*}
\vec e_r\,=\,\frac{1}{\abs{\vec r\,}}\vec r\,
               &=\,\frac{1}{r}(x\vec i+y\vec j+z\vec k) \\
               &=\,\sin{\theta} \cos{\varphi} \vec i + \sin{\theta}\sin{\varphi} \vec j 
                                                                  + \cos{\theta} \vec k.
\end{align*}
Tämän jälkeen $\vec e_\theta$ on laskettavissa tiedosta, että 
$\{\vec e_r,\vec e_\theta,\vec e_\varphi\}$ on ortonormeerattu, oikeakätinen systeemi: 
\[
\vec e_\theta=\vec e_\varphi\times\vec e_r.
\]
\begin{Exa} Pisteen $P$ karteesiset koordinaatit ovat $(x,y,z)=(-2,-3,-6)$. Määritä $P$:n
pallokoordinaatit sekä vektorin $\vec v=\vec i-\vec j$ koordinaatit $(v_r,v_\theta,v_\varphi)$
pallokoordinaatiston kannassa $\{\vec e_r,\vec e_\theta,\vec e_\varphi\}$ ko.\ pisteessä.
\end{Exa}
\ratk Jos merkitään $\vec r=\overrightarrow{OP}=-2\vec i-3\vec j-6\vec k$, niin
$r=\abs{\vec r\,}=7$ ja
\[
\vec e_r = \frac{1}{r}\,\vec r =\frac{1}{7}(-2\vec i-3 \vec j-6\vec k).
\]
Tällöin
\begin{align*}
\cos\theta\,&= \vec e_r\cdot\vec k = -\frac{6}{7}\,, \quad \sin\theta=\frac{\sqrt{13}}{7}\,, \\
\cos\varphi &= \frac{\vec i\cdot(-2\vec i-3\vec j)}{\sqrt{2^2+3^2}}=-\frac{2}{\sqrt{13}}\,,
                                                 \quad \sin\varphi =-\frac{3}{\sqrt{13}}\,,
\end{align*}
joten $P$:n pallokoordinaatit ovat
\[
(r,\theta,\varphi) \approx (7,\,149.0\aste,\,236.3\aste).
\]
Kantavektoreiden $\vec e_\theta,\vec e_\varphi$ tarkat arvot ovat
\begin{align*}
\vec e_\theta  &=\,-\frac{6}{7}\left(-\frac{2}{\sqrt{13}}\right)\vec i
                  -\frac{6}{7}\left(-\frac{3}{\sqrt{13}}\right)\vec j
                  -\frac{\sqrt{13}}{7}\vec k \\
               &=\,\frac{1}{7\sqrt{13}}\,(12\vec i+18\vec j-13\vec k), \\
\vec e_\varphi &=\,\frac{1}{\sqrt{13}}\,(3\vec i-2\vec j),
\end{align*}
joten kysytyt $\vec v$:n koordinaatit ovat
\begin{align*}
&v_r       = \vec v\cdot\vec e_r       =  \frac{1}{7}          \approx  0.143, \\
&v_\theta  = \vec v\cdot\vec e_\theta  = -\frac{6}{7\sqrt{13}} \approx -0.238, \\
&v_\varphi = \vec v\cdot\vec e_\varphi =  \frac{5}{\sqrt{13}}  \approx  1.387.
\end{align*}
Tarkistus:
\[
v_r^2+v_\theta^2+v_\varphi^2\,=\,\frac{1274}{13 \cdot 49}\,
                              =\,2\,=\,\abs{\vec v\,}^2. \quad\text{OK!} \loppu
\]

%\begin{multicols}{2} \raggedcolumns
\index{zza@\sov!Lentosuunta}
\begin{Exa}: \vahv{Lentosuunta}. Lentokone lähtee päiväntasaajan pisteestä $A:$ $90^\circ$ 
läntistä pituutta ja lentää pitkin isoympyrää, joka kulkeee pisteen  $B:$ $45^\circ$ itäistä
pituutta, $60^\circ$ pohjoista leveyttä, kautta. Määritä koneen lentosuunta pisteessä $B$ sekä
kannassa $\{\vec i, \vec j, \vec k\}$ että suhteessa ilmansuuntiin pisteessä $B$.
\end{Exa}
\begin{figure}[H]
\begin{center}
\import{kuvat/}{kuvaII-23.pstex_t}
\end{center}
\end{figure}
%\end{multicols}
\ratk Oletetaan pallon säteeksi $R=1$, jolloin (ks.\ kuva)
\begin{align*}
\Vect{OA} &= - \vec j, \\
\Vect{OB} &= \sin{30^\circ}\cos{45^\circ} \vec i 
+ \sin{30^\circ}\sin{45^\circ} \vec j + \cos{30^\circ} \vec k \\
&= \frac{1}{2\sqrt{2}} \vec i + \frac{1}{2\sqrt{2}} \vec j +
\frac{\sqrt{3}}{2} \vec k.
\end{align*}
Koneen lentorata on tasossa, joka kulkee pisteiden $O, A, B$ kautta, joten tason
normaalivektori on
\[
\vec n = \Vect{OA} \times \Vect{OB} 
       = -\frac{\sqrt{3}}{2}\vec i + \frac{1}{2\sqrt{2}}\vec k.
\]
Koska lentorata pisteessä $B$ on myös kohtisuorassa vektoria $\Vect{OB}$ vastaan, on
lentosuunnan oltava sama kuin vektorilla
\[
\vec n\times\Vect{OB} 
   = \left|\begin{array}{ccc}
     \vec i & \vec j & \vec k \\
     -\tfrac{\sqrt{3}}{2} & 0 & \tfrac{1}{2\sqrt{2}} \\
     \tfrac{1}{2\sqrt{2}} & \tfrac{1}{2\sqrt{2}} & \tfrac{\sqrt{3}}{2}
     \end{array} \right|
   = -\frac{1}{8}\vec i + \frac{7}{8}\vec j - \frac{1}{4}\sqrt{\frac{3}{2}}\vec k.
\]
Tämän vektorin pituus on $\sqrt{56}/8$, joten lentosuunnan osoittava yksikkövektori on
\[
\underline{\underline{\vec v\,^0 = \frac{1}{\sqrt{56}}(-\vec i + 7\vec j -\sqrt{6}\vec k)}}.
\]
Itään osoittava yksikkövektori pisteessä $B$ on
\[
\vec e_\varphi \,=\, - \sin{45^\circ} \vec i + \cos{45^\circ} \vec j 
               \,=\, \frac{1}{\sqrt{2}}(-\vec i + \vec j),
\]
ja etelään osoittava yksikkövektori on
\begin{align*}
\vec e_\theta &= \cos{30^\circ} \cos{45^\circ} \vec i +
\cos{30^\circ} \sin{45^\circ} \vec j - \sin{30^\circ}\vec k \\
&= \frac{1}{2}\sqrt{\frac{3}{2}}\,(\vec i + \vec j) - \frac{1}{2} \vec k.
\end{align*}
Koska
\begin{align*}
\vec v\,^0 \cdot \vec e_\varphi &= \frac{2}{\sqrt{7}}\ \approx\, \cos{41^\circ}, \\
\vec v\,^0 \cdot \vec e_\theta &= \sqrt{\frac{3}{7}}\, \approx\, \cos{49^\circ},
\end{align*}
on lentosuunta $B$:ssä kaakosta hieman itään päin. \loppu
\begin{figure}[H]
\begin{center}
\import{kuvat/}{kuvaII-24.pstex_t}
\end{center}
\end{figure}

\Harj
\begin{enumerate}

\item
Seuraavassa on annettu piste $P$ joko karteesisessa, lieriö- tai pallokoordinaatistossa.
Laske $P$:n koordinaatit muissa kahdessa koordinaatistossa (tarkasti, jos mahdollista,
muuten likiarvoina). \vspace{1mm}\newline 
$P=(x,y,z): \quad$ 
a)\, $(1,1,1) \qquad\,$ 
b)\, $(2,3,-1) \qquad$ 
c)\, $(-1,-\sqrt{3},-5)$ \newline
$P=(r,\varphi,z): \quad$
d)\, $(\sqrt{2},\frac{\pi}{2},1) \quad\,$ 
e)\, $(1,\frac{\pi}{3},-1) \qquad$
f)\, $(6\sqrt{2},\frac{7\pi}{4},3)$ \newline
$P=(r,\theta,\varphi): \quad $
g)\, $(1,\frac{\pi}{2},\pi) \qquad$
h)\, $(\sqrt{3},\frac{3\pi}{4},\frac{3\pi}{4}) \quad$
i)\, $(3,\frac{2\pi}{3},\frac{5\pi}{6})$

\item
a) Tutki, millainen on polaarikoordinaattien avulla ilmaistu tasokäyrä
\[
S=\{P=(r,\varphi) \mid r=\cos\varphi+\sin\varphi,\ 
                       \varphi\in[-\tfrac{\pi}{4},\tfrac{3\pi}{4}]\}.
\]
b) Tason ympyrän $S$ keskipiste on $(x_0,y_0)$ ja säde $=R$. Näytä, että
$P=(x,y) \in S$ täsmälleen kun jollakin $\varphi\in[0,2\pi)$ pätee
\[ 
\begin{cases} \,x=x_0+R\cos\varphi, \\ \,y=y_0+R\sin\varphi. \end{cases}
\]
Mikä on vastaava väittämä koskien avaruuden pallopintaa, jonka keskipiste on $(x_0,y_0,z_0)$
ja säde $R$\,?
%$\,S:\ (x-x_0)^2+(y-y_0)^2+(z-z_0)^2=R^2$\,?

\item
Pisteiden $P_1$ ja $P_2$ pallokoordinaatteja $(r_1,\theta_1,\varphi_1)$ ja 
$(r_2,\theta_2,\varphi_2)$ verrataan joukon 
$[0,\infty)\times[0,\pi]\times[0,2\pi)\subset\Rkolme$ alkioina. Luettele tapaukset,
joissa vertailu antaa tuloksen $P_1=P_2$, vaikka koordinaatit eivät ole samat.

\item
Pisteessä $P$, jonka karteesiset koordinaatit tunnetaan, vaikuttaa voima $\vec F$.
Muunna $\vec F$ annettuun kantaan ko.\ pisteessä. \vspace{1mm}\newline
a)\, $P=(-2,3,6), \quad \vec F=\vec i-\vec j+\vec k, \quad 
                        \{\vec e_r,\vec e_\varphi,\vec e_z\}$ \newline
b)\, $P=(-2,3,6), \quad \vec F=\vec i+\vec j+\vec k, \quad 
                        \{\vec e_r,\vec e_\theta,\vec e_\varphi\}$ \newline
c)\, $P=(6,-6,-3), \quad \vec F=-9\vec e_r\ \text{(pallok.)}, \quad 
                        \{\vec i,\vec j,\vec k\}$ \newline
d)\, $P=(-6,-6,3), \quad \vec F=-9\vec e_r\ \text{(lieriök.)}, \quad 
                        \{\vec e_r,\vec e_\theta,\vec e_\varphi\}$ \newline
e)\, $P=(-1,\sqrt{3},-2\sqrt{3}), \quad \vec F=\vec e_\theta-\vec e_\varphi\
                         \text{(pallok.)}, \quad \{\vec i,\vec j,\vec k\}$

\item
Karteesisen koordinaatiston origo on maapallon keskipisteessä ja vektori $\vec k$ osoittaa
pohjoisnavalle päin. Maapallon pinnalla eteläisen pallonpuoliskon pisteessä, jonka
pallonpintakoordinaatit ovat $\theta=135\aste,\ \varphi=120\aste$, laiva on matkalla
luoteeseen ja lentokone koilliseen. Määritä laivan ja lentokoneen liikesuunnat
yksikkövektoreina a) pallokoordinaatistossa vektorien $\{\vec e_\theta,\vec e_\varphi\}$ 
avulla, b) karteesisessa koordinaatistossa vektorien $\{\vec i,\vec j,\vec k\}$ avulla.

\item
Helsingistä $(60^{\circ}N,25^{\circ}E)$ lennetään lyhintä tietä Tokioon 
 $(36^{\circ}N,140^{\circ}E)$. Miten pitkä on matka ja mihin 
ilmansuuntaan on Helsingistä lähdettävä? Maapallon säde on $6370$ km.

\item (*)
Pallokoordinaatin $\theta $ avulla ilmaistu puolikartio $\theta =30^{\circ} $
leikataan tasolla $T: y+z-4=0$. Ilmaise syntyvä leikkauskäyrä  $S\subset E^3$ ensin 
pallokoordinaattien $(r,\theta ,\varphi)$ avulla. Muunna sitten käyrä tason $T$ 
napakoordinaatistoon, jossa origo on pisteessä $(0,0,4)$ ja napakulmaa mitataan sunnasta
$\vec i$, ja edelleen $T$:n vastaavaan karteesiseen koordinaatistoon. Hahmottele käyrä 
viimeksi mainitussa koordinaatistossa.

\end{enumerate}
 % Käyräviivaiset koordinaatistot

\chapter{Kompleksiluvut}

Siirtyminen reaaliluvista kompleksilukuihin on matemaattisen analyysin merkittävimpiä ja samalla
merkillisimpiä aluevaltauksia. Kyse on lukualueen laajennuksesta, ts. siirtymisestä jälleen 
uudelle 'todellisuuden' tasolle. Vaikka laajennusta voi pitää vain kuvitelmana, niin tämä 
kuvitelma on yksinkertaistanut matemaattista ajattelua siinä määrin, että sillä on lopulta ollut
syvällinen vaikutus kaikkeen matematiikkaan, myös käytännön laskentamenetelmiin. Matemaattisen
analyysin perinteessä lukualueen laajennus korostuu käsitteissä \kor{reaalianalyysi} ja 
\kor{kompleksianalyysi}. Molemmat ovat nykyään hyvin laajoja (ja hieman epämääräisiä) 
matematiikan alueita. Kompleksianalyysin osa-alueista maininnan arvoinen on kompleksimuuttujan
funktioiden teoria eli \kor{funktioteoria}\footnote[2]{Funktioteorian tutkimusperinne on 
Suomessa vahva. Tätä matematiikan suuntausta edusti myös Suomen historian tunnetuin 
matemaatikko, akateemikko \hist{Rolf Nevanlinna} (1895-1980). \index{Nevanlinna, R.|av}}.
 % Kompleksiluvut
\section{Osoitinkunta} \label{osoitinkunta}
\alku
\index{osoitinkunta|vahv}
\index{laskuoperaatiot!d@osoittimien|vahv}

Lähtökohtana on euklidinen taso ja siihen pystytetty karteesinen koordinaatisto,
koordinaatteina $(x,y)$. Kutsutaan tällä kertaa \kor{osoittimeksi} suuntajanaa, jolla on 
tietty pituus ja suunta. Pituutta merkitään symbolilla $r$ ja suunta mitataan $x$-akselin
suunnasta vastapäivään kiertäen
\index{vaihekulma}%
\kor{vaihekulmalla} (napakulmalla) $\varphi$. Merkitään
\[
\ptp = r \vkulma{\varphi}\,.
\]
\begin{figure}[H]
\begin{center}
\import{kuvat/}{kuvaII-25.pstex_t}
\end{center}
\end{figure}
Osoitin on siis toistaiseksi täsmälleen samanlainen olio kuin vektori: sillä on pituus ja 
suunta. Osoittimet myös samastetaan kuten vektorit. On huomioitava ainoastaan, että
$\varphi$ ja $\varphi+2\pi$ vastaavat samaa suuntaa, joten samastussäännöt ovat
\index{samastus '$=$'!f@osoittimien}% 
\[
r_1\vkulma{\varphi_1}=r_2\vkulma{\varphi_2} \qekv
\begin{cases}
\,r_1=r_2=0\ \ \text{tai} \\ 
\,r_1=r_2>0\,\ \wedge\,\ \varphi_1-\varphi_2=k \cdot 2\pi,\ \ k\in\Z.
\end{cases}
\]
Tapauksessa $r=0$ samastetaan siis kaikki suunnat, kuten nollavektorissa. Osoitinta
kutsutaankin tässä tapauksessa
\index{nollaosoitin}%
\kor{nollaosoittimeksi} ja merkitään
\[
0 \vkulma{\varphi} = \pointer{0}.
\]
Osoittimien joukkoa merkitään jatkossa $\Pkunta$:llä:
\[
\Pkunta=\{\ptp = r\vkulma{\varphi} \mid r\in[0,\infty),\ \varphi\in\R\}.
\]

Osoittimien yhteenlaskuoperaatio $(+)$ määritellään samalla tavoin kuin vektoreille, eli
kolmiodiagrammin avulla:
\begin{figure}[H]
\begin{center}
\import{kuvat/}{kuvaII-26.pstex_t}
\end{center}
\end{figure}
Myös skalaareilla eli reaaliluvuilla kertominen suoritetaan samoin kuin vektoreilla.
Erityisesti jos $\lambda \in \R$ ja $\lambda > 0$, niin
\[
\lambda(r \vkulma{\varphi}) = \lambda r \vkulma{\varphi}\,, \quad \lambda > 0.
\]
Tapauksessa $\lambda=0$ on kertolaskun tulos nollaosoitin. Jos $\lambda \in \R$ ja 
$\lambda < 0$, niin määritellään (vrt. vektorit)
\begin{align*}
\lambda (r \vkulma{\varphi}) &= \abs{\lambda} \cdot (-1) (r \vkulma{\varphi}) \\
                             &= \abs{\lambda} r \vkulma{(\varphi + \pi)}\,, \quad \lambda < 0.
\end{align*}
Tässä $r\vkulma(\varphi + \pi)$ on yhteenlaskun määritelmän mukaisesti osoittimen 
$\ptp = r \vkulma{\varphi}\,$
\index{vastaosoitin}%
\kor{vastaosoitin}:
\[
-\ptp = (-1)\ptp = r\vkulma{(\varphi + \pi)}\,.
\]
Tähänastisen perusteella osoittimien yhteenlaskusta ja skaalauksesta muodostuva algebra 
$(\Pkunta,+,\R)$ näyttää yksinkertaisesti tason vektoriavaruudelta. Näin onkin toistaiseksi,
ja vektorianalogia voidaan viedä hiukan pidemmällekin: Otetaan käyttöön tason vektoriavaruuden
ortonormeerattua kantaa $\{\vec i, \vec j\}$ vastaava osoitinkanta, jossa kantaosoittimia
merkittäköön
\begin{multicols}{2} \raggedcolumns
\begin{align*}\vec i\ \vast\ \ptr &= 1 \vkulma{0}, \\
\vec j\ \vast\ \pti &= 1 \vkulma{\pi/2}.
\end{align*}
\begin{figure}[H]
\setlength{\unitlength}{1cm}
\begin{center}
\begin{picture}(2,2.5)(0,0)
\put(0,0){\vector(1,0){2}} \put(1.75,-0.5){$\ptr$}
\put(0,0){\vector(0,1){2}} \put(0.2,1.6){$\pti$}
\end{picture}
\end{center}
\end{figure}
\end{multicols}
Jos nyt $\ptp = r \vkulma{\varphi} \in \Pkunta$, niin $\ptp$ voidaan ilmaista kantaosoittimien 
avulla yksikäsitteisesti muodossa
\begin{align} \label{vaellus1}
\ptp = r \vkulma{\varphi}\, &=\, r\cos{\varphi}\,\ptr + r\, \sin{\varphi}\,\pti \notag \\ 
                            &=\, x \ptr + y \pti.
\end{align}
\begin{figure}[H]
\begin{center}
\import{kuvat/}{kuvaII-27.pstex_t}
\end{center}
\end{figure}
Toisaalta jos tunnetaan $\ptp$:n koordinaatit $x,y$ osoitinkannassa, niin esitysmuoto 
$\ptp = r \vkulma{\varphi}$ saadaan lasketuksi kaavoilla
\begin{equation} \label{vaellus2}
r=\sqrt{x^2+y^2}\,, \quad \begin{cases}
                          \,\cos{\varphi} = x/r, \\ \,\sin{\varphi} = y/r.
                          \end{cases}
\end{equation}
Tämän mukaan $\varphi$ määräytyy $2\pi$:n monikertaa vaille yksikäsitteisesti, jos $r>0$
(eli $(x,y)\neq(0,0)$), joten osoitin määräytyy sovittujen samastussääntöjen nojalla
yksikäsitteisesti. Jatkossa käytetään osoittimen eri esityismuodoista nimityksiä
\begin{align*}
\ptp = r \vkulma{\varphi} \qquad\qquad &\text{\kor{polaarimuoto} (polaariesitys)}, \\
\ptp = x \ptr + y \pti    \qquad       &\text{\kor{komponenttimuoto}}.
\end{align*} 
Siirtyminen esitysmuodosta toiseen tapahtuu siis säännöillä \eqref{vaellus1} ja
\eqref{vaellus2}. Koska komponenttiesitys luo kääntäen yksikäsitteisen vastaavuuden
$\Pkunta \vast \Rkaksi$, niin osoittimia voidaan ajatella abstraktisti myös reaalilukujen
pareina:
\[
\ptp = x\ptr+y\pti\ \vast\ (x,y)\in\Rkaksi.
\]

Osoittimen komponenttimuoto on erityisen kätevä yhteenlaskussa:
\[
\boxed{\begin{aligned}
\quad \ptp_1 &= r_1 \vkulma{\theta_1} \ \ja \ \ptp_2 = r_2 \vkulma{\theta_2} \\
             &\impl\ \ptp_1 + \ptp_2 = (r_1 \cos{\theta_1} + r_2 \cos{\theta_2})\,\ptr 
                                     + (r_1 \sin{\theta_1} + r_2 \sin{\theta_2})\,\pti.
                                                                            \akehys\quad
\end{aligned} } \]
Polaarimuotoon päästään tästä takaisin säännöillä (\ref{vaellus2}) --- lopputulosta ei
yleisessä muodossa kannata kirjoittaa.

Tähän asti osoitinavaruuden ja tason vektoriavaruuden välinen analogia on täydellinen.
Tultaessa osoittimien \pain{kertolaskuun} tiet kuitenkin eroavat. Osoittimille ei määritellä
skalaarituloa eikä ristituloakaan, vaan toisen tyyppinen tulo, jota jatkossa sanotaan
\index{osoitintulo}%
\kor{osoitintuloksi}. Määritelmä on:
\[
\boxed{\kehys\quad \ptp_1 = r_1 \vkulma{\varphi_1} \ \ja \ \ptp_2 = r_2 \vkulma{\varphi} \qimpl 
                 \ptp_1\cdot\ptp_2 = r_1 r_2 \vkulma{(\varphi_1 + \varphi_2)}. \quad}
\]
Osoitintulo on siis funktio tyyppiä $\,\Pkunta \times \Pkunta \kohti \Pkunta$ --- kyse on
hieman omalaatuisesta reaalilukujen kertolaskun (pituudet) ja yhteenlaskun (suunnat)
yhdistelmästä (vrt.\ Harj.teht.\,\ref{kunta}:\ref{H-I-2: Big Ben}).

Osoitintulon määritelmästä seuraa suoraan normaali vaihdantalaki
\[
\ptp_1\ptp_2 = \ptp_2\ptp_1
\]
(tässä jätetty kertomerkki pois tavalliseen tapaan). Myös liitäntälaki on voimassa
(Harj.teht.\,\ref{H-III-1: osoitintulon liitäntälaki}). Lisäksi nähdään, että
\index{ykkösosoitin}%
\kor{ykkösosoitin}, eli kertolaskun ykkösalkio, on
\[
\pointer{1} = \ptr = 1 \vkulma{0}\,.
\]
Jokaisella $\ptp \in \Pkunta, \ \ptp \neq \pointer{0}$, on myös
\index{kzyzy@käänteisosoitin}%
\kor{käänteisosoitin} $\ptpinv$, joka toteuttaa
\[
\ptp\cdot\ptpinv = \pointer{1} = 1 \vkulma{0}\,.
\]
Nimittäin jos $\ptp = r \vkulma{\varphi},\ r \neq 0$, niin 
\begin{multicols}{2} \raggedcolumns
\[
\ptpinv = r^{-1}\vkulma{-\varphi}\,.
\]
\begin{figure}[H]
%\begin{center}
\import{kuvat/}{kuvaII-28.pstex_t}
%\end{center}
\end{figure}
\end{multicols}
Osoitintulon määritelmästä seuraa välittömästi, että jokaisella $n\in\N$ pätee potenssiin 
korotuksen laskusääntö
\[
\boxed{\kehys\quad (\ptp)^n = r^n\vkulma{n\varphi}. \quad}
\]
Tapauksessa $\ptp\neq\pointer{0}$ tämä on pätevä jokaisella $n\in\Z$, kun sovitaan normaaliin 
tapaan, että
\[
(\ptp)^0=\pointer{1}, \quad\ (\ptp)^{-n}=[\ptpinv]^n,\ \ n\in\N \quad (\ptp\neq\pointer{0}).
\]
\begin{Exa} Määritä osoitinyhtälön $(\ptp)^3=-\ptp$ kaikki ratkaisut.
\end{Exa}
\ratk Osoitintulon määritelmän ja samastussääntöjen nojalla päätellään
\begin{align*}
      &(\ptp)^3=-\ptp \\[2mm]
\qekv &r^3\vkulma{3\varphi} \,=\, r\vkulma{(\varphi+\pi)} \\[2mm]
\qekv &r^3=r=0 \quad \text{tai} \quad 
       r^3=r>0\ \ \ja\ \ 3\varphi=(\varphi+\pi)+k \cdot 2\pi,\,\ \ k\in\Z \\[1mm]
\qekv &r=0 \quad \text{tai} \quad r=1\ \ \ja\ \ \varphi=\frac{\pi}{2}+k\cdot\pi,\ \ k\in\Z.
\end{align*}
Tapauksessa $r=1$ saadaan erilaisia osoittimia vain $k$:n arvoilla $0,1$, joten ratkaisut
ovat
\[
\ptp=\pointer{0},\ \ptp=\pm\pti. \loppu
\]

Osoitintulon ja osoittimien yhteenlaskun määritelmistä seuraa vielä, että pätee myös
osittelulaki
\[
\ptp \cdot (\ptq_1 + \ptq_2) = \ptp \cdot \ptq_1 + \ptp \cdot \ptq_2\,.
\]
Tämä on seuraus siitä, että jos $\ptp = r \vkulma{\varphi}$, niin kertolasku $\ptp \cdot \ptq$
voidaan kirjoittaa muotoon
\[
\ptp \cdot \ptq = r (1\vkulma{\varphi}) \cdot \ptq.
\]
Tässä osoittimella $(1\vkulma{\varphi})$ kertominen on sama kuin kierto kulman $\varphi$
verran. Sekä tälle operaatiolle että skalaarilla $r$ kertomiselle pätee yhteenlaskun suhteen 
osittelulaki (vrt.\ ristitulon vastaavan osittelulain todistus Luvussa \ref{ristitulo}), joten 
väitetty osittelulaki seuraa.

Kun nyt $\Pkunta$:ssä on tullut määritellyksi sekä yhteenlasku että kertolasku, jotka 
toteuttavat normaalit vaihdanta-, liitäntä- ja osittelulait, ja lisäksi on konstruoitu
yhteenlaskun nolla-alkio ja vasta-alkio sekä kertolaskun ykkösalkio ja käänteisalkio, niin on 
tullut osoitetuksi, että $(\Pkunta,+,\cdot)$ on itse asiassa \pain{kunta}. Nähdään siis, että 
osoitintulon mukaan ottaminen muuttaa osoitinalgebran varsin radikaalisti. Osoittimia voikin
luonnehtia 'pyöriviksi luvuiksi', joita vain lasketaan yhteen kuten vektoreita. --- Erillistä
skalaarilla kertomisoperaatiota ei osoitintulon määrittelyn jälkeen enää tarvita, sillä
skalaarin $\lambda\in\R$ voi tulkita osoittimeksi
\[
\pointer{\lambda} \,=\, \begin{cases}
                        \,\lambda \vkulma{0},        &\text{jos }\ \lambda \ge 0, \\
                        \,\abs{\lambda}\vkulma{\pi}, &\text{jos }\ \lambda <0,
                        \end{cases}
\]
jolloin skaalaus on osoitintulon erikoistapaus: $\lambda\ptp=\pointer{\lambda}\ptp$.
 
Osoitintulolle saadaan varsin yksinkertainen esitystapa myös komponenttimuodossa. Nimittäin
jos
\begin{align*}
\ptp_1 &= r_1 \cos{\varphi_1} \ptr + r_1 \sin{\varphi_1} \pti = x_1 \ptr + y_1 \pti, \\
\ptp_2 &= r_2 \cos{\varphi_2} \ptr + r_2 \sin{\varphi_2} \pti = x_2 \ptr + y_2 \pti,
\end{align*}
niin osoitintulon määritelmästä ja trigonometristen funktioiden yhteenlaskukaavoista 
(ks.\ Luku \ref{trigonometriset funktiot}) seuraa
\begin{align*}
\ptp_1 \cdot \ptp_2 &= r_1r_2 \cos({\varphi_1 + \varphi_2})\,\ptr 
                                + r_1r_2 \sin({\varphi_1 + \varphi_2})\,\pti \\
                    &= r_1r_2 (\cos{\varphi_1} \cos{\varphi_2} 
                                - \sin{\varphi_1} \sin{\varphi_2})\,\ptr \\
                    & \qquad + r_1r_2 (\sin{\varphi_1} \cos{\varphi_2} 
                                + \cos{\varphi_1} \sin{\varphi_2})\,\pti \\
                    &= (x_1x_2-y_1y_2)\ptr + (x_1y_2 + y_1x_2)\pti.
\end{align*}
Näin ollen osoitintuloa vastaa $\Rkaksi$:ssä määritelty tulo
\begin{equation} \label{kertolasku}
\boxed{\kehys\quad (x_1,y_1) \cdot (x_2,y_2) = (x_1x_2-y_1y_2,\,x_1y_2 + y_1x_2). \quad}
\end{equation}
Yhdessä jo aiemmin määritellyn (vektorien) yhteenlaskun kanssa, ts.
\begin{equation} \label{yhteenlasku}
\boxed{\kehys\quad (x_1,y_1) +(x_2,y_2) = (x_1 + x_2,\,y_1 + y_2) \quad}
\end{equation}
on lukupareista saatu aikaan algebra $(\Rkaksi,+,\cdot)$, joka siis on kunta (!). Kunnan
$(\Rkaksi,+,\cdot)\,$ nolla- ja ykkösalkiot ovat
\[
(0,0)\vastaa\pointer{0}, \quad (1,0)\vastaa\pointer{1}.
\]

\Harj
\begin{enumerate}

\item \label{H-III-1: osoitintulon liitäntälaki}
Todista osoitintulon liitäntälaki $\,(\ptp_1\ptp_2)\ptp_3=\ptp_1(\ptp_2\ptp_3)$.

\item
Näytä suoraan osoitintulon määritelmästä, että pätee: \newline
a) \ $-\ptp=(-\pointer{1})\ptp \quad$ 
b) \ $\pointer{0}\ptp=\pointer{0} \quad$
c) \ $\ptp\ptq=\pointer{0}\ \impl\ \ptp=\pointer{0}\ \tai\ \ptq=\pointer{0}$

\item
Määritä seuraavien osoitinyhtälöiden kaikki ratkaisut: \newline
a) \ $(\ptp)^2=-\pointer{1} \quad$ 
b) \ $2(\ptp)^3=\pointer{1} \quad$
c) \ $(\ptp)^4+3\ptp=\pointer{0} \quad$
d) \ $(\ptp)^4=4\vkulma\pi$

\item
Olkoon $(x,y)\in\Rkaksi,\ (x,y) \neq (0,0)$. Lähtien lukuparien tulon määritelmästä 
määritä $(a,b)\in\Rkaksi$ siten, että $(x,y)\cdot(a,b)=(1,0)$, ts.\ $(a,b)=(x,y)^{-1}$.
 
\end{enumerate} % Osoitinkunta
\section{Kompleksiluvut ja niillä laskeminen} 
\label{kompleksiluvuilla laskeminen}
\alku
\sectionmark{Kompleksiluvut}
\index{laskuoperaatiot!e@kompleksilukujen|vahv}

\kor{Kompleksilukujen} määrittelyn lähtökohdaksi otetaan matematiikassa usein kunta 
$(\Rkaksi,+,\cdot)$, jossa laskutoimitukset on määritelty edellisen luvun säännöillä 
\eqref{yhteenlasku} ja \eqref{kertolasku}. Suoraviivaisin määritelmä on yksinkertaisesti sopia,
että $(\Rkaksi,+,\cdot)$ itse on kompleksilukujen kunta, ts. kompleksiluvut ovat $\Rkaksi$:n 
lukupareja, joiden väliset laskutoimitukset on määritelty mainitulla tavalla. Käytännön 
laskurutiineissa käytetään kuitenkin yleensä havainnollisempia kompleksilukujen esitystapoja. 
Näitä ovat edellä kuvattu polaarimuoto (osoitinmuoto) tai sitäkin yleisempi,
komponenttimuodosta johdettu kolmas esitystapa, joka seuraavassa otetaan kompleksilukujen
käytännöllisen määrittelyn lähtökohdaksi.

Aloitetaan merkitsemällä $\Rkaksi$:n '$x$-akselia' symbolilla $\Rkaksi_0$:
\[
\Rkaksi_0 = \{(x,0) \ | \ x \in \R\}.
\]
Havaitaan, että $\Rkaksi_0$ sisältää $(\Rkaksi,+,\cdot)$:n nolla-alkion $(0,0)$, samoin 
ykkösalkion $(1,0)$. Lisäksi jos $(x,0)$, $(y,0)$ ovat mitkä tahansa kaksi $\Rkaksi_0$:n 
alkiota, niin edellisen luvun laskusääntöjen perusteella näiden summa, erotus, tulo ja
osamäärä (jos $y \neq 0$) ovat myös $\Rkaksi_0$:ssa. Nämä tosiasiat yhdessä merkitsevät, että 
$(\Rkaksi_0,+,\cdot)$ on itsekin kunta ja siis $(\Rkaksi,+,\cdot)$:n \pain{alikunta} 
(vrt.\ Luku \ref{kunta}). Toisaalta havaitaan, että $\Rkaksi_0$:n ja  $\R$:n välinen ilmeinen
vastaavuus
\[
(x,0) \in \Rkaksi_0 \ \leftrightarrow \ x \in \R
\]  
ulottuu samanlaisena laskutoimituksiin: Jos $\Rkaksi_0$:ssa suoritetaan laskutoimitus kunnan 
$(\Rkaksi,+,\cdot)$ säännöillä, niin tätä vastaa $(\R,+,\cdot)$:n laskutoimitus normaalissa 
reaalilukujen kunnassa. Esimerkiksi kertolasku
\[
(x_1,0) \cdot (x_2,0) = (x_1x_2,0)
\]
(sovellettu edellisen luvun sääntöä \eqref{kertolasku}) vastaa kertolaskua $\R$:ssä, kun 
käytetään em.\ vastaavuusperiaatetta:
\[
\begin{array}{rccccc}
\Rkaksi_0: &(x_1,0) &\cdot &(x_2,0) &= &(x_1x_2,0) \\
&\updownarrow & &\updownarrow & &\updownarrow\\ 
\R: &x_1 &\cdot &x_2 &= &x_1x_2
\end{array}
\]
Sama pätee yhteelaskulle, joten kunnilla $(\Rkaksi_0,+,\cdot)$ ja $(\R,+,\cdot)$ ei käytännön 
laskennan kannalta ole mitään eroa. Tehdäänkin tämän perusteella kunnassa $(\Rkaksi,+,\cdot)$ 
samastussopimus
\[ (x,0) = x \quad \forall x \in \R. \]
Tällä sopimuksella reaalilukujen kunnasta $(\R,+,\cdot)$ tulee kunnan $(\Rkaksi,+,\cdot)$ 
alikunta. Merkitään vielä $(0,1)=i$, 
jolloin pätee
\begin{align*} 
(x,y) = (x,0) + (0,y) &= (x,0) + (y,0)(0,1) \\
                      &= x + yi = x+iy. 
\end{align*}
Kunnan $(\Rkaksi,+,\cdot)$ kertolaskun määritelmän ja tehdyn samastussopimuksen mukaan on
\[ 
i^2 = i \cdot i = (-1,0) = -1. 
\]
\begin{Def} \vahv{(Kompleksiluvut)} \label{kompleksilukujen määritelmä}
\index{kompleksiluvut|emph} \index{imaginaariluku $i$|emph}
Kompleksilukujen joukko $\C$ sisältää reaaliluvut ja lisäksi nk.\ \kor{imaginaariluvun} $i$,
joka toteuttaa
\[
i^2=i \cdot i = -1 \in \R. 
\]
Kompleksiluvut ovat kunta $(\C,+,\cdot)$, missä jokaisella $z\in\C$ on yksikäsitteinen
esitysmuoto
\[
z=x+iy, \quad x,y \in \R.
\]
Erityisesti on $x+0i=x\in\R$ ja $0+1i=i$.
\end{Def}
Määritelmän mukaisesti $(\C,+,\cdot)$ syntyy reaalilukujen kunnan $(\R,+,\cdot)$ laajennuksena,
kun lukujoukkoon $\R$ lisätään imaginaariluku $i$ ja huomioidaan kunnan perusaksiooma (K0)
--- vrt.\ Luvun \ref{kunta} Esimerkki \ref{muuan kunta}, jossa kuntaa $(\Q,+,\cdot)$ 
laajennettiin vastaavalla tavalla lisäämällä luku $a=\sqrt{2}$. Luvulle $i$ asetettu ehto 
$i^2=-1$ ei toteudu millekään $i\in\R$, joten $i$ on 'aidosti imaginaarinen' luku. 
Osoitinkunnassa $(\Pkunta, + , \cdot)$ laskusääntöä $i^2=-1$ vastaa tulos
\[
\pti \cdot \pti = 1 \vkulma{\pi} = -(1 \vkulma{0}) = - \pointer{1}.
\]

Määritelmän \ref{kompleksilukujen määritelmä} perusteella kompleksiluvuilla voi operoida 
jokseenkin normaalisti, eli reaalilukujen algebrasta tutulla tavalla. Normaalista poikkeaa
vain laskusääntö $i^2=-1$. Laskennassa kompleksiluvun esitysmuotoja voi vaihdella vapaasti 
edellisen luvun muunnossääntöjen puitteissa. Tavallisimmin käytetään joko Määritelmän 
\index{perusmuoto!bb@kompleksiluvun}%
\ref{kompleksilukujen määritelmä} mukaista kompleksiluvun \kor{perusmuotoa}, tai sitten
polaarimuotoa. Saman luvun eri esitysmuotojen välillä käytetään jatkossa joko
vastaavuusmerkintää '$\vastaa$' tai, sikäli kuin luontevaa, yksinkertaisesti samastusta '$=$'.
\begin{Exa} Saata kompleksiluku $(2+3i)\cdot(5-2i)$ perusmuotoon $x+iy$. \end{Exa}
\ratk Määritelmän \ref{kompleksilukujen määritelmä} perusteella
\[
(2+3i)\cdot(5-2i) = 2\cdot 5 - 2\cdot 2i + 5\cdot 3i - 3\cdot 2\cdot i^2 = 16+11i. \loppu
\]
\begin{Exa} Olkoon $z=a+ib \neq 0$. Laske $z^{-1}$ perusmuodossa. \end{Exa}
\ratk Jos $z^{-1}=x+iy$, niin on oltava
\begin{align*}
&(a+ib)(x+iy)=1 \\
\ekv \ &(ax-by)+i(bx+ay)=1=1+0i \\
\ekv \ &\left\{ \begin{array}{ll}
ax-by=1 \\
bx+ay=0
\end{array} \right. \\
\ekv \ & x=\frac{a}{a^2+b^2}\,, \quad y=-\frac{b}{a^2+b^2}\,.
\end{align*}
Siis
\[
z^{-1}=\frac{1}{a^2+b^2}(a-ib) = \frac{a}{a^2+b^2} - \frac{b}{a^2+b^2}\,i. \loppu
\]
\begin{Def} \label{kompleksilukujen terminologiaa}
\index{reaaliosa|emph} \index{imaginaariosa|emph} \index{itseisarvo|emph}
\index{vaihekulma|emph} \index{napakulma|emph} \index{liittoluku|emph} \index{konjugaatti|emph}
\index{moduuli (kompleksiluvun)|emph} \index{argumentti (kompleksiluvun)|emph}
Kompleksiluvun
\[
z=x+iy \in \C \ \vastaa \ (x,y) \in \Rkaksi \ \vastaa \ r \vkulma{\varphi} \in \Pkunta
\]
\begin{itemize}
\item[-] \kor{reaaliosa} on $\ \re z = x = r \cos{\varphi}$.
\item[-] \kor{imaginaariosa} on $\ \im z = y = r \sin{\varphi}$.
\item[-] \kor{itseisarvo} (moduuli) on $\ \abs{z}=\sqrt{x^2+y^2}=r$.
\item[-] \kor{vaihekulma} (argumentti, napakulma) on $\ \arg z = \varphi$.
\item[-] \kor{liittoluku} eli \kor{konjugaatti} on 
              $\overline{z}=x-iy \ \vastaa \ (x,-y) \ \vastaa \ r \vkulma{-\varphi}\,$.
\end{itemize}
\end{Def}
Kompleksiluvuilla laskettaessa seuraavia Määritelmään \ref{kompleksilukujen terminologiaa} 
liittyviä kaavoja tarvitaan usein:
\[ \boxed{ \begin{aligned}
\quad\ykehys &\text{(1)} \qquad \overline{z_1+z_2} = \overline{z_1} + \overline{z_2} \\
             &\text{(2)} \qquad \overline{z_1z_2} = \overline{z_1}\,\,\overline{z_2} \\
             &\text{(3)} \qquad z \overline{z} = \abs{z}^2 \\
             &\text{(4)} \qquad \abs{z_1z_2} = \abs{z_1} \abs{z_2} \\
             &\text{(5)} \qquad \abs{z^{-1}} = \abs{z}^{-1} \\
             &\text{(6)} \qquad \abs{\overline{z}} = \abs{z} \\
             &\text{(7)} \qquad z+\overline{z} = 2\,\text{Re} \, z, \quad z-\overline{z} 
                                               = 2i\,\text{Im} \, z \qquad\\
             &\text{(8)} \qquad \overline{\overline{z}} = z \akehys
           \end{aligned} } \]
Nämä voi helposti perustella määritelmistä (kaavat (2)--(5) suorimmin polaarimuodosta). 

\jatko \begin{Exa} (jatko) Esimerkissä laskettiin kompleksiluvun $z \neq 0$ käänteisluku 
$z^{-1}$. Koska $z \neq 0\,\ekv\,\overline{z} \neq 0$, niin kaavaa (3) käyttäen voidaan
päätellä myös seuraavasti:
\begin{align*}
           & zz^{-1} = 1 \\
\ekv \quad &\overline{z}(zz^{-1}) = \overline{z} \\
\ekv \quad &(\overline{z}z)z^{-1} = \overline{z} \\
\ekv \quad &\abs{z}^2z^{-1} = \overline{z} \\
\ekv \quad &z^{-1} = \abs{z}^{-2} \overline{z}.
\end{align*}
Tämä vastaa normaalia lavennusmenettelyä
\[
z^{-1} = \frac{1}{a+ib} = \frac{(a - ib)}{(a+ib)(a-ib)} 
= \frac{1}{a^2+b^2}(a-ib)\,.
\]
Vielä nopeampi on kuitenkin osoitinlasku:
\[
z = r \vkulma \varphi\ \impl\ z^{-1} = r^{-1} \vkulma{-\varphi} 
  = r^{-2}(r \vkulma{-\varphi}) = \abs{z}^{-2} \overline{z}. \loppu
\]
\end{Exa}
\begin{Exa} Saata $z=(1+i)^7$ perusmuotoon $x+iy$. \end{Exa}
\ratk Tässäkin polaariesitys on tehokkain: Koska
\[
1+i = r \vkulma{\varphi}\,,\quad \text{missä}\ \quad r
    = \sqrt{1+1} = \sqrt{2}, \quad \varphi = \frac{\pi}{4}\,, \\
\]
niin
\begin{align*}
(1+i)^7 \ &=\ (\sqrt{2})^7 \vkulma{7\pi/4} \\
          &=\ 8 \sqrt{2}\,\vkulma{-\pi/4} \\
          &=\ 8 \sqrt{2}\,(\cos\tfrac{\pi}{4} - i\,\sin\tfrac{\pi}{4}) \\[1mm]                
          &=\ 8(1-i). \loppu
\end{align*}
Kompleksialgebran kaavoista maininnan arvoinen on vielä
\index{de Moivren kaava}%
\kor{de Moivren kaava}
\[
\boxed{\kehys\quad \text{(9)} \qquad (\cos \varphi + i \sin \varphi)^n 
                                        = \cos n \varphi + i \sin n \varphi. \quad}
\]
Tämä seuraa välittömästi, kun osoitinlaskennan tulos
\[
(1 \vkulma{\varphi})^n = 1 \vkulma{n \varphi}
\]
esitetään kompleksilukujen perusmuodossa.
\begin{Exa} de Moivren kaavan ja binomikaavan mukaan
\begin{align*}
\cos 3 \varphi + i \sin 3 \varphi\ &=\ (\cos \varphi + i \sin \varphi)^3 \\
&=\ \cos^3 \varphi + 3i \cos^2 \varphi \sin \varphi + 3 i^2 \cos \varphi \sin^2 \varphi 
                                                    +i^3 \sin^3 \varphi \\
&=\ (\cos^3 \varphi - 3\cos \varphi \sin^2 \varphi) + i(3 \cos^2 \varphi \sin \varphi 
                                                    - \sin^3 \varphi) \\[3mm]
\impl \ &\left\{
\begin{aligned}
\cos 3 \varphi\ &=\ \cos^3 \varphi - 3 \cos \varphi \sin^2 \varphi, \\
\sin 3 \varphi\ &=\ 3 \cos^2 \varphi \sin \varphi - \sin^3 \varphi
\end{aligned} \right.
\end{align*}
(vrt. Luvun \ref{trigonometriset funktiot} Esimerkki \ref{sin kolme alpha}). \loppu
\end{Exa}

\subsection{Kolmioepäyhtälö $\C$:ssä}
\index{kolmioepäyhtälö!e@kompleksilukujen|vahv}

Kompleksilukujen yhteenlasku on samanlainen operaatio kuin tason vektorien yhteenlasku, ja
myös kompleksiluvun itseisarvo vastaa vektorin itseisarvoa. Tästä syystä kompleksiluvuille
pätee myös samaa muotoa oleva kolmioepäyhtälö kuin vektoreille:
\[
\boxed{\kehys\quad \abs{\abs{z_1}-\abs{z_2}} \le \abs{z_1 + z_2} \leq \abs{z_1} + \abs{z_2}, 
                                                                 \quad z_1,z_2 \in \C. \quad}
\]

Kolmioepäyhtälö on hyvin keskeinen työkalu kompleksilukuihin perustuvassa matemaattisessa
analyysissä eli \kor{kompleksianalyysissä}. Tyypillisenä esimerkkinä kolmioepäyhtälön käytöstä 
tarkasteltakoon väittämää, joka koskee yleistä
\index{kompleksimuuttujan!a@polynomi} \index{polynomi (kompleksimuuttujan)}%
\kor{kompleksimuuttujan polynomia} muotoa
\[
p(z) = c_0 + c_1z + \cdots + c_nz^n = \sum_{k=0}^{n} c_k z^k, \quad c_n \neq 0. 
\]
Tässä siis $z \in \C$ on \kor{kompleksimuuttuja} ja myös luvut $c_k$, eli polynomin
\index{kerroin (polynomin)} \index{aste (polynomin)}% 
\kor{kertoimet}, ovat kompleksilukuja. Luku $n\in\N\cup\{0\}$ on polynomin \kor{aste}.

Kun kompleksimuuttujan polynomiin sovelletaan kolmioepäyhtälöä sekä em.\ kaavoja (4),\,(5), 
tullaan väittämään, jonka mukaan polynomin itseisarvo $\abs{p(z)}$ kasvaa riittävän suurilla 
$\abs{z}$:n arvoilla kvalitatiivisesti samalla tavoin kuin polynomin korkeimman asteisen
termin itseisarvo, eli riittävän suurilla $\abs{z}$:n arvoilla
\[ 
\abs{p(z)} \sim \abs{c_n z^n} = \abs{c_n}\abs{z}^n 
\]
(tässä on käytetty kaavaa (4)). Väittämä voidaan muotoilla täsmällisemmin esim.\ seuraavasti
(ks.\ myös Harj.teht.\,\ref{H-III-2: polynomin kasvu}):
\begin{Prop} \label{polynomin kasvu} \vahv{(Polynomin kasvu)} Jos 
$p(z) = \sum_{k=0}^{n} c_k z^k$, missä  $c_k \in \C$, $n \in \N$ ja $c_n \neq 0$, niin on
olemassa $R \in \R_+\,$ siten, että pätee
\[
\frac{1}{2}\abs{c_n}\abs{z}^n\,\le\,\abs{p(z)}\,\le\,\frac{3}{2}\abs{c_n}\abs{z}^n,\quad 
                                        \text{kun} \ z \in \C\ \text{ja}\ \abs{z} \ge R.
\]
\end{Prop}
\tod Koska $c_n \neq 0$, voidaan kirjoittaa
\[
p(z) = c_n z^n(1 + b_{n-1}z^{-1} + \cdots + b_0z^{-n}),
\]
missä $b_k = c_k / c_n$. Käyttämällä kolmioepäyhtälön ensimmäistä ja toista osaa muodoissa
\[
\abs{z_1+z_2} \ge \abs{z_1}-\abs{z_2}, \quad -\abs{z_1 + z_2} \geq - \abs{z_1} - \abs{z_2}
\]
sekä kaavoja (4),\,(5), päätellään
\begin{align*}
\abs{p(z)} &=   \abs{c_n}\abs{z}^n \abs{1 + b_{n-1}z^{-1} + \cdots + b_0 z^{-n}} \\
           &\ge   \abs{c_n}\abs{z}^n (1 - \abs{b_{n-1}z^{-1} + \cdots + b_0 z^{-n}}) \\
           &\ge \abs{c_n}\abs{z}^n (1 - \abs{b_{n-1}z^{-1}} - \cdots - \abs{b_0 z^{-n}}) \\
           &= \abs{c_n}\abs{z}^n \left(1 - \frac{\abs{b_{n-1}}}{\abs{z}} - \cdots 
                                                     - \frac{\abs{b_0}}{\abs{z}^{n}}\right).
\end{align*}
Kun
\[
\abs{z} \geq \max \{1, \ 2(\abs{b_{n-1}} + \cdots + \abs{b_0}) \} = R,
\]
niin pätee
\[
\frac{\abs{b_{n-1}}}{\abs{z}} + \cdots + \frac{\abs{b_0}}{\abs{z}^n} \leq
\frac{\abs{b_{n-1}} + \cdots + \abs{b_0}}{\abs{z}} \leq \frac{1}{2}\,,
\]
joten seuraa väittämän ensimmäinen osa:
\[
\abs{p(z)} \ge \frac{1}{2} \abs{c_n}\abs{z}^n, \quad \text{kun} \ \abs{z} \geq R.
\]
Jälkimmäinen osa seuraa vastaavasti soveltamalla kolmioepäyhtälön jälkimmäistä osaa. \loppu
\begin{Exa} Jos $p(z) = z^5 -4i\,z^3 + 10z + (30+40i)$, niin em.\ todistuksen logiikkaa 
seuraamalla nähdään, että
\begin{align*}
\abs{p(z)} &\ge \abs{z}^5
                \left(1 - \frac{\abs{5i}}{\abs{z}^2} - \frac{10}{\abs{z}^4} 
                                                     - \frac{\abs{30+40i}}{\abs{z}^5}\right) \\
           &=   \abs{z}^5\left(1 - \frac{5}{\abs{z}^2} - \frac{10}{\abs{z}^4} 
                                                       - \frac{50}{\abs{z}^5}\right) \\
           &\ge \abs{z}^5\left(1 - \frac{5}{\abs{z}} - \frac{10}{\abs{z}} 
                                                     - \frac{50}{\abs{z}}\right) \\
           &=   \abs{z}^5\left(1 - \frac{65}{\abs{z}}\right), \quad \text{kun}\ \abs{z} \ge 1.
\end{align*}
Vastaavasti päätellään, että $\abs{p(z)}\le\abs{z}^5(1+65\abs{z}^{-1})$, kun $\abs{z} \ge 1$,
joten seuraa
\[
\frac{1}{2}\abs{z}^5\,\le\,\abs{p(z)}\,\le\,\frac{3}{2}\abs{z}^5, \quad 
                                            \text{kun}\ \abs{z} \ge 130 = R. \loppu
\]
\end{Exa}

\Harj
\begin{enumerate}

\item
Muunna seuraavat kompleksiluvut perusmuotoon $x+iy$ annetusta osoitinmuodosta tai 
osoitinmuotoon annetusta perusmuodosta. Käytä tarvittaessa likiarvoja.
\[
\text{a)}\,\ \sqrt{2}\vkulma{-\tfrac{5\pi}{4}} \quad\
\text{b)}\,\ \sqrt{3}-2i \quad\
\text{c)}\,\ 2\vkulma{700\aste} \quad\
\text{d)}\,\ -3-4i 
\]

\item
Saata seuraavat kompleksiluvut perusmuotoon $x+iy$ ja polaarimuotoon: \newline
a) \ $(1+i)(1-i)^5 \quad$ 
b) \ $(1-i)(1+i\sqrt{3})^{-1} \quad$
c) \ $(\sqrt{2}+1+i)^8$

\item
Laske itseisarvo ja vaihekulma käyttäen polaarimuotoa: \newline
a) \ $(1+i)^6 \quad$ 
b) \ $(1-i\sqrt{3})(1-i)^2 \quad$
c) \ $(2-2i)(\sqrt{3}+i)^{-2}$

\item
Olkoon $z=\tfrac{1}{2}(1+i\sqrt{3})$. Millä $n\in\Z$ pätee\, a) $z^n=z$, \ b) $z^n=-z$\,?

\item
Kompleksilukujen itseisarvoille pätee $\abs{z_1 z_2}^2=\abs{z_1}^2\abs{z_2}^2$. Tarkista kaavan
pätevyys suoraan kompleksilukujen perusmuodosta, eli kirjoittamalla \newline
$z_1=x_1+iy_1\,, \ z_2=x_2+iy_2\,$.

\item
Johda de Moivren kaavan avulla: \newline 
a) $\sin 5x$:lle lauseke $\sin x$:n polynomina, \newline
b) $\sin^5x$:lle lauseke muodossa $\,a\sin x+b\sin 3x+c\sin 5x,\ a,b,c\in\R$.

\item
Jos kompleksilukujen kunnassa on määritelty järjestysrelaatio, niin ensimmäisen
järjestysaksiooman (J1) mukaan on oltava joko $i>0$, $i=0$ tai $i<0$. Päättele, että
järjestysrelaatiota ei voi määritellä.

\item
Määritä jokin $R$ siten, että pätee
\[
\abs{z}^5\,\le\,\abs{2z^5+1000z^4-10^4(3+4i)}\,\le\,3\abs{z}^5, \quad 
                                                                    \text{kun} \abs{z} \ge R.
\]

\item (*)
Todista, että kaikilla $z_1,z_2\in\C$ pätee $(1+\abs{z_1}^2)(1+\abs{z_2}^2)\ge\abs{1+z_1z_2}^2$.

\item (*) \label{H-III-2: polynomin kasvu}
Näytä, että Propositon \ref{polynomin kasvu} oletuksin on olemassa $R\in\R_+$ siten, että
\[
0.999\abs{c_n}\abs{z}^n\,\le\,\abs{p(z)}\,\le\,1.001\abs{c_n}\abs{z}^n,\quad 
                                        \text{kun} \ z \in \C\ \text{ja}\ \abs{z} \ge R.
\]
Millaisesta vielä yleisemmästä väittämästä tämä on erikoistapaus?

\end{enumerate} % Kompleksiluvut
\section{Algebran peruslause} \label{III-3}
\alku

\index{funktio A!e@kompleksifunktio}%
Kompleksianalyysin alalaji \kor{funktioteoria} tutkii kompleksimuuttujan kompleksiarvoisia 
funktioita tyyppiä
\[
f: \C \kohti \C \quad \text{tai} \quad f: A \kohti \C, \quad A\subset\C.
\]
Yksinkertaisin (vaan ei vähäisin) esimerkki tällaisesta funktiosta on edellisessä luvussa
(Propositio \ref{polynomin kasvu}) tarkasteltu, koko $\C$:ssä määritelty polynomifunktio
\begin{equation} \label{polynomi}
p(z) = \sum_{k=0}^n c_k z^k, \quad c_k \in \C, \ c_n \neq 0.
\end{equation}

Jokainen kompleksimuuttujan kompleksiarvoinen funktio on esitettävissä muodossa
\[
f(z) = \re f(z) + i\,\im f(z) = u(x,y) + iv(x,y), \quad z = x + iy,
\]
missä $u,v$ ovat funktioita tyyppiä $u,v: \Rkaksi\kohti\R$ 
(vrt.\ Luku \ref{trigonometriset funktiot}). Itse $f$ on siis myös tulkittavissa funktioksi
tyyppiä $f: \Rkaksi\kohti\C$. Kompleksimuuttujan funktioita onkin luontevaa ajatella
geometrisesti tasossa määritellyiksi. Tasoa sanotaan tällaisissa tarkasteluissa
\index{kompleksitaso}%
\kor{kompleksitasoksi}. Kompleksitasossa siis piste $(x,y)$ tarkoittaa kompleksilukua
$z = x +iy$.
\begin{figure}[H]
\setlength{\unitlength}{1cm}
\begin{center}
\begin{picture}(8,6)(-1,-1)
\put(-1,0){\vector(1,0){8}} \put(5.5,-0.5){$x=\text{Re}\,z$}
\put(0,-1){\vector(0,1){6}} \put(0.2,4.8){$y=\text{Im}\,z$}
\put(3.9,2.9){$\bullet$} \put(4.2,2.9){$z=x+iy$}
\dashline{0.2}(0,3)(4,3) \dashline{0.2}(4,3)(4,0)
\put(3.9,-0.4){$x$} \put(-0.4,2.9){$y$}
\end{picture}
\end{center}
\end{figure}
Kompleksifunktion
\index{juuri (kompleksifunktion)}%
$f$ \kor{juureksi} (engl. root) tai yksinkertaisesti \kor{nollakohdaksi} 
sanotaan jokaista $z \in \C$, jolle
\[
f(z) = 0.
\]
Reaali- ja imaginaariosiin hajotetusta muodosta juuret $z = x +iy$ saadaan yhtälöryhmän
\[
\begin{cases}
 u(x,y) = 0 \\ v(x,y) = 0
\end{cases}
\]
ratkaisuista.
\begin{Exa} \label{kompleksipolynomin juuret} $f(z) = z^3 +1$. Juuret? \end{Exa}
\ratk Tässä
\begin{align*}
f(x+iy) &= (x+iy)^3 + 1 \\
&= (x^3 -3xy^2 +1) + (3x^2y -y^3)i,
\end{align*}
joten $f$:n nollakohdat ratkeavat yhtälöryhmästä
\[
\left\{ \begin{array}{ll}
x^3 -3xy^2 +1 = 0, \\
3x^2y -y^3 = 0.
\end{array} \right.
\]
Jokaista yhtälöryhmän reaalista ratkaisua $(x,y) \in \Rkaksi$ vastaa $f$:n juuri $x+iy \in \C$.
Juuret löydetään tässä helposti, koska yhtälöryhmän jälkimmäinen yhtälö on kirjoitettavissa 
muotoon $(3x^2-y^2)y=0$. Näin ollen on oltava joko $y=0$ tai 
$y^2 = 3x^2\ \ekv\ y = \pm\sqrt{3}\,x$. Sijoittamalla nämä edelliseen yhtälöön saadaan juuriksi
\begin{multicols}{2} \raggedcolumns
\begin{align*}
(x_1,y_1) = (-1,0)\qquad\,\ &\vastaa\ z_1 = -1, \\[3mm]
(x_2,y_2) = \left(\frac{1}{2}\,,\frac{\sqrt{3}}{2}\right)\quad\,  
                            &\vastaa\ z_2 = \frac{1}{2}\left(1 +i\sqrt{3}\right), \\
(x_3,y_3) = \left(\frac{1}{2}\,,-\frac{\sqrt{3}}{2}\right)\ 
                            &\vastaa\ z_3 = \frac{1}{2}\left(1 -i\sqrt{3}\right).
\end{align*}
\begin{figure}[H]
\setlength{\unitlength}{1cm}
\begin{center}
\begin{picture}(6,6)(-3.5,-3)
\put(-3,0){\vector(1,0){6}} \put(2.5,-0.5){$\text{Re}\,z$}
\put(0,-3){\vector(0,1){6}} \put(0.2,2.8){$\text{Im}\,z$}
\Thicklines
\put(2,0){\line(0,-1){0.3}} \put(1.9,-0.8){$1$}
\put(-2,0){\line(0,-1){0.3}} \put(-2.43,-0.8){$-1$}
\thinlines
\put(0,0){\bigcircle{4}}
\put(-2.1,-0.1){$\bullet$} \put(-1.8,0.2){$z_1$}
\put(0.9,1.63){$\bullet$} \put(1.1,1.8){$z_2$}
\put(0.9,-1.83){$\bullet$} \put(1.1,-2){$z_3$}
\end{picture}
\end{center}
\end{figure}
\end{multicols}

Esimerkissä $f$ oli polynomi astetta 3 ja sille löydettiin kolme juurta. Tulos on
erikoistapaus lauseesta, joka on sekä laajakantoinen että syvällinen:
\begin{*Lause} \vahv{(Algebran peruslause)} \label{algebran pl}
\index{Algebran peruslause|emph} Jos $p \, $ on polynomi muotoa \eqref{polynomi} ja astetta
$n \geq 1$, niin on olemassa $n$ kompleksilukua $z_1, \ldots, z_n$ siten, että pätee
\begin{align*}
p(z) &= c_n(z-z_1) \cdot \cdots \cdot (z-z_n) \\
     &= c_n \prod_{i=1}^n (z-z_i).
\end{align*}
\end{*Lause} 
Lauseen \ref{algebran pl} mukaisessa tulohajotelmassa luvut $z_i$ eivät välttämättä ole eri
suuret, joten polynomissa voi olla tekijä muotoa
\[
(z -z_k)^m, \quad 1 \leq m \leq n,
\]
liittyen juureen $z_k$. Jos $p$:llä on tämä tekijä mutta ei tekijää $(z -z_k)^{m+1}$, niin 
sanotaan, että juuri $z_k$ on \kor{$m$-kertainen} (engl. $m$-fold) tai että juuren 
\index{kertaluku!a@juuren (nollakohdan)}
\kor{kertaluku} (engl. order) on $m$. Jos kertaluku on $m=1$, sanotaan, että juuri on 
\index{yksinkertainen!a@nollakohta (juuri)}%
\kor{yksinkertainen} (engl. simple). Jos samaa juurta edustavat tekijät kootaan yhteen,
saadaan Lauseen \ref{algebran pl} tulohajotelmalle muoto
\begin{equation} \label{tulohajotelma}
p(z) = c_n \prod_{i=1}^ \nu (z -z_i)^{m_i},
\end{equation}
missä nyt $z_i \neq z_j$ kun $i \neq j$, $m_i \in \N$ ja
\[
\sum_{i=1}^ \nu m_i = n.
\]
Erityisesti jos hajotelmassa \eqref{tulohajotelma} on $\nu = n$, eli polynomilla on $n$ eri 
juurta, on kaikkien juurien oltava yksinkertaisia. Tämä tilanne oli Esimerkissä 
\ref{kompleksipolynomin juuret}.

Esimerkissä \ref{kompleksipolynomin juuret} aidosti kompleksiset juuret
($\text{Im}\,z \neq 0$) esiintyivät \kor{konjugaattiparina}. Näin on yleisemmin silloin, kun
(kuten esimerkissä) polynomi on \kor{reaalikertoiminen}, ts. $c_k \in \R$ lausekkeessa
\eqref{polynomi}. Tämä perustuu siihen, että reaalikertoimisen polynomin tapauksessa pätee
(vrt. edellisen luvun kaavat (1)--(2))
\[
\overline{p(z)} = p(\overline{z}) \quad (\text{reaalikertoiminen polynomi}).
\]
Tällöin
\[
p(z) = 0 \ \impl \ p(\overline{z}) = 0,
\]
ja juuret siis 'pariutuvat'. Jos konjugaattiparin muodostavat juuret
\[
z= a \pm ib,
\]
niin polynomissa on tekijänä
\[
(z-a-ib)(z-a+ib) = (z-a)^2 +b^2.
\]
Reaalikertoiminen polynomi voidaan siis aina hajottaa reaalikertoimisiin tekijöihin, jotka
ovat joko muotoa
\begin{align*}
&\text{(a)}\qquad (z-a_k)^m, \quad a_k \in \R,
\intertext{vastaten $m$-kertaista reaalijuurta $z_k=a_k$, tai muotoa}
&\text{(b)}\qquad [(z-a_k)^2+b_k^2\,]^m \quad a_k, b_k \in \R, \ b_k \neq 0,
\end{align*}
vastaten $m$-kertaista konjugaattijuuriparia $z_k=a_k \pm ib_k$.
\jatko \begin{Exa} (jatko) Esimerkin polynomi hajoaa muotoon
\begin{align*}
z^3+1 &= (z+1)(z^2-z+1) \\
&= (z+1)\left[\left(z-\frac{1}{2}\right)^2+\left(\frac{\sqrt{3}}{2}\right)^2\right]. \loppu
\end{align*}
\end{Exa}
'Algebran peruslause' edellä esitetyssä muodossa on itse asiassa oikean 
\kor{Algebran peruslauseen} tekninen seuraamus. Oikea peruslause muotoillaan:
\begin{*Lause} \vahv{(Algebran peruslause -- lyhyt muoto)} \label{algebran peruslause}
\index{Algebran peruslause|emph} Jos  $n \ge 1$, niin polynomilla \eqref{polynomi} on juuri
kompleksitasossa.
\end{*Lause} 
Lauseen \ref{algebran peruslause} perusteella polynomi purkautuu heti tekijöihin Lauseen
\ref{algebran pl} mukaisesti. Nimittäin jos yksi juuri $z_1$ on löydettävissä, niin $p(z)$ 
voidaan kirjoittaa muotoon
\[
p(z) = p(z) - p(z_1) = \sum_{k=1}^n c_k(z^k-z_1^k).
\]
Tässä on kunta-algebran perusteella (Propositio \ref{kuntakaava})
\[
z^k-z_1^k = (z-z_1)(z^{k-1} + z_1z^{k-2} + \ldots + z_1^{k-1}), \quad k = 1 \ldots n,
\]
joten nähdään, että $p(z)$ on kirjoitettavissa muotoon
\[
p(z) = (z-z_1)\,p_1(z),
\]
missä $p_1$ on polynomi astetta $n-1$ ja muotoa 
\[
p_1(z) = c_n z^{n-1} + [\,\text{alempiasteisia termejä}\,]. 
\]
Mikäli $n \geq 2$, on $p_1$:llä puolestaan edelleen juuri Lauseen \ref{algebran peruslause} 
perusteella, jolloin $p_1$:stä on erotettavissa tekijä $(z-z_2)$ jne, ja Lauseen 
\ref{algebran pl} väittämä siis seuraa.

Algebran peruslause on seuraamuksiltaan sen verran mittava, että matematiikkaa --- edes 
sovellettua matematiikkaa --- ilman sitä on nykyisin vaikea kuvitella. Lause on myös hyvin 
elegantti em. 'lyhyen kaavan' mukaan esitettynä. Edes todistusperiaatteen taakse kurkistaminen 
ei eleganssia vähennä, pikemminkin päinvastoin: Algebran peruslauseen voi nykyisin nähdä 
kompleksifunktioiden yleisemmän teorian seurauksena. Tarkemmin sanoen kyse on nk.\ 
\kor{analyyttisten} funktioiden teoriasta. Käsite 'analyyttinen' määritellään myöhemmin
Luvussa \ref{analyyttiset funktiot}; tässä yhteydessä todettakoon vain, että kyse on
äärimmäisen säännöllisistä kompleksifunktioista, esimerkiksi sellaisista kuin juuri polynomit.

Algebran peruslauseen todistuksen perusidea analyyttisten funktioiden teorian avulla
esitettynä on seuraava\footnote[2]{Algebran peruslauseen todistamiseen paneutui mm.\ 
'matemaatikkojen ruhtinaaksi' sanottu saksalainen \hist{Carl Friedrich Gauss} (1777-1855).
Erään hyvin uskottavan todistuksen Gauss esitti väitöskirjassaan v. 1799. Loogisesti täysin
auktoton todistus pystyttiin esittämään vasta myöhemmin 1800-luvulla reaali- ja
kompleksianalyysin kehityttyä riittävästi. --- Myöhemmin Luvussa V.10 esitetään Algebran
peruslauseelle melko suoraviivainen todistus, joka perustuu vain reaalianalyysiin, Propositioon
\ref{polynomin kasvu} ja toiseen kompleksialgebran väittämään, joka on esitetty
harjoitustehtävässä \ref{H-III-3: avaintulos}. \index{Gauss, C.F.|av}}: Tehdään vastaoletus,
että polynomilla $p$ ei ole nollakohtia. Siinä tapauksessa funktio
\[
f(z) = \frac{1}{p(z)}
\]
on koko kompleksitasossa säännöllinen. Itse asiassa $f$ on 'äärimmäisen säännöllinen' eli 
analyyttinen. Kun lisäksi käytetään Proposition \ref{polynomin kasvu} tulosta, niin voidaan 
todeta: Jos $p$:n aste on $n \ge 1$ ja $p$:llä ei ole nollakohtia, niin funktio $f=1/p$
toteuttaa
\begin{enumerate}
\item $f$ on analyyttinen koko kompleksitasossa.
\item $f(z) \rightarrow 0$, kun $\abs{z} \rightarrow \infty$.
\end{enumerate}
On olemassa ainakin yksi funktio, joka toteuttaa nämä molemmat ehdot, nimittäin $f(z) = 0$.
Kysymys kuulukin: Onko muita? Analyyttisten funktioiden teorian antama --- hieman yllättävä ---
vastaus tähän kysymykseen on:
\begin{itemize}
\item[] Ei!
\end{itemize}
Vastaoletus on näin ollen johtanut loogiseen ristiriitaan ja lause siis on tosi.

\subsection{Kompleksiluvun juuret}
\index{juuri (kompleksiluvun)|vahv}
\index{\ohje!a@--luvun tai osaluvun otsikkoon|vahv}
\index{\ohje!b@--lauseeseen tai määritelmään|emph}
\index{\ohje!c@--tekstiin}
\index{\ohje!d@--alaviitteeseen|av}

Algebran peruslauseen mukaan polynomiyhtälöllä
\begin{equation} \label{kompleksiluvun juuret}
z^n = a, \quad a \in \C,\ n \in \N,\ n \ge 2
\end{equation}
on ainakin yksi juuri. Itse asiassa osoittautuu, että jos $a \neq 0$ (mikä oletetaan), niin 
juuria on tasan $n$ kappaletta, eli kaikki juuret ovat yksinkertaisia. Merkitään niitä kaikkia 
symbolilla
\[
z=a^{1/n}=\sqrt[n]{a},
\]
ja sanotaan, että kyseessä ovat \kor{kompleksiluvun $a$ juuret}. Näitä siis tulee olemaan 
$n$ kpl, ja halutaan laskea ne perusmuotoon
\[
z_k = x_k + iy_k, \quad k=0 \ldots n-1.
\]
Tähän päästään helpoiten polaariesityksen kautta: Kirjoitetaan
\[
a= r \vkulma{\varphi} = r (\cos \varphi + i \sin \varphi),
\]
missä siis $r= \abs{a} \neq 0$. Kun merkitään $z=\abs{z}\vkulma{\psi}$, niin
\begin{align*}
z^n = a &\qekv \abs{z}^n = r \quad\ja\quad n\psi=\varphi+k\cdot 2\pi,\ k\in\Z \\
        &\qekv \abs{z}=\sqrt[n]{r} \quad\ja\quad 
               \psi=\frac{\varphi}{n}+k\cdot\frac{2\pi}{n},\ k\in\Z.
\end{align*}
Erilaisia ratkaisuja saadaan tästä indeksin arvoilla $k=0 \ldots n-1$. Nämä voidaan esittää
muodossa
\begin{align*}
z_k &= \sqrt[n]{r}\vkulma{\tfrac{\varphi}{n}+k\cdot\tfrac{2\pi}{n}} \\
    &= \left(\sqrt[n]{r}\vkulma{\tfrac{\varphi}{n}}\right)
       \left(1\vkulma{k\cdot\tfrac{2\pi}{n}}\right) \\
    &= \left(\sqrt[n]{r}\vkulma{\tfrac{\varphi}{n}}\right)
       \left(1\vkulma{\tfrac{2\pi}{n}}\right)^k \\
    &= z_0\rho^k, \quad k=0 \ldots n-1,
\end{align*}
missä siis
\begin{align*}
z_0  &\,=\,\sqrt[n]{r}\vkulma{\frac{\varphi}{n}} 
      \,=\,\sqrt[n]{r}\left(\cos\frac{\varphi}{n}+i\sin\frac{\varphi}{n}\right), \quad
                             r=|a|,\ \varphi=\arg a, \\
\rho &\,=\, 1\vkulma{\frac{2\pi}{n}} = \cos\frac{2\pi}{n}+i\sin\frac{2\pi}{n}\,.
\end{align*}
Perusmuodossa yhtälön (\ref{kompleksiluvun juuret}) ratkaisut ovat
\[
z_k = \sqrt[n]{r}\left[\cos\left(\frac{\varphi}{n} + k \cdot \frac{2 \pi}{n}\right) + 
 i \sin\left(\frac{\varphi}{n} + k\cdot\frac{2\pi}{n}\right)\right], \quad k = 0 \ldots n-1.
\]
Lähtien perusjuuresta $z_0$ juuret sijaitsevat tasavälein kompleksitason ympyrällä, jonka säde
$= \sqrt[n]{\abs{a}}$. Kuvassa on tapaus $n=8$.
\begin{figure}[H]
\setlength{\unitlength}{1cm}
\begin{center}
\begin{picture}(8,8)(-4,-4)
\put(-4,0){\vector(1,0){8}} \put(3.5,-0.5){$\text{Re}\,z$}
\put(0,-4){\vector(0,1){8}} \put(0.2,3.8){$\text{Im}\,z$}
\put(0,0){\bigcircle{6}}
\put(0,0){\vector(2,1){2.62}} \put(2.57,1.26){$\bullet$} \put(2.85,1.45){$z_0$}
\put(0,0){\vector(1,3){0.92}} \put(0.85,2.74){$\bullet$}
\put(0,0){\vector(-1,2){1.32}} \put(-1.45,2.58){$\bullet$}
\put(0,0){\vector(-3,1){2.82}} \put(-2.94,0.84){$\bullet$}
\put(0,0){\vector(-2,-1){2.62}} \put(-2.77,-1.46){$\bullet$}
\put(0,0){\vector(-1,-3){0.92}} \put(-1.05,-2.94){$\bullet$}
\put(0,0){\vector(1,-2){1.32}} \put(1.25,-2.78){$\bullet$}
\put(0,0){\vector(3,-1){2.82}} \put(2.74,-1.04){$\bullet$}
\put(0,0){\arc{2}{-0.464}{0}} \put(1.2,0.2){$\varphi/n$}
\put(0,0){\arc{1}{-1.249}{-0.464}} \put(0.4,0.55){$\frac{2\pi}{n}$}
\end{picture}
\end{center}
\end{figure}
\jatko \begin{Exa} (jatko) Tässä $a = -1 = 1 \vkulma{\pi}\,$, siis $r=1$, $\varphi = \pi$. 
Juuret ovat
\[
z_k = \cos\left(\frac{\pi}{3}+ k \cdot \frac{2\pi}{3}\right) + i \sin\left(\frac{\pi}{3}
                              + k \cdot \frac{2\pi}{3}\right), \quad k = 0,1,2,
\]
\begin{multicols}{2} \raggedcolumns
\begin{align*}
\text{eli} \quad &\left\{ \begin{array}{ll}
z_0 = \dfrac{1}{2} + \dfrac{\sqrt{3}}{2} i, \\
z_1 = -1, \\
z_2 = \dfrac{1}{2} - \dfrac{\sqrt{3}}{2} i. \quad \loppu
\end{array} \right.
\end{align*}
\begin{figure}[H]
\setlength{\unitlength}{1cm}
\begin{center}
\begin{picture}(6,6)(-3,-3)
\put(-3,0){\vector(1,0){6}} \put(2.5,-0.5){$\text{Re}\,z$}
\put(0,-3){\vector(0,1){6}} \put(0.2,2.8){$\text{Im}\,z$}
\Thicklines
\put(2,0){\line(0,-1){0.3}} \put(1.9,-0.8){$1$}
\put(-2,0){\line(0,-1){0.3}} \put(-2.43,-0.8){$-1$}
\thinlines
\put(0,0){\bigcircle{4}}
\put(-2.1,-0.1){$\bullet$} \put(-1.8,0.2){$z_1$}
\put(1.1,1.47){$\bullet$} \put(1.2,1.7){$z_0$}
\put(1.0,-1.76){$\bullet$} \put(1.1,-2){$z_2$}
\put(0,0){\vector(3,4){1.2}}
\put(0,0){\vector(-1,0){2}}
\put(0,0){\vector(2,-3){1.1}}
\put(0,0){\arc{1.4}{-0.93}{0}} \put(0.7,0.3){$\frac{\pi}{3}$}
\put(0,0){\arc{1}{-3.14}{-0.93}} \put(-0.6,0.8){$\frac{2\pi}{3}$}
\end{picture}
\end{center}
\end{figure}
\end{multicols}
\end{Exa}
\begin{Exa} $\sqrt{i} =\,?$
\end{Exa}
\ratk
Kyseessä on yhtälön $z^2 = i$ ratkaiseminen. Koska $\,i = 1\vkulma{\pi/2}\,$, niin ratkaisut
ovat
\begin{align*}
z_k              &= \cos\left(\frac{\pi}{4} + k \pi\right) + i \sin\left(\frac{\pi}{4}
                                                           + k \pi\right), \quad k=0,1 \\
\impl \ \sqrt{i} &= \pm \frac{1}{\sqrt{2}}(1+i).
\end{align*}
\pain{Tarkistus:} 
\[
(\sqrt{i})^2 = \frac{1}{2}(1+i)^2 = \frac{1}{2}(1-1+2i) = i. \quad \text{OK!} \loppu
\]
\begin{Exa} Ratkaise yhtälö $\ z^2+(2+2i)z-i=0$. \end{Exa}
\ratk Toisen asteen yhtälön ratkaisukaava perustuu vain kunta-algebraan, joten se on pätevä 
myös kompleksikertoimiselle polynomille. Ratkaisut siis ovat
\[
z_{1,2} = -(1+i) \pm\sqrt{(1+i)^2+i} = -(1+i)\pm\sqrt{3i},
\]
eli edellisen esimerkin perusteella
\begin{align*}
z_1 &= -(1+i) + \sqrt{\frac{3}{2}}\,(1+i) = \left(\sqrt{\frac{3}{2}}-1\right)(1+i), \\
z_2 &= -(1+i) - \sqrt{\frac{3}{2}}\,(1+i) = -\left(\sqrt{\frac{3}{2}}+1\right)(1+i). \loppu
\end{align*}

\Harj
\begin{enumerate}

\item
Määritä seuraavien funktioiden kaikki juuret kompleksitasossa jakamalla funktiot reaali- 
ja imaginaariosiin. \newline
a) \ $f(z)=z^2+i \quad$ 
b) \ $f(z)=z^3+8i \quad$ 
c) \ $f(z)=z^2+4\overline{z}-1$

\item
Jos $\re z=\im z=a$, niin millä $a$:n arvoilla pätee $\abs{z-i}<\abs{z-3}$\,? Kuvio!

\item
Tutki, millaiset kompleksitason pistejoukot tulevat määritellyiksi seuraavilla ehdoilla. 
Piirrä kuviot. \newline
a) \ $\abs{z+1+i}^2=2 \quad$ 
b) \ $\abs{z-i} \le 2 \quad$ 
c) \ $\abs{z+i}=\abs{z-1-i}$ \newline
d) \ $\text{Re}[(z-i)/(z+i)]=0$

\item
Jaa polynomi enintään toisen asteen reaalikertoimisiin tekijöihin: \newline
a) \ $z^3+2z^2+3z+2 \quad$ 
b) \ $z^4+2 \quad$ 
c) \ $z^4+2z^3+z^2-2z-2$ \newline
d) \ $z^8-256 \quad$ 
e) \ $(z^4+1)^2-z^4$

\item
Määritä a) kaikki reaalikertoimiset, b) kaikki polynomit $p$ astetta $\le 5$, joilla on
ominaisuudet: $p(0)=1$ ja $p$:n nollakohdat ovat $1$, $i$ ja $-i$.

\item
Määritä seuraavien juurien kaikki arvot: \newline
a) \ $\sqrt[4]{-4} \quad$ 
b) \ $\sqrt[6]{-64} \quad$
c) \ $\sqrt[3]{i-1} \quad$
d) \ $\sqrt{3+4i} \quad$ 
e) \ $\sqrt{-7+24i}$

\item
Juuren $\sqrt[6]{2+3i}\,$ eräs likiarvo on $1.2217+0.2019i$. Piirrä kuva, jossa kaikki juuren
arvot on sijoitettu kompleksitasoon (laskematta erikseen muiden juurien likiarvoja).

\item
Määritä seuraavien yhtälöiden kaikki ratkaisut perusmuodossa $x+iy$. \newline
a) \ $z^2+2iz-i-1=0 \quad$ 
b) \ $z^2-4iz-4+i=0 \quad$
c) \ $z^2-(3+5i)z=4-7i$ \newline
d) \ $z^4-2z^2+4=0 \quad $ 
e) \ $z^4+(1-2i\sqrt{3})z^2-3-i\sqrt{3}=0$

\item
Määritä $a\in\C$ siten, että $z=1+i$ on polynomin $p(z)=z^3+az+1$ juuri, ja laske ja sen
jälkeen muut juuret.

\item (*) a) Juuret $\sqrt[4]{3+4i}\,$ on mahdollista laskea tarkasti perusmuodossa
$z_k=x_k+iy_k\,$ siten, että $x_k$ ja $y_k$ ovat geometrisia lukuja. Laske! \vspace{1mm}\newline
b) Määritä tarkasti ne kompleksiluvut, joiden viides potenssi $=1$. Piirrä kuvio lukujen
sijainnista kompleksitasossa.

\item (*)
Olkoon $p(z)=c_n z^n + \ldots + c_0$ reaalikertoiminen polynomi, $c_n \neq 0$. Näytä, että 
polynomilla on ainakin yksi reaalinen nollakohta, jos joko $n$ on pariton tai $c_n c_0 \le 0$.

\item (*)
Todista, että jos kahdella polynomilla astetta $\le n$ on samat arvot $n+1$ eri kompleksitason
pisteessä, niin polynomit ovat samat.

\item (*) \label{H-III-3: avaintulos}
Olkoon $p(z)$ kompleksimuuttujan polynomi, joka ei ole vakio, ja olkoon $c\in\C$ ja 
$p(c)=a\in\C$. Näytä: \vspace{1mm}\newline
a) On olemassa $m\in\N$ ja $b\in\C,\ b \neq 0$ sekä polynomi $q(z)$ siten, että 
\[
p(z)=a+b(z-c)^m+(z-c)^{m+1}q(z)\ \ \forall z\in\C.
\]
b) Jos a)-kohdan hajotelmassa on $a\neq 0$ ja $q(z)=0$, niin $\abs{p(z)}$ pienenee johonkin
suuntaan $c$:stä lähdettäessä, t.s.\ on olemassa $\rho\in\C,\ \abs{\rho}=1$ ja $\delta>0$
siten, että
\[
\abs{p(c+t\rho)} < \abs{p(c)}, \quad \text{kun}\ 0<t<\delta.
\]
c) Jos a)-kohdan hajotelmassa on $a \neq 0$ ja $q(z) \ne 0$, niin b)-kohdan väittämä on
edelleen tosi. \newline
d) Jos $\abs{p(z)}$ saavuttaa paikallisen minimiarvon $c$:ssä, niin $p(c)=a=0$.

\end{enumerate} % Algebran peruslause
\section{*Kompleksikertoiminen vektoriavaruus}
\alku
\index{vektoric@vektoriavaruus!b@kompleksikertoiminen|vahv}

Luvussa \ref{tasonvektorit} esiteltiin algebra nimeltä (lineaarinen)
vektoriavaruus, symbolisesti $(U,\K)$, missä $U$ on vektorien muodostama joukko ja $\K$ on
nk.\ skalaarien muodostama vektoriavaruuden kertojakunta. Vektoriavaruudessa on määritelty
kaksi laskuoperaatiota, vektorien yhteenlasku ja skalaarilla kertominen, jotka ovat
funktioita tyyppiä

\begin{tabular}{ll}
yhteenlasku: &$U \times U \kohti U$, \\
skalaarilla kertominen: &$\K \times U \kohti U$.
\end{tabular}

Sisätuloavaruudeksi sanotaan vektoriavaruutta, jossa on määritelty myös skalaaritulo
(eli sisätulo, ks.\ Luku \ref{abstrakti skalaaritulo}) funktiona tyyppiä

\begin{tabular}{ll}
skalaaritulo: &$\qquad\qquad\,\ U \times U \kohti \K$.
\end{tabular}

Tähän asti on käsitelty lähinnä tapausta $\K = \R$, jolloin puhutaan \kor{reaalikertoimisesta} 
avaruudesta. Laajennus reaaliluvuista kompleksilukuihin mahdollistaa nyt myös 
\kor{kompleksikertoimisen} vektoriavaruuden, jossa $\K = \C$. Kompleksikertoiminen 
vektoriavaruus ei algebrana oleellisesti poikkea reaalikertoimisesta niin kauan kuin vain 
vektorien yhteenlasku ja skalaarilla kertominen on määritelty. Sen sijaan skalaaritulon 
määrittely funktiona tyyppiä $U \times U \kohti \C$ aiheuttaa teoriaan lisäpiirteitä, jotka on
otettava huomioon. Kompleksikertoimisen sisätuloavaruuden $(U, \C)$ sisätulon 
$\scp{\cdot}{\cdot}$ on toteutettava seuraavat ehdot (vrt.\ Luvun \ref{abstrakti skalaaritulo}
ehdot reaalikertoimiselle tapaukselle)\,:
\begin{enumerate}
\item \kor{Puolisymmetrisyys}
\begin{itemize}
\item[] $\scp{\mpu}{\mpv}=\overline{\scp{\mpv}{\mpu}} \quad \forall \ \mpu,\mpv \in U$.
\end{itemize}
\item \kor{Sekvilineaarisuus} \index{sekvilineaarisuus}
\begin{itemize}
\item[(a)] $\scp{\alpha \mpu + \beta \mpv}{\mw} = \alpha \scp{\mpu}{\mw} 
                                                                + \beta \scp{\mpv}{\mw}$,
\item[(b)] $\scp{\mpu}{\alpha \mpv + \beta \mw} = \overline{\alpha} \scp{\mpu}{\mpv} 
                                                        + \overline{\beta} \scp{\mpu}{\mw}$
\item[] \qquad\qquad $\forall \ \mpu, \mpv, \mw \in U,\ \alpha, \beta \in \C$.
\end{itemize}
\item \kor{Positiividefiniittisyys} \index{positiividefiniittisyys!a@skalaaritulon}
\begin{itemize}
\item[(a)] $\scp{\mpu}{\mpu} \geq 0 \ \forall \ \mpu \in U$,
\item[(b)] $\scp{\mpu}{\mpu} = 0 \ \ekv \ \mpu = \mathbf{0}$.
\end{itemize}
\end{enumerate}
Havaitaan, että reaalikertoimisesta tapauksesta poikkeavat vain symmetriaehto (1) sekä 
sekvilineaarisuusehto (2b) (sekvilineaarinen = $1\frac{1}{2}-$lineaarinen, engl.\ sesquilinear 
-- vrt. bilineaarinen = kaksoislineaarinen, engl.\ bilinear). Itse asissa vain symmetriaehdon
ero on olennainen, sillä (1) \& (2a) $\impl$ (2b) (Harj.teht.\,\ref{H-III-4: skalaaritulo}).
Huomattakoon, että symmetriaehto (1) pitää erikoistapauksena sisällään ehdon
$\scp{\mpu}{\mpv} = \scp{\mpv}{\mpu}$, kun $\scp{\mpu}{\mpv} \in \R$, joten aiemmin asetettuja
reaalikertoimisen tapauksen ehtoja ei erillisinä enää tarvita. Koska ehto (1) myös takaa, että 
$\scp{\mpu}{\mpu}\in\R\,\ \forall \mpu \in U$, niin ehto (3a) on mielekäs. Uusista ehdoista
aksioomina tarpeellisia ovat (1), (2a), (3a) ja (3b):n osa $[\impl]$, sillä (1) \& (2a)
$\impl$ (2b), kuten sanottu, ja (2a) $\impl$ (3b):n osa $[\Leftarrow]$.

Skalaaritulon keskeinen ominaisuus on Cauchyn--Schwarzin epäyhtälö, joka vapautettiin 
geometriasta Luvussa \ref{abstrakti skalaaritulo}. Varmistetaan nyt, että epäyhtälö pätee myös
kompleksikertoimisessa tapauksessa (vrt. Lause \ref{schwarzR}).
\begin{Lause} \label{schwarzC} \index{Cauchyn!f@--Schwarzin epäyhtälö|emph}
Jokaiselle aksioomat 1--3 toteuttavalle kompleksikertoimisen vektoriavaruuden $U$
skalaaritulolle pätee Cauchyn--Schwarzin epäyhtälö
\[
\abs{\scp{\mpu}{\mpv}} \leq \abs{\mpu} \abs{\mpv},
\]
missä
\[
\abs{\mpu}=\scp{\mpu}{\mpu}^{1/2}.
\]
\tod
Tapaus $\mpu = \mathbf{0}$ on jälleen selvä, joten voidaan olettaa $\mpu \neq \mathbf{0}$, 
jolloin on $\scp{\mpu}{\mpu}>0$ aksioomien (3a,b) perusteella. Lähdetään jälleen epäyhtälöstä
\[
\scp{\beta \mpu + \mpv}{\beta \mpu + \mpv} \geq 0,
\]
joka nyt on voimassa $\forall \beta \in \C$ (aksiooma (3a)).

Käyttämällä sekvilineaarisuusehtoja (2a,b), symmetriaehtoa (1) ja Luvun 
\ref{kompleksiluvuilla laskeminen} kaavoja (2),\,(3),\,(7) tämä purkautuu muotoon
\begin{align*}
\scp{\beta \mpu}{\beta \mpu} + \scp{\beta \mpu}{\mpv} + \scp{\mpv}{\beta \mpu} 
                                                      + \scp{\mpv}{\mpv} &\geq 0 \\
\ekv \ \abs{\beta}^2 \scp{\mpu}{\mpu} + 2 \; \text{Re} \; \{ \beta \scp{\mpu}{\mpv} \} 
                                      + \scp{\mpv}{\mpv} &\geq 0.
\end{align*}
Koska tämä on voimassa $\forall \beta \in \C$ ja koska $\scp{\mpu}{\mpu} > 0$, voidaan valita
\[
\beta = - \frac{\overline{\scp{\mpu}{\mpv}}}{\scp{\mpu}{\mpu}}\,,
\]
jolloin Luvun \ref{kompleksiluvuilla laskeminen} kaavojen (3),\,(6) perusteella seuraa
\begin{align*}
&-\frac{\abs{\scp{\mpu}{\mpv}}^2}{\scp{\mpu}{\mpu}} + \scp{\mpv}{\mpv} \geq 0 \\
& \ekv \ \abs{\scp{\mpu}{\mpv}}^2 \leq \scp{\mpu}{\mpu} \scp{\mpv}{\mpv} \\[2mm]
& \ekv \ \abs{\scp{\mpu}{\mpv}} \leq \abs{\mpu} \abs{\mpv}. \quad\loppu
\end{align*}
\end{Lause}
\begin{Exa} Yksinkertaisin esimerkki kompleksikertoimisesta sisätuloavaruudesta on $\C$ itse,
ts.\ $(U, \C)$, missä $U=\C$. Nimittäin koska kompleksilukujen yhteenlasku on samanlainen 
operaatio kuin tason vektoreiden yhteenlasku, ja kertolasku voidaan tulkita myös skalaarilla
kertomiseksi, niin $(\C,\C)$ on vektoriavaruus. Kun tässä avaruudessa määritellään skalaaritulo
\[
\scp{u}{v} = u \overline{v}, \quad u,v \in \C,
\]
toteutuvat em. ehdot 1--3 (vrt. Luvun \ref{kompleksiluvuilla laskeminen} kaavat). 
Cauchyn--Schwarzin epäyhtälö pelkistyy tässä erikoistapauksessa yhtälöksi:
\[
\abs{\scp{u}{v}} = \abs{u \overline{v}} = \abs{u}\abs{v}, \quad u,v \in \C. \loppu
\]
\end{Exa}
\begin{Exa} Kompleksikertoimisia sisätuloavaruuksia ovat myös avaruuksien $(\Rkaksi, \R)$ ja
$(\R^3,\R)$ laajennukset $(\C^2,\C)$ ja $(\C^3,\C)$. Esimerkiksi $(\C^3,\C)$:ssä skalaaritulo
määritellään
\begin{align*}
&\mpu =(u_1,u_2,u_3) \in \C^3, \quad \mpv =(v_1,v_2,v_3) \in \C^3\,: \\
&\scp{\mpu}{\mpv} = \sum_{i=1}^3 u_i \overline{v_i}.
\end{align*}
\end{Exa}
Avaruuden $(\C^n, \C),\ (n=2,3)$ 
\index{euklidinen!b@normi} \index{normi!euklidinen}%
\kor{euklidinen normi} on
\[
\abs{\mpu} = \scp{\mpu}{\mpu}^{1/2} = ( \sum_{i=1}^n \abs{u_i}^2 )^{1/2}. \loppu
\]

\Harj
\begin{enumerate}

\item \label{H-III-4: skalaaritulo}
Näytä, että skalaaritulon sekvilineaarisuusominaisuus (2b) seuraa ehdoista (1) ja (2a).

\item
Ovatko avaruuden $(\C^2,\C)$ vektorit 
\[
\mpu=(3-4i,\,7-i), \ \ \mpv=(2-11i,\,13-9i)
\]
lineaarisesti riippumattomat?
 
\item
Jaa sisätuloavaruuden $(\C^2,\C)$ vektori $\mpu=(1+i,2-3i)$ kahteen komponenttiin siten, että
toinen komponentti on vektorin $\mpv=(1-i,2+i)$ suuntainen ja toinen $\mpv$:tä vastaan
kohtisuora.

\end{enumerate} % *Kompleksikertoiminen vektoriavaruus

\chapter{Reaalimuuttujien funktiot}

Matemaattisten funktioiden päätyypit ovat
\begin{itemize}
\item \kor{yhden} reaali\kor{muuttujan} reaaliarvoiset \kor{funktiot}, eli reaalifunktiot
\item \kor{useamman} reaali\kor{muuttujan} reaaliarvoiset \kor{funktiot}
\item yhden tai useamman reaalimuuttujan \kor{vektoriarvoiset funktiot}
\item \kor{kompleksifunktiot}, eli kompleksimuuttujan kompleksiarvoiset funktiot
\end{itemize}
Tässä luvussa aloitetaan funktoiden tutkimus tarkastelemalla yhden tai useamman, toistaiseksi
kahden tai kolmen, reaalimuuttujan reaaliarvoisia funktioita sekä yhden tai kahden muuttujan
vektoriarvoisia funktioita. Tarkasteltaville funktiotyypeille on yhteistä niiden saama 'näkyvä'
muoto, kun luvut, lukuparit ja lukukolmikot muuttujina tai vektorit funktion arvoina ymmärretään
euklidisten pisteavaruuksien tai vastaavien vektoriavaruuksien olioina.

Funktioiden tutkimus on matematiikassa hyvin keskeistä, siksi myös tähän liittyvä käsitteistö
ja keinovalikoima on huomattavan laaja. Tässä luvussa ei koko 'teknologiaa' oteta vielä käyttöön,
vaan rajoitutaan toistaiseksi kaikkein yksinkertaisimpiin algebran ja geometrian keinoihin.
Toisaalta sovelluksia (etenkin fysiikan sovelluksia) ajatellen tässä luvussa tarkasteltavien
funktioiden tyyppivalikoima on jo melko edustava. Tarkoituksena on tämän valikoiman puitteissa
käydä läpi mm.\ sellaiset funktioiden algebran käsitteet kuin funktioiden 
\kor{algebralliset yhdistelyt}, \kor{yhdistetty funktio}, \kor{käänteisfunktio} ja
\kor{implisiittifunktio}. Selvimmin 'geometrisia funktioita' ovat Luvussa
\ref{parametriset käyrät} esiteltävät \kor{parametriset käyrät} ja \kor{parametriset pinnat}. % Reaalimuuttujien funktiot
\section{Yhden muuttujan funktiot} \label{yhden muuttujan funktiot}
\alku
\index{funktio A!b@reaalifunktio|vahv}
\index{yhden muuttujan funktio|vahv}

Yhden (reaali)muuttujan (reaaliarvoisella) funktiolla eli reaalifunktiolla tarkoitetaan
funktiota tyyppiä
\[
f:\DF_f \kohti \R, \quad \DF_f \subset \R.
\]
\index{mzyzy@määrittelyjoukko}%
Tässä $\DF_f$ on $f$:n \kor{määrittelyjoukko} (lähtöjoukko, engl. domain) ja joukko
\[
\RF_f=\{y\in\R \mid y=f(x)\,\ \text{jollakin}\ x\in\DF_f\}
\]
\index{arvojoukko}%
on $f$:n \kor{arvojoukko} (engl.\ range eli 'kantama').\footnote[2]{Suomenkielisissä
teksteissä määrittely- ja arvojoukoille käytetään myös merkintöjä $M_f,A_f$.} Funktion $f$ 
\index{kuvaaja}%
\kor{kuvaaja} (engl.\ graph) joukossa $A \subset \DF_f$ on euklidisen tason (yleensä
karteesisen) koordinaatiston avulla määritelty pistejoukko
\[
G_{f,A}=\{\,P \vastaa (x,y) \in \R^2 \mid x \in A\,\ja\, y=f(x)\,\} \subset \Ekaksi.
\]
Kuvaajan tarkoituksena on 'geometrisoida' funktio niin, että näköhavainnot tulevat 
mahdollisiksi.\footnote[3]{Kuvaaja on geometrinen vastine reaalifunktion joukko-opilliselle
määritelmälle $\Rkaksi$:n osa\-joukkona, ks.\ Luku \ref{trigonometriset funktiot}.}
Esimerkiksi arvojoukko $\RF_f$ on kuvaajasta helppo hahmottaa, ja sellainen
usein luontevalta tuntuva sanonta kuin '$f$ pisteessä $x$' ($=f(x)$) sisältää myös 
geometrisointiajatuksen ($x \vastaa P \in E^1$, vrt.\ Luku \ref{tasonvektorit}).
\setlength{\unitlength}{1mm}
\begin{figure}[H]
\begin{center}
\begin{picture}(130,70)(-50,-20)
\put(0,-20){\vector(0,1){70}} \put(3,47){$y$}
\put(-50,0){\vector(1,0){130}} \put(77,-5){$x$}
\spline(-40,10)(-20,30)(0,20)(20,-20)(40,30)(50,39.4)(60,30)
\put(41,30){$\bullet$} \put(-1,30){$\bullet$} \dashline{2}(-1,31)(41,31) 
\put(41,-1){$\bullet$} \dashline{2}(42,31)(42,0)
\dashline{2}(-40,10)(-40,0) \dashline{2}(60,30)(60,0) \dashline{2}(50,37)(0,37) 
\dashline{2}(20,-9)(0,-9)
\put(8,22){\vector(-1,1){7}}\put(10,21){$f$ pisteessä $x$}
\put(50,-9){\vector(-1,1){7}}\put(52,-10){piste $x$}
\put(-40,-12.2){$\underbrace{\hspace{100mm}}_{\D \DF_f}$}
\put(-12,13){$\RF_f \left\{ \begin{array}{c} \vspace{40mm} \end{array} \right.$}
\end{picture}
%\caption{Funktion 'geometrisointi'}
\end{center}
\end{figure}

Reaalimuuttujan funktiota $f$ tutkitaan useimmiten jollakin avoimella, suljetulla tai 
puoliavoimella välillä, jolle $f$:n määrittelyjoukko voidaan haluttaessa ajatella rajatuksi
ko.\ tarkastelussa. Kerrattakoon Luvusta \ref{reaaliluvut} merkinnät
\begin{align*}
\text{avoin väli:} \quad       &(a,b)\ =\ \{x\in\R \mid a<x<b\}, \\
\text{suljettu väli:} \quad    &[a,b]\,\ =\ \{x\in\R \mid a\leq x\leq b\}, \\
\text{puoliavoin väli:} \quad  &(a,b]\,\ \text{tai}\,\ [a,b)
\intertext{ja myös yleisessä käytössä olevat 'äärettömän välin' merkinnät}
              (0,\infty)\ =\,\ &\{x\in\R \mid x>0\} = \R_+, \\
             (-\infty,0)\ =\,\ &\{x\in\R \mid x<0\} = \R_-, \\
        (-\infty,\infty)\ =\,\ &\R.
\end{align*}
Usein yhden muuttujan reaalifunktioista ilmoitetaan vain laskusääntö (liittämissääntö)
muodossa 'funktio $f(x)$'. Tällöin oletetaan (ellei toisin mainita), että määrittelyjoukko
$\DF_f$ on suurin mahdollinen ts.
\[
\DF_f=\{x\in\R \ | \ y=f(x) \text{ määritelty yksikäsitteisesti ja } y\in\R\}.
\]
\begin{Exa} \label{reaalifunktioita} \index{funktio B!d@paloittain määritelty}
\index{paloittainen!a@funktion määrittely}
\begin{align*}
&\text{a)}\ \ f(x)=x^4+x^2+1 \qquad 
 \text{b)}\ \ f(x)=x^2/(x-1)^2 \qquad 
 \text{c)}\ \ f(x)=\cot x \\
&\text{d)}\ \ f(x)=x^{3/4} \quad\
 \text{e)}\ \ f(x) = \begin{cases} \,\cos x,     &\text{kun}\ x<0 \\ 
                                   \,x-x^2,\,\   &\text{kun}\ 0 \le x \le 1 \\
                                   \,2-\sqrt{x}, &\text{kun}\ x>1
                      \end{cases} \quad\
 \text{f)}\ \ f(x) = \sum_{k=1}^\infty \frac{x^k}{k^2}
\end{align*}
\begin{itemize}
\item[a)] Polynomi: $\ \DF_f=\R$, $\,\RF_f=[1,\infty)$.
\item[b)] \kor{Rationaalifunktio}: $\ \DF_f=\{x\in\R \mid x \neq 1\}$, $\,\RF_f=[0,\infty)$.
\item[c)] Trigonometrinen funktio: 
           $\ \DF_f=\{x\in\R \mid x/\pi\not\in\Z\}$, $\,\RF_f=\R$.
\item[d)] \kor{Potenssifunktio}: $\ \DF_f=\RF_f=[0,\infty)$.
\item[e)] \kor{Paloittain määritelty} funktio: $\ \DF_f = \R$, $\,\RF_f = (-\infty,1]$.
\item[f)] Potenssisarja: Määrittelyjoukko on $\,\DF_f = [-1,1]$
          (vrt.\ Luku \ref{potenssisarja}). Arvojoukko on vaikeampi määrittää, mutta
          osoittautuu: $\,\RF_f = [-\tfrac{\pi^2}{12},\tfrac{\pi^2}{6}]$. (Sivuutetaan
          perustelut.) \loppu
\end{itemize}
\end{Exa}
\index{rationaalifunktio} \index{potenssifunktio}%
Esimerkin b-kohdan rationaalifunktion yleisempi muoto on $f(x)=p(x)/q(x)$, missä $p$ ja $q$
ovat (reaalikertoimisia) polynomeja. Määrittelyjoukko on tällöin
$\DF_f=\{x\in\R \mid q(x) \neq 0\}$. Potenssifunktion (esimerkin d-kohta) yleinen muoto on
$f(x)=x^\alpha$, missä (toistaiseksi) $\alpha\in\Q$. Määrittelyjoukko on joko
$\DF_f=[0,\infty)$ ($\alpha>0,\ \alpha\not\in\N$), 
$\DF_f=(0,\infty)=\R_+$ ($\alpha<0,\ \alpha\not\in\Z$),
$\DF_f=\{x\in\R\ |\ x \neq 0\}$ ($\alpha\in\Z,\ \alpha \le 0$) tai
$\DF_f=\R$ ($\alpha\in\N$). 
\begin{Exa} \label{erään polynomin arvojoukko} Määritä funktion $f(x) = x^2-7x+11$ arvojoukko
välillä $[1,4]$. \end{Exa}
\ratk Tehtävän asettelun mukaisesti rajataan funktion määrittelyjoukko tässä väliksi
$A = [1,4]$, jolloin arvojoukolle luonteva merkintä on $f(A)$. Koska
\[ 
f(x) = \left(x - \dfrac{7}{2}\right)^2 - \dfrac{49}{4} + 11 
                                       = \left(x - \dfrac{7}{2}\right)^2 - \dfrac{5}{4}\,, 
\]
niin $f(A)$:n minimi on
\[ 
f_{\text{min}} = \min\,\{f(x) \mid x \in [1,4]\,\} = f(7/2) = -5/4, 
\]
ja $f(A)$:n maksimi saavutetaan mahdollisimman etäällä pisteestä $x = 7/2$, eli pisteessä 
$x=1$\,:
\[ 
f_{\text{max}} = \max\,\{f(x) \mid x \in [1,4]\,\} = f(1) = 5. 
\]
Tähän asti on päätelty: $y \in f(A)\,\impl\,-5/4 \le y \le 5$, eli $f(A) \subset [-5/4,5]$. 
Toisaalta jos $y \in [-5/4,5]$, niin yhtälöllä $f(x)=y$ on ratkaisu
\[
x = \dfrac{7}{2} - \sqrt{y + \dfrac{5}{4}}\,,
\]
joka on välillä $[1,4]$. Siis pätee $y \in [-5/4,5]\,\impl\,y \in f(A)$, eli 
$[-5/4,5] \subset f(A)$. Koska on sekä $f(A) \subset [-5/4,5]\,$ että 
$\,[-5/4,5] \subset f(A)$, niin on 
\[
f(A) = [-5/4,5]. \loppu 
\]
Esimerkissä on kyse tyypillisestä 'funktiotutkimuksesta', jossa on annettu joukko
$A \subset \DF_f$ (reaalifunktion tapauksessa usein väli) ja on määrättävä $B=f(A)$ eli
funktion arvojoukko, kun määrittelyjoukko rajataan $A$:ksi. Ongelma voi olla asetettu myös
käänteisesti niin, että tunnetaan joukko $B$, ja on määrättävä joukko 
$A = \{x \in \DF_f \mid f(x) \in B\}$. Esimerkiksi funktion nollakohtia määrättäessä on 
ratkaistava käänteinen ongelma, kun $B=\{0\}$, eli on \pain{ratkaistava} y\pain{htälö} 
$f(x)=0$. Jos $B=[0,\infty)$, niin on \pain{ratkaistava} \pain{e}p\pain{ä}y\pain{htälö}
$f(x) \ge 0$. 
\jatko \begin{Exa} (jatko) Jos esimerkissä asetetaan $\DF_f=\R$ ja $B=\{0,1\}$, niin
\begin{align*}
A &= \{x\in\R \mid f(x) \in B\} \\
  &= \{x\in\R \mid x^2-7x+11=0\,\tai\,x^2-7x+11=1\} \\
  &= \{2,\,\tfrac{1}{2}(7-\sqrt{5}),\,\tfrac{1}{2}(7+\sqrt{5}),\,5\}. \loppu
\end{align*}
\end{Exa}

\subsection{Monotoniset funktiot}
\index{funktio B!a@monotoninen|vahv}

Yhtälöitä ja epäyhtälöitä ratkaistaessa, ja muutenkin funktioita tutkittaessa, on käytännössä 
suurta hyötyä seuraavasta yhden muuttujan funktioille 'luonnetta' antavasta määritelmästä
(vrt.\ Määritelmä \ref{monotoninen jono} lukujonoille).
\begin{Def} \label{monotoninen funktio} \index{monotoninen!b@reaalifunktio|emph}
\index{aidosti kasvava, vähenevä, monotoninen!b@reaalifunktio|emph}
Yhden reaalimuuttujan funktio on \kor{kasvava} (engl.\ increasing) \kor{välillä}
$A \subset \DF_f$, jos
\[
\forall x_1,x_2 \in A\,\bigl[\,x_1 < x_2 \ \impl \ f(x_1) \leq f(x_2)\,\bigr]
\]
ja \kor{aidosti} (engl. strictly) \kor{kasvava}, jos
\[
\forall x_1,x_2 \in A\,\bigl[\,x_1 < x_2 \ \impl \ f(x_1) < f(x_2)\,\bigr].
\]
Vastaavasti $f$ on \kor{vähenevä} (engl.\ decreasing) \kor{välillä} $A \subset \DF_f$, jos
\[
\forall x_1,x_2 \in A\,\bigl[\,x_1 < x_2 \ \impl \ f(x_1) \geq f(x_2)\,\bigr]
\]
ja \kor{aidosti vähenevä}, jos
\[
\forall x_1,x_2 \in A\,\bigl[\,x_1 < x_2 \ \impl \ f(x_1) > f(x_2)\,\bigr].
\]
Jos $f$ on välillä A jompaa kumpaa tyyppiä, niin sanotaan, että $f$ on ko.\ välillä (aidosti) 
\kor{monotoninen}. \end{Def}
\begin{figure}[H]
\setlength{\unitlength}{1mm}
\begin{center}
\begin{picture}(140,45)(0,10)
\put(20,20){\vector(1,0){40}} \put(58,16){$x$}
\put(20,20){\vector(0,1){30}} \put(22,48){$y$}
\put(80,20){\vector(1,0){40}} \put(118,16){$x$}
\put(80,20){\vector(0,1){30}} \put(82,48){$y$} 
\curve(20,30,25,32,30,38)
\drawline(30,38)(40,38)
\drawline(40,38)(45,45)
\curve(80,45,90,32,105,25)
\put(95,32){$y=f(x)$}
\put(35,32){$y=f(x)$}
\put(30,10){$f$ kasvava}
\put(83,10){$f$ aidosti vähenevä}
\end{picture}
%\caption{Esimerkkejä}
\end{center}
\begin{Exa} Identiteetistä
\[
\sqrt{x_1}-\sqrt{x_2}=\frac{x_1-x_2}{\sqrt{x_1}+\sqrt{x_2}}\,, 
                               \quad x_1,x_2 \ge 0, \ x_1 \neq x_2
\]
nähdään, että funktio $f(x)=\sqrt{x}$ on välillä $[0,\infty)$ (eli määrittelyjoukossaan)
aidosti kasvava. \loppu
\end{Exa}
\end{figure}
\begin{multicols}{2} \raggedcolumns
Jos $f$ on välillä $A\subset\DF_f$ aidosti kasvava ja $c\in A$, niin epäyhtälöllä
\[
f(x) \leq c
\]
on välille $A$ rajattuna helppo ratkaisu:
\[
x\in A \cap (-\infty,a],\ \text{ missä } f(a)=c.
\]
\begin{figure}[H]
\setlength{\unitlength}{1mm}
\begin{center}
\begin{picture}(40,35)(-5,-5)
\put(-5,0){\vector(1,0){40}} \put(33,-4){$x$}
\put(0,-5){\vector(0,1){30}} \put(2,23){$y$}
\curve(0,2,15,10,30,25)
\drawline(15,10)(15,0)
\drawline(15,10)(0,10)
\put(14,-4){$a$} \put(-4,9){$c$}
\dottedline[$\shortmid$]{1}(1,0.2)(15,0.2)
\end{picture}
%\caption{$f$ aidosti kasvava}
\end{center}
\end{figure}
\end{multicols}
\jatko\begin{Exa} (jatko) Jos $a>0$, niin epäyhtälön $\sqrt{x} \le a$ ratkaisu on
\[
\{x\in\R \mid \sqrt{x} \le a\} \,=\, \{x\in[0,\infty) \mid x \le a^2\} 
                               \,=\, [0,a^2\,]. \loppu
\]
\end{Exa}
Jos funktio ei ole koko tarkasteltavalla välillä monotoninen, on funktiotutkimuksen
ensimmäinen askel usein välin jakaminen sellaisiin osaväleihin, joilla monotonisuus toteutuu.
\begin{Exa}
$f(x)=\sqrt{\abs{x}} \quad (\DF_f=\R)$ on aidosti vähenevä välillä $(-\infty,0]$ ja aidosti 
kasvava välillä $[0,\infty)$. \loppu
\end{Exa}
\begin{Exa} Jos $f(x) = -x^2+6x+1$, niin kirjoittamalla
\[ f(x) = -(x-3)^2 + 10 \]
nähdään, että $f$ on aidosti kasvava välillä $(-\infty,3]$ ja aidosti vähenevä välillä 
$[3,\infty)$. \loppu \end{Exa} 
\begin{Exa}
Trigonometristen funktioiden määritelmien (Luku \ref{trigonometriset funktiot}) perusteella 
seuraavat väittämät ovat uskottavia (myös tosia).
\begin{itemize}
\item[$\sin x$:] Aidosti monotoninen väleillä 
                 $[(k-\frac{1}{2})\pi,(k+\frac{1}{2})\pi],\ k\in\Z$\,: kasvava kun $k$ on
                 parillinen ja vähenevä kun $k$ on pariton.
\item[$\cos x$:] Aidosti monotoninen väleillä $[k\pi,(k+1)\pi],\ k\in\Z$\,: vähenevä kun $k$
                 on parillinen ja kasvava kun $k$ on pariton.
\end{itemize}
\begin{figure}[H]
\setlength{\unitlength}{1cm}
\begin{picture}(14,4)(-2,-2)
\put(-2,0){\vector(1,0){14}} \put(11.8,-0.4){$x$}
\put(0,-2){\vector(0,1){4}} \put(0.2,1.8){$y$}
%\linethickness{0.5mm}
\multiput(3.14,0)(3.14,0){3}{\drawline(0,-0.1)(0,0.1)}
\put(0.1,-0.4){$0$} \put(3.05,-0.4){$\pi$} \put(6.10,-0.4){$2\pi$} \put(9.20,-0.4){$3\pi$}
\curve(
   -1.5708,   -1.0000,
   -1.0708,   -0.8776,
   -0.5708,   -0.5403,
   -0.0708,   -0.0707,
    0.4292,    0.4161,
    0.9292,    0.8011,
    1.4292,    0.9900,
    1.9292,    0.9365,
    2.4292,    0.6536,
    2.9292,    0.2108,
    3.4292,   -0.2837,
    3.9292,   -0.7087,
    4.4292,   -0.9602,
    4.9292,   -0.9766,
    5.4292,   -0.7539,
    5.9292,   -0.3466,
    6.4292,    0.1455,
    6.9292,    0.6020,
    7.4292,    0.9111,
    7.9292,    0.9972,
    8.4292,    0.8391,
    8.9292,    0.4755,
    9.4292,   -0.0044,
    9.9292,   -0.4833,
   10.4292,   -0.8439,
   10.9292,   -0.9978)
\curvedashes[1mm]{0,1,2}
\curve(
   -1.5708,    0.0000,
   -1.0708,    0.4794,
   -0.5708,    0.8415,
   -0.0708,    0.9975,
    0.4292,    0.9093,
    0.9292,    0.5985,
    1.4292,    0.1411,
    1.9292,   -0.3508,
    2.4292,   -0.7568,
    2.9292,   -0.9775,
    3.4292,   -0.9589,
    3.9292,   -0.7055,
    4.4292,   -0.2794,
    4.9292,    0.2151,
    5.4292,    0.6570,
    5.9292,    0.9380,
    6.4292,    0.9894,
    6.9292,    0.7985,
    7.4292,    0.4121,
    7.9292,   -0.0752,
    8.4292,   -0.5440,
    8.9292,   -0.8797,
    9.4292,   -1.0000,
    9.9292,   -0.8755,
   10.4292,   -0.5366,
   10.9292,   -0.0663)
\put(1,1.2){$y=\sin x$}
\put(5.5,1.2){$y=\cos x$}
\end{picture} 
\end{figure}
\begin{itemize}
\item[$\tan x$:] Aidosti kasvava väleillä
                 $((k-\frac{1}{2})\pi,(k+\frac{1}{2})\pi), \ k \in \Z$.
\item[$\cot x$:] Aidosti vähenevä väleillä $((k\pi,(k+1)\pi), \ k \in \Z$. \loppu
\end{itemize}
\begin{figure}[H]
\setlength{\unitlength}{1cm}
\begin{picture}(14,4)(-2,-2)
\put(-2,0){\vector(1,0){14}} \put(11.8,-0.4){$x$}
\put(0,-2){\vector(0,1){4}} \put(0.2,1.8){$y$}
\multiput(3.14,0)(3.14,0){3}{\drawline(0,-0.1)(0,0.1)}
\put(0.1,-0.4){$0$} \put(3.05,-0.4){$\pi$} \put(6.10,-0.4){$2\pi$} \put(9.20,-0.4){$3\pi$}
\multiput(0,0)(3.14,0){4}{
\curve(
   -1.1,-2,     
   -1.0708,   -1.8305,
   -0.9708,   -1.4617,
   -0.8708,   -1.1872,
   -0.7708,   -0.9712,
   -0.6708,   -0.7936,
   -0.5708,   -0.6421,
   -0.4708,   -0.5090,
   -0.3708,   -0.3888,
   -0.2708,   -0.2776,
   -0.1708,   -0.1725,
   -0.0708,   -0.0709,
    0.0292,    0.0292,
    0.1292,    0.1299,
    0.2292,    0.2333,
    0.3292,    0.3416,
    0.4292,    0.4577,
    0.5292,    0.5848,
    0.6292,    0.7279,
    0.7292,    0.8935,
    0.8292,    1.0917,
    0.9292,    1.3386,
    1.0292,    1.6622,
        1.1,2)}
\curvedashes[1mm]{0,1,2}
\multiput(0,0)(3.14,0){4}{
\curve(
    0.5000,   1.8305,
    0.6000,    1.4617,
    0.7000,    1.1872,
    0.8000,    0.9712,
    0.9000,    0.7936,
    1.0000,    0.6421,
    1.1000,   0.5090,
    1.2000,   0.3888,
    1.3000,    0.2776,
    1.4000,   0.1725,
    1.5000,    0.0709,
    1.6000,  -0.0292,
    1.7000,  -0.1299,
    1.8000,   -0.2333,
    1.9000,   -0.3416,
    2.0000,   -0.4577,
    2.1000,   -0.5848,
    2.2000,   -0.7279,
    2.3000,   -0.8935,
    2.4000,   -1.0917,
    2.5000,   -1.3386,
    2.6000,   -1.6622,
    2.7000,   -2.1154)}
\put(-2.2,-0.4){$y=\tan x$}
\put(1.3,0.5){$y=\cot x$}
\end{picture}
%\caption{Trigonometriset funktiot}
\end{figure}
\end{Exa}

\subsection{Yhdistetty funktio}
\index{funktio B!e@yhdistetty|vahv}
\index{yhdistetty funktio|vahv}

Kahden funktion $f,g$ \kor{yhdistetty} (engl. composite) \kor{funktio} merkitään
$f \circ g$ ja määritellään laskusäännöllä
\[
(f \circ g)(x)=f(g(x)).
\]
Tavallinen ääntämistapa on hieman arkinen 'f pallo g'. 
\begin{Exa}
Palautuva lukujono muotoa
\[
x_0 \in \R, \quad x_n=f(x_{n-1}), \quad n=1,2,\ldots
\]
voidaan tulkita 'sisäkkäisten', ts. yhdistettyjen funktioiden avulla:
\begin{align*}
x_1 &= f(x_0), \\ 
x_2 &= f(x_1) = f(f(x_0))=(f \circ f)(x_0), \\
x_3 &= f(f(f(x_0)))=(f \circ f \circ f)(x_0), \quad \text{jne.} \loppu
\end{align*}
\end{Exa}
%Jos esimerkiksi $f(x)=1/(1+x)$, niin syntyy nk. \kor{ketjumurtoluku}
%\[
%\left.\begin{aligned}
%x_n=\cfrac{1}{1+
%     \cfrac{1}{1+
%      \cfrac{1}{1+\dotsb}
%        }}& \\
%        & \cfrac{\ddots \quad }{1+x_0}
%\end{aligned} \quad \right\} n \text{ kpl} \quad\loppu
%\]
Yhdistetyn funktion määrittelyjoukko on aina rajattava niin, että funktiot eivät 'riitele'.
Tällöin suurimmaksi mahdolliseksi määrittelyjoukoksi (joka yleensä oletetaan, ellei toisin 
mainita) tulee
\[
\DF_{f\circ g}=\{x \in \DF_g \ | \ g(x) \in \DF_f\} \subset \DF_g.
\]
Jos näin määritelty joukko $\DF_{f \circ g}$ on tyhjä, ei yhdistettyä funktiota voi määritellä.
\begin{Exa} Määrittele yhdistetyt funktiot $f \circ g$ ja $g \circ f$, kun 
$f(x)=\sqrt{a-x},\quad$ $g(x)=\sqrt{x-b}\ (a,b\in\R)$. \end{Exa}
\ratk Laskusäännöt ovat
\[
(f\circ g)(x) = \sqrt{a-\sqrt{x-b}}, \quad (g\circ f)(x) = \sqrt{\sqrt{a-x}-b}.
\]
Koska $\DF_f=(-\infty,a]$, $\DF_g=[b,\infty)$, on
\[
\DF_{\,f \circ g}=\{x \in [b,\infty) \mid \sqrt{x-b} \in (-\infty,a]\,\}.
\]
Jos $a<0$, niin $\DF_{f\circ g}=\emptyset$, muuten
\[
\DF_{f\circ g}\ =\ \{\,x\in\R \mid x \geq b \ \ja \ x-b \leq a^2\,\}\ 
                =\ [b,b+a^2] \quad (a \geq 0).
\]
Siis
\[
\DF_{f\circ g}=\begin{cases}
               \,\emptyset, &\text{ jos } a <0, \\
               \,[b,b+a^2], &\text{ jos } a \geq 0.
               \end{cases}
\]
Vastaavalla tavalla päätellään
\begin{align*}
\DF_{g\circ f}&=\{\,x\in (-\infty,a] \ | \ \sqrt{a-x} \in [b,\infty)\,\} \\
              &=\begin{cases}
                \,(-\infty,a],     &\text{ jos } b \le 0, \\
                \,(-\infty,a-b^2], &\text{ jos } b > 0.
                \end{cases} \qquad\quad \loppu
\end{align*}

\subsection{Muuttujan vaihto}
\index{funktio B!g@muuttujan vaihto|vahv}
\index{muuttujan vaihto (sijoitus)|vahv}

Jos tutkimuskohteena oleva funktio $f$ joukossa $A\subset\DF_f$ on esitettävissä
yhdistettynä funktiona $f(x)=g(v(x))$, niin usein auttaa \kor{muuttujan vaihto} eli
\kor{sijoitus} $t=v(x)$. Tällöin voidaan siirtyä tutkimaan (mahdollisesti yksinkertaisempaa)
funktiota $g(t)$ joukossa $B=v(A)$.
\begin{Exa} Määritä funktion $f(x)=x-7\sqrt{x}+11$ arvojoukko $f(A)$ välillä $A=[1,16]$.
\end{Exa}
\ratk Sijoituksella $t=\sqrt{x}$ tutkimuskohteeksi tulee funktio $g(t)=t^2-7t+11$ välillä
$B=[1,4]$. Siis $f(A)=g(B)=[-5/4,5]$ (Esimerkki \ref{erään polynomin arvojoukko}). \loppu


\subsection{Funktioiden yhdistely laskutoimituksilla}
\index{funktio B!f@yhdistely laskutoimituksilla|vahv}

Reaaliarvoisia funktioita on mahdollista yhdistellä peruslaskutoimituksilla samalla tavoin kuin 
lukujonoja. Tällöin ajatellaan, että kun lukujonoja lasketaan yhteen, kerrotaan ja jaetaan 
termeittäin, niin funktioita yhdistellään vastaavalla tavalla \kor{pisteittäin}. Näin ajatellen
saadaan määritellyksi funktioiden väliset peruslaskutoimitukset. Yhden reaalimuuttujan
funktioille täsmällisempi määritelmä on seuraava:
\begin{Def} (\vahv{Funktioiden yhdistely}) \label{funktioiden yhdistelysäännöt}
\index{laskuoperaatiot!f@reaalifunktioiden|emph}
Jos $f:\DF_f \rightarrow \R$, $g:\DF_g\rightarrow\R$, $\DF_f,\DF_g \subset \R$, niin funktiot
$\lambda f$ $(\lambda\in\R)$, $f+g$, $fg$ ja $f/g$ määritellään
\[
\begin{array}{clll}
(1) & (\lambda f)(x)&=\ \lambda f(x), \quad & x \in \DF_f \\ \\
(2) & (f+g)(x)&=\ f(x)+g(x), \quad & x\in \DF_f \cap \DF_g \\ \\ 
(3) & (fg)(x)&=\ f(x)g(x), \quad & x\in \DF_f \cap \DF_g \\ \\
(4) & (f/g)(x)&=\ f(x)/g(x), \quad & x\in \DF_f \cap \DF_g\ \wedge\ g(x) \neq 0
\end{array} 
\]
\end{Def}
Ainoa uusi piirre lukujonoihin nähden on, että funktioita yhdisteltäessä on määrittelyjoukkoa 
rajoitettava, ellei ole $\DF_f=\DF_g$. Perusmuotoisten 'jonofunktioiden' tapauksessa tätä
ongelmaa ei ollut, koska määrittelyjoukko oli aina sama $(=\N)$. Huomattakoon erityisesti, että
funktion $f/g$ määrittelyjoukko on ym. määritelmän mukaisesti
\[
\DF_{f/g}=\{x\in\R \ | \ x\in\DF_f\ \wedge\ x\in\DF_g\ \wedge\ g(x)\neq 0\}.
\]
Tässä vaatimus $g(x)\neq 0$ esiintyi jo lukujonojen yhteydessä, vrt. Lause 
\ref{raja-arvojen yhdistelysäännöt}.
\begin{Exa}
Määritelmän \ref{funktioiden yhdistelysäännöt} säännön (4) mukaisesti on
\[
\tan=\sin/\cos, \quad \cot = \cos/\sin. \loppu
\]
\end{Exa}

\subsection{Parilliset ja parittomat funktiot}
\index{funktio B!b@parillinen, pariton|vahv}
\index{parillinen, pariton!b@funktio|vahv}

\begin{Def} \label{parilliset ja parittomat funktiot} Jos $f: \DF_f \kohti \R$ ja $f$:n 
määrittelyjoukko $\DF_f \subset \R$ on origon suhteen symmetrinen, ts. pätee 
$-x \in \DF_f\ \forall x\in\DF_f$, niin sanotaan, että $f$ on \kor{parillinen} (engl. even), jos
\[
f(-x)=f(x) \quad \forall x\in\DF_f,
\]
ja \kor{pariton} (engl. odd), jos
\[
f(-x)=-f(x) \quad \forall x\in\DF_f.
\] \end{Def}
\begin{Exa} Trigonometrisistä funktioista kosini on parillinen ja sini pariton. Potenssifunktio
$f(x) = x^n,\ n \in \Z$, on parillinen/pariton kun $n$ on parillinen/pariton. Funktio 
$f(x)=0\ \forall x\in\DF_f$ (esim.\ $\DF_f=\R$) on funktioista ainoa, joka on sekä parillinen
että pariton. \loppu 
\end{Exa}

Yhdisteltäessä funktioita Määritelmän \ref{funktioiden yhdistelysäännöt} mukaisesti ovat 
seuraavat säännöt helposti todennettavissa:
\begin{itemize}
\item[(1)]  $f$ ja $g$ parillisia/parittomia $\ \impl\ $ $f+g$, $\lambda f$ ja $1/f$ 
            parillisia/parittomia
\item[(2a)] $f$ ja $g$ parillisia/parittomia $\ \impl\ $ $fg$ parillinen
\item[(2b)] $f$ parillinen ja $g$ pariton $\ \impl\ $ $fg$ pariton
\end{itemize}
\jatko \begin{Exa} (jatko) Ym.\ sääntöjen perusteella (ja muutenkin) voidaan päätellä: 
a) Trigonometriset funktiot $\tan$ ja $\cot$ ovat parittomia. b) Polynomi 
$f(x) = \sum_{k=0}^n a_k x^k$ on parillinen/pariton täsmälleen kun $a_k = 0$ jokaisella 
parittomalla/parillisella indeksin $k$ arvolla. \loppu 
\end{Exa} 
Jos funktio on origon suhteen symmetrisesti määritelty, mutta parillisuuden suhteen 'epäpuhdas',
niin se voidaan aina esittää parillisen ja parittoman funktion summana. Nimittäin 
$f = f_+ + f_-$, missä
\[
f_+(x)=\frac{1}{2}[f(x)+f(-x)], \quad f_-(x)=\frac{1}{2}[f(x)-f(-x)].
\]
\begin{Exa}
\[
f(x)\ =\ \begin{cases} 0, & \text{kun } x \leq 0 \\ x, & \text{kun } x > 0 \end{cases}\ \
      =\ \ \frac{1}{2}\,\abs{x} + \frac{1}{2}x\,\quad \forall x \in \R. \loppu
\]
\end{Exa}
\begin{figure}[H]
\setlength{\unitlength}{1mm}
\begin{center}
\begin{picture}(120,35)
\put(18,20){\vector(1,0){24}} \put(41,16){$x$}
\put(30,20){\vector(0,1){12}} \put(32,30){$y$}
\put(48,20){\vector(1,0){24}} \put(71,16){$x$}
\put(60,20){\vector(0,1){12}} \put(62,30){$y$} 
\put(78,20){\vector(1,0){24}} \put(101,16){$x$}
\put(90,20){\vector(0,1){12}} \put(92,30){$y$}
\linethickness{0.5mm}
\curve(18,20,30,20)
\curve(30,20,42,32)
\curve(48,26,60,20)
\curve(60,20,72,26)
\curve(78,14,102,26)
\put(29,5){$f$}
\put(59,5){$f_+$}
\put(89,5){$f_-$}
\end{picture}
%\caption{Funktion esittäminen parillisen ja parittoman funktion summana}
\end{center}
\end{figure}

\subsection{Jaksolliset funktiot}

\begin{Def} \index{funktio B!c@jaksollinen|emph} \index{jaksollinen funktio|emph}
\index{periodinen funktio} 
Reaalifunktio $f$ on \kor{jaksollinen} eli \kor{periodinen}, jos jollakin
$a\in\R_+$ pätee $\,x \pm a\in\DF_f\ \forall x\in\DF_f$ ja
\[
f(x+a)=f(x) \quad \forall x\in\DF_f.
\]
Tällöin $a$ on $f$:n \kor{jakso}.
\end{Def}
Määritelmän mukaisesti myös jokainen jakson monikerta on jakso. Jaksoista pienin on nimeltään
\kor{perusjakso} (usein vain 'jaksoksi' sanottu).
\begin{Exa} Funktiot $\,\abs{\sin x}$, $\,\abs{\cos x}$, $\,\tan x$ ja $\,\cot x\,$ ovat 
jaksollisia, \newline (perus)jaksona $a=\pi$. \loppu
\end{Exa} 
\begin{Exa} Jos $\DF_f=\R$, $f$:n jakso on $a=1$ ja $f(x)=x$, kun $x\in[0,1)$, niin
\[
f(x) = x-k, \quad \text{kun}\ x\in[k,k+1),\ k\in\Z.
\]
\end{Exa} 
\begin{figure}[H]
\setlength{\unitlength}{1cm}
\begin{picture}(14,3)(-2,-1)
\put(-2,0){\vector(1,0){14}} \put(11.8,-0.4){$x$}
\put(0,-2){\vector(0,1){4}} \put(0.2,1.8){$y$}
%\multiput(3.14,0)(3.14,0){3}{\drawline(0,-0.1)(0,0.1)}
\multiput(-1,0)(1,0){12}{\line(1,1){1}}
\multiput(-1.07,-0.07)(1,0){12}{$\scriptstyle{\bullet}$}
\put(-1.2,-0.4){$-1$} \put(0.1,-0.4){$0$} \put(1,-0.4){$1$} \put(2,-0.4){$2$}
\end{picture}
\end{figure}

\Harj
\begin{enumerate}

\item
Määritä algebran keinoin seuraavien reaalifunktioiden arvojoukot: \newline
a) \ $x^2+2x+8,\quad$
b) \ $1-x-4x^2 \quad$
c) \ $1/(2+x+x^2) \quad$
d) \ $1/(1-\sqrt{x})$ \newline
e) \ $x^2/(1-x^2) \quad$
f) \ $(x+1)/(x+2) \quad$
g) \ $\abs{x}+\abs{x+1} \quad$
h) \ $\abs{x}-\abs{x+2}$ \newline
i) \ $\sum_{k=0}^\infty x^k \quad$
j) \ $\sum_{k=0}^\infty (-1)^k 2^k x^k \quad$ 
k) \ $\sum_{k=0}^\infty x^{2k} \quad$
l) \ $\sum_{k=0}^\infty (-1)^k x^{2k}$

\item
Selvitä algebran keinoin, millä väleillä seuraavat funktiot ovat aidosti kasvavia ja millä
aidosti väheneviä: \,\ a)\, $1/x^4$, \,\ b)\, $x/(x+1)$, \,\ c)\, $\abs{x^2+x-2}$, \newline 
d)\, $1/(x^2+2x+2)$, \,\ e)\, $1/(x^2+3x+2)$, \,\ f)\, $x^4/(2-x^4)$ \,\ 
g)\, $\max\{0,\sin x\}$

\item
Mitkä ovat seuraavien yhdistettyjen funktioiden määrittely- ja arvojoukot? \newline
a) \ $\sqrt{8-2x} \quad$ 
b) \ $\sqrt{1-x-x^2}\quad$
c) \ $1/(1-\sqrt{2-x}) \quad$
d) \ $\cos(\sin x)$

\item
Määritä yhdistettyjen funktioiden $f \circ g$ ja $g \circ f$ laskusäännöt ja määrittelyjoukot
seuraavissa tapauksissa: \newline
a) \ $f(x)=1/\sqrt{x+1},\,\ g(x)=\sqrt{x-1}$ \newline
b) \ $f(x)=\sqrt{x+1},\,\ g(x)=x/(1-2x) \quad$ \newline
c) \ $f(x)=\sqrt{2x+1},\,\ g(x)=\sin x$

\item
a) Olkoon $f(x)=1-4\sqrt{x}$. Mitkä ovat funktioiden $f \circ f$ ja $f \circ f \circ f$
määrittelyjoukot? \ b) Olkoon $g(x)=\frac{1}{2}x+1$. Millainen reaalifunktio on 
$f(x)=\lim_n g_n(x)$, missä $g_1=g$, $g_2=g \circ g$, $g_3=g \circ g \circ g$, jne.\,?

\item
Määritä algebran keinoin seuraavien funktioiden arvojoukot: \newline
a) \ $1-\sqrt[4]{17x}+\sqrt{x} \quad$
b) \ $2+5x^{48}-x^{96} \quad$
c) \ $x-4\sqrt{x}+3-4\abs{\sqrt{x}-1}$ \newline
d) \ $x^4/(2-x^4) \quad$
e) \ $\cos x-7\sin x \quad$
f) \ $\sin x + \cos 2x \quad$ 
g) \ $2\sin x - \abs{\cos 2x}$ 

\item
a) Näytä, että jos $g$ on kasvava välillä $A\subset\DF_g$ ja $f$ on kasvava välillä 
$B=g(A)\subset\DF_f$, niin $f \circ g$ on kasvava välillä $A$. Sovella väittämää funktioon
$f(x)=\sqrt{x^2+x-2}$ välillä $[1,\infty)$. \newline
b) Näytä, että jos $f$ ja $g$ ovat (aidosti) kasvavia välillä $A\subset\DF_f\cap\DF_g$, niin
samoin on funktio $f+g$. Miten on funktion $fg$ laita?

\item
Määrittele $f+g$, $fg$ ja $f/g$ (määrittelyjoukko ja sievennetty laskusääntö), kun \ 
a) \ $f(x)=g(x)=x+1$, \ \ 
b) \ $f(x)=\abs{x}+x,\ g(x)=\abs{x}-x$, \newline
c) \ $f(x)=\sin x \sin 2x,\ g(x)=2\cos^3 x$.

\item
Näytä: a) Funktion jako parilliseen ja parittomaan osaan on yksikäsitteinen.
b) Jos $g$ on parillinen, niin $f \circ g$ on parillinen tai ei määritelty.
c) Jos $f$ on parillinen/pariton ja $g$ on pariton, niin $f \circ g$ on parillinen/pariton
   tai ei määritelty.

\item
Jaa seuraavat funktiot parilliseen ja parittomaan osaan: \newline
a) \ $f(x)=2+x-3x^2+x^4+x^5+\sin x-3\cos x$ \newline
b) \ $f(x)=|x+1|+|x-1|-x+2x^2$\newline
c) \ $f(x)=\sin(x+\frac{\pi}{4})+\cos(x+\frac{\pi}{3})$

\item \label{H-IV-1: näyttöjä}
a) Näytä, että jos $f$ on (aidosti) kasvava/vähenevä väleillä $A_1$ ja $A_2$ ja
$A_1 \cap A_2 \neq \emptyset$, niin $f$ on (aidosti) kasvava/vähenevä välillä 
$A=A_1 \cup A_2$. \newline
b) Olkoon $\DF_f=\R$ ja $f$ (aidosti) kasvava välillä $[0,\infty)$. Näytä, että jos lisäksi
$f$ on parillinen/pariton, niin $f$ on (aidosti) vähenevä välillä $(-\infty,0]$\,/\, 
(aidosti) kasvava $\R$:ssä.

\item
Funktio $f(x)=\sin\frac{x}{5}+\cos\frac{x}{7}$ on jaksollinen. Mikä on perusjakso? 

\item
Funktion $g$ jakso on $a=2$ ja $g(x)=1-\abs{x}$, kun $x\in[-1,1]$. Määritä funktion
$f(x)=g(x)+cx-5$ nollakohdat, kun\, a) $c=1$, \ b) $c=1/2$.

\item
Määritellään porrasfunktio
\[
f(x)=k, \quad \text{kun}\ \ 2k-2 \le x < 2k,\ k\in\Z.
\]
Määritä $a\in\R$ ja jaksollinen funktio $g$ siten, että $f(x)=ax+g(x),\ x\in\R$.

\item (*) \label{H-IV-1: funktioalgebran haasteita}
Todista pelkin algebran keinoin: \vspace{1mm}\newline
a) Funktion $f(x)=(3x^2+3)/(x^2+x+1)$ arvojoukko on $\RF_f=[2,6]$. \newline
b) Funktio $f(x)=x^2/(x+1)$ on aidosti kasvava väleillä $(-\infty,-2]$ ja $[0,\infty)$
ja aidosti vähenevä väleillä $[-2,-1)$ ja $(-1,0]$.

\end{enumerate} % Yhden muuttujan funktiot
\section{Käänteisfunktio. Implisiittifunktiot} \label{käänteisfunktio}
\sectionmark{Käänteisfunktio} 
\alku
\index{funktio B!i@käänteisfunktio|vahv}
\index{kzyzy@käänteisfunktio|vahv}

Sanotaan, että reaalifunktio $f: \DF_f\kohti\R$ on 1-1 ('yksi yhteen', engl.\
one to one) eli \kor{kääntyvä} (engl.\ invertible) eli
\index{injektio}%
\kor{injektio} eli \kor{injektiivinen}, jos pätee
\[
\forall x_1, x_2 \in \DF_f\ [\,x_1 \neq x_2 \ \impl \ f(x_1) \neq f(x_2)\,].
\]
Jos $f$ on 1-1 ja $y\in \RF_f$, niin yhtälöllä
\[
f(x)=y \quad (x \in \DF_f)
\]
on y\pain{ksikäsitteinen} ratkaisu. Koska jokaiseen $y\in\RF_f$ liittyy tällä tavoin
yksikäsitteinen $x\in\DF_f$, niin kyseessä on funktioriippuvuus $y \map x$, joka
merkitään 
\[
x=f^{-1}(y).
\]
Sanotaan, että $f^{-1}$ (luetaan '$f$ miinus 1', engl.\ '$f$ inverse') on $f$:n
\kor{käänteisfunktio}. Käänteisfunktion määrittelyjoukko on siis $\DF_{\inv{f}}=\RF_f$ ja
arvojoukko $\RF_{\inv{f}}=\DF_f$.
\begin{Exa} Funktiolle $f(x)=1/x\ (\DF_f=\{x\in\R \mid x \neq 0\})$ pätee
\[
f(x_1)-f(x_2)\,=\,\frac{1}{x_1}-\frac{1}{x_2}\,=\,\frac{x_2-x_1}{x_1x_2}\,, \quad 
                                                            x_1\,,x_2 \in \DF_f\,.
\]
Tämän perusteella on $f(x_1)-f(x_2) \neq 0\,$ aina kun $x_1,x_2 \neq 0\,$ ja $x_1 \neq x_2$,
joten $f$ on 1-1. Käänteisfunktio löydetään ratkaisemalla yhtälö $f(x)=y$\,:
\[
x^{-1}=y \qimpl x=y^{-1}, \quad \text{jos}\,\ y \neq 0.
\]
Jos $y=0$, ei yhtälöllä ole ratkaisua, joten käänteisfunktion määrittelyjoukko ($=f$:n 
arvojoukko) $=\{y\in\R \mid y \neq 0\}$, ja $f^{-1}(y)=1/y$. Koska siis $\DF_{f^{-1}}=\DF_f$ ja 
$f^{-1}(y)=f(y)\ \forall y \in \DF_{f^{-1}}=\DF_f$, niin $f^{-1}=f$. \loppu
\end{Exa}

Määritelmän \ref{monotoninen funktio} nojalla on selviö, että jos funktion $f$ 
määrittelyjoukko on \pain{väli} ja $f$ on ko.\ välillä aidosti kasvava tai aidosti vähenevä
(eli aidosti monotoninen), niin $f$ on 1-1. Aito monotonisuus onkin käytännössä tavallisin
injektiivisyyden olomuoto silloin kun funktion määrittelyjoukko on väli (tai väliksi rajattu).
\begin{Exa} \label{x^m:n käänteisfunktio} Näytä, että funktio $f(x)=x^m,\ m\in\N$ on injektio,
kun määrittelyjoukko rajataan väliksi $[0,\infty)$. Määritä käänteisfunktio.
\end{Exa}
\ratk Kirjoitetaan (ks.\ Propositio \ref{kuntakaava})
\[
f(x_1)-f(x_2) \,=\, x_1^m-x_2^m \,=\, (x_1-x_2)(x_1^{m-1}+x_1^{m-2}x_2+\cdots +x_2^{m-1}).
\]
Jos tässä on $x_1\,,x_2 \ge 0$ ja $x_1 \neq x_2$, niin viimeksi kirjoitetun tulon 
jälkimmäisessä tekijässä on jokainen yhteenlaskettava ei-negatiivinen ja ainakin yksi on
positiivinen (koska joko $x_1>0$ tai $x_2>0$), joten ko.\ tekijä on positiivinen. Päätellään,
että $f$ on välillä $[0,\infty)$ aidosti kasvava:
\[
\forall x_1,x_2 \ge 0\,\ [\,x_1<x_2\ \impl\ f(x_1)<f(x_2)\,].
\]
Siis $f$ on injektio, joten yhtälöllä $f(x)=y\ (y\in\R)$ on enintään yksi ratkaisu $x$
välillä $[0,\infty)$. Jos $y<0$, ei ratkaisua ole. Jos $y=0$, on ratkaisu $x=0$. Lopulta
jos $y>0$, on ratkaisu myös olemassa ja merkitään $x=\sqrt[m]{y}$. (Luku $\sqrt[m]{y}\in\R_+$
on laskettavissa esim.\ kymmenjakoalgoritmilla, vrt.\ Luku \ref{reaaliluvut}.) Kysytty
käänteisfunktio on siis $\inv{f}(x)=\sqrt[m]{x}\,$, määrittelyjoukkona
$\DF_{\inv{f}}=\RF_f=[0,\infty)$. \loppu
\jatko \begin{Exa} (jatko) Jos $m$ on p\pain{ariton}, niin $f(x)=x^m,\ \DF_f=\R$, on pariton
ja näin muodoin aidosti kasvava koko $\R$:ssä
(ks.\ Harj.teht. \ref{yhden muuttujan funktiot}:\ref{H-IV-1: näyttöjä}b). Siis $f$ on
injektio. Yhtälöllä $f(x)=y$ on tällöin (yksikäsitteinen) ratkaisu jokaisella $y\in\R$, sillä
jos $y<0$, niin ratkaisu on $x=-\sqrt[m]{-y}$. Käänteisfunktio merkitään yleensä
yksinkertaisesti $f^{-1}(x) = \sqrt[m]{x}$, jolloin siis sovitaan, että
\[
\sqrt[m]{-x}=-\sqrt[m]{x}, \quad \text{kun $x>0$ ja $m$ on pariton}.\footnote[2]{Juurilukujen
määrittelyn laajentaminen mainitulla tavalla ei ole aivan ongelmatonta, kuten nähdään laskusta
\[
1=\sqrt[6]{1}=\sqrt[6]{(-1)^2}=(-1)^{2/6}=(-1)^{1/3}=\sqrt[3]{-1}=-1.
\]
Tämän tyyppisten ristiriitojen välttämiseksi on selvintä sopia, että reaalinen
potenssifunktio $f(x)=x^\alpha$ on määritelty välillä $(-\infty,0)$ vain kun $\alpha\in\Z$.}
\loppu \]
\end{Exa}
Seuraava yleinen sääntö on Määritelmästä \ref{monotoninen funktio} ja käänteisfunktion
määritelmästä helposti johdettavissa (Harj.teht.\,\ref{H-IV-2: todistus}a)\,:
\[
\boxed{
\begin{aligned} 
\quad\ykehys f\,\ &\text{aidosti kasvava/vähenevä välillä}\ A\subset\DF_f \\ 
                  &\qimpl \inv{f}\,\ \text{aidosti kasvava/vähenevä väleillä}\ 
                                                          B \subset f(A). \quad\akehys
\end{aligned} }
\]
\jatko \begin{Exa} (jatko) Koska $f(x)=x^m\ (m\in\N)$ on aidosti kasvava välillä
$A=[0,\infty)$, niin $\inv{f}(x)=\sqrt[m]{x}$ on samoin aidosti kasvava välillä
$[0,\infty)=f(A)$. \loppu
\end{Exa}
Käänteisfunktion määritelmästä helposti todennettavissa ovat myös yleiset funktioalgebran lait
\[
\boxed{ \begin{aligned}
\ykehys\quad(\inv{f} \circ f)(x) &= x \quad \forall x\in \DF_f, \quad\\
            (f \circ \inv{f})(y) &= y \quad \forall y\in \RF_f, \\
                 \inv{(\inv{f})} &= f. \akehys
\end{aligned} }
\]
Käänteisfunktion $\inv{f}$ kuvaaja saadaan vaihtamalla $x$ ja $y$ $f$:n kuvaajassa, eli 
peilaamalla $f$:n kuvaaja suoran $y=x$ suhteen. Kuvassa $f(x)=x^2,\ \DF_f = [0,\infty)$.
\begin{figure}[H]
\setlength{\unitlength}{1cm}
\begin{center}
\begin{picture}(7,6.5)(-1,-1.5)
\put(0,0){\vector(1,0){6}} \put(5.8,-0.4){$x$}
\put(0,-1.5){\vector(0,1){6.5}} \put(0.2,4.8){$y$}
\multiput(1,0)(1,0){2}{\drawline(0,-0.1)(0,0.1)}
\multiput(0,1)(0,1){2}{\drawline(-0.1,0)(0.1,0)}
\put(0.9,-0.5){$1$} \put(1.9,-0.5){$2$}
\put(-0.5,-0.15){$0$} \put(-0.5,0.9){$1$} \put(-0.5,1.9){$2$}
\curve(0,0,1,1,2,4) \put(1.5,4.2){$y=f(x)$}
\curve(0,0,1,1,4,2) \put(4.3,2){$y=\inv{f}(x)$}
\dashline{1}(0,0)(4,4) \put(4,4.2){$y=x$}
\end{picture}
%\caption{Käänteisfunktion $\inv{f}$ kuvaaja, kun $f(x)=x^2$}
\end{center}
\end{figure}
\begin{Exa} \label{algebrallinen käänteisfunktio} Tutki funktion $f(x)=x^5+3x,\ \DF_f=\R$ 
(mahdollista) käänteisfunktiota. \end{Exa}
\ratk Koska $f$ on pariton ja 
\[
f(x_1)-f(x_2)=(x_1-x_2)(x_1^4+x_1^3x_2+x_1^2x_2^2+x_1x_2^3+x_2^4+3),
\]
niin päätellän kuten Esimerkissä \ref{x^m:n käänteisfunktio}, että $f$ on aidosti 
kasvava koko $\R$:ssä. Käänteisfunktio $\inv{f}: \RF_f \kohti \R$ on siis olemassa. Kun
$y \in \RF_f$, niin funktioevaluaatio $y \map x = \inv{f}(y)$ tarkoittaa yhtälön
\[
x^5+3x=y
\]
ratkaisemista. Ratkeavuus jokaisella $y>0$ on osoitettavissa
(Harj.teht.\,\ref{H-IV-2: ratkeavuus kymmenjaolla}), ja koska $f$ on pariton, niin yhtälö
ratkeaa myös jokaisella $y \le 0$, t.s.\ $\RF_f=\DF_{\inv{f}}=\R$. Nähdään myös, että 
esimerkiksi $\inv{f}(0)=0$, $\inv{f}(4)=1$, ja $\inv{f}(-38)=-2$. Sen sijaan vaikkapa lukua
$a=\inv{f}(1)$ ei voi määrätä 'tarkasti' edes juurilukujen avulla, vaan kyseessä on yleisempi
(algebrallinen, vrt.\ Luku \ref{reaalilukujen ominaisuuksia}) luku. Tällaisen luvun
määrittelyssä on tyydyttävä reaaliluvun yleiseen määritelmään, esim.\ äärettömänä
desimaalilukuna (ks.\ Harj.teht.\,\ref{H-IV-2: ratkeavuus kymmenjaolla}). Symbolinen laskenta
luvulla $a$ on toki myös mahdollista, mutta tällainen laskenta määritelmän $a^5 + 3a = 1$
perusteella on varsin rajoitettua. --- Laihan lohdun suokoon tulos
\[
\inv{a}=a^4+3. \loppu
\]

\subsection{Injektio, surjektio ja bijektio}
\index{injektio|vahv} \index{surjektio|vahv} \index{bijektio|vahv}

Kuten Esimerkissä \ref{x^m:n käänteisfunktio} edellä, on käänteisfunktioita tutkittaessa varsin 
tavallista, että $f$ ei välttämättä ole injektiivinen koko määrittelyjoukossaan, mutta on 
kuitenkin injektio, jos määrittelyjoukkoa sopivasti rajoitetaan. Käänteisfunktiotarkastelujen 
lähtökohdaksi voidaan tällöin ottaa rajoitettu kuvaus
\[
f:A\rightarrow B,
\]
missä $A\subset D_f$ ja $B \supset f(A)$. Yleensä $A$ on $D_f$:n jokin \pain{osaväli}. Joukkoon
$A$ rajattua funktiota voidaan haluttaessa merkitä $f_{|A}$ ja sanoa, että kyseessä on 
\index{rajoittuma (funktion)}%
\kor{f:n rajoittuma $A$:lle}. Jos tässä $B$ on vielä onnistuttu valitsemaan siten, että
$B=f(A)$, niin sanotaan, että $f:A \kohti B$ on \kor{surjektio B:lle} (ransk. sur jeter = 
heittää päälle; englanninkielinen 'f onto B' sisältää saman ajatuksen). Termiä käytetään myös
silloin kun $f$ ei ole injektio. Siis funktiosta $f: A \kohti B$ saadaan surjektio
yksinkertaisesti rajaamalla maalijoukko $f(A)$:ksi. Jos $f:A \kohti B$ ($=f_{|A}$) on sekä
injektio että surjektio, niin sanotaan, että $f:A \kohti B$ on \kor{bijektio} (engl. one to one
and onto). Tällöin myös käänteisfunktio $\inv{f}:B \kohti A$ (tässä siis 
$\inv{f} = (f_{|A})^{-1}$) on bijektio.\footnote[2]{Termit injektio, surjektio ja bijektio eivät
rajoitu reaalifunktioihin, vaan ne voidaan liittää yhtä hyvin yleiseen joukko-opilliseen
funktiokäsitteeseen (ks.\ Luku \ref{trigonometriset funktiot}). Jos $A$ ja $B$ ovat joukkoja,
ja on olemassa bijektio $f:A \kohti B$, niin sanotaan, että $A$:n ja $B$:n välillä on
(bijektion $f$ luoma) \kor{kääntäen yksikäsitteinen vastaavuus} ja merkitään $A \vast B$.
Esim.\ Luvun \ref{tasonvektorit} merkinnässä $\Ekaksi\vast\Rkaksi$ tarkoitetaan tällaista
vastaavuutta annetussa tason koordinaatistossa.
\index{vastaavuus ($\ensuremath  {\leftrightarrow }$)|av}} 
\begin{Exa} Jos $f(x)=x^m$, niin $f:\,[0,\infty)\kohti[0,\infty)$ on bijektio jokaisella
$m\in\N$, samoin esim.\ $f:\,(0,\infty)\kohti(0,\infty)$, $f:\,[0,1]\kohti[0,1]$ ja
$f:\,[1,\infty)\kohti[1,\infty)$ (vrt.\ Esimerkki \ref{x^m:n käänteisfunktio}). \loppu
\end{Exa} 
\begin{Exa} Jos $f(x)=x^2$, niin \vspace{2mm}\newline
$f:\ [-2,3]\kohti[0,10]\ \ $     on funktio, mutta ei injektio eikä surjektio. \newline
$f:\ [-2,3]\kohti[0,9]\quad\,$   on surjektio, ei injektio. \newline
$f:\ [0,3]\kohti[0,10]\quad\,\ $ on injektio, ei surjektio: käänteisfunktio
     $f^{-1}(x)=\sqrt{x}$. 
\newline
$f:\ [0,3]\kohti[0,9]\qquad$     on bijektio: käänteisfunktio $f^{-1}(x)=\sqrt{x}$. \newline
$f:\ [-2,-1]\kohti[1,4]\,\ $     on bijektio: käänteisfunktio $f^{-1}(x)=-\sqrt{x}$. \loppu
\end{Exa}

\subsection{Trigonometriset käänteisfunktiot}
\index{funktio C!d@$\Arcsin$, $\Arccos$, $\Arctan$, $\Arccot$|vahv}

Trigonometriset funktiot eivät ole injektiivisiä koko määrittelyjoukossaan, joten ne eivät ole 
'kääntyviä' tavallisessa mielessä. Trigonometrisilla käänteisfunktioilla tarkoitetaankin 
tavallisimmin funktioita, jotka saadaan rajoittamalla trigonometrinen funktio joko välille 
$[-\frac{\pi}{2},\frac{\pi}{2}]$ tai $(-\frac{\pi}{2},\frac{\pi}{2})$ ($\sin$, $\tan$) tai 
välille $[0,\pi]$ tai $(0,\pi)$ ($\cos$, $\cot$), jolloin funktio on ko. välillä aidosti 
monotoninen. Näin saatuja käänteisfunktioita sanotaan
\index{arkusfunktiot} \index{syklometriset funktiot}%
\kor{arkusfunktioiksi} tai \kor{syklometrisiksi funktioiksi}. Määritelmät ovat
\begin{alignat*}{2}
x&=\sin y\ \ \ja \ &y \in \left[-\tfrac{\pi}{2},\tfrac{\pi}{2}\right] \ & \ekv\ y=\Arcsin x, \\
x&=\cos y\,\ \ja \ &y \in \left[0,\pi\right]\quad\ \ & \ekv \ y=\Arccos x, \\
x&=\tan y \ \ja \ &y \in \left(-\tfrac{\pi}{2},\tfrac{\pi}{2}\right) \ & \ekv\ y=\Arctan x, \\
x&=\cot y\,\ \ja \ &y \in \left(0,\pi\right)\quad\ & \ekv \ y=\Arccot x.
\end{alignat*}
Nämä luetaan 'arkus sini' jne. Kyseessä ovat trigonometristen käänteisfunktioiden nk.\
\index{pzyzy@päähaara (funktion)}%
\kor{päähaarat}.\footnote[2]{ Muille väleille rajoitettujen trigonometristen funktioiden 
käänteisfunktioista käytetään tässä tekstissä yhteismerkintää $\arcsin$, jne. --- ks.\ tarkempi
määrittely luvun lopussa. Kirjallisuudessa trigonometristen käänteisfunktioiden merkintätavat
ovat hieman kirjavat. Vanhempia päähaarojen merkintöjä ovat $\overline{\text{arc}}$sin, 
$\overline{\text{arc}}$cos, $\overline{\text{arc}}$tan, $\overline{\text{arc}}$cot. Laskimissa
tavallisia merkintöjä $\,\sin^{-1}$, $\,\cos^{-1}$ ja $\,\tan^{-1}$ näkee myös käytettävän 
kirjallisuudessa. --- Tässä tekstissä määritellään $\,\sin^n x=(\sin x)^n\ \forall n\in\Z$,
jolloin $\sin^{-1}x=1/\sin x$.}

Arkusfunktioiden määrittely- ja arvojoukot ovat määritelmien perusteella seuraavat:
\[
\begin{array}{llll}
\Arcsin x &: \quad & \DF_f=[-1,1], \quad & \RF_f=[-\tfrac{\pi}{2},\tfrac{\pi}{2}]. \\
\Arccos x &: \quad & \DF_f=[-1,1], \quad & \RF_f=[0,\pi]. \\
\Arctan x &: \quad & \DF_f=\R, \quad     & \RF_f=(-\tfrac{\pi}{2},\tfrac{\pi}{2}). \\
\Arccot x &: \quad & \DF_f=\R, \quad     & \RF_f=(0,\pi).
\end{array}
\]
Funktiot $\Arcsin$ ja $\Arctan$ ovat määrittelyvälillään aidosti kasvavia (kuten $\sin$ ja
$\tan$), muut kaksi aidosti väheneviä.
\begin{figure}[H]
\setlength{\unitlength}{1.3cm}
\begin{picture}(5,5.4)(-2.2,-2.7)
\put(-2,0){\vector(1,0){4}} \put(2.2,-0.1){$x$}
\put(0,-2){\vector(0,1){4.4292}} \put(0,2.6292){$y$}
\put(-0.9,-2.6){$y=\Arcsin x$}
\put(-1,0){\line(0,1){0.07}} \put(1,0){\line(0,1){0.07}}
\put(0,-1.5708){\line(1,0){0.07}} \put(0,1.5708){\line(1,0){0.07}}
\put(-1.3,-0.4){$-1$} \put(0.95,-0.4){$1$}
\put(0.15,-1.66){$-\tfrac{\pi}{2}$} \put(0.25,1.5){$\tfrac{\pi}{2}$}
\curve(
-1.0000,  -1.5708,
-0.8776,  -1.0708,
-0.5403,  -0.5708,
-0.0707,  -0.0708,
 0.0707,   0.0708,
 0.5403,   0.5708,
 0.8776,   1.0708,
 1.0000,   1.5708)
\put(4,-1.5708){\vector(1,0){4}} \put(8.2,-1.6708){$x$}
\put(6,-1.5708){\vector(0,1){4}} \put(6,2.6292){$y$}
\put(5,-1.5708){\line(0,1){0.07}} \put(7,-1.5708){\line(0,1){0.07}}
\put(6,0){\line(1,0){0.07}} \put(6,1.5708){\line(1,0){0.07}}
\put(4.7,-1.9708){$-1$} \put(6.95,-1.9708){$1$}
\put(6.25,-0.0708){$\tfrac{\pi}{2}$} \put(6.25,1.5292){$\pi$}
\put(5.1,-2.6){$y=\Arccos x$}
\curve(
 5.0000,   1.5708,
 5.0200,   1.3705,
 5.0500,   1.2532,
 5.1000,   1.1198,
 5.1500,   1.0160,
 5.4000,   0.6435,
 5.7000,   0.3047,
 6.0000,   0.0000,
 6.3000,  -0.3047,
 6.6000,  -0.6435,
 6.8500,  -1.0160,
 6.9000,  -1.1198, 
 6.9500,  -1.2532,
 6.9800,  -1.3705,
 7.0000,  -1.5708)
\end{picture} 
\end{figure}

\begin{figure}[H]
\setlength{\unitlength}{1cm}
\begin{picture}(15,4.5)(-2.2,-2)
\put(-1,0){\vector(1,0){12}} \put(11.2,-0.1){$x$}
\put(5,-2){\vector(0,1){4}} \put(5,2.2){$y$}
\put(7.5,1.6){$y=\Arctan x$}
\put(4,0){\line(0,1){0.1}} \put(6,0){\line(0,1){0.1}}
\put(5,-1.5708){\line(1,0){0.1}} \put(5,1.5708){\line(1,0){0.1}}
\put(3.7,-0.4){$-1$} \put(5.95,-0.4){$1$}
\put(5.15,-1.66){$-\tfrac{\pi}{2}$} \put(5.25,1.5){$\tfrac{\pi}{2}$}
\curve(
-0.5000,  -1.3909,
 1.0000,  -1.3258,
 2.0000,  -1.2491,
 3.0000,  -1.1072,
 3.5000,  -0.9828,
 4.0000,  -0.7854,
 4.3000,  -0.6107,
 4.6000,  -0.3805,
 4.8000,  -0.1974,
 5.0000,   0.0000,
 5.2000,   0.1974,
 5.4000,   0.3805,
 5.7000,   0.6107,
 6.0000,   0.7854,
 6.5000,   0.9828,
 7.0000,   1.1072,
 8.0000,   1.2491,
 9.0000,   1.3258,
10.5000,   1.3909) 
\end{picture} 
\end{figure}

Funktiopareista $\,(\Arcsin,\,\Arccos)\,$ ja $\,(\Arctan,\,\Arccot)\,$ riittää sovelluskäyttöön
valita yksi edustaja kummastakin parista (tavallisimmin valitaan $\Arcsin$ ja $\Arctan$), sillä
parien 'jäsentenvälinen' ratkeaa yksinkertaisiin yhteyksiin
\[ 
\boxed{ \begin{aligned}
\ykehys\quad \Arcsin x + \Arccos x &= \frac{\pi}{2} \quad \forall x\in [-1,1], \quad \\
\akehys\quad \Arctan x + \Arccot x &= \frac{\pi}{2} \quad \forall x\in\R.
\end{aligned} }
\]  
Näistä esimerkiksi ensimmäinen seuraa päättelyllä (vrt.\ kuviot edellä)
\begin{align*}
&\begin{cases}
\,\dfrac{\pi}{2}-\Arccos x \in \left[-\dfrac{\pi}{2},\dfrac{\pi}{2}\right] \quad 
                           &\forall x\in[-1,1] \\[2mm]
\,\sin\left(\dfrac{\pi}{2} - \Arccos x\right) = \cos (\Arccos x) = x \quad 
                           &\forall x\in[-1,1]
\end{cases} \\
&\qimpl \frac{\pi}{2}- \Arccos x =\Arcsin x \quad \forall x\in[-1,1].
\end{align*}
\begin{Exa} Hahmottele funktion $\Arcsin(\cos x)$ kuvaaja. \end{Exa}
\ratk Sievennetään ensin $\,y=\Arcsin(\cos x)$\,:
\begin{align*}
y&\,=\,\Arcsin (\cos x) \\[1mm]
 &\,\ekv\ \ y \in \left[-\frac{\pi}{2},\frac{\pi}{2}\right] \ \ja \ \sin y = \cos x \\
 &\,\ekv\ \ y \in \left[-\frac{\pi}{2},\frac{\pi}{2}\right] \ \ja \ 
                                             \sin y = \sin \left(\frac{\pi}{2} -x\right) \\
 &\,\ekv\ \ y \in \left[-\frac{\pi}{2},\frac{\pi}{2}\right] \ \ja \ 
                                    y = \frac{\pi}{2} \pm x + n \cdot 2\pi, \ n\in\Z.
\end{align*}
Tämän perusteella kuvaaja koostuu joukosta suoria rajoitettuna välille 
$y\in[-\tfrac{\pi}{2},\tfrac{\pi}{2}]$ (ks.\ kuvio). \loppu
\begin{figure}[H]
\begin{center}
\setlength{\unitlength}{1cm}
\begin{picture}(14,5.5)(-7,-2)
\put(-7,0){\vector(1,0){14}} \put(6.8,-0.4){$x$}
\put(0,-2.5){\vector(0,1){6}}  \put(0.2,3.3){$y$}
\drawline(-6.8,1.05)(-6.28,1.57)(-3.14,-1.57)(0,1.57)(3.14,-1.57)(6.28,1.57)(6.8,1.05)
\multiput(-1.57,0)(3.14,0){2}{\drawline(0,-0.1)(0,0.1)}
\put(-2,-0.6){$-\tfrac{\pi}{2}$} \put(1.45,-0.6){$\tfrac{\pi}{2}$} 
\put(-0.7,1.4){$\tfrac{\pi}{2}$}
\put(-5.5,1.5){\vector(0,-1){0.5} $n=-1$ $(-)$} \put(-2.5,-1.5){\vector(0,1){0.5} $n=0$ $(+)$} 
\put(0.8,1.5){\vector(0,-1){0.5} $n=0$ $(-)$} \put(3.8,-1.5){\vector(0,1){0.5} $n=-1$ $(+)$}
\end{picture}
\end{center}
\end{figure}

\subsection{Implisiittifunktio $y(x)$}
\index{funktio B!j@implisiittifunktio|vahv}
\index{implisiittifunktio|vahv}

Jos yhtälöstä muotoa
\[
F(x,y)=0
\]
on $y$ ratkaistavissa yksikäsitteisesti jokaisella $x \in A$ ($A\subset\R$), niin
sanotaan, että yhtälö määrittelee $A$:ssa \kor{implisiittifunktion} $y=f(x)$. Termillä
tarkoitetaan, että funktio on määritelty epäsuorasti eli \kor{implisiittisesti}. Tällöin
\pain{ei} edellytetä, että funktio on kirjoitettavissa suoraan (eli eksplisiittisesti)
tunnettuna lausekkeena. Yhtälössä $F$ on kahden reaalimuuttujan reaaliarvoinen funktio --- 
käytännössä lauseke, jossa $x$ ja $y$ esiintyvät. (Kahden muuttujan funktioita itsenäisinä
olioina käsitellään seuraavassa luvussa.)

Implisiittifunktiota merkittäessä halutaan usein säilyttää 'paikan tuntu' käyttämällä
yleismerkinnän $f(x)$ sijasta merkintää $y(x)$. Tämän mukaisesti siis funktioriippuvuus
$x \map y(x)$ määritellään epäsuoralla laskusäännöllä
\[
F(x,y(x))=0, \quad x \in A.
\]
Yksinkertainen esimerkki implisiittifunktiosta on käänteisfunktio, sillä jos
käänteisfunktion määrittelyssä vaihdetaan $x$ ja $y$, niin määritelmä on
\[
F(x,y)=x-f(y)=0,\,\ y\in\DF_f \qekv y(x)=\inv{f}(x).
\]
--- Itse asiassa myös tavallinen funktio muuttuu 'implisiittiseksi', kun $y=f(x)$ 
kirjoitetaan muotoon $F(x,y)=f(x)-y=0$.
\begin{Exa} Jos $b \neq 0$, niin tason suoran perusmuotoinen yhtälö
\[
ax+by+c=0
\]
voidaan tulkita implisiittifunktion $y(x)=-(ax+c)/b$ määrittelyksi. \loppu
\end{Exa} 
\begin{Exa} \label{muuan implisiittifunktio} Tarkasteltaessa yhtälöä
\[
F(x,y)=x+y+\cos x\sin y=0
\]
osoittautuu (tarkemmat perustelut sivuutetaan), että yhtälöllä on yksikäsitteinen ratkaisu
$y(x)$ jokaisella $x\in\R$, t.s.\ yhtälö määrittelee koko $\R$:ssä implisiittifunktion $y(x)$.
Yhtälöstä on pääteltävissä, että $y(x)=-x$ kun $x=(n+\tfrac{1}{2})\pi$ ja samoin kun
$x=n\pi$, $\,n\in\Z$. Muilla $x$:n arvoilla funktioevaluaatio $x \map y(x)$ on työläämpi, sillä
kyseessä on nk.\ 
\index{transkendenttinen yhtälö}%
\kor{transkendenttinen yhtälö}, joka ratkeaa vain numeerisin keinoin.
Yhtälöstä on myös $x$ ratkaistavissa yksikäsitteisesti jokaisella $y\in\R$. Näin määräytyy
(joskaan ei lausekkeena) funktio $y \map x(y)$, joka on funktion $x \map y(x)$ käänteisfunktio.
\loppu \end{Exa}

\subsection{Monihaaraiset implisiittifunktiot}
\index{funktio B!j@implisiittifunktio|vahv}
\index{implisiittifunktio|vahv}
\index{monihaarainen implisiittifunktio|vahv}

Implisiittifunktion määrittely-yrityksissä varsin tavallinen on tilanne, että yhtälö $F(x,y)=0$
ratkeaa $y$:n suhteen annetulla $x$, mutta ei yksikäsitteisesti. Olkoon tällaisessa tilanteessa
$A\subset\R$ väli (suljettu, avoin tai puoliavoin) ja tarkastellaan yhtälöä, kun $x \in A$.
Oletetetaan, että on olemassa välille $A$ ominainen indeksijoukko $\Lambda\subset\N$ 
(voi olla myös $\Lambda=\N$) ja funktioj\pain{oukko}
%S_i=\{P \vastaa (x,y) \in \R^2 \mid x \in A \ \ja \ y=f_i(x)\},
\[
\mathcal{Y}=\{y_i, \ i \in \Lambda \}
\]
siten, että pätee
\[
\forall x \in A\ [\,F(x,y)=0\ \ekv\ y\in\{y_i(x),\ i\in\Lambda\}\,].
\]
Sanotaan tällöin, että $\mathcal{Y}$ on yhtälön $F(x,y)=0$ välillä $A$ määrittelemä 
\kor{monihaarainen} (moniarvoinen) \kor{implisiittifunktio}. Monihaarainen implisiittifunktio
ei siis ole funktio vaan sellaisten joukko. Joukon $\mathcal{Y}$ alkioita $y_i$ sanotaan
\index{haara (funktion)}%
$\mathcal{Y}$:n \kor{haaroiksi} (engl. branch).\footnote[2]{Monihaarainen funktio $\mathcal{Y}$
voidaan myös tulkita \kor{joukkoarvoiseksi} määrittelemällä
\[ 
\mathcal{Y}(x) = \{y_i(x),\ i \in \Lambda\},
\]
missä $y_i,\ i \in \Lambda$ ovat $\mathcal{Y}$:n haarat tarkasteltavalla välillä $A$. Koska
$\mathcal{Y}(x)$ on (joukkona) yksikäsitteinen jokaisella $x \in A$, niin $\mathcal{Y}$ on
funktio tyyppiä $\mathcal{Y}: A \kohti \mathcal{B}$, missä maalijoukko $\mathcal{B}$ on
'joukkojen joukko' 
$\mathcal{B} = \{\,\{t_i, \ i \in \Lambda\} \mid t_i \in \R\ \ \forall i \in \Lambda\,\}$.
\index{joukkoarvoinen (funktio)|av}} 
Kuviossa haaroja on kolme välillä $A=[a,b]$.\footnote[3]{Jotta monihaaraisen funktion 
$\mathcal{Y}=\{y_i\}$ haarat $y_i$ olisivat yksiselitteiset, on määrittelyssä estettävä 
'hyppiminen haaralta toiselle'. Ehdon voi tulkita geometris--intuitiivisesti niin, että
kuvaajien $G_i=\{P=(x,y)\in\Ekaksi \mid y=y_i(x)\ \ja\ x \in A\}$ on oltava \kor{yhtenäisiä}
tason pistejoukkoja, ts.\ kuvaajissa ei sallita 'katkoksia'. (Täsmällisemmin tarkoitetaan,
että funktioiden $y_i$ on oltava \kor{jatkuvia}, ks.\ Luku \ref{jatkuvuuden käsite}
jäljempänä.)}
\begin{figure}[H]
\begin{center}
\setlength{\unitlength}{1cm}
\begin{picture}(8,7.5)(-1,-2)
\put(-1,0){\vector(1,0){10}}\put(8.8,-0.4){$x$}
\put(0,-2){\vector(0,1){7.5}}\put(0.2,5.3){$y$}
\put(1,0){\drawline(0,-0.1)(0,0.1)} \put(6,0){\drawline(0,-0.1)(0,0.1)}
\put(0.9,-0.5){$a$} \put(5.9,-0.5){$b$}
\put(0,-0.4){\curve(
    1.0000,   -1.1137,
    1.2000,   -1.1013,
    1.4000,   -1.0586,
    1.6000,   -0.9883,
    1.8000,   -0.9004,
    2.0000,   -0.8108,
    2.2000,   -0.7383,
    2.4000,   -0.7012,
    2.6000,   -0.7129,
    2.8000,   -0.7791,
    3.0000,   -0.8952,
    3.2000,   -1.0466,
    3.4000,   -1.2100,
    3.6000,   -1.3572,
    3.8000,   -1.4598,
    4.0000,   -1.4947,
    4.2000,   -1.4487,
    4.4000,   -1.3217,
    4.6000,   -1.1282,
    4.8000,   -0.8954,
    5.0000,   -0.6600,
    5.2000,   -0.4619,
    5.4000,   -0.3379,
    5.6000,   -0.3146,
    5.8000,   -0.4034,
    6.0000,   -0.5976)}
\put(0,-0.747){\curve(
    1.0000,    3.0000,
    1.2000,    2.8198,
    1.4000,    2.6562,
    1.6000,    2.5080,
    1.8000,    2.3740,
    2.0000,    2.2531,
    2.2000,    2.1440,
    2.4000,    2.0456,
    2.6000,    1.9566,
    2.8000,    1.8760,
    3.0000,    1.8025,
    3.2000,    1.7349,
    3.4000,    1.6720,
    3.6000,    1.6127,
    3.8000,    1.5558,
    4.0000,    1.5000,
    4.2000,    1.4442,
    4.4000,    1.3873,
    4.6000,    1.3280,
    4.8000,    1.2651,
    5.0000,    1.1975,
    5.2000,    1.1240,
    5.4000,    1.0434,
    5.6000,    0.9544,
    5.8000,    0.8560,
    6.0000,    0.7469)}
\put(0,-1.3){\curve(
    1.0000,    4.0000,
    1.2000,    5.1424,
    1.4000,    5.8307,
    1.6000,    6.1511,
    1.8000,    6.1831,
    2.0000,    5.9988,
    2.2000,    5.6630,
    2.4000,    5.2337,
    2.6000,    4.7614,
    2.8000,    4.2895,
    3.0000,    3.8545,
    3.2000,    3.4854,
    3.4000,    3.2042,
    3.6000,    3.0256,
    3.8000,    2.9574,
    4.0000,    3.0000,
    4.2000,    3.1467,
    4.4000,    3.3836,
    4.6000,    3.6898,
    4.8000,    4.0369,
    5.0000,    4.3898,
    5.2000,    4.7058,
    5.4000,    4.9352,
    5.6000,    5.0213,
    5.8000,    4.8999,
    6.0000,    4.5000)}
\put(6,-1.4){$y=y_1(x)$}
\put(6,0.5){$y=y_2(x)$}
\put(6,3.7){$y=y_3(x)$}
\end{picture}
%\caption{Monihaarainen implisiittifunktio}
\end{center}
\end{figure}

\begin{Exa}
Yhtälö $\,x^2+y^2=1\,$ määrittelee välillä $[-1,1]$ kaksihaaraisen implisiittifunktion
$\mathcal{Y}=\{y_+,y_-\}$, missä $y_\pm = \pm \sqrt{1-x^2}$. \loppu
\end{Exa}
\begin{Exa} Millaisen monihaaraisen implisiittifunktion määrittelee yhtälö \newline
$y^4-2y^2+x^2-x=0\,$ eri $\R$:n osaväleillä?
\end{Exa}
\ratk Jos $x\in\R$, niin yhtälön mahdolliset ratkaisut $y\in\R$ ovat
\[
y=\pm\sqrt{1\pm\sqrt{1+x-x^2}}.
\]
Tässä on
\begin{align*}
1+x-x^2 &\ge 0 \qekv \frac{1}{2}(1-\sqrt{5}) \le x \le \frac{1}{2}(1+\sqrt{5}), \\
1+x-x^2 &\ge 1 \qekv 0 \le x \le 1.
\end{align*}

Päätellään, että välillä $[0,1]$ funktio on kaksihaarainen: $\mathcal{Y}=\{y_1\,,\,y_2\}$,
missä
\[
y_1(x)=\sqrt{1+\sqrt{1+x-x^2}}\,, \quad y_2(x)=-\sqrt{1+\sqrt{1+x-x^2}}\,.
\]
Väleillä $\,[\tfrac{1}{2}(1-\sqrt{5}),\,0]\,$ ja $\,[1,\,\tfrac{1}{2}(1+\sqrt{5})]\,$ on
$0 \le 1+x-x^2 \le 1$, joten näillä väleillä funktio on nelihaarainen: 
$\mathcal{Y}=\{y_1\,,\,y_2\,,\,y_3\,,\,y_4\}$, missä $y_1$ ja $y_2$ ovat samat kuin edellä ja
\[
y_3(x)=\sqrt{1-\sqrt{1+x-x^2}}\,, \quad  y_4(x)=-\sqrt{1-\sqrt{1+x-x^2}}\,.
\]
Jos $x\in(-\infty,\,\tfrac{1}{2}(1-\sqrt{5}))$ tai $x\in(\tfrac{1}{2}(1+\sqrt{5}),\,\infty)$,
ei yhtälö toteudu millään $y\in\R$, joten näillä väleillä yhtälö ei määrittele mitään
(reaali)funktiota. \loppu 

\subsection{Funktiot $\arcsin$, $\arccos$, $\arctan$}
\index{funktio C!e@$\arcsin$, $\arccos$, $\arctan$|vahv}

Funktiomerkinnöillä $\,\arcsin$, $\arccos$ ja $\arctan$ tarkoitetaan yhtälöiden $x-\sin y =0$,
$x-\cos y=0$ ja $x-\tan y=0$ määrittelemiä äärettömän monihaaraisia funktioita,
määrittelyjoukkona joko väli $[-1,1]$ ($\arcsin,\,\arccos$) tai $\R$ ($\arctan$). Esimerkiksi
$\arctan$  määritellään
\[
\arctan = \{y_n, \ n\in\Z\},
\]
missä
\[
y_n(x)= \Arctan x + n\pi.
\]
Tässä $y_0(x)=\Arctan x\,$ on $\,\arctan$:n
\index{pzyzy@päähaara (funktion)}%
\kor{päähaara} (engl.\ principal branch).
\begin{Exa} \label{napakulman kaava} Jos $\varphi(x,y)$ on karteesisen koordinaatiston
pistettä $(x,y)$ vastaava napakulma, $\varphi\in[0,2\pi)$, niin funktioriippuvuus 
$(x,y)\map\varphi(x,y)$ on
\[
\varphi(x,y)=
\begin{cases}
\Arctan(y/x),      &\text{jos}\ \ x>0\ \ja\ y\geq 0, \\
\Arctan(y/x)+\pi,  &\text{jos}\ \ x<0, \\
\Arctan(y/x)+2\pi, &\text{jos}\ \ x>0\ \ja\ y<0, \\
\pi/2,             &\text{jos}\ \ x=0\ \ja\ y>0, \\
3\pi/2,            &\text{jos}\ \ x=0\ \ja\ y<0.
\end{cases}
\]
Tässä tarvittiin siis peräti kolmea $\arctan$:n haaraa (!).
\loppu
\begin{figure}[H]
\begin{center}
\input{plots/arctanhaarat.tex}
\end{center}
\end{figure}
\end{Exa}

\Harj
\begin{enumerate}

\item \label{H-IV-2: todistus}
Todista: a) Jos $f$ on välillä $A\subset\DF_f$ aidosti kasvava tai vähenevä, niin samoin
on $f^{-1}$ väleillä $B \subset f(A)$.\, b) Jos $f$ on 1-1 ja pariton, niin samoin on
$\inv{f}$.

\item
a) Funktio $f(x)=x^2+x+2$ on kääntyvä, kun se rajoitetaan välille $(-\infty,a]$ tai välille
$[a,\infty)$. Määritä $a$ ja käänteisfunktio kummallakin välillä. \vspace{1mm}\newline
b) Näytä algebran keinoin, että funktio $f(x)=x^3+2x$ on koko määrittelyjoukossaan ($=\R$)
aidosti kasvava. Määritä $f^{-1}(-1)$, $f^{-1}(3)$, $f^{-1}(10)$, $f^{-1}(12)$ ja 
$f^{-1}(4\sqrt{2})$ tarkasti, jos mahdollista, muuten likimäärin yhden desimaalin
tarkkuudella. \vspace{1mm}\newline
c) Olkoon $f(x)=(1-x^n)^{1/n},\,\ \DF_f=[0,1],\ n\in\N$. Näytä, että $f=\inv{f}$.
\vspace{1mm}\newline
d) Olkoon $f(x)=(ax+b)/(cx+d)$. Millä vakioiden $a,b,c,d$ arvoilla on $f=\inv{f}$\,?

\item
Tarkastellaan funktioita $f: A \kohti R$. Määritä seuraavissa tapauksissa $B$ siten, että
$f: A \kohti B$ on surjektio. Onko $f: A \kohti B$ tällöin myös bijektio? \vspace{1mm}\newline
a) \ $f(x)=-x^5,\ A=(-4,-3]$ \newline
b) \ $f(x)=x^2,\ A=[\sqrt{2}-\sqrt{3}\,,\sqrt{2}+\sqrt{3}\,]$ \newline
c) \ $f(x)=-6x-x^2,\ A=(-5,-2)$ \newline
d) \ $f(x)=(x+5)/(x+2),\ A=[-1,1]$ \newline
e) \ $f(x)=3+2\sin x,\ A=[3\pi/2,11\pi/6]$ \newline
f) \ $f(x)=x+\Arcsin x,\ A=[-1,1]$

\item
Laske tarkka arvo lausekkeelle tai ratkaise yhtälö:
\begin{align*}
&\text{a)}\ \sin\left(\Arccos\left(-\frac{3}{5}\right)\right) \qquad\quad\ \
 \text{b)}\ \cos\left(\Arcsin\left(-\frac{3}{5}\right)\right) \\
&\text{c)}\ \sin\left(\Arctan\left(-\sqrt{3}\right)\right) \qquad\quad\,\
 \text{d)}\ \sin\left(\Arccot\left(-\sqrt{3}\right)\right) \\
&\text{e)}\ \sin\left(\Arctan 2-\Arctan 3\right) \quad\ \ \
 \text{f)}\ \cos\left(\Arccot\frac{2}{5}+\Arctan\frac{3}{7}\right) \\
&\text{g)} \,\ \Arctan x=\frac{\pi}{4}-\Arctan 3 \quad\,\ \
 \text{h)} \,\ \Arccos x=\Arctan 2+\Arccos\frac{3}{4}
\end{align*}

\item
Saata seuraavat funktiot muotoon $R\sin(x+\alpha)$:
\begin{align*}
&\text{a)}\ \ f(x)=\sin x+\sin\left(x-\Arctan\frac{4}{3}\right) \\
&\text{b)}\ \ f(x)=\sqrt{5}\sin\left(x+\Arctan\frac{1}{2}\right)
                +2\sqrt{2}\sin\left(x+\frac{3\pi}{4}\right)+\sqrt{3}\sin(x-\pi)
\end{align*}

\item
Määritä $R$ ja $\alpha$ siten, että
\[
2\sin\left(x+\Arctan 2\sqrt{2}\right)+R\sin(x+\alpha)=6\sin x, \quad x\in\R.
\]

\item
Sievennä lauseke ja piirrä kuvaaja:
\begin{align*}
&\text{a)}\ \ y=\Arccos(\sin x) \qquad\qquad\ \
 \text{b)}\ \ y=\Arctan(\tan x) \\
&\text{c)}\ \ y=\Arcsin\left(2\cos^2 x-1\right) \quad\ 
 \text{d)}\ \ y=\Arctan\frac{2\tan x}{1-\tan^2 x}
\end{align*}

\item
Näytä, että Esimerkin \ref{muuan implisiittifunktio} implisiittifunktio on esitettävissä
muodossa $y(x)=u(x)-x$, missä $u$ on jaksollinen. Mikä on perusjakso? Onko $u(x)$
parillinen/pariton? Hahmottele likimäärin funktion $y(x)$ kuvaaja. 

\item
Seuraavassa on määritelty implisiittifunktioita, mahdollisesti useampihaaraisia. Määritä näiden
haarat funktioina $x \map y(x)$ eri $\R$:n osaväleillä. Hahmottele funktiot myös graafisesti!
\begin{align*}
&\text{a)}\ \ x^2+y^2+6x=0 \qquad\ \ \
 \text{b)}\ \ x^2-y^2-4x=0 \\[1mm]
&\text{c)}\ \ xy^4+3y^2-x=0 \qquad\ \
 \text{d)}\ \ \abs{x}^{2/3}+\abs{y}^{2/3}=1 \\[2mm]
&\text{e)}\ \ x^2-4y^2+y^4=0 \qquad\quad
 \text{f)}\ \ xy^4+3x^3+6y^2-12x=0 \\
&\text{g)}\ \ x^2+\frac{\sin y}{3+\sin y}=0 \qquad\,\
 \text{h)}\ \ \arctan x + \arctan y=\frac{\pi}{4}
\end{align*}

\item
Mitkä ovat monihaaraisen funktion $\arccos(\cos x)$ arvot, kun $x=13\pi/4$\,?

\item
a) Määrittele monihaaraisen funktion $\arcsin=\{y_n\,,\ n\in\Z\}$ haarat päähaaran $y_0$
avulla. \ b) Ilmaise karteesisen koordinaatiston pisteiden ja napakulman välinen riippuvuus
$\varphi(x,y)$ funktioiden $y_n$ avulla.

\item(*) \label{H-IV-2: ratkeavuus kymmenjaolla}
Olkoon $f(x)=x^3+3x$ ja $y\in(0,\infty)$. Todista yhtälön $f(x)=y$ ratkeavuus käyttäen
kymmenjakoalgoritmia (ks.\ Luku \ref{reaaliluvut}). \kor{Vihje}: Lähde tuloksista:
$f$ on aidosti kasvava välillä $[0,\infty)$ ja $\lim_n f(n)=\infty$.

\item (*)
Tutki, millä arvoilla $x\in\R$ tai $(x,y)\in\Rkaksi$ pätee
\begin{align*}
&\text{a)}\ \ 2\Arcsin x=\Arccos(1-2x^2) \\
&\text{b)}\ \ \Arctan x+\Arctan y=\Arctan\frac{x+y}{1-xy} \hspace{4cm}
\end{align*}

\item(*)
Tutki, mitkä seuraavista funktioista tyyppiä $\,f: \N\kohti\Q$ ovat injektioita: \newline
a) \ $f(n)=n^22^{-n}\,\ $ b)\ $f(n)=n^{333}2^{-n}\,\ $ c)\ $f(n)=n^{243}3^{-n}\,\ $ 
d) \ $f(n)=n^{80}3^{-n}$

\end{enumerate} % Käänteisfunktio
\section{Kahden ja kolmen muuttujan funktiot} 
\label{kahden ja kolmen muuttujan funktiot}
\alku
\index{funktio A!c@kahden ja kolmen muuttujan|vahv}
\index{kahden muuttujan funktio|vahv}

\kor{Kahden reaalimuuttujan} funktio on tyyppiä
\[ 
f:\Rkaksi \kohti \R \quad \text{tai} \quad f:A \kohti \R, \quad A\subset\Rkaksi. 
\]
\index{kuvaaja}%
Tällaista funktion \kor{kuvaaja} on Euklidisen avaruuden $\Ekolme$ pistejoukko
\[
G_{f,A}=\{P \vastaa (x,y,z) \in \Rkolme \mid z=f(x,y), \ (x,y)\in A\} \subset \Ekolme,
\]
jonka yleisnimi on \kor{pinta}. Funktiota voidaan 'katsella' esimerkiksi kehittämällä
(yleensä tietokoneen avulla) $G_{f,A}$:n perspektiivikuva. Toinen tapa tehdä kahden muuttujan
funktio geometrisesti havainnolliseksi on piirtää funktion
\index{tasa-arvokäyrä}%
\kor{tasa-arvokäyriä} euklidiseen tasoon. Tasa-arvokäyrä on tason pistejoukko
\[
S_{f,c}=\{P \vastaa (x,y)\in \Rkaksi \mid f(x,y)=c\,\} \subset \Ekaksi \quad (c\in\R).
\]
Kartalla tasa-arvokäyriä ovat \pain{korkeusviivat} ($f=$ maaston 
korkeusfunktio).\footnote[2]{Geometrian käsite \kor{käyrä} on intuitiivisesti ymmärrettynä
'yhtenäinen viiva vailla leveyttä'. Käsitettä on yllättävän vaikea määritellä
täsmällisesti ja samalla yleispätevästi, eikä määrittely edes onnistu pelkin algebran ja 
geometrian keinoin. Sovellustilanteissa kohdataan tyypillisisesti vain käyrien
erikoistapauksia, kuten suoria, toisen asteen käyriä, jne., jolloin käsitteellisiä ongelmia ei
yleensä ole. Tässä ja seuraavissa luvuissa käytetään termiä 'käyrä' intuitiivisesti,
ajatellen lähinnä ongelmattomia erikoistapauksia. --- Termi 'tasa-arvokäyrä' sen sijaan on
sovelluksiakin ajatellen tahallisen huolimaton, sillä kyse voi hyvin olla pistejoukosta, jossa
on 'läiskiä' tai erillisiä pisteitä, tai joukko voi olla tyhjä. Itse asiassa jos $A$ on mikä
tahansa $\Rkaksi$:n osajoukko, niin on helppo määritellä funktio $f$ siten, että $S_{f,c}=A$
annetulla $c$. \index{kzyyrzy@käyrä|av}}
\begin{Exa} \label{saari}
Meren pohjasta kohoavaa saarta kuvaa funktio
\[
h(x,y) = -\frac{1}{10}\,(x^2+2y^2+2xy+4x+14y),
\]
missä $h(x,y)=$ maaston korkeus merenpinnan tasosta (vedenalaisissa osissa $h(x,y)<0$). 
Pituusyksikkö = 100 m. Hahmottele saaren sijainti ja korkeuskäyriä. Missä ja kuinka korkealla 
on saaren korkein kohta?
\end{Exa}
\ratk
Merkitään $\,f(x,y) = -10\,h(x,y)=x^2+2y^2+2xy+4x+14y\,$ ja yritetään ensin kirjoittaa $f$
hieman selkeämpään muotoon
\[
f(x,y)=(x-x_0)^2+2(y-y_0)^2+2(x-x_0)(y-y_0)+c,
\]
eli
\[
f(x,y)=g(\xi,\eta)=\xi^2+2\eta^2+2\xi\eta+c, \quad \begin{cases} \xi=x-x_0,\\ \eta=y-y_0, 
                                                   \end{cases}
\]
missä $x_0,y_0,c\in\R$ on määrättävä. Muunnos $(x,y) \map (\xi,\eta)$ vastaa siis 
koordinaatiston origon siirtoa (toistaiseksi tuntemattomaan) pisteeseen $(x_0,y_0)$.
Vertaamalla $f$:n lausekkeita nähdään, että $x_0,y_0,c$ on valittava siten, että pätee
\begin{align*}
f(x,y) &= x^2+2y^2+2xy-(2x_0+2y_0)x-(2x_0+4y_0)y+(x_0^2+2y_0^2+2x_0y_0+c) \\
       &= x^2+2y^2+2xy+4x+14y \quad \forall (x,y) \in \R^2,
\end{align*}
eli on oltava
\[ \begin{cases} 2x_0+2y_0=-4 \\ 2x_0+4y_0=-14 \\ x_0^2+2y_0^2+2x_0y_0+c=0 \end{cases}
\qekv \begin{cases} x_0=3\\y_0=-5\\c=-29 \end{cases}
\]
Siis $\,\xi=x-3,\ \eta=y+5$, ja
\begin{align*}
f(x,y) = g(\xi,\eta) &= \xi^2+2\eta^2+2\xi\eta-29 \\
                     &=(\xi+\eta)^2+\eta^2-29.
\end{align*}
Viimeksi kirjoitetusta lausekkeesta nähdään, että $g$:n minimikohta $=h$:n maksimikohta on 
pisteessä
\[
(\xi,\eta)=0 \ \ekv \ (x,y)=(3,-5),
\]
ja saaren korkein kohta on 2.9 km meren pinnasta. Saaren rantaviiva on pistejoukko
\[ 
S = \{P \vastaa (x,y) \in \R^2 \mid f(x,y) = 0\}. 
\]
Hahmottelemalla muitakin korkeusviivoja saadaan saaresta yleiskuva. \loppu
\begin{figure}[H]
\begin{center}
\epsfig{file=kuvat/saari.eps}
\end{center}
\end{figure}
\begin{Exa} Määritä funktion $f(x,y)=x+2y+2$ arvojoukko yksikköneliössä
\[
A=\{(x,y)\in\R^2 \; | \; 0\leq x\leq 1,\; 0\leq y\leq 1\}.
\]
\end{Exa}
\ratk
Tasa-arvokäyrät $S: f(x,y)=c\ $ ovat suoria
%\begin{multicols}{2} \raggedcolumns
\[
x+2y+2=c,
\]
\begin{multicols}{2} \raggedcolumns
joten arvojoukko on (kuva!)
\[
f(A)=[2,5]. \loppu
\]
\begin{figure}[H]
\begin{center}
\import{kuvat/}{kuvaV-1.pstex_t}
\end{center}
\end{figure}
\end{multicols}

\subsection{Kolmen muuttujan funktiot}
\index{kolmen muuttujan funktio|vahv}

Kolmen reaalimuuttujan funktio on tyyppiä
\[
f:\R^3\kohti\R \quad \text{tai} \quad f:A\kohti\R, \quad A\subset\R^3. 
\]
Kolmen muuttujan funktioita voi havainnollistaa esimerkiksi \pain{vii}p\pain{aloimalla}: 
Valitaan äärellinen joukko muuttujan $z$ arvoja $z_i$ ja tutkitaan kahden muuttujan funktioita
\[
g_i(x,y)=f(x,y,z_i).
\]
\index{zza@\sov!Tomografia}%
\begin{Exa}: \vahv{Tomografia.} \label{tomografia} \ Lääketieteessä paljon käytetyllä 
(tietokone)tomo\-grafialla määritetään nk.\ varjostumafunktio $f$, jonka määrittelyjoukko on
ihmisruumis tai sen osa. Varjostumafunktion arvot ovat reaalilukuja, jotka kertovat
varjostuman tummuusasteen. Funktiosta $f$ saadaan mittausten avulla likimäärin selville
viipaloitujen funktioiden $g_i(x,y)=f(x,y,z_i)$ arvot valituilla (äärellisen monella) $z_i$:n
arvoilla. Funktiot $g_i$ määrätään yhdistämällä suuri joukko erisuuntaisia röntgen(varjo)kuvia 
laskennallisin keinoin (tietokoneen avulla).

Olkoon tarkastelun kohteena (idealisoitu) ihmisen pää
\[
A=\{P \vastaa (x,y,z)\in \R^3 \mid x^2+y^2+z^2\leq 7^2\}
\]
(pituusyksikkö = cm). Tomografiakuvaus tuottakoon varjostumafunktion
\[
f(x,y,z)=1+0.02(2x-3y+6z).
\]
Missä on kirkkainta ja missä tumminta?
\end{Exa}
\ratk Funktio $f$ on saatu käytännössä yhdistelemällä joukko viipalefunktioita
$g_i(x,y)=f(x,y,z_i)$, missä $-7<z_i<7$. Sikäli kuin mittaukset ja niistä lasketut funktiot
$g_i$ ja $f$ katsotaan tarkoiksi, on siis oltava
\[
g_i(x,y)=0.02(2x-3y)+c_i, \quad (x,y) \in A_i\,,
\]
missä $c_i=0.12z_i+1$, ja $A_i$ on kiekon muotoinen leikkauskuvio
\[
A_i=
\{(x,y) \in \R^2 \mid x^2+y^2\leq r_i^2\}, \quad r_i=\sqrt{7^2-z_i^2}.
\]
Funktion $g_i$ tasa-arvokäyrät ovat suoria
\[
2x-3y=\text{vakio},
\]
joten $g_i$:n maksimi ja minimi löytyvät näiden suorien normaalin $3x+2y=0$ ja $A_i$:n 
reunaviivan leikkauspisteistä.
\begin{figure}[H]
\begin{center}
\setlength{\unitlength}{1cm}
\begin{picture}(8,8)(-4,-4)
\put(0,0){\vector(1,0){4}} \put(3.8,-0.4){$x$}
\put(0,0){\vector(0,1){4}} \put(0.2,3.8){$y$}
\put(0,0){\bigcircle{6}}
\multiput(-1,1.5)(1,-1.5){3}{\drawline(-3.5,-2.33)(3.5,2.33)}
\drawline(-2.33,3.5)(2.33,-3.5)
\put(-1.75,2.4){$\bullet$} \put(1.58,-2.6){$\bullet$ $g_i=$max!}
\put(-1.75,3.1){$g_i=$min!}
\put(3,3){$g_i(x,y)=$vakio} \put(2,-4){$3x+2y=0$}
\end{picture}
\end{center}
\end{figure}
Eri viipalekuvia tutkimalla löydetään likimain myös $f$:n maksimi- ja minimiarvot. Tässä
\index{tasa-arvopinta}%
$f$ kuitenkin tunnetaan tarkasti, joten voidaan suoraan tutkia $f$:n \kor{tasa-arvopintoja}.
Nämä ovat tasoja
\[
2x-3y+6z=\text{vakio},
\]
joten voidaan geometrisesti päätellä, että $f$:n maksimi ja minimi löytyvät origon kautta
kulkevan tasa-arvopintojen yhteisen normaalin
\[
\begin{cases} x=2t, \\ y=-3t, \\ z=6t \end{cases}
\]
ja kappaleen reunapinnan
\[
x^2+y^2+z^2=7^2
\]
leikkauspisteistä. Nämä pisteet vastaavat $t$:n arvoja
\[ 
(2t)^2+(-3t)^2+(6t)^2=7^2 \qekv t= \pm 1. 
\]
Varjostuman maksimi- ja minimiarvoiksi päätellään näin muodoin
\begin{alignat*}{3}
&f(2,-3,6)  &\ = 1.98 &= f_{\text{max}}\,, \\
&f(-2,3,-6) &\ = 0.02 &= f_{\text{min}}\,. \qquad\loppu
\end{alignat*}

\subsection{Funktiot käyräviivaisissa koordinaatistoissa} 
\index{funktio B!h@käyräv.\ koordinaatistoissa|vahv}
\index{muuttujan vaihto (sijoitus)!a@käyräv.\ koordinaatteihin|vahv}
\index{kzyyrzy@käyräviivaiset koordinaatistot!a@--funktiot|vahv}

Kahden ja kolmen reaalimuuttujan funktioita tutkittaessa voi olla apua siirtymisestä 
käyräviivaiseen napa-, lieriö- tai pallokoordinaatistoon silloin kun funktion 
määrittelyjoukon geometria on sellaiseen muunnokseen sopiva. Esimerkiksi, jos kahden
reaalimuuttujan funktio on määritelty pyörähdyssymmetrisessä joukossa $A\subset\Rkaksi$
(voi olla myös $A=\Rkaksi$), voi napakoordinaatteihin siirtyminen auttaa. Siirtyminen tapahtuu
muunnoksella (vrt. Luku \ref{koordinaatistot})
\begin{multicols}{2} \raggedcolumns
\begin{align*}
f(x,y)&=f(r\cos\varphi,r\sin\varphi) \\[2mm]
      &=g(r,\varphi).
\end{align*}
\begin{figure}[H]
\setlength{\unitlength}{1cm}
\begin{center}
\begin{picture}(4,3)(-1,0)
\put(-1,0){\vector(1,0){4}} \put(2.8,-0.4){$x$}
\put(0,0){\vector(0,1){3}} \put(0.2,2.8){$y$}
\put(0,0){\vector(2,3){1.5}}
\arc{1}{5.25}{6.2}
\put(0.5,1.1){$r$} \put(0.5,0.25){$\varphi$} \put(1.45,2.15){$\bullet$ $(x,y)$}
\end{picture}
\end{center}
\end{figure}
\end{multicols}
Huomattakoon, että muunnettu funktio $g$ on itse asiassa yhdistetty funktio
\[
g(r,\varphi)=f(x(r,\varphi),y(r,\varphi)),
\]
missä $x(r,\varphi)=r\cos\varphi,\ y(r,\varphi)=r\sin\varphi$. Siirtyminen 
polaarikoordinaatistosta karteesiseen tapahtuu käänteismuunnoksella
\[
f(x,y)=g(r(x,y),\varphi(x,y)),
\]
missä $\,r(x,y)=\sqrt{x^2+y^2}\,$ ja $\,\varphi(x,y)\,$ on pistettä $(x,y)$ vastaava 
napakulma\footnote[2]{Merkinnöissä $x(r,\varphi)$, $y(r,\varphi)$, $r(x,y)$ ja $\varphi(x,y)$
on alunperin riippumattomista muuttujista $x,y$ tai $r,\varphi$ tehty funktiosymboleja.
Koordinaattimuunnoksissa (myös implisiittifunktioissa, vrt.\ edellinen luku) tällaiset
merkinnät ovat tavallisia, koska ne selkeyttävät laskemista.}
($\varphi$:n laskukaava on hieman konstikas, ks.\ edellisen luvun Esimerkki 
\ref{napakulman kaava}).

Kolmen muuttujan funktioita tarkasteltaessa voidaan siirtyä lieriö- tai 
pallokoordinaatistoon muunnoksilla
\begin{align*}
\text{Lieriö:} \quad  f(x,y,z) &= f(r\cos\varphi,r\sin\varphi,z) =g(r,\varphi,z). \\
\text{Pallo:} \quad\  f(x,y,z) &= f(r\sin\theta\cos\varphi,r\sin\theta\sin\varphi,r\cos\theta)
                                = g(r,\theta,\varphi).
\end{align*}
Lieriökoordinaatiston tapauksessa muunnos $(x,y)\ext(r,\varphi)$, samoin käänteismuunnos
$(r,\varphi)\ext(x,y)$ on sama kuin napakoordinaatistossa. Pallokoordinaatistossa 
käänteismuunnos on
\begin{align*}
g(r,\theta,\varphi) &= g(r(x,y,z),\theta(x,y,z),\varphi(x,y)) \\
&=f(x,y,z),
\end{align*}
missä
\begin{align*}
     r(x,y,z) &= \sqrt{x^2+y^2+z^2}, \\
\theta(x,y,z) &= \Arccos\left(\frac{z}{\sqrt{x^2+y^2+z^2}}\right),
\end{align*}
ja $\varphi(x,y)$ on sama kuin napakoordinaatistossa. Suuntakulma $\theta$ ei ole määritelty
origossa eikä kulma $\varphi$ $z$-akselilla.
\begin{Exa}
Vuoristoisen maaston korkeus merenpinnasta on origon ympäristössä
\[
f(x,y)=x^2-2xy-3y^2+0.2x-0.1y+10
\]
(yksikkö = 100 m). Origossa on kahden tien risteys. Tiet ovat kartalla suoria ja myös niiden
sivuprofiili on suora. Mihin suuntiin tiet kulkevat ja mikä on teiden kaltevuus?
\end{Exa}
\ratk
Siirrytään napakoordinaatistoon:
\[
f(x,y)=h(r,\varphi)=r^2(\cos^2 \varphi-2\cos\varphi\sin\varphi-3\sin^2\varphi) 
                                      + r(0.2\cos\varphi-0.1\sin\varphi) +10.
\]
Tiet noudattavat origosta lähdettäessä puolisuoria $\varphi=\varphi_1$ ja $\varphi=\varphi_2$
(koska ovat kartalla suoria). Koska teiden sivuprofiilikin on suora, on oltava
\[
\cos^2 \varphi-2\cos\varphi\sin\varphi-3\sin^2\varphi=0.
\]
Tässä ei $\,\cos\varphi=0\,$ ole ratkaisu, joten voidaan jakaa $\cos^2\varphi$:lla:
\[
1-2\tan\varphi-3\tan^2\varphi=0 
           \qekv \begin{cases}
                  \,\tan\varphi_1=1/3 \ & \impl \ \ \varphi_1 \approx 18\aste\ 
                                                              \text{tai}\ 198\aste, \\
                  \,\tan\varphi_2=-1 \  & \impl \ \ \varphi_2 =135\aste\
                                                              \text{tai}\ 315\aste.
                  \end{cases}
\]
Kaltevuuskulmat suuntiin $\varphi_1 \approx 18\aste$ ja $\varphi_2=135\aste$ saadaan
ratkaisemalla
\[
\tan\alpha_i=0.2\cos\varphi_i-0.1\sin\varphi_i
                  \qimpl \begin{cases}
                         \,\alpha_1\approx 9\aste, \\
                         \,\alpha_2\approx -12\aste.
                         \end{cases} \qquad\loppu
\]

\begin{Exa}
Etsi (jos mahdollista) pisteet, joissa funktio
\[
f(x,y,z)=xyz(3x^2+3y^2-z^2)/(x^2+y^2)
\]
saavuttaa maksimi- tai minimiarvonsa joukossa
\[
A=\{(x,y,z)\in\R^3 \ | \ x^2+y^2+z^2\leq 96, \ (x,y)\neq (0,0)\}.
\]
\end{Exa}
\ratk Siirrytään pallokoordinaatistoon. Koska pallokoordinaateissa on
\[
x^2+y^2=r^2\sin^2\theta(\cos^2\varphi + \sin^2\varphi)=r^2\sin^2\theta,
\]
niin (vrt.\ Luku \ref{trigonometriset funktiot})
\begin{align*}
f(x,y,z)\,=\,g(r,\theta,\varphi) 
         &=\,r^3\cos\theta\,(3\sin^2\theta-\cos^2\theta)\cos\varphi\sin\varphi \\[2mm]
         &=\,r^3\,(3\cos\theta-4\cos^3\theta)\cos\varphi\sin\varphi \\
         &=-\frac{1}{2}\,r^3\cos 3\theta\sin 2\varphi.
\end{align*}
Funktio $g$ saavuttaa maksimiarvonsa $192\sqrt{6}$ pallon pinnalla $(r=\sqrt{96}=4\sqrt{6})$, 
kun joko $\cos 3\theta=-1$, $\sin 2\varphi=1$ tai $\cos 3\theta=1$, $\sin 2\varphi=-1$. 
Maksimikohtia on neljä:
\[
\begin{array}{llllrr}
(r, & \theta, & \varphi) & (x, & y, & z) \\ \\
(4\sqrt{6}, & \pi/3, & \pi/4) \quad & (\ \: \, 6, &6, &2\sqrt{6}) \\ 
(4\sqrt{6}, & \pi/3, & 5\pi/4) \quad & (-6, &-6, &2\sqrt{6}) \\ 
(4\sqrt{6}, & 2\pi/3, & 3\pi/4) \quad & (-6, & 6, &-2\sqrt{6}) \\ 
(4\sqrt{6}, & 2\pi/3, & 7\pi/4) \quad & (\ \: \, 6, &-6, &-2\sqrt{6})
\end{array}
\]
Minimikohtia (joissa $g=g_{\text{min}}=-192\sqrt{6}$) on samoin neljä. Nämä saadaan vaihtamalla
yhden koordinaatin merkki maksimipisteiden karteesisessa esitysmuodossa. Funktio $g$ saavuttaa
maksimi- tai minimiarvonsa myös pallonpintakoordinaattien arvoilla 
$(\theta,\varphi) \in  \{0,\pi\} \times \{\pi/4,\ 3\pi/4,\ 5\pi/4,\ 7\pi/4\}$. Nämä vastaavat 
kuitenkin karteesisen koordinaatiston pisteitä $(0,0,z)$, jotka ovat alkuperäisen funktion $f$ 
määrittelyjoukon ulkopuolella. \loppu

\subsection{Usean muuttujan funktioiden yhdistely}
\index{funktio B!e@yhdistetty|vahv}
\index{yhdistetty funktio|vahv}

Useamman reaalimuuttujan reaaliarvoisia funktioita voi yhdistellä yhdistetyiksi funktioiksi 
samoilla periaatteilla kuin yhden muuttujan tapauksessa. Lisäehtona on kuitenkin, että funktiot
ovat tyypiltään yhteensopivia. Esimerkiksi funktioiden $f(x)$ ja $g(x,y)$ yhdistetty funktio 
$F = f \circ g$  voidaan määritellä: 
\[ 
F(x,y) = (f \circ g)(x,y) = f(g(x,y)), \quad \DF_F = \{(x,y) \in \DF_g \mid g(x,y) \in \DF_f\}.
\]
Voidaan myös määritellä yhdistetyt funktiot
\begin{align*}
&G_1(x) = g(x,f(x)),   \quad \DF_{G_1} = \{x \in \DF_f \mid (x,f(x)) \in \DF_g\}, \\
&G_2(y) = g(f(y),y), \,\quad \DF_{G_2} = \{y \in \DF_f \mid (f(y),y)\,\in \DF_g\}, \\
&G_3(x,y) = g(f(x),f(y)),\quad \DF_{G_3} 
                               = \{(x,y) \in \DF_f \times \DF_f \mid (f(x),f(y)) \in \DF_g\}.
\end{align*}
Funktiota $g \circ f$ sen sijaan ei voi määritellä, ei myöskään funktiota $g \circ g$.
\begin{Exa} Jos $f(x) = \sqrt{x},\ \DF_f = [0,\infty)$ ja $g(x,y) = x-y^2,\ \DF_g = \R^2$, niin
\begin{align*}
&F(x,y) = (f \circ g)(x,y) = \sqrt{x-y^2}, \quad \DF_F = \{(x,y) \in \R^2 \mid x \ge y^2\}, \\
&G(x) = g(x,f(x)) = 0, \quad \DF_G = [0,\infty). \loppu
\end{align*} \end{Exa}
\begin{Exa} Määritä funktion $f(x,y) = x^2 + 2y^2 + 2xy + 4x + 14y$ minimiarvo suoralla 
$S:\ y = 2x-5$. (Vrt.\ Esimerkki \ref{saari}.) \end{Exa}
\ratk Kun käytetään suoralla $S$ vallitsevalle funktioriippuvuudelle $x \map y$ 
merkintää $y(x)$, niin kyse on yhdistetystä funktiosta $F(x) = f(x,y(x))$, $\DF_F = \R$.
Koska
\begin{align*}
 F(x) = f(x,2x-5) &= x^2 + 2(2x-5)^2 + 2x(2x-5) + 4x + 14(2x-5) \\[3mm]
                  &= 13x^2 - 18x - 20 \\
                  &= 13\left(x-\dfrac{9}{13}\right)^2 - \dfrac{341}{13}\,,
\end{align*}
niin nähdään, että kysytty minimiarvo on
\[ 
F_{\text{min}} = -\dfrac{341}{13} \approx -26.2. 
\]
Minimikohta on suoran pisteessä $(9/13,-47/13) \approx (0.69,-3.62)$. \loppu

Useamman muuttujan funktioita voidaan yhdistellä myös laskutoimituksin samalla tavoin kuin
yhden muuttujan funktioita (Määritelmä \ref{funktioiden yhdistelysäännöt}).  Ellei
yhdisteltävien funktioiden muuttujien lukumäärä täsmää, on ajateltava, että funktioissa on 
'näkymättömiä' muuttujia. 
\begin{Exa} Jos $f(x) = \tan x\,$ ja $\,g(x,y) = \sin(x+y^2)$, niin kirjoitettaessa
\[ 
F(x,y) = \tan x + \sin (x+y^2) 
\]
tarkoitetaan tämän laskusäännön ilmaisemaa funktiota $F$, määrittelyjoukkona 
\[ 
\DF_F = \{(x,y) \in \R^2 \mid x \neq (n+1/2)\,\pi\,\ \forall n \in \Z\}.
\]
Samaan tulokseen tullaan myös funktioiden yhdistelyn kautta ajattelemalla, että
$F = \tilde{f} + g,\ \DF_F = \DF_{\tilde{f}} \cap \DF_g$, missä $\tilde{f}$ on saatu $f$:stä
lisäämällä toinen muuttuja:
\[ 
\tilde{f}(x,y) = \tan x, \quad \DF_{\tilde{f}} = \DF_f \times \R. \loppu
\] 
\end{Exa}

\subsection{Kahden muuttujan implisiittifunktiot}
\index{funktio B!j@implisiittifunktio|vahv}
\index{implisiittifunktio|vahv}

Jos $F$ on kolmen muuttujan funktio ja yhtälö
\[
F(x,y,z)=0
\]
ratkeaa $z$:n suhteen, kun $(x,y)\in A\subset\R^2$, niin yhtälö määrittelee $A$:ssa
kahden muuttujan (mahdollisesti monihaaraisen) implisiittifunktion. Jos tälle käytetään
funktiomerkintää $z(x,y)$, niin määritelmä on siis
\[
F(x,y,z(x,y))=0, \quad (x,y)\in A.
\]
\begin{Exa}
Yhtälö $x^2+y^2+z^2=R^2$ määrittelee kaksihaaraisen implisiittifunktion
$z(x,y)=\pm\sqrt{R^2-x^2-y^2}\,$ kiekossa $A:\ x^2+y^2 \le R^2$. \loppu
\end{Exa}
\begin{Exa} Jos $m\in\N,\ m \ge 2$, niin yhtälö
\[
y,z\in\C: \quad y^m=z
\]
määrittelee $m$-haaraisen kompleksifunktion $y=\sqrt[m]{z}$, vrt.\ Luku \ref{III-3}.
Jos kirjoitetaan $z=x+iy$ ja $y=u(x,y)+iv(x,y)$, niin $u$ ja $v$ ovat $m$-haaraisia
funktioita tyyppiä $u,v:\ \Rkaksi\kohti\R$. \loppu
\end{Exa}

\Harj
\begin{enumerate}

\item
Määritä seuraavien funktioiden arvojoukot annetulla janalla $AB$\,: \newline
a) \ $f(x,y)=x^2+2xy-y^2,\,\ A=(1,2),\ B=(-1,3)$ \newline
b) \ $f(x,y,z)=x+xy+yz+z^2,\,\ A=(1,1,1),\ B=(-1,2,-3)$

\item
Määritä seuraavien funktioiden arvojoukot annetussa joukossa $A$: \newline
a) \ $f(x,y)=x+3y-1$, \ $A=$ kolmio, jonka kärjet $(0,1)$, $(4,0)$ ja $(3,4)$ \newline
b) \ $f(x,y,z)=x+2y-3z$, \ $A=\{(x,y,z) \mid (x-1)^2+(y+1)^2+(z-2)^2 \le 9\}$ \newline
c) \ $f(x,y,z)=x+xy+yz+z^2$, \ $A=$ suora $\,2x-2=y+1=4-2z$

\item
Tasangolla $z=0$ on järvi $A=\{(x,y)\in\Rkaksi \mid h(x,y)<0\}$, missä
\[
h(x,y) = 4x^2+y^2+24x+8y
\]
on järven pohjan korkeusprofiili. Tässä $h$ on ilmaistu metreinä ja $x,y$ kilometreinä.
Järven poikki kulkee moottoritie suoraa $y=x+2$ pitkin. Missä on järven syvin kohta ja mikä
on syvyys tässä kohdassa? Hahmottele järven rantaviiva ja laske, missä pisteissä moottoritie
leikkaa rantaviivan.

\item
Muunna seuraavat funktiot napa- tai pallokoordinaatistoon:
\begin{align*}
&\text{a)}\ \ f(x,y)=x+2y \qquad \text{b)}\ \ f(x,y)=xy^2 \qquad 
 \text{c)}\ \ f(x,y)=\frac{xy}{x^2+y^2} \\
&\text{d)}\ \ f(x,y)=\max\{0,x,y\} \qquad
 \text{e)}\ \ f(x,y)=\begin{cases} x^2/y, &\text{kun}\ x > 0\ \text{ja}\ y>0 \\
                                   0,     &\text{muulloin}
                     \end{cases} \\
&\text{f)}\ \ f(x,y)=\begin{cases} xy^2-y^3, &\text{kun}\ 0 \le x \le y \\
                                   0,        &\text{muulloin}
                     \end{cases} \qquad
 \text{g)}\ \ f(x,y,z)=\frac{xyz}{x^2+y^2+z^2} \\
&\text{h)}\ \ f(x,y,z)=\frac{xy^2z^3}{x^2+y^2} \qquad 
 \text{i)}\ \ f(x,y,z)=\begin{cases}
                      \,z^2-xyz, &\text{kun}\ z>0 \\ \,0, &\text{kun}\ z \le 0
                      \end{cases}
\end{align*}

\item
Muunna seuraavat napakoordinaateissa ilmaistut funktiot karteesiseen koordinaatistoon:
\begin{align*}
&\text{a)}\ \ g(r,\varphi)=r^3(\cos\varphi-\sin\varphi) \qquad
 \text{b)}\ \ g(r,\varphi)=\sin 2\varphi-r^2\cos 2\varphi \\
&\text{c)}\ \ g(r,\varphi)=r^2(\tan\varphi+\cot\varphi) \qquad
 \text{c)}\ \ g(r,\varphi)=\frac{\sin\varphi}{2+\cos\varphi}
\end{align*}

\item
Olkoon $f(x,y)=x+y$ ja $g(x,y)=xy,\ (x,y)\in\Rkaksi$. Määrittele sievennettyinä lausekkeina 
(laskusääntöinä) seuraavat yhdistetyt funktiot: \newline
a) \ $f(x,g(x,y))\quad$ b) \ $f(g(x,y),y)\quad$ c) \ $f(g(x,y),g(x,y))$ \newline
d) \ $g(x,f(x,y))\quad$ e) \ $g(f(x,y),y)\quad$ f) \ $g(f(x,y),f(x,y))$ \newline
g) \ $f(f(x,y),f(x,y))\quad$ h) \ $g(g(x,y),g(x,y))$

\item
Olkoon $f(x)=\sqrt{2-x}$ ja $g(x,y)=\sqrt{x-y^2}\ (x,y\in\R)$. Määrittele yhdistetty funktio
$F=f \circ g$ (laskusääntö ja määrittelyjoukko).

\item
Olkoon $f(x)=\sqrt{25-x^2}$ ja $g(x,y)=3x+4y-1\ (x,y\in\R)$. Määritä yhdistetyn funktion
$F=f \circ g$ pienin ja suurin arvo yksikkökiekon neljänneksessä
$A= \{\,(x,y)\in\Rkaksi \mid x \ge 0\,\ja\,y \ge 0\,\ja\,x^2+y^2 \le 1\,\}$.

\item
a) Esitä kompleksifunktion $f(z)=(z+1)^2+(z+i)^2$ reaaliosa, imaginaariosa ja itseisarvo
funktioina tyyppiä $g:\ \Rkaksi \kohti \R$. \vspace{1mm}\newline
b) Määrittele kaksihaaraiset funktiot $u=\{u_1,u_2\}$ ja $v=\{v_1,v_2\}$ siten, että
$u(x,y)+iv(x,y)=\sqrt{z}\ \ \forall z=x+iy\in\C$.

\item
Yhtälö $x+3xyz^2+z^4=0\ (x,y,z\in\R)$ määrittelee implisiittifunktion $z=f(x,y)$. \ a) Laske 
$f$:n arvot pisteissä $(0,0)$, $(2,-1)$ ja $(1,1)$. \ b)  Millaisia ovat $f$:n haarat 
(määrittelyjoukot ja laskusäännöt) yleisemmin?

\item (*)
Vuoristoisen maaston korkeus merenpinnasta on origon $O$ lähellä funktio
\[
f(x,y)=\frac{1}{20}(3x^2-5xy-2y^2)+2.
\]
Origossa on kahden tien $S_1,S_2$ risteys. Tiet ovat maastoon sovitettuja ja vaakasuoria, ja 
lisäksi ne ovat kartallakin suoria. Teiden $S_1,S_2$ poikki kulkee suora rautatie pisteissä
$A\in S_1$, $B\in S_2$. Molemmat pisteet ovat $O$:sta etäisyydellä $1$ (yksikkö = km).
Mikä on suurin pudotuskorkeus laaksoon rautatiesillalta, jonka päät ovat pisteissä $A,B$\,?

\item (*)
Halutaan selvittää, missä pisteissä funktio $f(x,y)=x^2-2xy+3y^2$ saavuttaa suurimman ja 
pienimmän arvonsa\, a) ympyrällä \mbox{$S: x^2+y^2=9$}, \ \ b) kiekossa 
$A=\{(x,y)\in\Rkaksi \mid x^2+y^2 \le 9\}$. Ratkaise ongelma napakoordinaatteja käyttäen 
(vrt.\ Harj.teht.\,\ref{trigonometriset funktiot}:\ref{H-II-5: minmax}).

\item (*)
Funktiosta $f(x)=x-x^3\,$ tiedetään, että välillä $[0,1]$ $f$ saavuttaa suurimman arvonsa
pisteessä $x=1/\sqrt{3}$. Mihin suuntaan origosta lähdettäessä funktio\, a) $f(x,y)=xy^2$,\,
b) $f(x,y,z)=xy^2z^3$ kasvaa nopeimmin suhteessa kuljettuun matkaan?

\item (*) \index{zzb@\nim!Funktio avaimenreiässä} 
(Funktio avaimenreiässä) Olkoon
\[
A=\{\,(x,y,z)\in \R^3 \mid (x,y)\in B_1 \cup B_2,\,\ z\in[0,10]\,\},
\]
missä 
\begin{align*}
B_1 &= \{\,(x,y)\in\Rkaksi \mid x^2+(y-1)^2 \le 2\,\ja\, y \ge 0\,\}, \\
B_2 &= \{\,(x,y)\in\Rkaksi \mid x\in[-1,1]\,\ja\,y\in [-4,0]\,\}.
\end{align*}
Missä $A$:n pisteissä funktio $f(x,y,z)=x-3y+2z$ saavuttaa suurimman ja missä pienimmän arvonsa?

\end{enumerate}

 % Kahden ja kolmen muuttujan funktiot
\section{Parametriset käyrät ja pinnat} \label{parametriset käyrät}
\alku
\index{funktio A!f@parametrinen käyrä ja pinta|vahv}
\index{parametrinen käyrä|vahv}
\index{parametri(sointi)!c@käyrän|vahv}
\index{kzyyrzy@käyrä|vahv}

Tason tai avaruuden \kor{parametriseksi käyräksi} sanotaan \pain{funktiota} muotoa
\[ 
t \in A\ \map\ P(t)\in E^d, 
\]
missä $A \subset \R$ on \pain{väli} (usein suljettu väli) ja $d=2$ tai $d=3$. Muuttujaa $t$ 
sanotaan tässä \kor{parametriksi}. Käyrä on \kor{tasokäyrä} jos $d=2$, \kor{avaruuskäyrä} jos 
$d=3$. Käyttäen jo tutuksi tulleita geometrisia vastaavuuksia voidaan kirjoittaa
\[
P(t)\ \vastaa\ \Vect{OP}(t)\ 
                = \left\{ \begin{array}{lrlll} 
                   x(t)\vec{i}+y(t)\vec{j} & \vastaa & (x(t),y(t)),      & & (d=2) \\
                   x(t)\vec{i}+y(t)\vec{j}+z(t)\vec{k} & \vastaa & (x(t),y(t),z(t)), & & (d=3)
                          \end{array} \right.
\]
missä $x,y,z$ ovat funktioita tyyppiä $f: A \kohti \R$. Tämän mukaisesti parametrinen käyrä 
\index{vektoric@vektoriarvoinen funktio}%
voidaan tulkita reaalimuuttujan \kor{vektoriarvoiseksi} funktioksi, jolloin luonteva esitystapa
on myös vektorimerkintä
\[ 
\vec{r}\,(t)\ =\ \begin{cases} 
   x(t)\vec{i}+y(t)\vec{j},\ \ t \in A,                   &\text{(tasokäyrä)} \\
   x(t)\vec{i}+y(t)\vec{j}+z(t)\vec{k},\ \ t \in A. \quad &\text{(avaruuskäyrä)}
               \end{cases}
\]
Liittyen vastaavuuteen $\mathit{E^d} \vast \R^d$ voidaan parametrinen käyrä esittää yhtä hyvin 
yhtälöryhmänä, esim.\ tasokäyrän tapauksessa
\[ 
(x,y)\ =\ (x(t),y(t)) \qekv 
        \begin{cases} \,x = x(t), \\ \,y = y(t). \end{cases} \quad 
\text{(tasokäyrä)}\footnote[2]{Merkinnöissä $x=x(t)$ ja $y=y(t)$ on symboleja $x,y$ käytetty
kahdessa eri merkityksessä: oikealla funktion, vasemmalla ko.\ funktion arvojoukon alkion
symbolina. Tämän tyyppisiä epäloogisuuksia pidetään matematiikan käytännössä siedettävinä,
syystä että ne yksinkertaistavat merkintöjä.}
\]
\begin{Exa} Reaalimuuttujan funktio $f: [a,b] \kohti \R$ voidaan tulkita parametriseksi
tasokäyräksi
\[ 
\vec{r} = \vec{r}\,(t) = t\vec{i} + f(t)\vec{j} \qekv 
                           \begin{cases} \,x = x(t) = t, \\ \,y = y(t) = f(t),\ \ t \in [a,b]. 
                           \end{cases} \loppu 
\] 
\end{Exa}
\begin{Exa} Luvuista \ref{suorat ja tasot} ja \ref{koordinaatistot} tuttuja parametrisia 
avaruuskäyriä ovat
\begin{align*}
&\text{avaruussuora:}\quad \vec{r}\,(t)\ 
          =\ (x_0 + \alpha t)\vec{i} + (y_0 + \beta t)\vec{j} + (z_0 + \gamma t)\vec{k}\ 
          =\ \vec{r}_0 + t \vec{v}, \quad t \in \R, \\ 
&\text{ruuviviiva:}\quad \begin{cases}
                          \,x = x(\varphi) = R\cos\varphi, \\ 
                          \,y = y(\varphi) = R\sin\varphi, \\ 
                          \,z = z(\varphi) = a\varphi,\ \ \varphi \in \R. \loppu
                         \end{cases} 
\end{align*} 
\end{Exa}
Parametrisen käyrän euklidiseen avaruuteen jättämä 'geometrinen jälki' on arvojoukko 
$S = \{P(t) \vastaa \vec{r}\,(t) \mid t \in A\} \subset E^d$, johon voidaan viitata sellaisilla 
termeillä kuin \kor{käyrä} (engl.\ curve) tai (käyrän) \kor{kaari} (engl.\ arc). 
Yksinkertaisimmillaan $S$ on jostakin pisteestä $A$ alkava ja toiseen pisteeseen $B$ päättyvä,
\index{yksinkertainen!c@käyrä, kaari} \index{kaari (käyrän)}%
itseään leikkaamaton ja yhtenäinen viiva, eli nk.\ \kor{yksinkertainen kaari} (engl.\ 
simple arc). Tällaisia ovat esim.\ jana tai ympyrän kaari. Viiva voi myös olla päättymätön, 
kuten suora, tai umpinainen
\index{suljettu käyrä}%
\kor{suljettu käyrä} (engl.\ closed curve), kuten tason tai 
avaruuden ympyräviiva. Joukko $S$ voi myös olla itseään leikkaava, eli siinä voi olla 
'silmukoita'.\footnote[2]{Muodossa $S = \{P(t) \vastaa \vec{r}\,(t) \mid t \in A\}$
määriteltyjen taso- tai avaruuskäyrien geometrista luokittelua ei todellisuudessa voi
täsmentää ilman funktion $\vec{r}$ koordinaattifunktioile $x,y,z$ asetettavia lisäehtoja.
Vrt.\ käyriä koskeva alaviite edellisessä luvussa.}
Jos lähtökohdaksi otetaan vain tällainen 'viiva' $S$, eli pelkästään geometrinen objekti, niin 
funktiota $\vec{r}\,(t),\ t \in A$, jonka arvojoukko $=S$, sanotaan $S$:n 
\kor{parametriesitykseksi} eli \kor{parametrisoinniksi} (parametrisaatioksi). 
Parametrisoivan funktion $t \in A \map \vec{r}\,(t) \in S$ ei tarvitse olla injektiivinen, ts.\ 
samaan pisteeseen $P \in S$ voidaan päätyä monella (jopa äärettömän monella) parametrin
arvolla.
\begin{Exa} \label{ympyrän parametrisaatio} Tason ympyräviivan 
\[ 
S = \{P \vastaa (x,y) \in \R^2 \mid (x-x_0)^2 + (y-y_0)^2 = R^2\} 
\]
luonteva parametrisointi on
\[ 
\begin{cases} x = x_0 + R \cos t, \\ y = y_0 + R\,\sin t,\ \ t \in [0,2\pi). \end{cases} 
\]
Tässä voi välin $[0,2\pi)$ tilalla olla myös esim.\ $A = (-\pi,\pi]$, $A = [0,2\pi]$ tai
$A = \R$. Kahdessa jälkimmäisessä vaihtoehdossa parametrisointi ei ole injektio. \loppu
\end{Exa}
\begin{Exa} Jos $S$ on avaruussuora, niin tämän tavanomaisin parametrisaatio on 
$\vec{r}\,(t) = \vec{r}_0 + t \vec{v},\ t \in \R$, missä $P_0 \vastaa \vec{r}_0$ on suoran
piste ja $\vec{v} \neq \vec{0}$ suoran suuntavektori (vrt.\ Luku \ref{suorat ja tasot}).
Tällaisiakin parametrisointeja on jo äärettömän monta, mutta mahdollisuudet eivät lopu tähän:
Jos $\vec{r}_0$ ja $\vec{v}$ täyttävät mainitut ehdot, niin parametrisoinniksi voidaan
yleisemmin valita
\[ 
\vec{r}\,(t)\ =\ \vec{r}_0 + f(t)\,\vec{v}, \quad t \in A, 
\]
missä funktion $f: A \kohti \R$ valintaa rajoittaa vain ehto $\RF_f = \R$. Esimerkiksi voidaan 
valita $f(t) = \tan t,\ A = (-\pi/2,\pi/2)$ ($f$ injektio), tai $f(t) = t \sin t,\ A = \R$ 
($f$ ei injektio).  \loppu \end{Exa}
\jatko \begin{Exa} (jatko) Jos halutaan parametrisoida suoralla $S$ oleva jana, jonka 
päätepisteet ovat $A \vastaa \vec{r}_1$ ja $B \vastaa \vec{r}_2$, niin tämä käy muodossa
\[ 
\vec{r}\,(t) = f(t)\,\vec{r}_1 + [1-f(t)]\,\vec{r}_2,\ \ t \in A, 
\]
missä $f$ on funktio tyyppiä $f: A \kohti \R,\ A \subset \R$, ja $\RF_f = [0,1]$.
Yksinkertaisin parametrisointi saadaan, kun valitaan $A = [0,1]$ ja $f(t) = t$. \loppu 
\end{Exa}
Jos tasokäyrän yhtälöistä $x = x(t),\ y = y(t),\ t \in A$ pystytään eliminoimaan parametri
$t$, on tuloksena yhtälö muotoa
\[ 
F(x,y) = 0, 
\]
missä siis $F$ on kahden muuttujan funktio, jolle pätee $F(x(t),y(t)) = 0\,\ \forall t \in A$.
Jos $S = \{P \in \Ekaksi \mid P \vastaa (x(t),y(t))\ \text{jollakin}\ t \in A\}$, niin
sanotaan tällöin, että ym.\ yhtälö on \kor{käyrän} $S$ (tai pistejoukon $S$) \kor{yhtälö}.
Avaruuskäyrän tapauksessa johtaa parametrin eliminointi (onnistuessaan) yhtälöryhmään muotoa
\[ 
\begin{cases} \,F_1(x,y,z) = 0, \\ \,F_2(x,y,z) = 0. \end{cases} 
\]
\begin{Exa} Jos $a,b>0$, niin parametrinen tasokäyrä
\[ 
S:\ \begin{cases} \,x = a\cos t, \\ \,y = b\,\sin t,\ \ t \in [0,2\pi] \end{cases} 
\]
on nimeltään \kor{ellipsi} (tapauksessa $a=b$ ympyrä). Eliminoimalla $t$ saadaan $S$:n
yhtälöksi
\[ 
\frac{x^2}{a^2} + \frac{y^2}{b^2} = 1. \loppu
\]
\end{Exa}

\subsection{Liikerata}
\index{liikerata|vahv}

Tyypillisessä parametrisen käyrän fysikaalisessa sovellustilanteessa parametri $t$ on 
\pain{aika}muuttuja, $A$ on tarkasteltava \pain{aikaväli}, ja $P(t) \vastaa \vec{r}\,(t)$ on 
liikkuvan pisteen (esim.\ pistemäiseksi ajatellun partikkelin tai liikkuvan kiinteän kappaleen 
pisteen) p\pain{aikka} hetkellä $t$. Tällöin funktion $P(t),\ t \in A$, arvojoukko $S$ on ko.\
pisteen \pain{liikerata} aikavälillä $A$. Funktio $t\map\vec r\,(t),\ t \in A$ on $S$:n
parametrisointi, joka kertoo koko \pain{liikehistorian}. 
\index{zza@\sov!Heittoparaabeli}%
\begin{Exa}: \vahv{Heittoparaabeli}. \label{heittoparaabeli}
Kivi heitetään tornista korkeudelta $h$ alkuvauhdilla $v_0$ ja kulmassa $\alpha$ vaakasuuntaan
nähden. Millainen on lentorata, jos ilmanvastusta ei huomioida?
\end{Exa}
\ratk Tarkastellaan liikettä (avaruustason) koordinaatistossa, jossa $x$ mittaa vaakasuoraa
etäisyyttä lähtöpisteestä ja $y$ korkeutta maan pinnan tasosta. Liikelakien mukaan kiven
paikka $P(t)=(x(t),y(t))$ on lentoajan $t$ funktiona parametrinen käyrä
\[
\begin{cases}
\,x(t)=v_0t\cos\alpha, \\
\,y(t)=h+v_0t\sin\alpha-\tfrac{1}{2}gt^2,
\end{cases}
\]
missä $g=$ maan vetovoiman kiihtyvyys. Eliminoimalla $t$ ja huomioimalla, että
$1/\cos^2\alpha=1+\tan^2\alpha\,$ saadaan lentoradan yhtälö muotoon
\[
y=h+kx-(1+k^2)\frac{x^2}{2a}\,,
\]
\index{paraabeli}%
missä $\,k=\tan\alpha\,$ ja $\,a=v_0^2/g$. Lentorata on \kor{paraabelin} kaari. Kuvan
tapauksessa $\alpha=0$ kivi törmää maahan hetkellä $t=\sqrt{2h/g}$. \loppu
\begin{figure}[H]
\setlength{\unitlength}{1cm}
\begin{center}
\begin{picture}(8,5)(0,0)
\put(0,0){\vector(1,0){8}} \put(7.8,-0.5){$x$}
\put(1,0){\vector(0,1){5}} \put(1.2,4.8){$y$}
\linethickness{0.05cm}
\multiput(0,0)(1,0){2}{\line(0,1){3.7}}
\multiput(0,3.7)(0.4,0){3}{\line(1,0){0.2}}
\multiput(0.2,3.7)(0.2,0){4}{\line(0,-1){0.2}}
\multiput(0.2,3.5)(0.4,0){2}{\line(1,0){0.2}}
\thinlines
\curve(1,3.7,4,2.8,6,0)
\end{picture}
\end{center}
\end{figure}
\index{zza@\sov!Sykloidi}%
\begin{Exa}: \vahv{Sykloidi}. \label{sykloidi}
$R$-säteinen pyörä vierii liukumatta pitkin tasoa siten, että pyörän keskipisteen liikenopeus
on $v_0\vec i$, $v_0=\text{vakio}$. Määritä pyörän ulkokehän pisteen $P$ paikka ajan $t$
funktioina.
\end{Exa}
\ratk Oletetaan, että pyörä vierii pitkin $x$-akselia ja että $P$ on origossa, kun $t=0$. 
Tällöin ratakäyrän parametriesitys on
\begin{multicols}{2} \raggedcolumns
\[
\begin{cases} x(t) = v_0t-R\sin \varphi(t), \\ y(t) = R-R\cos \varphi(t), \end{cases}
\]

\vspace{1mm}

missä $\varphi(t)$ on vierimiskulma. 

\begin{figure}[H]
\setlength{\unitlength}{1cm}
\begin{center}
\begin{picture}(5,3)(-1,0)
\put(0,0){\vector(1,0){4}} \put(3.8,-0.5){$x$}
\put(0,0){\vector(0,1){3}} \put(0.2,2.8){$y$}
\put(2,1.25){\circle{2.5}}
\dashline{0.2}(0,2.5)(4,2.5) \put(-0.5,2.4){$\scriptstyle{2R}$}
\dashline{0.1}(2,1.25)(0.8,1.6)
\put(2,1.25){\vector(1,0){1}} \put(2.6,1.4){$\scriptstyle{v_0\vec i}$}
\dashline{0.1}(2,0)(2,1.25)
\put(2,1.25){\arc{0.6}{1.59}{3.43}}
\put(1.2,0.85){$\scriptstyle{\varphi(t)}$}
\put(1.93,1.18){$\scriptstyle{\bullet}$} \put(0.73,1.53){$\scriptstyle{\bullet}$} 
\put(0.5,1.7){$\scriptstyle{P}$}
\end{picture}
\end{center}
\end{figure}
\end{multicols}
Koska liukumista ei tapahdu, on oltava $\,R\varphi(t)=v_0t$, joten $P$:n paikkavektori
hetkellä $t$ on
\[
\vec r\,(t)=\left[v_0t-R\sin(\frac{v_0t}{R})\right]\,\vec i 
                                      + R\left[1-\cos(\frac{v_0t}{R})\right]\,\vec j.
\]
Jos parametriksi otetaan vierimiskulma $\varphi$, niin liikeradan parametriesitys on
\[ \left\{ \begin{aligned}
x&=x(\varphi)=R(\varphi-\sin\varphi), \\
y&=y(\varphi)=R(1-\cos\varphi).
\end{aligned} \right. \]
\index{sykloidi}%
Tätä sanotaan \kor{sykloidiksi}. \loppu
\begin{figure}[H]
\setlength{\unitlength}{1cm}
\begin{center}
\begin{picture}(8,3)(-0.5,0)
\put(0,0){\vector(1,0){7.5}} \put(7.3,-0.5){$x$}
\put(0,0){\vector(0,1){3}} \put(0.2,2.8){$y$}
\dashline{0.2}(0,2)(7.5,2) \put(-0.5,1.9){$\scriptstyle{2R}$}
\curve(
      0,         0,
    0.0206,    0.1224,
    0.1585,    0.4597,
    0.5025,    0.9293,
    1.0907,    1.4161,
    1.9015,    1.8011,
    2.8589,    1.9900,
    3.8508,    1.9365,
    4.7568,    1.6536,
    5.4775,    1.2108,
    5.9589,    0.7163,
    6.2055,    0.2913,
    6.2794,    0.0398,
    6.2849,    0.0234,
    6.3430,    0.2461,
    6.5620,    0.6534,
    7.0106,    1.1455)
\put(6.05,-0.4){$\scriptstyle{2\pi R}$}
\end{picture}
\end{center}
\end{figure}
\begin{Exa} Pistemäinen partikkeli on hetkellä $t=0$ ($t$:n yksikkö s) pisteessä $(1,1,1)$ 
(yksikkö m) ja liikkuu suoraviivaisesti vakionopeudella (vauhdilla) $v=10$ (yksikkö m/s)
siten, että eräällä ajan hetkellä partikkeli on pisteessä $(2,-1,0)$. Määritä partikkelin
sijainti $(x(t),y(t),z(t))$, kun $t \ge 0$. 
\end{Exa}
\ratk Partikkeli liikkuu suoralla, jonka suuntavektori on 
$(2\vec{i}-\vec{j})-(\vec{i}+\vec{j}+\vec{k})=\vec{i}-2\vec{j}-\vec{k}$. Liikesuuntaan
osoittava yksikkövektori on siis
\[ 
\vec{e} = \dfrac{1}{\sqrt{6}}(\vec{i}-2\vec{j}-\vec{k}),
\]
ja partikkelin paikkavektori hetkellä $t \ge 0$ näin ollen
\[
\vec{r}\,(t) = \vec{i}+\vec{j}+\vec{k} + (vt)\,\vec{e} \qekv
               \begin{cases} 
                \,x(t) = 1 + \dfrac{10t}{\sqrt{6}}, \\[3mm] 
                \,y(t) = 1 - \dfrac{20t}{\sqrt{6}}, \\[3mm] 
                \,z(t) = 1 - \dfrac{10t}{\sqrt{6}}.
               \end{cases} \quad\loppu
\]

\subsection{Parametriset pinnat}
\index{parametri(sointi)!d@pinnan|vahv}
\index{parametrinen pinta|vahv}

Euklidisen avaruuden $\Ekolme$ \kor{parametriseksi pinnaksi} sanotaan kuvausta (funktiota)
tyyppiä
\[
(u,v) \in A \map P(u,v) \in E^3,
\]
missä $A \subset \R^2$ ja muuttujia $u,v$ sanotaan parametreiksi. Liittyen vastaavuuksiin 
$P \in \Ekolme \vast \vec{r} \in V \vast (x,y,z) \in \R^3$
($V = \{\text{avaruuden vektorit}\}$) voidaan kuvauksen maalijoukoksi yhtä hyvin ajatella $V$
tai $R^3$. Kuvauksesta voidaan tällöin käyttää joko vektorimerkintää
\[
\vec r=\vec r\,(u,v),\quad (u,v)\in A,
\]
tai vastaavaa koordinaattimuotoista esitystä
\[
\begin{cases}
\,x=x(u,v), \\
\,y=y(u,v), \\
\,z=z(u,v), &(u,v)\in A.
\end{cases}
\]
\index{pinta}%
Funktion $(u,v) \in A \map P(u,v) \in \Ekolme$ arvojoukko $S \subset \Ekolme$ on \kor{pinta}
(engl.\ surface) geometrisena oliona.\footnote[2]{Pintojen täsmällisemmässä määrittelyssä
on samat ongelmat kuin käyrien, vrt.\ alaviitteet edellä. Tässä yhteydessä ei mihinkään
täsmennysyrityksiin ryhdytä, vaan nojaudutaan geometriseen intuitioon.}
Itse funktio on tällöin $S$:n eräs \kor{parametrisointi}. Jos lähtökohtana on pinta $S$, niin
parametrisointi pyritään usein valitsemaan siten, että lähtöjoukko $A$ on geometrialtaan 
mahdollisimman yksinkertainen, esim.\ suorakulmio. 
\kor{Pinnan} $S$ \kor{yhtälöksi} sanotaan yhtälöä muotoa
\[ 
F(x,y,z) = 0, 
\]
joka toteutuu jokaisella $(x,y,z) \vastaa P \in S$. Yhtälöön päädytään, jos parametrit $u,v$
pystytään eliminoidaan ym.\ koordinaattimuotoisesta esityksestä. Jos alunperin tunnetaan kolmen
reaalimuuttujan funktio $F$, niin sanotaan yleisemmin, että yhtälö $F(x,y,z) = c\ (c \in \R)$
\index{tasa-arvopinta}%
määrittelee $F$:n \kor{tasa-arvopinnan} (sikäli kuin kyseessä on pinta, ks.\ alaviite).
\begin{Exa} 'Kaikkien pintojen äiti' on \kor{taso}, jonka yleinen parametriesitys on muotoa 
(vrt. Luku \ref{suorat ja tasot})
\[ \begin{cases} 
    \,x(u,v) = x_0 + \alpha_1\,u + \alpha_2\,v, \\ 
    \,y(u,v) = y_0\,+ \beta_1\,u + \beta_2\,v, \\ 
    \,z(u,v) = z_0\,+ \gamma_1\,u\,+ \gamma_2\,v.
   \end{cases} \]
Näin määritellen taso kulkee pisteen $\vec r_0\vastaa (x_0,y_0,z_0)$ kautta ja sen 
suuntavektorit ovat $\vec v_1 = \alpha_1\,\vec i + \beta_1\,\vec j + \gamma_1\,\vec k$ ja 
$\vec v_2 = \alpha_2\,\vec i + \beta_2\,\vec j + \gamma_2\,\vec k$. Eliminoimalla parametrit 
(olettaen $\vec v_1$ ja $\vec v_2$ lineaarisesti riippumattomiksi) saadaan tasolle johdetuksi
yhtälö muotoa $F(x,y,z)=ax+by+cz+d=0$ (vrt.\ Luku \ref{suorat ja tasot}). \loppu
\end{Exa}
\begin{Exa} Jos $f: \DF_f \kohti \R,\ \DF_f \subset \R^2$ on kahden reaalimuuttujan funktio,
niin $f$:n \kor{kuvaaja} joukossa $A\subset\DF_f$ on pinta, jonka yhtälö on
\[
z=f(x,y),\quad (x,y)\in A.
\]
\begin{figure}[H]
\begin{center}
\import{kuvat/}{kuvaDD-1.pstex_t}
\end{center}
\end{figure}
Pinnan luonnollinen parametrisointi on tässä tapauksessa
\[
x=u,\quad y=v,\quad z=f(u,v), \quad (u,v) \in A. \loppu 
 \]
\end{Exa}
\begin{Exa} Avaruuden yleisen pallopinnan yhtälö on
\[ 
(x-x_0)^2 + (y-y_0)^2 + (z-z_0)^2 = R^2. 
\]
Luontevin parametrisointi perustuu pallonpintakoordinaatteihin:
\[ \begin{cases} \,x = x(\theta,\varphi) = x_0 + R \sin \theta \cos \varphi, \\
                 \,y = y(\theta,\varphi) = y_0 + R \sin \theta \sin \varphi, \\
                 \,z = z(\theta,\varphi) 
                   = z_0 + R \cos \theta, \quad (\theta,\varphi) \in [0,\pi] \times [0,2\pi].
   \end{cases} \]
Pallokoordinaatistossa, jonka origo on pisteessä $(x_0,y_0,z_0)$ on pinnan yhtälö kaikkein 
yksinkertaisin: $\,r = R$. \loppu \end{Exa}
\begin{Exa} Jos $a,b,c>0$, niin yhtälö
\[ 
\frac{x^2}{a^2} + \frac{y^2}{b^2} + \frac{z^2}{c^2} = 1 
\]
määrittelee pinnan nimeltä
\index{ellipsoidi}%
\kor{ellipsoidi}. Pallonpintakoordinaatteihin perustuva parametrisointi on
\[ \begin{cases} 
     \,x = a \sin \theta \cos \varphi, \\ 
     \,y = b \sin \theta \sin \varphi, \\ 
     \,z = c \cos \theta, \quad (\theta,\varphi) \in [0,\pi] \times [0,2\pi]. \loppu
   \end{cases} \]
\end{Exa}

\subsection{Pyörähdyspinnat}
\index{pyzzrzy@pyörähdyspinta|vahv}
\index{kzyyrzy@käyräviivaiset koordinaatistot!b@--pyörähdyspinnat|vahv}

\kor{Pyörähdyspinta} syntyy, kun tasokäyrä pyörähtää tasossa olevan suoran ympäri. Olkoon
käyrä annettu muodossa
\[
K=\{(x,y)\in\R^2 \ | \ y=f(x) \ \ja \ x\in [a,b]\},
\]
missä $f(x)\geq 0 \ \forall x\in [a,b]$. Tällöin käyrän pyörähtäessä $x$-akselin ympäri syntyy 
pinta $S$, jonka luonnolliset parametrit ovat $u=x$ ja $v=\varphi=\text{pyörähdyskulma}$, 
jolloin pinnan parametrisoinniksi tulee
\begin{multicols}{2} \raggedcolumns
\[
\begin{cases}
\,x=u, \\
\,y=f(u)\cos\varphi, \\
\,z=f(u)\sin\varphi,
\end{cases}
\]
missä
\[
(u,\varphi)\in A=[a,b]\times [0,2\pi].
\]
\begin{figure}[H]
\begin{center}
\import{kuvat/}{kuvaDD-2.pstex_t}
\end{center}
\end{figure}
\end{multicols}
Eliminoimalla parametrit $u,\varphi$ saadaan \kor{pyörähdyspinnan yhtälö}
\[
\boxed{\kehys\quad y^2+z^2=[f(x)]^2,\quad x\in [a,b]. \quad}
\]
\begin{Exa} Parametrisoi pyörähdyspinta
\[
S: \quad x^2+y^2=z,\quad z\geq 0.
\] \end{Exa}
\ratk Pinta $S$ syntyy kun $yz$-tason käyrä $\,K=\{(y,z)\in\R^2 \mid z=y^2,\ y \ge 0\}$
pyörähtää $z$-akselin ympäri. Luonteva parametrisointi saadaan lieriökoordinaattien avulla:
\begin{multicols}{2} \raggedcolumns
\[
\begin{cases}
\,x=r\sin\varphi, \\
\,y=r\cos\varphi, \\
\,z=r^2,
\end{cases}
\]
missä
\[
(r,\varphi)\in A=[0,\infty)\times [0,2\pi].
\]
\index{paraboloidi}%
Pinta on (pyörähdys)\kor{paraboloidi}. \loppu
\begin{figure}[H]
\begin{center}
\import{kuvat/}{kuvaDD-3.pstex_t}
\end{center}
\end{figure}
\end{multicols}
Esimerkki on erikoistapaus yleisemmästä pyörähdyspinnasta, joka syntyy, kun $yz$-tason 
käyrä \,$K:\ z=f(y),\ y \in B \subset [0,\infty)$, pyörähtää $z$-akselin ympäri.
Lieriökoordinaatteihin perustuva pinnan (luontevin) parametrisointi on
\[
\begin{cases}
\,x=r\cos\varphi, \\
\,y=r\sin\varphi, \\
\,z=f(r), \quad (r,\varphi) \in A = B \times [0,2\pi].
\end{cases}
\]
Näistä yhtälöistä viimeinen on itse asiassa pinnan yhtälö lierökoordinaateissa (!).
\begin{figure}[H]
\begin{center}
\import{kuvat/}{kuvaDD-4.pstex_t}
\end{center}
\end{figure}

\subsection{Viivoitinpinnat}
\index{viivoitinpinta|vahv}

\kor{Viivoitinpinta} syntyy, kun suora tai jana liikkuu avaruudessa siten, että suoran/janan
piste $P_0\vastaa\vec r_0$ ja suuntavektori $\vec t$ ovat yhdestä parametrista ($u$) riippuvia.
Pinnan luonnollinen parametrisaatio on tällöin
\begin{align*}
\vec r\,(u,v) &= \vec r_0(u)+v\,\vec t\,(u) \\
              &= x(u,v)\vec i +y(u,v)\vec j+z(u,v)\vec k.
\end{align*}
\index{zza@\sov!Jzyzy@Jäähdytystorni}%
\begin{Exa}: \vahv{Jäähdytystorni}. \label{jäähdytystorni}
Jana, jonka päätepisteet ovat $A=(2,0,0)$ ja $B=(0,1,3)$ pyörähtää $z$-akselin ympäri.
Millainen parametrisoitu pinta syntyy? Kyseessä on myös pyörähdyspinta --- millainen?
\end{Exa}
\ratk Kulman $u$ verran (kuvio) pyörähtänyt suuntajana on
\begin{multicols}{2} \raggedcolumns
\begin{align*}
\vec t\,(u) 
&= \overrightarrow{A'B'} \\
&= (-\sin u\,\vec i + \cos u\,\vec j + 3\vec k) -(2\cos u\,\vec i + 2\sin u\, \vec j) \qquad \\
&=-(2\cos u+\sin u)\vec i + (\cos u-2\sin u)\vec j +3\vec k.
\end{align*}
\begin{figure}[H]
\begin{center}
\import{kuvat/}{kuvaDD-5.pstex_t}
\end{center}
\end{figure}
\end{multicols}
Pinnalle saadaan näin ollen parametrisointi
\begin{align*}
\vec r
&= \vec r\,(u,v)=\vec r_0(u)+v\vec t\,(u) \\
&= 2\cos u\,\vec i+2\sin u\,\vec j +v\vec t\,(u) \\
&= [(2-2v)\cos u-v\sin u]\vec i+[v\cos u +(2-2v)\sin u]\vec j+3v\vec k \\ \\
\ekv \ &\begin{cases}
\,x=(2-2v)\cos u-v\sin u, \\
\,y=v\cos u+(2-2v)\sin u, \\
\,z=3v,
\end{cases} \quad (u,v) \in [0,2\pi] \times [0,1].
\end{align*}
Parametriesityksestä nähdään, että
\begin{align*}
[x(u,v)]^2+[y(u,v)]^2\ &=\ (2-2v)^2+v^2 \\
                       &=\ 5v^2-8v+4.
\end{align*}
Koska tässä $v=z(u,v)/3$, niin nähdään, että pinta voidaan esittää lieriökoordinaatistossa 
muodossa
\begin{align*}
r^2 &=\ \frac{5}{9}\,z^2-\frac{8}{3}\,z+4 \\
    &=\ \frac{5}{9}\left(z-\frac{12}{5}\right)^2 + \frac{4}{5}\,.
\end{align*}
Tämä on pyörähdyspinta, joka syntyy, kun $yz$-tason käyrä
\[
K:\quad y^2-\frac{5}{9}\left(z-\frac{12}{5}\right)^2 =\ \frac{4}{5}\,,\quad z \in [0,3]
\]
pyörähtää $z$-akselin ympäri. Käyrä $K$ on
\index{hyperbeli} \index{hyperboloidi} \index{yksivaippainen hyperboloidi}%
\kor{hyperbelin} kaari, ja pyörähdyspinta on \kor{yksivaippaisen hyperboloidin} osa.
\begin{figure}[H]
\begin{center}
\epsfig{file=kuvat/hyperboloidi.eps}
\end{center}
\end{figure}
Pinnan kapein kohta on korkeudella $z=12/5$. \loppu

\pagebreak

\Harj
\begin{enumerate}

\item
Hahmottele seuraavien parametristen tasokäyrien kulku. Eliminoimalla parametri johda myös
käyrän yhtälö karteesisessa koordinaatistossa.
\begin{align*}
&\text{a)}\ \ x=2-t,\ y=t+1,\ t\in\R \qquad\ \text{b)}\ \ x=t^2,\ y=2-t,\ t\in[0,\infty) \\
&\text{c)}\ \ x=\frac{1}{t}\,,\ y=t-1,\ t\in(0,4) \qquad\,
 \text{d)}\ \ x=\frac{1}{1+t^2}\,,\ y=\frac{t}{1+t^2}\,,\ t\in\R \\
&\text{e)}\ \ \vec r=3\sin\pi t\,\vec i+4\cos\pi t\,\vec j,\ t\in[-1,1] \\[1mm]
&\text{f)}\ \ x=1-\sqrt{4-t^2}\,,\ y=2+t,\ t\in[-2,2] \\[1mm] 
&\text{g)}\ \ \vec r=t\cos t\,\vec i+t\sin t\,\vec j,\ t\in[0,4\pi]
\end{align*}

\item
a) Tasokäyrän eräs parametrisointi on $\vec r=\cos 2t\,\vec i+\sin^2 t\,\vec j,\ t\in\R$. 
Anna käyrälle vaihtoehtoinen, mahdollisimman yksinkertainen parametrisointi.
\vspace{1mm}\newline
b) Totea, että $\,\vec r=(t-1)\vec i+\sqrt{2t-t^2}\,\vec j,\ t\in[0,2]\,$ ja 
$\,\vec r=t\sqrt{2-t^2}\,\vec i+(1-t^2)\vec j$, \newline $t\in[-1,1]$ ovat saman käyrän 
parametrisointeja. Mikä käyrä on kyseessä?

\item \index{Cartesiuksen lehti}
Tasokäyrä $S:\,x^3+y^3=3xy\,$ on nimeltään \kor{Cartesiuksen lehti}. Johda käyrälle
parametriesitys kirjoittamalla $y=tx$. Hahmottele käyrän kulku parametrimuodosta ja merkitse
kuvioon, mitkä käyrän osat vastaavat parametrin arvoja väleillä $(-\infty,-1)$, $(-1,0)$ ja 
$[0,\infty)$. Miksei $t=-1$ vastaa mitään käyrän pistettä?

\item \index{venytetty sykloidi}
Ympyrä, jonka säde on $R=1$, vierii liukumatta pitkin positiivista $x$-akselia. Ympyrän mukana
pyörii siihen kiinnitetty jana, jonka toinen päätepiste on ympyrän keskipisteessä ja keskipiste
on ympyrän kehällä. Määritä janan toisen (ympyrän ulkopuolella olevan) päätepisteen sijainti
parametrisena käyränä $x=x(t),\ y=y(t)$, missä $t$ on ympyrän vierimiskulma mitattuna 
alkutilanteesta, jossa janan päätepisteet ovat $(0,1)$ ja $(0,3)$. Hahmottele käyrä
graafisesti. Missä pisteessä käyrä leikkaa ensimmäisen kerran itsensä? (Käyrää sanotaan 
\kor{venytetyksi sykloidiksi}.)

\item
Parametrisoi tason $T\,:x+y=4$ ja kartion $K:\,xy+yz+xz=0$ leikkauskäyrä ottamalla
parametriksi\, a) $t=x$, \ b) $t=x-y$.

\item
a) Avaruuskäyrän $S$ parametrisointi on $\vec r = \vec r_0+\cos t\,\vec a+\sin t\,\vec b$,
$\,t\in[0,2\pi]$, missä $\vec a$, $\vec b$ ja $\vec r_0$ ovat avaruusvektoreita. Täsmälleen
millä ehdoilla $S$ on avaruusympyrä?
\vspace{1mm}\newline
b) Avaruusympyrän keskipiste on $(1,1,2)$, säde on $R=3$ ja ympyrä on tasossa $x-y-3z+6=0$.
Johda ympyrälle jokin parametrisointi muotoa $\vec r=\vec r_0+\cos t\,\vec a+\sin t\,\vec b$,
$\,t\in[0,2\pi]$.

\item
Näytä, että yhtälöryhmä
$\D \ \begin{cases} \,x^2+y+z=2 \\ \,xy+z=1 \end{cases} $ \vspace{1mm}\newline
määrittelee kaksi leikkaavaa avaruuskäyrää, joista toisen parametrisointi on
$\vec r=t\vec i+(1+t)\vec j+(1-t-t^2)\vec k,\ t\in\R$. Millainen on toinen käyrä?

\item \index{zzb@\nim!Kiukkulintu ja kuulantyöntäjät} (Kiukkulintu ja kuulantyöntäjät)\, 
a) Kiukkulintu lennätetään alkupisteestä $(x,y)$ $=(0,1)$ (pituusyksikkö = cm) venyttämällä 
heittoparaabelissa (ks.\ Esimerkki\,\ref{heittoparaabeli}) vakion $a$ arvoksi $8$ cm ja
tähtäämällä porsaaseen, joka on pisteessä $(4,3)$. Millä $k$:n arvoilla tulee
osuma? \vspace{1mm}\newline 
b) Teekkarit Yrjölä ja Ståhlberg kisaavat kuulantyönnössä. Ratkaise, kumpi voitti, kun
kisaajien parhaissa työnnöissä heittoparaabelin parametrit ovat \newline
Yrjölä:   $\,\ \qquad h=2.00$ m, $\ \alpha=60.0\aste$, $\ a=8.00$ m \newline
Ståhlberg: $\quad h=1.80$ m, $\ \alpha=30.0\aste$, $\ a=6.65$ m

\item
Esitä jokin parametrisointi seuraavien yhtälöiden määräämille pinnoille: \newline
a)\, $x^3y^2z=5,\,\ $ b)\, $(x-z)(x+z)+y+2z=0,\,\ $ c)\, $x\sin z+xy^5+y=1$

\item
a) Johda pinnalle $S$ yhtälö muotoa $F(x,y,z)=0$ parametrisoinnista
\[ 
S:\ \begin{cases}
    \,x=3+2\sin\theta\cos\varphi, \\
    \,y\,=-1+\sin\theta\sin\varphi, \\
    \,z=2+3\cos\theta, \quad (\theta,\varphi)\in\Rkaksi.
    \end{cases}
\]
b) Pallon $K$ keskipiste on $(1,1,1)$ ja säde on $R=2$. Kuvan piirtoa varten halutaan
parametrisoida pallon $xy$-tason yläpuolinen ($z \ge 0$) osa. Esitä parametrisointi!
\vspace{1mm}\newline
c) Pinnan $S$ yhtälö lieriökoordinaateissa on $\,r=\varphi,\ (\varphi,z) \in A$, missä
$A=[0,4\pi]\times[-5,5]$. Parametrisoi $S$ viivoitinpintana. Millainen on $S$:n ja
$xy$-tason leikkauskäyrä? \vspace{1mm}\newline
d) Pinnan yhtälö lieriökoordinaateissa on $r=z^2\abs{\cos\varphi}$. Esitä pinnan yhtälö
karteesisissa koordinaateissa. Millaisia ovat pinnan ja tasojen $z=c$ ($c\in\R$)
leikkauskäyrät?

\item \index{hyperboloidi} \index{kaksivaippainen hyperboloidi}
a) Määritä sen viivoitinpinnan yhtälö (muodossa $F(x,y,z)=0$), joka syntyy, kun suora
$S:\ x=z,\ y=1$ pyörähtää $x$-akselin ympäri. Totea, että sama pinta (nimeltään yksivaippainen
hyperboloidi) syntyy myös, kun eräs $xy$-tason käyrä $K$ pyörähtää $x$-akselin ympäri. 
Hahmottele $K$ graafisesti. \vspace{1mm}\newline 
b) Tasokäyrän $K: x^2-y^2=1$ pyörähtäessä $x$-akselin ympäri syntyy pinta nimeltä
\kor{kaksivaippainen hyperboloidi}. Määritä ko.\ pinnan yhtälö. Missä pisteissä suora
$x=y=z$ leikkaa pinnan?

\item
a) Puolikartion $K$ kärki on origossa, symmetria-akseli on positiivinen $z$-akseli ja
puolisuora $x=2y=3z,\ x \ge 0$ on pinnalla $K$. Parametrisoi $K$ pyörähdyspintana ja
viivoitinpintana. Mikä on $K$:n yhtälö lieriökoordinaatistossa? \vspace{1mm}\newline
b) Parametrisoi kartio $K:\ xy+yz+xz=0\,$ viivoitinpintana.

\item (*) \index{asteroidi} \index{hyposykloidi}
Ympyrän keskipiste on origossa ja säde on $a$. Ympyrää pitkin sen sisäpuolella vierii liukumatta
toinen ympyrä, jonka säde on $b<a$. Tällöin vierivän ympyrän kiinteä piste $P$ piirtää 
tasokäyrän nimeltä \kor{hyposykloidi}. \ a) Näytä, että pisteen $(a,0)$ kautta kulkevan
hyposykloidin parametriesitys on
\[
x=(a-b)\cos t+b\cos\left(\frac{a-b}{b}\,t\right), \quad
y=(a-b)\sin t-b\sin\left(\frac{a-b}{b}\,t\right).
\]
b) Päättele, että tapauksessa $a=2b$ piste $P$ liikkuu pitkin janaa. \newline
c) Näytä, että tapauksessa $a=4b$ parametriesitys yksinkertaistuu muotoon 
\[
x=a\cos^3 t,\ \ y=a\sin^3 t.
\] 
Hahmottele tämän käyrän --- nimeltään \kor{asteroidi} --- kulku. Mikä on asteroidin yhtälö
karteesisissa koordinaateissa? 

\item (*) \index{zzb@\nim!Sotaharjoitus 1}
(Sotaharjoitus 1) Origosta ammutun tykinkuulan lentorata on ajan $t$ funktiona (yksiköt km ja s)
\[ \begin{cases}
x(t)=(\sin\theta\cos\varphi+a)\,t, \\
y(t)=(\sin\theta\sin\varphi+b)\,t, \\
z(t)=(\cos\theta)\,t-0.005\,t^2,
\end{cases} \]
missä $\theta,\varphi$ ovat suuntauskulmat ja $a,b$ ovat tuuliparametreja. Maastoesteet
asettavat suuntaukselle rajoituksen $\tan\theta > 0.2$. Miten suuntaus on valittava 
tuulettomassa säässä ($a=b=0$), jotta ammus osuisi pisteessä $(10,20,0)$ olevaan maaliin? 
Kuinka korkealla ammus käy? Kuinka kauas maalista ammus osuu tällä suuntauksella, jos 
$a=0.002$ ja $b=-0.001$\,? %Miten suuntausta olisi (likimain) muutettava?

\item (*)
Jana, jonka pituus on $20$, liikkuu seuraavasti: Janan keskipiste liikkuu $z$-akselia pitkin
positiiviseen suuntaan vakionopeudella. Liikkuessaan jana pysyy $xy$-tason suuntaisena ja pyörii
tasaisesti (kulmanopeus vakio) siten, että keskipisteen liikkuessa $30$ pituusyksikköä
jana pyörii täyden kierroksen positiivisen $z$-akselin suunnasta katsottuna vastapäivään.
Esitä janan avaruuteen piirtämän viivoitinpinnan $S$ parametrisointi, kun tiedetään lisäksi,
että piste $(1,0,0)$ on tällä pinnalla. Leikkaako suora $z=25,\ x+y=4$ pinnan $S$\,?

\end{enumerate} % Parametriset käyrät ja pinnat
\section{*Funktioavaruus} \label{funktioavaruus}
\alku
\index{funktioavaruus|vahv}

Tarkastellaan \pain{samassa} joukossa $A$ määriteltyjä yhden, kahden tai kolmen 
reaalimuuttujan reaaliarvoisia funktioita ja merkitään näiden joukkoa $V$:llä:
\[ 
V\ =\ \{\,\text{funktiot}\ f:\ A \kohti \R\,\}.
\]
Joukossa $V$ on määritelty funktioiden yhteenlasku $f,g \map f+g$ ja skalaarilla kertominen
$f \map \lambda f$ aiemmin kerrotulla tavalla (Määritelmän \ref{funktioiden yhdistelysäännöt}
säännöt (1) ja (2), kun $A \subset \R$). Näiden laskuoperaatioiden perusteella $V$ on
\index{vektorib@vektori (algebrallinen)!d@funktioavaruuden}%
mahdollista tulkita vektoriavaruudeksi. Samassa joukossa määritellyt funktiot voidaan siis 
mieltää 'vektoreiksi', jolloin puhutaan \kor{funktioavaruudesta} (engl.\ function space). 
\index{nollafunktio}%
Tällaisen avaruuden nolla-alkio on nk.\ (algebrallinen) \kor{nollafunktio}, joka määritellään
\[ 
\mathbf{0}(x) = 0\,\ \forall x \in A .
\]
Äärellinen funktiojoukko $\{ f_i,\ i = 1 \ldots n\} \subset V$ on (albegrallisesti) 
\index{lineaarinen riippumattomuus}%
\kor{lineaarisesti riippumaton}, jos pätee
\[ 
\sum_{i=1}^n \lambda_i f_i = \mathbf{0} \qimpl \lambda_i = 0,\ i = 1 \ldots n. 
\]
\begin{Exa} Enintään kolmannen asteen polynomit (määrittelyjoukko $= \R$) muodostavat 
funktioalgebran yhdistelysääntöjen perusteella funktioavaruuden
\[
V\ =\ \{ f = \lambda_1 f_1 + \lambda_2 f_2 + \lambda_3 f_3 + \lambda_4 f_4 
                                                                   \mid \lambda_i \in \R \},
\]
missä $f_1(x) = 1$, $f_2(x) = x$, $f_3(x) = x^2$ ja  $f_4(x) = x^3$. Algebran
peruslauseesta (ks.\ Luku \ref{III-3}) on helposti pääteltävissä, että funktiot $f_i$ ovat
lineaarisesti riippumattomia, joten $\{f_i,\ i = 1 \ldots 4\}$ on $V$:n kanta. Siis $V$ on
neliulotteinen vektoriavaruus: dim $V = 4$. \loppu \end{Exa}
\begin{Exa} Funktiot muotoa $f(x)=c_1\sin x+c_2\cos x,\ c_1,c_2\in\R$ muodostavat 2-ulotteisen
funktioavaruuden, luonnollisena kantana $\{\sin x,\cos x\}$. \loppu
\end{Exa}
\begin{Exa} Funktioavaruus
\[
V=\{f(x)=c_1+c_2\cos^2 x+c_3\sin^2 x,\ c_i\in\R\}
\]
ei ole 3-ulotteinen, sillä funktiosysteemi $\{f_1\,,f_2\,,f_3\}=\{1,\cos^2 x,\sin^2 x\}$ on 
lineaarisesti riippuva:
\[
f_1-f_2-f_3=\mathbf{0}.
\]
Mikä tahansa pari mainitusta kolmesta funktiosta sen sijaan on lineaarisesti riippumaton,
joten $V$ on 2-ulotteinen, kantana esim.\ $\{1,\cos^2 x\}$. Eräs $V$:n 1-ulotteinen aliavaruus
on
\[
W=\{\lambda\cos 2x \mid \lambda\in\R\},
\]
sillä $\,\cos 2x=2f_2-f_1 \in V$. \loppu
\end{Exa}
\begin{Exa} \label{polynomiavaruus}
Funktioavaruus
\[
V=\{f(x,y)=c_1+c_2x+c_3y+c_4x^2+c_5xy+c_6y^2,\ c_i\in\R\}
\]
koostuu kahden muuttujan polynomeista enintään astetta $2$ ja tämän aliavaruus
\[
W=\{f(x,y)=c_1+c_2x+c_3y,\ c_i\in\R\}
\]
enintään ensimmäisen asteen polynomeista. Funktiosysteemi $\{1,x,y,x^2,xy,y^2\}$ on 
lineaarisesti riippumaton (Harj.teht.\,\ref{H-IV-8: polynomit}), joten dim $V=6$ ja dim $W=3$.
\loppu
\end{Exa} 

\Harj
\begin{enumerate}

\item
Näytä, että funktiot $f_1(x)=x$ ja $f_2(x)=\abs{x}$ ovat lineaarisesti riippumattomat, jos
määrittelyjoukkona on väli $[-1,1]$ ja lineaarisesti riippuvat, jos määrittelyjoukko on
väli $[0,1]$.

\item
Montako alkiota on joukossa $A\subset\R$ oltava, jotta $A$:ssa määritellyt funktiot
$f_i(x)=x^{i-1},\ i=1 \ldots n$ ovat lineaarisesti riippumattomat?

\item
a) Näytä, että $\{1,\sin x,\sin^2 x,\sin^3 x\}$ on erään 4-ulotteisen funktioavaruuden
$V$ kanta (funktioiden määrittelyjoukko $=\R$). \ b) Sikäli kuin 
$f(x)=2+\cos 2x-\sin 3x \in V$, määrää $f$:n koordinaatit (kertoimet) mainitussa kannassa.

\item \label{H-IV-8: polynomit}
Näytä, että Esimerkin \ref{polynomiavaruus} polynomiavaruus $V$ on 6-ulotteinen, ts.\ näytä, että
$c_1+c_2x+c_3y+c_4x^2+c_5xy+c_6y^2=0\,\ \forall\,(x,y)\in\Rkaksi\,\ \impl\,\ c_1 = \ldots 
=c_6=0$.

\item (*) Olkoon $V=\{f(x)=c_1+c_2x+c_3x^2 \mid c_i\in\R\}$. Näytä, että
\[
\scp{f}{g}=f(0)g(0)+f(1)g(1)+f(2)g(2), \quad f,g \in V
\]
määrittelee $V$:n skalaaritulon. Laske funktion $u(x)=2+3x-2x^2$ ortogonaaliprojektio $w$ 
(ko.\ skalaaritulon mielessä) $V$:n aliavaruuteen $W$, jonka kanta on $\{1,x\}$. Piirrä samaan
kuvaan funktioiden $u$ ja $w$ kuvaajat välillä $[0,2]$. Tarkista, että $u-w \perp W$, ts.\ 
$\scp{u-w}{v}=0\ \forall v \in W$.

\end{enumerate} % *Funktioavaruus

\chapter{Jatkuvuus ja derivoituvuus}  \label{jatkuvat funktiot}

Funktion \kor{jatkuvuus} ja \kor{derivoituvuus} ovat käsitteellisiä peruslähtökohtia 
matematiikan suuntauksessa, jota kutsutaan väljästi \kor{analyysiksi}. Jatkuvuudesta, tai 
yleisemmin funktion \kor{säännöllisyydestä} puhuttaessa on kyse funktion arvojen
ennustettavuudesta muuttujan (muuttujien) arvojen vaihdellessa. Jatkuvuuden ja derivoituvuuden
käsitteet tuovat matemaattisten funktioiden teoriaan kokonaan oman 'makunsa' verrattuna
tähän asti tarkasteltuun funktioiden algebraan.

Tässä luvussa tarkastelun kohteena ovat pääosin vain yhden reaalimuuttujan funktiot. Näille
funktioille määritellään ensin jatkuvuus peruskäsitteenä ja jatkuvuuteen läheisesti liittyvä
\kor{funktion raja-arvon} käsite. Pelkkää jatkuvuutta vahvempina säännöllisyyden lajeina
määritellään mm.\ derivoituvuus (Luku \ref{derivaatta}) ja derivoituvuutta vahvemmat 
\kor{sileyden} lajit (Luku \ref{ääriarvot}). Näiden käsitteiden pohjalta luonnehditaan
funktioita, mm.\ esitetään kaksi reaalimuuttujan analyysin keskeistä \kor{väliarvolausetta} ja
tarkastellaan näiden lauseiden seuraamuksia yhtälöiden ja myös yksinkertaisten 
\kor{differentiaaliyhtälöiden} ratkeavuusteoriassa. 

Derivaatta tuo mukanaan myös derivoimissäännöt eli uuden ulottuvuuden funktioalgebraan.
Luvuissa \ref{derivaatta}--\ref{kaarenpituus} johdetaan kaikki keskeiset derivoimissäännöt,
mukaanlukien implisiittifunktiot, potenssisarjan summana määritellyt funktiot ja
(Luvussa \ref{kaarenpituus} erikseen) trigonometriset funktiot. 

Luvussa \ref{kiintopisteiteraatio} tarkastelun kohteena ovat yhtälöitten likimääräisessä
ratkaisussa yleisesti käytettävien algoritmien, \kor{kiintopisteiteraatioiden},
toimintaperiaatteet ja suppenevuusteoria. Luvussa \ref{analyyttiset funktiot} tarkastellaan
lyhyesti jatkuvuuden ja derivoituvuuden käsitteiden laajentamista kompleksifunktioihin ja
määritellään tähän liittyen \kor{analyyttisen} kompleksifunktion käsite. Viimeisessä
osaluvussa todistetaan jatkuvien funktioiden teorian keskeisimpiä väittämiä kuten 
\kor{Weierstrassin lause}. Tässä yhteydessä esitetään myös lyhyesti, miten Algebran peruslause
on todistettavissa kompleksialgebran ja analyysin tuloksia yhdistelemällä.
 % Jatkuvuuvs ja derivoituvuus
\section{Jatkuvuuden käsite} \label{jatkuvuuden käsite}
\alku

Funktion jatkuvuuden ongelma tulee eteen niinkin yksinkertaisessa tehtävässä kuin funktion
arvon numeerisessa määrittämisessä, eli numeerisessa funktioevaluaatiossa. Tarkastellaan Luvun 
\ref{käänteisfunktio} Esimerkin \ref{algebrallinen käänteisfunktio} käänteisfunktiota, 
joka nyt kirjoitettakoon muotoon
\[
y=f(x) \ \ekv \ y^5+3y=x, \quad x\in\R.
\]
Tehtävänä olkoon laskea likimäärin luku $a=f(\pi)$, eli evaluoida $f$ numeerisesti $\pi$:ssä. 
Tähän itse asiassa sisältyy kaksi numeerista ongelmaa: Ensinnäkin luvulle $\pi$ ei ole olemassa
'tarkkaa' numeerista arvoa, ja toiseksi yhtälöä ei yleensä voi ratkaista $y$:n suhteen
tarkasti, vaikka $x$ olisi rationaaliluku. Käytännössä menetellään (laskin/tietokone 
menettelee) seuraavasti: Valitaan $\pi$:lle edustaja $x_n$ rationaalilukujonosta $\{x_n\}$,
jolle pätee $x_n\rightarrow \pi$. Lasketaan $a_n=f(x_n)$ likimäärin käyttämällä jotakin
numeerista algoritmia yhtälön $y^5+y=x_n$ ratkaisuun. Algoritmi tuottaa käytännössä
rationaalilukujonon $\{b_k\}$, jolle pätee $b_k \kohti a_n$ kun $k\rightarrow\infty$. Poimitaan
tästä jonosta luvun $a_n$ likiarvoksi $b_m$ jollakin $m$ (esim.\ $m=10$), ja ilmoitetaan
lopputuloksena tämän luvun likiarvo äärellisenä desimaalilukuna (katkaistuna tai pyöristettynä
liukulukuna). 

Jos em.\ laskussa ei huomioida numeerisia pyöristysvirheitä liukulukulaskennassa, niin 
lopputuloksen $b_m$ virhe koostuu kahdesta osasta:
\[
b_m-f(\pi)\,=\,[b_m-f(x_n)]+[f(x_n)-f(\pi)].
\]
Tässä ensimmäinen osa on approksimaation $b_m \approx f(x_n)$ virhe, eli kyse on yhtälön
$y^5+3y=x_n$ ratkaisualgoritmin tarkkuudesta. Virheen jälkimmäisessä osassa sen sijaan on kyse,
paitsi approksimaation $x_n \approx x$ tarkkuudesta, myös funktiosta $f$: Kyse on funktion 
j\pain{atkuvuudesta}. Kvalitatiivisesti väittämä '$f$ on jatkuva $x$:ssä' tarkoittaa, että
funktioevaluaatio $x \map f(x)$ on luotettava, kun muuttujalle sallitaan pieni vaihtelu, ts.\
pätee
\[
x_n\approx x \ \impl \ f(x_n)\approx f(x).
\]

Em.\ funktion tapauksessa jatkuvuuskysymys ratkeaa seuraavasti: Koska
\[
\begin{cases}
a^5+a=\pi, \\
a_n^5+a_n=x_n,
\end{cases}
\]
saadaan vähennyslaskulla (vrt. Luvun \ref{käänteisfunktio} tarkastelut)
\[
(a-a_n)(a^4+a^3a_n+a^2a_n^2+aa_n^3+a_n^4+3)=\pi-x_n.
\]
Tässä voidaan turvallisesti olettaa, että $a,a_n \ge 0$, joten seuraa
\[
\abs{a_n-a}\le \tfrac{1}{3}\abs{x_n-\pi}\,\ \ekv\,\ 
\abs{f(x_n)-f(\pi)}\le\tfrac{1}{3}\abs{x_n-\pi}.
\]
Jatkuvuudelle on näin saatu jopa kvantitatiivinen varmistus: Approksimaation $f(x_n)$ $\approx$
$f(\pi)$ virhe on enintään kolmasosa approksimaation $x_n\approx \pi$ virheestä. 

Esimerkki johdattaa seuraavaan funktion jatkuvuuden määritelmään (vaihtoehtoinen määritelmä 
jäljempänä).
\begin{Def} \vahv{(Jatkuvuus: jonokriteeri)}\ \label{funktion jatkuvuus}
\index{jatkuvuus (yhden muuttujan)|emph} \index{epzy@epäjatkuvuus (funktion)|emph}
Funktio $f:\DF_f\rightarrow\R,\ \DF_f\subset\R$, on \kor{jatkuva} (engl.\ continuous, ruots.\ 
kontinuerlig) \kor{pisteessä} $a\in\DF_f$ (tai $a$:ssa), jos kaikille reaalilukujonoille
$\seq{x_n}$ pätee
\[
x_n\in\DF_f\ \forall n\,\ \ja\,\ x_n \kohti a\,\ \impl\,\ f(x_n) \kohti f(a).
\]
Jos $f$ ei ole jatkuva pisteessä $a\in\DF_f$, niin $f$ on \kor{epäjatkuva}
(engl.\ discontinuous) pisteesä $a$.
\end{Def}
\begin{Exa} \label{epäjatkuva funktio} Funktio \
$\D f(x)= \begin{cases} \,x-1, &\text{jos}\ x<\pi \\ \,x, &\text{jos}\ x\ge\pi \end{cases}$
\vspace{1mm}\newline
on epäjatkuva pisteessä $x=\pi$, sillä jos $x_n \kohti \pi$ ja $x_n<\pi\ \forall n$, niin
$f(x_n)=x_n-1 \kohti \pi-1 \neq f(\pi)=\pi$. Muissa pisteissä $f$ on jatkuva, sillä jos esim.\
$a<\pi$ ja $x_n \kohti a$, niin jostakin indeksistä $N$ eteenpäin on $\abs{x_n-a}<\pi-a$ 
(lukujonon suppenemisen määritelmässä valittu $\eps=\pi-a$), jolloin on erityisesti 
$x_n-a < \pi-a\ \ekv\ x_n<\pi$, kun $n>N$. Tällöin on $f(x_n)=x_n-1,\ n>N$ ja siis
$f(x_n) \kohti a-1=f(a)$. \loppu
\end{Exa}
\begin{Exa} Olkoon $\DF_f=\{a\}$ ($a\in\R$) ja $f(a)=c\in\R$. Määritelmän
\ref{funktion jatkuvuus} mukaisesti $f$ on jatkuva pisteessä $a$\,(!). Yleisemmin
reaalifunktio on jatkuva jokaisessa määrittelyjoukkonsa nk.\ \kor{eristetyssä pisteessä},
ks.\ Harj.teht.\,\ref{H-V-1: eristetty piste}. \loppu
\end{Exa}
Jatkuvuus voidaan määritellä myös suoraan vetoamatta lukujonoihin, jolloin määritelmästä tulee 
lukujonon suppenemisen määritelmää (Määritelmä \ref{jonon raja}) muistuttava. Vaihtoehtoinen
määritelmä on seuraava.
\begin{Def} \vahv{(Jatkuvuus: $(\eps,\delta)$-kriteeri)}\ \label{vaihtoehtoinen jatkuvuus}
\index{jatkuvuus (yhden muuttujan)|emph}
Funktio $f:\DF_f\kohti\R,\ \DF_f\subset\R$, on jatkuva pisteessä $a\in\DF_f$, jos jokaisella
$\eps>0$ on olemassa $\delta>0$ siten, että jokaisella $x\in\R$ pätee
\[
x\in\DF_f\,\ \ja\,\ \abs{x-a}<\delta\,\ \impl\,\ \abs{f(x)-f(a)}<\eps.
\]
\end{Def}
Määritelmiä \ref{funktion jatkuvuus} ja \ref{vaihtoehtoinen jatkuvuus} verrattaessa ei näytä
aivan ilmeiseltä, että pätee (ks.\ todistus luvun lopussa)
\begin{*Lause} \label{jatkuvuuskriteerien yhtäpitävyys} Jatkuvuuden määritelmät
\ref{funktion jatkuvuus} ja \ref{vaihtoehtoinen jatkuvuus} ovat yhtäpitävät.
\end{*Lause} 
\jatko\jatko \begin{Exa} (jatko) Jos $a\neq\pi$, niin esimerkin funktiolle pätee
\[
\abs{f(x)-f(a)}=\abs{x-a}, \quad \text{kun}\ \abs{x-a}<\abs{\pi-a},
\]
joten Määritelmän \ref{vaihtoehtoinen jatkuvuus} ehto on voimassa, kun valitaan
$\delta=\min\{\eps,\abs{\pi-a}\}>0$. Siis $f$ on jatkuva pisteissä $a\neq\pi$ Määritelmän
\ref{vaihtoehtoinen jatkuvuus} mukaisesti. \loppu
\end{Exa} \seur
\begin{Exa} Näytä, että $f(x)=x^2$ on jatkuva jokaisessa pisteessä $a\in\R$
käyttäen jatkuvuuden a) jonokriteeriä,\, b) $(\eps,\delta)$-kriteeiä.
\end{Exa}
\ratk \ a) Määritelmän \ref{funktion jatkuvuus} mukainen jatkuvuus on seuraus Lauseesta
\ref{raja-arvojen yhdistelysäännöt}:
\[
x_n \kohti a \qimpl x_n^2 \kohti a^2\,\ \impl\,\ f(x_n) \kohti f(a).
\]
b) Jos $a\in\R$ ja $\abs{x-a} \le 1$, niin kunta-algebran ja kolmioepäyhtälön nojalla
\begin{align*}
\abs{f(x)-f(a)}\,=\,\abs{x-a}\abs{x+a}\,
                &=\,\abs{x-a}\abs{2a+(x-a)} \\
                &\le\,\abs{x-a}(2\abs{a}+\abs{x-a})\,
                 \le\,\abs{x-a}(2\abs{a}+1).
\end{align*}
Näin ollen jos $\eps>0$, niin pätee $\abs{f(x)-f(a)}<\eps$ aina kun
$\abs{x-a}<\delta=\min\{1,\eps/(2\abs{a}+1)\}$ (jolloin myös $\abs{x-a}<1$). Koska tässä on
$\delta>0$ aina kun $\eps>0$, niin $f$ on jatkuva $a$:ssa
Määritelmän \ref{vaihtoehtoinen jatkuvuus} mukaisesti. \loppu

Jatkuvuuden määrittely jonokriteerillä (Määritelmä \ref{funktion jatkuvuus}) on kätevää
monissa teoreettisissa tarkasteluissa, jotka tällä tavoin palautuvat suppenevien
lukujonojen teoriaan. (Tämä teoria on kylläkin tunnettava, mitä voi pitää myös
haittapuolena.) Määritelmä \ref{vaihtoehtoinen jatkuvuus} on jatkuvuuden perinteisempi
'koulumääritelmä'. Tämä on usein Määritelmää \ref{funktion jatkuvuus} selkeämpi silloin kun
halutaan selvittää, miltä jatkuvat funktiot 'näyttävät'. Esimerkiksi seuraava usein käytetty
tulos, joka kertoo jatkuvan funktion 'jäykkyydestä', on tästä määritelmästä helppo johtaa.
Todistus jätetään harjoitustehtäväksi (Harj.teht.\,\ref{H-V-1: todistuksia}a).
\begin{Prop} \label{jatkuvan funktion jäykkyys} Jos $f$ on jatkuva pisteessä $a$ ja $f(a)>0$
($f(a)<0$), niin $\exists\delta>0$ siten, että $f(x)>0$ ($f(x)<0$) aina kun 
$x\in(a-\delta,a+\delta) \cap \DF_f\,$.
\end{Prop}

Jatkuvuus yksittäisessä pisteessä voidaan laajentaa koskemaan joukkoa $A\subset\DF_f$\,:
Funktio on jatkuva $A$:ssa, jos se on jatkuva $A$:n jokaisessa pisteessä. 
Jatkuvuustarkasteluille on kuitenkin tyypillistä, että tarkastelu rajoittuu \pain{välille} 
$A\subset\DF_f$ siten, että funktion ominaisuuksilla välin ulkopuolella ei ole lainkaan
merkitystä. Sen vuoksi on tapana asettaa mainitusta yleisestä säännöstä hieman poikkeava
\begin{Def} (\vahv{Jatkuvuus välillä}) \label{jatkuvuus välillä}
\index{jatkuvuus (yhden muuttujan)!a@välillä|vahv}
Funktio $f:\DF_f\kohti\R,\ \DF_f\subset\R$, on \kor{jatkuva välillä} $A\subset\DF_f$, jos
jokaisella $x \in A$ ja jokaiselle reaalilukujonolle $\seq{x_n}$ pätee
\[
x_n \in A\ \ja\ x_n \kohti x\,\ \impl\,\ f(x_n) \kohti f(x).
\]
\end{Def}
Tämän mukaisesti jatkuvuus välillä $A$ tarkoittaa samaa kuin Määritelmän
\ref{funktion jatkuvuus} mukainen jatkuvuus $A$:n jokaisessa pisteessä siinä tapauksessa, että
funktion määrittelyjoukko rajataan väliksi $A$. (Jatkuvuuden vaihtoehtoisessa määritelmässä
\ref{vaihtoehtoinen jatkuvuus} korvataan ehto $\,x\in\DF_f\,$ ehdolla $x\in\DF_f \cap A$.)
Jatkuvuus välillä määritellään siis jatkuvuutena 'sisältä päin' ko.\ joukossa. Jos väli on
avoin, ei tämä rajaus ole tarpeen, sillä jos $x\in(a,b)=A$ ja $x_n \kohti x$, niin jollakin $N$
pätee $x_n \in A\ \forall n>N$ (Määritelmä \ref{jonon raja}, $\eps=\min\{x-a,b-x\}>0$) eli
määritelmän ehto toteutuu indeksistä $N$ eteenpäin joka tapauksessa. Sen sijaan jos väli on
suljettu, on ehdolla $x_n \in A$ merkitystä välin päätepisteissä (ei sisäpisteissä).
\jatko\jatko \jatko \begin{Exa} (jatko) Jos $b>\pi$, on esimerkin funktio $f$ jatkuva välillä
$[\pi,b]$ (Määritelmä \ref{jatkuvuus välillä}) vaikka $f$ ei ole jatkuva pisteessä $\pi$
(Määritelmä \ref{funktion jatkuvuus}). Jos $a<\pi$, on $f$ jatkuva välillä $[a,\pi)$ mutta ei
välillä $(a,\pi]$. \loppu
\end{Exa} \seur\seur
\begin{Exa} \label{Dirichlet'n funktio} \index{Dirichleta@Dirichlet'n funktio}
Funktio $\D
f(x)=\begin{cases}
     \,1, \ &\text{ jos } x\in\Q,  \\
     \,0, &\text{ jos } x\in\R, \ x\notin\Q
     \end{cases}$ \vspace{2mm}\newline
on esimerkki 'patologisesta' funktiosta, joka on määritelty koko $\R$:ssä, mutta joka ei ole 
missään pisteessä jatkuva. Funktion nimi on \kor{Dirichlet'n funktio}.  \loppu
\end{Exa}

\subsection{Jatkuvien funktioiden yhdistely}

Esimerkin \ref{Dirichlet'n funktio} vastapainoksi voidaan kysyä, millaiset 'normaalit' funktiot 
tyypillisesti ovat jatkuvia. Ensimmäinen tuntuma näihin saadaan yhdistelemällä yksinkertaisia 
funktioita peruslaskutoimituksilla Määritelmän \ref{funktioiden yhdistelysäännöt} mukaisesti.
Koska funktion jatkuvuudessa on viime kädessä kyse lukujonon $\seq{f(x_n)}$ suppenemisesta,
on seuraava lause välitön seuraus Lauseesta \ref{raja-arvojen yhdistelysäännöt}
(Harj.teht.\,\ref{H-V-1: todistuksia}b).
\begin{Lause} (\vahv{Jatkuvuuden yhdistelysäännöt}) \label{jatkuvuuden yhdistelysäännöt}
\index{jatkuvuus (yhden muuttujan)!b@yhdistelysäännöt|emph}
Jos $f:\DF_f\kohti\R$ on jatkuva pisteessä $x\in\DF_f$, niin $\lambda f$ on jatkuva $x$:ssä 
$\forall\lambda\in\R$. Jos lisäksi $x\in\DF_g$ ja $g:\DF_g\kohti\R$ on jatkuva pisteessä $x$,
niin $f+g$ ja $fg$ ovat jatkuvia $x$:ssä. Jos edelleen $g(x)\neq 0$, niin myös $f/g$ on
jatkuva $x$:ssä.
\end{Lause}
\begin{Exa} Jokainen polynomi tai rationaalifunktio voidaan määritellä palautuvasti
algebrallisena yhdistelmänä perusfunktioista $f_0(x)=1$ ja $f_1(x)=x$. Koska nämä ovat
ilmeisen jatkuvia $\R$:ssä, niin päätellään Lauseesta \ref{jatkuvuuden yhdistelysäännöt},
että jokainen polynomi on jatkuva $\R$:ssä ja jokainen rationaalifunktio
määrittelyjoukossaan, eli muualla kuin nimittäjänsä nollakohdissa. \loppu 
\end{Exa}
\begin{Exa} \label{trig yhdistely} Jos pidetään tunnettuna, että trigonometriset funktiot
$\sin$ ja $\cos$ ovat jatkuvia $\R$:ssä, niin Lauseen \ref{jatkuvuuden yhdistelysäännöt}
perusteella funktiot $\tan=\sin/\cos$ ja $\cot=\cos/\sin$ ovat samoin jatkuvia koko
määrittelyjoukossaan. \loppu 
\end{Exa}
\begin{Lause} (\vahv{Yhdistetyn funktion jatkuvuus}) \label{yhdistetyn funktion jatkuvuus}
\index{jatkuvuus (yhden muuttujan)!c@yhdistetyn funktion|emph}
Jos $g$ on jatkuva pisteessä $x\in\DF_g$, $g(x)\in\DF_f$ ja $f$ on jatkuva pisteessä $g(x)$,
niin yhdistetty funktio $f\circ g$ on jatkuva $x$:ssä.
\end{Lause}
\tod Koska $\,x_n\in\DF_{f \circ g}\ \ekv\ x_n\in\DF_g\,\wedge\,g(x_n)\in\DF_f$, niin
jonokriteerin avulla päätellään:
\begin{align*}
x_n \in\DF_{f\circ g}\ \ja\ x_n \kohti x 
           &\qimpl g(x_n)\in\DF_f\ \ja\ x_n\in\DF_g\ \ja\ x_n \kohti x \\
           &\qimpl g(x_n)\in\DF_f\ \ja\ g(x_n) \kohti g(x) \\
           &\qimpl f(g(x_n)) \kohti f(g(x)). \quad\loppu
\end{align*}
\begin{Exa} \label{itseisarvon jatkuvuus} Näytä, että jos $f$ on jatkuva $x$:ssä, niin myös
$\abs{f}$ on jatkuva $x$:ssä. 
\end{Exa}
\ratk Funktio $g(x)=\abs{x}$ on helposti osoitettavissa jatkuvaksi $\R$:ssä suoraan
jatkuvuuden määritelmistä (tai vedoten Propositioon \ref{jatkuvuuspropositio} alla), joten
Lauseen \ref{yhdistetyn funktion jatkuvuus} nojalla yhdistetty funktio 
$(g \circ f)(x) = \abs{f(x)}$ on jatkuva jokaisessa pisteessä, jossa $f$ on jatkuva. \loppu

Jatkuvuuden periytyvyyden p\pain{aloittain} (eli eri väleillä erilaisilla laskusäännöillä)
määritellyn funktion osalta ratkaisee
(Harj.teht.\,\ref{H-V-1: todistuksia}c; vrt.\ myös Esimerkki \ref{epäjatkuva funktio} edellä).
\begin{Prop} \label{jatkuvuuspropositio}
Olkoon $f_1$ ja $f_2$ jatkuvia avoimella välillä $(a,b)$ ja
\[
f(x) = \begin{cases} 
       \,f_1(x), &\text{kun}\ x \in (a,c), \\ 
       \,k, &\text{kun}\ x=c, \\ 
       \,f_2(x), &\text{kun}\ x \in (c,b),
       \end{cases} 
\]
missä $a<c<b$. Tällöin $f$ on jatkuva $c$:ssä täsmälleen kun $\,f_1(c)=f_2(c)=k$. 
\end{Prop}
\begin{Exa} Funktio
\[ f(x) = \begin{cases}
          \ x+a,\ &\text{kun}\ x \le a, \\ -(x+1)^2,\ &\text{kun}\ x > a 
          \end{cases} 
\] 
on polynomi $f_1(x)=x+a$ välillä $(-\infty,a)$ ja polynomi $f_2(x)=-(x+1)^2$ välillä
$(a,\infty)$, joten näillä väleillä funktio on jatkuva. Proposition \ref{jatkuvuuspropositio}
mukaan ehto funktion jatkuvuudelle pisteessä $x=a$ (ja siis ehto jatkuvuudelle koko $\R$:ssä)
on
\[
f_1(a)=f_2(a) \qekv a^2+4a+1=0 \qekv a = -2 \pm \sqrt{3}. \loppu
\]
\end{Exa}

\subsection{Funktio $f(x) = \sqrt[m]{x}$}
\index{jatkuvuus (yhden muuttujan)!d@funktion $\sqrt[m]{x}$|vahv}

Osoitetaan seuraavaksi, että funktio $f:[0,\infty)\rightarrow [0,\infty)$, joka määritellään
\[
f(x)=x^{1/m}=\sqrt[m]{x},\quad m\in\N, \ m\geq 2
\]
on koko määrittelyvälillään jatkuva (kuvassa $f(x)$, kun $m=8$).
\begin{figure}[H]
\begin{center}
\epsfig{file=plots/8throotx.eps}
%\caption{$y=\sqrt[8]{x}$}
\end{center}
\end{figure}
Aloitetaan pisteestä $x=0$, joka on jatkuvuuden kannalta kriittisin (vrt. kuva). Koska f on 
aidosti kasvava (ks.\ Luku \ref{käänteisfunktio}, Esimerkki \ref{x^m:n käänteisfunktio}),
niin jokaisella $a>0$ pätee
\[
0 \le x < a^m\ \ekv\ 0 \le f(x) < f(a^m) = a.
\]
Jos nyt $x_n\geq 0$ ja $x_n\rightarrow 0$, niin jokaisella $\eps>0$ $\exists N\in\N$ siten,
että pätee
\[
n>N \ \impl \ 0 \le x_n < \eps^m \ \impl\ 0 \le f(x_n) < \eps.
\]
Näin ollen lukujono $\{f(x_n)\}$ on suppeneva ja $f(x_n) \kohti 0=f(0)$, joten $f$ on jatkuva 
$0$:ssa.

Muualla kuin origossa $f$:n jatkuvuus voidaan päätellä kunta-algebran avulla: Jos 
$x_n \ge 0$ ja $x_n \kohti x>0$, niin merkitsemällä $y=f(x)$, $y_n=f(x_n)$ seuraa
\[
x-x_n \,=\, y^m-y_n^m \,=\, (y-y_n)(y^{m-1}+y^{m-1}y_n+\cdots +y_n^{m-1}).
\]
Tässä on $y_n \ge 0$, $y>0$, joten päätellään
\[
\abs{x-x_n}\,\ge\,\abs{y-y_n}y^{m-1} \,\ \ekv\,\ \abs{y-y_n}\,\le\,y^{1-m}\abs{x-x_n}.
\]
Näin ollen $x_n \kohti x \ \impl\ y_n \kohti y$, eli $f$ on jatkuva pisteessä $x$.
\begin{Exa} \label{epsilon ja delta} Olkoon $\eps = 10^{-6}$. Määritä suurin $\delta$ siten,
että Määritelmän \ref{vaihtoehtoinen jatkuvuus} ehto toteutuu funktiolle $f(x)=\sqrt[8]{x}$
pisteessä $x=0$.
\end{Exa}
\ratk Koska $f$ on aidosti kasvava välillä $[0,\infty)=\DF_f$, niin $\forall x\in\DF_f$ pätee
$|f(x)-f(0)|=\sqrt[8]{x} < \eps\ \ekv\ 0 \le x < \eps^8\ \ekv\ x\in(-\eps^8,\eps^8)\cap\DF_f$.
Siis $\,\delta_{max} = \eps^8 = 10^{-48}$. \loppu

Funktion $f(x)=x^{1/m}$ jatkuvuuden tultua todetuksi seuraa Lauseesta 
\ref{yhdistetyn funktion jatkuvuus} ja funktion $g(x)=\abs{x}$ jatkuvuudesta, että yhdistetty 
funktio $(f\circ g)(x) = \abs{x}^{1/m}$ on jatkuva koko määrittelyjoukossaan ($= \R$). Kun
huomioidaan myös Lause \ref{jatkuvuuden yhdistelysäännöt}, niin seuraa
\begin{Prop} \label{potenssifunktion jatkuvuus}
Funktio $f(x)=\abs{x}^\alpha$, $\alpha=p/q\in\Q$ on koko määrittelyjoukossaan ($\DF_f=\R$ jos
$\alpha>0$, $\DF_f=\{x\in\R \mid x \neq 0\}$ jos $\alpha \leq 0$) jatkuva.
\end{Prop}
Lauseiden \ref{jatkuvuuden yhdistelysäännöt}--\ref{yhdistetyn funktion jatkuvuus}, Proposition
\ref{potenssifunktion jatkuvuus} ja Esimerkkien \ref{trig yhdistely} ja
\ref{itseisarvon jatkuvuus} perusteella voidaan vetää se yleisempi johtopäätös, että kaikki
toistaiseksi tunnetut 'yhden lausekkeen' funktiot ovat jatkuvia koko määrittelyjoukossaan (!).
\begin{Exa} Ilman tapauskohtaista tarkastelua voidaan esimerkiksi seuraavat funktiot
päätellä jatkuviksi määrittelyjoukkonsa jokaisessa pisteessä:
\[
\sqrt[3]{\abs{1-\sqrt{x}\,}}\,, \quad 
\sqrt[6]{\frac{1+\sqrt{x}}{2-x^2}}\,, \quad
\frac{|\sin x|^{3/4}}{x-\pi}\,, \quad 
\frac{\cot x}{\sqrt[4]{x+50\cos(\tan x)}}\,. \loppu
\]
\end{Exa}

\subsection{Jatkuvien funktioiden päälauseet}

Tässä osaluvussa esitetään kolme matemaattisen analyysin keskeistä lausetta, jotka kaikki
koskevat suljetulla välillä jatkuvia funktioita. Ensimmäinen lauseista on myös ensimmäinen
yhden reaalimuuttujan funktioita koskevista \kor{väliarvolauseista}, joita on kaikkiaan kolme.
(Muut kaksi esitetään myöhemmin.) Tässä esitettävistä päälauseista jälkimmäiset kaksi ovat
erikoistapauksia yleisemmistä väittämistä, jotka perustuvat jatkuvuuden syvällisempään
logiikkaan. Näiden todistuksia ei vielä esitetä, vaan lauseet muotoillaan ja todistetaan
jäljempänä Luvussa \ref{jatkuvuuden logiikka} tässä esitettyä yleisemmässä muodossa. 
 
Tarkastellaan suljetulla välillä $[a,b]$ jatkuvaa funktiota $f$. Olkoon $f(a) \neq f(b)$ ja 
$c < \eta < d$, missä
\[ 
c = \min \{f(a),f(b)\}, \quad d = \max\{f(a),f(b)\}. 
\]
Koska $f$ on jatkuva, niin tuntuu luonnolliselta ajatella, että $f$:n kuvaaja välillä $[a,b]$
on 'jatkuva lanka', joka yhdistää pisteet $A=(a,f(a))$ ja $B=(b,f(b))$. Tämän intuition
mukaisesti tuntuu selvältä, että kuvaajan on leikattava suora $y=\eta$ ainakin kerran välillä
$(a,b)$. Toisin sanoen, probleemalla
\[ 
x \in [a,b]: \quad f(x) = \eta 
\]
on oltava ainakin yksi ratkaisu $x=\xi \in (a,b)$ jokaisella $\eta \in (c,d)$ (kuva).
\begin{figure}[H]
\setlength{\unitlength}{1cm}
\begin{center}
\begin{picture}(10,6)(-1,-1)
\put(-1,0){\vector(1,0){10}} \put(8.8,-0.4){$x$}
\put(0,-1){\vector(0,1){6}} \put(0.2,4.8){$y$}
\curve(
    1.0000,    1.0000,
    1.5000,    2.7969,
    2.0000,    3.4643,
    2.5000,    3.4174,
    3.0000,    3.0000,
    3.5000,    2.4844,
    4.0000,    2.0714,
    4.5000,    1.8906,
    5.0000,    2.0000,
    5.5000,    2.3862,
    6.0000,    2.9643,
    6.5000,    3.5781,
    7.0000,    4.0000,
    7.5000,    3.9308,
    8.0000,    3.0000)
\put(0.9,0.9){$\bullet$} \put(0.5,0.6){$A$}
\put(7.9,2.9){$\bullet$} \put(8.2,3.1){$B$}
\dashline{0.2}(5.5,0)(5.5,2.39)(0,2.39)
\dashline{0.05}(1,0)(1,1)(0,1)
\dashline{0.05}(8,0)(8,3)(0,3)
\put(0.9,-0.5){$a$} \put(5.4,-0.5){$\xi$} \put(7.9,-0.5){$b$}
\put(-0.4,0.9){$c$} \put(-0.4,2.3){$\eta$} \put(-0.4,2.9){$d$}
\end{picture}
%\caption{Väliarvolauseen väittämän havainnollistaminen}
\end{center}
\end{figure}
Ym.\ geometriselle intuitiolle vahvistuksen antaa
%\begin{figure}[H]
%\setlength{\unitlength}{1mm}
%\begin{center}
%\begin{picture}(140,45)(0,10)
%\put(20,20){\vector(1,0){40}} \put(58,16){$x$}
%\put(20,20){\vector(0,1){30}} \put(22,48){$y$}
%\put(80,20){\vector(1,0){40}} \put(118,16){$x$}
%\put(80,20){\vector(0,1){30}} \put(82,48){$y$} 
%\curve(25,25,45,32,55,45)
%%\curve(85,45,95,32,115,25)
%%\curve(25,15,45,50,55,45)
%\curve(85,55,95,20,115,32)
%\dashline{1}[0.2](30,20)(30,26)(20,26)
%\dashline{1}[0.2](53,20)(53,41)(20,41)
%\dashline{1}[0.2](87,20)(87,40)(80,40)
%\dashline{1}[0.2](110,20)(110,26)(80,26)
%\put(29,16){$a$} \put(16,25){$c$} \put(52,16){$b$} \put(16,40){$d$}
%\put(86,16){$a$} \put(76,39){$c$} \put(109,16){$b$} \put(76,25){$d$}
%\end{picture}
%\end{center}
%\end{figure}
\begin{Lause} \label{ensimmäinen väliarvolause}
\vahv{(Ensimmäinen väliarvolause --- Jatkuvien funktioiden väliarvolause)}
\index{vzy@väliarvolauseet!a@jatkuvien funktioiden|emph}
Jos $f:\DF_f\kohti\R$, $\DF_f\subset\R$, on jatkuva välillä $[a,b]\subset\DF_f$ ja 
$c=\min\{f(a),f(b)\}<d=\max\{f(a),f(b)\}$ (eli $f(a) \neq f(b)$), niin jokaisella 
$\eta\in (c,d)$ on olemassa $\xi \in (a,b)$ siten, että $f(\xi)=\eta$.\footnote[2]{Lause
\ref{ensimmäinen väliarvolause} varmistaa, että välillä $[a,b]$ jatkuvan funktion kuvaaja 
\[
G_f=\{P=(x,y) \mid x\in[a,b]\,\ja\,y=f(x)\}
\]
on geometrisen intuition mukainen \kor{käyrä} eli 'yhtenäinen viiva vailla leveyttä'. $G_f$ on
'vailla leveyttä', koska $f(x)$ on yksikäsitteinen $\forall x$ (eli $f$ on funktio), ja
'yhtenäinen', koska $f$ on jatkuva. --- Jatkuvuus (yhtenäisyyden takaajana) liitetään käyrän
käsitteeseen yleisemminkin. Esimerkiksi yleisessä parametrisessa käyrässä
$\vec r(t)=x(t)\vec i+y(t)\vec j+z(t)\vec k$ koordinaattifunktiot $x(t),y(t),z(t)$ oletetaan
jatkuviksi tarkasteltavalla välillä. Vrt.\ alaviitteet Luvuissa
\ref{käänteisfunktio}--\ref{parametriset käyrät}. \index{kzyyrzy@käyrä|av}} 
\end{Lause}
Jos $f$ ja $\eta$ täyttävät Lauseen \ref{ensimmäinen väliarvolause} ehdot, niin nähdään, että 
lauseen väittämä seuraa, kun seuraavaa pelkistetympää tulosta sovelletaan funktioon 
$g(x)=f(x)-\eta$ tai $g(x)=\eta-f(x)$.
\begin{Lause} \vahv{(Bolzano)} \label{Bolzanon lause} \index{Bolzanon lause|emph}
Jos $f(a)<0$, $f(b)>0$, ja $f$ on jatkuva välillä $[a,b]$, niin $f(\xi)=0$ jollakin
$\xi\in (a,b)$.
\end{Lause}
\tod on puhtaasti konstruktiivinen ja toimii käytännössäkin algoritmina, joka 
etsii yhden $f$:n nollakohdista välillä $(a,b)$. Tavallisin todistustapa on nk.\  
\pain{haarukointi} eli puolitusmenetelmä (kymmenjakoalgoritmikin toimisi, vrt.\ Luku
\ref{reaaliluvut}). Ensinnäkin voidaan funktioevaluaatioiden (oletetaan tarkoiksi\,!)
perusteella todeta, että joko valinnalla $a_1=a$, $b_1=\frac{1}{2}(a+b)$ tai valinnalla 
$a_1=\frac{1}{2}(a+b)$, $b_1=b$ pätee
\[
f(a_1) \le 0 \ \ja \ f(b_1) \ge 0.
\]
Jos tässä on $f(a_1)=0$ tai $f(b_1)=0$, on $\xi$ löydetty. Muussa tapauksessa on $f(a_1)<0$ ja
$f(b_1)>0$, jolloin voidaan jatkaa $\xi$:n 'haarukointia' välillä $(a_1,b_1)$ samalla
periaatteella. Mikäli konstruktio ei katkea $\xi$:n löytymiseen, syntyy kaksi jonoa
$\seq{a_n}$, $\seq{b_n}$. Konstruktion  perusteella $\seq{a_n}$ on kasvava, $\seq{b_n}$ on
vähenevä ja
\[
b_n-a_n=\left(\frac{1}{2}\right)^n(b-a) \kohti 0,
\]
joten jonoilla on yhteinen raja-arvo: $a_n\kohti\xi$ ja $b_n\kohti\xi$ jollakin $\xi\in\R$.
Koska on $a \le a_n \le b\ \forall n$, on myös oltava $a \le \xi \le b$ 
(Propositio \ref{jonotuloksia}:\,V1), eli $\xi \in [a,b]$. Tällöin koska $f$ on välillä $[a,b]$ 
jatkuva (Määritelmä \ref{jatkuvuus välillä}), pätee
\[
f(a_n) \kohti f(\xi)\ \ \text{ja}\ \ f(b_n) \kohti f(\xi).
\]
Mutta lukujonojen $\seq{a_n}$ ja $\seq{b_n}$ konstruktion perusteella
\begin{align*}
f(a_n)<0\,\ \forall n \ &\impl \ f(\xi)\le 0, \\
f(b_n)>0\,\ \forall n \ &\impl \ f(\xi)\ge 0
\end{align*}
(Propositio \ref{jonotuloksia}:\,V1), joten on oltava $f(\xi)=0$. Tässä oli $\xi\in [a,b]$,
mutta koska $f(a)<0$ ja $f(b)>0$, niin $\xi \in (a,b)$. Bolzanon lause on näin todistettu.
\loppu
\begin{Exa}
Funktiolle $f(x)=x^5+3x\,$ pätee $f(0)=0$ ja $f(1)=4$, joten Lauseen 
\ref{ensimmäinen väliarvolause} mukaan yhtälöllä
\[ 
x^5 + 3x = 2 
\]
on ratkaisu $x=\xi$ välillä $(0,1)$. Ratkaisu on itse asiassa yksikäsitteinen, koska 
$f$ on 1-1 (vrt.\ Esimerkki \ref{algebrallinen käänteisfunktio} Luvussa \ref{käänteisfunktio}).
\loppu \end{Exa}

Seuraava lause, jolla on myös yleisempiä muotoja, on reaalianalyysin merkittävimpiä tuloksia
(todistus yleisemmässä muodossa Luvussa \ref{jatkuvuuden logiikka}).
\begin{*Lause} \vahv{(Weierstrass)} \label{Weierstrassin peruslause}
\index{Weierstrassin lause|emph} 
Suljetulla välillä $[a,b]$ jatkuva funktio $f$ saavuttaa ko.\ välillä pienimmän ja suurimman
arvonsa, ts.\ on olemassa $x_1 \in [a,b]$ ja $x_2 \in [a,b]$ siten, että
\[ 
f(x_1) = \min\{f(x) \mid x \in [a,b]\}, \quad f(x_2) = \max\{f(x) \mid x \in [a,b]\}. 
\]
\end{*Lause}
\begin{Exa} Funktio
\[
f(x)=\begin{cases} \,x, &\text{kun}\ x \le 0, \\ \,1-x, &\text{kun}\ x>0 \end{cases}
\]
on jatkuva välillä $[a,b]$, jos joko $a < b \le 0$ tai $0<a<b$. Tällöin funktio saavuttaa
maksimi- ja minimiarvonsa välin päätepisteissä. Muussa tapauksessa, eli jos $a \le 0$ ja
$b>0$, ei $f$ ole jatkuva välillä $[a,b]$, sillä jos $x_n\in(0,b]$ ja $x_n \kohti 0$, niin
$f(x_n) \kohti 1 \neq f(0)$. Tässä tapauksessa $f$ ei myöskään saavuta maksimiarvoa
välillä $[a,b]$, ainoastaan minimiarvon ($=\min\{f(a),f(b)\}$). \loppu
\end{Exa} 
Yhdistämällä Lauseiden \ref{ensimmäinen väliarvolause} ja \ref{Weierstrassin peruslause} 
väittämät tullaan seuraavaan maininnan arvoiseen päätelmään:
\begin{Kor} \label{eräs jatkuvuuskorollaari} Jos $f$ on jatkuva välillä $A=[a,b]$ ja $f$ ei
ole vakio, niin $f(A)$ on suljettu väli. \end{Kor}

Seuraava lause antaa erään riittävän kriteerin käänteisfunktion jatkuvuudelle. Todistus
(hieman yleisemmin muotoillulle lauseelle) esitetään Luvussa \ref{jatkuvuuden logiikka}.
\begin{*Lause} \label{käänteisfunktion jatkuvuus}
\index{jatkuvuus (yhden muuttujan)!e@käänteisfunktion|emph}
\vahv{(Käänteisfunktion jatkuvuus)} Jos $f:[a,b] \kohti [c,d]$ on jatkuva bijektio, niin myös 
$\inv{f}:[c,d] \kohti [a,b]$ on jatkuva bijektio.
\end{*Lause} 
\begin{Exa} Funktio $f(x)=x^{1/m},\ m \in \N,\ m \ge 2\,$ pääteltiin edellä jatkuvaksi välillä 
$[0,\infty)$. Tämä seuraa myös Lauseesta \ref{käänteisfunktion jatkuvuus}, sillä $f=\inv{g}$,
missä $g(x)=x^m$ on jatkuva bijektio $\,g: [0,a] \kohti [0,a^m]$ jokaisella $a>0$. \loppu
\end{Exa}

\subsection{*Lauseen \ref{jatkuvuuskriteerien yhtäpitävyys} todistus}

Lauseen \ref{jatkuvuuskriteerien yhtäpitävyys} väittämä on kaksiosainen: Väitetään, että
jos $f$ on jatkuva pisteessä $a\in\DF_f$ Määritelmän \ref{funktion jatkuvuus}
jonokriteerillä, niin $f$ on jatkuva $a$:ssa myös Määritelmän
\ref{vaihtoehtoinen jatkuvuus} $(\eps,\delta)$-kriteerillä ja että tämä
implikaatio pätee myös kääntäen.
 
\tod\, \fbox{$\impl$} Oletetaan, että $f$ on jatkuva pisteessä $a\in\DF_f$  Määritelmän 
\ref{funktion jatkuvuus} mukaisesti. Tehdään vastaoletus: $f$ ei ole jatkuva $a$:ssa
Määritelmän \ref{vaihtoehtoinen jatkuvuus} kriteerillä. Tällöin voidaan
samalla tavoin kuin Lauseen \ref{negaatioperiaate} todistuksessa päätellä, että jollakin 
$\eps>0$ pätee:
\[
\forall \delta>0 \ \exists x\in\DF_f\ \text{siten, että}\ \ 
                   \abs{x-a}<\delta \ \ja\ \abs{f(x)-f(a)}\ge\eps.
\]
Kun nyt $\delta$:n arvoista muodostetaan jono $\{\delta_n\}$, joka suppenee kohti nollaa 
(esim.\ $\delta_n=1/n$), niin vastaavasti on siis löydettävissä jono $\{x_n\}$ siten, että
jokaisella $n$
\[
x_n\in\DF_f\ \ja\ \abs{x_n-a}<\delta_n \ \ja \ \abs{f(x_n)-f(a)}\ge\eps.
\]
Näin on löydetty jono, jolle pätee
\[
x_n\in\DF_f\ \ja\ x_n \kohti a\ \ja\ f(x_n) \not\kohti f(a).
\]
Tämä on looginen ristiriita, koska oletettiin, että $f$ on jatkuva $a$:ssa Määritelmän
\ref{funktion jatkuvuus} mukaisesti. Tehty vastaoletus on siis väärä, eli $f$ on jatkuva
$a$:ssa Määritelmän \ref{vaihtoehtoinen jatkuvuus} kriteerillä. 

\fbox{$\Leftarrow$} Oletetaan, että $f$ on jatkuva pisteessä $a\in\DF_f$ Määritelmän
\ref{vaihtoehtoinen jatkuvuus} kriteerillä. Olkoon $\eps>0$ ja valitaan $\delta>0$ siten, että
Määritelmän \ref{vaihtoehtoinen jatkuvuus} ehto on täytetty. Edelleen olkoon $\seq{x_n}$
lukujono, jolle pätee $x_n\in\DF_f\ \forall n$ ja $x_n \kohti a$. Tällöin jostakin indeksistä
$n=N$ eteenpäin on $\abs{x_n-a}<\delta$ (koska oli $\delta>0$), jolloin tehtyjen oletusten
mukaan on myös $\abs{f(x_n)-f(a)}<\eps$. On siis löydetty indeksi $N\in\N$ siten, että pätee
\[
\abs{f(x_n)-f(a)} < \eps, \quad \text{kun}\ n>N.
\]
Tässä $\eps>0$ oli mielivaltainen, joten $f(x_n) \kohti f(a)$ (Määritelmä \ref{jonon raja}).
Myös jono $\seq{x_n}$ oli tehtyjen oletusten puitteissa vapaasti valittu, joten jokaiselle 
tällaiselle jonolle on voimassa $f(x_n) \kohti f(a)$. Siis $f$ on jatkuva pisteessä $a$
Määritelmän \ref{funktion jatkuvuus} kriteerillä. \loppu

\Harj
\begin{enumerate}

\item
Todista suoraan Määritelmän \ref{funktion jatkuvuus} ja lukujonoja koskevien tulosten
perusteella kukin seuraavista funktioista joko jatkuvaksi tai epäjatkuvaksi pisteessä $x=0$.
Hahmottele myös funktioiden kuvaajat joukossa $A=\DF_f\cap[-1,1]$.
\begin{align*}
&\text{a)}\,\ f(x)=x^7+7x+49 \qquad 
 \text{b)}\,\ f(x)=x^5-6x^3+\frac{x^3+20}{100x^2-1} \\
&\text{c)}\,\ f(x)=\begin{cases} 
                   \,\dfrac{1}{x}\left(\dfrac{1}{1+7x}-\dfrac{1}{1+5x}\right), 
                                                       &\text{kun}\ x \neq 0, \\[2mm]
                   \,2,                                &\text{kun}\ x=0
                   \end{cases} \\
&\text{d)}\,\ f(x)=\begin{cases}
                   \,x\cos\dfrac{1}{x}\,, &\text{kun}\ x \neq 0, \\[2mm]
                   \,0,                   &\text{kun}\ x=0
                   \end{cases} \qquad
 \text{e)}\,\ f(x)=\begin{cases}
                   \,\sin\dfrac{1}{x}\,, &\text{kun}\ x \neq 0, \\[2mm]
                   \,0,                  &\text{kun}\ x=0
                   \end{cases}
\end{align*}

\item \label{abs{x}:n jatkuvuus}
Näytä funktio $f(x)=1/x$ jatkuvaksi määrittelyjoukossaan vedoten suoraan a) jatkuvuuden
jonokriteeriin, b) jatkuvuuden $(\eps,\delta)$-kriteeriin.

\item \label{H-V-1: todistuksia}
Käyttäen a)- kohdassa jatkuvuuden $(\eps,\delta)$-kriteeriä, b)-kohdassa jonokriteeriä
ja c)-kohdassa jompaa kumpaa todista: \vspace{1mm}\newline
a) Propositio \ref{jatkuvan funktion jäykkyys}. $\quad$
b) Lause \ref{jatkuvuuden yhdistelysäännöt}. $\quad$
c) Propositio \ref{jatkuvuuspropositio}.

\item \label{H-V-1: eristetty piste} \index{eristetty piste (joukon)}
Sanotaan, että $a \in A$ on joukon $A\subset\R$ \kor{eristetty piste}, jos jollakin
$\delta>0$ on $(a-\delta,a+\delta) \cap A = \{a\}$. Näytä, että reaalifunktio on jatkuva
jokaisessa määrittelyjoukkonsa eristetyssä pisteessä.

\item
Hae sellaiset parametrin $a$ arvot, joilla funktiosta tulee jatkuva koko $\R$:ssä, ja
hahmottele kuvaaja näillä $a$:n arvoilla:
\begin{align*}
&\text{a)}\,\ f(x)=\begin{cases}
                   \,x-1, &\text{kun}\ x \le a, \\ \,1-x^2, &\text{kun}\ x>a
                   \end{cases} \qquad
 \text{b)}\,\ f(x)=\begin{cases}
                   \,a-x, &\text{kun}\ x \le -2, \\ \,x^3-4x, &\text{kun}\ x>-2
                   \end{cases} \\
&\text{c)}\,\ f(x)=\begin{cases}
                   \,\dfrac{x^2-4}{x^3-8}\,, &\text{kun}\ x<2, \\[2mm]\,a,&\text{kun}\ x \ge 2
                   \end{cases} \quad\ \
 \text{d)}\,\ f(x)=\begin{cases}
                   \,\sin x, &\text{kun}\ x \le \pi/3, \\ \,\cos ax, &\text{kun}\ x>\pi/3
                   \end{cases} \\
&\text{e)}\,\ f(x)=\begin{cases}
                   \,\dfrac{x^3+4x^2+5x+2}{x^2+3x+2}\,, &\text{kun}\ x<a, \\[2mm] 
                   \,ax-10,                             &\text{kun}\ x \ge a
                   \end{cases}
\end{align*}

\item
Totea Lauseiden \ref{jatkuvuuden yhdistelysäännöt}--\ref{yhdistetyn funktion jatkuvuus} ja 
Proposition \ref{potenssifunktion jatkuvuus} perusteella seuraavat funktiot jatkuviksi koko 
määrittelyjoukossaan. --- Mikä määrittelyjoukko on?
\[
\text{a)}\,\ f(x)=\sqrt{1+\frac{1}{2-x}} \qquad 
\text{b)}\,\ f(x)=\sqrt[4]{1-\frac{1}{\sqrt{2-x}}}
\]

\item
Funktio $f(x)=\sin x$ on aidosti kasvava välillä $[-\frac{\pi}{2},\frac{\pi}{2}]$, aidosti
vähenevä välillä $[\frac{\pi}{2},\frac{3\pi}{2}]$ ja jatkuva koko $\R$:ssä. Täsmälleen montako
ratkaisua yhtälöllä $\,\sin x=y\in[-1,1]$ on näiden tietojen ja Lauseen 
\ref{ensimmäinen väliarvolause} perusteella välillä $[-\frac{\pi}{2}\,,\frac{\pi}{6}]$ eri
$y$:n arvoilla?

\item
a) Transkendenttisella yhtälöllä $x=\cos x$ on yksikäsitteinen ratkaisu välillä $[0,1]$.
Seuraten Bolzanon lauseen todistuskonstruktiota laske ratkaisulle lähin ala- ja ylälikiarvo 
äärellisinä binaarilukuina muotoa $p/64,\ p\in\N$. \vspace{1mm}\newline
b) Montako nollakohtaa on funktiolla $f(x)=x-\cos 49x$\,? Jos nollakohtaa etsitään väliltä
$[0,1]$ Bolzanon lauseen todistuksessa käytetyllä algoritmilla, niin mikä nollakohdista
löytyy?

\pagebreak

\item
a) Näytä, että polynomin $p(x)=x^3-15x+1$ kaikki juuret ovat reaalisia ja sijaitsevat
välillä $[-4,4]$. \vspace{1mm}\newline
b) Näytä, että polynomi $f(x)=(x-a)^2(x-b)^2+x$ saa jossakin pisteessä arvon $(a+b)/2$.

\item
Näytä, että jos Bolzanon lauseen todistuskonstruktiossa oletetaan ainoastaan, että $f$ on
määritelty välillä $[a,b]$ ja $f(a)<0,\ f(b)>0$ (eli ei oleteta jatkuvuutta), niin
konstruktio tuottaa tässä tapauksessa pisteen $\xi\in[a,b]$, jossa on joko $f(\xi)=0$ tai
--- toinen vaihtoehto? Anna esimerkki funktiosta, jolle konstruktion tulos on $\xi=a$.

\item \label{H-V-1: Weierstrassin seuraus}
Todista Lauseen \ref{ensimmäinen väliarvolause} ja Weierstrassin lauseen avulla väittämä:
Jos $f$ on jatkuva välillä $[a,b]$ ja $f(x) \neq 0\ \forall x\in[a,b]$, niin $\exists c>0$
siten, että joko $f(x) \ge c\ \forall x\in[a,b]$ tai $f(x) \le -c\ \forall x\in[a,b]$.

\item (*) \label{H-V-1: pelkistyvä funktio}
Funktio $f$ on jatkuva koko $\R$:ssä ja
\begin{align*}
&f(x) = \frac{1}{x}\cdot\frac{x-1+\sqrt{x^2+1}}{x+1+\sqrt{x^2+1}}\,, \quad 
                                                    \text{kun}\ x \neq 0. \\
\intertext{Näytä, että yksinkertaisempi $f$:n määritelmä on}
&f(x) = \frac{1}{1+\sqrt{x^2+1}}\,, \quad x\in\R.
\end{align*}

\item (*) \label{H-V-1: kiintopisteongelmia}
Olkoon $f$ on jatkuva välillä $A=[a,b]$. Näytä, että \newline
a) jos $f(A) \subset A$, niin $f(c)=c$ jollakin $c \in A$. \newline
b) jos $f(a)=f(b)$, niin $f(c)=f(c+\tfrac{b-a}{2})$ jollakin $c\in[a,\tfrac{a+b}{2}]$.

\item (*) Olkoon $\{a_k\,,\ k=0,1,2,\ldots\}$ reaalilukujono, jolle pätee $\lim_k a_k=\infty$
ja olkoon
\[
H(x)= \begin{cases} 
      \,1-\abs{x}, &\text{kun}\ \abs{x} \le 1 \\ \,0, &\text{kun}\ \abs{x}>1
      \end{cases}
\]
Näytä, että funktio
\[
f(x)=\sum_{k=0}^\infty H(x-a_k)
\]
on jatkuva koko $\R$:ssä. Mikä on $f$:n maksimiarvo ja missä se saavutetaan, kun 
$a_k=(5/4)^k$\,? 



\end{enumerate} % Jatkuvuuden käsite
\section{Funktion raja-arvo} \label{funktion raja-arvo} \alku
 
Jatkuvuuteen liittyy läheisesti käsite \kor{funktion raja-arvo}. Raja-arvo kertoo funktion 
arvojen käyttäytymisestä lähestyttäessä jotakin pistettä $a\in\R$ funktion määrittelyjoukosta 
käsin. Pisteen $a$ ei tarvitse olla määrittelyjoukossa, riittää että sitä voidaan lähestyä 
mielivaltaisen lähelle. Tyypillinen sovellustilanne onkin juuri tällainen. Toisaalta, jos
piste on määrittelyjoukossa, ei raja-arvo silti riipu funktion arvosta tässä pisteessä.

Funktion raja-arvoja on kahdentyyppisiä, varsinaisia eli raja-arvoja ilman lisämääreitä, ja 
\index{toispuolinen raja-arvo}%
\kor{toispuolisia} raja-arvoja (engl. one-sided limit). Määritelmät ovat seuraavat.
\begin{Def} \vahv{(Funktion raja-arvo)}\ \label{funktion raja-arvon määritelmä}
\index{funktion raja-arvo|emph} \index{raja-arvo!b@reaalifunktion|emph}
\index{vasemmanpuoleinen raja-arvo|emph} \index{oikeanpuoleinen raja-arvo|emph}
Funktiolla $f:\DF_f\kohti\R$, $\DF_f\subset\R$, on \kor{pisteessä} $a$ (tai $a$:ssa) 
\kor{raja-arvo} $A\in\R$, jos jokaiselle reaalilukujonolle $\seq{x_n}$ pätee
\[
\bigl(\,x_n\in\DF_f\ \ja\ x_n\neq a\,\bigr)\,\forall n\,\ \ja\,\ x_n\kohti a\,\ 
                          \impl\,\ f(x_n)\kohti A,
\]
ja oletus on voimassa jollekin jonolle $\{x_n\}$. Funktiolla on pisteessä $a$ 
\kor{vasemmanpuoleinen raja-arvo} $A_-\in\R$, jos jokaiselle reaalilukujonolle $\seq{x_n}$
pätee
\[
\bigl(\,x_n\in\DF_f\ \ja\ x_n< a\,\bigr)\,\forall n\,\ \ja\,\ x_n\kohti a\,\ 
                          \impl\,\ f(x_n)\kohti A_-,
\]
ja oletus on voimassa jollekin jonolle $\{x_n\}$. Funktiolla on pisteessä $a$
\kor{oikeanpuoleinen raja-arvo} $A_+\in\R$, jos jokaiselle reaalilukujonolle $\seq{x_n}$
pätee
\[
\bigl(\,x_n\in\DF_f\ \ja\ x_n> a\,\bigr)\,\forall n\,\ \ja\,\ x_n\kohti a\,\
                          \impl\,\ f(x_n)\kohti A_+,
\]
ja oletus on voimassa jollekin jonolle $\{x_n\}$.
\end{Def}
Raja-arvon määritelmän lisäehto 'oletus voimassa jollekin jonolle $\seq{x_n}$' tarkoittaa,
että lähestyminen ($x_n \kohti a$) oletetulla tavalla on mahdollista, ts.\ että $a$ \pain{ei}
\pain{ole} \pain{eristett}y p\pain{iste} joukossa $\DF_f \cup \{a\}$
(ks.\ Harj.teht.\,\ref{jatkuvuuden käsite}:\ref{H-V-1: eristetty piste}). Toispuolisissa
raja-arvoissa lisäehto tarkoittaa vastaavasti, että $a$ ei ole eristetty piste joukossa
$[\DF_f\cap(-\infty,a)]\cup\{a\}$ (vasemmanpuoleinen raja-arvo) tai joukossa
$[\DF_f\cap(a,\infty)]\cup\{a\}$ (oikeanpuoleinen raja-arvo). Eristetyssä pisteessä
raja-arvoa ei siis määritellä, olipa piste määrittelyjoukossa $\DF_f$ tai ei.

Määritelmässä esiintyville raja-arvoille käytetään merkintöjä
\[
A = \lim_{x\kohti a} f(x), \quad A_\pm = \lim_{x\kohti a^\pm} f(x),
\]
tai merkitään (vrt.\ lukujonon raja-arvomerkinnät)
\[
f(x) \kohti A, \quad \text{kun}\ x \kohti a, \qquad 
f(x) \kohti A_\pm, \quad \text{kun}\ x \kohti a^\pm.
\]
Toispuolisille raja-arvoille kätevä merkintätapa on myös
\[
A_+ = f(a^+), \quad A_- = f(a^-).
\]
\begin{Exa} \label{helppo raja-arvo} Määritä $\,\lim_{x \kohti 1}f(x)$, kun 
$f(x)=(x-1)/(x^2+x-2)$.
\end{Exa}
\ratk Määrittelyjoukko on $\DF_f=\{x\in\R \mid x \neq 1\,\ja\,x \neq -2\}$ ja
$f(x)=1/(x+2)\ \forall x\in\DF_f$. Näin ollen jos $x_n\in\DF_f\ \forall n$
($\,\impl\, x_n \neq 1\ \forall n$) ja $x_n \kohti 1$, niin $f(x_n)=1/(x_n+2) \kohti 1/3$.
Siis $\,\lim_{x \kohti 1}f(x)=1/3$ (Määritelmä \ref{funktion raja-arvon määritelmä}). \loppu

Kuten lukujonon, myös funktion raja-arvomerkinnöissä, voi olla $A$:n tai $A_\pm$:n
tilalla $\infty/-\infty$, jolloin tarkoitetaan Määritelmän 
\ref{funktion raja-arvon määritelmä} mukaisesti, että $f(x)$ kasvaa/vähenee rajatta,
kun $x \kohti a$ tai $x \kohti a^\pm$. 
\jatko \begin{Exa} (jatko) Esimerkin funktiolle pätee
\[
\lim_{x \kohti\,-2^-} f(x) = -\infty, \quad \lim_{x \kohti\,-2^+} f(x) = \infty. \loppu
\]
\end{Exa}

Sikäli kuin raja-arvo $\lim_{x \kohti a} f(x)$ on olemassa, yhtyvät myös toispuoliset 
raja-arvot (sikäli kuin määriteltävissä) tähän. Toisaalta on mahdollista, että molemmat 
toispuoliset raja-arvot ovat olemassa pisteessä $a$ mutta eri suuret, jolloin raja-arvoa
pisteessä $a$ ei ole. Mitä tulee jatkuvuuden ja raja-arvon väliseen yhteyteen, nähdään
Määritelmistä \ref{funktion jatkuvuus} ja \ref{funktion raja-arvon määritelmä}, että sikäli
kuin raja-arvo $\lim_{x \kohti a} f(x)$ on määriteltävissä (eli $a$ ei ole eristetty piste
joukossa $\DF_f \cup \{a\}$), pätee
\[ \boxed{ \begin{aligned}
\ykehys\quad f \text{ jatkuva pisteessä } a\in\DF_f \ \ekv \ 
       \lim_{x\kohti a} f(x) = f(a) \quad \\ (\text{$a$ ei eristetty piste}). \quad\akehys
\end{aligned} } \]
Jos $f$:llä on toispuolinen raja-arvo $f(a^+)$ tai $f(a^-)$ ja pätee joko $f(a)=f(a+)$ tai 
\index{oikealta jatkuva} \index{vasemmalta jatkuva}
\index{jatkuvuus (yhden muuttujan)!ea@oikealta, vasemmalta}%
$f(a)=f(a^-)$, niin sanotaan vastaavasti, että $f$ on \kor{oikealta jatkuva} tai 
\kor{vasemmalta jatkuva} pisteessä $a$. Jos $a\in\DF_f$ ja pistettä $a$ voidaan lähestyä 
molemmista suunnista $\DF_f$:stä käsin, niin on ilmeistä, että $f$ on jatkuva $a$:ssa
täsmälleen kun $f$ on sekä vasemmalta että oikealta jatkuva $a$:ssa. Edelleen nähdään, että
$f$ on jatkuva suljetulla välillä $[a,b]$ (Määritelmä \ref{jatkuvuus välillä}) täsmälleen kun
$f$ on jatkuva avoimella välillä $(a,b)$ ja lisäksi oikealta jatkuva $a$:ssa ja vasemmalta
jatkuva $b$:ssä.

Kuten jatkuvuus, myös raja-arvo on määriteltävissä vaihtoehtoisella 
$(\eps,\delta)$-kritee\-rillä, vrt.\ Määritelmä \ref{vaihtoehtoinen jatkuvuus}. Vaihtoehtoinen
määritelmä muotoillaan tässä lauseena raja-arvolle $\lim_{x\kohti a} f(x)= A\in\R$. Todistus
sivuutetaan, sillä se on hyvin samanlainen kuin Lauseen \ref{jatkuvuuskriteerien yhtäpitävyys}
todistus edellisessä luvussa. (Toispuolisille raja-arvoille on muotoiltavissa vastaava tulos,
samoin tapauksille $\lim_{x\kohti a} f(x) = \pm\infty$.)
\begin{*Lause} \vahv{(Raja-arvon $(\eps,\delta)$-kriteeri)} \label{approksimaatiolause}
\index{funktion raja-arvo|emph} \index{raja-arvo!b@reaalifunktion|emph}
Funktiolla $f:\DF_f\kohti\R$, $\DF_f\subset\R$, on raja-arvo $\lim_{x\kohti a} f(x)=A\in\R$
täsmälleen kun $\DF_f\cap[(a-\delta,a)\cup(a,a+\delta)]\neq\emptyset$ $\forall \delta>0$ ja
jokaisella $\eps>0$ on olemassa $\delta>0$ siten, että jokaisella $x\in\R$ pätee
\[
x\in\DF_f \ \ja \ 0<\abs{x-a}<\delta \ \impl \ \abs{f(x)-A}<\eps.
\]
\end{*Lause}
Lukujonoista tiedetään, että suppeneva lukujono on rajoitettu 
(Lause \ref{suppeneva jono on rajoitettu}). Funktiolle, jolla on raja-arvo, pätee Lauseen
\ref{approksimaatiolause} perusteella vastaava tulos:
\begin{Lause} \label{raja-arvo ja rajoitettu funktio} \index{rajoitettu!c@funktio|emph} 
Jos $\lim_{x \kohti a} f(x) = A\in\R$, niin $f$ on \kor{rajoitettu} jossakin pisteen
ympäristössä, ts.\ $\exists \delta>0$ ja $C\in\R_+$ siten, että
\[ 
|f(x)| \le C \quad \forall\ x\in(a-\delta,a+\delta)\cap\DF_f. 
\]
\end{Lause}
\tod Valitaan $\eps=1$ ja vastaava $\delta>0$ niin, että Lauseen \ref{approksimaatiolause}
ehto on voimassa. Tällöin seuraa kolmioepäyhtälöstä, että väittämä on tosi, kun
$C=\abs{A}+1$. \loppu

\subsection{Funktion approksimointi raja-arvolla}
%\index{funktion approksimointi!a@raja-arvolla|vahv}

Jos funktiosta tunnetaan raja-arvo $A = \lim_{x\kohti a} f(x)$, niin Lauseen 
\ref{approksimaatiolause} mukaan voidaan raja-arvopisteen lähellä käyttää approksimaatiota 
$f(x) \approx A$. Lause ei tosin anna (eikä tehdyin oletuksin voikaan antaa) mitään 
kvantitatiivista tietoa approksimaation tarkkuudesta, koska $\delta$:n riippuvuutta
$\eps$:sta ei tunneta. Tyypillisissä esimerkkitapauksissa funktiosta $f$ kuitenkin yleensä
tiedetään lauseessa oletettua enemmän, jolloin approksimaatiolle on ehkä mahdollista johtaa 
kvantitatiivinen virhearvio tämän lisätiedon perusteella. Likimääräisessä funktioevaluaatiossa
raja-arvotieto voi auttaa etenkin silloin, kun funktion laskukaava suoraan käytettynä on
altis numeerisille pyöristysvirheille raja-arvopisteen lähellä.
\begin{Exa} \label{raja-arvolla approksimointi} Funktio $f(x) = (1-\cos x)/x^2$ on määritelty,
kun $x \neq 0$. Myöhemmin (Luku \ref{kaarenpituus}) osoitetaan, että 
$\lim_{x \kohti 0} f(x) = 1/2$. Tähän tulokseen perustuva approksimaatio
\[
f(0.000000003) \approx \lim_{x \kohti 0} f(x) = 0.5
\]
on huomattavasti turvallisempi kuin $f$:n laskukaavan suora käyttö, sillä lasku\-operaatiossa
$x \map 1-\cos x$ tapahtuu huomattava merkitsevien numeroiden kato, kun $\abs{x}$ on pieni 
(vrt.\ Luku \ref{desimaaliluvut}). Paitsi turvallinen, raja-arvoon perustuva
\mbox{approksimaatio} on tässä tapauksessa myös hyvin tarkka: virhe on alle $10^{-18}$. \loppu
\end{Exa}

\subsection{Raja-arvojen yhdistely}
\index{funktion raja-arvo!a@raja-arvojen yhdistely|vahv}

Raja-arvojen laskemista helpottavat seuraavat lauseet, jotka ovat Lauseiden 
\ref{jatkuvuuden yhdistelysäännöt} ja \ref{yhdistetyn funktion jatkuvuus} vastineita. 
Todistukset ovat Määritelmän \ref{funktion raja-arvon määritelmä} perusteella suoraviivaisia
(Harj.teht.\,\ref{H-V-2: todistuksia}; ks.\ myös 
Harj.teht.\,\ref{jatkuvuuden käsite}:\ref{H-V-1: eristetty piste}). Lauseet pätevät ilmeisin
muutoksin myös toispuolisille raja-arvoille.
\begin{Lause} \label{funktion raja-arvojen yhdistelysäännöt}
Jos $\lim_{x\kohti a} f(x) = A\in\R$, niin
\begin{align*}
\lim_{x\kohti a} (\lambda f)(x) &= \lambda A \quad \forall \lambda\in\R. \\
\intertext{Jos lisäksi $\lim_{x\kohti a} g(x)=B\in\R$ ja $a$ ei ole joukon
$(\DF_f\cap\DF_g)\cup\{a\}$ eristetty piste, niin}
\lim_{x\kohti a} (f+g)(x)       &=A+B, \\  
\lim_{x\kohti a} (fg)(x)        &=AB.
\intertext{Jos lisäksi $B\neq 0$ ja $a$ ei ole joukon
$\DF_{f/g}\cup\{a\}=\{x\in\DF_f\cap\DF_g \mid g(x) \neq 0\}\cup\{a\}$ eristetty piste, niin}
\lim_{x\kohti a} (f/g)(x)       &=A/B.
\end{align*}
\end{Lause}
\begin{Lause} \label{yhdistetyn funktion raja-arvo} Jos $\lim_{x\kohti a} g(x) = A\in\R$, $f$
on jatkuva pisteessä $x=A$ ja $a$ ei ole joukon $\DF_{f \circ g}\cup\{a\}$ eristetty piste,
niin $\lim_{x\kohti a} (f\circ g)(x) = f(A)$. 
\end{Lause}
\jatko \begin{Exa} (jatko) Koska $g(x) = \sqrt{\abs{x}}$ on jatkuva $\R$:ssä 
(Propositio \ref{potenssifunktion jatkuvuus}), niin esimerkin raja-arvotuloksesta ja Lauseesta
\ref{yhdistetyn funktion raja-arvo} seuraa
\[
\lim_{x \kohti 0} \frac{\sqrt{1-\cos x}}{\abs{x}} 
      = \lim_{x \kohti 0} \sqrt{\frac{1-\cos x}{x^2}} = \frac{1}{\sqrt{2}}\,. \loppu 
\]
\end{Exa}
\begin{Exa} Funktioille $f_0(x)=1$ ja $f_1(x)=x$ ovat voimassa ilmeiset raja-arvotulokset
\[
\lim_{x \kohti a} f_0(x) = 1, \quad \lim_{x \kohti a} f_1(x) = a, \quad a \in \R.
\]
Näiden ja Lauseen \ref{funktion raja-arvojen yhdistelysäännöt} perusteella seuraa
\[
\lim_{x \kohti a} \frac{(x+1)^2}{x+2} = \frac{(a+1)^2}{a+2}\,, \quad \text{kun}\ a \neq -2.
\]
Tulos on selvä myös Lauseen \ref{jatkuvuuden yhdistelysäännöt} perusteella, sillä tämän mukaan
rationaalifunktio $f$ on jatkuva koko määrittelyjoukossaan, jolloin 
$\lim_{x\kohti a} f(x) = f(a)$, kun $a\in\DF_f$. \loppu
\end{Exa}

\subsection{Raja-arvon laskeminen sijoituksella}
\index{funktion raja-arvo!b@laskeminen sijoituksella|vahv}

Raja-arvon $\,\lim_{x \kohti a} f(x)$ laskemista on usein mahdollista helpottaa tekemällä
\kor{sijoitus} eli \kor{muuttujan vaihto} $\,x=u(t)$. Muuttujaa vaihdettaessa tukeudutaan
seuraavaan lauseeseen, joka on helposti muotoiltavissa myös toispuolista raja-arvoa
koskevaksi (Harj.teht.\,\ref{H-V-2: todistuksia}c).
\begin{Lause} \label{raja-arvo sijoituksella} Olkoon $u:\ U\kohti[a-\delta,a+\delta]$ 
($\delta>0$) jatkuva bijektio, missä $U$ on suljettu väli. Tällöin jos $u(\alpha)=a$, niin
pätee
\[
\lim_{x \kohti a} f(x) = \lim_{t\kohti\alpha} f(u(t)) = \lim_{t\kohti\alpha} F(t)=A,
\]
sikäli kuin raja-arvo oikealla on olemassa ($A\in\R$ tai $A=\pm\infty$).
\end{Lause}
\tod Lauseen \ref{käänteisfunktion jatkuvuus} perusteella myös käänteisfunktio
$u^{-1}:\ [a-\delta,a+\delta] \kohti U$ on jatkuva bijektio. Olkoon nyt $x_n\in\DF_f$,
$x_n \neq a\ \forall n$ ja $x_n \kohti a$. Tällöin jostakin indeksistä eteenpäin on 
$\abs{x_n-a}<\delta$, jolloin voidaan kirjoittaa $\,x_n=u(t_n),\ t_n \in U$. Koska
$u:\ U \kohti [a-\delta,a+\delta]$ on injektio, niin $x_n \neq a\ \impl\ t_n\neq\alpha$.
Koska $u^{-1}$ on jatkuva pisteessä $a$, niin 
$x_n \kohti a\ \impl\ t_n=u^{-1}(x_n) \kohti u^{-1}(a)=\alpha$. Tällöin oletuksen
$\lim_{t\kohti\alpha} F(t)=A$ perusteella pätee $f(x_n)=f(u(t_n))=F(t_n) \kohti A$. On näytetty,
että $\,\lim_{x \kohti a} f(x)=A$ (Määritelmä \ref{funktion raja-arvon määritelmä},
kun $A\in\R$; päättely toimii myös, kun $A=\pm\infty$). \loppu

\begin{Exa} \label{raja-arvo muuttujan vaihdolla} Määritä raja-arvo
$\D \,A=\lim_{x \kohti 81} \frac{\sqrt{x}-9}{\sqrt[4]{x}-3}\,$.
\end{Exa}
\ratk Tässä sopiva sijoitus on $\,\sqrt[4]{x}=t$, jolloin on $\,x=t^4=u(t)$, ja arvoa
$x=81\,(=a)$ vastaa $t=3\,(=\alpha)$. Koska
\[
F(t)=f(t^4)=\frac{t^2-9}{t-3}=t+3=G(t), \quad \text{kun}\ t \neq 3,
\]
niin funktion $G$ jatkuvuuden perusteella on $\lim_{t \kohti 3} F(t)=G(3)=6$. Lauseen
\ref{raja-arvo sijoituksella} oletukset ovat voimassa ($\delta \le 3$), joten kysytty
raja-arvo on $A=6$. \loppu
\begin{Exa} Sijoituksella $x=2t$ saadaan (vrt.\ Luku \ref{trigonometriset funktiot})
\[
\lim_{x \kohti 0^+} \frac{\sin\frac{x}{2}}{\sqrt{1-\cos x}}
  = \lim_{t \kohti 0^+} \frac{\sin t}{\sqrt{1-\cos 2t}}
  = \lim_{t \kohti 0^+} \frac{\sin t}{\sqrt{2\sin^2 t}}
  = \lim_{t \kohti 0^+} \frac{1}{\sqrt{2}} 
  = \frac{1}{\sqrt{2}}\,.
\]
Tässä loppusievennys perustui päättelyyn
$\,t\in(0,\pi]\,\impl\,\sqrt{\sin^2 t}=\sin t$. \loppu
\end{Exa}

Esimerkeissä suoritettiin raja-arvolaskuille hyvin tyypillinen nelivaiheinen lasku\-operaatio
\[
\lim_{x \kohti a} f(x) = \lim_{t\kohti\alpha} f(u(t)) = \lim_{t\kohti\alpha} F(t)
                       = \lim_{t\kohti\alpha} G(t) = G(\alpha).
\]
Tässä
\begin{enumerate}
\item Sijoitetaan $x=u(t)$ [\,tai $v(x)=t\,\impl\,x=u(t)$\,] ja lasketaan $\alpha=u^{-1}(a)$.
\item Sievennetään $f(u(t))$ lausekkeeksi $F(t)$.
\item Pelkistetään $F(t)$ lausekkeeksi $G(t)$, kun $t\neq\alpha$.
\item Lasketaan raja-arvo vedoten $G$:n j\pain{atkuvuuteen} pisteessä $t=\alpha$.
\end{enumerate}
Sikäli kuin muuttujaa ei vaihdeta, lasku supistuu vaiheiksi 3--4 ($F=f,\,t=x$), kuten
Esimerkissä \ref{helppo raja-arvo} edellä.

\subsection{Funktion jatkaminen}
\index{jatkaminen (funktion)|vahv}

Jos $a\in\R$ ei ole funktion $f$ määrittelyjoukossa, mutta on olemassa (aito) raja-arvo
$\lim_{x\kohti a} f(x)=A\in\R$, niin on luonnollista sisällyttää $a$ määrittelyjoukkoon
asettamalla $f(a)=A$. Näin menetellen $f$:stä tulee Määritelmän \ref{funktion jatkuvuus}
mukaisesti jatkuva pisteessä $a$. Funktion määrittelyjoukon laajentamista tällä tavoin
kutsutaan \kor{funktion jatkamiseksi}.
\begin{Exa} Esimerkissä \ref{helppo raja-arvo} todettiin, että funktiolla
$f(x)=(x-1)/(x^2+x-2)$ on raja-arvo $\lim_{x \kohti 1}f(x)=1/3$. Kun piste $x=1$ sisällytetään
$f$:n määrittelyjoukkoon asettamalla $f(1)=1/3$, niin $f$ tulee jatketuksi funktioksi
$f(x)=1/(x+2)$. Enempää ei määrittelyjoukkoa voi tässä laajentaa jatkamalla, koska
$f$:llä ei ole reaalista (tai yleisempääkään) raja-arvoa, kun $x \kohti -2$. \loppu
\end{Exa}
\begin{Exa} Esimerkin \ref{raja-arvolla approksimointi} raja-arvotiedon
(ja Esimerkin \ref{jatkuvuuden käsite}:\ref{trig yhdistely} tiedon) perusteella funktio
\[ 
f(x) = \begin{cases} \,\dfrac{1-\cos x}{x^2}\,, &\text{kun}\ x \neq 0, \\[2mm]
                     \,\dfrac{1}{2}\,,          &\text{kun}\ x=0
       \end{cases}
\]
on jatkuva $\R$:ssä. Funktion arvo $0$:ssa on määrätty jatkamalla. \loppu
\end{Exa}             

\subsection{Paloittainen jatkuvuus}
\index{jatkuvuus (yhden muuttujan)!f@paloittainen|vahv} \index{paloittainen!b@jatkuvuus|vahv}

Funktion toispuoliset raja-arvot tulevat käyttöön erityisesti sellaisissa sovellustilanteissa,
joissa (usein fysikaalista perua oleva) funktio on jatkuva muualla paitsi erillisissä 
\index{hyppyepäjatkuvuus}%
pisteissä, joissa sillä on yksinkertainen nk.\ \kor{hyppyepäjatkuvuus}
(engl.\ jump discontinuity). Asetetaan tällaisia käytännön tarpeita silmällä pitäen
\begin{Def}
Funktio $f:\DF_f\kohti\R$, $\DF_f\subset\R$, on välillä $[a,b]$ \kor{paloittain jatkuva} 
(engl.\ piecewise continuous), jos $\exists$ pisteet $c_k$, $k=0\ldots n$, $n\in\N$ siten,
että
\[
a=c_0<c_1<\ldots<c_n=b
\]
ja pätee
\begin{itemize}
\item[(i)] $(c_{k-1},c_k)\subset\DF_f$ ja $f$ on jatkuva välillä
$(c_{k-1},c_k), \quad k=1 \ldots n$,
\item[(ii)] $\exists$ toispuoliset raja-arvot
\begin{align*}
\lim_{x\kohti c_k^+} f(x) &= f(c_k^+), \quad k=0 \ldots n-1, \\ 
\lim_{x\kohti c_k^-} f(x) &= f(c_k^-), \quad k=1 \ldots n.
\end{align*}
\end{itemize}
\end{Def}
Huomattakoon, että raja-arvot $f(c_k^\pm)$ eivät raja-arvon määritelmän mukaisesti riipu $f$:n
mahdollisista arvoista pisteissä $c_k$. Jos molemmat toispuoliset raja-arvot ovat olemassa,
mutta eivät yhdy, on kyseessä hyppyepäjatkuvuus (hyppy= $f(c_k^+)-f(c_k^-)$). Jos yhtyvät, on
funktio pisteessä $c_k$ joko jatkuva tai määriteltävissä jatkuvaksi raja-arvon avulla
(jatkamismenettely).
\begin{Exa} \label{sahafunktio} Olkoon $[x]=$ suurin kokonaisluku, jolle pätee $[x] \le x$.
Funktio $f(x)=x-[x]$, eli
\[
f(x)=x-k,\quad \text{kun } x\in[k,k+1), \quad k\in\Z,
\]
on jokaisella välillä $[a,b]\subset\R$ paloittain jatkuva. Pisteissä $k\in\Z$ funktio
voitaisiin määritellä miten tahansa (tai jättää määrittelemättä) ilman, että sillä 
vaikutettaisiin toispuolisiin raja-arvoihin $f(k^+)=0$ ja $f(k^-)=1$. \loppu
\begin{figure}[H]
\setlength{\unitlength}{1cm}
\begin{center}
\begin{picture}(14,3)(-1,-1)
\put(-1,0){\vector(1,0){14}} \put(12.8,-0.4){$x$}
\put(0,-1){\vector(0,1){3}} \put(0.2,1.8){$y$}
\multiput(0,0)(1,0){12}{\drawline(0,0)(0,-0.1)} \put(0.9,-0.5){$1$} \put(1.9,-0.5){$2$}
\drawline(-0.1,1)(0,1) \put(-0.4,0.9){$1$}
\multiput(0,0)(1,0){12}{\drawline(0,0)(1,1)}
\end{picture}
\end{center}
\end{figure}
\end{Exa}
\index{zza@\sov!Kiihdytys}%
\begin{Exa} \vahv{Kiihdytys}. Auto on paikallaan moottorin käydessä. Ajan hetkellä $t=0$
polkaistaan kaasupoljin pohjaan. Kiihtyvyys $f(t)=$ ?
\end{Exa}
\ratk Idealisoidun matemaattisen mallin mukaan on
\[
f(t)=
\begin{cases}
\,0,              &\text{kun}\ t<0, \\
\,a=\text{vakio}, &\text{kun}\ t\ge 0.
\end{cases}
\]
Todellisuudessa hyppyepäjatkuvuutta ei hetkellä $t=0$ esiinny. Matemaattinen malli (jos hyvä)
riippuukin esimerkissä olennaisesti siitä, millaisessa \pain{aikaskaalassa} tapahtumia
tarkastellaan. Eri skaaloissa funktio $f$ voi näyttää hyvin erilaiselta. \loppu
\begin{figure}[H]
\setlength{\unitlength}{1cm}
\begin{center}
\begin{picture}(14,4)(0,0)
\multiput(0,0)(8,0){2}{
\put(0,1){\vector(1,0){6}} \put(5.8,0.6){$t$}
\put(1,0){\vector(0,1){4}} \put(1.2,3.8){$y$}
}
\multiput(2,1)(1,0){4}{\drawline(0,0)(0,-0.1)} \put(1.7,0.5){$1$ s}
\drawline(0.9,3)(1,3) \put(0.6,2.9){$a$}
\multiput(10,1)(1,0){4}{\drawline(0,0)(0,-0.1)} \put(9.5,0.5){$1$ ms}
\drawline(8.9,3)(9,3) \put(8.6,2.9){$a$}
\dashline{0,2}(9,2.97)(14,2.97)
\spline(1,1)(1.1,3)(1.3,2.9)(1.4,3)(2,3)(3,3)(4,3)(5,3)
\spline(9,1)(9.5,2)(9.7,1.5)(10,2.3)(10.3,2.2)(10.6,2.5)(11,2.7)(11.5,2.8)(12,2.9)(13,2.95)
\put(3,3.3){$y=f(t)$} \put(11,3.3){$y=f(t)$}
\end{picture}
\end{center}
\end{figure}

\subsection{Raja-arvot $\displaystyle{\lim_{x\kohti\pm\infty}f(x)}$}
\index{funktion raja-arvo|vahv} \index{raja-arvo!b@reaalifunktion|vahv}

Raja-arvo $\lim_{x\kohti\infty} f(x)$ määritellään samoin kuin edellä, eli sijoittamalla
yleiseen määritelmään $a=\infty$. Tällöin $x_n\kohti\infty$ tarkoittaa, että $\{x_n\}$ kasvaa
rajatta. Raja-arvo $\lim_{x\kohti -\infty} f(x)$ määritellään vastaavasti. Lause 
\ref{funktion raja-arvojen yhdistelysäännöt} on pätevä myös raja-arvoille 
$\lim_{x\kohti\pm\infty} f(x)$, samoin Lause \ref{yhdistetyn funktion raja-arvo}, kun 
määrittelyjoukkoa $\DF_{f\circ g}$ koskeva oletus muutetaan joko ehdoksi
$\DF_{f\circ g}\cap(M,\infty)\neq\emptyset\ \forall M\in\R$ (jos $x \kohti \infty$) tai ehdoksi 
$\DF_{f\circ g}\cap(-\infty,M)\neq\emptyset\ \forall M\in\R$ (jos $x \kohti- \infty$).

Raja-arvoja $\lim_{x\kohti\infty} f(x)$ määrättäessä usein kätevä on sijoitus $x=t^{-1}$, eli
siirtyminen tarkastelemaan funktiota $F(t)=f(1/t)$. Koska 
$x\kohti\pm\infty\ \ekv\ t\kohti 0^\pm$, niin raja-arvojen määritelmistä nähdään, että pätee
(vrt.\ Lause \ref{raja-arvo sijoituksella})
\[
\lim_{x\kohti\pm\infty} f(x) = \lim_{t\kohti 0^\pm} F(t) = F(0^\pm), \quad F(t)=f(1/t).
\]
\begin{Exa} (jatko) Muuttujan vaihdolla $x=t^{-1}$ päätellään Propostitioon 
\ref{potenssifunktion jatkuvuus} ja Lauseisiin \ref{jatkuvuuden yhdistelysäännöt} ja 
\ref{yhdistetyn funktion jatkuvuus} vedoten:
\[
\lim_{x\kohti\infty} \frac{\sqrt[3]{x}+\sqrt[4]{x}}{\sqrt[3]{8x+3}} 
                 = \lim_{t\kohti 0^+} \frac{1+\sqrt[12]{t}}{\sqrt[3]{8+3t}}
                 = \frac{1+0}{\sqrt[3]{8+0}} = \frac{1}{2}\,. \loppu
\]
\end{Exa}
Lukujonojen teoriasta tiedetään, että monotoninen ja rajoitettu lukujono suppenee.
Raja-arvoja $\,\lim_{x\kohti\infty}f(x)\,$ koskeva vastaava väittämä on
\begin{Lause} \label{monotonisen funktion raja-arvo} Jos funktio $f$ on monotoninen ja
rajoitettu välillä $[a,\infty)$, niin on olemassa raja-arvo $\lim_{x\kohti\infty}f(x)\in\R$.
\end{Lause}
\tod Olkoon $f$ esim.\ kasvava välillä $[a,\infty)$. Tällöin jos merkitään $y_k=f(k)$,
$k\in\N$ ja $k \ge a$, niin $\seq{y_k}$ on kasvava ja rajoitettu lukujono, joten
$y_k \kohti y\in\R$. Olkoon nyt $\seq{x_n}$ mikä tahansa lukujono, jolle pätee
$x_n\ge a\ \forall n$ ja $x_n\kohti\infty$. Tällöin jos $x_n\in[k,k+1)$, $k\in\N$, niin
$y_k \le f(x_n) \le y_{k+1}$, koska $f$ on kasvava. Tässä $k\kohti\infty$ kun $n\kohti\infty$
(koska $x_n\kohti\infty$), joten päätellään (Propositio \ref{jonotuloksia}[V2]), että
$\lim_nf(x_n)=\lim_ky_k=y$. Koska tämä pätee jokaiselle mainitut ehdot täyttävälle jonolle
$\seq{x_n}$, niin $\,\lim_{x\kohti\infty}f(x)=y$. Jos $f$ on vähenevä välillä $[a,\infty)$, niin
päättely on vastaava. \loppu

\subsection{Asymptootit}
\index{asymptootti|vahv}

Sanotaan, että funktio $g(x)$ on funktion $f(x)$ \kor{asymptootti}, jos 
\[
\lim_{x\kohti \infty} [f(x)-g(x)]=0, \ \text{ tai }\ \lim_{x\kohti -\infty} [f(x)-g(x)]=0.
\]
Asymptootin ideana on approksimoida funktiota jollakin (mieluiten) yksinkertaisella
funktiolla $g(x)$, kun joko $x$ on hyvin suuri (merkintä $x \gg 1$) tai $-x$ on hyvin suuri 
(merkintä $x \ll -1$)\,:
\[
f(x)\approx g(x), \quad \text{kun}\ x\gg 1\ 
                        \text{tai}\ x\ll -1.\footnote[2]{Asymptootin perinteisempi ja
rajoitetumpi geometrinen määritelmä on tasokäyrään liittyen \pain{suora}, jota 'käyrä lähestyy
äärettömyydessä'. Esimerkiksi käyrän $S:\,y=x^2/(x+1)$ asymptootteja ovat tämän tulkinnan
mukaan suorat $y=x-1$ ja $x=-1$. Jälkimmäinen on nk.\ pystysuora asymptootti, jolla ei ole
funktiovastinetta.}
\]
Kyse on Lauseesta \ref{approksimaatiolause}, joka on yleistettävissä myös raja-arvoja 
$\lim_{x\kohti\pm\infty} f(x)$ koskevaksi: Esimerkiksi jos $\DF_f\cap\DF_g\supset[a,\infty)$ ja
$\lim_{x\kohti\infty} [f(x)-g(x)]=0$, niin jokaisella $\eps>0$ on olemassa $M\in[a,\infty)$
siten, että $\,\abs{f(x)-g(x)}<\eps\ \forall x>M$.
\begin{Exa} Funktion $f(x)=\sqrt{x^2+4x}\,$ eräs asymptootti, kun $|x| \gg 1$, on $g(x)=|x+2|$,
sillä rajoilla $x\kohti\pm\infty\,$ on
\begin{align*}
f(x)-g(x) \,=\, \sqrt{x^2+4x}-|x+2| 
          &\,=\, \frac{x^2+4x-(x+2)^2}{\sqrt{x^2+4x}+|x+2|} \\
          &\,=\, -\frac{4}{\sqrt{x^2+4x}+|x+2|}\ \kohti\ 0. \loppu
\end{align*}
\end{Exa}

\Harj
\begin{enumerate}

\item \label{H-V-2: todistuksia}
a) Todista Lause \ref{funktion raja-arvojen yhdistelysäännöt}. \
b) Todista Lause \ref{yhdistetyn funktion raja-arvo}. \
c) Muotoile ja todista Lauseen \ref{raja-arvo sijoituksella} vastine koskien toispuolista
raja-arvoa $\lim_{x \kohti a^+} f(x)$.

\item
Funktiosta $f$ tiedetään, että $[-1,1]\subset\DF_f$, $f(0)=0$ ja 
$\,\sqrt{2-x^2} \le f(x) \le \sqrt{2+9\abs{x}}$, kun $0 < \abs{x} \le 1$. Näytä, että
$\lim_{x \kohti 0}f(x)=\sqrt{2}$.

\item
a) Funktiosta $f$ tiedetään, että $\,\lim_{x \kohti 0^+}f(x)=A$. Näytä, että jos $f$ on 
parillinen, niin $\,\lim_{x \kohti 0^-}f(x)=A$, ja jos pariton, niin
$\,\lim_{x \kohti 0^-}f(x)=-A$. \newline
b) Funktiosta $f$ tiedetään, että $\,\lim_{x \kohti 0^+}f(x)=A$ ja
$\,\lim_{x \kohti 0^-}f(x)=B$. Laske raja-arvot $\lim_{x \kohti 0^+} f(x^2-x)$ ja
$\lim_{x \kohti 0^-} f(x^2+x^3)$. 

\item
Määritä seuraavat raja-arvot, joko reaalilukuna tai muodossa $\pm\infty$, tai totea 
vaihtoehtoisesti, ettei raja-arvoa ole. Vaihda tarvittaessa muuttujaa.
\begingroup
\allowdisplaybreaks

\begin{align*}
&\text{a)}\ \lim_{x \kohti 4} (x^2-4x+1) \qquad
 \text{b)}\ \lim_{x \kohti 3} \frac{x+3}{x+6} \qquad
 \text{c)}\ \lim_{x\kohti\pi} \frac{(x+\pi)^2}{\pi x} \\[1mm]
&\text{d)}\ \lim_{x \kohti -2} \frac{x^2+2x}{x^2-4} \qquad
 \text{e)}\ \lim_{x \kohti 2} \left(\frac{1}{x-2}-\frac{4}{x^2-4}\right) \qquad
 \text{f)}\ \lim_{x \kohti \frac{\pi}{2}} \frac{\sin(2x+\pi)}{\cot x} \\
&\text{g)}\ \lim_{t \kohti 0} \frac{(t+1)^2-(t-1)^2}{t} \qquad
 \text{h)}\ \lim_{s \kohti 0} \frac{s^2+3s}{(s+2)^2-(s-2)^2} \\
&\text{i)}\ \lim_{x \kohti -3} \abs{x-3} \qquad
 \text{j)}\ \lim_{x \kohti 2} \frac{\abs{x^2-4x+3}}{x^2+2x-3} \qquad
 \text{k)}\ \lim_{x \kohti 1} \frac{\abs{x^2-4x+3}}{x^2+2x-3} \\
&\text{l)}\ \lim_{x \kohti 0} \frac{\sqrt{1+2x+3x^2}-\sqrt{1-x}}{x} \qquad
 \text{m)}\ \lim_{y \kohti 1} \frac{y-4\sqrt{y}+3}{y^2-1} \\
&\text{n)}\ \lim_{x \kohti 0} \frac{\sqrt{2-x}-\sqrt{2+x}}{x\sqrt{x}} \qquad
 \text{o)}\ \lim_{x \kohti 2} \frac{\sqrt{4-4x+x^2}}{x-2} \qquad 
 \text{p)}\ \lim_{t \kohti 8} \frac{t^{2/3}-4}{t^{1/3}-2} \\
&\text{q)}\ \lim_{x \kohti 3^+} \frac{\abs{x-3}}{3-x} \qquad
 \text{r)}\ \lim_{x \kohti 2^-} \frac{\sqrt{4-4x+x^2}}{x-2} \qquad
 \text{s)}\ \lim_{x \kohti \pi^+} \frac{\sqrt{1+\cos x}}{\cos\frac{x}{2}} \\
&\text{t)}\ \lim_{x \kohti -0.4^+} \frac{2x+5}{5x+2} \qquad
 \text{u)}\ \lim_{x \kohti 1^+} \frac{x}{\sqrt{x^2-1}} \qquad
 \text{v)}\ \lim_{x \kohti 1^+} \frac{\sqrt{x^2-x}}{x-x^2} \\
&\text{x)}\ \lim_{x\kohti\infty} \frac{1+x+x^4}{2+30x+200x^3} \qquad
 \text{y)}\ \lim_{x\kohti\infty} \left(\frac{x^2}{x+1}-\frac{x^2}{x-1}\right) \\
&\text{z)}\ \lim_{x \kohti -\infty} \frac{2x-1}{\sqrt{3x^2+x+1}} \qquad
 \text{å)}\ \lim_{x\kohti\infty} \frac{x\sqrt{x+1}\,(1-\sqrt{2x+3})}{7-6x+4x^2} \\[1mm]
&\text{ä)}\ \lim_{x\kohti\infty} \left(\sqrt{x^2+9x}-\sqrt{x^2-5x}\right) \qquad
 \text{ö)}\ \lim_{x \kohti -\infty} \left(\sqrt{x^2-4x}-\sqrt{x^2+10x}\right)
\end{align*}%
\endgroup

\item \label{H-V-2: väittämiä}
Todista Määritelmän \ref{funktion raja-arvon määritelmä} ja lukujonojen teorian
avulla: \vspace{1mm}\newline
a) Jos $f$ on monotoninen ja rajoitettu välillä $(a,b)\subset\DF_f$, niin on olemassa
raja-arvot $f(a^+)$ ja $f(b^-)$. \newline
b) Jos $f(x) \le g(x)\ \forall x\in(a,b)\subset\DF_f\cap\DF_g$, niin $f(a^+) \le g(a^+)$ ja
$f(b^-) \le g(b^-)$ sikäli kuin raja-arvot ovat olemassa reaalilukuina.

\item
Laajenna funktion $\,\D f(x)=\frac{x^2-1}{\sqrt{x+3}-2}\,$
määrittelyjoukko väliksi $[-3,\infty)$ jatkamalla. Mikä on jatketun funktion sievennetty
laskusääntö?

\item
Olkoon $f$ määritelty välillä $\,[1,\infty)\,$ ja olkoon $\,y_n=f(n)$, $n\in\N$. Näytä, että
pätee $\,\lim_{x\kohti\infty} f(x)=A\ \impl\ \lim_ny_n=A$. Näytä vastaesimerkillä, että
käänteinen implikaatio ei ole tosi.

\item
Määritä seuraaville funktioille $f(x)$ tai $y(x)$ asymptootti $g(x)$, joka on annettua
muotoa ($a,b\in\R$) ja mahdollisimman tarkka (ellei yksikäsitteinen). Tarkastele erikseen
tapaukset $x\kohti\infty$ ja $x \kohti -\infty$. \vspace{2mm}\newline
a) $\D \,f(x)=\frac{\abs{x+3}}{2x-1}\,, \quad g(x)=a$ \newline
b) $\D \,f(x)=\frac{(x+3)^2}{\abs{3x+1}}\,, \quad g(x)=ax+b$ \vspace{1mm}\newline
c) $\D \,f(x)=\sqrt{2x^2+3x+\cos x}, \quad g(x)=ax+b$ \vspace{3mm}\newline
d) $\D \,f(x)=\frac{1}{\sqrt{3x^2+4x}+\sqrt{x^2+x+1}}\,, \quad g(x)=\frac{a}{x}$ 
                                                         \vspace{1mm}\newline
e) $\D \,f(x)=\frac{\abs{3x+2}}{2x+3}\,, \quad g(x)=a+\frac{b}{x}$ \vspace{2mm}\newline
f) $\D \,9(x-1)^2-16(y+2)^2=25,\ y(x) \ge -2, \quad g(x)=ax+b$ \vspace{3mm}\newline
g) $\D \,(2x^2+1)y+x\,\cos y=(2x+1)^3, \quad g(x)=ax+b$

\item
Funktiolla $f(x)=\sqrt{x^4+8x^3+35x^2+78x+98}$ on asymptootteina toisen asteen polynomeja. 
Koeta löytää mahdollisimman tarkka tällainen asymptootti $p(x)$ ja arvioi approksimaation 
$f(x)\approx p(x)$ virhe, kun $|x|\ge 100$.

\item (*)
Olkoon $a>0$ ja $m,n\in\N$. Laske raja-arvo
$\displaystyle{\,\lim_{x \kohti a} \frac{\sqrt[m]{x}-\sqrt[m]{a}}{\sqrt[n]{x}-\sqrt[n]{a}}\,}$.

\item (*) Reaalifunktiosta $f$ tiedetään, että $f(x)=x+1\ \forall x\in\R,\ x \not\in X$, missä
$X\subset\R$ on äärellinen joukko. Joukkoa $X$ ei tunneta, eikä pisteistä $x \in X$ tiedetä
muuta kuin että $f$ on näissä pisteissä joko määritelty jollakin tuntemattomilla tavalla
(reaaliarvoisena) tai jätetty määrittelemättä. Näytä, että $\,\lim_{x \kohti 0} f(x)=1$.

\end{enumerate} % Funktion raja-arvo
\section{Derivaatta} \label{derivaatta}
\alku

Tarkastellaan funktion $f$ approksimoimista pisteen $a\in\DF_f$ lähellä perustuen
erilaisiin olettamuksiin funktion ennustettavuudesta. Ensinnäkin jos $f$ jatkuva
pisteessä $a$, niin Määritelmän \ref{vaihtoehtoinen jatkuvuus} mukaan $f$ on $a$:n lähellä
likimain vakio: $f(x) \approx f(a)$ kun $x \approx a$. Tämän approksimaation virhe on
$a$:n lähellä tyypillisesti muotoa $f(x)-f(a) \approx L(x-a)$, missä $L$ on $f$:stä ja $a$:sta
riippuva vakio. Erikoistapauksia (kuten $f(x)=$ vakio tai $f(x)=x^2,\ a=0$) lukuunottamatta
mainittu virhearvio on yleisesti paras mahdollinen.
\begin{Exa} Jos $f(x)=x^3$ ja $a \neq 0$, niin
\[
f(x)-f(a) \,=\, x^3-a^3 \,=\, (x-a)(x^2+ax+a^2) \,=\, (x-a)g(x).
\]
Tässä $g(x)=x^2+ax+a^2$ on (polynomina) jatkuva, joten pisteen $x=a$ lähellä on
$f(x)-f(a) \approx L(x-a)$, missä $L=g(a)=3a^2$. \loppu
\end{Exa} 
Jos jatkuvalle funktiolle halutaan pisteen $a$ lähellä olennaisesti tarkempi approksimaatio
kuin $f(x) \approx f(a)$, on approksimaation oltava jotakin muuta tyyppiä kuin 
$f(x) \approx A =$ vakio, sillä näistä vaihtoehdoista valinta $A=f(a)$ on paras. Koska 
vakio = polynomi astetta $n=0$, niin luonteva seuraava yritys on käyttää approksimaatiossa
polynomia $p$ astetta $n=1$. Kun ennakkoehdoksi asetetaan $p(a)=f(a)$, niin approksimaatio
saa muodon
\[
f(x) \approx f(a) + k(x-a).
\]
Tässä kerroin $k \in \R$ määrätään (jos mahdollista) niin, että virheelle
$g(x)=f(x)-f(a)-k(x-a)$ saadaan $a$:n lähellä olennaisesti parempi arvio kuin
$\abs{g(x)} \le L\abs{x-a}$. Asetetaan minimiehdoksi
\[
\lim_{x \kohti a}\,\frac{g(x)}{x-a} = 0 
                  \,\ \ekv\,\ \lim_{x \kohti a} \left[\frac{f(x)-f(a)}{x-a}-k\right] = 0
                  \,\ \ekv\,\ \lim_{x \kohti a} \frac{f(x)-f(a)}{x-a} = k.
\]
Riippuen funktiosta $f$ kerroin $k$ siis joko määräytyy tästä ehdosta yksikäsitteisesti
(raja-arvo oikealla olemassa  reaalilukuna) tai sitten kerrointa ei voi määrätä, jolloin
ym.\ approksimaatioyritys katsotaan epäonnistuneeksi.
\jatko \begin{Exa} (jatko) Esimerkin funktiolle pätee
\begin{align*}
f(x) &\,=\, f(a)+(x-a)(x^2+ax+a^2) \\
     &\,=\, f(a)+3a^2(x-a)+(x-a)(x^2+ax-2a^2) \\
     &\,=\, f(a)+3a^2(x-a)+(x-a)^2(x+2a).
\end{align*}
Siis jos valitaan $k=3a^2$, niin $g(x)/(x-a)=(x-a)(x+2a) \kohti 0$ kun $x \kohti a$, joten
approksimaatio onnistui. \loppu
\end{Exa}
\begin{Def} \vahv{(Derivaatta)} \label{derivaatan määritelmä}
\index{derivaatta|emph} \index{derivoituvuus|emph} \index{linearisaatio (funktion)|emph}
%\index{funktion approksimointi!ab@linearisaatiolla|emph}
Funktio $f:\DF_f\kohti\R$, $\DF_f\subset\R$, on pisteessä $a\in\DF_f$ \kor{derivoituva} 
(engl.\ differentiable), jos $(a-\delta,a+\delta)\subset\DF_f$ jollakin $\delta>0$ ja on
olemassa raja-arvo
\[
\lim_{x\kohti a} \frac{f(x)-f(a)}{x-a} = k \in \R.
\]
Lukua $k$ kutsutaan $f$:n \kor{derivaataksi} (engl. derivative) pisteessä $a$, ja merkitään 
$k=f'(a)$. Jos $f$ on pisteessä $a$ derivoituva, niin polynomia 
\[ 
p(x) = f(a) + f'(a)(x-a) 
\]
sanotaan $f$:n \kor{linearisoivaksi approksimaatioksi} eli \kor{linearisaatioksi}
pisteessä $a$.
\end{Def}
Linearisoivan approksimaation geometrinen vastine on pisteen $P=(a,f(a))$ kautta kulkeva suora,
jonka
\index{kulmakerroin} \index{tangentti (käyrän)}%
\kor{kulmakerroin} on $k=f'(a)\ (=p'(a))$. Sanotaan, että ko.\ suora on käyrän
$S: y=f(x)$ \kor{tangentti} pisteessä $P$. Tämän mukaisesti siis 'derivaatta on tangentin
kulmakerroin'.
\begin{figure}[H]
\setlength{\unitlength}{1cm}
\begin{picture}(13,7)(-4,-2)
\put(0,-1.5){\vector(0,1){5.5}} \put(0.2,3.7){$y$}
\put(-2,0){\vector(1,0){10}} \put(7.7,-0.4){$x$}
\curve(0,0,0.5,0.0625,1,0.25,1.5,0.5625,2,1,2.5,1.5625,3,2.25,3.5,3.0625,4,4)
\put(0,-1){\line(1,1){4}}
\put(1.9,-0.4){$a$} \dashline{0.1}(2,0)(2,1)
\put(4.2,4){$S: y=f(x)$} \put(4,2.6){$y=p(x)$}
\put(1.93,0.93){$\scriptstyle{\bullet}$} \put(2.2,0.8){$P$}
\end{picture}
\end{figure}
\jatko \begin{Exa} (jatko) Esimerkin tuloksen perusteella funktio $\,f(x)=x^3\,$ on 
derivoituva jokaisessa pisteessä $a\in\R$ ja $f'(a)=3a^2$. Käyrän $S: y=x^3$ tangentin
yhtälö pisteessä $x=a$ on siis $y=p(x)=a^3+3a^2(x-a)$. \loppu \end{Exa}
\begin{Exa} \label{ei-derivoituva f} Funktio $f(x)=\abs{x-a}$ ei ole derivoituva pisteessä
$a$, sillä
\[
\frac{f(x)-f(a)}{x-a} \,=\, \begin{cases} 
                            \ \ 1, \quad &\text{kun}\ x>a, \\
                               -1, \quad &\text{kun}\ x<a. 
                            \end{cases} \loppu
\]
\end{Exa}
Jos $f$ on derivoituva pisteessä $a$, niin Määritelmän \ref{derivaatan määritelmä} ja
Lauseen \ref{funktion raja-arvojen yhdistelysäännöt} perusteelle pätee
\[
\lim_{x \kohti a} f(x) \,=\, \lim_{x \kohti a}\left[f(a)+(x-a)\,\frac{f(x)-f(a)}{x-a}\right]
                      \,=\, f(a) + 0 \cdot f'(a) = f(a).
\]
Siis derivoituvuus on jatkuvuutta vahvempi ominaisuus:
\[
\boxed{\kehys\quad f \text{ derivoituva pisteessä } a \ 
                              \impl \ f \text{ jatkuva pisteessä } a. \quad}
\]

\subsection{Derivaatta operaattorina}

Kun derivaatan määritelmässä kirjoitetaan $x-a=\Delta x$ ja asetetaan $a$:n tilalle $x$, niin
määritelmä saa muodon
\[
f'(x) = \lim_{\Delta x\kohti 0} \frac{f(x+\Delta x)-f(x)}{\Delta x}\,.
\]
Derivaatan muita merkintöjä ovat
\[
\boxed{\quad f'(x)=\frac{\ykehys df}{\akehys dx}=\frac{d}{dx}f=\dif f(x). \quad}
\]
Kahdessa viimeksi kirjoitetussa merkinnässä tulkitaan derivaatta kohteestaan erillisenä
\kor{operaattorina} eli 'funktion funktiona':
\[
\dif:f \map f', \quad \dif = \frac{d}{dx}\,.
\]
Operaattoreiksi sanotaan yleisemmin sellaisia funktioita, joiden sekä määrittely- että 
maalijoukkona ovat (esim.\ reaalimuuttujan) funktiot. Derivoinnin suorittavaa
'funktion funktiota' $\dif$ sanotaan
\index{differentiaalioperaattori!a@derivoinnin}%
\kor{differentiaalioperaattoriksi}.

Differentiaalioperaattori voi määritelmänsä mukaisesti toimia vain sellaisissa pisteissä,
joissa $f$ on derivoituva, ts. $f'$:n määrittelyjoukossa
\[
\DF_{f'}=\{x\in\DF_f \ | \ f \text{ derivoituva pisteessä } x\}.
\]
Useat tavalliset funktiot ovat derivoituvia 'melkein kaikkialla' niin, että $\DF_{f'}$
saadaan $\DF_f$:stä poistamalla enintään äärellinen tai numeroituva määrä pisteitä. 
\begin{Exa} Funktio $f(x)=x-[x]$, missä $[x]=$ suurin kokonaisluku, jolle pätee $[x] \le x$,
on derivoituva muualla paitsi pisteissä $k \in \Z$, eli $D_{f'}=\{x\in\R \mid x\not\in\Z\}$. 
Tässä joukossa on $f'(x)=1$. (Vrt.\ Esimerkki \ref{sahafunktio} edellisessä luvussa.) \loppu
\end{Exa}

Jos $f$:n derivaatta $f'$ on edelleen derivoituva pisteessä $x$, voidaan määritellä $f$:n 
\kor{toinen derivaatta} pisteessä $x$:
\[
f''(x)=\lim_{\Delta x\kohti 0} \frac{f'(x+\Delta x)-f'(x)}{\Delta x} = \dif f'(x)
      = \dif(\dif f(x)) = \dif^2 f(x).
\]
Pisteissä, joissa $f''$ on edelleen on derivoituva, voidaan määritellä kolmas derivaatta
$f'''(x)$ jne. Yleisesti funktion \kor{$n$:s derivaatta} pisteessä $x$ on
\[
f^{(n)}=\frac{d^n}{dx^n}f(x)=\dif^n f(x).
\]
\index{kertaluku!ba@derivaatan}%
Sanotaan, että $n$ on derivaatan $f^{(n)}$ \kor{kertaluku}, ja sovitaan, että $f^{(0)}=f$.

\subsection{Derivoimissäännöt}
\index{derivoimissäännöt!a@summa, tulo, osamäärä|vahv}

Yleisissä \kor{derivoimissäännöissä} esitetään laskukaavat funktioiden algebrallisten
johdannaisten 
\[
f,g,\lambda\ \map\ \lambda f,\ f+g,\ fg,\ f/g,\ f \circ g,\ f^{-1}
\]
derivaattojen laskemiseksi $f'$:n ja $g'$:n avulla. Derivoimissäännöistä yksinkertaisimmat
ovat
\begin{align*}
\dif(\lambda f) &= \lambda f', \quad \lambda\in\R, \\
\dif(f+g)       &= f'+g',
\end{align*}
jotka voidaan yhdistää säännöksi
\begin{equation} \label{D1}
\boxed{\kehys\quad \dif(\alpha f + \beta g)=\alpha \dif f + \beta \dif g, \quad 
                   \alpha,\beta\in\R. \quad}
\end{equation}
\index{lineaarisuus!a@derivoinnin}%
Sääntö \eqref{D1} merkitsee, että $\dif$ on \kor{lineaarinen} operaattori. Sääntö on pätevä
jokaisessa pisteessä, jossa $f$ ja $g$ ovat molemmat derivoituvia, kuten voidaan helposti
todeta raja-arvojen yhdistelysääntöjen (Lause \ref{funktion raja-arvojen yhdistelysäännöt})
perusteella. Samoin ehdoin pätee tulon derivoimissääntö
\begin{equation} \label{D2}
\boxed{\kehys\quad \dif(fg)=f'g+fg'. \quad}
\end{equation}
Tämän perustelemiseksi kirjoitetaan
\begin{multline*}
f(x+\Delta x)g(x+\Delta x)-f(x)g(x) \\
              =[f(x+\Delta x)-f(x)]g(x+\Delta x)+f(x)[g(x+\Delta x)-g(x)],
\end{multline*}
jaetaan puolittain $\Delta x$:llä, ja sovelletaan mainittuja raja-arvojen 
yhdistelysääntöjä. Huomioidaan myös, että $g$ on (derivoituvana) jatkuva $x$:ssä. 
\begin{Exa} \label{potderiv 1} Lähtemällä ilmeisestä derivoimissäännöstä $\dif x = 1$ ja 
soveltamalla sääntöä \eqref{D2} funktiojonoon
$f_1(x) = x,\ f_n(x)=xf_{n-1}(x),\ n = 2,3,\ldots\,$ nähdään induktiolla oikeaksi sääntö 
\[
\dif x^m=mx^{m-1}, \quad m\in\N\cup\{0\}.
\]
Toinen tapa johtaa tämä tulos on lähteä suoraan derivaatan määritelmästä ja käyttää
binomikaavaan perustuvaa hajotelmaa:
\begin{align*}
\frac{(x+\Delta x)^m-x^m}{\Delta x}\,=\ 
                      &mx^{m-1}+\binom{m}{2}x^{m-2}\Delta x + \ldots +(\Delta x)^{m-1} \\
              \kohti\ &mx^{m-1}, \quad \text{kun}\ \Delta x \kohti 0. \loppu
\end{align*}
\end{Exa}
Kun esimerkin tuloksen ohella huomioidaan myös sääntö \eqref{D1}, nähdään että polynomin
derivaatta on toinen polynomi (astetta alempi).
\begin{Exa}
\begin{align*}
f(x)       &=x^5-4x^4+6x^3+x^2-2x+3 \\
f'(x)      &=5x^4-16x^3+18x^2+2x-2 \\
f''(x)     &=20x^3-48x^2+36x+2 \\
f'''(x)    &=60x^2-96x+36 \\
f^{(4)}(x) &=120x-96 \\
f^{(5)}(x) &=120 \\
f^{(n)}(x) &=0, \quad n\geq 6 \loppu
\end{align*}
\end{Exa}

Jos $f$ ja $g$ ovat $n$ kertaa derivoituvia pisteessä $x$, niin soveltamalla säännön
\eqref{D2} oikealla puolella uudelleen samaa sääntöä yhdessä säännön \eqref{D1} kanssa
saadaan johdetuksi binomikaavaa (Propositio \ref{binomikaava}) muistuttava
\kor{Leibnizin sääntö} \index{Leibnizin sääntö}
\[
\boxed{\quad \dif^n(fg) = \sum_{k=0}^n \binom{\ykehys n}{\akehys k}\,f^{(k)}g^{(n-k)}. \quad}
\]

Jos $\,h(x)=(f/g)(x)$, niin tulon derivoimissääntö \eqref{D2} sovellettuna identiteettiin
$h(x)g(x)=f(x)\,$ antaa tuloksen $h'g+hg'=f'$. Pisteissä, joissa $g(x) \neq 0$, voidaan tästä
ratkaista $h$:n eli osamäärän derivoimissäännöksi
\begin{equation} \label{D3}
\boxed{\quad \dif\left(\frac{f}{g}\right)
           =\frac{\ykehys f'}{\akehys g}-\frac{fg'}{g^2}=\frac{f'g-fg'}{g^2}\,. \quad}
\end{equation}
Säännöstä \eqref{D3} sekä polynomin derivoimissäännöstä nähdään, että jokainen
rationaalifunktio $f=p/q$ ($p$ ja $q$ polynomeja) on koko määrittelyjoukossaan, eli joukossa
$\{x\in\R \mid q(x) \neq 0\}$, derivoituva mielivaltaisen monta kertaa ja että derivaatat ovat
ovat samassa joukossa määriteltyjä rationaalifunktioita.
\begin{Exa} \label{ratfunktion derivaatta} Säännön \eqref{D3} ensimmäisen laskukaavan mukaan
\[
\frac{d}{dx}\left(\frac{x}{1-x^2}\right) = \frac{1}{1-x^2}+\frac{2x^2}{(1-x^2)^2}
                                         = \frac{1+x^2}{(1-x^2)^2}\,. \loppu
\]
\end{Exa}
\begin{Exa} \label{potder 2} Soveltamalla sääntöä \eqref{D3} tapaukseen $f(x)=1$,
$g(x)=x^m,\ m\in\N$ nähdään, että Esimerkin \ref{potderiv 1} sääntö $\dif x^m = mx^{m-1}$ on
pätevä $\forall m\in\Z$. \loppu
\end{Exa}
\begin{Lause} (\vahv{Yhdistetyn funktion derivaatta}) \label{yhdistetyn funktion derivaatta}
\index{derivoimissäännöt!b@yhdistetty funktio|emph}% 
Jos $g$ on derivoituva pisteessä $x$ ja $f$ on derivoituva pisteessä $g(x)$, niin $f \circ g$
on derivoituva pisteessä $x$ ja
\[
\boxed{\kehys\quad \dif f(g(x))=f'(g(x))\,g'(x). \quad}
\]
\end{Lause}
\tod Merkitään $g(x+\Delta x)-g(x)=\Delta g$. Tällöin koska $f$ on derivoituva
pisteessä $g(x)$, niin pätee (vrt.\ derivaatan määritelmä edellä)
\[
f(g(x+\Delta x)) = f(g(x)+\Delta g) = f(g(x))+f'(g(x))\Delta g + h(\Delta g)\Delta g,
\]
missä $h(\Delta g) \kohti 0$, kun $\Delta g \kohti 0$. Koska $g$ on derivoituva pisteessä
$x$ ja koska $\,\Delta x \kohti 0\ \impl\ \Delta g \kohti 0$, niin seuraa
\begin{align*}
\dif (f\circ g)(x) &= \lim_{\Delta x \kohti 0}\,\frac{f(g(x+\Delta x))-f(g(x))}{\Delta x} \\
                   &= \lim_{\Delta x \kohti 0} \left[f'(g(x))\frac{\Delta g}{\Delta x} 
                                     + h(\Delta g)\frac{\Delta g}{\Delta x}\right] \\[2mm]
                   &= f'(g(x))g'(x) + 0 \cdot g'(x) = f'(g(x))g'(x). \loppu
\end{align*}

Soveltamalla Lauseen \ref{yhdistetyn funktion derivaatta} sääntöä ja Esimerkin \ref{potder 2}
tulosta funktioon $g\circ f$, missä $g(x)=x^m$, saadaan derivoimissääntö
\begin{equation} \label{D4}
\boxed{\kehys\quad \dif[f(x)]^m=m[f(x)]^{m-1}f'(x),\quad m\in\Z. \quad}
\end{equation}
\begin{Exa} Jatkamalla Esimerkistä \ref{ratfunktion derivaatta} säännöillä \eqref{D2},
\eqref{D4} ja \eqref{D1} saadaan
\begin{align*}
\frac{d^2}{dx^2}\left(\frac{x}{1-x^2}\right)=\frac{d}{dx}\left(\frac{1+x^2}{(1-x^2)^2}\right) 
                   &=\frac{2x}{(1-x^2)^2}+(1+x^2)\,\frac{(-2)(-2x)}{(1-x^2)^3} \\
                   &=\frac{6x+2x^3}{(1-x^2)^3}\,. \loppu
\end{align*}
\end{Exa}
Jos $f(x)=x^{m/n}$, $m\in\Z$, $n\in\N$, $x>0$, niin $\,[f(x)]^n=x^m$. Derivoimalla tässä
puolittain säännöllä \eqref{D4} saadaan
\[
n[f(x)]^{n-1}f'(x) = mx^{m-1} \qimpl f'(x) = (m/n)\,x^{m/n-1}.
\]
Näin on johdettu yleisen potenssifunktion derivoimissääntö
\[ 
\boxed{\kehys\quad \dif x^\alpha = \alpha x^{\alpha-1}, \quad x\in\R_+\,, \ \alpha\in\Q. \quad}
\]

\begin{Exa} Derivoi $\D f(x) = \sqrt[3]{1+\sqrt{x}}$. \end{Exa}
\ratk Potenssifunktion ja yhdistetyn funktion derivoimissääntöjen perusteella
\begin{align*}
f'(x) = \frac{1}{3}(1+\sqrt{x})^{-2/3}\cdot\frac{1}{2}\,x^{-1/2}
      &=\,\frac{(1+\sqrt{x})^{1/3}}{6\sqrt{x}\,(1+\sqrt{x})} \\
      &=\,\frac{\sqrt[3]{1+\sqrt{x}}}{6(x+\sqrt{x})}\,, \quad x>0. \loppu
\end{align*}

Yhdistetyn funktion derivoimissäännöstä (Lause \ref{yhdistetyn funktion derivaatta}) voi
johtaa myös yleisen käänteisfunktion derivoimissäännön: Oletetaan, että $f$ on derivoituva
pisteessä $x$ ja että käänteisfunktio $g=\inv{f}$ on derivoituva pisteessä $f(x)$. Tällöin
identiteetistä $\,g(f(x))=x\,$ seuraa derivoimalla puolittain, että $\ g'(f(x))f'(x)=1$.
Siis tehdyin oletuksin ja lisäoletuksella $f'(x) \neq 0$ pätee
\begin{equation} \label{D5}
\boxed{\quad \dif\inv{f}(y)=\frac{1}{f'(x)},\quad y=f(x). \quad}
\end{equation}
Käänteisfunktion derivoituvuuden perustelee tarkemmin seuraava lause.
\begin{Lause} (\vahv{Käänteisfunktion derivaatta}) \label{käänteisfunktion derivoituvuus}
\index{derivoimissäännöt!c@käänteisfunktio|emph}%
Olkoon $f: [a-\delta,a+\delta] \kohti U$ jatkuva bijektio jollakin $\delta>0$ ja olkoon
$f$ derivoituvua pisteessä $a$. Tällöin $\inv{f}: U \kohti [a-\delta,a+\delta]$ on
derivoituva pisteessä $b=f(a)$ täsmälleen kun $f'(a) \neq 0$, jolloin 
$\,\dif \inv{f}(b)=1/f'(a)$.
\end{Lause}
\tod Koska $f$ on jatkuva ja $1-1$ välillä $[a-\delta,a+\delta]$ ja $f$:n arvojoukko
ko.\ valillä $=U$, niin Lauseen \ref{ensimmäinen väliarvolause} mukaan pätee
$[b-\eps,b+\eps] \subset U$, missä
\[
\eps \,=\, \min\{\,\abs{f(a-\delta)-f(a)},\,\abs{f(a+\delta)-f(a)}\,\} \,>\, 0.
\]
Tällöin jos $f'(a) \neq 0$, niin tekemällä välillä $[b-\eps,b+\eps]$ muuttujan
vaihto $y=f(x)$ seuraa lauseiden \ref{raja-arvo sijoituksella} ja
\ref{funktion raja-arvojen yhdistelysäännöt} perusteella
\[
\lim_{y \kohti b} \frac{\invf(y)-\invf(b)}{y-b} 
           \,=\,\lim_{x \kohti a} \frac{x-a}{f(x)-f(a)}
           \,=\,\left[\lim_{x \kohti a} \frac{f(x)-f(a)}{x-a}\right]^{-1} 
           \,=\, \frac{1}{f'(a)}\,,
\]
joten $\inv{f}$ on derivoituva $b$:ssä ja $\dif\inv{f}(b)=1/f'(a)$ (Määritelmä
\ref{derivaatan määritelmä}). Jos $f'(a)=0$, seuraa Lauseesta \ref{raja-arvo sijoituksella}
vastaavasti, että
\[
\lim_{y \kohti b} \left|\frac{\invf(y)-\invf(b)}{y-b}\right|
            \,=\, \lim_{x \kohti a} \left|\frac{x-a}{f(x)-f(a)}\right|
            \,=\ \infty,
\]
joten tässä tapauksessa $\invf$ ei ole derivoituva $b$:ssä. \loppu
\begin{Exa} \label{algebrallinen käänteisfunktio: derivaatta} Funktiolle $f(x)=x^5+3x$ 
Lauseen \ref{käänteisfunktion derivoituvuus} oletukset ovat voimassa jokaisella $a\in\R$
(myös jokaisella $\delta>0$, ks.\ Esimerkki \ref{algebrallinen käänteisfunktio} Luvussa
\ref{käänteisfunktio}), joten sääntö \eqref{D5} on soveltuva: 
\[
\dif\inv{f}(y)=\frac{1}{5x^4+3}, \quad y=x^5+3x.
\]

Esimerkiksi $\dif\inv{f}(0)=1/3$, koska $y=0 \impl x=0$. Jos halutaan laskea 
$\dif\inv{f}(1)$, on ensin ratkaistava (numeerisesti) yhtälö $x^5+3x=1$. Jos ratkaisu
merkitään $x=a$, niin $\dif\inv{f}(1)=1/(5a^4+3)$. \loppu 
\end{Exa}

\subsection{Implisiittinen derivointi}
\index{implisiittinen derivointi|vahv} \index{derivoimissäännöt!d@implisiittifunktio|vahv}

Jos funktio $y(x)$ on määritelty implisiittisesti muodossa (vrt.\ Luku \ref{käänteisfunktio})
\[ 
F(x,y) = 0 \qekv y = y(x), 
\]
niin derivaatta $y'(x)$ on mahdollista laskea derivoimalla puolittain yhtälö
\[ 
F(x,y(x)) = 0.
\]
Tällaista epäsuoraa menettelyä sanotaan \kor{implisiittiseksi derivoinniksi}. Toistaiseksi
tunnetuilla derivointisäännöillä implisiittinen derivointi onnistuu esim.\ silloin kun
funktio F on (kahden muuttujan) polynomi tai rationaalifunktio.
\jatko \begin{Exa} (jatko) Jos esimerkissä merkitään $\inv{f}(x)=y(x)$, niin $y(x)$ voidaan
tulkita implisiittifunktioksi, joka määräytyy yhtälöstä $\,y^5+3y=x$, eli
\[
[y(x)]^5+3y(x)=x, \quad x\in\R.
\]
Kun tässä derivoidaan puolittain soveltaen yhdistetyn funktion derivoimissääntöä ja sääntöä
\eqref{D1}, niin seuraa
\[ 
\bigl[5[y(x)]^4+3\bigr]\,y'(x)=1\ \impl\ \ y'(x)=\frac{1}{5y^4+3} \quad (y=y(x)). \loppu
\]
\end{Exa}
\begin{Exa} Yhtälö 
\[ 
x^2 + y^2 = 1 
\]
määrittelee välillä $[-1,1]$ kaksihaaraisen implisiittifunktion, jonka kumpikin haara on
derivoituva välillä $(-1,1)$. Jos $x\in(-1,1)$, niin $y'(x)$ lasketaan helposti
implisiittisellä dervoinnilla:
\begin{align*}
 \dif[x^2 + (y(x))^2] &= 2x + 2y(x)y'(x) = \dif\,1 = 0
                      &\impl \quad y'(x) = -\,\dfrac{x}{y}\,,\ \ y=y(x).
\end{align*}
Tulos on pätevä kummallakin implisiittifunktion haaralla
\[ 
y(x) = \pm \sqrt{1-x^2}\,, 
\]
kuten myös suoralla derivoinnilla voi todeta:
\[
y(x) = \pm\sqrt{1-x^2}\,\ \impl\,\ y'(x) = \pm\left(\frac{-x}{\sqrt{1-x^2}}\right) 
                                         = -\,\frac{x}{y(x)}\,, \quad x \in (-1,1). \loppu
\]
\end{Exa}

\subsection{Potenssisarjan derivointi}

Tähänastisten tarkastelujen perusteella voidaan pitää pääsääntönä, että tavanomaiset
(algebran keinoin lausekkeina määritellyt) funktiot ovat derivoituvia määrittelyjoukossaan,
lukuunottamatta mahdollisia erillisiä pisteitä, joita voi olla äärellinen tai joskus
numeroituva määrä. Näytetään luvun lopuksi, että pääsäännöstä eivät tee poikkeusta myöskään
funktiot, jotka on määritelty potenssisarjojen summina. Tällaisten funktioiden myötä
derivoituvien funktioiden joukko laajenee itse asiassa merkittävästi, kuten myöhemmin tullaan
näkemään.

Olkoon reaalifunktio $f$ määritelty muodossa
\[
f(x) = \sum_{k=0}^\infty a_k x^k.
\]
Oletetaan, että tässä potenssisarjan suppenemissäde on joko $\rho\in\R_+$ tai $\rho=\infty$.
Tällöin suurin avoin väli, joka sisältyy $f$:n määrittelyjoukkoon, on $(-\rho,\rho)$
(vrt.\ Luku \ref{potenssisarja}). Jatkossa näytetään, että $f$ on derivoituva koko tällä
välillä ja että derivaatta voidaan laskea yksinkertaisesti derivoimalla sarja termeittäin,
ts.\ pätee
\[
f'(x) = \sum_{k=1}^\infty k a_k x^{k-1}, \quad x \in (-\rho,\rho).
\]
Derivoinnin tuloksena saadun sarjan suppenemissäde on myös $\rho$ (Lause
\ref{potenssisarjan skaalaus}), joten $f'$ on edelleen derivoituva välillä $(-\rho,\rho)$,
ja derivoinnin tulos on siis
\[
f''(x) = \sum_{k=2}^\infty k(k-1)\,a_k x^{k-2}, \quad x \in (-\rho,\rho).
\]
Myös tämän sarjan suppenemissäde $=\rho$ (Lause \ref{potenssisarjan skaalaus}),
joten $f''$ on välillä $(-\rho,\rho)$ edelleen derivoituva, jne. Päätellään, että
potenssisarjan summana määritelty funktio on avoimella suppenemisvälillään itse asiassa
mielivaltaisen monta kertaa derivoituva funktio (!). Päätelmä perustui siis potenssisarjojen
suppenemisteoriaan ja seuraavaan väittämään, joka on todistettavissa suoraan derivaatan
määritelmästä lähtien (eli aiempiin derivoimissääntöihin vetoamatta).
\begin{Lause} \label{potenssisarja on derivoituva} \vahv{(Potenssisarjan derivaatta)}\,
\index{potenssisarja!b@derivointi|emph} \index{derivoimissäännöt!e@potenssisarja|emph}%
Jos potenssisarjan $\sum_{k=0}^\infty a_k x^k$ suppenemissäde on $\rho>0$, niin sarjan summana 
määritelty funktio $f(x)$ on derivoituva välillä $(-\rho,\rho)$ ja ko.\ välillä pätee 
$f'(x)=\sum_{k=1}^\infty k a_k x^{k-1}$. 
\end{Lause}
\tod Olkoon $x\in (-\rho,\rho)$ ja valitaan $\Delta x \neq 0$ siten, että
\[
\abs{x}+\abs{\Delta x} \le \rho_0<\rho.
\]
Binomikaavan mukaan
\begin{align*}
(x+\Delta x)^k\ &=\ \sum_{l=0}^k \binom{k}{l} (\Delta x)^l x^{k-l} \\
                &=\ x^k + kx^{k-1}\Delta x + \sum_{l=2}^k \binom{k}{l} (\Delta x)^l x^{k-l}.
\end{align*}
Kirjoitetaan tässä viimeinen termi summausindeksin vaihdolla muotoon
\begin{align*}
\sum_{l=2}^k \binom{k}{l} (\Delta x)^l x^{k-l}\ 
              &= \sum_{i=0}^{k-2} \binom{k}{i+2} (\Delta x)^{i+2} x^{k-2-i} \\
              &=\ (\Delta x)^2 \sum_{i=0}^{k-2} \binom{k}{i+2} (\Delta x)^i x^{k-2-i},
\end{align*}
ja edelleen
\[
\sum_{i=0}^{k-2} \binom{k}{i+2} (\Delta x)^i\,x^{k-2-i} 
                = \sum_{i=0}^{k-2} c_i\,\binom{k-2}{i} (\Delta x)^i\,x^{k-2-i},
\]
missä
\[
c_i = \binom{k}{i+2} \binom{k-2}{i}^{-1} 
    = \frac{k!}{(k-i-2)!\,(i+2)!} \cdot \frac{(k-i-2)!\,i!}{(k-2)!} 
    = \frac{k(k-1)}{(i+1)(i+2)}\,.
\]
Koska
\[
c_i \le \frac{1}{2}\,k(k-1) < \frac{1}{2}\,k^2, \quad i = 0 \ldots k-2,
\]
ja oli $\abs{x}+\abs{\Delta x} \le \rho_0$, niin saadaan jokaisella $k \ge 2$ arvio
\begin{align*}
\left|\frac{(x+\Delta x)^k - x^k}{\Delta x} - kx^{k-1}\right|\ 
 &=\ \abs{\Delta x} \left|\sum_{i=0}^{k-2} c_i\,\binom{k-2}{i} (\Delta x)^i\,x^{k-2-i}\right| \\
 &\le\ \abs{\Delta x} \sum_{i=0}^{k-2} c_i\,\binom{k-2}{i} \abs{\Delta x}^i\,\abs{x}^{k-2-i} \\
 &\le\ \frac{1}{2}k^2\abs{\Delta x} 
               \sum_{i=0}^{k-2} \binom{k-2}{i} \abs{\Delta x}^i\,\abs{x}^{k-2-i} \\
 &= \frac{1}{2}k^2\abs{\Delta x}(\abs{x}+\abs{\Delta x})^{k-2} \\
 &\le \frac{1}{2}k^2\abs{\Delta x}\rho_0^{k-2}.
\end{align*}
Näin ollen
\begin{align*}
\left| \frac{f(x+\Delta x)-f(x)}{\Delta x} - \sum_{k=1}^\infty ka_kx^{k-1} \right|\ 
       &=\ \left|\sum_{k=2}^\infty a_k
                 \left[\frac{(x+\Delta x)^k - x^k}{\Delta x} - kx^{k-1}\right]\right| \\
       &\le\ \sum_{k=2}^\infty \abs{a_k}
             \left|\frac{(x+\Delta x)^k - x^k}{\Delta x} - kx^{k-1}\right| \\
       &\le\ \abs{\Delta x}\sum_{k=2}^\infty \frac{1}{2}k^2\abs{a_k}\rho_0^{k-2}.
\end{align*}
Tässä oikealla oleva sarja suppenee Lauseen \ref{potenssisarjan skaalaus} perusteella, koska
$\rho_0<\rho$, joten
\[
\left|\frac{f(x+\Delta x)-f(x)}{\Delta x} - \sum_{k=1}^\infty ka_kx^{k-1}\right| 
       \le\ C\,\abs{\Delta x}, \quad\ C = \sum_{k=2}^\infty \frac{1}{2}k^2\abs{a_k}\rho_0^{k-2}.
\]
Saatu arvio on pätevä, kun $x \in (-\rho,\rho)$ ja $\abs{\Delta x} < \delta$, missä esim.\ 
\[ 
\delta = \frac{1}{2}\,(\rho-\abs{x}) > 0. 
\]
Siis $f$ on jokaisessa pisteessä $x \in (-\rho,\rho)$ derivoituva ja
\[
f'(x) = \lim_{\Delta x \kohti 0}\,\frac{f(x+\Delta x)-f(x)}{\Delta x}\ 
      =\ \sum_{k=1}^\infty ka_kx^{k-1}. \loppu
\]
\begin{Exa} Laske $\,\sum_{k=1}^\infty k q^k$, kun $\abs{q}<1$. \end{Exa}
\ratk Derivoidaan funktio
\[ 
f(x) = \sum_{k=0}^\infty x^k = \frac{1}{1-x}\,, \quad x \in (-1,1) 
\]
toisaalta Lauseen \ref{potenssisarja on derivoituva} perusteella ja toisaalta 
rationaalifunktiona:
\[ 
f'(x) \,=\, \sum_{k=1}^\infty kx^{k-1} \,=\, \frac{1}{(1-x)^2}\,, \quad x\in(-1,1). 
\]
Valitaan $x=q$ ja kerrotaan puolittain $q$:lla:
\[
q\sum_{k=1}^\infty kq^{k-1} \,=\, \sum_{k=1}^\infty kq^k \,=\, \frac{q}{(1-q)^2}\,. \loppu
\]

\Harj
\begin{enumerate}

\item
Laske suoraan määritelmästä (derivoimissääntöjä käyttämättä) linearisoiva approksimaatio
seuraaville funktioille pisteessä $a=1$. Piirrä myös kuva! 
\[
\text{a)}\,\ f(x)=\frac{x}{1+2x} \qquad 
\text{b)}\,\ f(x)=\frac{x}{2+x^2} \qquad
\text{c)}\,\ f(x)=\sqrt{10+6x}
\]

\item
Laske seuraavien funktioiden derivaatat pisteessä $a\in\DF_f$ suoraan derivaatan
määritelmästä:
\[
\text{a)}\,\ f(x)=x^3+3x^2 \qquad 
\text{b)}\,\ f(x)=\frac{1}{x} \qquad
\text{c)}\,\ f(x)=\frac{1}{\sqrt{x}}
\]

\item
Olkoon $f$ derivoituva pisteessä $a$. Määritä seuraavat raja-arvot $f(a)$:n ja $f'(a)$:n
avulla:
\begin{align*}
&\text{a)}\ \lim_{h \kohti 0} \frac{f(a+h^2)-f(a-h)}{h} \qquad
 \text{b)}\ \lim_{x \kohti a} \frac{xf(a)-af(x)}{x-a} \\
&\text{c)}\ \lim_{t \kohti 0^+} \frac{f(a+\alpha t)-f(a+\beta t)}{t}\,,\,\ \alpha,\beta\in\R
\end{align*}

\item \label{H-V-4: trig-esim}
Lähtien derivaatan määritelmästä määritä $f'(0)$ tai näytä, että $f$ ei ole derivoituva
$0$:ssa:
\begin{align*}
&\text{a)}\,\ f(x)=\begin{cases}
                   \,x(1+\sqrt{\abs{x}}\cos\dfrac{1}{x}, &\text{kun}\ x \neq 0, \\[2mm] 
                   \,0,                                  &\text{kun}\ x=0
                   \end{cases} \\[2mm]
&\text{b)}\,\ f(x)=\begin{cases}
                   \,x\sin^2\dfrac{1}{x}\,, &\text{kun}\ x \neq 0, \\[2mm] 
                   \,0,                     &\text{kun}\ x=0
                   \end{cases} \\[2mm]
&\text{c)}\,\ f(x)=\begin{cases}
                   \,1+x\cos\dfrac{1}{x}\sqrt[100]{x\left|\sin\dfrac{1}{x}\right|},
                                            &\text{kun}\ x>0, \\[2mm]
                   \,1+x^2,                 &\text{kun}\ x \le 0
                   \end{cases}
\end{align*}

\item
Palauta osamäärän derivoimissääntö tulon derivoimissääntöön johtamalla ensin funktion
$1/f$ derivoimissääntö suoraan derivaatan määritelmästä.

\item
Laske kohdissa a)--c) $f^{(n)}(x)$ kun $n=1,2$ ja kohdissa d)--f) derivaatan
$f^{(n)}(x),\ n\in\N$ yleinen lauseke:
\begin{align*}
&\text{a)}\,\ f(x)=\frac{x}{\sqrt{a^2-x^2}} \qquad 
 \text{b)}\,\ f(x)=\frac{1}{\sqrt{x}+1} \qquad
 \text{c)}\,\ f(x)=\sqrt{x+\sqrt{x}} \\
&\text{d)}\,\ f(x)=\frac{1}{1+x} \qquad\quad\ \ 
 \text{e)}\,\ f(x)=\sqrt{x+1} \qquad\
 \text{f)}\,\ f(x)=\frac{1-x}{1+x}
\end{align*}

\item
Tiedetään, että $f(2)=1$, $f'(2)=3$ ja $f''(2)=-1$. Laske
\[
\text{a)}\ \left[\frac{d}{dx}\left(\frac{x^2}{f(x)}\right)\right]_{x=2} \qquad
\text{b)}\ \left[\frac{d}{dx}\left(\frac{f(x)}{x^2}\right)\right]_{x=2} \qquad
\text{c)}\ \left[\frac{d^2}{dx^2}\left(\frac{f(x)}{x^2}\right)\right]_{x=2}
\]

\item
Funktio $f$ määritellän funktion $g(x)=x^2-3x+4$ kaksihaaraisena käänteisfunktiona. Laske
$f'(2)$ molemmilla haaroilla \ a) ratkaisemalla ensin yhtälö $g(x)=2$ ja käyttämällä
käänteisfunktion derivoimissääntöä, \ b) ratkaisemalla $\,g(y)=x\,\impl\,y=f(x)$,
derivoimalla ja sijoittamalla $x=2$. 

\item
Seuraavat yhtälöt määrittelevät implisiittifunktion $y(x)$. Laske implisiittisellä
derivoinnilla $y'(x)$ annetussa pisteessä $(x,y)=(x,y(x))$\,: \newline
a) \ $2x^2+3y^2=5,\ (x,y)=(1,1)$ \newline
b) \ $2x^2+3y^2=5,\ (x,y)=(1,-1)$ \newline
c) \ $x^2y^3-x^3y^2=12,\ (x,y)=(-1,2)$ \newline
d) \ $(x-1)(x+2y+1)=y^2,\ (x,y)=(2,-1)$

\item
Funktio $g$ on funktion $f$ käänteisfunktio eli $y=f(x)\,\ekv\, x=g(y)$. Johda
derivoimissääntö
\[
g''(y)=-\frac{f''(x)}{[f'(x)]^3}\,, \quad y=f(x)
\]
ja laske säännöllä $g''(1)$, kun $f(x)=x^5+2x+1$.

\item
Laske implisiittisesti derivoimalla $y''$ $x$:n ja $y$:n avulla:
\[
\text{a)}\,\ xy=x+y \qquad \text{b)}\,\ x^2+y^2=1 \qquad \text{c)}\,\ x^3-y^2+y^3=x
\]

\item \label{H-V-3: cos ja sin potenssisarjoina}
a) Näytä, että $\,xy''=y$, kun määritellään
\[
y(x)=\sum_{k=1}^\infty \frac{x^k}{k!(k-1)!}\,, \quad x\in\R.
\]
b) Näytä, että $\,u'=-v$ ja $\,v'=u$, kun määritellään
\[
u(x)=\sum_{k=0}^\infty (-1)^k \frac{x^{2k}}{(2k)!}\,, \quad
v(x)=\sum_{k=0}^\infty (-1)^k \frac{x^{2k+1}}{(2k+1)!}\,, \quad x\in\R.
\]
 
\item
Seuraavat funktiot ovat rationaalifunktioita välillä $(-1,1)$. Laske funktioiden lausekkeet
potenssisarjaa $\sum_{k=0}^\infty x^k$ derivoimalla.
\begin{align*}
&\text{a)}\ f(x)=\sum_{k=2}^\infty kx^k \qquad 
 \text{b)}\ f(x)=\sum_{k=3}^\infty (-1)^k kx^k \qquad
 \text{c)}\ f(x)=\sum_{k=1}^\infty k^2 x^k \\
&\text{d)}\ f(x)=\sum_{k=0}^\infty (k+2)^2 x^k \quad\ \
 \text{e)}\ f(x)=\sum_{k=1}^\infty k^3 x^k \quad\ \
 \text{f)}\ f(x)=\sum_{k=0}^\infty (k+1)^3 x^k 
\end{align*}

\item (*)
Todista Leibnizin sääntö.

\item (*) 
Olkoon $a\in\R$ reaalikertoimisen polynomin $p$ nollakohta ja olkoon $m\in\N,\ m\ge 2$.
Näytä, että $a$ on $m$-kertainen nollakohta täsmälleen kun $p^{(k)}(a)=0,\ k=1 \ldots m-1$
ja $p^{(m)}(a) \neq 0$.

\item (*)
a) Näytä, että jos potenssisarja $\sum_k a_k x^k$ suppenee välillä $(-\rho,\rho)$ ja
funktio $g$ on derivoituva välillä $(a,b)\subset(-\rho,\rho)$, niin funktion
\[
f(x) = \sum_{k=0}^\infty x^k g(x), \quad x\in(a,b)
\]
derivaatta välillä $(a,b)$ on laskettavissa derivoimalla sarja
termeittäin.\vspace{1mm}\newline
b) Määritä $f'(x)$ ja $f''(x)$ välillä $(0,\infty)$, kun
\[
f(x) = \sum_{k=0}^\infty \frac{1}{k!}\,x^{k+\tfrac{1}{2}}, \quad x \ge 0.
\]

\end{enumerate} % Derivaatta
\section{Trigonometristen funktioiden derivointi} \label{kaarenpituus}
\alku

Trigonometriset funktiot reaalifunktioina perustuvat kaarenpituuden käsitteeseen
(yksikköympyrällä, ks.\ Luku \ref{trigonometriset funktiot}). Käsite on aiemmin määritelty
lyhyesti Luvussa \ref{geomluvut}. Seuraavassa tarkastellaan aluksi kaarenpituutta hieman
yleisemmältä kannalta ja johdetaan näiden tarkastelujen pohjalta raja-arvotulos
\[
\lim_{x \kohti 0} \frac{\sin x}{x} = 1.
\]
Trigonometristen funktioiden kaikki derivoimissäännöt ovat tästä perustuloksesta ja Luvun
\ref{derivaatta} säännöistä johdettavissa.

\subsection{Kaarenpituus}
\index{kaarenpituus|vahv}

Olkoon $f:\DF_f\kohti\R$, $\DF_f\subset\R$, jatkuva suljetulla välillä $[a,b]\subset \DF_f$.
Merkitään
\[
A=(a,f(a)), \quad B=(b,f(b))
\]
ja sanotaan, että euklidisen tason käyrä
\[
S=\{P=(x,y)\in\Ekaksi \mid x\in [a,b] \ \ja \ y=f(x)\}
\]
\index{kaari (käyrän)} \index{yksinkertainen!c@käyrä, kaari}%
on pisteitä $A,B$ yhdistävä (yksinkertainen) \kor{kaari} (tai käyrän kaari, engl.\ arc).
\begin{figure}[H]
\setlength{\unitlength}{1cm}
\begin{center}
\begin{picture}(10,5)(-1,-1)
\put(-1,0){\vector(1,0){10}} \put(8.8,-0.4){$x$}
\put(0,-1){\vector(0,1){5}} \put(0.2,3.8){$y$}
\put(0.9,1.9){$\bullet \ A$} \put(7.9,2.9){$\bullet \ B$}
\curve(
    1.0000,    2.0000,
    1.5000,    2.7635,
    2.0000,    3.0000,
    2.5000,    2.8848,
    3.0000,    2.5679,
    3.5000,    2.1733,
    4.0000,    1.8000,
    4.5000,    1.5211,
    5.0000,    1.3843,
    5.5000,    1.4117,
    6.0000,    1.6000,
    6.5000,    1.9202,
    7.0000,    2.3179,
    7.5000,    2.7130,
    8.0000,    3.0000)
\put(4.5,1.8){$S$}
\end{picture}
\end{center}
\end{figure}
Olkoon edelleen $\{P_0,\ldots,P_n\}$, $n\in\N$, äärellinen, järjestetty pistejoukko kaarella
$S$ siten, että
\[
P_i=(x_i,f(x_i)),\quad i=0,\ldots,n,
\]
missä
\[
a=x_0<x_1<\ldots <x_n=b,
\]
jolloin siis $P_0=A$, $P_n=B$. Merkitään tällaista pistejoukkoa symbolisesti $\mathcal{P}$:llä,
\[
\mathcal{P}=\{P_0,\ldots,P_n\}\subset S,
\]
ja kaikkien tällaisten pistejoukkojen joukkoa $\mathcal{M}$:llä. Joukko $\mathcal{M}$ on 
ylinumeroituva, sillä jo pisteellä $P_1$ on ylinumeroituva määrä vaihtoehtoja.

Pistejoukon $\mathcal{P}\in\mathcal{M}$ kautta kulkevaa murtoviivaa, päätepisteinä $A$ ja $B$, 
sanottakoon pisteistön $\mathcal{P}$ kautta kulkevaksi kaaren $S$ \kor{oikopoluksi}.
\begin{figure}[H]
\setlength{\unitlength}{1cm}
\begin{center}
\begin{picture}(10,5)(-1,-1)
\put(-1,0){\vector(1,0){10}} \put(8.8,-0.4){$x$}
\put(0,-1){\vector(0,1){5}} \put(0.2,3.8){$y$}
\put(0.9,1.9){$\bullet \ A$} \put(7.9,2.9){$\bullet \ B$}
\curve(
    1.0000,    2.0000,
    1.5000,    2.7635,
    2.0000,    3.0000,
    2.5000,    2.8848,
    3.0000,    2.5679,
    3.5000,    2.1733,
    4.0000,    1.8000,
    4.5000,    1.5211,
    5.0000,    1.3843,
    5.5000,    1.4117,
    6.0000,    1.6000,
    6.5000,    1.9202,
    7.0000,    2.3179,
    7.5000,    2.7130,
    8.0000,    3.0000)
\put(3.7,1.2){$S$}
\put(2.4,2.8){$\bullet$} \put(2.4,3.1){$P_1$}
\put(5.4,1.3){$\bullet$} \put(5.4,0.9){$P_2$}
\path(1,2)(2.5,2.88)(5.5,1.41)(8,3)
\end{picture}
\end{center}
\end{figure}
Jokaisella oikopolulla on (geometrinen) \kor{pituus}, jota merkitään symbolilla
$s_\mathcal{P}$ ja määritellään
\[
s_\mathcal{P}=\sum_{i=1}^n \sqrt{(x_i-x_{i-1})^2+[f(x_i)-f(x_{i-1})]^2}\,.
\]
\index{suoristuva (käyrä)}%
Sanotaan, että kaari $S$ on \kor{suoristuva} (engl. rectifiable), jos joukko
\[
\mathcal{S}=\{s_\mathcal{P} \ | \ \mathcal{P}\in\mathcal{M}\}
\]
on (reaalilukujoukkona) \pain{ra}j\pain{oitettu}. Tässä tapauksessa kaarelle $S$ voidaan
määritellä  \kor{kaarenpituus}. Nimittäin, koska rajoitetulla joukolla on pienin yläraja eli
supremum (vrt.\ Luku \ref{reaalilukujen ominaisuuksia}), niin luontevalta tuntuva määritelmä on
\[
\boxed{\kehys\quad S\text{:n pituus} = s = \sup\mathcal{S}. \quad}
\]
Pelkästään $f$:n jatkuvuudesta \pain{ei} seuraa, että kaari $S$ on suoristuva (vaikka
vasta\-esimerkkejä ei ole aivan helppo löytää). Jatkossa tehdään funktiosta $f$ sen vuoksi 
hieman voimakkaampi oletus, joka takaa kaarenpituuden olemassaolon. Oletus on lievennettävissä
koskemaan välin $[a,b]$ osavälejä (ks.\ alaviite jäljempänä), joten tavanomaisia funktioita 
ajatellen lisäoletus on melko viaton.
\begin{Lause} \label{kaarenpituuslause}
Jos $f$ on välillä $[a,b]$ jatkuva ja lisäksi monotoninen, niin kaari
$S=\{P=(x,y)\in\Ekaksi \mid x\in [a,b] \ja y=f(x)\}$ on suoristuva, ja kaarenpituudelle $s$
pätee $\,b-a \,\le\, s \,\le\, b-a + \abs{f(b)-f(a)}$.
\end{Lause}
\tod  Koska $f$ on monotoninen, niin $s_\mathcal{P}$:n lausekkeessa luvut $f(x_i)-f(x_{i-1})$
ovat samanmerkkisiä (tai nollia), jolloin epäyhtälöä $\,\sqrt{x^2+y^2}\le|x|+|y|\,$ ensin
soveltaen seuraa
\begin{align*}
s_\mathcal{P}\, &\le\, \sum_{i=1}^n [\,(x_i-x_{i-1})+|f(x_i)-f(x_{i-1})|\,] \\
               &=\, \sum_{i=1}^n (x_i-x_{i-1})+\left|\sum_{i=1}^n [f(x_i)-f(x_{i-1})]\right|
             \,=\, b-a+\abs{f(b)-f(a)}.
\end{align*}
Todetaan myös, että $s_\mathcal{P} \ge b-a\ \forall\ \mathcal{P}$, sillä pienin mahdollinen
$s_\mathcal{P}$ (jokaisella $\mathcal{P}$) saadaan, kun $f$ on vakio. Tehdyin oletuksin 
siis pätee
\[
b-a\ \le\ s_\mathcal{P}\ \le\ b-a+\abs{f(b)-f(a)} \quad \forall\ s_\mathcal{P}\in\mathcal{S}.
\]
Joukolle $\mathcal{S}$ on näin saatu alaraja ja yläraja, joten luku $s=\sup\mathcal{S}$ on
olemassa ja saadut rajat pätevät myös tälle luvulle. \loppu

Ellei $S$ itse satu olemaan murtoviiva (jolloin $s\in\mathcal{S}$), on kaarenpituus määrättävä
käytännössä konstruoimalla jono $\mathcal{P}_n\in\mathcal{M}$, $n=1,2,\ldots$ niin, että 
$s_{\mathcal{P}_n}=s_n\kohti s$, kun $n\kohti\infty$. Esimerkiksi jos valitaan 
$\mathcal{P}_n = \{a + k(b-a)/n,\ k = 0 \ldots n\}$, niin algoritmi toimiikin tavanomaisissa
tapauksissa (vrt.\ Harj.teht.\,\ref{H-V-4: numeerinen
kaarenpituus}).\footnote[2]{Kaarenpituuden laskemiseen palataan myöhemmin toisessa
asiayhteydessä. Tällöin myös näytetään, että kaarenpituus on tarkasteltavan välin suhteen
\kor{additiivinen}, ts.\ jos $a<c<b$, niin $s=s_1+s_2$, missä $s$, $s_1$ ja $s_2$ ovat
vastaavasti kaarenpituudet välillä $[a,b]$, $[a,c]$ ja $[c,b]$. Additiivisuuden ja Lauseen
\ref{kaarenpituuslause} perusteella käyrän $S: y=f(x)$ kaaren suoristuvuuteen välillä $[a,b]$
riittää jatkuvuusoletuksen lisäksi, että väli $[a,b]$ on jaettavissa äärellisen moneen
osaväliin, joilla $f$ on monotoninen.}

Jos koko välin $[a,b]$ sijasta tarkastellaan väliä $[a,x]$, $a<x \le b$, ja määritellään
\[
S_x=\{P=(t,f(t)) \ | \ a\leq t\leq x\},
\]
niin $S_x$:n kaarenpituudesta $s(x)$ tulee välillä $[a,b]$ määritelty funktio, kun vielä 
asetetaan $s(a)=0$. Koska $s(x_2)-s(x_1)$ tällöin merkitsee kaarenpituutta välillä $[x_1,x_2]$
(ks.\ alaviite), on Lauseen \ref{kaarenpituuslause} mukaan
\[
s(x_2)-s(x_1)\,\ge\,x_2-x_1,\quad a \le x_1 < x_2 \le b,
\]
joten $s(x)$ on välillä $[a,b]$ aidosti kasvava. Lisäksi nähdään Lauseen
\ref{kaarenpituuslause} yläraja-arviosta, että $s(x)$ on jatkuva välillä $[a,b]$.

Siirrytään nyt tarkastelemaan jatkon kannalta tärkeää erikoistapausta, missä $[a,b]=[0,1]$ ja 
funktio $f$ määritellään
\[
f(x)=1-\sqrt{1-x^2}, \quad x\in [0,1].
\]
Lauseen \ref{kaarenpituuslause} oletukset toteutuvat tällöin, ja $S$ on $x$-akselia origossa
sivuavan 1-säteisen ympyrän kaari.
\begin{figure}[H]
\setlength{\unitlength}{1cm}
\begin{center}
\begin{picture}(6,5)(-1,-1)
\put(-1,0){\vector(1,0){6}} \put(4.8,-0.4){$x$}
\put(0,-1){\vector(0,1){5}} \put(0.2,3.8){$y$}
\put(3,0){\line(0,-1){0.15}} \put(2.92,-0.5){$\scriptstyle{1}$}
\put(0,3){\line(-1,0){0.15}} \put(-0.4,2.9){$\scriptstyle{1}$}
\put(0,3){\arc{6}{0}{1.57}}
\put(-0.1,-0.1){$\bullet$} \put(2.9,2.9){$\bullet$}
\path(0,3)(2.25,1.05)(0,1.05)
\dashline{0.1}(2.25,1.05)(2.25,0)
\put(0,1.05){$\overbrace{\hspace{2.25cm}}^x$}
\put(1,0.5){$\scriptstyle{t}$} \put(3,2){$S$} \put(2.18,-0.2){$\scriptstyle{x}$}
\end{picture}
\end{center}
\end{figure}
Kun merkitään $t=$ kaarenpituus välillä $[0,x]$, niin (ks.\ kuvio)
\[
x=\sin t.
\]
Kun $x$:n ja $t$:n välille oletetaan tämä yhteys välillä 
$x\in[0,1]\ \ekv\ t\in[0,\tfrac{\pi}{2}]$, niin ko.\ välillä voidaan kirjoittaa
\[
f(x) \,=\, 1+\sqrt{1-x^2} \,=\, 1-\cos t
\]
ja arvioida Lauseen \ref{kaarenpituuslause} perusteella
\[
\begin{cases} \,t \ge x, \\ \,t \le x+f(x) \end{cases} \qekv
\begin{cases} \,t \ge \sin t, \\ \,t \le \sin t + 1-\cos t \end{cases}
\]
eli
\[
t-(1-\cos t) \,\le\, \sin t \,\le\, t, \quad t\in[0,\tfrac{\pi}{2}].
\]

Tässä on viitattu aiemmin (Luku \ref{geomluvut}) sovittuun merkintään ($\pi$:n määritelmä!)
\[
S\text{:n pituus}=\tfrac{\pi}{2}.
\]
Kun saadun epäyhtälön vasemmalla puolella käytetään trigonometrian kaavaa
$\,1-\cos t = 2\sin^2\tfrac{t}{2}$ (ks.\ Luku \ref{trigonometriset funktiot}) ja sovelletaan
epäyhtälön jälkimmäistä osaa, seuraa
\[
t-(1-\cos t) \,=\, t-2\sin^2\tfrac{t}{2} 
             \,\ge\, t-2\left(\tfrac{t}{2}\right)^2 
             \,=\, t-\tfrac{1}{2}t^2, \quad t\in[0,\pi].
\]
Yhdistämällä epäyhtälöt on tultu päätelmään
\begin{align*}
t-\tfrac{1}{2}t^2                &\,\le\ \sin t \,\le\ t, \quad t\in[0,\tfrac{\pi}{2}] \\[1mm]
\impl\quad\ 1-\tfrac{1}{2}\,t    &\,\le\, \frac{\sin t}{t} \,\le\, 1,
                                    \quad t\in(0,\tfrac{\pi}{2}] \\
\impl\quad 1-\tfrac{1}{2}\abs{t} &\,\le\, \frac{\sin t}{t} \,\le\, 1, 
                                    \quad 0<\abs{t}\le\tfrac{\pi}{2}.
\end{align*}
Tässä viimeinen päättely perustui funktion $\sin t/t$ parillisuuteen.

Jos nyt $t_n\in [-\tfrac{\pi}{2},\tfrac{\pi}{2}]$, $t_n\neq 0$ ja $t_n\kohti 0$, niin viimeksi
kirjoitetun epäyhtälön perusteella
\[
1\ \ge\ \frac{\sin t_n}{t_n}\ \ge\ 1-\tfrac{1}{2}\abs{t_n} \kohti 1.
\]
Raja-arvon määritelmän (Määritelmä \ref{funktion raja-arvon määritelmä}) perusteella on 
johdettu raja-arvotulos
\begin{equation} \label{sinin raja-arvotulos}
\boxed{\quad \lim_{t\kohti 0}\,\frac{\ykehys\sin t}{\akehys t}=1. \quad}
\end{equation}
Kun sovelletaan tätä tulosta, kaavaa $1-\cos t=2\sin^2\tfrac{t}{2}$ ja raja-arvojen
yhdistelysääntöjä (Lause \ref{funktion raja-arvojen yhdistelysäännöt}), niin seuraa
\begin{equation} \label{kosinin raja-arvotulos}
\boxed{\quad \lim_{t\kohti 0}\,\frac{\ykehys 1-\cos t}{\akehys t^2}=\frac{1}{2}\,. \quad}
\end{equation}
Edellä olennaisesti johdettiin myös seuraavat, kaikilla $t\in\R$ pätevät epäyhtälöt:
\[
\boxed{\kehys\quad \abs{\sin t}\,\le\,\abs{t}, \quad 
                   0\,\le\,1-\cos t \le \frac{1}{2}\,t^2, \quad t\in\R. \quad}
\]
Kun raja-arvotulokset \eqref{sinin raja-arvotulos},\,\eqref{kosinin raja-arvotulos} yhdistetään
trigonometrisiin yhteenlaskukaavoihin
\begin{align*}
\sin (x+t) &=\sin x\cos t+\cos x\sin t, \\
\cos (x+t) &=\cos x\cos t-\sin x\sin t
\end{align*}
(ks.\ Luku \ref{trigonometriset funktiot}) saadaan yleisemmät raja-arvotulokset
(Harj.teht.\,\ref{H-V-4: sinin ja kosinin erotusosamäärät})
\begin{equation} \label{sinin ja kosinin erotusosamäärät}
\boxed{\begin{aligned}
\quad \lim_{t\kohti 0}\,\frac{\ykehys \sin(x+t)-\sin x}{t} &= \cos x, \\
       \lim_{t\kohti 0}\,\frac{\cos(x+t)-\cos x}{\akehys t}&= -\sin x. \quad
\end{aligned}}
\end{equation}

\subsection{Trigonometristen funktioiden derivaatat}
\index{derivoimissäännöt!f@trigonometriset funktiot|vahv}

Raja-arvotulosten \eqref{sinin ja kosinin erotusosamäärät} perusteella funktioiden $\sin$ ja 
$\cos$ derivoimissäännöt ovat
\[ \boxed{ \begin{aligned}
\quad \dif\sin x &= \cos x, \quad x\in\R, \\
      \dif\cos x &= -\sin x, \quad x\in\R. \quad
\end{aligned} } \]
Näiden sääntöjen ja Luvun \ref{derivaatta} sääntöjen perusteella on
\[
\dif\left(\frac{\sin x}{\cos x}\right)=1+\frac{\sin^2 x}{\cos^2 x}\,.
\]
Tämä ja vastaava $\cot$-funktion derivoimissääntö annetaan yleensä muodossa
\[ \boxed{ \begin{aligned}
\quad \dif\tan x\ &=\ 1+\tan^{\ykehys 2}x\ =\ \dfrac{1}{\cos^2 x}\,, \\
      \dif\cot x\ &=\ -(1+\cot^2 x)\ =\ -\dfrac{1}{\akehys \sin^2 x}\,. \quad
\end{aligned} } \]

Trigonometristen käänteisfunktioiden derivaatat voidaan laskea edellisen luvun säännöllä
\eqref{D5}. Esim.\ jos $y\in(-1,1)$ ja $\Arcsin y=x\in(-\tfrac{\pi}{2},\tfrac{\pi}{2})$,
niin mainittu sääntö yhdessä trigonometrian kaavojen kanssa antaa
\[
\frac{d}{dy}\Arcsin y = \frac{1}{\dif\sin x} 
                      = \frac{1}{\cos x} 
                      = \frac{1}{\sqrt{1-\sin^2 x}}
                      = \frac{1}{\sqrt{1-y^2}}\,.
\]
Kun $\Arccos$ ja $\Arctan$ derivoidaan vastaavalla tavalla on tulos
(Harj.teht.\,\ref{H-V-4: Arccos ja Arctan})\,:
\[ \boxed{ \begin{aligned}
\quad \dif\Arcsin x\ &=\ \dfrac{1}{\sqrt{1-x^2}}\,, \quad x\in (-1,1), \\
      \dif\Arccos x\ &=\ -\dfrac{1}{\sqrt{1-x^2}}\,, \quad x\in (-1,1), \quad \\
      \dif\Arctan x\ &=\ \dfrac{1}{1+x^2}\,, \quad x\in\R.
\end{aligned} } 
\]

Em.\ derivoimissääntöjen antamia funktioita edelleen derivoimalla nähdään, että kaikki 
säännöissä mainitut funktiot ovat mielivaltaisen monta kertaa derivoituvia koko
määrittelyjoukossaan, jos tämä koostuu avoimista väleistä, muuten kaikilla määrittelyjoukon
avoimilla osaväleillä. Sama pätee myös yleisemmille trigonometrisille funktioille, jotka
saadaan yhdistelemällä perusfunktioita laskuoperaatioilla tai yhdistettyinä funtioina.
\begin{Exa} Funktioiden $\,\sin\,$ ja $\,\cos\,$ sekä edellisen luvun dervioimissääntöjen
nojalla on
\[
\dif\,\frac{\sin x}{1+\cos x} 
       \,=\, \frac{\cos x}{1+\cos x}+\frac{\sin^2x}{(1+\cos x)^2}
       \,=\, \frac{\cos x+\cos^2x+\sin^2x}{(1+\cos x)^2}
       \,=\, \frac{1}{1+\cos x}\,.
\]
Lasku on pätevä jokaisella $x\in\R$, jolla $\cos x \neq -1$, eli derivoitavan funktion
koko määrittelyjoukossa. \loppu
\end{Exa}
\begin{Exa} Funktiot $\,u(x)=\cos x\,$ ja $\,v(x)=\sin x\,$ ovat (ainakin eräs) ratkaisu
ongelmaan
\[
\begin{cases} \,u'=-v,\ v'=u, \quad x\in\R \\ \,u(0)=1,\ v(0)=0 \end{cases}
\]
(Ongelma on ratkaistavissa toisellakin tavalla, ks.\
Harj.teht.\,\ref{derivaatta}:\ref{H-V-3: cos ja sin potenssisarjoina}b.) \loppu
\end{Exa}
 

\Harj
\begin{enumerate}

\item \label{H-V-4: numeerinen kaarenpituus}
Tutki, kuinka tarkasti kaaren
\[
S=\{P=(x,y) \mid y=x^2\ \ja\ x\in[0,1]\}
\]
pituus saadaan lasketuksi käyttämällä pisteistöä $\{(x_i,f(x_i)),\ i=0 \ldots 10\}$, missä 
$x_i=i/10$. Tarkka arvo on $s=1.47894..$

\item
Origosta siirrytään käyrää $r=4\cos\varphi$ (napakoordinaatit) seuraten pisteeseen, jonka
$x$-koordinaatti $=3$. Mikä on matkan pituus lyhintä reittiä?

\item \label{H-V-4: sinin ja kosinin erotusosamäärät}
Johda raja-arvotulokset \eqref{sinin ja kosinin erotusosamäärät}.

\item
Näytä, että jos $a,b\in\R$ ja $b \neq 0$, niin 
$\,\displaystyle{\lim_{x \kohti 0}\frac{\sin ax}{\sin bx}=\frac{a}{b}}\,$.

\item
Määritä raja-arvot
\begin{align*}
&\text{a)}\ \lim_{x \kohti 0} \frac{x\sin x}{1-\cos x} \qquad
 \text{b)}\ \lim_{x \kohti 0} \frac{x\sin 2x}{1-\cos 2x} \qquad
 \text{c)}\ \lim_{x \kohti 0^+} \frac{x-\sqrt{x}}{\sqrt{\sin x}} \\
&\text{d)}\ \lim_{x\kohti\frac{\pi}{2}}(1+\cos 2x)\tan^2 x \qquad
 \text{e)}\ \lim_{x\kohti\infty} x\sin\frac{1}{x}
\end{align*}

\item \label{H-V-4: Arccos ja Arctan}
Johda funktioiden $\Arccos$ ja $\Arctan$ derivoimissäännöt.

\item
Derivoi ja sievennä lopputulos:
\begin{align*}
&\text{a)}\ \frac{1-\cos x}{\sin x} \qquad
 \text{b)}\ \frac{1+\cos x}{1-\cos x} \qquad
 \text{c)}\ \frac{\sin x-\cos x}{\sin x+\cos x} \\[1mm]
&\text{d)}\ \Arcsin(\cos x) \qquad
 \text{e)}\ \Arccos(\sin x) \qquad
 \text{f)}\ \Arctan(\cot x)
\end{align*} 

\item
Määritä implisiittifunktion $y(x)$ derivaatta annetussa pisteessä:
\begin{align*}
&\text{a)}\,\ y+2\sin y+\cos y=x,\ \ (x,y)=(1,0) \\
&\text{b)}\,\ 2x+y-\sqrt{2}\sin(xy)=\frac{\pi}{2}\,,\ \ (x,y)=\left(\frac{\pi}{4}\,,1\right) \\
&\text{c)}\,\ x\sin(xy-y^2)=x^2-1,\ \ (x,y)=(1,1) \\
&\text{d)}\,\ \tan(xy^2)=\frac{2xy}{\pi}\,,\ \ (x,y)=\left(-\pi,\frac{1}{2}\right)
\end{align*}

\item
Näytä induktiolla, että $\dif^n\tan x=p(\tan x)$, missä $p$ on polynomi astetta $n+1$ ja
muotoa $\,p(t)=n!\,t^{n+1}\,+\,c_{n-1}\,t^{n-1}\,+\,c_{n-3}\,t^{n-3}\,+\,\ldots$

\item
Totea funktio
\[
f(x)=\begin{cases} 
     x^2\sin\dfrac{1}{x^2}\,, &\text{kun}\ x \neq 0, \\ 0, &\text{kun}\ x=0
     \end{cases} \]
esimerkiksi funktiosta, joka on derivoituva jokaisessa pisteessä $x\in\R$, mutta derivaatta
ei ole jokaisessa pisteessä jatkuva. Hahmottele $f$:n ja $f'$:n kuvaajat!

\item (*)
Olkoon $s_n$ käyrän $\,S: y=x^n\,$ kaarenpituus välillä $[0,1]$, kun $n\in\N$. Näytä, että
$\lim_n s_n = 2$.

\item (*)
Näytä sopivalla muuttujan vaihdolla: \vspace{1mm}\newline
$\D
\text{a)}\ \ \lim_{x\kohti\infty} x\left(\frac{\pi}{2}-\Arctan x\right)=1 \qquad 
\text{b)}\ \ \lim_{x \kohti 1^-} \frac{\pi-2\Arcsin x}{\sqrt{1-x}}=2\sqrt{2}$

\end{enumerate} % Trigonometristen funktioiden derivointi
\section{Ääriarvot. Sileys} \label{ääriarvot}
\alku

Derivaatta on mitä mainioin työkalu, kun halutaan luonnehtia tavallisia reaalifunktioita,
jotka useimmiten ovat derivoituvia 'melkein kaikkialla'. Tässä ja seuraavassa luvussa
tarkastellaan derivaatan käyttöä funktiotutkimuksessa.

\subsection{Paikalliset ääriarvot}

Aloitetaan määritelmästä.
\begin{Def} \label{paikallinen ääriarvo}
\index{paikallinen maksimi, minimi, ääriarvo|emph}
\index{maksimi (funktion)!a@paikallinen|emph} 
\index{minimi (funktion)!a@paikallinen|emph}
\index{zyzy@ääriarvo (paikallinen)|emph}
\index{suhteellinen ääriarvo|emph}
\index{oleellinen ääriarvo(piste)|emph}
\index{epzyozy@epäoleellinen ääriarvo(piste)|emph}
Funktiolla $f:\DF_f\kohti\R$, $\DF_f\subset\R$, on pisteessä $c\in\DF_f$ \kor{paikallinen} eli
\kor{suhteellinen ääriarvo}, eli $c$ on $f$:n \kor{paikallinen ääriarvopiste} 
(ääri\-arvokohta), jos jollakin $\delta>0$ ja $\forall x\in\R$ pätee
$(c-\delta,c+\delta)\subset\DF_f$ ja
\begin{align*}
\text{joko:}\quad 0<\abs{x-c}<\delta \ &\impl \ f(x) \le f(c), \\
 \text{tai:}\quad 0<\abs{x-c}<\delta \ &\impl \ f(x) \ge f(c).
\end{align*}
Edellisessä tapauksessa on kyseessä \kor{paikallinen maksimi}, jälkimmäisessä 
\kor{paikallinen minimi}. Jos lisäksi jollakin $\delta>0$ on $f(x) \neq f(c)$ aina
kun $0<\abs{x-c}<\delta$, niin ääriarvo (ääriarvopiste) on \kor{oleellinen}, muulloin
\kor{epäoleellinen}.
\end{Def}
\begin{Exa} Funktio
\[
f(x)=\begin{cases}
     \,\sin^2\dfrac{1}{x}\,, &\text{kun}\ x \neq 0, \\[2mm] \,0, &\text{kun}\ x=0
     \end{cases}
\]
saavuttaa pienimmän arvonsa $f_{min}=0$ pisteessä $x=0$ sekä pisteissä 
\[
a_n=\frac{1}{n\pi}\,, \quad n\in\Z,\ n \neq 0.
\]
Suurimman arvonsa $f_{max}=1$ funktio saavuttaa pisteissä 
\[
b_n=\frac{1}{(n+\frac{1}{2})\pi}\,, \quad n\in\Z.
\]
Pisteet $a_n$ ja $b_n$ ovat myös Määritelmän \ref{paikallinen ääriarvo} mukaisia oleellisia
paikallisia minimi- ja maksimikohtia, sillä $f$:n määrittelyn ja $\sin$-funktion tunnettujen
ominaisuuksien perusteella on $\,0<f(x)<1$ kaikissa muissa kuin mainituissa pisteissä tai
pisteessä $x=0$. Tällöin jos esim.\ $c=b_n\,,\ n\in\Z$, niin Määritelmän
\ref{paikallinen ääriarvo} paikallisen maksimin ehto on voimassa, kun valitaan $\delta$ siten,
että $0<\delta<\abs{b_{n \pm 1}-b_n}$. Pisteessä $x=0$ on $f$:llä Määritelmän
\ref{paikallinen ääriarvo} mukainen epäoleellinen paikallinen minimi, sillä jos valitaan mikä
tahansa $\delta>0$, niin $a_n\in(0,\delta)$ kun $n>(\pi\delta)^{-1}$, jolloin välillä
$(0,\delta)$ on aina pisteitä, joissa $f(x)=f(0)$. \loppu
\end{Exa}
Seuraavan lauseen mukaan derivoituvan funktion paikallinen ääriarvokohta on välttämättä myös
derivaatan nollakohta. Tähän liittyen sanotaankin derivaatan nollakohtia funktion
\index{kriittinen piste}%
\kor{kriittisiksi pisteiksi}.
\begin{Lause} \label{ääriarvolause}
Jos $c\in\DF_f$ on $f$:n paikallinen ääriarvopiste ja $f$ on derivoituva $c$:ssä, 
niin $f'(c)=0$.
\end{Lause}
\tod Väittämän loogisesti ekvivalentti muoto on: Jos $f$ on $c$:ssä derivoituva ja 
$f'(c) \neq 0$, niin $c$ ei ole $f$:n paikallinen ääriarvopiste. Todistetaan väittämä tässä
muodossa, eli oletetaan, että $f'(c) = k \neq0$. Derivaatan määritelmän perusteella
(vrt.\ Luku \ref{derivaatta}) $f$ on tällöin määritelty välillä $(c-\delta,c+\delta)$ jollakin
$\delta>0$ ja ko. välillä pätee 
\[
f(x)=f(c)+k(x-c)+g(x),
\]
missä $\lim_{x\kohti c} g(x)/(x-c) =0$. Tällöin koska $k \neq 0$, niin Lauseen
\ref{approksimaatiolause} mukaan jollakin $\delta>0$ (vastaten valintaa $\eps=|k|/2>0$
ko.\ lauseessa) pätee myös
\[
\abs{g(x)/(x-c)} \le \tfrac{1}{2}\abs{k}\,\ 
   \impl\,\ \abs{g(x)} \le \tfrac{1}{2}\abs{k}\abs{x-c}, \quad \text{kun}\ 0<\abs{x-c}<\delta.
\] 
Näin ollen jos $k>0$, niin kolmioepäyhtälön nojalla
\[
\begin{cases}
\,f(x)-f(c)\ \ge\ k(x-c) - \frac{k}{2}\,\abs{x-c}
           \ =\ \frac{k}{2}\,(x-c)\ >\ 0 \quad \forall x\in(c,c+\delta), \\
\,f(x)-f(c)\ \le\ k(x-c) + \frac{k}{2}\,\abs{x-c} 
           \ =\ \frac{k}{2}\,(x-c)\ <\ 0 \quad \forall x\in(c-\delta,c).
\end{cases}
\]
Jos $k<0$, päätellään samalla tavoin, että $f(x)-f(c)<0$ kun $x \in (c,c+\delta)$ ja 
$f(x)-f(c)>0$ kun $x \in (c-\delta,c)$. Määritelmän \ref{paikallinen ääriarvo} mukaan 
kummassakaan tapauksessa $c$ ei ole $f$:n paikallinen ääriarvopiste. \loppu
\jatko \begin{Exa} (jatko) Esimerkin funktiolle pätee
\[
f'(x) = -\frac{2}{x^2}\,\sin\frac{1}{x}\cos\frac{1}{x}\,, \quad \text{kun}\ x \neq 0,
\]
joten $f'(a_n)=f'(b_n)=0$ Lauseen \ref{ääriarvolause} väittämän mukaisesti. Pisteesä $x=0$ ei
$f$ ole derivoituva --- tämäkin on sopusoinnussa Lauseen \ref{ääriarvolause} kanssa. \loppu
\end{Exa}
\begin{Exa} \label{kriittiset pisteet: esim} Rationaalifunktio $f(x)=x^2/(x+1)$ on derivoituva
koko määrittelyjoukossaan, joten $f$:n mahdolliset paikalliset ääriarvokohdat ovat väistämättä
kriittisiä pisteitä. Koska
\[
f'(x) \,=\, \frac{2x}{1+x}-\frac{x^2}{(1+x)^2} 
      \,=\, \frac{x^2+2x}{(x+1)^2}\,, \quad x \neq -1,
\]
niin kriittiset pisteet ovat $x=-2$ ja $x=0$. Osoittautuu, että edellisessä pisteessä $f$:llä
on paikallinen maksimi, jälkimmäisessä paikallinen minimi. (Ks.\ seuraavan luvun Esimerkki
\ref{monotonisuus: esim}.) \loppu
\end{Exa}

\subsection{Ääriarvotehtävät}
\index{zyzy@ääriarvotehtävä|vahv}

Kysymys funktion paikallisista ääriarvokohdista nousee yleensä esiin osatehtävänä, kun halutaan
ratkaista \kor{ääriarvotehtävä}: On annettu funktio $f$ ja joukko $A\subset\DF_f$, ja haluttaa
määrittää $f$:n suurin ja/tai pienin arvo $A$:ssa. Sikäli kuin tehtävä ratkeaa, eli 
suurin/pienin arvo on löydettävissä, sanotaan ko.\ arvoa $f$:n
\index{maksimi (funktion)!b@absoluuttinen} \index{minimi (funktion)!b@absoluuttinen}
\index{absoluuttinen maksimi, minimi}%
\kor{absoluuttiseksi} maksimiksi/minimiksi $A$:ssa. Absoluuttinen maksimi ja minimi löytyvät
aina, jos joukko $A$ on \pain{äärellinen}, sillä tällöin kyse on vain valinnasta äärellisessä
joukossa. Jos $A$ on ääretön joukko, esim.\ väli, on tilanne ongelmallisempi. Rajoitutaan tässä
sovelluksissa usein esiintyvään perustehtävään, jossa $A$ on \pain{sul}j\pain{ettu} \pain{väli}
ja $f$ on ko.\ \pain{välillä} j\pain{atkuva} Määritelmän \ref{jatkuvuus välillä} mukaisesti.
Tällöin ääriarvotehtävän ratkeavuuden takaa kummankin ääriarvon osalta Weierstrassin lause
(Lause \ref{Weierstrassin peruslause}).

Jatkuvaa funktiota $f$ ja suljettua väliä $[a,b]$ koskevan ääriarvotehtävän ratkaiseminen
helpottuu käytännössä huomattavasti, jos $f$ on paitsi jatkuva myös derivoituva tai
'melkein derivoituva' välillä $(a,b)$. Tulos on muotoiltavissa Lauseen \ref{ääriarvolause}
korollaarina seuraavasti.
\begin{Kor} \label{ääriarvokorollaari} Olkoon $f$ on jatkuva välillä $[a,b]$ ja olkoon
$X\subset[a,b]$ joukko, jolle pätee
\begin{align*}
&\text{1.} \ \ a \in X\ \text{ja}\ b \in X, \\
&\text{2.} \,\ \{c\in(a,b) \mid \text{$f$ derivoituva $c$:ssä ja $f'(c)=0$}\} \subset X, \\
&\text{3.} \,\ \{c \in (a,b) \mid \text{$f$ ei derivoituva $c$:ssä}\} \subset X.
\end{align*}
Tällöin $\D\ \underset{x\in [a,b]}{\max/\min} \; f(x)=\underset{x\in X}{\max/\min} \; f(x)$.
\end{Kor}
Korollaarin \ref{ääriarvokorollaari} ehdot täyttävä joukko $X$ siis sisältää välin $[a,b]$
päätepisteet ja lisäksi kaikki välillä $(a,b)$ olevat $f$:n kriittiset pisteet sekä pisteet,
joissa $f$ ei ole derivoituva. Yleensä joukko $X$ voidaan valita niin, että se on äärellinen,
jolloin $f$:n maksimi- ja minimiarvojen haku pelkistyy korollaarin mukaisesti valinnaksi
äärellisessä joukossa $X$. --- Huomattakoon, että joukkoon $X$ (sikäli kuin äärellinen) voidaan
sen määrittelyn mukaisesti sisällyttää myös sellaisia pisteitä, joissa $f$:n derivoituvuus
on pelkästään epäilyksen alaista. Haluttaessa vain selvittää funktion maksimi- ja minimikohdat
ja -arvot välillä $[a,b]$ ei tarkempaa tutkimusta 'epäillyistä' tarvita. Avoimeksi voidaan myös
jättää kysymys, ovatko joukkoon $X$ sisällytettävät pisteet $f$:n paikallisia ääriarvokohtia
vai eivät.
\begin{Exa} \label{ääriarvoesimerkki} Määritä funktion $f(x) = \max\,\{2-x,\,6x-3x^2\}$ 
maksimi- ja minimikohdat ja -arvot välillä $[0,3]$. 
\end{Exa}
\ratk Kysessä on jatkuva ja paloittain polynomiarvoinen funktio. Koska \newline
$2-x = 6x-3x^2\ \ekv\ 3x^2-7x+2=0\ \ekv\ x=\tfrac{1}{3}\ \tai\ x=2$, niin päätellään, että
\[
f(x) = \begin{cases} 
       \,6x-3x^2,\ &\text{kun}\,\ \tfrac{1}{3} \le x \le 2, \\ 
       \,2-x,\     &\text{kun}\,\ x \le \tfrac{1}{3}\,\ \text{tai}\,\ x \ge 2. 
       \end{cases}
\]
Väleillä $(0,\tfrac{1}{3})$ ja $(2,3)$ on $f'(x)=-1 \neq 0$. Välillä $(\tfrac{1}{3},2)$ on
$f'(x)=6-6x$, joten tällä välillä on $f$:llä kriittinen piste $c=1$. Pisteissä
$x=\tfrac{1}{3}$ ja $x=2$ ei $f$ mahdollisesti (eikä todellisuudessakaan) ole derivoituva,
muissa välin $(0,3)$ pisteissä on, joten Korollaarissa \ref{ääriarvokorollaari} voidaan valita
$X=\{0,\tfrac{1}{3},1,2,3\}$. Laskemalla $f$:n arvot näissä pisteissä todetaan, että
$f_{max}=f(1)=3$, $f_{min}=f(3)=-1$. \loppu

Esimerkissä mahdollisia paikallisia ääriarvokohtia välillä $(0,3)$ ovat Lauseen 
\ref{ääriarvolause} mukaan pisteet $\tfrac{1}{3},1,2$. Funktion lähempi tutkimus näiden
pisteiden ympäristössä (algebran keinoin tai jatkossa esitettävin menetelmin) osoittaa, että
$c=\tfrac{1}{3}$ on oleellinen paikallinen minimi, $c=1$ on oleellinen paikallinen maksimi ja
$c=2$ ei ole paikallinen ääriarvokohta.

\subsection{Toispuoliset derivaatat}
\index{derivaatta!toispuolinen|vahv}
\index{toispuolinen derivaatta|vahv}

Jos Esimerkissä \ref{ääriarvoesimerkki} tarkkaillaan niitä pisteitä, joissa $f$ ei ole 
derivoituva, nähdään että $f$:llä on näissäkin pisteissä \kor{toispuoliset} (engl. one-sided) 
derivaatat seuraavan määritelmän mielessä (vrt. toispuoliset raja-arvot Luvussa 
\ref{funktion raja-arvo}).
\begin{Def} \label{toispuoliset derivaatat} \index{derivoituvuus!a@vasemmalta, oikealta|emph}
\index{vasemmalta derivoituva|emph} \index{oikealta derivoituva|emph}
Funktio $f:\DF_f\kohti\R$, $\DF_f\subset\R$, on pisteessä $x\in\DF_f$
\kor{vasemmalta derivoituva}, jos $(x-\delta,x]\subset\DF_f$ jollakin $\delta>0$ ja $\exists$
raja-arvo
\[
\dif_-f(x)=\lim_{\Delta x\kohti 0^-} \frac{f(x+\Delta x)-f(x)}{\Delta x},
\]
ja \kor{oikealta derivoituva}, jos $[x,x+\delta)\subset\DF_f$ jollakin $\delta>0$ ja $\exists$ 
raja-arvo
\[
\dif_+f(x)=\lim_{\Delta x\kohti 0^+} \frac{f(x+\Delta x)-f(x)}{\Delta x}.
\]
\end{Def}
Määritelmien \ref{toispuoliset derivaatat} ja \ref{derivaatan määritelmä} perusteella $f$ on
pisteessä $x$ derivoituva täsmälleen kun $f$ on $x$:ssä sekä vasemmalta että oikealta
derivoituva ja $\dif_-f(x)=\dif_+f(x)$ ($=f'(x)$).

Jos $f$ on pisteessä $c$ jatkuva ja lisäksi sekä vasemmalta että oikealta derivoituva, niin 
pätee (vrt.\ johdatus derivaattaan Luvussa \ref{derivaatta})
\[
\begin{cases}
\,f(x)=f(x)+\dif_-f(c)(x-c)+g(x), \quad \text{kun}\ x \in (c-\delta,c], \\
\,f(x)=f(x)+\dif_+f(c)(x-c)+g(x), \quad \text{kun}\ x \in [c,c+\delta),
\end{cases}
\]
missä $g(x)/(x-c) \kohti 0$, kun $x \kohti c^-$ tai $x \kohti c^+$. Tästä nähdään oikeaksi
\begin{Lause} \label{ääriarvolause 2} Jos $f$ on pisteessä $c$ jatkuva ja lisäksi sekä 
vasemmalta että oikealta derivoituva, niin pätee
\begin{itemize}
\item[(a)] $\dif_-f(c)\,\dif_+f(c)<0 \ \impl \ f$:llä on pisteessä $c$ paikallinen ääriarvo,
joka on
\begin{itemize}
\item[-] oleellinen maksimi,\,  jos $\dif_-f(c)>0$ ja $\dif_+f(c)<0$,
\item[-] oleellinen minimi, \ \ jos $\dif_-f(c)<0$ ja $\dif_+f(c)>0$.
\end{itemize}
\item[(b)] $\dif_-f(c)\,\dif_+f(c)>0 \ \impl\ c$ ei ole $f$:n paikallinen ääriarvopiste.
\end{itemize}
\end{Lause}
\jatko \begin{Exa} (jatko) Esimerkissä on $\dif_-f(\tfrac{1}{3})=-1,\ \dif_+f(\tfrac{1}{3})=4$
ja $\dif_-f(2)=-6,\ \dif_+f(2)=-1$, joten Lauseen \ref{ääriarvolause 2} mukaan $f$:llä on
pisteessä  $x=\tfrac{1}{3}$ oleellinen paikallinen minimi ja $x=2$ ei ole paikallinen
ääriarvokohta. \loppu
\end{Exa}
Jos $f$ on pisteessä $c$ derivoituva ja $f'(c) \neq 0$, niin
$\dif_-f(c)\,\dif_+f(c) = [f'(c)]^2 > 0$, joten Lauseen \ref{ääriarvolause 2} väittämä (b)
sisältää myös Lauseen \ref{ääriarvolause} väittämän muodossa
\[
f'(c) \neq 0 \qimpl \text{$f$:llä ei ole paikallista ääriarvoa $c$:ssä}.
\]
Jos $\dif_-f(c)=0$ tai $\dif_+f(c)=0$ (myös kun $f'(c)=0$), voi $c$ olla paikallinen
ääriarvopiste tai ei, riippuen tapauksesta.
\begin{Exa} Funktioille
$\displaystyle{\
f(x)=\begin{cases}
\,x^2,  &x<0, \\
\,x,    &x\ge 0
     \end{cases} \quad \text{ja} \quad
g(x)=\begin{cases}
-x^2, &x<0, \\
\,x,    &x\ge 0
     \end{cases}}$

pätee $\dif_-f(0)=\dif_-g(0)=0$ ja $\dif_+f(0)=\dif_+g(0)=1$. Funktiolla $f$ on origossa
(oleellinen) paikallinen minimi, $g$:llä ei ole paikallista ääriarvoa origossa.
\begin{figure}[H]
\setlength{\unitlength}{1cm}
\begin{center}
\begin{picture}(10,5.5)(0,-2.5)
\multiput(0,0)(6,0){2}{
\put(0,0){\vector(1,0){4}} \put(3.8,-0.4){$x$}
\put(2,-2){\vector(0,1){4}} \put(2.2,1.8){$y$}}
\curve(0.586,2,1,1,1.5,0.25,2,0) \put(2,0){\line(1,1){2}}
\curve(6.586,-2,7,-1,7.5,-0.25,8,0)\drawline(8,0)(10,2)
\put(1.3,-3){$y=f(x)$} \put(7.3,-3){$y=g(x)$}
\end{picture}
\end{center}
\end{figure}
\end{Exa}

\subsection{Sileys}
\index{sileys(aste)|vahv}

Seuraava määritelmä yhdistävää suljetulla välillä jatkuvuuden 
(Määritelmä \ref{jatkuvuus välillä}) ja derivoituvuuden käsitteitä.
\begin{Def} \label{sileys}
\index{jatkuvasti derivoituvuus|emph} \index{derivoituvuus!b@jatkuvasti derivoituvuus|emph}
Funktio $f:\DF_f\kohti\R$, $\DF_f\subset\R$ on \kor{jatkuvasti derivoituva} 
(engl.\ continuously differentiable) \kor{suljetulla välillä} $[a,b]\subset\DF_f$, jos $f$ on
derivoituva välillä $(a,b)$, oikealta derivoituva pisteessä $a$, vasemmalta derivoituva 
pisteessä $b$, ja derivaatta $f'$, määriteltynä toispuolisena välin $[a,b]$ päätepisteissä, on
välillä $[a,b]$ jatkuva funktio. Yleisemmin jos $m\in\N$, niin $f$ on välillä $[a,b]$ $m$ 
\kor{kertaa jatkuvasti derivoituva}, jos $f$ on $m$ kertaa derivoituva välillä $(a,b)$, $m$ 
kertaa oikealta derivoituva pisteessä $a$, $m$ kertaa vasemmalta derivoituva pisteessä $b$, ja
derivaatat $f^{(k)}$, määriteltynä toispuolisina välin päätepisteissä, ovat välillä $[a,b]$
jatkuvia, kun $\,k=1 \ldots m$. Edelleen jos $A\subset\DF_f$ on puoliavoin tai avoin väli,
niin sanotaan, että $f$ on $m$ kertaa jatkuvasti derivoituva $A$:ssa, jos $f$ on $m$ kertaa
jatkuvasti derivoituva jokaisella $A$:n suljetulla osavälillä. 
\end{Def}
Sovellettaessa määritelmää suljetun välin tapauksessa, kun $m \ge2$, ajatellaan derivaatat
$f^{(k)}$, $k=1\ldots m-1$ määritellyksi välin päätepisteissä palautuvasti (toispuolisina)
alkaen indeksistä $k=1$. Indeksiä $m=0$ vastaava termi '$0$ kertaa jatkuvasti derivoituva'
tulkitaan pelkäksi jatkuvuudeksi. 
\begin{Exa} \label{sileysesimerkki} Olkoon $n\in\N$ ja tarkastellaan funktiota
\[
f(x) = \begin{cases} \,0, &\text{kun } x\leq 0, \\ \,x^n, &\text{kun } x>0. \end{cases}
\]
Derivoimalla pistessä $x=0$ toispuolisesti todetaan, että $f$ on tässä pistessä (ja siis koko
$\R$:ssä) $n-1$ kertaa derivoituva ja pätee
\[
f^{(k)}(x) = \begin{cases}
             \,0,                           &\text{kun}\ x \le 0, \\[2mm]
             \,\dfrac{n!}{(n-k)!}\,x^{n-k}, &\text{kun}\ x>0, \quad k=1 \ldots n-1.
             \end{cases}
\]
Koska derivaatat $f^{(k)}(0)=0,\ k=1 \ldots n-1$ ovat $\R$:ssä jatkuvia ja koska $f^{(n-1)}$ 
ei ole derivoituva pisteessä $x=0$, niin päätellään, että $f$ on $\R$:ssä täsmälleen $n-1$
kertaa (ei $n$ kertaa) jatkuvasti derivoituva. \loppu
\end{Exa}
Jos $f$ on $m$ kertaa mutta ei $m+1$ kertaa jatkuvasti derivoituva välillä $A$, niin 
indeksiä $m$ kutsutaan usein $f$:n \kor{sileysasteeksi} (engl.\ degree of smoothness), ja 
voidaan myös sanoa, että $f$ on \kor{sileä astetta $m$} ko.\ välillä. Tällöin 'sileä astetta
nolla' tarkoittaa siis pelkkää jatkuvuutta. Esimerkissä $f$:n sileysaste ($\R$:ssä) on
$m=n-1$. Jos sileysasteella ei ole ylärajaa, ts.\ $f$ on 'äärettömän sileä', niin 
sileysasteeksi voidaan merkitä $m=\infty$. Termillä \kor{sileä} (ilman lisämääreitä) on usein
juuri tämä merkitys. Esimerkissä on $m=\infty$ esim.\ välillä $[0,\infty)$.
\begin{Exa} Polynomi on sileä (= 'äärettömän sileä') jokaisella välillä, eli $\R$:ssä, samoin
trigonometriset funktiot $\sin$, $\cos$ ja $\Arctan$. Trigonometrinen funktio $\,\tan\,$ on
sileä väleillä $((n-\tfrac{1}{2})\pi,(n+\tfrac{1}{2})\pi),\ n\in\Z$. \loppu
\end{Exa}
\begin{Exa} Potenssisarjan summana määritelty funktio on sileä välillä $(-\rho,\rho)$
(eli väleillä $[a,b] \subset (-\rho,\rho)$), missä $\rho=$ sarjan suppenemissäde, vrt.\ Luku 
\ref{derivaatta}. \loppu
\end{Exa}

Edellä Esimerkissä \ref{sileysesimerkki} määritelty funktio $f$ on
\index{paloittainen!c@sileys}
\kor{paloittain sileä} (engl.\ piecewise smooth) millä tahansa indeksillä $m\in\N$ mitaten.
Tämä tarkoittaa, että olipa $m\in\N$ mikä tahansa, niin $f$ täyttää seuraavan määritelmän ehdot.
\begin{Def} \label{paloittainen sileys} \index{jatkuvasti derivoituvuus|emph}
\index{derivoituvuus!b@jatkuvasti derivoituvuus|emph}
Funktio $f:\DF_f\kohti\R$, $\DF_f\subset\R$ on \kor{$m$ kertaa paloittain jatkuvasti
derivoituva} välillä $[a,b]$, jos $\exists$ pisteet $c_k$, $k=0 \ldots n$, $n\in\N$ ja 
funktiot $f_k$, $k=1 \ldots n$ siten, että pätee
\begin{itemize}
\item[(i)]   $a=c_0<c_1<\ldots<c_n=b$\, ja $\,(c_{k-1},c_k)\subset\DF_f, \quad k=1 \ldots n$,
\item[(ii)]  $[c_{k-1},c_k]\subset\DF_{f_k}$ ja $f_k$ on $m$ kertaa jatkuvasti derivoituva
             välillä $[c_{k-1},c_k]$, $k=1\ldots n$,
\item[(iii)] $f(x)=f_k(x)$, \,kun $x\in (c_{k-1},c_k), \quad k=1 \ldots n$.
\end{itemize}
\end{Def}
\begin{Exa} Jos $a<0$ ja $b>0$, niin Esimerkin \ref{sileysesimerkki} funktiolle Määritelmän 
\ref{paloittainen sileys} ehdot ovat voimassa jokaisella $m\in\N$, kun asetetaan $n=1$,
$c_1=0$, $f_1(x)=0$ ja $f_2(x) = x^n$. \loppu 
\end{Exa}
Jos $f$ on välillä $[a,b]$ paloittain sileä astetta $m$ millä tahansa $m\in\N$, kuten 
Esimerkissä \ref{sileysesimerkki}, niin voidaan sanoa yksinkertaisesti, että $f$ on 
\kor{paloittain sileä} ilman lisämääreitä. Myös Esimerkin \ref{ääriarvoesimerkki} funktio on 
tätä tyyppiä.
\begin{Exa}
Funktio
\[
f(x)=\begin{cases}
x^3, &x<0, \\
x^5\sqrt{x}, &x\geq 0
\end{cases}
\]
on välillä $[-1,1]$ sileä astetta $m=2$ (täsmälleen) ja paloittain sileä astetta $m=5$ 
(täsmälleen). Tässä paloittaista sileyttä rajoittaa, että $f$ on origossa vain viidesti 
derivoituva oikealta. \loppu
\end{Exa}

\Harj
\begin{enumerate}

\item
Tutki origon laatu mahdollisena paikallisena ääriarvopisteenä:
\begin{align*}
&\text{a)}\ \ f(x)=100+x^{99}-x^{98} \qquad
 \text{b)}\ \ f(x)=x^4-x^3\cos x \\
&\text{c)}\ \ f(x)=\begin{cases}
                  x^3\sin\dfrac{1}{x}\,, &\text{kun}\ x \neq 0, \\[2mm]
                  0,                     &\text{kun}\ x=0
                  \end{cases} \quad\
 \text{d)}\ \ f(x)=\begin{cases} 0.000001, &\text{kun}\ x=0, \\[2mm]
                  x\cos\dfrac{1}{x}\,,    &\text{kun}\ x \neq 0
                  \end{cases}
\end{align*}

\item
Määritä seuraavien funktioiden maksimi- ja minimiarvot annetulla välillä sekä pisteet
joissa maksimi/minimi saavutetaan:
\begin{align*}
&\text{a)}\ f(x)=(x-1)^3(x+1)^3, \quad \text{väli}\ [-2,4] \\[4mm]
&\text{b)}\ f(x)=\abs{x^5-80x+1}, \quad \text{väli}\ [-1,1] \\[2mm]
&\text{c)}\ f(x)=\frac{2-x}{5-4x+x^2}\,,\quad \text{väli}\ [-100,2] \\
&\text{d)}\ f(x)=\begin{cases}
                 8x^2-16x-14, &\text{kun}\ x \le 2, \\ 3x^2-24x+22, &\text{kun}\ x>2,
                 \end{cases} \quad \text{väli}\ [0,5] \\[1mm]
&\text{e)}\ f(x)=\min\{2x^3-2x,\,3-2x\}, \quad \text{väli}\ [-1,2] \\[5mm]
&\text{f)}\ f(x)=\abs{\sin x}-\cos x, \quad \text{väli}\ [0,2\pi] \\[5mm]
&\text{g)}\ f(x)=x+5\sin x, \quad \text{väli}\ [0,2\pi]
\end{align*}

\item
Mikä on lausekkeen $\sqrt{x}+2\sqrt{y}$ pienin arvo, kun $x,y \ge 0$ ja $x+y=5/6$\,?

\item
Suoran tien varressa, kohtisuorassa tietä vastaan, on mainostaulu, jonka leveys on $10$ metriä
ja lähin etäisyys tiellä kulkevien ajoradasta on $20$ m. Mikä on suurin kulma, jossa tiellä
liikkuja taulun näkee?

\item
Piste $P=(x,y)$ sijaitsee Cartesiuksen lehdellä $\,S: x^3+y^3=3xy$. Mikä on suurin mahdollinen
$x$:n arvo, jos $y \ge 0$\,?

\item 
Määritä \vspace{1mm}\newline
a) funktion $f(x,y)=(2x-y)(x+y-1)(x-y-1)$ maksimi- ja minimiarvot janalla, jonka päätepisteet
ovat $(-1,0)$ ja $(3,1)$, \vspace{1mm}\newline
b) funktion $f(x,y)=\abs{x^2+2y}+2x^2$ pienin arvo käyrällä $S:\ xy=1$, \vspace{1mm}\newline
c) funktion $f(x,y,z)=x+y+z(2x+y+z)(x+3y+z)(x+y+4z)$ pienin arvo suoralla $S:\ x=2y=3z$.

\item
Funktio $f$ on määritelty yksikkökiekon neljänneksessä 
$A=\{(x,y)\in\R^2 \mid x^2+y^2 \le 1\ \&\ x \ge 0\ \&\ y \ge 0\}$ seuraavasti
(napakoodinaatit!): $f(r,\phi)=(3r^2-2r)(2\phi^2-3\phi +1)$. Missä $A$:n pisteissä $f$ 
saavuttaa suurimman ja missä pienimmän arvonsa?

\item
Määritä napa- tai pallokoordinaatistoon siirtymällä seuraavien funktioiden maksimi- ja 
minimiarvot annetussa joukossa $A$. Anna myös karteesisessa koordinaatistossa pisteet, 
joissa nämä arvot saavutetaan.
\begin{align*}
&\text{a)}\ f(x,y)=x^3y^4,\ \ A=\{(x,y) \mid x^2+y^2 \le 1\} \\
&\text{b)}\ f(x,y)=x^2y^4,\ \ A=\{(x,y) \mid x^2+y^2 \le 25\} \\
&\text{c)}\ f(x,y,z)=xyz^3,\ \ A=\{(x,y,z) \mid x^2+y^2+z^2 \le 1\}
\end{align*}

\item
Mikä on funktion 
\[
f(x)=|x-3|^{111/11} + (x+1)^3|x+1| +\max\{1,x^2-2x+2\}
\]
sileysaste välillä \ a) $[-2,0]$, \ b) [0,2], \ c) [2,4]\,?

\item
Määritä seuraavien funktioiden sileysasteet välillä $[-1,1]$:
\begin{align*}
&\text{a)}\,\ x^2\abs{\sin x} \qquad
 \text{b)}\,\ \abs{x}^3(1-\cos x) \qquad
 \text{c)}\,\ \sin^2 x\abs{\tan x} \\
&\text{d)}\,\ f(x) = \begin{cases} 
                     \,1-\tfrac{1}{2}x^2, &\text{kun}\ x<0, \\
                     \,\cos x,            &\text{kun}\ x \ge 0
                     \end{cases} \quad\ \
 \text{e)}\,\ f(x) = \begin{cases}
                     \,\sin x,            &\text{kun}\ x \le 0, \\
                     \,x-\tfrac{1}{6}x^3, &\text{kun}\ x>0
                     \end{cases}
\end{align*} 

\item
Olkoon $f_1(x)=2x^3-7x^2+9x$ ja $f_2(x)=ax^2+bx+c$. Määritä kertoimet $a,b,c$ siten, että 
funktio
\[
f(x)=\begin{cases}
     \,f_1(x),  &\text{kun}\ x \le 1, \\ \,f_2(x),  &\text{kun}\ x>1
     \end{cases}
\]
on kahdesti jatkuvasti derivoituva välillä $[0,2]$. Piirrä samaan kuvaan $f$:n kuvaaja ko.\
välillä sekä katkoviivalla $f_2$:n kuvaaja välillä $[0,1]$ ja $f_1$:n kuvaaja välillä $[1,2]$.

\item (*)
a) Funktiosta $f: \R \kohti \R$ tiedetään, että $f$ on derivoituva $\R$:ssä ja että
\[
\lim_{x \kohti -\infty} f(x)=A_-\,, \quad \lim_{x \kohti\infty} f(x)=A_+\,,
\]
missä $A_-\in\R$ ja $A_+\in\R$. Näytä, että $f$ saavuttaa $\R$:ssä absoluuttisen
maksimiarvon täsmälleen kun jossakin $f$:n kriittisessä pisteessä $c$ on \newline
$f(c) \ge \max\{A_-\,,\,A_+\}$. \ b) Millä $a$:n arvoilla funktio
\[
f(x)=\frac{x^4+ax^2}{(x^2+7)^2}
\]
saa absoluuttisen maksimiarvon jollakin $x\in\R$\,?

\item (*)
Määritä funktion $f(x)=4\abs{\cos x}-3\sin x+2\cos x\,$ kriittiset pisteet, derivaatan
epäjatkuvuuskohdat, paikalliset ääriarvokohdat ja absoluutiset maksimi- ja minimiarvot
välillä $[0,2\pi]$. Hahmottele käyrän $y=f(x)$ kulku.

\item (*)
Määrittele välillä $(0,\infty)$ funktio $g$ siten, että funktion $\,f(x)=x+a\sin x\,$
pienin arvo välillä $[0,a]$ on $f_{min}=g(a)$. Hahmottele $g$:n kuvaaja.

\item (*) \label{H-V-5: jatkaminen polynomeilla}
a) Olkoon $p_1$ ja $p_2$ polynomeja astetta $\le n$. Näytä, että jos funktio
\[
f(x)=\begin{cases}
     \,p_1(x),  &\text{kun}\ x<a,  \\ \,p_2(x),  &\text{kun}\ x \ge a
     \end{cases}
\]
on $n$ kertaa jatkuvasti derivoituva välillä $[a-1,a+1]$, niin $p_1=p_2$.\vspace{1mm}\newline
b) Funktio $f$ olkoon $m$ kertaa jatkuvasti derivoituva välillä $[a,b]$. Näytä, että on
olemassa yksikäsitteiset polynomit $p_1$ ja $p_2$ astetta $\le m$ siten, että funktio
\[
g(x) = \begin{cases}
       \,p_1(x), &\text{kun}\ x<a, \\ 
       \,f(x), &\text{kun}\ x\in[a,b], \\ 
       \,p_2(x), &\text{kun}\ x>b
       \end{cases}
\]
on $m$ kertaa jatkuvasti derivoituva $\R$:ssä.

\item (*) \index{zzb@\nim!Vasikka-aitaus}
(Vasikka-aitaus) Maanviljelijä haluaa aidata navettansa viereen suorakulmion muotoisen 
aitauksen vasikoiden laidunmaaksi. Aitauksen mitat valitaan aidan kokonaispituuden ($x$ m) 
funktiona siten, että laitumen pinta-ala on mahdollisimman suuri ja navetan $40$ m:n pituinen
seinä käytetään hyväksi mahdollisimman hyvin. Näin suunnitellussa aitauksessa olkoon
$f(x)$ navetan seinän suuntaisen ja seinää vastaan kohtisuoran sivun pituuksien suhde.
Määritä $f(x)$ välillä $(0,\infty)$. Mikä on $f$:n sileysaste välillä $[0,\infty)$, kun
asetetaan $f(0)=f(0^+)$\,?

\item (*) \index{zzb@\nim!Pisin heitto?}
(Pisin heitto?) Jos ilmanvastusta ei huomioida, niin ilmaan heitetyn kappaleen lentorata
noudattaa heittoparaabelia 
\[
y=h+kx-(1+k^2)\frac{x^2}{2a}\,,
\]
missä $x$ mittaa vaakasuoraa etäisyyttä lähtöpisteestä, $y$ korkeutta maan
pinnan tasosta ja vakiot $h$, $k$ ja $a$ ovat yksittäiselle heitolle ominaisia
(ks.\ Luku \ref{parametriset käyrät}, Esimerkki \ref{heittoparaabeli}). Jos
$h$ ja $a$ kiinnitetään, niin millä $k$:n arvolla kappale lentää pisimmälle --- ja kuinka
pitkälle? --- ennen kuin törmää maahan?

\end{enumerate}
 % Ääriarvot. Sileys
\section{Differentiaalilaskun väliarvolause} \label{väliarvolause 2}
\alku

Seuraava lause kuuluu matemaattisen analyysin huomattavimpiin ja myös hyvin usein käytettyihin
perustuloksiin. Lause on väliarvolauseiden sarjan toinen --- vrt.\ Ensimmäinen väliarvolause 
Luvussa \ref{jatkuvuuden käsite} (Lause \ref{ensimmäinen väliarvolause}). 
\begin{Lause} \label{toinen väliarvolause}
\index{Differentiaalilaskun väliarvolause|emph}
\index{vzy@väliarvolauseet!b@differentiaalilaskun|emph} 
\vahv{(Toinen väliarvolause -- Differentiaalilaskun väliarvolause)} Jos $f:\DF_f\kohti\R$, 
$\DF_f\subset\R$, on jatkuva välillä $[a,b]\subset\DF_f$ ja derivoituva välillä $(a,b)$, niin 
jollakin $\xi\in (a,b)$ on
\[
f(b)-f(a)=f'(\xi)(b-a).
\]
\end{Lause} 
\vspace{1mm}
\begin{multicols}{2} \raggedcolumns
Lauseen väittämä on geometrisesti hyvin uskottava (vrt.\ kuvio), mutta todistuksessa joudutaan 
kuitenkin tekemisiin jatkuvuuden syvällisemmän logiikan kanssa.
\begin{figure}[H]
\setlength{\unitlength}{1cm}
\begin{center}
\begin{picture}(6,3.5)
\put(0,0){\vector(1,0){6}} \put(5.8,-0.4){$x$}
\put(0,0){\vector(0,1){3.5}} \put(0.2,3.3){$y$}
% f(x)=0.5(x+1)+0.2(x-1)(x-2.5)(x-5)
\curve(
    1.0000,    1.0000,
    1.2500,    1.3534,
    1.5000,    1.6000,
    1.7500,    1.7406,
    2.0000,    1.8000,
    2.2500,    1.7969,
    2.5000,    1.7500,
    2.7500,    1.6781,
    3.0000,    1.6000,
    3.2500,    1.5344,
    3.5000,    1.5000,
    3.7500,    1.5156,
    4.0000,    1.6000,
    4.2500,    1.7719,
    4.5000,    2.0500,
    4.7500,    2.4531,
    5.0000,    3.0000)  
\put(0.9,0.85){$\bullet$}\put(4.9,2.85){$\bullet$}
\drawline(1,1)(5,3)
\drawline(3,1.1)(5,2.1)
\dashline{0.2}(4,0)(4,1.6)
\dashline{0.2}(1,1)(1,0) \dashline{0.2}(5,3)(5,0)
\put(0.9,-0.5){$a$} \put(4,-0.5){$\xi$} \put(4.9,-0,5){$b$}
\put(1,2.5){$y=f(x)$}
\end{picture}
\end{center}
\end{figure}
\end{multicols}
Lauseen \ref{toinen väliarvolause} todistamiseksi palautetaan väittämä ensinnäkin 
yksinkertaisemmaksi määrittelemällä
\[
g(x)=f(x)-f(a)\cdot\frac{b-x}{b-a}-f(b)\cdot\frac{x-a}{b-a}, \quad x\in [a,b].
\]
Tällöin on $g(a)=g(b)=0$ ja pätee
\[
f(b)-f(a) = f'(\xi)(b-a) \qekv g'(\xi) = 0.
\]
Lauseen \ref{toinen väliarvolause} väittämä saa näin seuraavan pelkistetymmän muodon 
(vrt.\ Ensimmäinen väliarvolause ja sen pelkistetty muoto, Bolzanon lause, Luvussa 
\ref{jatkuvuuden käsite})\,:
\begin{Lause} \label{Rollen lause} \vahv{(Rollen lause)} \index{Rollen lause|emph}
Jos $f$ on jatkuva välillä $[a,b]$ ja derivoituva välillä $(a,b)$ ja $f(a)=f(b)=0$, niin
$f'(\xi)=0$ jollakin $\xi\in (a,b)$. 
\end{Lause}
\tod Koska $f$ on jatkuva välillä $[a,b]$, niin Weierstrassin lauseen 
(Lause \ref{Weierstrassin peruslause}) perusteella $f$ saavuttaa minimi- ja maksimiarvonsa 
välillä $[a,b]$. Kun suljetaan pois ilmeinen tapaus $f(x)=0 \ \forall x\in [a,b]$ (jolloin
$f'(\xi)=0 \ \forall \xi\in (a,b)$), niin on oltava
\begin{align*}
\text{joko:} \quad f(\xi) &= f_{\text{min}}<0, \quad \xi\in (a,b), \\
\text{tai:} \quad f(\xi) &= f_{\text{max}}>0, \quad \xi\in (a,b).
\end{align*}
Kummassakin tapauksessa on oltava $f'(\xi)=0$ (Lause \ref{ääriarvolause}). Lause 
\ref{Rollen lause} on näin todistettu, ja tämän välittömänä seurauksena myös Lause 
\ref{toinen väliarvolause}. --- Huomattakoon, että todistus oli verrattain mutkaton vain siksi, 
että siinä oli 'kova ydin' (Weierstrassin lause). \loppu
\begin{Exa} Funktio $f(x)=\sqrt{\abs{x}}$ toteuttaa Lauseen \ref{toinen väliarvolause} ehdot
välillä $[0,1]$, ja väitetty $\xi$ on yksikäsitteinen: $\xi=\tfrac{1}{4}\in(0,1)$. Välillä
$[-1,1]$ Lauseen \ref{toinen väliarvolause} ehdot eivät toteudu, koska $f$ ei ole derivoituva
pisteessä $x=0$. Väitettyä pistettä $\xi$ (jossa olisi oltava $f'(\xi)=0$) ei tässä tapauksessa
myöskään ole. \loppu
\end{Exa}
Differentiaalilaskun väliarvolauseella on hyvin monia käyttömuotoja tutkittaessa derivoituvien
(tai 'melkein derivoituvien') funktioiden ominaisuuksia. Jatkossa esitellään näistä 
käyttömuodoista keskeisimmät.

\subsection{Funktion monotonisuus}
\index{funktio B!a@monotoninen|vahv} \index{monotoninen!b@reaalifunktio|vahv}

Yksinkertaisissa tapauksissa voidaan pelkin algebran keinoin selvittää, millä väleillä annettu
funktio on monotoninen (kasvava tai vähenevä, vrt.\ Luku \ref{yhden muuttujan funktiot}).
Yleisemmin tehtävä helpottuu huomattavasti, kun otetaan käyttöön seuraava Lauseesta
\ref{toinen väliarvolause} johdettava väittämä.
\begin{Lause} \label{monotonisuuskriteeri}
Olkoon $f$ jatkuva välillä $[a,b]$ ja olkoon $X\subset (a,b)$ äärellinen pistejoukko siten,
että pätee
\begin{itemize}
\item[(1)] $f$ on derivoituva jokaisessa pisteessä $x\in (a,b),\ x \notin X$.
\item[(2)] On voimassa $(\star)\ \forall x\in (a,b),\ x\notin X$, missä ($\star$) on jokin 
           seuraavista vaihtoehdoista:
           \[
           \begin{array}{ll}
           \text{(a)}\quad f'(x)\geq 0, \quad &\text{(b)}\quad f'(x)>0, \\
           \text{(c)}\quad f'(x)\leq 0, \quad &\text{(d)}\quad f'(x)<0. \\ 
           \end{array}
           \]
\end{itemize}
Tällöin $f$ on (a) kasvava, (b) aidosti kasvava, (c) vähenevä, (d) aidosti vähenevä välillä 
$[a,b]$.
\end{Lause}
\tod Ol. $x_1,x_2\in [a,b]$, $x_1<x_2$. Koska joukko $X$ on äärellinen, niin voidaan valita 
pistejoukko $\{t_1,\ldots,t_{n-1}\}\subset X$ (mahdollisesti tyhjä joukko) siten, että
\[
x_1=t_0<t_1<\ldots <t_n=x_2\ \ \text{ja}\ \ (t_{k-1},t_k)\cap X=\emptyset, \ k=1\ldots n.
\]
Tällöin kun kirjoitetaan $f(x_2)-f(x_1)$ teleskooppisummaksi ja sovelletaan Lausetta
\ref{toinen väliarvolause}, niin seuraa
\[
f(x_2)-f(x_1) \,=\, \sum_{k=1}^n [f(t_k)-f(t_{k-1})]
              \,=\, \sum_{k=1}^n f'(\xi_k)(t_k-t_{k-1}), \quad \xi_k\in (t_{k-1},t_k),
\]
jolloin tehtyjen oletuksien perusteella päätellään
\[
f(x_2)-f(x_1)=\begin{cases}
\ge 0 &\text{(a)}, \\
> 0   &\text{(b)}, \\
\le 0 &\text{(c)}, \\
< 0   &\text{(d)}. \\
\end{cases} \quad\loppu
\]

Sovellettaessa Lausetta \ref{monotonisuuskriteeri} voidaan joukkoon $X$ aina lukea $f'$:n 
nollakohdat, sikäli kuin niitä on äärellinen määrä. Äärellinen määrä $f'$:n nollakohtia ei siis
häiritse funktion (aitoakaan) monotonisuutta, kunhan $f'$:n merkki ei nollakohdissa vaihdu.
\begin{Exa} Funktio
$\D f(x)=\begin{cases} 
         \,x^3, &\text{kun}\ -1\leq x\leq 1, \\ \,x, &\text{kun} \quad 1<x\leq 2
         \end{cases}$

on jatkuva välillä $[-1,2]$ ja derivoituva välillä $(-1,2)$ lukuunottamatta pistettä $x=1$.
Kun valitaan $X=\{0,1\}$, niin Lauseen \ref{monotonisuuskriteeri} oletus (b) on voimassa,
joten $f$ on välillä $[-1,2]$ aidosti kasvava. \loppu
\end{Exa}
\begin{Exa} \label{monotonisuus: esim} Edellisen luvun Esimerkin
\ref{kriittiset pisteet: esim} mukaan funktiolle $f(x)=x^2/(x+1)$ pätee: $f'(x)>0$
väleillä $(-\infty,-2)$ ja $(0,\infty)$ ja $f'(x)<0$ väleillä $(-2,-1)$ ja $(-1,0)$. Lauseen
\ref{monotonisuuskriteeri} perusteella päätellään, että $f$ on aidosti kasvava väleillä
$(-\infty,-2]$ ja $[0,\infty)$ (eli näiden välien suljetuilla osaväleillä) ja vastaavasti
aidosti vähenevä väleillä $[-2,-1)$ ja $(-1,0]$. 
(Samaan tulokseen tullaan pelkin algebran keinoinkin, mutta työläämmin:
Harj.teht.\,\ref{yhden muuttujan funktiot}:\ref{H-IV-1: funktioalgebran haasteita}b.)
\loppu \end{Exa}

\subsection{Kriittisen pisteen laatu}
\index{kriittinen piste!a@luokittelu|vahv}

Lauseen \ref{monotonisuuskriteeri} monotonisuuskriteereillä voidaan yleensä selvittää helposti,
onko funktion kriittinen piste paikallinen ääriarvokohta (ja minkälaatuinen) vai ei. Nimittäin
asia selviää (ellei $f$ ole poikkeuksellisen 'pahantapainen') tutkimalla derivaatan merkkiä 
kriittisen pisteen lähiympäristössä. Esimerkiksi jos $f'(x)<0$ välillä $(c-\delta_1,c)$ ja 
$f'(x)>0$ välillä $(c,\,c+\delta_2)$ joillakin $\delta_1\,,\delta_2>0$, niin $f$ on välillä 
$[c-\delta_1,c]$ aidosti vähenevä ja välillä $[c,c+\delta_2]$ aidosti kasvava, jolloin $c$:n 
on oltava paikallinen minimikohta. Päättelyssä riittää, että $f$ on pisteessä $c$ jatkuva, ts.\
derivoituvuutta ei tarvitse olettaa (vrt.\ Lause \ref{ääriarvolause 2}).
\jatko \begin{Exa} (jatko) Esimerkissä $f$:n kriittiset pisteet ovat $-2$ ja $0$. Derivaatan
merkin perusteella päätellään, että $x=-2$ on $f$:n paikallinen maksimikohta ja $x=0$ on
paikallinen minimikohta. \loppu
\end{Exa}

\subsection{Funktion (käyrän) kaareutuvuus}
\index{kaareutuvuus (funktion, käyrän)|vahv}

Jos $f'(x)$:n merkki kertoo, onko $f$ kasvava tai vähenevä, niin $f''(x)$:n merkki puolestaan 
kertoo funktion $f$ (tai käyrän $y=f(x)$) \kor{kaareutumissuunnan}. Jos $f$ on derivoituva
avoimella välillä $(a,b)$, niin sanotaan, että $f$ on ko.\ välillä \kor{ylös}(päin) 
\kor{kaareutuva} (egl.\ concave up), jos $f'$ on ko.\ välillä aidosti kasvava, ja
\kor{alas}(päin) \kor{kaareutuva} (engl.\ concave down), jos $f'$ on aidosti vähenevä välillä
$(a,b)$. Jos $f$ on kahdesti derivoituva, lukuunottamatta mahdollisesti äärellistä
pistejoukkoa, niin $f$:n kaareutumissuunta voidaan päätellä $f''$:n merkistä Lauseen
\ref{monotonisuuskriteeri} mukaisesti. Kaareutuvuuden geometrinen tulkinta on ilmeinen,
vrt.\ kuvio.\footnote[2]{Kaareutuvuuden synonyyminä käytetään matemaattisissa teksteissä
myös termiä \kor{kuperuus} (ylös tai alas), mutta tällöin saattaa jäädä epäselväksi, kumpaa
kaareutumissuuntaa tarkoitetaan. Kaareutuvuuden (kuperuuden) käsite voidaan määritellä 
yleisemmin myös derivaatoista riippumatta, jolloin käytetään useammin termejä \kor{konveksi}
ja \kor{konkaavi}. Funktiota $f$ sanotaan konveksiksi välillä $[a,b]\in\DF_f$, jos $f$:llä on
ominaisuus
\[ 
f\bigl(\,tx_1 + (1-t)x_2\bigr) \le t f(x_1) + (1-t)f(x_2), \quad 
                   \text{kun}\ x_1,x_2\in[a,b]\ \text{ja}\ t \in [0,1]. 
\]
Jos tässä ehdossa epäyhtälö toteutuu muodossa '$<$' aina kun $x_1 \neq x_2$ ja $t \in (0,1)$,
niin $f$ on välillä $[a,b]$ \kor{aidosti konveksi}. Geometrisesti aito konveksisuus
tarkoittaa, että jos $x_1,\,x_2\in[a,b]$ ja $x_1 < x_2$, niin funktion $f$ kuvaaja on
pisteiden $(x_1,f(x_1))$ ja $(x_2,f(x_2))$ kautta kulkevan suoran alapuolella avoimella
välillä $(x_1,x_2)$. Välillä $[a,b]$ jatkuva ja välillä $(a,b)$ ylös kaareutuva (riittävästi
derivoituva) funktio on määritelmien mukaisesti aidosti konveksi. Alaspäin kaareutuvuutta
vastaava käsite konkaavius määritellään vastaavasti.\index{kuperuus|av}
\index{konveksi, konkaavi|av} \index{aidosti konveksi|av}} 
\begin{figure}[H]
\setlength{\unitlength}{1cm}
\begin{center}
\begin{picture}(10,4.5)(0,-1)
\multiput(0,0)(6,0){2}{
\put(0,0){\vector(1,0){4}} \put(3.8,-0.4){$x$}
\put(0,0){\vector(0,1){3}} \put(0.2,2.8){$y$}
}
\curve(0.5,0.5,2.5,1.3,4,3) \put(1,2){$y=f(x)$}
\curve(6.5,0.5,8,2.8,10,2.6) \put(8,2){$y=f(x)$}
\put(1.3,-1){$f''>0$} \put(7.3,-1){$f''<0$}
\put(0.3,-1.6){(ylös kaareutuva)} \put(6.4,-1.6){(alas kaareutuva)}
\end{picture}
\end{center}
\end{figure}
Pistettä, jossa kaareutumissuunta vaihtuu, sanotaan (funktion/käyrän)
\index{kzyzy@käännepiste}
\kor{käännepisteeksi} (engl.\ inflection point). Jos $f''$ on jatkuva käännepisteessä $c$, niin
on oltava $f''(c)=0$.
\begin{Exa} Derivoimalla todetaan, että kolmannen asteen polynomilla 
$f(x)=x^3+ax^2+bx+c\ (a,b,c\in\R)$ on täsmälleen yksi käännepiste: $\,x=-\tfrac{1}{3}a$.
Välillä $(-\infty,-\tfrac{1}{3}a)$ $f$ on alas ja välillä $(-\tfrac{1}{3}a,\infty)$ ylös
kaareutuva. \loppu
\end{Exa}

\subsection{Lipschitz-jatkuvuus}

Luvussa \ref{jatkuvuuden käsite} määritelty funktion jatkuvuus voidaan tulkita funktiota
koskevaksi \pain{minimi}oletukseksi, kun halutaan taata fuktioevaluaation $x \map f(x)$
luotettavuus. Seuraavassa esitellään 'pelkkää jatkuvuutta' vahvempi jatkuvuuden laji, jolla
jatkossa on silloin tällöin (etenkin teoreettista) käyttöä.
\begin{Def} \label{funktion l-jatkuvuus} \index{Lipschitz-jatkuvuus|emph}
Funktio $f$ on \kor{Lipschitz-jatkuva} eli \kor{Lipschitz}\footnote[2]{\hist{Rudolf Lipschitz}
(1832-1903) oli saksalainen matemaatikko.\index{Lipschitz, R.|av}} \kor{välillä}
$[a,b]\subset\DF_f$, jos jollakin $L\in\R_+$ pätee
\[
\abs{f(x_1)-f(x_2)}\leq L\abs{x_1-x_2} \quad \forall x_1,x_2\in [a,b].
\]
\end{Def}
Määritelmän lukua $L$ sanotaan $f$:n \kor{Lipschitz-vakioksi} välillä $[a,b]$. Vakio $L$ ei
ole yksikäsitteinen, sillä jos $L$ on $f$:n Lipschitz-vakio, niin määritelmän mukaan samoin
on jokainen $L_1 > L$. Jos $f$ on Lipschitz välillä $[a,b]$, niin $f$:n pienin
Lipschitz-vakion arvo on pienin yläraja (määritelmän perusteella rajoitetulle)
reaalilukujoukolle
\[
A = \left\{ y = \frac{\abs{f(x_1)-f(x_2)}}{\abs{x_1-x_2}}\ 
                              \Big\vert\ x_1,x_2 \in [a,b]\ \ja\ x_1 \neq x_2\right\},
\]
ts.\ $\ L_{min} = \sup A$ (vrt.\ Luku \ref{reaalilukujen ominaisuuksia}).

Lipschitz-jatkuvuus on käsitteenä lähellä derivoituvuutta, ja Lipschitz-jatkuvia funktioita
voikin luonnehtia 'melkein derivoituviksi'.
\begin{Exa} Funktio $f(x)=\abs{x}$ on jokaisella välillä Lipschitz-jatkuva vakiolla $L=1$
(= pienin $L$:n arvo), sillä kolmioepäyhtälön nojalla
\[
\abs{f(x_1)-f(x_2)} \,=\, \abs{\abs{x_1}-\abs{x_2}} 
                    \,\le\, \abs{x_1-x_2}, \quad x_1,\,x_2\in\R. \loppu
\]
\end{Exa}
Esimerkin funktio on myös paloittain jatkuvasti derivoituva jokaisella välillä
(Määritelmä \ref{paloittainen sileys}, $\,m=1$). Yhdessä jatkuvuuden kanssa tämä on
yleisemminkin riittävä tae Lipschitz-jatkuvuudelle:
\begin{Lause} \label{Lipschitz-kriteeri} Jos $f$ on välillä $[a,b]$ jatkuva ja paloittain
jatkuvasti derivoituva, niin $f$ on Lipschitz-jatkuva välillä $[a,b]$.
\end{Lause}
Todistuksen (Harj.teht.\,\ref{H-V-6: Lipschitz-kriteeri}) johdannoksi näytettäköön toteen
seuraava väittämä, jonka osavättämän (ii) Lause \ref{Lipschitz-kriteeri} yleistää. Todistus
on tyypillinen esimerkki Diferentiaalilaskun väliarvolauseen soveltamisesta.
\begin{Lause} \label{Lipschitz-kriteeri 1}
Jos $f$ on välillä $[a,b]$ jatkuvasti derivoituva, niin
\begin{align*}
\text{(i)}  \quad 
            &\dif_+f(a) = f'(a^+),\,\ \dif_-f(b) = f'(b^-), \\
\text{(ii)} \quad 
            &\text{$f$ on välillä $[a,b]$ Lipschitz-jatkuva vakiolla, jonka pienin arvo on} \\
            &L=\max_{x\in [a,b]} \abs{f'(x)}.
\end{align*}
\end{Lause}
\tod (i) Olkoon $\seq{x_n}$ jono, jolle pätee $a < x_n \le b\ \forall n$ ja $x_n \kohti a^+$. 
Tällöin Lauseen \ref{toinen väliarvolause} mukaan jokaisella $n$ on olemassa 
$\xi_n \in (a,x_n)$ siten, että
\[
\frac{f(x_n)-f(a)}{x_n-a} = f'(\xi_n).
\]
Tässä $\xi_n \kohti a^+$ kun $x_n \kohti a^+$, joten toispuolisen derivaatan $\dif_+f(a)$ 
määritelmän ja $f'$:n oletetun toispuolisen jatkuvuuden perusteella seuraa
\[
\dif_+f(a) = \lim_{x_n\kohti a^+}\,\frac{f(x_n)-f(a)}{x_n-a} 
           = \lim_{x_n\kohti a^+} f'(\xi_n) = f'(a^+).
\]
Väittämän (i) toinen osa todistetaan vastaavasti.

(ii) Jos $x_1,x_2\in [a,b]$ ja $x_1<x_2$, niin Lauseen \ref{toinen väliarvolause} mukaan
\[
f(x_2)-f(x_1)=f'(\xi)(x_2-x_1),\quad \xi\in (x_1,x_2).
\]
Koska $f'$ on jatkuva välillä $[a,b]$, niin Weierstrassin lauseen (Lause
\ref{Weierstrassin peruslause}) mukaan $\abs{f'(x)}$ saavuttaa välillä $[a,b]$ maksimiarvonsa.
Kun tämä merkitään $=L$, niin nähdään, että $L$ kelpaa $f$:n Lipschitz-vakion arvoksi.
Valitsemalla $x_1$ ja $x_2$ $\abs{f'(x)}$:n maksimikohdan läheltä (vrt.\ osaväittämän (i)
todistus) nähdään myös helposti, että tämä $L$:n arvo on pienin mahdollinen. \loppu
\begin{Exa} Jos $0<a<b$, niin funktio $f(x)=\sqrt{x}$ on Lauseen \ref{Lipschitz-kriteeri 1}
mukaan välillä $[a,b]$ Lipschitz-jatkuva vakiolla $L=1/(2\sqrt{a})$. Väleillä $[0,b]$ $f$ ei
ole Lipschitz-jatkuva (ainoastaan jatkuva), sillä jos valitaan $x_1=0$ ja $x_2>0$, niin
\[
\frac{\abs{f(x_1)-f(x_2)}}{\abs{x_1-x_2}} = \frac{1}{\sqrt{x_2}} \kohti \infty, \quad
                                            \text{kun}\ x_2 \kohti 0^+. \loppu
\]
\end{Exa}

\subsection{Differentiaaliyhtälö $y'=f(x)$. Integraalifunktio}
\index{differentiaaliyhtälö|vahv} \index{integraalifunktio|vahv}

Derivoimissääntöjen perusteella 'vakion derivaatta on nolla', ts.\ jos $f(x)=C$, $x\in(a,b)$
jollakin $C\in\R$, niin $f'(x)=0,\ x\in(a,b)$. Differentiaalilaskun väli\-arvo\-lauseesta 
seuraa, että väittämä pätee myös käänteisesti muodossa: \pain{vain} vakion derivaatta $=0$. 
Tämä yksinkertainen tulos osoittautuu seuraamuksiltaan huomattavaksi.
\begin{Lause} (\vahv{Integraalilaskun\footnote[2]{\kor{Integraalilaskenta} on matematiikan
laji, jota tarkastellaan perusteellisemmin Luvussa \ref{Integraali}.} peruslause}) 
\label{Integraalilaskun peruslause} \index{Integraalilaskun peruslause|emph}
Jos $f$ on derivoituva välillä $(a,b)$ ja $f'(x)=0\ \forall x\in(a,b)$, niin jollakin $C\in\R$
on $f(x)=C,\ x\in(a,b)$.
\end{Lause}
\tod Jos $c\in(a,b)$ ja $x\in(a,b),\ x \neq c$, niin Differentiaalilaskun väliarvolauseen
oletukset ovat voimassa välillä $[c,x]$ ($x>c$) tai välillä $[x,c]$ ($x<c$). Näin ollen
$f(x)-f(c)=f'(\xi)(x-c)$ jollakin $\xi\in(c,x)\subset(a,b)$ tai $\xi\in(x,c)\subset(a,b)$.
Koska $f'(\xi)=0\ \forall\xi\in(a,b)$, niin seuraa $f(x)=f(c)=C,\ x\in(a,b)$. \loppu

Lause \ref{Integraalilaskun peruslause} voidaan tulkita väittämänä, joka koskee hyvin
yksinkertaista (itse asiassa kaikkein yksinkertaisinta) \kor{differentiaaliyhtälöä}
\[
y'=0.
\]
Tässä $y(x)$ on tarkasteltavalla välillä $(a,b)$ derivoituva (tuntematon)
funktio\footnote[3]{Differentiaaliyhtälöissä käytetään yleensä funktiosymbolia $y$ (ei $f$),
koska kyseeessä on epäsuorasti määritelty funktio (tai funktiojoukko), vrt.\ 
implisiittifunktiot (Luku \ref{käänteisfunktio}).}. 
Differentiaaliyhtälön \kor{ratkaisu} on jokainen funktio $y(x)$, joka toteuttaa yhtälön ko.\
välillä. Lauseen \ref{Integraalilaskun peruslause} mukaan diferentiaaliyhtälön $y'=0$ jokainen
ratkaisu on vakio, eli ko.\ differentiaaliyhtälön \kor{yleinen ratkaisu} välillä $(a,b)$ on
\[
y(x)=C, \quad x\in(a,b),
\]
missä $C$ on nk.\ \kor{määräämätön vakio} ($C\in\R$). 

Em.\ tulos on helposti yleistettävissä koskemaan differentiaaliyhtälöä
\[
y'(x)=f(x), \quad x\in(a,b),
\]
missä $f$ on tunnettu funktio välillä $(a,b)$.
\begin{Kor} \label{toiseksi yksinkertaisin dy} Jos $F$ on välillä $(a,b)$ derivoituva funktio
ja $\,F'(x)=f(x)$, $x\in(a,b)$, niin differentiaaliyhtälön $y'=f(x)$ yleinen ratkaisu välillä
$(a,b)$ on
\[
y(x)=F(x)+C, \quad x\in(a,b).
\]
\end{Kor}
\tod Jos $y$ differentiaaliyhtälön ratkaisu ja merkitään $u(x)=y(x)-F(x)$, niin
$\,u'(x)=f(x)-f(x)=0$, $x\in(a,b)$, jolloin Lauseen \ref{Integraalilaskun peruslause} mukaan
on oltava
\[
u(x)=C\,\ \ekv\,\ y(x)=F(x)+C, \quad x\in(a,b). \loppu
 \]
\begin{Exa} \label{V-6: dyex1} Differentiaaliyhtälön
\[
y'=x^3-2x, \quad x\in\R
\]
(tässä $(a,b)=(-\infty,\infty)$) ratkaisemiseksi on etsittävä funktio $F$, jolle pätee
$F'(x)=x^3-2x,\ x\in\R$. Derivoimissääntöjen perusteella nähdään, että voidaan valita
$F(x)=\frac{1}{4}x^4-x^2$, joten yleinen ratkaisu on
\[
y(x)=\frac{1}{4}x^4-x^2+C. \loppu
\]
\end{Exa}
\begin{Exa} Mikä on differentiaaliyhtälön
\[
y''=0, \quad x\in\R
\]
yleinen ($\R$:ssä kahdesti derivoituva) ratkaisu?
\end{Exa}
\ratk Jos merkitään $u(x)=y'(x)$, niin differentiaaliyhtälö pelkistyy muotoon $u'=0$, joten
$u(x)=C_1\,,\ x\in\R$ (Lause\ref{Integraalilaskun peruslause}), ja näin ollen
$y'=C_1\,,\ x\in\R$. Korollaarin \ref{toiseksi yksinkertaisin dy} ja derivoimissääntöjen mukaan
on tällöin oltava
\[
y(x)=C_1x+C_2\,, \quad x\in\R.
\]
Tässä $C_1$ ja $C_2$ ovat molemmat määräämättömiä (toisistaan riippumattomia) vakioita, joten
yleinen ratkaisu koostuu kaikista polynomeista astetta $\le 1$. \loppu

Differentiaaliyhtälön $y'=f(x)$ ratkaisuja sanotaan funktion $f$ \kor{integraalifunktioiksi}.
Integraalifunktio on siis määräämätöntä vakiota, nk.\ 
\index{integroimisvakio}%
\kor{integroimisvakiota} vaille
yksikäsitteinen. Sovelluksissa integroimisvakio määräytyy usein (sovelluksesta peräisin
olevasta) 
\index{alkuehto (DY:n)} \index{alkuarvotehtävä}%
\kor{alkuehdosta} muotoa $y(x_0)=y_0$. Kyseessä on tällöin nk.\ \kor{alkuarvotehtävä}
(engl.\ initial value problem)
\[ \begin{cases}
   \,y'=f(x), \quad x\in(a,b), \\ \,y(x_0)=y_0,
   \end{cases} \]
missä $x_0\in(a,b)$ ja $y_0\in\R$ on annettu. Ratkaisu on
\[
y(x)=F(x)-F(x_0)+y_0\,,
\]
missä $F$ on (mikä tahansa) $f$:n integraalifunktio. Ratkaisu on yksikäsitteinen, sillä
$F(x)-F(x_0)$ ei muutu, jos $F$:ään lisätään vakio.
\jatko \begin{Exa} (jatko) Alkuarvotehtävän
\[ \begin{cases}
   \,y''=0, \quad x\in\R, \\ \,y(1)=0,\ y'(1)=2
   \end{cases} \]
yksikäsitteinen ratkaisu on $y(x)=2x-2$. \loppu
\end{Exa}

Annetun funktion $f$ 'integroiminen' eli integraalifunktion etsiminen on hyvin perinteinen
matematiikan taitolaji, jonka laajempi esittely kuuluu integraalilaskun yhteyteen
(ks.\ Luvut \ref{integraalifunktio}--\ref{osamurtokehitelmät} jäljempänä). Toistaiseksi
todettakoon ainoastaan, että derivoimissääntöjen perusteella helposti integroitavissa
ovat mm.\ trigonometriset funktiot $\,\sin\,$ ja $\,\cos$, samoin polynomit (vrt.\ Esimerkki
\ref{V-6: dyex1}). Myös funktio, joka on määritelty välillä $(-\rho,\rho)$ suppenevan
potenssisarjan summana on ko.\ välillä (yllättävänkin) helposti integroitavissa. Esimerkki
valaiskoon asiaa.
\begin{Exa} Funktion $f(x)=x^2/(1-x)$ integraalifunktio ei ole (lausekkeena) esitettävissä
toistaiseksi tunnettujen funktioiden avulla. Välillä $(-1,1)$ on integraalifunktio
kuitenkin määrättävissä, sillä tällä välillä pätee
\[
f(x)=\frac{x^2}{1-x}=x^2\sum_{k=0}^\infty x^k=\sum_{k=0}^\infty x^{k+2}, \quad x\in(-1,1).
\]
Potenssisarjan derivoimissäännön (Lause \ref{potenssisarja on derivoituva}) perusteella
päätellään, että $f$:n integraalifunktio välillä $(-1,1)$ on
\[
F(x)\,=\,C+\sum_{k=0}^\infty \frac{x^{k+3}}{k+3}\,
      =\,C+\frac{x^3}{3}+\frac{x^4}{4}+ \ldots \loppu
\]
\end{Exa}

\subsection{l'Hospitalin säännöt}

\kor{l'Hospitalin säännöillä} tarkoitetaan derivointiin perustuvia raja-arvojen
laskusääntöjä muotoa
\[
\lim_{x \kohti a^\pm} \frac{f(x)}{g(x)} = \lim_{x \kohti a^\pm} \frac{f'(x)}{g'(x)}\,.
\]
Hankalien raja-arvojen laskemisessa nämä säännöt ovat varteenotettava --- myös helposti
muistettava --- vaihtoehto esim.\ muuttujan vaihdolle (vrt.\ Luku \ref{funktion raja-arvo}).
Säännöt pätevät lievin lisäehdoin sellaisissa tapauksissa, joissa $f(a)/g(a)$ on joko muotoa
$0/0$ tai $\infty/\infty$.

l'Hospitalin sääntöjen taustalla on jälleen Differentiaalilaskun väliarvolause, tarkemmin
seuraava nk.\ \kor{yleistetty väliarvolause}, joka sekin on Rollen lauseesta johdettavissa
(Harj.teht.\,\ref{H-V-6: yleistetty väliarvolause}).
\begin{Lause} \label{yleistetty väliarvolause} 
\index{vzy@väliarvolauseet!b@differentiaalilaskun|emph}
\index{Differentiaalilaskun väliarvolause!yleistetty väliarvolause|emph} Jos $f$ ja $g$ ovat
molemmat jatkuvia välillä $[a,b]$ ja derivoituvia välillä $(a,b)$ ja lisäksi $g'(x) \neq 0$
$\forall x\in(a,b)$, niin on olemassa $\xi\in(a,b)$ siten, että
\[
\frac{f(b)-f(a)}{g(b)-g(a)} = \frac{f'(\xi)}{g'(\xi)}\,.
\]
\end{Lause}
\begin{Lause} (\vahv{l'Hospitalin\footnote[2]{Ranskalainen markiisi ja matemaatikko
\hist{Guillaume de l'Hospital} (1661-1704) tunnetaan ennen muuta hänen julkaisemastaan
ensimmäisestä differentiaalilaskennan oppikirjasta (1696).
\index{l'Hospital G. de|av}} säännöt}) \label{Hospital} \index{l'Hospitalin säännöt|emph}
\begin{itemize}
\item[1.] Olkoot $f$ ja $g$ derivoituvia välillä $(a,a+\delta),\ \delta>0$, ja olkoon
          $g'(x) \neq 0$ tällä välillä. Edelleen olkoon
          $\lim_{x \kohti a^+}f(x)=\lim_{x \kohti a^+}g(x) = 0$. Tällöin pätee
          \[
          \lim_{x \kohti a^+} \frac{f(x)}{g(x)} = \lim_{x \kohti a^+} \frac{f'(x)}{g'(x)} = A
          \]
          sikäli kuin raja-arvo oikealla on olemassa ($A\in\R$ tai $A=\pm\infty$).
\item[2.] Olkoot $f$ ja $g$ derivoituvia välillä $(M,\infty),\ M\in\R$, ja olkoon
          $g'(x) \neq 0$ tällä välillä. Edelleen olkoon
          $\lim_{x\kohti\infty}f(x)=\lim_{x\kohti\infty}g(x)=0$. Tällöin pätee
          \[
          \lim_{x\kohti\infty} \frac{f(x)}{g(x)} = \lim_{x\kohti\infty} \frac{f'(x)}{g'(x)} = A
          \]
          sikäli kuin raja-arvo oikealla on olemassa ($A\in\R$ tai $A=\pm\infty$).
\item[3.] Säännöt 1--2 ovat päteviä myös, kun funktioiden $f$ ja $g$ raja-arvoja koskevat
          oletukset muutetaan muotoon: $\,\lim|f(x)|=\lim|g(x)|=\infty$.
\end{itemize}
\end{Lause}
\tod \ Sääntö 1. Asetetaan $f(a)=g(a)=0$, jolloin oletusten perusteella $f$ ja $g$ ovat
jatkuvia välillä $[a,b]$, kun $a<b<a+\delta$. Näin ollen jos $a<x<a+\delta$, niin
Lauseen \ref{yleistetty väliarvolause} perusteella pätee jollakin $\xi\in(a,x)$
\[
\frac{f(x)}{g(x)} = \frac{f(x)-f(a)}{g(x)-g(a)} = \frac{f'(\xi)}{g'(\xi)}\,.
\]
Tässä $\xi \kohti a^+$ kun $x \kohti a^+$, joten sääntö seuraa.

Sääntö 2. Tehdään muuttujan vaihto $x=1/t$ ja sovelletaan 1.\ sääntöä:
\[
\lim_{x\kohti\infty}\frac{f(x)}{g(x)} 
   \,=\, \lim_{t \kohti 0^+} \frac{f(\tfrac{1}{t})}{g(\tfrac{1}{t})}
   \,=\, \lim_{t \kohti 0^+} \frac{-\tfrac{1}{t^2}f'(\tfrac{1}{t})}
                                 {-\tfrac{1}{t^2}g'(\tfrac{1}{t})}
   \,=\, \lim_{t \kohti 0^+} \frac{f'(\tfrac{1}{t})}{g'(\tfrac{1}{t})}
   \,=\, \lim_{x\kohti\infty} \frac{f'(x)}{g'(x)}\,.
\]
Sääntö 3. Myös tämä perustuu Lauseeseen \ref{yleistetty väliarvolause}, mutta todistus on
melko työläs. Todistus sivuutetaan (ks.\ Harj.teht\,\ref{H-V-6: Hospital 3}). \loppu

Lauseen \ref{Hospital} säännöillä on ilmeiset vastineensa raja-arvoille 
$\lim_{x \kohti a^-}f(x)/g(x)$ ja $\lim_{x \kohti -\infty}f(x)/g(x)$. Säännöt ovat samoin pätevät
raja-arvolle $\lim_{x \kohti a}f(x)/g(x)$ ($a\in\R$) olettaen, että $f$ ja $g$ ovat 
derivoituvia väleillä $(a-\delta,a)$ ja $(a,a+\delta)$ ja $g'(x) \neq 0$ näillä väleillä.
\begin{Exa} l'Hospitalin 1. säännöllä laskien saadaan 
(vrt.\ Esimerkki \ref{funktion raja-arvo}:\ref{raja-arvo muuttujan vaihdolla})
\[
\lim_{x \kohti 81} \frac{\sqrt{x}-9}{\sqrt[4]{x}-3} 
  \,=\, \lim_{x \kohti 81} \frac{\tfrac{1}{2}x^{-1/2}}{\tfrac{1}{4}x^{-3/4}}
  \,=\, \lim_{x \kohti 81} 2x^{1/4} \,=\, 6.
\] 
Samaa sääntöä soveltaen seuraa myös
\[
\lim_{x \kohti 0} \frac{\sin x}{x} \,=\, \lim_{x \kohti 0}\frac{\cos x}{1} \,=\, 1.
\]
Tämä lasku kuitenkin kätkee kehäpäätelmän: Raja-arvoa laskettaessa käytetään
derivoimissääntöä, joka perustui ko.\ raja-arvoon (!) (ks.\ Luku \ref{kaarenpituus}). \loppu
\end{Exa}

\Harj
\begin{enumerate}

\item
Määritä seuraavien funktioiden kriittiset pisteet ja välit, joilla funktiot ovat aidosti
kasvavia tai väheneviä. Määritä myös funktioiden absoluuttiset minimi- ja maksimiarvot
määrittelyjoukossaan, sikäli kuin olemassa. Hahmottele funktioiden kuvaajat.
\begin{align*}
&\text{a)}\ f(x)=1-6x+9x^2-5x^3 \qquad
 \text{b)}\ f(x)=x^4+x \qquad
 \text{c)}\ f(x)=\frac{9+x^2}{1+x} \\
&\text{d)}\ f(x)=\frac{x^4+x+1}{x^4+1} \qquad 
 \text{e)}\ f(x)=x\sqrt{4-x^2} \qquad
 \text{f)}\ f(x)=\sqrt{3x^2-x^3} \\
&\text{g)}\ f(x)=\frac{x^2}{\sqrt{4-x^2}} \qquad
 \text{h)}\ f(x)=\frac{x}{\sqrt{x^4+1}} \qquad
 \text{i)}\ f(x)=x-\sin x \\
&\text{j)}\ f(x)=x+2\cos x \qquad
 \text{k)}\ f(x)=x-\frac{2}{\sin x} \qquad
 \text{l)}\ f(x)=x-2\tan x \\
&\text{m)}\ f(x)=1-\frac{1}{2}x^2-\cos x
\end{align*}

\item
a) Määritä funktion $f(x)=3x^4-2x^3+15x^2+10x-20$ pienin Lipschitz-vakio välillä $[0,6]$. \
b) Näytä, että jos $f$ on Lipscitz-jatkuva välillä $[a,b]$ vakiolla $L$, niin käyrän
$S: y=f(x)$ kaari välillä $[a,b]$ on suoristuva ja kaarenpituudelle pätee arvio
$s \le \sqrt{1+L^2}\,(b-a)$. \ c) Näytä, että jos $f$ ja $g$ ovat Lipschitz-jatkuvia
välillä $[a,b]$ vakioilla $L_1$ ja $L_2$, niin $\alpha f+\beta g$ on välillä $[a,b]$
Lipschitz-jatkuva vakiolla $L=\abs{\alpha}L_1+\abs{\beta}L_2$ $(\alpha,\beta\in\R)$.

\item \label{H-V-6: Lipschitz-kriteeri}
Näytä, että jos $f$ on jatkuva ja paloittain jatkuvasti derivoituva välillä $[a,b]$, niin
$f$ on välillä $[a,b]$ Lipschitz-jatkuva vakiolla, jonka pienin arvo on
$L=\max_{x\in[a,b]} g(x)$, missä $g(a)=\abs{\dif_+f(a)}$, $\,g(b)=\abs{\dif_-f(b)}$ ja
$g(x)=$ $\max\{\abs{\dif_-f(x)},\,\abs{\dif_+f(x)}\},\ x\in(a,b)$.
Laske tulosta soveltaen funktioiden $\,f(x)=\abs{x-1}+\abs{x}+\abs{x+1}\,$ ja
$\,g(x)=\max\{2x^3-2x,\,-2x^2-4x,\,3-2x\}\,$ pienin Lipschitz-vakio välillä $[-2,2]$.

\item
Määritä seuraavien funktioiden käännepisteet ja kaareutumissuunnat eri $\R$:n
osaväleillä: \vspace{1mm}\newline
a) \ $f(x)=3x^5+35x^4+100x^2-200x\ \quad$   b) \ $f(x)=3x^5-10x^4+10x^2$ \newline
c) \ $f(x)=2x^3+x+1-1/x\ \qquad\qquad\quad$ d) \ $f(x)=x/(x^2+1)$

\item
Johda kaava $\,\Arcsin x + \Arccos x=\frac{\pi}{2},\ x\in(-1,1)\,$ derivoimissäännöistä ja
Lauseesta \ref{Integraalilaskun peruslause}.

\item
Ratkaise (yleinen ratkaisu tai alkuarvotehtävän ratkaisu) joko lausekkeena tai potenssisarjan
avulla:
\begin{align*}
&\text{a)}\,\ y'=4x^7-5x^3+3,\,\ x\in\R \qquad
 \text{b)}\,\ y'=\frac{x^5-1}{x-1}\,,\,\ x\in(1,\infty),\,\ y(1)=1 \\[1mm]
&\text{c)}\,\ y'=\sqrt{x}-\sqrt[3]{x},\,\ x\in(0,\infty) \quad\,\ \
 \text{d)}\,\ y'=\frac{1}{\sqrt{x}}\,,\,\ x\in(0,\infty),\,\ y(1)=-1 \\[1mm]
&\text{e)}\,\ y'=\sin x+\cos x,\,\ x\in\R \qquad\,\ \
 \text{f)}\,\ y'=\frac{\cot x}{\sin x}\,,\,\ x\in(0,\pi),\,\ y(\tfrac{\pi}{2})=1 \\[1mm]
&\text{g)}\,\ y'=\sum_{k=1}^\infty \frac{x^k}{k}\,,\,\ x\in(-1,1) \qquad\,\
 \text{h)}\,\ y'=\sum_{k=0}^\infty \frac{x^k}{k!}\,,\,\ x\in\R,\,\ y(0)=1 \\
&\text{i)}\,\ y'=\sum_{k=0}^\infty \frac{(-1)^k}{k!}x^k,\,\ x\in\R \qquad\,\ \
 \text{j)}\,\ y'=\frac{1}{1+x}\,,\,\ x\in(-1,1),\,\ y(0)=2 \\
&\text{k)}\,\ y'=\frac{1}{1-x^2}\,,\,\ x\in(-1,1) \qquad\,\
 \text{l)}\,\ y'=\frac{2+3x}{1-x^2}\,,\,\ x\in(-1,1),\,\ y(0)=0
\end{align*}

\item
Näytä, että jos $y(x)$ on $\R$:ssä $n$ kertaa derivoituva ($n\in\N$) ja $y^{(n)}=0$, niin
$y$ on polynomi astetta $\le n$. Mikä on $y(x)$:n lauseke alkuehdoilla $y(x_0)=c_0$,
$y'(x_0)=c_1\,,\ \ldots,\ y^{(n-1)}(x_0)=c_{n-1}$\,?

\item
Näytä Korollaarin \ref{toiseksi yksinkertaisin dy} avulla, että välillä $(-1,1)$ pätee:
\[
\Arctan x=\sum_{k=0}^\infty \frac{(-1)^{k}}{2k+1}\,x^{2k+1}
         = x-\frac{x^3}{3}+\frac{x^5}{5}- \ldots
\]

\item \label{H-V-6: yleistetty väliarvolause}
Todista Lause \ref{yleistetty väliarvolause} soveltamalla Rollen lausetta funktioon
\[
h(x) = [f(b)-f(a)][g(x)-g(a)]-[g(b)-g(a)][f(x)-f(a)].
\]
Miksei voi olla $g(a)=g(b)$\,?

\item
Yritä laskea l'Hospitalin säännöllä raja-arvo $\,\lim_{x\kohti\infty} x/\sqrt{x^2+1}$.

\item
Laske l'Hospitalin säänöillä:
\begin{align*}
&\text{a)}\ \lim_{x \kohti -3} \frac{x^2+3x}{x^2-9} \qquad
 \text{b)}\ \lim_{x \kohti 3^-} \frac{\abs{x^2-4x+3}}{2x^2-5x-3} \qquad
 \text{c)}\ \lim_{t \kohti 8} \frac{t^{2/3}-4}{t^{1/3}-2} \\
&\text{d)}\ \lim_{y \kohti 1} \frac{y-4\sqrt{y}+3}{y^2-1} \qquad
 \text{e)}\ \lim_{x\kohti\infty} \frac{2x^2+x+3}{1-x^2} \qquad
 \text{f)}\ \lim_{x \kohti 0} \frac{\sin 2x}{\sin 3x} \\
&\text{g)}\ \lim_{x \kohti 0} \frac{1-\cos 4x}{1-\cos 3x} \qquad
 \text{h)}\ \lim_{x \kohti 0} \frac{x-\sin x}{x^3} \qquad
 \text{i)}\ \lim_{x \kohti 0} \frac{2-x^2-2\cos x}{x^4} \\
&\text{j)}\ \lim_{x \kohti 0} \frac{x-\sin x}{x-\tan x} \qquad
 \text{k)}\ \lim_{x \kohti 0^+} \frac{\sin^2 x}{\abs{x-\tan x}} \qquad
 \text{l)}\ \lim_{t \kohti 0} \frac{3\sin t-\sin 3t}{3\tan t-\tan 3t} \\
&\text{m)}\ \lim_{x \kohti 1^-} \frac{\Arccos x}{\sqrt{1-x^2}} \qquad
 \text{n)}\ \lim_{x \kohti \infty} x(2\Arctan x-\pi)
\end{align*}

\item (*)
Olkoon $f$ ja $g$ Lipschitz-jatkuvia välillä $[a,b]$. Todista: \ a) $fg$ on Lipschitz
välillä $[a,b]$.\, b) Jos lisäksi $g(x) \neq 0\ \forall x\in[a,b]$, niin $f/g$ on Lipschitz
välillä $[a,b]$.

\item (*)
a) Funktio $y(x)$ on alkuarvotehtävän
$y'=\frac{1}{1+x^7}\,,\,\ x\in(-1,1),\,\ y(0)=1$
ratkaisu. Laske $y(\tfrac{1}{2})$ kahdeksan merkitsevän numeron tarkkuudella.
\vspace{1mm}\newline
b) Määritä differentiaaliyhtälön $y'=\frac{1}{\sqrt{x}\,(1+x)}$
yleinen ratkaisu sarjamuotoisena välillä $(0,1)$.

\item (*) \label{H-V-6: Hospital 3}
Halutaan todistaa, että laskusääntö 
\[
\lim_{x \kohti a^+} \frac{f'(x)}{g'(x)} = A\in\R \qimpl
\lim_{x \kohti a^+} \frac{f(x)}{g(x)} = A
\]
(Lause \ref{Hospital}: 1.\ sääntö, kun $A\in\R$) on pätevä myös, kun funktioiden $f$ ja $g$
raja-arvoja koskeva oletus muutetaan muotoon: $\,\lim_{x \kohti a^+}\abs{f(x)}=\infty$ ja
$\lim_{x \kohti a^+}\abs{g(x)}=\infty$. Tarkista ja täydennä todistukseksi päättely: \newline
Jos $a<x<t<a+\delta$, niin jollakin $\xi\in(x,t)$ pätee
\begin{align*}
\frac{f(x)-f(t)}{g(x)-g(t)}\,
  &=\, \frac{f'(\xi)}{g'(\xi)} \\[2mm]
\qimpl\ \ \frac{f(x)}{g(x)}-A\,
  &=\, \frac{f'(\xi)}{g'(\xi)}-A
     + \frac{1}{g(x)}\left(f(t)-g(t)\,\frac{f'(\xi)}{g'(\xi)}\right) \\[2mm]
\qimpl \left|\frac{f(x)}{g(x)}-A\right|\,
  &\le\, \left|\frac{f'(\xi)}{g'(\xi)}-A\right|
       + \frac{1}{|g(x)|}\left(|f(t)|+|g(t)|\left|\frac{f'(\xi)}{g'(\xi)}\right|\right).
\end{align*}
Jos nyt $\eps>0$, niin viimeksi kirjoitetussa epäyhtälössä voidaan valita ensin $t$ ja sitten
$x$ niin, että epäyhtälön oikea puoli on pienempi kuin $\eps$.

\end{enumerate} % Differentiaalilaskun väliarvolause
\section{Kiintopisteiteraatio. Newtonin menetelmä} \label{kiintopisteiteraatio}
\sectionmark{Kiintopisteiteraatio}
\alku
\index{kiintopisteiteraatio|vahv}
\index{suppeneminen!b@kiintopisteiteraation|vahv}

Aiemmin Luvuissa \ref{jono}--\ref{monotoniset jonot} on tarkasteltu esimerkkejä palautuvista
lukujonoista, jotka määräytyvät alkuarvosta $x_0\in\R$ ja (johonkin reaalifunktioon $f$
liittyen) palautuskaavasta
\begin{equation} \label{kp-iteraatio}
x_{n+1}=f(x_n),\quad n=0,1,\ldots \tag{$\star$}
\end{equation}
Jatkossa tarkastellaan tällaisia lukujonoja ja niiden suppenemisen ehtoja eiempaa yleisemmältä
kannalta.

Oletetaan aluksi, että $f$ on jatkuva suljetulla välillä $[a,b]$ ja että 
$x_n\in [a,b]\ \forall n$. Tällöin jos jono $\{x_n\}$ on suppeneva, niin on oltava 
\[
x_n\kohti c\in [a,b].
\]
Koska $f$ on välillä $[a,b]$ jatkuva, niin
\[
x_n\kohti c \ \impl \ f(x_n)\kohti f(c),
\]
joten kaavan \eqref{kp-iteraatio} mukaan
\[
c=f(c).
\]
Sanotaan, että $c$ on $f$:n
\index{kiintopiste}%
\kor{kiintopiste} (engl. fixed point) ja lukujen $x_n$ laskemista 
palautuskaavasta \eqref{kp-iteraatio} sanotaan tämän vuoksi \kor{kiintopisteiteraatioksi} 
(lat.\ itero = toistaa, tehdä uudelleen).

Jos halutaan löytää annetun funktion $f$ kiintopiste, niin luonnollinen algoritmi (joskaan ei 
aina toimiva, ks.\ tarkastelut jäljempänä) on kiintopisteiteraatio \eqref{kp-iteraatio}
jostakin alkuarvauksesta $x_0$.
\begin{multicols}{2} \raggedcolumns
\begin{Exa} \label{kp-esim 1} Ratkaise (transkendenttinen) yhtälö $x=\cos x$ 
kiintopisteiteraatiolla. 
\end{Exa}
\ratk Valitaan alkuarvaukseksi $x_0=0$, jolloin saadaan iteraatio 
\[
x_0=0,\quad x_{n+1}=\cos x_n,\quad n=0,1,\ldots
\]
Tämä suppenee hitaahkosti kohti funktion $f(x)=\cos x$ (ainoaa) kiintopistettä 
$c = 0.7390851332..$
\begin{figure}[H]
\setlength{\unitlength}{1cm}
\begin{center}
\begin{picture}(5,4)(-1,-1)
\put(-1,0){\vector(1,0){5}} \put(3.8,-0.4){$x$}
\put(0,-1){\vector(0,1){4}} \put(0.2,2.8){$y$}
\put(-0.5,-0.67){\line(3,4){2.2}} \put(1.4,2.4){$y=x$}
\curve(
   0,       2,
0.75,  1.7552,
 1.5,  1.0806,
2.25, 0.14147,
   3,-0.83229)
\put(2.25,0.5){$y=\cos x$}
\dashline{0.1}(1.1,0)(1.1,1.47) \put(1.05,-0.4){$\scriptstyle{c}$}
\put(1.5,0){\line(0,-1){0.15}} \put(1.45,-0.4){$\scriptstyle{1}$}
\put(0,2){\line(-1,0){0.15}} \put(-0.4,1.9){$\scriptstyle{1}$} 
\put(2.35,0){\line(0,-1){0.15}} \put(2.22,-0.4){$\scriptstyle{\frac{\pi}{2}}$}
\end{picture}
\end{center}
\end{figure}
\end{multicols}
\[
\begin{array}{ll}
x_1=1 \quad        &x_{10}=0.73140404.. \\
x_2=0.5403.. \quad &x_{20}=0.73893775.. \\
x_3=0.8575.. \quad &x_{30}=0.73908229.. \\
%x_4=0.6542.. \quad &x_{40}=0.73908507.. \\
%x_5=0.7934.. \quad &x_{50}=0.73908513.. \\
\ \vdots                 &\ \vdots \qquad \loppu
\end{array}
\]
Esimerkki herättää kysymyksen, millaisilla ehdoilla kiintopisteiteraatio \eqref{kp-iteraatio} 
yleensä suppenee, ja kuinka nopeasti, jos $x_0 \neq c$. (Tapaus $x_0=c$ on triviaali, koska 
tällöin $x_n=c\ \forall n$.) Suppenemistarkasteluille suunnan antakoon
\begin{Exa} Olkoon $f$ ensimmäisen asteen polynomi. Tällöin jos $f(c)=c$, eli $c$ on
kiintopiste, niin jollakin $k \in \R$ on $\,f(x) = c + k(x-c)$, jolloin iteraatiokaavan
\eqref{kp-iteraatio} mukaan on
\[
x_{n+1} - c = k(x_n - c), \quad n = 0,1,\ldots \qimpl x_n-c = k^n(x_0-c), \quad n=1,2,\ldots
\]
Siis $x_n \kohti c$ alkuarvosta $x_0$ riippumatta, jos $\abs{k}<1$. Jos $\abs{k} \ge 1$, niin 
$x_n \kohti c$ vain kun $x_0=c$. \loppu
\end{Exa}
Esimerkin mukaan $f$:n säännöllisyys (edes sileys) ei ole sovelias kriteeri
kiintopisteiteraation suppenemiselle, vaan tarvitaan toisen tyyppisiä ehtoja. Seuraavissa
kahdessa lauseessa asetetaan, esimerkin tulosta mukaillen, riittävät ehdot sekä
kiintopisteiteraation suppenemiselle kohti haluttua kiintopistettä $c$ että iteraation
epäonnistumiselle ($ x_n \not\kohti c$).
\begin{Lause} \label{kp-lause 1} Jos (i) funktiolla $f$ on kiintopiste $c$, ja (ii) $f$ on 
määritelty välillä $[c-a,c+a],\ a>0\,$  ja toteuttaa ehdon
\[
\abs{f(x)-f(c)} \le L\abs{x-c\,}, \quad x \in [c-a,c+a],
\]
missä $0 \le L<1$\footnote[2]{Ilman lisäehtoa $L<1$ Lauseen \ref{kp-lause 1} ehtoa (ii)
sanotaan \kor{Lipschitz-ehdoksi pisteessä} $c$, vrt.\ Lipschitz-jatkuvuuden ehto suljetulla
välillä: Määritelmä \ref{funktion l-jatkuvuus}. \index{Lipschitz-ehto (pisteessä)|av}}, niin
\begin{itemize}
\item[(1)] $c$ on $f$:n ainoa kiintopiste välillä $[c-a,c+a]$,
\item[(2)] iteraatio \eqref{kp-iteraatio} suppenee kohti kiintopistettä jokaisella 
$x_0\in [c-a,c+a]$ ja pätee
\[ 
\abs{x_n-c} \le L^n\abs{x_0-c\,}, \quad n=1,2,\ldots 
\]
\end{itemize}
\end{Lause}
\tod (1) \ Jos $c_1\in [c-a,c+a]$ on myös kiintopiste, niin
\begin{align*}
c=f(c)\ \ja\ c_1=f(c_1) &\qimpl\ \abs{c-c_1}=\abs{f(c)-f(c_1)} \,\le\, L\abs{c-c_1} \\
                        &\qimpl\  (1-L)\,\abs{c-c_1} \le 0.
\end{align*}
Koska on $0 \le L < 1$, niin on oltava $c-c_1=0$.

(2) \ Oletuksien perusteella pätee ensinnäkin
\[
\abs{f(x)-c\,} = \abs{f(x)-f(c)} \,\le\, L\abs{x-c\,} \le \abs{x-c} \le a, 
                                          \quad \text{kun}\ x\in[c-a,c+a]. 
\]
Näin ollen jos $x_0 \in [c-a,c+a]$, niin  iteraatiolle \eqref{kp-iteraatio} pätee 
$\,x_n \in [c-a,c+a]\ \forall n$, joten oletuksien perusteella voidaan arvioida
\[
\abs{x_n-c\,} = \abs{f(x_{n-1})-f(c)} \,\le\, L\abs{x_{n-1}-c\,} \,\le\, \ldots\ 
                                      \,\le\, L^n\abs{x_0-c\,}.
\]
Koska $L<1$, niin $x_n \kohti c$. \loppu
\begin{Lause} \label{kp-lause 2} Jos (i) funktiolla $f$ on kiintopiste $c$, ja (ii) $f$ on 
määritelty välillä $[c-a,c+a],\ a>0\,$ ja toteuttaa ehdon
\[
\abs{f(x)-f(c)} \ge L\abs{x-c\,}, \quad x \in [c-a,c+a],
\]
missä $L>1$, niin kiintopisteiteraatio \eqref{kp-iteraatio} suppenee kohti kiintopistettä $c$ 
ainoastaan siinä tapauksessa, että $x_n=c$ jollakin $n$. 
\end{Lause}
\tod Jos $x_k=c$, niin iteraatiokaavan \eqref{kp-iteraatio} mukaan on $x_n=c\ \forall n \ge k$,
jolloin $x_n \kohti c$. Oletetaan siis, että $x_n \neq c\ \forall n$, jolloin väittämä on, että
$x_n \not\kohti c$. Jos $\abs{x_n-c\,} > a\ \forall n$, niin tämä on tosi. Voidaan siis olettaa,
että $x_n \in [c-a,c+a]$ jollakin $n$. Tällöin oletuksien mukaan
\[
\abs{x_{n+1}-c\,} = \abs{f(x_n)-f(c)} \ge L\abs{x_n-c\,}.
\]
Jos $x_{n+1} \in [c-a,c+a]$, voidaan edelleen arvioida $\abs{x_{n+2}-c\,} \ge L^2\abs{x_n-c\,}$,
jne. Koska oletettiin, että $x_n \neq c$ ja $L>1$, niin päätellään, että jollakin 
$m \in \N,\ m>n$ pätee
\[
\abs{x_m-c\,} \ge L^{m-n}\abs{x_n-c\,} > a.
\]
Edellä on päätelty, että jos $x_n \neq c\ \forall n$, niin mistä tahansa indeksistä $N$ 
eteenpäin on aina löydettävissä jokin indeksi $n>N$ siten, että $\abs{x_n-c\,} > a$. Tällöin 
$x_n \not\kohti c$. \loppu

Jos $f$ on derivoituva kiintopisteen ympäristössä tai ainakin kiintopisteessä (niinkuin usein),
niin Lauseissa \ref{kp-lause 1} ja \ref{kp-lause 2} asetettujen ehtojen pätevyyttä voidaan 
tutkia helposti derivaatan avulla. Ensinnäkin jos oletetaan, että $f$ on jatkuvasti derivoituva
välillä $[c-a,c+a]$, niin Differentiaalilaskun väliarvolauseen mukaan on 
$f(x)-f(c) = f'(\xi)(x-c)$ jollakin $\xi \in (c-a,c+a)$, kun $x \in [c-a,c+a]$. Näin ollen
voidaan päätellä:
\begin{align*}
&\max_{x \in [c-a,c+a]}\,\abs{f'(x)} = L<1 
                      \qimpl \text{Lauseen \ref{kp-lause 1} ehdot voimassa}. \\
&\min_{x \in [c-a,c+a]}\,\abs{f'(x)} = L>1 
                      \qimpl \text{Lauseen \ref{kp-lause 2} ehdot voimassa}.
\end{align*}
\jatko\jatko \begin{Exa} (jatko) Esimerkissä on $\abs{f'(x)} = \abs{\sin x} \le L<1$ esim.\
välillä $[c-0.5,c+0.5]$. Esimerkin iteraatiolle ovat näin ollen voimassa Lauseen
\ref{kp-lause 1} ehdot (kun $a=0.5$) indeksistä $n=1$ alkaen. \loppu
\end{Exa} \seur
Jos Lauseiden \ref{kp-lause 1} ja \ref{kp-lause 2} ehdot asetetaan muodossa 'jollakin $a>0$', 
ts.\ väliä $[c-a,c+a]$ ei kiinnitetä etukäteen, niin ehtojen toteutumiselle saadaan seuraava 
yksinkertainen kriteeri:
\begin{Prop} \label{kp-prop} Jos $f$ on derivoituva kiintopisteessä $c$, niin pätee:
\begin{align*}
\abs{f'(c)} < 1 \qekv \text{Lauseen \ref{kp-lause 1} ehdot voimassa jollakin $a>0$}. \\[1mm]
\abs{f'(c)} > 1 \qekv \text{Lauseen \ref{kp-lause 2} ehdot voimassa jollakin $a>0$}.
\end{align*}
\end{Prop}
\tod Jos $\abs{f'(c)}<1$, niin derivaatan määritelmän ja raja-arvon $(\eps,\delta)$-kriteerin 
(Lause \ref{approksimaatiolause}) mukaan jokaisella $\eps>0$ on olemassa $\delta>0$ siten,
että pätee
\[
\left|\frac{f(x)-f(c)}{x-c} - f'(c)\right| < \eps, \quad 
                        \text{kun}\ \abs{x-c} < \delta\ \ja\ x \neq c.
\]
Kun tässä valitaan $\eps = (1-\abs{f'(c)})/2>0$ ja käytetään kolmioepäyhtälöä, niin nähdään, 
että Lauseen \ref{kp-lause 1} oletukset ovat voimassa jokaisella $a \in (0,\delta)$ 
(esim.\ $a=\delta/2$), kun valitaan $L = (1+\abs{f'(c)})/2<1$. Tämä todistaa ensimmäisen 
väittämän osan \fbox{$\impl$}\,. Osa \fbox{$\Leftarrow$} seuraa, kun Lauseen \ref{kp-lause 1}
oletuksen (ii) perusteella päätellään, että on oltava $|f'(c)| \le L < 1$ (Määritelmät
\ref{derivaatan määritelmä} ja \ref{funktion raja-arvon määritelmä} sekä Lause
\ref{jonotuloksia} [V1]). Toinen väittämä todistetaan vastaavasti. \loppu

Proposition \ref{kp-prop} ja Lauseen \ref{kp-lause 1} mukaisesti kiintopisteiteraatio 
\eqref{kp-iteraatio} suppenee kohti kiintopistettä $c$, jos $\abs{f'(c)}<1$ ja lisäksi $x_0$ on
\pain{riittävän} \pain{lähellä} kiintopistettä. Jos taas $\abs{f'(c)}>1$, niin $x_n \kohti c$ on
Lauseen \ref{kp-lause 2} mukaisesti tosi vain siinä (melko onnekkaassa) tapauksessa, että 
\index{attraktiivinen (kiintopiste)}%
$x_n=c$ jollakin $n$. Näiden tulosten perusteella kiintopistettä sanotaan \kor{attraktiiviseksi}
\index{hylkivä (kiintopiste)} \index{repulsiivinen (kiintopiste)}%
(eli puoleensa vetäväksi), jos $\abs{f'(c)}<1$ ja \kor{hylkiväksi} eli \kor{repulsiiviseksi},
jos $\abs{f'(c)}>1$. Luokittelun ulkopuolelle (tapauskohtaisesti tutkittaviksi) jäävät siis 
ainoastaan sellaiset kiintopisteet, joissa $f'(c)=\pm 1$.
\begin{Exa}
Funktion $f(x)=\sqrt{x+1}$ ainoa kiintopiste $c$ on
\[
c\geq 0 \ \ja \ c=\sqrt{c+1} \ \ekv \ c=\frac{1}{2}(\sqrt{5}+1).
\]
Koska
\[
f'(x) = \frac{1}{2\sqrt{1+x}}\ \impl\ 0 < f'(c) < \frac{1}{2}\,,
\]
niin kyseessä on attraktiivinen kiintopiste. Tarkempi tutkimus paljastaa, että 
kiintopisteiteraatio $x_{n+1}=f(x_n)$ suppenee jokaisella $x_0 \in D_f = [-1,\infty)$. 
\loppu \end{Exa}
\begin{Exa} Funktiolla $f(x)=1-x^2$ on kaksi kiintopistettä:
\[
c=1-c^2 \ \ekv \ c=\frac{1}{2}(-1\pm\sqrt{5}).
\]
Koska $f'(c)=-2c=1\pm\sqrt{5}$, niin nähdään, että molemmat kiintopisteet ovat hylkiviä. Näin
ollen päätellään (Lause \ref{kp-lause 2}), että kiintopisteiteraatio
\[
x_0 \in \R, \quad x_{n+1} = 1-x_n^2, \quad n=0,1,\ldots
\]
voi olla suppeneva vain jos $x_n=\frac{1}{2}(1\pm\sqrt{5})$ jollakin $n$. Tämä mahdollisuus on
pois suljettu esim.\ jos $x_0\in\Q$, koska tällöin $\seq{x_n}$ on rationaalilukujono. \loppu
\end{Exa}

\subsection{Asymptoottinen suppenemisnopeus}

Tarkastellaan kiintopisteiteraatiota \eqref{kp-iteraatio} olettaen, että (i) $f$ on derivoituva
kiintopisteessä $c$ ja $\abs{f'(c)}<1$, (ii) $x_n \kohti c$, ja (3) $x_n \neq c\ \forall n$. 
Tällöin iteraatiokaavasta \eqref{kp-iteraatio} ja derivaatan määritelmästä seuraa
\[
\lim_{n\kohti\infty}\frac{x_{n+1}-c}{x_n-c} 
                        = \lim_{n\kohti\infty}\frac{f(x_n)-f(c)}{x_n-c} = f'(c).
\]
Tällä perusteella voidaan sanoa, että luku $q=f'(c)$ määrää kiintopisteiteraation 
\index{asymptoottinen suppenemisnopeus}%
\kor{asymptoottisen suppenemisnopeuden}: Suurilla $n$:n arvoilla on likimain
\[
 x_n -c\ \sim\ \text{vakio} \times q^n \quad \text{($n$ suuri)}.
\]
Tästä nähdään myös, että jos $q>0$, niin jono $\seq{x_n}$ on asymptoottisesti (eli suurilla $n$)
monotoninen. Jos $q<0$, niin jono on asymptoottisesti 'hyppelehtivä', vrt.\ kuvio.
\begin{figure}[H]
\setlength{\unitlength}{1cm}
\begin{center}
\begin{picture}(11,6)(0,-2)
\multiput(0,0)(6,0){2}
{
\put(0,0){\vector(1,0){4}} \put(3.8,-0.4){$x$}
\put(0,0){\vector(0,1){4}} \put(0.2,3.8){$y$}
}
\curve(0.5,1.5,2,2.5,4,2.8) \put(0.2,2.6){$y=f(x)$}
\curve(6.5,2.5,7.8,1.5,10,0.6) \put(6.2,2.8){$y=f(x)$}
\put(0,0){\line(1,1){3.5}} \put(6,0){\line(1,1){3.5}} \put(3,3.6){$y=x$} \put(9,3.6){$y=x$}
\put(1,-1.6){$0<q<1$} \put(7,-1.6){$-1<q<0$}
\path(0.8,0)(0.8,1.8)(1.8,1.8)(1.8,2.4)(2.4,2.4)(2.4,2.6)(2.6,2.6)
\dashline{0.1}(1.8,0)(1.8,1.8)
\dashline{0.1}(2.4,0)(2.4,2.4)
\dashline{0.2}(2.7,0)(2.7,2.7)
\put(0.6,-0.6){$x_0$} \put(1.6,-0.6){$x_1$} \put(2.2,-0.6){$x_2$} \put(2.62,-0.3){$c$}
\path(9.4,0)(9.4,0.8)(6.8,0.8)(6.8,2.2)(8.2,2.2)(8.2,1.3)(7.3,1.3)(7.3,1.8)(7.8,1.8)(7.8,1.5)
(7.5,1.5)(7.5,1.7)
\dashline{0.1}(6.8,0)(6.8,0.8)
\dashline{0.1}(8.2,0)(8.2,2.2)
\dashline{0.1}(7.3,0)(7.3,1.3)
\dashline{0.1}(7.8,0)(7.8,1.8)
%\dashline{0.1}(7.5,0)(7.5,1.5)
\dashline{0.2}(7.6,0)(7.6,1.6)
\put(9.2,-0.6){$x_0$} \put(6.6,-0.6){$x_1$} \put(8,-0.6){$x_2$} \put(7.52,-0.3){$c$}
\end{picture}
%\caption{Kiintopisteiteraation geometria}
\end{center}
\end{figure}
\begin{Exa} Esimerkissä \ref{kp-esim 1} oli $\,f'(c)=-\sin c \approx -0.67$. Suppenemisen 
verkkaisuus ja 'hyppelehtivyys' sai näin selityksensä. \loppu
\end{Exa} 
Kiintopisteiteraatio $x_{n+1}=f(x_n)$ suppenee asymptoottisesti erityisen nopeasti silloin, kun 
kiintopisteessä on $f'(c)=0$. Jotta myös tämä tapaus tulisi tarkemmin tutkituksi, oletettakoon
yleisemmin, että jollakin $m\in\N$ ja $A\in\R,\ A \neq 0$ on voimassa
\[
\lim_{x \kohti c}\,\frac{f(x)-f(c)}{(x-c)^m} \,=\, A.
\]
Jos oletetaan samoin kuin edellä, että kiintopisteiteraatiolle pätee $x_n \kohti c$ ja 
$x_n \neq c\ \forall n$, niin oletuksen perusteella pätee
\[
\lim_{n\kohti\infty} \frac{x_{n+1}-c}{(x_n-c)^m}\ 
                =\ \lim_{n\kohti\infty} \frac{f(x_n)-f(c)}{(x_n-c)^m}\ = A,
\]
jolloin suurilla $n$:n arvoilla on likimain
\[
x_{n+1}-c\ \approx A(x_n-c)^m \quad \text{($n$ suuri)}.
\]
Tämän perusteella sanotaan, että $m$ on suppenemisnopeuden (asymptoottinen) 
\index{kertaluku!aa@suppenemisnopeuden}
\kor{kertaluku}. --- Huomattakoon, että jos $f'(c) \neq 0$, niin ym.\ oletus on
(derivaatan määritelmän nojalla, vrt.\ Luku \ref{derivaatta}) voimassa kun $m=1$ ja $A=f'(c)$.
Suppenemista tässä tapauksessa (kun $\abs{A}<1$) sanotaankin
\index{lineaarinen suppeneminen} \index{superlineaarinen (suppeneminen)}%
\kor{lineaariseksi} (kertaluku $=1$), ja muissa tapauksissa \kor{superlineaariseksi}.
Superlineaarisista tavallisin on tapaus $m=2$, jolloin sanotaan, että kiintopisteiteraatio
suppenee \kor{kvadraattisesti}. \index{kvadraattinen!a@suppeneminen}
\begin{Exa} \label{neliöjuuri a} Luvun \ref{monotoniset jonot} Esimerkissä 
\ref{sqrt 2 algoritmina} tarkasteltiin tapauksessa $a=2$ palautuvaa lukujonoa
\[
x_0=a, \quad x_{n+1} = \frac{1}{2}\left(x_n + \frac{a}{x_n}\right), \quad n=0,1,\ldots
\]
Olkoon nyt yleisemmin $a>0,x_0>0$ ja tulkitaan lukujono kiintopisteiteraatioksi funktiolle 
$f(x)=\frac{1}{2}(x+\frac{a}{x})$. Kiintopisteitä on kaksi: 
\[
c = \frac{1}{2}\left(c + \frac{a}{c}\right)\ \ekv\ c^2=a\ \ekv\ c=\pm\sqrt{a}.
\]
Näistä vain $c=\sqrt{a}$ on mahdollinen jonon $\seq{x_n}$ raja-arvo, kun $x_0>0$, koska tällöin
on $x_n>0\ \forall n$. Tällä $c$:n arvolla nähdään, että
\begin{align*}
f(x)-f(c) = \frac{1}{2}\left(x + \frac{a}{x}\right) - \sqrt{a} 
                       &= \frac{1}{2x}\,(x^2-2\sqrt{a}\,x+a) \\
                       &= \frac{1}{2x}\,(x-c)^2.
\end{align*}
Tämän perusteella $(x-c)^{-2}[f(x)-f(c)] \kohti 1/(2c)$, kun $x \kohti c=\sqrt{a}$.
Siis jos lukujono $\seq{x_n}$ suppenee, niin se suppenee kvadraattisesti. --- Tarkempi
tutkimus osoittaa, että $x_n\kohti\sqrt{a}$ aina kun $x_0>0$. \loppu
\end{Exa}

\subsection{Newtonin menetelmä}
\index{Newtonin menetelmä|vahv}

Jos funktio on derivoituva pisteessä $c$, niin sitä voidaan approksimoida pisteen $c$ 
ympäristössä perustuen linearisaatioon (vrt. Luku \ref{derivaatta})
\[
f(x)\approx f(c)+f'(c)(x-c).
\]
Tähän linearisointiajatukseen perustuu epälineaaristen yhtälöiden ratkaisussa hyvin yleisesti 
käytetty ja tehokas menetelmä, \kor{Newtonin menetelmä}. Newtonin menetelmässä etsitään 
yhtälölle
\[
f(x)=0
\]
ratkaisua pisteen $x_0$ (alkuarvaus) lähistöltä. Algoritmissa $f$ linearisoidaan pisteessä $x_n$
(aluksi $n=0$) ja ratkaistaan linearisoitu yhtälö
\[
f(x_n)+f'(x_n)(x-x_n)=0.
\]
Tämä ratkeaa, jos $f'(x_n)\neq 0$. Kun ratkaisua merkitään $x=x_{n+1}$, saadaan algoritmiksi
\begin{equation} \label{N-iteraatio}
\boxed{\quad x_{n+1}=x_n-\frac{f(x_n)}{f'(x_n)},\quad n=0,1,2,\ldots \quad} \tag{$\star\star$}
\end{equation}
Laskimien ja tietokoneohjelmien komentojen 'Solve' tai 'FindRoot' takana on yleensä joko tämä
menetelmä tai jokin sen variaatio, kuten \kor{sekanttimenetelmä}, ks.\ kommentit edempänä.
\begin{Exa} \label{neliöjuuri a - Newton} Jos $f(x)=x^2-a$, $a>0$, niin Newtonin algoritmi
$f$:n nollakohdan $c=\sqrt{a}$ määrämiseksi on
\[
x_0>0, \quad x_{n+1}=x_n-\frac{x_n^2-a}{2x_n}
                    =\frac{1}{2}\left(x_n+\frac{a}{x_n}\right),\quad n=0,1,2,\ldots
\]
Esimerkissä \ref{neliöjuuri a} oli siis kyse Newtonin menetelmästä. \loppu
\end{Exa}
Iteraatiokaavan \eqref{N-iteraatio} mukaisesti Newtonin algoritmi on kiintepistoiteraatio
sovellettuna funktioon
\[
F(x)=x-\frac{f(x)}{f'(x)}.
\]
Algoritmin suosio perustuu siihen, että sikäli kuin iteraatio suppenee, suppeneminen on melko
yleisin edellytyksin kvadraattista. Edellytys kvadraattiselle suppenemiselle on, että $f$
\index{yksinkertainen!a@nollakohta (juuri)}%
on nollakohdan $c$ lähellä riittävän säännöllinen ja että nollakohta on \kor{yksinkertainen},
ts.\ $f'(c) \neq 0$. Seuraavan täsmällisen suppenemislauseen todistus perusuu
differentiaalilaskennan väittämään, jota ei vielä ole käytettävissä. Sen vuoksi todistuksessa
rajoitutaan toistaiseksi erikoistapaukseen, jossa $f$ on polynomi. (Yleisempi todistus,
ks.\ Harj.teht.\,\ref{taylorin lause}:\ref{H-dif-4: Newtonin konvergenssi}.)
\begin{Lause} \label{Newtonin konvergenssi} Jos $f$ on kahdesti jatkuvasti derivoituva välillä 
$[c-a,c+a]$ jollakin $a>0$ ja $f(c)=0$ ja $f'(c)\neq 0$, niin Newtonin iteraatio 
\eqref{N-iteraatio} suppenee $c$:tä kohti, kun $x_0$ on $c$:tä riittävän lähellä, ja pätee
\[
\lim_{n \kohti \infty} \frac{x_{n+1}-c}{(x_n-c)^2} = \frac{f''(c)}{2f'(c)}\,.
\]
\end{Lause}
\underline{Todistus}, kun $f$ on polynomi: Koska $f(c)=0$, niin $f(x)=(x-c)g(x)$, missä
$g$ on polynomi, ja samalla perusteella $g(x)-g(c)=(x-c)h(x)$ ja $f'(x)-f'(c)=(x-c)r(x)$,
missä $h$ ja $r$ ovat polynomeja (Lause \ref{algebran pl}). Derivoimalla nähdään, että
$f'(c)=g(c)$, joten saadaan hajotelmat
\begin{align*}
f(x)  \,&=\, (x-c)[g(c)+(x-c)h(x)] \,=\, f'(c)(x-c)+(x-c)^2h(x), \\
f'(x) \,&=\, f'(c)+(x-c)r(x).
\end{align*}
Sijoittamalla nämä $F$:n lausekkeeseen ja huomioimalla, että $F(c)=c$, seuraa
\begin{align*}
F(x)-F(c)\ &=\ x-c - \frac{f(x)}{f'(x)} \\
           &=\ x-c - \frac{f'(c)(x-c)+(x-c)^2h(x)}{f'(c)+(x-c)r(x)} \\
           &=\ \frac{(x-c)^2[r(x)-h(x)]}{f'(c)+(x-c)r(x)}\,.
\end{align*}
Tämän perusteella
\[
\lim_{x \kohti c}\,\frac{F(x)-F(c)}{(x-c)^2} \,=\, \frac{r(c)-h(c)}{f'(c)}\,.
\]
Derivoimalla em.\ $f$:n ja $f'$:n hajotelmia nähdään edelleen, että $h(c)=\tfrac{1}{2}f''(c)$
ja $r(c)=f''(c)$, joten väite seuraa. \loppu

Lauseen \ref{Newtonin konvergenssi} perusteella Newtonin iteraation konvergenssi on lauseen
oletuksin kvadraattista, tai jopa 'superkvadraattista' (jos $f''(c)=0$). Jos $f$ täyttää
Lauseen \ref{Newtonin konvergenssi} säännöllisyysehdon mutta nollakohta $c$ on
\index{kaksinkertainen nollakohta}%
\kor{kaksinkertainen}, ts.\
\[
f(c)=f'(c)=0, \ f''(c)\neq 0,
\]
niin Newtonin algoritmi suppenee tässäkin tapauksessa (riittävän läheltä $c$:tä), mutta
suppeneminen hidastuu lineaariseksi. Tarkemmin pätee tässä tapauksessa:
$\lim_n (x_{n+1}-c)/(x_n-c) = 1/2$ 
(Harj.teht.\,\ref{H-V-7: moninkertainen nollakohta ja Newton}; vrt.\ myös Esimerkki
\ref{neliöjuuri a}, kun $a=0$). 
\begin{Exa} Ratkaise Esimerkin \ref{kp-esim 1} yhtälö $x=\cos x$ Newtonin menetelmällä.
\end{Exa}
\ratk Kun valitaan $\,f(x)=x-\cos x$, niin Lauseen \ref{Newtonin konvergenssi} ehdot ovat
voimassa ja $f''(c)=\cos c \neq 0$, joten Newtonin iteraatio suppenee (sikäli kuin suppenee)
kvadraattisesti. Iteraatiokaava on
\[
x_{n+1}\ =\ x_n - \frac{x_n-\cos x_n}{1+\sin x_n}\
         =\ \frac{x_n\sin x_n+\cos x_n}{1+\sin x_n},\quad n=0,1,2,\ldots
\]
Alkuarvauksella $x_0=0$ on tulos (vrt.\ Esimerkki \ref{kp-esim 1})
\begin{align*}
x_0    &= 0 \\
x_1    &= 1 \\
x_2    &= 0.7503638678.. \\
x_3    &= 0.7391128909.. \\
x_4    &= 0.7390851333.. \\
x_5    &= 0.7390851332.. \\
\vdots & \loppu
\end{align*}

Jos derivaatta $f'$ on nopeasti muuttuva $f$:n nollakohdan lähellä, voi Newtonin menetelmä
olla hyvin herkkä alkuarvaukselle, eikä iteratio välttämättä suppene lainkaan. Laskentaohjelma
antaa silloin tuloksen 'failed to converge'. Tällöin yleensä yksinkertaisesti vaihdellaan
alkuarvoa $x_0$, kunnes onni kääntyy. Toinen mahdollisuus on käyttää jotakin varmempaa
menetelmää hyvän alkuarvauksen hakuun, jolloin Newtonin iteraation tehtäväksi jää 
'loppukiihdytys'. Esimerkiksi Bolzanon lauseen (Lause \ref{Bolzanon lause}) todistuksessa
käytetty puolitus\-konstruktio on aloitusmenetelmänä oivallinen --- hidas mutta varma.
\begin{Exa}
Jos funktion $f(x)=x/(1+x^2)$ nollakohtaa haetaan Newtonin menetelmällä, tulee
iteraatiokaavaksi
\[
x_{n+1}=F(x_n)=-\frac{2x_n^3}{1-x_n^2},\quad n=0,1,2,\ldots
\]
Suppenemisalueen rajalle joudutaan, jos valitaan $x_0=a\neq 0$ siten, että $x_1=-a$, jolloin 
iteraatiokaavan mukaan on $x_n=(-1)^na$. Näin käy siis kun
\[
a=\frac{2a^3}{1-a^2} \ \ja \ a\neq 0 \ \ekv \ a=\pm \frac{1}{\sqrt{3}}\,.
\]
Jos $\abs{x_0}<1/\sqrt{3}$, niin iteraatio suppenee: $x_n\kohti 0$. (Suppenemisnopeuden
kertaluku on $m=3$, sillä $\,\lim_{x \kohti 0}\,x^{-3}F(x)=-2$.) Jos $\abs{x_0}\geq 1/\sqrt{3}$,
on tulos 'failed to converge'. Geometrisestikin nähdään, että jos $\abs{x_0}>1/\sqrt{3}$, niin
itse asiassa $\abs{x_n}\kohti\infty$ (kuvassa $x_n\kohti -\infty$).
\begin{figure}[H]
\setlength{\unitlength}{1cm}
\begin{center}
\begin{picture}(12,4)(-6,-2)
\put(-6,0){\vector(1,0){12}} \put(5.8,-0.4){$x$}
\put(0,-1.5){\vector(0,1){3.5}} \put(0.2,1.8){$y$}
\curve(
   -6.0000,   -0.6000,
   -5.5000,   -0.6423,
   -5.0000,   -0.6897,
   -4.5000,   -0.7423,
   -4.0000,   -0.8000,
   -3.5000,   -0.8615,
   -3.0000,   -0.9231,
   -2.5000,   -0.9756,
   -2.0000,   -1.0000,
   -1.5000,   -0.9600,
   -1.0000,   -0.8000,
   -0.5000,   -0.4706,
         0,         0,
    0.5000,    0.4706,
    1.0000,    0.8000,
    1.5000,    0.9600,
    2.0000,    1.0000,
    2.5000,    0.9756,
    3.0000,    0.9231,
    3.5000,    0.8615,
    4.0000,    0.8000,
    4.5000,    0.7423,
    5.0000,    0.6897,
    5.5000,    0.6423,
    6.0000,    0.6000)
\multiput(-2,0)(4,0){2}{\drawline(0,0)(0,-0.1)}
\multiput(-1.15,0)(2.3,0){2}{\linethickness{0.6mm} \line(0,-1){0.15}}
\drawline(-1.24,-0.89)(1.55,0)
\drawline(1.55,0.97)(-4.64,0)
\dashline{0.1}(-1.24,0)(-1.24,-0.89)
\dashline{0.1}(1.55,0)(1.55,0.97)
\dashline{0.1}(-4.64,0)(-4.64,-0.72)
\put(-2.4,-0.6){$-1$} \put(1.93,-0.6){$1$}
\put(-1.35,0.15){$x_0$} \put(1.4,-0.4){$x_1$} \put(-4.74,0.15){$x_2$}
\put(-1.05,-1.6){$\underbrace{\hspace{2.3cm}}_{(-\frac{1}{\sqrt{3}},\frac{1}{\sqrt{3}})}$}
\put(1.1,-2.25){= suppenemisväli}
\end{picture}
%\caption{Newtonin menetelmä funktiolle $f(x)=x/(1+x^2)$}
\end{center}
\end{figure}
\end{Exa}

\subsection{Sekanttimenetelmä}
\index{sekanttimenetelmä|vahv}

Jos funktion derivoituvuudessa on ongelmia, tai jos derivaattoja on hankala määrätä, voidaan
käyttää Newtonin mentelmän lähisukulaista, \kor{sekanttimenetelmää}. Tässä ideana on käyttää
pisteiden $(x_{n-1},f(x_{n-1}))$ ja $(x_n,f(x_n))$ kautta kulkevaa suoraa eli käyrän $y=f(x)$ 
\index{sekantti (käyrän)}%
\kor{sekanttia} funktion approksimointiin määrättäessä seuraavaa pistettä $x_{n+1}$. Lauseen 
\ref{Newtonin konvergenssi} oletuksilla sekanttimenetelmän iteraatio suppenee lähes yhtä
nopeasti kuin Newtonin.\footnote[2]{Sekanttimenetelmän asymptoottinen suppenemisnopeus on
lineaarisen ja
kvadraattisen suppenemisen välimuoto; tarkemmin on osoitettavissa, että Lauseen
\ref{Newtonin konvergenssi} oletuksin pätee
\[
\lim_n \frac{|x_{n+1}-c|}{|x_n-c|^\alpha} = \left|\frac{f''(c)}{2f'(c)}\right|,
\]
missä $\alpha=\tfrac{1}{2}(\sqrt{5}+1) \approx 1.62$. (Potenssifunktio
$f(x)=x^\alpha,\ \alpha\not\in\Q$ määritellään jäljempänä Luvussa
\ref{yleinen eksponenttifunktio}.)} Algoritmin käyntiin saattamiseksi on sekanttimenetelmässä
annettava kaksi alkuarvausta $x_0,x_1$.
\begin{figure}[H]
\setlength{\unitlength}{1cm}
\begin{center}
\begin{picture}(12,6)
\drawline(0,2)(5,2) \drawline(7,2)(12,2)
\curve(0.5,1.7,3,3,4,6) \drawline(2,1.8)(4,5)
\curve(7.5,1.7,10,3,11,6) \drawline(11.4,4.55)(8.9,1.8)
\dashline{0.1}(9.9,2)(9.9,2.9)
\dashline{0.1}(3.8,2)(3.8,4.65)
\dashline{0.1}(2.38,2)(2.38,2.43)
\put(2.28,1.6){$x_n$} \put(3.7,1.6){$x_{n-1}$} \put(9.8,1.6){$x_n$}
\put(2,1){$x_{n+1}$} \put(9,1){$x_{n+1}$}
\dashline{0.1}(2.12,1.3)(2.12,2)
\dashline{0.1}(9.08,1.3)(9.08,2)
\put(1,0){Sekanttimenetelmä} \put(9,0){Newton}
\end{picture}
\end{center}
\end{figure}


\Harj
\begin{enumerate}

\item
Seuraavat yhtälöt voidaan ratkaista kiintopisteiteraatiolla. Määritä asymptoottiset 
suppenemisnopeudet tarkasti (jos mahdollista) tai yhden desimaalin tarkkuudella:
\begin{align*}
&\text{a)}\ \ x=\frac{12}{1+x} \qquad\ \
 \text{b)}\ \ x=\sqrt{3+x} \qquad
 \text{c)}\ \ x=\frac{1}{2+x^2} \\
&\text{d)}\ \ x=\sqrt[3]{x+9} \qquad 
 \text{e)}\ \ x=\cos\frac{x}{3} \qquad\quad 
 \text{f)}\ \ x=1+\frac{1}{4}\sin x
\end{align*}

\item
Tutki, mitkä polynomin $p(x)=x^3+8x^2-44x-10$ nollakohdista voidaan tarkentaa
kiintopisteiteraatiolla
\[
x_{n+1}=\frac{1}{44}(x_n^3+8x_n^2-10), \quad n=0,1,\ldots
\]
olettaen, että käytettävissä on riittävän hyvä alkuarvaus $x_0$. Miten tähän 
iteraatiomenetelmään on päädytty?

\item
Yhtälön $x^3+x=1$ reaalista ratkaisua voidaan yrittää hakea kiintopisteiteraatiolla
hajottamalla yhtälö muotoon
\[
\text{a)}\,\ x=1-x^3 \quad\ 
\text{b)}\,\ x=x^{-2}-x^{-1} \quad\
\text{c)}\,\ x=\sqrt[3]{1-x} \quad\
\text{d)}\,\ x=\frac{1}{1+x^2}
\]
Tutki, miten iteraatiot (asymptoottisesti) suppenevat tai hajaantuvat olettaen, että
alkuarvaus on hyvin lähellä kiintopistettä.

\item
a) Yhtälö $\,x=\cos x\,$ voidaan kirjoittaa muotoon $y=\Arccos y$ ja yrittää ratkaista 
kiintopisteiteraatiolla $\,y_{n+1}=\Arccos y_n\,$. Toimiiko menetelmä? \newline
b) Jos yhtälö ratkaistaan iteraatiolla $\,x_0=0,\ x_{n+1}=\cos x_n$, niin mitä lukua kohti ja 
kuinka nopeasti iteraatio suppenee, jos funktioevaluaatioissa $x_n \map \cos x_n$ muuttujan
yksikkö on aste, ts.\ $\cos x=\cos x\aste$\,? Kokeile valisemalla laskimeen astemoodi ja
painelemalla \fbox{$\cos$} -- näppäintä!

\item
Näytä, että jos $x_0$ on rationaaliluku, niin kiintopisteiteraatio
\[
x_{n+1}=(x_n-2)^2, \quad n=0,1,\ldots
\]
suppenee vain, jos $x_0$ on jokin luvuista $0,1,2,3,4$.

\item
Johda Newtonin iteraatiokaava luvun $\sqrt[m]{a}$ määräämiseksi funktion $f(x)=x^m-a$
nollakohtana ($a>0,\ m\in\N,\ m \ge 2$). Päättele suppeneminen kvadraattiseksi. Päättele myös 
geometrisesti, että iteraatio suppenee aina kun $x_0>0$.

\item
Etsi seuraavien funktioiden nollakohdat annetulta väliltä neljän desimaalin tarkkuudella
käyttäen Newtonin menetelmää:
\begin{align*}
&\text{a)}\ \ f(x)=x^3+2x-1, \quad c\in[0,1] \\
&\text{b)}\ \ f(x)=x^4-8x^2-x+16, \quad c\in[1,3] \\
&\text{c)}\ \ f(x)=\cos x-x^2, \quad c\in(-\infty,\infty) \\
&\text{d)}\ \ f(x)=3\sin x-x-1, \quad c\in[0,\infty)
\end{align*}

\item
Laske seuraavien funktioiden maksimi- ja minimiarvot tarkasti, jos mahdollista, muuten
kuuden desimaalin tarkkuudella:
\[
\text{a)}\,\ \frac{\sin x}{1+x^2} \qquad 
\text{b)}\,\ \frac{\cos x}{1+x^2} \qquad
\text{c)}\,\ f(x)=\begin{cases} 
             \dfrac{\sin x}{x}\,, &\text{kun}\ x \neq 0 \\ \,1, &\text{kun}\ x=0
             \end{cases}
\]

\item
Millä $a$:n arvoilla yhtälöllä $\cos x=ax$ on täsmälleen kaksi ratkaisua?

\item
Laske (likimäärin)\, a) funktion $f(x,y)=xy^2+y^4$ maksimiarvo ympyräviivalla
$S:\ x^2+y^2=1$, \, b) funktion $f(x,y)=(x-y)(x+y)^2$ pienin ja suurin arvo ympyräviivalla 
$x=2\cos t,\, y=1+2\sin t,\, t\in [0,2\pi)$.

\item 
Millaisen algoritmisen muodon saa jakolaskuoperaatio $a \map a^{-1}$, kun se suoritetaan 
soveltamalla Newtonin iteraatiota funktioon $f(x)=x^{-1}-a$\,? Tarvitaanko algoritmissa 
jakolaskuja? Kokeile, kun $a=3$.

\item
Funktioevaluaatio $y \map \Arctan y$ halutaan toteuttaa Newtonin menetelmään perustuvalla
algoritmilla, joka sisältää $\R$:n kuntaoperaatioiden lisäksi ainoastaan funktioevaluaatioita 
$x \map \cos x$ ja $x \map \sin x$. Esitä tällainen algoritmi.

\item
Laske luvulle $\sqrt{2}$ approksimaatio iteroimalla neljä kertaa sekanttimenetelmällä
alkuarvauksista $x_0=2,\ x_1=1.5$ (funktio $f(x)=x^2-2$). Vertaa Newtonin menetelmään.

\item (*) \label{H-V-7: kontraktiokuvauslause} \index{kontraktio(kuvaus)}
\index{Kontraktiokuvauslause}
Sanotaan, että funktio $f$ on \kor{kontraktio} välillä $[a,b]$, jos $f$ on välillä $[a,b]$
Lipschitz-jatkuva vakiolla $L<1$. Todista \kor{Kontraktiokuvauslause}: Jos $f$ on kontraktio
välillä $A=[a,b]$ ja lisäksi $f(A) \subset A$, niin pätee:
\begin{itemize}
\item[(i)]  $f$:llä on täsmälleen yksi kiintopiste $c$ välillä $A$.
\item[(ii)] Kiintopisteiteraatio $x_{n+1}=f(x_n),\ n=0,1,\ldots$ suppenee kohti $c$:tä
            jokaisella $x_0 \in A$.
\end{itemize}
\kor{Vihje}: Sovella ensin Bolzanon lausetta funktioon $g(x)=f(x)-x$.

\item (*) \label{H-V-7: yksinkertaistettu Newton}
Funktiosta $f$ tiedetään, että $f$ on (tuntemattoman) nollakohdan $c$ lähellä jatkuvasti
derivoituva ja että $f'(c) \neq 0$. Etsitään nollakohtaa kiintopisteiteraatiolla
\[
x_{n+1}=x_n-kf(x_n), \quad n=0,1,\ldots
\]
Näytä, että jos $x_0$ on riittävän lähellä $c$:tä ja valitaan $k=1/f'(x_0)$, niin iteraatio
suppenee kohti $c$:tä ainakin lineaarisesti. Näytä edelleen, että rajalla $x_0 \kohti c$
suppeneminen muuttuu superlineaariseksi, ts.\ suurilla $n$ pätee $x_n-c \sim q^n$, missä
$q \kohti 0$ kun $x_0 \kohti c$.

\item (*) \label{H-V-7: moninkertainen nollakohta ja Newton}
Näytä, että jos $f$ on polynomi ja $c$ on $f$:n $m$-kertainen nollakohta, $m \ge 2$, niin
Newtonin iteraatio suppenee $c$:tä kohti aina kun alkuarvaus on riittävän lähellä $c$:tä 
ja suppeneminen on asymptoottisesti lineaarista, tarkemmin 
\[
q = \lim_n \frac{x_{n+1}-c}{x_n-c} = \frac{1}{m}\,.
\] 

\item (*) \label{H-V-7: kuutiollisia iteraatioita} \index{kuutiollinen suppeneminen}
Luku $\sqrt{a}$ voidaan määrätä iteraatioilla
\begin{align*}
&\text{a)}\ \ x_{n+1}=\frac{x_n^3+3ax_n}{3x_n^2+a}\,, \quad n=0,1\ldots \\
&\text{b)}\ \ x_{n+1}=\frac{3x_n}{8}+\frac{3a}{4x_n}-\frac{a^2}{8x_n^3}\,, \quad n=0,1,\ldots
\end{align*}
Näytä, että jos $x_n\kohti\sqrt{a}$, niin suppeneminen on kummassakin tapauksessa
\kor{kuutiollista} (kertaluku=3). Onko iteraatioilla muita kiintopisteitä ja minkälaatuisia ne
ovat? Kokeile menetelmiä käytännössä, kun $a=2$, $x_0=1.5$, ja vertaa Esimerkin
\ref{neliöjuuri a} kvadraattiseen menetelmään.

\item (*) \index{zzb@\nim!Laskiainen, 1.\ lasku}
(Laskiainen, 1.\ lasku) Lumilautailija haluaa rakentaa mäen, jota pitkin voi laskea $xy$-tason
origosta pisteeseen $(5,-1)$ nopeinta mahdollista reittiä (gravitaation suunta $-\vec j$, ei
kitkaa). Ryhdy konsultiksi käyttäen vanhaa tietoa\footnote[2]{Lyhimmän ajan käyrän eli
\kor{brakistokronin} ongelman ratkaisi sveitsiläinen matemaatikko \hist{Johann Bernoulli}
(1667-1748) v.\ 1697. Ratkaisemisessa kilpaili myös Johann B:n veli \hist{Jakob} (1654-1705),
jonka mukaan mm.\ Bernoullin epäyhtälö (Propositio \ref{Bernoulli}) on nimetty.
\index{Bernoulli, J.|av} \index{brakistokroni|av} \index{sykloidi!brakistokroni|av}},
jonka mukaan oikea mäen profiili on sykloidin kaari
\[
\begin{cases} \,x=R(t-\sin t), \\ \,y=-R(1-\cos t). \end{cases}
\]
Laske siis $R$ ja esittele graafisesti ehdotuksesi optimaaliseksi mäeksi.

\end{enumerate} % Kiintopisteiteraatio
\section{Analyyttiset kompleksifunktiot} \label{analyyttiset funktiot}
\alku \sectionmark{Analyyttiset funktiot}
\index{funktio A!e@kompleksifunktio|vahv}

Kompleksifunktiolla tarkoitetaan kompleksimuuttujan kompleksiarvoista funktiota eli
funktiota tyyppiä $f: \DF_f\kohti\C,\ \DF_f\subset\C$. Tällainen on esimerkiksi Luvussa
\ref{III-3} tarkasteltu (koko $\C$:ssä määritelty) polynomi. Koska $z$ on
tulkittavissa tason (kompleksitason) pisteeksi, niin kompleksifunktio voidaan ymmärtää
myös kahden reaalimuuttujan kompleksiarvoisena funktiona kirjoittamalla (vrt.\ Luvun
\ref{III-3} Esimerkki \ref{kompleksipolynomin juuret})
\[
f(x+iy) = u(x,y)+iv(x,y),
\]
missä $u(x,y)=\text{Re}\,f(x+iy)$ ja $v(x,y)=\text{Im}\,f(x+iy)$. Huolimatta tästä erosta
suhteessa yhden reaalimuuttujan funktioihin voidaan kompleksifunktioille määritellä käsitteet
jatkuvuus, raja-arvo ja derivoituvuus aivan samalla tavoin kuin reaalifunktioille. Esimerkiksi
$f$ on jatkuva pisteessä $z\in\DF_f$ täsmälleen kun kaikille kompleksilukujonoille $\seq{z_n}$
pätee (vrt.\ Määritelmä \ref{funktion jatkuvuus})
\[ 
z_n\in\DF_f\ \ja\ z_n \kohti z \qimpl f(z_n) \kohti f(z).
\]
Tässä lukujonojen $\seq{z_n}$ ja $\seq{f(z_n)}$ suppeneminen viittaa Määritelmään
\ref{jonon raja}, joka toimii sellaisenaan myös kompleksilukujen jonoille, kunhan merkinnän
$\abs{\cdot}$ tulkitaan tarkoittavan kompleksiluvun itseisarvoa. Määritelmän mukaan pätee
erityisesti (kuten reaalilukujonoillekin)
\[ 
z_n \kohti z \qekv \abs{z_n - z} \kohti 0.
\]
Jos tässä on $z=0$, niin pätee siis
\[
z_n=r_n\vkulma{\varphi_n} \kohti 0 \qekv r_n \kohti 0.
\]
Tämän mukaan lähestyminen kohti kompleksitason pistettä (tässä origoa) voi tapahtua
äärettömän monesta eri suunnasta ($\varphi_n=\varphi\in[0,2\pi)\ \forall n$) tai suuntia
vaihdellen (esim.\ spiraalimainen lähestyminen). Jatkossa nähdään, että tämä kvalitatiivinen
ero suhteessa reaalilukujonoon (jolla mahdollisia lähestymissuuntia kohti raja-arvoa on vain
kaksi) voi tuottaa kompleksifunktoiden raja-arvotarkasteluissa yllätyksiä.
\begin{Exa} Funktio $f(z)=\overline{z}$ (eli $f(x+iy)=x-iy$) on jatkuva $\C$:ssä, sillä
kompleksialgebran (ks.\ Luku \ref{kompleksiluvuilla laskeminen}) mukaan
\[
\abs{f(z_n)-f(z)} = \abs{\overline{z_n}-\overline{z}} 
                  = \abs{\overline{z_n-z}} = \abs{z_n-z} \kohti 0
\]
aina kun $z_n \kohti z$. \loppu
\end{Exa}
Myös jatkuvuuden vaihtoehtoinen määritelmä (Määritelmä \ref{vaihtoehtoinen jatkuvuus})
toimii sellaisenaan kompleksifunktioille, samoin funktion raja-arvon määritelmä.
Raja-arvon käsitteeseen perustuva funktion derivaatta määritellään myös samoin kuin
reaalifunktioille:
\[ 
f'(z) = \lim_{\Delta z \kohti 0} \frac{f(z + \Delta z) - f(z)}{\Delta z}. 
\]
\index{derivoituvuus}%
Jos raja-arvo on olemassa, niin sanotaan, että $f$ on \kor{derivoituva} pisteessä $z$. 
\begin{Exa} \label{kompleksifunktioiden derivoituvuus} Tutki kompleksifunktioiden
\[ 
\text{a)}\ f(z) = \bar{z}, \quad \text{b)}\ f(x+iy) = x^2 + iy^2, \quad 
\text{c)}\ f(x+iy) = x^2-y^2 + 2ixy 
\]
derivoituvuutta. \end{Exa}
\ratk a) Jos $\Delta z_n = \Delta r_n\vkulma{\varphi_n}
= \Delta r_n(\cos\varphi_n + i\sin\varphi_n) \neq 0$ ($\Delta r_n \neq 0$), niin
\[ 
\frac{f(z + \Delta z_n) - f(z)}{\Delta z_n} 
             = \frac{\cos\varphi_n - i\sin\varphi_n}{\cos\varphi_n + i\sin\varphi_n}. 
\]
Nähdään, ettei vaadittua raja-arvoa ole, eli $f$ ei ole missään pisteessä derivoituva.

b) Jos tässä merkitään $\Delta z_n = \Delta x_n + i\Delta y_n$, niin
\[ 
\frac{f(z + \Delta z_n) - f(z)}{\Delta z_n} 
    = 2\,\frac{x\Delta x_n + iy\Delta y_n}{\Delta x_n + i\Delta y_n}
                           + \frac{(\Delta x_n)^2 + (\Delta y_n)^2}{\Delta x_n + i\Delta y_n}. 
\]
Siirtymällä polaariesitykseen nähdään, että jälkimmäinen termi oikealla $\kohti 0$ aina kun 
$\abs{\Delta z_n} = r_n \kohti 0$. Ensimmäisellä termillä sen sijaan on vain lähestymissuunnasta
riippuvia suunnattuja raja-arvoja, ellei ole $x=y=t$, jolloin ko.\ termi yksinkertaistuu muotoon
$2t$. Päätellään siis, että $f$ on derivoituva ainoastaan kompleksitason suoralla
$z = t(1+i),\ t \in \R$, ja tällä suoralla $f'(z) = 2Rez$. 

c) Tässä esimerkissä ollaan onnekkaampia, sillä $f(x+iy) = (x+iy)^2$, jolloin
\[ 
\frac{f(z + \Delta z) - f(z)}{\Delta z} = \frac{(z+\Delta z)^2 - z^2}{\Delta z} 
                                        = 2z + \Delta z, \quad \Delta z \neq 0. 
\]
Siis $f$ on derivoituva jokaisessa pisteessä $z \in \C$ ja $f'(z) = 2z$. \quad \loppu

Esimerkki kertoo, että kompleksifunktion derivoituvuus on kaikkea muuta kuin 'läpihuutojuttu'
siinäkin tapauksessa, että funktion reaali- ja imaginaariosat ovat säännöllisiä funktioita
(kuten esimerkissä polynomeja). Derivoituvuus onkin kom\-pleksifunktiolle paljon vaativampi
ominaisuus kuin reaalifunktiolle.

Seuraavassa määritelmässä laajennetaan derivoituvuusehto koskemaan kompleksitason
\kor{avointa} osajoukkoa $G\subset\C$.
\begin{Def} (\vahv{Analyyttinen kompleksifunktio}) \label{analyyttinen funktio}
\index{analyyttinen kompleksifunktio|emph} \index{avoin joukko|emph}
\index{ympzy@($\delta$-)ympäristö} 
Kompleksitason osa\-joukko $G\subset\C$ on \kor{avoin}, jos jokaisella $z_0 \in G$ on
\kor{ympäristö}
\[ 
U_{\delta}(z_0) = \{z \in \C \mid \abs{z-z_0}<\delta\},\ \ \delta>0
\]
siten, että $U_{\delta} \subset G$. Kompleksifunktio, joka on derivoituva ei-tyhjässä,
avoimessa joukossa $G\subset\C$, on \kor{analyyttinen} $G$:ssä.
\end{Def}
Analyyttiset funktiot muodostavat kompleksifunktioiden tärkeän ja paljon tutkitun
'vähemmistön'.\footnote[2]{Analyyttisten kompleksifunktioiden teorian tarkempi esittely
kuuluu toisiin yhteyksiin. Mainittakoon teorian tuloksista kuitenkin, että jos $f$ on
analyyttinen $G$:ssä, niin samoin on $f'$, jolloin $f$ on itse asiassa mielivaltaisen monta
kertaa derivoituva $G$:ssä. --- Tämä tulos kertoo osaltaan, että kompleksifunktion analyytisyys
on paljon voimakkaampi ominaisuus kuin reaalifunktion derivoituvuus.}
\jatko \begin{Exa} (jatko) Esimerkissä c-kohdan funktio $f(z) = z^2$ on analyyttinen koko 
kompleksitasossa ($G = \C$), kun taas a- ja b- kohtien funktiot eivät ole analyyttisiä
missään. \loppu 
\end{Exa}
\index{kokonainen funktio}%
Jos $f$ on analyyttinen koko kompleksitasossa, niin sanotaan, että $f$ on  \kor{kokonainen} 
(engl.\ entire) funktio. Kaikki polynomit $p(z)$ (myös kompleksikertoimiset) ovat kokonaisia 
funktioita, sillä polynomin derivoimissääntö perustuu vain kunta-algebraan, joka ei lainkaan 
muutu siirryttäessä reaalimuuttujasta kompleksimuuttujaan (vrt.\ Esimerkki 
\ref{kompleksifunktioiden derivoituvuus}, c-kohta). Algebraan perustuvat myös funktioiden
summan, tulon, osamäärän ja yhdistetyn funktion derivoimissäännöt reaalifunktioille, joten
niissäkin voidaan reaalimuuttujan $x$ tilalle vaihtaa kompeksimuuttuja $z$ säännön 
\index{kompleksimuuttujan!b@rationaalifunktio} \index{rationaalifunktio}%
muuttumatta. Näiden sääntöjen perusteella voidaan erityisesti jokainen (kompleksikertoiminen)
\kor{kompleksimuuttujan rationaalifunktio} derivoida samalla tavoin kuin reaalimuuttujan 
tapauksessa. Rationaalifunktiot ovatkin Määritelmän \ref{analyyttinen funktio} mukaisesti
analyyttisiä koko määrittelyjoukossaan, sillä jos $f(z) = p(z)/q(z)$ ($p$ ja $q$ polynomeja)
ja $z_0\in\DF_f$, niin $q(z_0) \neq 0$, jolloin $q(z) \neq 0$ myös jossakin ympäristössä
$U_{\delta}(z_0)$. Tällöin $U_{\delta}(z_0)\subset\DF_f$, eli $\DF_f$ on avoin joukko.
\begin{Exa} Rationaalifunktiot
\[
f(z) = \frac{1}{z^2 + 1}\,, \quad g(z) = \frac{i}{z^2 + iz}
\]
on määritelty koko kompleksitasossa lukuunottamatta pisteitä $\pm i$ ($f$) ja $0,-i$ ($g$).
Määrittelyjoukot ovat avoimia, ja molemmat funktiot ovat määrittelyjoukossaan derivoituvia,
siis analyyttisiä. Derivaatat lasketaan kuten reaalimuuttujan tapauksessa:
\[ 
f'(z) = -\frac{2z}{(z^2 + 1)^2}\,, \quad g'(z) = -\frac{i(2z+i)}{(z^2 + iz)^2}\,. \loppu 
\] 
\end{Exa}

Jos kompleksifunktio on analyyttinen nollakohtansa ympäristössä, niin nollakohtaa voidaan
etsiä Newtonin iteraatiolla
\[
z_{n+1} = z_n - \frac{f(z_n)}{f'(z_n)}\,, \quad n=0,1,\ldots
\]
Esimerkiksi polynomin yksinkertaista nollakohtaa etsittäessä algoritmi toimii erinomaisesti,
kunhan alkuarvaus on riittävän hyvä
(ks.\ Harj.teht.\,\ref{H-V-8: Newton 1}--\ref{H-V-8: Newton 3}).

\Harj
\begin{enumerate}

\item
Näytä suoraan kompleksifunktion derivaatan määritelmästä, että \newline
a) \ $\dif z^{-1}=-z^{-2},\quad$ b) \ $\dif (z+i)^{-2}=-2(z+i)^{-3}.$

\item
Missä kompleksitason osajoukossa seuraavat funktiot ovat analyyttisiä?
\begin{align*}
&\text{a)}\ \ f(z)=z^3+iz \qquad \text{b)}\ \ f(z)=\frac{z+1}{z^2+z+1} \qquad
 \text{c)}\ \ f(z)=\frac{1}{z^3-8} \\
&\text{d)}\ \ f(z)=\frac{z}{\abs{z}^2} \qquad 
 \text{e)}\ \ f(z)=\frac{\overline{z}}{\abs{z}^2} \qquad
 \text{f)}\ \ f(z)=(z+\overline{z})^2-2\abs{z}^2-\overline{z}^2
\end{align*}

\item \label{H-V-8: Newton 1}
Näytä, että jos $a\in\C,\ a \neq 0$, niin Newtonin iteraatio
\[
z_0\in\C, \quad z_{n+1} = \frac{1}{2}\left(z_n + \frac{a}{z_n}\right), \quad n=0,1,\ldots
\]
(vrt.\ Luku \ref{kiintopisteiteraatio}, Esimerkki \ref{neliöjuuri a}) suppenee kvadraattisesti
kohti funktion $f(z)=z^2-a$ nollakohtaa, jos alkuarvaus $z_0$ on nollakohtaa (kumpaa tahansa)
riittävän lähellä. Kokeile iteraation toimivuuttaa tapauksessa $a=i$ valinnoilla a) $z_0=1$, \
b) $z_0=i$, \ c) $z_0=-1-i$.

\item \label{H-V-8: Newton 2}
Etsi polynomin $f(z)=z^4+z+4$ nollakohdat likimäärin Newtonin menetelmällä. Huomaa, että
Newtonin iteraatio ei tässä tapauksessa suppene reaalisilla alkuarvauksilla --- miksei?

\item (*) \label{H-V-8: Newton 3}
Todista Lauseen \ref{Newtonin konvergenssi} vastine analyyttiselle kompleksifunktiolle
$f(z)$ tapauksessa, jossa $f$ on polynomi.

\end{enumerate} % Analyyttiset kompleksifunktiot
\section{*Jatkuvuuden logiikka} \label{jatkuvuuden logiikka}
\alku

Tässä luvussa todistetaan Luvussa \ref{jatkuvuuden käsite} esitetyt kaksi jatkuvien
funktioiden päälausetta, Weierstrassin lause (Lause \ref{Weierstrassin peruslause}) ja
käänteisfunktion jatkuvuutta koskeva Lause \ref{käänteisfunktion jatkuvuus}. Todistukset
perustuvat Luvussa \ref{Cauchyn jonot} esitettyyn osajonojen teoriaan, ja ne ovat melko
vaativia. Erityisesti rajoitettuja reaalilukujonoja koskeva Bolzanon--Weierstrassin lause
(Lause \ref{B-W}) on todistuksissa ahkerassa käytössä. Weierstrassin lauseen todistuksen
jatkoksi näytetään, että sama todistustekniikka, yhdistettynä eräisiin polynomeja koskeviin
teknisempiin väittämiin, johtaa jopa Algebran peruslauseen todistukseen. Luvun lopussa
määritellään vielä käsite \kor{tasainen jatkuvuus}, jota voi pitää Lipschitz-jatkuvuuden
minimalistisena vastineena. Tasaisen jatkuvuuteen liittyen todistetaan eräs reaalianalyysin
hämmästyttävimmistä lauseista.

Koska tarkoituksena on todistaa Lauseet \ref{Weierstrassin peruslause} ja 
\ref{käänteisfunktion jatkuvuus} hieman yleisemmässä muodossa, määritellään aluksi
suljettua väliä yleisempi \kor{kompaktin} joukon käsite. Asiayhteyden vuoksi esitellään
samalla muitakin $\R$:n nk.\ \kor{topologisia} peruskäsitteitä.
 
\subsection{Avoimet, suljetut ja kompaktit joukot}

Avointa väliä
\[ 
U_{\delta}(x) = (x-\delta,x+\delta) \quad (\delta>0)
\] 
\index{ympzy@($\delta$-)ympäristö}%
sanotaan pisteen $x \in \R$ (avoimeksi) \kor{ympäristöksi} (tai tarkemmin 
$\delta$-ympäristöksi, engl.\ $\delta$-neighbourhood; vrt.\ vastaava kompleksitason käsite
Määritelmässä \ref{analyyttinen funktio}).
\begin{Def} \label{avoimet ym. joukot}
\index{avoin joukko|emph} \index{suljettu joukko|emph} \index{rajoitettu!b@joukko|emph}
\index{kompakti joukko|emph}
Joukko $A\subset\R$ on
\begin{itemize}
\item[-] \kor{avoin}, jos $\forall x \in A$ pätee
         $\,(x-\delta,x+\delta)\subset A\,\ \text{jollakin}\ \delta>0$,
\item[-] \kor{suljettu}, jos kaikille reaalilukujonoille $\seq{x_n}$ pätee: \newline
         $\,x_n \in A\ \forall n\ \ja\ x_n\kohti x\in\R\ \impl\ x\in A$,
\item[-] \kor{rajoitettu}, jos $\exists C\in\R_+$ siten, että 
         $\,\abs{x} \le C\ \forall x \in A$,
\item[-] \kor{kompakti}, jos $A$ on suljettu ja rajoitettu.
\end{itemize}
\end{Def}
\begin{Exa} Avoin väli $(a,b)$ on avoin joukko, sillä jos $x\in(a,b)$, nii
$U_\delta(x)\subset(a,b)$, kun $0<\delta\le\min\{x-a,b-x\}$. Suljettu väli on vastaavasti
suljettu joukko, sillä jos $x_n\in[a,b]\ \forall n$ ja $x_n \kohti x\in\R$, niin
$a \le x \le b$ (Lause \ref{jonotuloksia} [V1]) eli $x\in[a,b]$. Koska suljettu väli on myös
rajoitettu ($\abs{x} \le C = \max\{\abs{a},\abs{b}\}\ \forall x\in[a,b]$), niin suljettu
väli on kompakti joukko. \loppu
\end{Exa}
\begin{Exa}
\[
\begin{array}{ll}
\left[0,1\right] \cup [2,3]   &\text{on kompakti (ei avoin)}, \\[1mm]
(0,1) \cup (2,3)              &\text{on avoin ja rajoitettu (ei suljettu)}, \\[1mm]
\R                            &\text{on avoin ja suljettu (ei rajoitettu)}, \\[1mm]
[0,1)                         &\text{on rajoitettu (ei avoin eikä suljettu)}, \\[1mm]
\left[0,\infty\right)         &\text{on suljettu (ei avoin eikä rajoitettu)}, \\[1mm]
(0,\infty)                    &\text{on avoin (ei suljettu eikä rajoitettu)}, \\[1mm]
\Q                            &\text{ei avoin, ei suljettu eikä rajoitettu}, \\[1mm]
\emptyset                     &\text{(tyhjä) on avoin ja kompakti}. \qquad\quad\loppu
\end{array}
\]
\end{Exa}

\begin{Def}
\index{komplementti (joukon)|emph} \index{sulkeuma (joukon)|emph}
\index{reuna (joukon, alueen)|emph} \index{siszy@sisäpiste|emph}
Joukon $A\subset\R$
\begin{itemize}
\item[-] \kor{komplementti} on $\ \complement(A)=\{x\in\R \mid x\notin A\}$.
\item[-] \kor{sulkeuma} (engl. closure) on 
\[
\overline{A}=\{x\in\R \mid \exists\ \text{jono}\ \{x_n\}\ \text{siten, että}\ 
                                     x_n\in A\ \forall n\,\ \ja\,\ x_n \kohti x\}.
\]
\item[-] \kor{reuna} (engl. boundary) on 
         $\ \partial A=\overline{A}\cap \overline{\complement(A)}$.
\item[-] \kor{sisäpisteiden joukko} on $\ A_0 = \complement(\,\overline{\complement(A)}\,)$.
\end{itemize}
\end{Def}
Joukon sulkeuma on nimensä mukaisesti suljettu joukko, ja toinen määritelmä onkin:
$\overline{A}$ on pienin suljettu joukko, joka sisältää $A$:n. Sisäpisteiden joukko $A_0$ on
avoin (ks.\ Lause \ref{avoin vs suljettu} jäljempänä); tämän vaihtoehtoisia määritelmiä ovat:
$A_0$ on suurin $A$:n avoin osajoukko, tai
\[ 
A_0 = \{x \in A \mid x \not\in \partial A\}. 
\]
Pätee myös $\,\overline{A} = A_0 \cup \partial A,\ A_0 \cap \partial A = \emptyset$, samoin
pätee
\[ 
A\ \text{suljettu}\ \ekv\ A = \overline{A}\ \ekv\ \partial A \subset A.
\]
\begin{Exa}
Joukkojen
\[
A=(0,1],\quad B=(-1,0)\cup (0,1),\quad C=\Q
\]
sulkeumat, reunat ja sisäpisteiden joukot ovat
\[
\begin{array}{rclrclrcl}
\overline{A} &=& [0,1],\   & \overline{B} &=& [-1,1],\      & \overline{C} &=& \R, \\
  \partial A &=& \{0,1\},\ & \ \partial B &=&  \{-1,0,1\},\ & \ \partial C &=& \R, \\
         A_0 &=& (0,1),    & B_0          &=& B,\           & C_0          &=& \emptyset. 
                                                                           \quad\loppu  
\end{array}
\]
\end{Exa}
Haluttaessa selvittää, onko annettu joukko avoin, suljettu tai kompakti, helpottuu tehtävä
usein huomattavasti seuraavia loogisia väittämiä hyödyntämällä. Väittämistä ensimmäisen voi
tulkita myös suljetun tai avoimen joukon määritelmäksi, jos vain toinen käsitteistä on
määritelty erikseen. 
\begin{Lause} \label{avoin vs suljettu} Pätee
\[ 
A\,\text{ avoin }\ \ekv\ \complement(A)\,\text{ suljettu}, \qquad 
A\,\text{ suljettu }\ \ekv\ \complement(A)\,\text{ avoin}. 
\]
\end{Lause}
\begin{Lause} \label{unionit ja leikkaukset} Pätee
\begin{align*}
&A,B\,\text{ avoimia/suljettuja/kompakteja } \\
& \ \impl\ A\cup B\,\text{ ja } A\cap B\text{ avoimia/suljettuja/kompakteja}.
\end{align*}
\end{Lause}
\tod Väittämiin sisältyy yhteensä kymmenen implikaatioväittämää, joiden todistukset ovat
kaikki melko lyhyitä. Todistetaan esimerkkinä ainoastaan väittämä
$\,A$ avoin $\impl\ \complement (A)$ suljettu\, eli: Jos $A$ on avoin, niin jokaiselle
reaalilukujonolle $\seq{x_n}$ pätee
\[ 
x_n \in \complement(A)\ \forall n\,\ \ja\,\ x_n \kohti x \qimpl x \in \complement (A). 
\]
Tehdään vastaoletus: $x\not\in\complement (A)\ \ekv\ x \in A$. Tällöin 
$(x-\delta,x+\delta) \subset A$ jollakin $\delta>0$, koska $A$ oli avoin. Tällöin koska
$x_n\in\complement(A)\ \forall n$, on oltava
\[ 
\abs{x_n-x} \ge \delta\ \ \forall n \qimpl x_n \not\kohti x.
\]
Tämä on looginen ristiriita, koska oletettiin, että $x_n \kohti x$. \loppu
\begin{Exa} Joukko $A=(-\infty,a)\cup(b,\infty)$ on Lauseen \ref{unionit ja leikkaukset}
(tai suoraan Määritelmän \ref{avoimet ym. joukot}) mukaan avoin. Lauseen
\ref{avoin vs suljettu} mukaisesti komplementti
\[
\complement (A) = \begin{cases} 
                  \,[a,b], &\text{kun}\ a<b, \\ 
                  \,\{a\}, &\text{kun}\ a=b, \\
                  \,\emptyset, &\text{kun}\ a>b
                  \end{cases}
\]
on suljettu. \loppu
\end{Exa}
\begin{Exa} Yhden alkion sisältävä joukko $A=\{a\}$ on Määritelmän \ref{avoimet ym. joukot}
mukaan kompakti, joten Lauseen \ref{unionit ja leikkaukset} mukaan samoin on jokainen kahden,
kolmen, jne.\ alkion joukko. Siis jokainen äärellinen joukko on kompakti. \loppu
\end{Exa}
\begin{Exa} Rationaalifunktion $f(x)=p(x)/q(x)$ ($p$ ja $q$ polynomeja) määrittelyjoukko on
$\DF_f = \complement (A)$, missä $A = \{q\text{:n nollakohdat}\}$. Äärellisenä joukkona $A$
on suljettu (jopa kompakti), joten Lauseen \ref{avoin vs suljettu} mukaan $\DF_f$ on avoin.
\loppu \end{Exa} 

\subsection{Weierstrassin lause}
\index{Weierstrassin lause|vahv}

Funktion $f$ jatkuvuus suljetulla välillä $[a,b]\subset\DF_f$ tarkoitti Määritelmän
\ref{jatkuvuus välillä} mukaisesti jatkuvuutta 'sisältä päin' ko.\ välillä. Asetetaan
vastaavalla tavalla yleisempi määritelmä.
\begin{Def} \label{jatkuvuus kompaktissa joukossa}
\index{jatkuvuus (yhden muuttujan)!g@kompaktissa joukossa|emph}
Reaaliunktio $f$ on \kor{jatkuva kompaktissa joukossa} $K\subset\DF_f$, jos jokaiselle
reaalilukujonolle $\seq{x_n}$ pätee
\[
x_n \in K\ \forall n\,\ \ja\,\ x_n \kohti x\,\ 
                               \impl\,\ f(x_n) \kohti f(x).\footnote[2]{Huomautettakoon,
että määritelmän ehto toteutuu mille tahansa reaalifunktiolle $f$, jos $K\subset\DF_f$ on
äärellinen (ja siis kompakti) joukko. Vrt.\ alkuperäinen jatkuvuuden määritelmä
(Määritelmä \ref{funktion jatkuvuus}), jonka mukaan funktio on jatkuva määrittelyjoukkonsa
eristetyissä pisteissä.}
\]
\end{Def}
\begin{*Lause} \label{kompaktissa joukossa jatkuva funktio on rajoitettu}
Jos $f$ on jatkuva kompaktissa joukossa $K\subset\DF_f$, niin $f(K)$ on rajoitettu joukko.
\end{*Lause}
\tod Käytetään epäsuoraa todistustapaa, eli tehdään vastaoletus: $f(K)$ ei ole rajoitettu.
Tällöin on olemassa jono $\seq{x_n}$ siten, että
\[
x_n \in K\ \forall n\,\ \ja\,\ \abs{f(x_n)}\kohti\infty.
\]
Koska $A$ on (kompaktina joukkona) rajoitettu, on jono $\seq{x_n}$ rajoitettu. Tällöin
jonolla on Bolzanon--Weierstrassin lauseen (Lauseen \ref{B-W}) mukaan suppeneva osa\-jono. Kun
indeksoidaan tämä osajono jonoksi $\seq{x_k}$, niin pätee siis $x_k \kohti x\in\R$. Koska $K$
on (kompaktina joukkona) suljettu, on oltava $x \in K$. On siis löydetty jono $\seq{x_k}$,
jolle pätee $x_k \in K\ \forall k$ ja $x_k \kohti x \in K$. Tällöin jatkuvuusoletuksen ja
Määritelmän \ref{jatkuvuus kompaktissa joukossa} mukaan on oltava $f(x_k) \kohti f(x)$.
Toisaalta koska alkuperäiselle jonolle $\seq{f(x_n)}$ pätee $\abs{f(x_n)}\kohti\infty$, niin
myös $\abs{f(x_k)}\kohti\infty$. Oletuksista ja vastaoletuksesta seurasi siis, että on
olemassa jono $\seq{x_k}$, jolle pätee sekä $f(x_k) \kohti f(x)\in\R$ että
$\abs{f(x_k)}\kohti\infty$. Tämä on looginen ristiriita, joten lause on todistettu. \loppu
 
Seuraava tulos sisältää erikoistapauksena Weierstrassin lauseen
\ref{Weierstrassin peruslause}.
\begin{*Lause} \label{weierstrass} Jos reaalifunktio $f$ on jatkuva kompaktissa joukossa
$K\subset\DF_f$, niin $f$ saavuttaa $K$:ssa pienimmän ja suurimman arvonsa. 
\end{*Lause}
\tod Koska $f(K)$ on Lauseen \ref{kompaktissa joukossa jatkuva funktio on rajoitettu} mukaan
rajoitettu, niin Lauseen \ref{supremum-lause} mukaan tällä joukolla on supremum:
$\,\sup f(K) = y\in\R$. Supremumin määritelmän (ks.\ Luku \ref{reaalilukujen ominaisuuksia})
mukaisesti on olemassa jono $\seq{y_n}$, jolle pätee $\,y_n \in f(K)\ \forall n$ ja
$y_n \kohti y$. Koska $y_n \in f(K)$, on $f(x_n) = y_n$ jollakin $x_n \in K$. Koska $K$ on
(kompaktina joukkona) rajoitettu, on jono $\seq{x_n}$ on rajoitettu, joten sillä on Lauseen
\ref{B-W} mukaan suppeneva osajono. Indeksoidaan tämä uudelleen jonoksi $\seq{x_k}$, jolloin
pätee $x_k \kohti x\in\R$. Koska $K$ on (kompaktina joukkona) suljettu, niin on oltava
$x \in K$. Tällöin oletuksen ja Määritelmän \ref{jatkuvuus kompaktissa joukossa} mukaan
$f(x_k) \kohti f(x)$. Mutta jono $\seq{f(x_k)}$ on alkuperäisen jonon $\seq{f(x_n)}$ osajono,
joten pätee myös $f(x_k) \kohti y$ (Lause \ref{suppenevat osajonot}). Siis $f(x)=y$ 
(koska lukujonon raja-arvo on yksikäsitteinen). Koska $y$ on $f(K)$:n yläraja, on 
$f(t) \le y\,\ \forall t \in K$ --- siis $f$ saavuttaa joukossa $K$ suurimman arvonsa
pisteessä $x$. Lause on näin todistettu maksimiarvon osalta. Minimiarvon osalta lause
todistetaan joko vastaavalla päättelyllä tai soveltamlla jo todistettua väittämää funktioon
$-f$. \loppu
\begin{Exa} Funktio 
\[ f(x)= \begin{cases} \,1/x,\,\ &\text{kun}\ x>0, \\ \,0, &\text{kun}\ x=0 
         \end{cases} \]on jatkuva (kompaktissa) joukossa $K=[a,b]$, jos $0<a<b$. Tällöin $f$:n maksimimiarvo $K$:ssa
$= f(a)$ ja minimiarvo $= f(b)$. Kompaktilla välillä $[0,1]\subset\DF_f$ ei $f$ ole 
Määritelmän \ref{jatkuvuus kompaktissa joukossa} mukaisesti jatkuva, eikä $f$ myöskään saavuta
maksimiarvoaan tällä välillä (minimiarvo $=0$). Välin $A=(0,1]$ (rajoitettu, ei suljettu)
jokaisessa pisteessä $f$ on jatkuva, mutta $f$ ei saavuta $A$:ssa maksimiarvoaan
(minimiarvo $=1$). Välillä A=(1,2) (rajoitettu, ei suljettu) $f$ ei saavuta kumpaakaan
ääriarvoaan. Välillä $A=[1,\infty)$ (suljettu, ei rajoitettu) $f$ on jatkuva jokaisessa
pisteessä, mutta $f$ ei saavuta $A$:ssa minimiarvoaan (maksimiarvo $=1$). \loppu
\end{Exa}

\subsection{Algebran peruslause}
\index{Algebran peruslause|vahv}

Käsitteet avoin, suljettu ja kompakti joukko määritellään kompleksitasossa aivan samalla
tavoin kuin $\R$:ssä. Erona on ainoastaan, että avoimen joukon määrittelyssä tarvittava
ympäristö on $\C$:ssä kiekon muotoinen: $\,U_\delta(c)=\{z\in\C \mid \abs{z-c}<\delta\}$
(Määritelmä \ref{analyyttinen funktio}). Lauseet \ref{avoin vs suljettu} ja
\ref{unionit ja leikkaukset} pätevät myös kompleksitasossa.
\begin{Exa} Kiekko $K=\{z\in\C \mid \abs{z} \le R\}$ on suljettu, ts.\ jokaiselle
kompleksilukujonolle $\seq{z_n}$ pätee: 
$z_n \in K\,\forall n\,\ja\,z_n \kohti z\ \impl\ z \in K$. $K$ on myös rajoitettu
($\abs{z} \le C=R\ \forall z \in K$), joten $K$ on kompakti joukko. Komplementti
$\complement(K)=\{z\in\C \mid \abs{z}>R\}$ on avoin joukko. \loppu
\end{Exa}
Kompleksifunktion jatkuvuus määriteltiin edellisessä luvussa vastaavalla tavalla kuin
reaalifunktioille, ja myös jatkuvuus kompaktissa joukossa (Määritelmä 
\ref{jatkuvuus kompaktissa joukossa}) yleistyy vastaavasti koskemaan kompleksifunktioita.
Weierstrassin lauseen vastine kompleksifunktioille on seuraava väittämä, jonka todistus
noudattaa hyvin tarkoin Lauseen \ref{weierstrass} todistuksen logiikkaa.
(Todistus sivuutetaan.)
\begin{*Lause} \label{weierstrass kompleksifunktioille}
Jos $f$ on kompaktissa joukossa $K\subset\C$ jatkuva kompleksiarvoinen funktio, niin
$\abs{f}$ saavuttaa $K$:ssa pienimmän ja suurimman arvonsa.
\end{*Lause}
Algebran peruslauseen (Lause \ref{algebran peruslause}) todistus voidaan perustaa tähän
tulokseen sekä Luvun \ref{III-3} tarkasteluihin: Olkoon $p(z)$ polynomi astetta $n \ge 1$
(ei vakio). Tällöin $\abs{p(z)}$ saavuttaa Lauseen \ref{weierstrass kompleksifunktioille}
mukaan minimiarvonsa kompaktissa joukossa
\[
K=\{z\in\C \ | \ \abs{z}\leq R\}
\]
jokaisella $R \in \R_+$. Koska $\abs{p(z)} \sim \abs{z}^n$, kun $\abs{z}\kohti\infty$ 
(ks.\ Propositio \ref{polynomin kasvu}), niin päätellään, että $R$:n ollessa riittävän suuri
on $\abs{p}$:n minimikohta $K$:ssa samalla $\abs{p}$:n absoluuttinen minimikohta $\C$:ssä. Siis 
$\abs{p}$ saavuttaa jossakin pisteessä $c \in \C$ absoluuttisen minimiarvonsa. Tällöin on
seuraavan väittämän mukaan oltava $p(c)=0$, jolloin Algebran peruslause on todistettu.
\begin{Lause} \label{polynomitulos} Jos kompleksimuuttujan polynomi $p$ ei ole vakio, niin 
$p(c)=0$ jokaisessa pisteessä $c \in \C$, jossa $\abs{p}$:llä on paikallinen minimi. 
\end{Lause}
\tod Ks. Harj.teht.\,\ref{III-3}:\ref{H-III-3: avaintulos}. \loppu

\subsection{Käänteisfunktion jatkuvuus}
\index{jatkuvuus (yhden muuttujan)!e@käänteisfunktion|vahv}

Seuraava käänteisfunktion jatkuvuuden takaava lause on yleistys Lauseesta 
\ref{käänteisfunktion jatkuvuus}. Todistuksessa on Bolzanon--Weierstrassin lause jälleen
keskeisessä roolissa.
\begin{*Lause} \label{R:n käänteisfunktiolause} Jos $f$ on jatkuva kompaktissa joukossa
$A\subset\DF_f$ ja $f:\,A \kohti B$ on bijektio, niin $B$ on kompakti ja
$\inv{f}:\,B \kohti A$ on samoin jatkuva kompaktissa joukossa $B$.
\end{*Lause}
\tod  Näytetään ensin, että $B$ on kompakti joukko. Lauseen
\ref{kompaktissa joukossa jatkuva funktio on rajoitettu} mukaan $B$ on rajoitettu, joten 
riittää osoittaa, että $B$ on suljettu. Oletetaan siis, että $y_n \in B\ \forall n$ ja 
että $y_n \kohti y\in\R$. Koska $y_n \in B=f(A)$, niin jokaisella $n$ on $y_n=f(x_n)$ jollakin
$x_n \in A$. Koska $A$ on (kompaktina joukkona) rajoitettu, niin jono $\seq{x_n}$ on
rajoitettu, jolloin tällä jonolla on osajono $\seq{x_k}$, joka suppenee: $x_k \kohti x\in\R$
(Lause \ref{B-W}). Koska $A$ on (kompaktina joukkona) suljettu, niin $x \in A$. Tällöin koska
$f$ on $x$:ssä jatkuva Määritelmän \ref{jatkuvuus kompaktissa joukossa} mukaisesti, niin
 $f(x_k) \kohti f(x)$. Mutta $\seq{f(x_k)}=\seq{y_k}$ on jonon $\seq{y_n}$ osajono, joten
$f(x_k)=y_k \kohti y$. Lukujonon raja-arvon yksikäsitteisyyden perusteella on silloin
$f(x)=y$, joten $y \in f(A)=B$. On näytetty, että jokaiselle lukujonolle $\seq{y_n}$ pätee: 
$y_n \in B\ \forall n\,\ \ja\,\ y_n \kohti y\ \impl\ y \in B$. Siis $B$ on suljettu.

Käänteisfunktion $\inv{f}$ väitetyn jatkuvuuden osoittamiseksi on näytettävä: Jos
$y_n \in B,\ n=1,2,\ldots\,$ ja $f(x_n)=y_n\,,\ x_n \in A$, niin pätee
\[
y_n\kohti y \in B \qimpl x_n\kohti x\in A \ \ja \ x=f^{-1}(y).
\]
Koska $x_n \in A$ ja $A$ on kompakti, niin jono $\seq{x_n}$ on rajoitettu, joten jonolla on
suppeneva osajono $\seq{x_k}:\ x_k \kohti x \in \R$. Koska $A$ on kompakti, niin $x \in A$,
jolloin $f$:n jatkuvuuden nojalla $f(x_k) \kohti f(x)$. Mutta jono $\seq{f(x_k)}$ on jonon 
$\seq{f(x_n)} = \seq{y_n}$ osajono, joten oletuksen $y_n \kohti y$ mukaan on oltava myös 
$f(x_k) \kohti y$. Siis $f(x)=y\ \ekv\ x=f^{-1}(y)$. On päätelty, että jonolla $\seq{x_n}$
on ainakin osajono $\seq{x_k}$, jolle pätee $x_k \kohti x \in A$ ja $x = f^{-1}(y)$. Näytetään
nyt, että myös alkuperäiselle jonolle pätee $x_n \kohti x$, jolloin lause on todistettu. 
Tehdään vastaoletus: $x_n \not\kohti x$. Tällöin Lauseen \ref{negaatioperiaate}
mukaan on olemassa toinen osajono $\{x_l\}$ ja $\eps>0$ siten, että pätee
\[
\abs{x_l-x}\geq\eps\quad\forall l\in\N.
\]
Tälläkin osajonolla on kuitenkin suppeneva osajono $\{x_\nu\}$, jolle siis pätee 
$x_\nu\kohti x'\neq x$ kun $\nu\kohti\infty$. Tällöin on jälleen $x'\in A$, koska $A$ on
suljettu, joten $f$:n jatkuvuuden perusteella $f(x_\nu)\kohti f(x')$. Toisaalta koska jono
$\seq{f(x_\nu)}$ on edelleen alkuperäisen jonon $\seq{f(x_n)} = \seq{y_n}$ osajono, niin
oletuksesta $y_n \kohti y$ seuraa, että myös $f(x_\nu) \kohti y$. Siis pätee sekä 
$f(x_\nu) \kohti f(x')$ että $f(x_\nu) \kohti y$, jolloin on oltava $f(x')=y=f(x)$.
Oletuksista ja vastaoletuksesta on näin johdettu päätelmä: On olemassa $x \in A$ ja 
$x' \in A$ siten, että $x \neq x'$ ja $f(x)=f(x')$. Mutta (toistaiseksi käyttämättömän)
oletuksen mukaan $f$ on injektio, joten  $x \neq x'\ \impl\ f(x) \neq f(x')$. Siis
 $f(x)=f(x')$ ja $f(x) \neq f(x')$ --- looginen ristiriita, joka osoittaa tehdyn
vastaoletuksen vääräksi. Siis $x_n \kohti x=f^{-1}(y)$. \loppu
\begin{Exa}
Määritellään funktio $f$ joukossa
\begin{multicols}{2} \raggedcolumns
\[
D_f=A=[-1,0]\cup (1,2]
\]
seuraavasti:
\[
f(x)=\begin{cases}
x &,\text{ kun } x\in [-1,0] \\
x-1 \ &, \text{ kun } x\in (1,2]
\end{cases}
\]
\begin{figure}[H]
\setlength{\unitlength}{1cm}
\begin{center}
\begin{picture}(4,4)(-2,-2)
\put(-2,0){\vector(1,0){4}} \put(1.8,-0.4){$x$}
\put(0,-2){\vector(0,1){4}} \put(0.2,1.8){$y$}
\drawline(-1,-1)(0,0)
\drawline(1,0)(2,1)
\put(-0.1,-0.1){$\bullet$}
\end{picture}
%\caption{$y=f(x)$}
\end{center}
\end{figure}
\end{multicols}
Tällöin $f$ on koko määrittelyjoukossaan jatkuva ja $f:A \kohti [-1,1]$ on bijektio, joten myös 
$\inv{f}:[-1,1] \kohti A$ on bijektio (vrt. Luku \ref{käänteisfunktio}). Mutta \inv{f} ei ole 
jatkuva pisteessä $x=0$. \loppu
\end{Exa}
Esimerkissä on käänteisfunktion jatkuvuuden kannalta ongelmana, että määrittelyjoukon osaväli 
$(1,2]$ ei ole suljettu. Ongelmaa ei voi poistaa ottamalla $x=1$ mukaan määrittelyjoukkoon,
sillä jos asetetaan $f(1)=0$, niin $f(0)=f(1)$, jolloin $f$ ei ole injektio, ja jos asetetaan 
$f(1)=c \neq 0$, niin $f$ ei ole (oikealta) jatkuva pisteessä $x=1$.

\subsection{Tasainen jatkuvuus}
\index{tasainen jatkuvuus|vahv} \index{jatkuvuus (yhden muuttujan)!h@tasainen|vahv}

Tarkastellaan vielä uutta jatkuvuuden käsitettä, jota voi pitää Luvussa 
\ref{väliarvolause 2} määritellyn Lipschitz-jatkuvuuden minimalistisena vastineena 
(vrt.\ Määritelmä \ref{funktion l-jatkuvuus} ja Lause \ref{tasaisen jatkuvuuden käsite} alla).
\begin{Def} Reaalifunktio $f$ on joukossa $A\subset\DF_f$ \kor{tasaisesti jatkuva} 
(engl. uniformly continuous), jos kaikille reaalilukujonoille $\seq{x_n}$ ja $\seq{t_n}$ pätee
\[
x_n,t_n\in A \ \ja \ x_n-t_n\kohti 0 \ \impl \ f(x_n)-f(t_n)\kohti 0.
\]
\end{Def}
Jos $f$ on tasaisesti jatkuva välillä $(c-\delta,c+\delta)$, $\delta>0$, niin valitsemalla 
$y_n=c$ ym.\ määritelmässä nähdään, että $f$ on jatkuva pisteessä $c$. Tasainen jatkuvuus 
on siis tässä mielessä vahvempi ominaisuus kuin Määritelmän \ref{funktion jatkuvuus} mukainen
\index{pisteittäinen jatkuvuus}%
jatkuvuus, jota myös tavataan sanoa \kor{pisteittäiseksi} (engl. pointwise).
\begin{Exa}
$f(x)=1/x$ on koko määrittelyjoukossaan (pisteittäin) jatkuva, mutta ei tasaisesti jatkuva: 
Esim.\ jos $x_n=1/n$ ja $t_n=2/n$, niin $x_n-t_n \kohti 0$, mutta 
$f(x_n)-f(t_n) = n/2 \not\kohti 0$.
\end{Exa}
Tasaisen jatkuvuuden merkitys näkyy selvemmin seuraavasta tuloksesta, jota voi myös pitää 
tasaisen jatkuvuuden vaihtoehtoisena määritelmänä, vrt. Lause \ref{approksimaatiolause}.
Todistus (joka sivuutetaan) on idealtaan sama kuin Lauseen
\ref{jatkuvuuskriteerien yhtäpitävyys} todistus.
\begin{*Lause} \label{tasaisen jatkuvuuden käsite}
Funktio $f:\DF_f\kohti\R$, $\DF_f\subset\R$, on joukossa $A\subset\DF_f$ tasaisesti jatkuva 
täsmälleen kun jokaisella $\eps>0$ on olemassa $\delta>0$ siten, että $\forall x_1,x_2\in\R$
pätee
\[
x_1,x_2\in A \ \ja \ \abs{x_1-x_2}<\delta \ \impl \ \abs{f(x_1)-f(x_2)}<\eps.
\]
\end{*Lause}
\begin{Exa} Suljetulla välillä $A=[a,b]$ Lipschitz-jatkuva funktio (Määritelmä
\ref{funktion l-jatkuvuus}) on ko.\ välillä myös tasaisesti jatkuva, sillä Lauseen
\ref{tasaisen jatkuvuuden käsite} ehto toteutuu jokaisella $\eps>0$, kun valitaan
$\delta=\eps/L$, missä $L=$ Lipschitz-vakio. \loppu
\end{Exa}
Seuraava lause --- joka jälleen nojaa Bolzanon--Weierstrassin lauseeseen --- kuuluu
tavanomaisen reaalianalyysin hämmästyttävimpiin tuloksiin.
\begin{*Lause} \label{kompaktissa joukossa jatkuva on tasaisesti jatkuva}
Jos reaalifunktio $f$ on jatkuva kompaktissa joukossa $K\subset\DF_f$, niin $f$ on
$K$:ssa tasaisesti jatkuva.
\end{*Lause}
\tod Tehdään vastaoletus: $f$ ei ole $K$:ssa tasaisesti jatkuva. Tällöin on olemassa jonot 
$\{x_n\}$ ja $\{t_n\}$ siten, että
\[
x_n,t_n\in K\ \forall n\,\ \ja\,\ x_n-t_n\kohti 0\,\ \ja\,\ f(x_n)-f(t_n) \not\kohti 0.
\]
Koska $f(x_n)-f(t_n) \not\kohti 0$, niin jonolla $\{f(x_n)-f(t_n)\}$ on Lauseen 
\ref{negaatioperiaate} mukaan osajono $\{f(x_k)-f(t_k)\}$ siten, että jollakin $\eps>0$ pätee
\[
\abs{f(x_k)-f(t_k)} \ge \eps \quad\forall k\in\N.
\]
Mutta jonot $\seq{x_k}$ ja $\seq{t_k}$ ovat rajoitettuja (koska $x_k,t_k \in K$ ja $K$ on
rajoitettu), joten ensinnäkin jonolla $\seq{x_k}$ on suppeneva osajono $\seq{x_l}$
(Lause \ref{B-W}). Kun nyt siirrytään tarkastelemaan jonon $\seq{t_k}$ vastaavaa osajonoa
$\seq{t_l}$, niin tällä on edelleen suppeneva osajono $\seq{t_\nu}$. Mutta tällöin
vastaava jonon $\seq{x_l}$ osajono $\seq{x_\nu}$ suppenee myös (koska on suppenevan jonon
osajono). Näin on löydetty lukuparien $(x_k,t_k)$ muodostamasta jonosta osajono
$\seq{(x_\nu,t_\nu)}$, jossa pätee:  
\[
x_\nu \kohti x\in \R\,\ \ja\,\ t_\nu \kohti t\in\R.
\]
Tässä on oltava $x,t \in K$, koska $K$ on suljettu, ja on myös oltava $x=t$, koska 
$x_n-t_n \kohti 0\ \impl\ x_\nu-t_\nu \kohti 0$. Koska siis $x_\nu \kohti x$ ja 
$t_\nu \kohti x$, $x \in K$, niin $f$:n jatkuvuuden perusteella
\[
f(x_\nu)-f(t_\nu) \kohti f(x)-f(x) = 0.
\]
On siis päätelty, että $f(x_\nu)-f(t_\nu) \kohti 0$, ja toisaalta 
$\abs{f(x_\nu)-f(t_\nu)}\ge\eps>0\ \forall \nu$. Tämä on  looginen ristiriita, joten
vastaoletus on väärä ja lause siis todistettu. \loppu

\Harj
\begin{enumerate}

\item
Mitä ominaisuuksista (avoin, suljettu, rajoitettu, kompakti) on seuraavilla 
$\R$:n osajoukoilla:
\newline
a) \ $A=(0,1) \cup \{\pi\}$ \newline
b) \ $A=[-1000,\,10)$ \newline
c) \ $A=[-1000,\,1000000000000000000000000]$ \newline
d) \ $A=(-0.00000000002,\,0.00000000000000007)$ \newline
e) \ $A=\N$ \newline
f) \ $A=\{\frac{1}{n} \mid n\in\N\}$ \newline
g) \ $A=(-\infty,\,3]$ \newline
h) \ $A=\{\frac{1}{n} \mid n\in\N\}\cup[-1,0]$

\item
Näytä, että jos $A\subset\R$, niin pätee
\[
A\ \text{kompakti} \qimpl \sup A=\max A\,\ \ja\ \inf A=\min A.
\]
Näytä vastaesimerkillä, että implikaatio ei päde kääntäen.

\item
Todista seuraavat Lauseiden \ref{avoin vs suljettu} ja \ref{unionit ja leikkaukset}
osaväittämät: \vspace{1mm}\newline
a) \ $A$ suljettu $\impl$ $\complement(A)$ avoin \newline
b) \ $A$ ja $B$ avoimia $\impl$ $A \cap B$ avoin \newline
c) \ $A$ ja $B$ suljettuja $\impl$ $A \cup B$ suljettu

\item
Määrittele seuraavien joukkojen reuna $\partial A\,$: \vspace{1mm}\newline
a) \ $A=\{1,3,7\} \quad$ 
b) \ $A=\{\frac{1}{n} \mid n\in\N\} \quad$ 
c) \ $A=(0,1)\cap\Q$

\item
Näytä, että jos $f$ on $\R$:ssä jatkuva funktio ja $B\subset\R$ on suljettu joukko, niin
myös $A=\{x\in\R \mid f(x) \in B\}$ on suljettu. Päättele, että erityisesti $f$:n
nollakohtien joukko on suljettu.

\item
Näytä, että yhtälö
\[
8-63y^5+90y^7-35y^9=x
\]
määrittelee välillä $[0,8]$ funktion $y=f(x)$ ja että $f:\ [0,8] \kohti [0,a]$ on jatkuva
bijektio eräällä --- millä? --- $a$:n arvolla.

\item \index{Hzz@Hölder-jatkuvuus}
Funktio $f$ on välillä $[a,b]$ \kor{Hölder--jatkuva indeksillä} $\alpha$,
$\alpha\in(0,1]\cap\Q$, jos jollakin $C\in\R_+$ pätee
\[
|f(x_1)-f(x_2)| \,\le\, C|x_1-x_2|^\alpha, \quad x_1,x_2\in[a,b].
\]
a) Näytä, että välillä $[a,b]$ Hölder--jatkuva funktio
on ko.\ välillä tasaisesti jatkuva. \ b) Näytä, että funktio $f(x)=\sqrt[n]{x},\ n\in\N$ on
välillä $[0,1]$ Hölder--jatkuva täsmälleen indekseillä $0 < \alpha \le 1/n$.

\item (*)
Olkoon $f$ jatkuva välillä $[0,1]$ ja muodostetaan lukujono $\seq{y_n}$ seuraavasti:
\[
y_n = \max_i\{f(i/2^n),\ i=0 \ldots 2^n\}, \quad n=0,1,\ldots
\]
Näytä, että $\seq{y_n}$ on monotonisesti kasvava lukujono ja että
\[
\max_{x\in[0,1]}f(x)=\lim_n y_n\,.
\]
Löytyykö myös $f$:n maksimikohta tällä tavoin (algoritmisesti)\,?

\end{enumerate} % *Jatkuvuuden logiikka

\chapter{Eksponenttifunktio}  \label{eksponenttifunktio}
\index{eksponenttifunktio|vahv}

Tässä luvussa määriteltävää \kor{eksponenttifunktiota} voi syystä pitää
\pain{f}y\pain{siikan} yleisimpänä funktiona. Eksponenttifunktio esiintyy mitä
moninaisimmissa matemaattisissa malleissa, jotka kuvaavat loputonta kasvua tai 
vähenemistä/vaimenemista. Fysiikan ohella tällaisia \kor{eksponentiaalisen kasvun} tai 
\kor{eksponentiaalisen vaimenemisen} malleja on paljon biologiassa, taloustieteissä, ym.

Matemaattisten funktioiden paljoudessakin eksponenttifunktio erottuu siinä määrin 
erikoislaatuisena, että se ansaitsee oman lukunsa. Jatkossa tarkastellaan tämän funktion
ja sen käänteisfunktion, \kor{logaritmifunktion}, matemaattista määrittelyä ja ominaisuuksia
ensin reaalifunktioina (Luvut \ref{yleinen eksponenttifunktio}--\ref{exp(x) ja ln(x)}). Sen
jälkeen laajennetaan eksponenttifunktio kompleksifunktioksi ja tarkastellaan laajennuksen
synnyttämiä johdannaisfunktioita sekä ---  hieman yllättäviä --- yhteyksiä trigonometrisiin
funktioihin (Luku \ref{kompleksinen eksponenttifunktio}). Viimeisessä osaluvussa käydään
lyhyesti läpi eräitä reaalisen eksponenttifunktion sovellusesimerkkejä fysiikassa. %Eksponenttifunktio
\section{Yleinen eksponenttifunktio $E(x)$} \label{yleinen eksponenttifunktio}
\sectionmark{Eksponenttifunktio $E(x)$}
\alku

\begin{Def} \label{reaalinen E(x)} \index{eksponenttifunktio!a@aksioomat|emph}
Reaalifunktio $E$ on \kor{eksponenttifunktio}, jos
\begin{itemize}
\item[(E1)] $E(x)$ on määritelty $\forall x\in\R$,
\item[(E2)] $E(0)\neq 0$,
\item[(E3)] $E$ on jatkuva pisteessä $x=0$,
\item[(E4)] $E(x+y)=E(x)E(y)\quad\forall x,y\in\R$.
\end{itemize}
\end{Def}
Eksponenttifunktion keskeisintä aksioomaa E4 voi sovellustilanteissa usein pitää 
jopa luonnonlakina. Myös aksiooman E3 taustalla voi nähdä fysikaalisia syitä,
vrt.\ sovellusesimerkit jäljempänä Luvussa \ref{eksponenttifunktio fysiikassa}.
\begin{Lause} \label{eksponenttifunktion ominaisuudet}
Jokaisella eksponenttifunktiolla $E$ on seuraavat ominaisuudet:
\begin{itemize}
\item[(a)] $E(x)>0\quad\forall x\in\R$.
\item[(b)] $E(0)=1$.
\item[(c)] $E$ on jatkuva koko $\R$:ssä.
\item[(d)] Jos $x\in\Q$, niin $E(x)$ määräytyy yksikäsitteisesti luvusta $E(1)=b$ ja
\[
\boxed{\kehys\quad E(x)=b^x\quad\forall x\in\Q. \quad}
\]
\item[(e)] $E$ on $\R$:ssa aidosti kasvava, kun $b=E(1)>1$, ja aidosti vähenevä, kun $b<1$.
           Jos $b=1$, on $E(x)=1\ \forall x\in\R$.
\end{itemize}
\end{Lause}
\tod (a) \ Koska $E(x)\cdot E(-x)=E(x-x)=E(0)\neq 0$ (E4,\,E2), on oltava 
$E(x)\neq 0 \ \forall x\in\R$. Toisaalta on myös
$E(x)=E(\tfrac{x}{2}+\tfrac{x}{2})=[E(\tfrac{x}{2})]^2 \ge 0$ (E4), joten on oltava
$E(x)>0 \ \forall x\in\R$.

(b) \ Koska $E(0)\neq 0$ (E2) ja $E(0)=E(0+0)=[E(0)]^2$ (E4), on oltava $E(0)=1$.

(c) \ Jos $x_n \kohti x\in\R$, niin $x_n-x \kohti 0\ \impl\ E(x_n-x) \kohti E(0)=1$
(E3,(b)). Aksiooman E4 perusteella pätee tällöin 
\[
E(x_n) = E(x)E(x_n-x) \kohti E(x)\cdot 1 = E(x).
\]
Koska tämä on tosi jokaiselle reaalilukujonolle, jolle $\lim_n x_n=x\in\R$, niin $E$ on
jatkuva $x$:ssä ja siis koko $\R$:ssä.

(d) \ Jos $E(1)=b$, niin aksiooman E4 mukaan
\begin{align*}
b&=E\left(\sum_{k=1}^n \frac{1}{n}\right)
                          = \left[E\left(\frac{1}{n}\right)\right]^n \quad \forall n\in\N \\
 &\impl \ E\left(\frac{1}{n}\right) = b^{1/n}\quad \forall n\in\N \\
 &\impl \ E\left(\frac{m}{n}\right)
            = E\left(\sum_{k=1}^m \frac{1}{n}\right) 
            = (b^{1/n})^m = b^{m/n}\quad \forall m,n\in\N.
\end{align*}
Koska $E(0)=1=b^{0/n}$ ja
\[
E\left(\frac{m}{n}\right)\cdot E\left(-\frac{m}{n}\right)=E(0)=1 \ 
   \impl \ E\left(-\frac{m}{n}\right) = \left[E\left(\frac{m}{n}\right)\right]^{-1} = b^{-m/n},
\]
niin on osoitettu:
\[
E\left(\frac{m}{n}\right) = b^{m/n} \quad \forall m\in\Z,\ n\in\N \qekv \text{väite (d)}.
\]

(e) \ Olkoon $b>1$ ja $x_1<x_2$. Oletetaan ensin, että $x_1,x_2\in\Q$, jolloin on
$x=x_2-x_1\in\Q$ ja $x>0$, joten $x=p/q$ jollakin $p,q\in\N$. Tällöin on väittämän (d) ja
murtopotenssien laskusääntöjen perusteella
\[
E(x_1)-E(x_2) \,=\, b^{x_1}-b^{x_2}  \,=\, b^{x_1}\left(1-b^x\right) 
                                    \,=\, b^{x_1}\left(1-\sqrt[q]{b^p}\right).
\]
Tässä on $\sqrt[q]{b^p}>1$, koska $b>1$, joten $E(x_1)-E(x_2)<0$. On päätelty, että jos
$b>1$, niin $\forall x_1,x_2\in\Q$ pätee: $x_1<x_2\ \impl\ E(x_1)<E(x_2)$.

Seuraavaksi olkoon $x_1,x_2\in\R$ ja $x_1<x_2$ (edelleen $b>1$). Valitaan $t_1,t_2\in\Q$
siten, että $x_1<t_1<t_2<x_2$ ja rationaalilukujonot $\seq{\alpha_n}$ ja $\seq{\beta_n}$
siten, että $\lim_n\alpha_n=x_1$, $\lim_n\beta_n=x_2$ ja lisäksi $\alpha_n<x_1\ \forall n$ ja
$\beta_n>x_2\ \forall n$. Tällöin koska $\alpha_n,t_1,t_2,\beta_n\in\Q$ ja
$\alpha_n<t_1<t_2<\beta_n$, niin aiemman päättelyn perusteella on
$E(\alpha_n)<E(t_1)<E(t_2)<E(\beta_n)\ \forall n$. Toisaalta koska $\alpha_n \kohti x_1$ ja
$\beta_n \kohti x_2$, niin väittämän (c) perusteella $E(\alpha_n) \kohti E(x_1)$ ja
$E(\beta_n) \kohti E(x_2)$. Koska tässä on $E(\alpha_n)<E(t_1)\ \forall n$ ja
$E(\beta_n)>E(t_2)\ \forall n$, niin seuraa $E(x_1) \le E(t_1)$ ja $E(x_2) \ge E(t_2)$
(Lause \ref{jonotuloksia} [V1]), joten on näytetty, että
\[
E(x_1) = \lim_n E(\alpha_n) \le E(t_1) < E(t_2) \le \lim_n E(\beta_n) = E(x_2).
\]
Siis $E(x_1)<E(x_2)$. Tämä perustui vain oletukseen, että $x_1,x_2\in\R$ ja $x_1<x_2$, joten
väittämä (e) on todistettu tapauksessa $b>1$. Tapauksessa $b<1$ on todistus vastaava.
Tapauksessa $b=1$ on väittämien (d) ja (c) mukaan $E(x)=1\ \forall x\in\Q$
$\impl\ E(x)=1\ \forall x\in\R$. \loppu

Lauseesta \ref{eksponenttifunktion ominaisuudet} on syytä huomauttaa, että siinä ei oteta
tarkemmin kantaa, millainen eksponenttifunktioiden joukko on. Tulee ainoastaan näytetyksi,
että funktio $E(x)=1\ \forall x\in\R$ kuuluu joukkoon, jolloin jää jopa se mahdollisuus, että
tämä on eksponenttifunktioista ainoa (!). Eksponenttifunktion olemassaolokysymys ehdolla
$E(1)=b \neq 1$ onkin oma ongelmansa, jonka ratkaiseminen edellyttää syvällisempiä
lukujonoteoreettisia tarkasteluja. Nämä tarkastelut esitetään luvun lopussa, jolloin tulee
todistetuksi
\begin{*Lause} \label{eksponenttifunktion olemassaolo} Aksioomat E1--E4 toteuttava
eksponenttifunktio $E(x)$ on olemassa ja määräytyy yksikäsitteisesti ehdosta $E(1)=b$
jokaisella $b\in\R_+$.
\end{*Lause}
Jatkossa käytetään eksponenttifunktiolle merkintää $E(x)=b^x$ myös kun $x\not\in\Q$. Tällöin
siis tarkoitetaan funktiota, joka toteuttaa aksioomien E1--E4 lisäksi ehdon $E(1)=b$.
Mainitun merkinnän myötä tulee määritellyksi myös yleinen potenssifunktio
$f(x)=x^\alpha,\ x>0,\ \alpha\in\R$, eli rajoituksesta $\alpha\in\Q$ voidaan luopua.
\begin{Exa} Luku $\pi^\pi$ on eksponenttifunktion $E(x)=\pi^x$ ja potenssifunktion
$f(x)=x^\pi$ yhteinen arvo $\pi$:ssä. Lauseen \ref{eksponenttifunktion ominaisuudet} väittämän
(c) perusteella $\pi^\pi$ on laskettavissa esimerkiksi raja-arvona $\pi^\pi=\lim_n a_n$, missä
\begin{align*}
\seq{a_n}\ &=\ \{\,\pi^3,\,\pi^{3.1},\,\pi^{3.14},\,\ldots\,\} \\
           &=\ \{\,\pi^3,\,\sqrt[10]{\pi^{31}}\,,\,\sqrt[100]{\pi^{314}}\,,\,\ldots\,\}.
\end{align*}
Algoritmina tämä on hitaanpuoleinen (tarkka arvo $\,36.46215960..\,$)\,:
\begin{align*}
a_0 = &31.00627668.. \\
a_1 = &34.76679088.. \\
a_2 = &36.39574388.. \\
a_3 = &36.43743103.. \\
a_4 = &36.45829251.. \\
a_5 = &36.46204884.. \quad \loppu
\end{align*}
\end{Exa}

Eksponenttifunktion $b^x$ keskeiset laskusäännöt ovat
\[
\boxed{\kehys\quad b^xb^y=b^{x+y}, \quad a^xb^x=(ab)^x, \quad(b^x)^y=b^{xy},
                                                       \quad a,b\in\R_+,\ x,y\in\R. \quad}
\]
Näistä säännöistä ensimmäinen on perusaksiooma E4, joka yleistää murtopotensseille tutun
laskusäännön $b^xb^y=b^{x+y}\ \forall x,y\in\Q$. Muutkin säännöt voi tulkita vastaavien
murtopotenssien laskusääntöjen (vrt.\ Luku \ref{kunta}) yleistyksiksi. Näiden perustelu
jätetään harjoitustehtäviksi (Harj.teht.\,\ref{H-exp-1: sääntö 1}--\ref{H-exp-1: sääntö 2}).

Koska eksponenttifunktio $E(x)$ on jatkuva ja aidosti monotoninen, kun $E(1)=b\neq 1$, niin 
Ensimmäisestä väliarvolauseesta (Lause \ref{ensimmäinen väliarvolause}) ja ilmeisistä 
raja-arvotuloksista
\[
\lim_{x\kohti\infty} b^x=\begin{cases}
\infty, &\text{jos } b>1, \\
0, &\text{jos } b<1
\end{cases}
\]
on pääteltävissä:
\[ 
\boxed{\kehys\quad E(x)=b^x\ \ \text{on bijektio}\ \ E:\R\kohti\R_+ 
             \quad \text{jokaisella}\ b\in\R_+,\ b \neq 1. \quad}
\]
\begin{figure}[H]
\setlength{\unitlength}{1cm}
\begin{center}
\begin{picture}(8,6)(-4,-1)
\put(-4,0){\vector(1,0){8}} \put(3.8,-0.4){$x$}
\put(0,-1){\vector(0,1){6}} \put(0.2,4.8){$y$}
\curve(
   -2.0000,    0.1353,
   -1.5000,    0.2231,
   -1.0000,    0.3679,
   -0.5000,    0.6065,
         0,    1.0000,
    0.5000,    1.6487,
    1.0000,    2.7183,
    1.5000,    4.4817)
\curve(
   -2.0000,    4.0000,
   -1.5000,    2.8284,
   -1.0000,    2.0000,
   -0.5000,    1.4142,
         0,    1.0000,
    0.5000,    0.7071,
    1.0000,    0.5000,
    1.5000,    0.3536,
    2.0000,    0.2500,
    2.5000,    0.1768)  
\put(1.5,4){$y=b^x$, $b>1$}
\put(1.5,0.5){$y=b^x$, $b<1$}
\multiput(-1,0)(2,0){2}{\line(0,-1){0.1}}
\put(0,1){\line(-1,0){0.1}}
\put(-1.3,-0.4){$\scriptstyle{-1}$} \put(0.93,-0.4){$\scriptstyle{1}$}
\put(-0.4,0.9){$\scriptstyle{1}$}
\end{picture}
\end{center}
\end{figure}

\subsection{*Eksponenttifunktion konstruktio}

Jatkossa esitettävä Lauseen \ref{eksponenttifunktion olemassaolo} todistus on
konstruktiivinen, ts.\ todistuksessa konstruoidaan luku $E(x)\in\R$
jokaisella $x\in\R$ niin, että aksioomat E1--E4 sekä lisäehto $E(1)=b\in\R_+$ toteutuvat.
Koska Lauseen \ref{eksponenttifunktion ominaisuudet} perusteella jo tiedetään, että
$E(x)=b^x\ \forall x\in\Q$, niin riittää määritellä $E(x)$ myös irrationaalisilla $x$:n
arvoilla ja varmistaa, että näin määritellylle funktiolle aksioomat E3 ja E4 ovat voimassa.
Todistetaan ensin aputulos.
\begin{Lem} \label{exp-aputulos} Jos $b\in\R_+$, niin funktio $F(x)=b^x$, $\DF_F=\Q$, on
jatkuva pisteessä $x=0$, ts.\ jokaiselle rationaalilukujonolle $\seq{x_n}$ pätee:
$x_n \kohti 0\ \impl\ F(x_n) \kohti 1$.
\end{Lem}
\tod Jos $\seq{x_n}$ on rationaalilukujono, jolle pätee $x_n \kohti 0$, niin
$\forall k\in\N$ on olemassa indeksi $N_k\in\N$ siten, että $\abs{x_n}<1/k\ \forall n>N_k$ 
(Määritelmä \ref{jonon raja}, $\eps=1/k$). Koska funktio $F(x)=b^x$ on monotoninen $\Q$:ssa
(jopa aidosti, jos $b \neq 1$, ks.\ Lauseen \ref{eksponenttifunktion ominaisuudet} väittämän
(c) todistus), niin tällöin pätee
\[
\abs{b^{x_n}-1}\, \le \,\abs{b^{1/k}-1}, \quad \text{kun}\ n>N_k.
\]
Tässä $b^{1/k} \kohti 1$, kun $k \kohti \infty$ (Propositio \ref{juurilemma}), joten
$\forall \eps>0$ on olemassa indeksi $m$ siten, että $\abs{b^{1/m}-1}<\eps$. Valitsemalla
$k=m$ seuraa siis
\[
\abs{b^{x_n}-1} < \eps, \quad \text{kun}\ n>N_m\,.
\]
Tässä $\eps>0$ on mielivaltainen ja $N_m\in\N$, joten lukujonon suppenemisen määritelmän
mukaan $\,b^{x_n} \kohti 1$. \loppu

\vahv{Konstruktio}. \ Halutaan määritellä $E(x)$, kun $x\in\R,\ x\not\in\Q$. Koska tiedetään,
että $E$ (sikäli kuin olemasa) on jatkuva $\R$:ssä
(Lause \ref{eksponenttifunktion ominaisuudet} (c)), niin määrittelyn perustaksi voidaan ottaa
jatkaminen (vrt.\ Luku \ref{funktion raja-arvo})\,: Lähtien funktion $b^x$ alkuperäisestä
määrittelyjoukosta $\Q$ laajennetaan määrittelyjoukko $\R$:ksi jatkuvuuden perusteella.
Tämä merkitsee, että jokaiselle rationaalilukujonolle $\seq{x_n}$ on oltava voimassa
\begin{equation} \label{exp-jatko}
x_n \kohti x\in\R \qimpl \lim_n b^{x_n} \,=\, \lim_n E(x_n) \,=\, E(x).
\end{equation}
Jotta tämä ehto voisi toimia $E(x)$:n määritelmänä, on ensinnäkin varmistettava, että 
$\forall x\in\R$ ja jokaiselle rationaalilukujonolle $\seq{x_n}$ pätee
\begin{equation} \label{exp-väite 1}
x_n \kohti x \qimpl \text{$\seq{E(x_n)}\,$ on Cauchy}.
\end{equation}
Sikäli kuin tämä on tosi, niin Cauchyn suppenemiskriteerin (Lause \ref{Cauchyn kriteeri})
mukaan $E(x_n) \kohti y\in\R$. Tämä ei kuitenkaan vielä todista funktioriippuvuutta
$x \map y$, sillä on mahdollista, että $y$ riippuu jonosta $\seq{x_n}$ eikä vain sen
raja-arvosta $x$. Siksi on vielä näytettävä, että jokaisella $x\in\R$ ja kaikille
rationaalilukujonoille $\seq{x_n}$ ja $\seq{x_n'}$ pätee
\begin{equation} \label{exp-väite 2}
x_n,x_n'\in\Q\,\ \ja\ x_n \kohti x\ \ja\ x_n' \kohti x \qimpl E(x_n)-E(x_n') \kohti 0.
\end{equation}
Jos ehdot \eqref{exp-väite 1} ja \eqref{exp-väite 2} toteutuvat kaikille
rationaalilukujonoille $\seq{x_n}$ ja $\seq{x_n'}$, niin on näytetty, että jokaisella
$x\in\R$ (myös kun $x\in\Q$) on olemassa yksikäsitteinen $y\in\R$ siten, että jokaiselle
rationaalilukujonolle $\seq{x_n}$ pätee
\[
x_n \kohti x\in\R \qimpl E(x_n) \kohti y.
\]
Tällöin $y$ riippuu vain raja-arvosta $x$, jolloin riippuvuus $x \map y$ voidaan ilmaista
funktiosymbolilla $E$, eli  voidaan kirjoittaa $y=E(x)$ jatkamisperiaatteen \eqref{exp-jatko}
mukaisesti.

Jatkamisen perustana olevat väittämät \eqref{exp-väite 1} ja \eqref{exp-väite 2}
saadaan todistetuksi vedoten jo todettuun aksiooman E4 pätevyyteen murtopotenssien
laskusääntönä (eli rationaalisilla $x$:n ja $y$:n arvoilla) ja Lemmaan \ref{exp-aputulos}.
Ensinnäkin jos $\seq{x_n}$ on rationaalilukujono, niin (E4):n mukaan
\[
E(x_n)-E(x_m) = E(x_m)[E(x_n-x_m)-1].
\]
Jos tässä $x_n \kohti x\in\R$, niin $\seq{x_n}$ on Cauchy, joten $x_n-x_m \kohti 0$, kun 
$n,m \kohti \infty$. Tällöin myös $E(x_n-x_m) \kohti 1$ (Lemma \ref{exp-aputulos}). Koska
$\seq{E(x_n)}=\seq{b^{x_n}}$ on epäilemättä rajoitettu jono, niin päätellään
(ks.\ Lause \ref{jonotuloksia} [V3])
\begin{align*}
n,m \kohti \infty &\qimpl [E(x_n-x_m)-1]\ \kohti\ 0 \\
                  &\qimpl E(x_m)[E(x_n-x_m)-1]\ =\ E(x_n)-E(x_m)\ \kohti\ 0,
\end{align*}
mikä todistaa väittämän \eqref{exp-väite 1}. Väittämä \eqref{exp-väite 2} näytetään toteen 
vastaavalla tavalla lähtien hajotelmasta
\[
E(x_n)-E(x_n') = E(x_n')[E(x_n-x_n')-1].
\]   
Funktio $E(x)$ on näin määritelty yksikäsitteisesti koko $\R$:ssä tunnetun funktion 
$b^x,\ x\in\Q$, $b\in\R_+$ jatkona, jolloin myös ehto $E(1)=b$ toteutuu. Vielä on osoitettava,
että näin määritelty $E$ on eksponenttifunktio.

\vahv{Aksiooma E3}. Jatkuvuuden määritelmän mukaisesti on näytettävä, että jokaiselle
reaalilukujonolle $\seq{x_n}$ pätee
\[
x_n \kohti 0 \qimpl E(x_n) \kohti E(0)=1.
\]
Olkoon siis $\seq{x_n}$ reaalilukujono, jolle $\lim_n x_n=0$. Valitaan rationaalilukujono
$\seq{\alpha_n}$ siten, että $\lim_n\alpha_n=0$ ja lisäksi jokaisella $n$ on joko (i)
$0 \le x_n \le \alpha_n$ tai (ii) $\alpha_n \le x_n \le 0$. Lauseen 
\ref{eksponenttifunktion ominaisuudet} väittämän (e) mukaan edellä konstruoitu funktio $E$ on
$\R$:ssä monotoninen, sillä todistus perustui vain funktion $F(x)=b^x,\ x\in\Q\,$
ominaisuuksiin sekä jatkuvuusehtoon \eqref{exp-jatko}. Oletuksien (i) tai (ii) ja
monotonisuuden perusteella
\[
\abs{E(x_n)-1} \le \abs{E(\alpha_n)-1}\,\ \forall n.
\]
Tässä on $\alpha_n\in\Q\ \forall n$ ja $\alpha_n \kohti 0$, joten $E(\alpha_n) \kohti 1$
(Lemma \ref{exp-aputulos}). Siis myös $E(x_n) \kohti 1$, eli $E$ on jatkuva pisteessä $x=0$.

\vahv{Aksiooma E4}. \ Olkoon $x,y\in\R$ ja $\seq{x_n},\seq{y_n}$ rationaalilukujonoja,
joille pätee $x_n \kohti x$ ja $y_n \kohti y$. Tällöin jokaisella $n$ pätee
$E(x_n+y_n)=E(x_n)E(y_n)$, joten $E$:n määritelmän \eqref{exp-jatko} ja lukujonojen
raja-arvojen yhdistelysääntöjen perusteella
\begin{align*}
E(x+y)-E(x)E(y) &= \lim_n E(x_n+y_n) - \lim_n E(x_n) \cdot \lim_n E(y_n) \\
                &= \lim_n [E(x_n+y_n)-E(x_n)E(y_n)] \\
                &= \lim_n 0 = 0. \ \loppu
\end{align*}

\Harj
\begin{enumerate}

\item
Tarkista, mitkä eksponenttifunktion aksioomat toteuttaa 
\begin{align*}
&\text{a)}\ \ E(x)=0,\,\ x\in\R, \\
&\text{b)}\ \ E(x)=\begin{cases} 
                   b^x, &\text{jos}\ x\in\Q, \\ 0, &\text{jos}\ x\in\R,\ x\not\in\Q. 
                   \end{cases}
\end{align*}
Mikä tunnettu funktio sisältyy jälkimmäiseen joukkoon?

\item
Olkoon $\alpha\in\R$. Näytä eksponenttifunktion ominaisuuksiin vedoten, että funktio
$f(x)=x^\alpha$ on välillä $(0,\infty)$ aidosti kasvava, jos $\alpha>0$, ja aidosti vähenevä,
jos $\alpha<0$.

\item
Näytä, että approksimaation
\[
\pi^e \approx \sqrt[100]{\pi^{271}}\,=\,a
\]
suhteelliselle virheelle pätee arvio
\[
0\,<\,\frac{\pi^e-a}{\pi^e}\,<\,\sqrt[100]{\pi}-1.
\]

\item \label{H-exp-1: sääntö 1}
Perustele eksponenttifunktion jatkuvuuteen vedoten laskusääntö $a^xb^x=(ab)^x$,
$x\in\R,\ a,b\in\R_+$. Oletetaan sääntö tunnetuksi, kun $x\in\Q$.

\item (*) \label{H-exp-1: sääntö 2}
Halutaan todistaa laskusääntö $(b^x)^y=b^{xy}\ (b\in\R_+\,,\ x,y\in\R)$ eli sääntö
\[
E(x)^y=E(xy), \quad x,y\in\R,
\]
kun $E(x)=b^x$. \ a) Näytä, että jos $y\in\Q$, niin sääntö seuraa aksioomasta E4. \
b) Näytä eksponenttifunktion jatkuvuuteen vedoten, että sääntö on pätevä myös, kun 
$y\in\R,\ y\not\in\Q$.

\end{enumerate} 
 %Yleinen eksponenttifunktio $E(x)$
\section{Funktiot $e^x$ ja $\ln x$} \label{exp(x) ja ln(x)}
\alku
\index{funktio C!f@$\exp$ ($e^x$, $e^z$)|vahv}

Edellisessä luvussa eksponenttifunktio $E(x)$ konstruoitiin jatkamalla tunnettu funktio
$\,b^x,\ x\in\Q$. Tällä tavoin tuli varmistetuksi, että eksponenttifunktioita on olemassa,
että luku $b=E(1)$ määrää $E(x)$:n yksikäsitteisesti, ja että eksponenttifunktion arvot
voidaan laskea numeerisesti suoraan määritelmästä käsin. Mitä \pain{ei} tullut varmistetuksi
on esimerkiksi, onko $E(x)$ mahdollisesti myös derivoituva, eikä vain jatkuva. Yleisemminkin
jäi avoimeksi, kuinka säännöllisestä funktiosta on ylipäänsä kyse. Seuraavassa etsitään
vastausta näihin kysymyksiin lähestymällä eksponenttifunktiota toisesta suunnasta.

Lähtökohdaksi otetaan eksponenttifunktion aksiooman E3 (ks.\ edellinen luku) vahvistaminen
muotoon \index{eksponenttifunktio!a@aksioomat}
\begin{itemize}
\item[(E3')]  $E$ on derivoituva pisteessä $x=0$.
\end{itemize}
Koska tiedetään, että $E(0)=1$ (Lause \ref{eksponenttifunktion ominaisuudet} (b)), niin
aksiooman E3' mukaan on olemassa luku $a\in\R$ siten, että
\[
E'(0) \,=\, \lim_{x \kohti 0} \frac{E(x)-1}{x} \,=\, a.
\]
Kun yhdistetään tämä tulos ja aksiooma E4, niin seuraa
\[
\lim_{\Delta x \kohti 0} \frac{E(x+\Delta x)-E(x)}{\Delta x}
 \,=\, \lim_{\Delta x \kohti 0} E(x)\,\frac{E(\Delta x)-1}{\Delta x} \,=\, aE(x), \quad x\in\R.
\]
Siis $E$ on derivoituva jokaisessa pisteessä $x\in\R$ ja
\index{derivoimissäännöt!g@eksponenttifunktio}%
\[
E'(x)=aE(x), \quad x\in\R.
\]
On päätelty, että jos aksioomat E1,\,E2,\,E3',\,E4 toteuttava funktio $E(x)$ on
olemassa, niin tämä on derivoituva $\R$:ssä ja jollakin $a\in\R$ ratkaisu probleemalle
\[ 
\text{(P)} \quad \begin{cases} \,y'=ay, \quad x\in\R, \\ \,y(0)=1. \end{cases}
\]
Tässä on kyse alkuarvotehtävästä, jossa on ratkaistava differentiaaliyhtälö
$y'=ay$ alkuehdolla $y(0)=1$ (vrt.\ Luvussa \ref{väliarvolause 2} tarkasteltu
vastaava probleema differentiaaliyhtälölle $y'=f(x)$).
\index{differentiaaliyhtälö!a@eksponenttifunktion}%
\begin{Lause} \label{exp-dy} Probleemalla (P) on ratkaisu $y(x)=E(x)$, missä
\[ 
E(x) = \sum_{k=0}^\infty \frac{(ax)^k}{k!} = 1 + ax + \frac{(ax)^2}{2} + \ldots 
\]
Tämä ratkaisu on yksikäsitteinen, ja $E(x)$ on eksponenttifunktio.
\end{Lause}
\tod Potenssisarjan derivoimissäännöstä (Lause \ref{potenssisarja on derivoituva}) nähdään
välittömästi, että väitetty $y(x)$ on probleeman (P) ratkaisu. Nähdään myös, että funktiolla
$E(x)$ on eksponenttifunktiolta vaadittavat ominaisuudet (E1)--(E3), sillä $E(0)=1$ ja $E$
on määritelty koko $\R$:ssä suppenevan potenssisarjan summana, siis derivoituvana ja jopa
sileänä funktiona.

Näytetään seuraavaksi, että $E(x)$ toteuttaa myös eksponenttifunktion aksiooman (E4). Tätä
silmällä pitäen tarkastellaan ensin funktiota
\[
u(x) = E(x)E(-x), \quad x\in\R.
\]
Koska $E'=aE$, niin tulon derivoimissääntöä käyttäen todetaan, että
\begin{equation} \label{exp-apuy 1}
u'(x)=0, \quad x\in\R.
\end{equation}
Koska $u(0)=1$ ja siis $u'=0$, niin seuraa, että $u(x)=1\ \forall x\in\R$
(Lause \ref{Integraalilaskun peruslause}). Siis $E(x)E(-x)=1\ \forall x\in\R$, ja näin ollen 
$E(x) \neq 0\ \forall x\in\R$ ja
\begin{equation} \label{exp-apuy 2}
E(-x) = [E(x)]^{-1}, \quad x\in\R.
\end{equation}
Olkoon seuraavaksi $t\in\R$ kiinnitetty ja tarkastellaan funktiota
\[
u(x) = E(t+x)E(-t)E(-x), \quad x\in\R.
\]
Derivoimalla muuttujan $x$ suhteen ja käyttämällä differentiaaliyhtälöä $E'=aE$ todetaan, että 
jälleen pätee \eqref{exp-apuy 1}. Koska tuloksen \eqref{exp-apuy 2} mukaan on $u(0)=1$, niin 
päätellään kuten edellä ja käyttämällä uudelleen tulosta \eqref{exp-apuy 2}, että
\[
u(x)=E(t+x)E(-t)E(-x)=1\,\ \ekv\,\ E(t+x)= E(t)E(x), \quad x\in\R.
\]
Tämä pätee jokaisella $t\in\R$, joten aksiooma E4 on voimassa funktiolle $E(x)$. Siis 
$E(x)$ on eksponenttifunktio (jokaisella $a\in\R$).

Enää on todistamatta probleeman (P) ratkaisun yksikäsitteisyys. Tätä silmällä pitäen
olkoon $y(x)$ (P):n (mikä tahansa) ratkaisu ja tarkastellaan funktiota
\[
u(x)=y(x)E(-x), \quad x\in\R.
\]
Derivoimalla ja käyttämällä differentiaaliyhtälöitä $y'=ay$ ja $E'=aE$ todetaan, että jälleen 
pätee \eqref{exp-apuy 1}. Koska $u(0)=1$, niin seuraa
\[
u(x)=y(x)E(-x)=1\,\ \ekv\,\ y(x)=E(x), \quad x\in\R. \loppu
\]

Sanotaan jatkossa \kor{peruseksponenttifunktioksi}, symboli $\,\exp$, probleeman (P) ratkaisua,
kun $a=1$. Määritelmä potenssisarjana on siis
\[
\boxed{\quad \exp(x) = \sum_{k=0}^\infty \frac{x^{\ygehys k}}{\agehys k!}\,. \quad}
\]
Jos merkitään $e=\exp(1)$, niin edellisen luvun merkinnöin voidaan kirjoittaa
\[
\boxed{\quad\kehys \exp(x) = e^x. \quad}
\]
Tässä $e$ on todellakin Neperin luku, sillä pätee (ks.\ Propositiot \ref{Neperin jonot} ja
\ref{e:n sarja})
\[
\exp(1) = \sum_{k=0}^\infty \frac{1}{k!} = \lim_{n\kohti\infty}\left(1+\frac{1}{n}\right)^n
                                         = e = 2.71828182845905..
\]
Tässä jälkimmäinen raja-arvotulos on erikoistapaus yleisemmästä tuloksesta
(ks.\ Harj.teht.\,\ref{H-exp-2: eksoottisia raja-arvoja})
\[
\boxed{\Akehys\quad \lim_n \Bigl(1+\frac{\ygehys x}{n}\Bigr)^n = e^x, \quad x\in\R. \quad }
\]

Peruseksponenttifunktion $e^x$ määritelmästä seuraa derivoimissääntö
\[
\boxed{\kehys\quad \dif e^x = e^x. \quad}
\]
Monet funktion $e^x$ ominaisuuksista voidaan helposti johtaa tästä säännöstä tai 
potenssisarjaesityksestä. Esimerkiksi potenssisarjaesityksestä seuraa 
\[
e^x > x^m/m!\ \ \forall m \in \N, \quad \text{kun}\ x>0, 
\]
mistä on edelleen helposti pääteltävissä raja-arvotulokset
\begin{equation} \label{exp-raja-arvot}
\boxed{\quad \lim_{x\kohti\infty} x^{\ygehys -\alpha}e^x = \infty 
  \qekv \lim_{x\kohti\infty}x^\alpha e^{-x} = 0 \quad \forall \alpha \in \R_{\agehys}. \quad}
\end{equation}
Näiden mukaan funktio $e^x$ kasvaa nopeammin, ja funktio $e^{-x}$ vähenee nopeammin, kuin mikään
potenssifunktio, kun $x\kohti\infty$. Potenssisarjan perusteella voidaan myös $e^x$:n 
numeroarvoja laskea helposti, sillä pienillä $\abs{x}$:n arvoilla (esim.\ kun $\abs{x} \le 1$) 
sarja suppenee nopeasti ja suuremmilla taas voidaan käyttää kaavaa $e^x = (e^{x/n})^n$. 
Derivoimissäännöstä (myös potenssisarjasta) nähdään, että funktio $e^x$ on $\R$:ssä 
mielivaltaisen monta kertaa derivoituva (sileä) funktio. 
\begin{Exa} Jatka funktio $f(x)=(e^x-1-x)/x^2$ pisteeseen $x=0$.
\end{Exa}
\ratk $e^x$:n potenssisarjasta nähdään, että
\[
f(x)= \frac{1}{2!} + \frac{x}{3!} + \frac{x^2}{4!} + \ldots\ 
                           =\ \sum_{k=0}^\infty \frac{x^k}{(k+2)!}\,, \quad x \neq 0.
\]
Kun asetetaan $f(0)=1/2$, niin $\forall x\in\R$ pätee siis
\[
f(x)\,=\,\sum_{k=0}^\infty \frac{x^k}{(k+2)!} 
    \,=\,\begin{cases}
         \,\dfrac{e^x-1-x}{x^2}\,, &\text{kun}\ x \neq 0, \\ \tfrac{1}{2}\,, &\text{kun}\ x=0.
         \end{cases} 
\]
Tässä potenssisarjan suppenemissäde on $\rho=\infty$, joten $f$ ei ole vain jatkuva, vaan
$\R$:ssä (myös pisteessä $x=0$\,!) mielivaltaisen monta kertaa derivoituva. \loppu

Edellisessä luvussa todettiin, että yleinen eksponenttifunktio on muotoa $E(x)=b^x$, missä 
$b=E(1) \in \R_+$. Laskuissa usein kätevämpi on peruseksponenttifunktiosta johdettu 
vaihtoehtoinen esitystapa
\[
\boxed{\kehys\quad E(x)=e^{ax}, \quad E(1)=b=e^a. \quad}
\]
Luku $a\in\R$ määräytyy yksikäsitteisesti yhtälöstä $e^a=b$, koska $e^x:\ \R\kohti\R_+$ on 
bijektio. Jokainen eksponenttifunktio on siis muotoa $E(x)=e^{ax}$ jollakin $a\in\R$. 

Todettakoon vielä lähtökohtana olleesta probleemasta (P), että sen ratkaisuina saadaan siis 
kaikki eksponenttifunktiot $y(x)=e^{ax}$, kun $a$:n arvoa vaihdellaan. Jos ratkaisuksi
halutaan tietty eksponenttifunktio $E(x)$, niin $a$ määräytyy joko em.\ tavalla luvun
$b=E(1)$ kautta tai ehdosta $E'(0)=a$. Jos $a$ on kiinnitetty, niin differentiaaliyhtälön
$y'=ay$ y\pain{leinen} \pain{ratkaisu} (=kaikkien ratkaisujen joukko) on
(Harj.teht.\,\ref{H-exp-2: exp-dy})
\[
y(x)=Ce^{ax}, \quad C\in\R.
\]
Alkuehdon $y(0)=y_0$ toteuttava ratkaisu on tämän mukaan yksikäsitteinen ja saadaan
valitsemalla $C=y_0$.

\subsection{Logaritmifunktio}
\index{logaritmifunktio|vahv}
\index{funktio C!g@$\log$, $\ln$|vahv}

Koska eksponenttifunktio $E:\R\kohti\R_+$ on bijektio kun $E(1)=b\neq 1$, on sillä 
käänteisfunktio, jota sanotaan \kor{logaritmifunktioksi}. Käänteisfunktiota merkitään
symbolilla $\log_b\,$:
\[
y=b^x \ \ekv \ x=\log_b y.
\]
\index{luonnollinen logaritmi}%
Peruseksponenttifunktion $e^x$ käänteisfunktiota sanotaan \kor{luonnolliseksi logaritmiksi}, 
symbolina $\ln$ (toisinaan $\log$):
\[
\boxed{\kehys\quad y=e^x \ \ekv \ x=\ln y. \quad}
\]
Luonnollinen logaritmi, samoin kuin muutkin logaritmifunktiot, on koko määrittelyjoukossaan 
($=\R_+$) jatkuva ja aidosti kasvava, ja $\ln x \kohti \infty$ kun $x\kohti\infty$ ja 
$\ln x \kohti -\infty$ kun $x\kohti 0^+$. Sijoituksilla $x = \pm \ln t\ \ekv\ t = e^{\pm x}$ 
voidaan raja-arvotuloksista \eqref{exp-raja-arvot} päätellä, että rajoilla $x\kohti\infty$ ja
$x\kohti 0^+$ funktio $\abs{\ln x}$ kasvaa hitaammin kuin mikään potenssifunktio:
\begin{equation} \label{ln-raja-arvot}
\boxed{\quad \lim_{\akehys x\kohti\infty}x^{\ykehys -\alpha}\ln x\ =
             \lim_{x\kohti 0^+}x^\alpha \ln x = 0 \quad \forall \alpha\in\R_+. \quad}
\end{equation}
\begin{figure}[H]
\setlength{\unitlength}{1cm}
\begin{center}
\begin{picture}(8,4)(-1,-1)
\put(-1,0){\vector(1,0){8}} \put(6.8,-0.4){$x$}
\put(0,-2){\vector(0,1){5}} \put(0.2,2.8){$y$}
\put(1,0){\line(0,-1){0.1}}
\put(0,1){\line(-1,0){0.1}}
\put(0.93,-0.4){$\scriptstyle{1}$}
\put(-0.4,0.9){$\scriptstyle{1}$}
\put(5,1.3){$\scriptstyle{y=\ln x}$}
\curve(
    0.2500,   -1.3863,
    0.7500,   -0.2877,
    1.2500,    0.2231,
    1.7500,    0.5596,
    2.2500,    0.8109,
    2.7500,    1.0116,
    3.2500,    1.1787,
    3.7500,    1.3218,
    4.2500,    1.4469,
    4.7500,    1.5581,
    5.2500,    1.6582,
    5.7500,    1.7492,
    6.2500,    1.8326,
    6.7500,    1.9095)
\end{picture}
%\caption{$y=\ln x$}
\end{center}
\end{figure}
Eksponentti- ja logaritmifunktioilla laskettaessa käytetään yleisimmin 
peruseksponenttifunktiota $e^x$ ja sen käänteisfunktiota $\ln x$. Useimmin tarvittavat, suoraan
määritelmästä ja eksponenttifunktion ominaisuuksista seuraavat kaavat ovat
(Harj.teht.\,\ref{H-exp-2: kaavat})\,:
\begin{alignat}{3}
e^{\ln x} &= x,               &\quad &x\in\R_+.            \label{exp-kaava 1}\\
\ln(xy)   &= \ln x + \ln y,   &\quad &x,y\in\R_+.         \label{exp-kaava 2}\\
\ln (1/x) &= -\ln x,          &\quad &x\in\R_+.           \label{exp-kaava 3}\\
\ln x^y   &= y\ln x,          &\quad &x\in\R_+, \ y\in\R. \label{exp-kaava 4}
\end{alignat}
Kaavat \eqref{exp-kaava 1} ja \eqref{exp-kaava 4} yhdistämällä saadaan laskukaava
\begin{equation} \label{exp-kaava 5}
\boxed{\kehys\rule{0mm}{4.5mm}\kehys\quad x^y=e^{y\ln x}, \quad x\in\R_+\,,\ y\in\R. \quad}
\end{equation}
Kaavasta \eqref{exp-kaava 1}, eksponenttifunktion derivoimissäännöstä ja yhdistetyn funktion 
derivoimissäännöstä saadaan johdetuksi logaritmifunktion $\ln x$ derivoimissääntö:
\index{derivoimissäännöt!h@logaritmifunktio}% 
\[ 
1=\dif x = \dif e^{\ln x} = e^{\ln x} \dif \ln x 
         = x \dif \ln x \qimpl \dif \ln x = 1/x, \quad x>0. 
\]
Tämän perusteella on $\dif \ln(-x) = (-1)/(-x) = 1/x,\ x<0$, joten saadaan yleisempi sääntö
\begin{equation} \label{exp-kaava 6}
\boxed{\kehys\quad \dif \ln \abs{x} = \dfrac{1}{x}\,, \quad x \neq 0. \quad} 
\end{equation}
\begin{Exa} Kaavan \eqref{exp-kaava 5} ja säännön \eqref{exp-kaava 6} perusteella
\[ 
\dif x^x = \dif e^{x\ln x} = e^{x\ln x}\dif(x\ln x) = (\ln x + 1)\,x^x, \quad x>0.
\]
Tuloksesta nähdään, että $f(x)=x^x$ on aidosti vähenevä välillä $(0,1/e]$ ja aidosti kasvava
välillä $[1/e,\infty)$. Siis $f$:n absoluuttinen minimiarvo on $f_{min}=f(1/e)=e^{-1/e}$.
Rajalla $x \kohti 0^+$ on $f(0^+)=1$ (Harj.teht.\,\ref{H-exp-2: raja-arvoja}e). \loppu
\end{Exa}
\begin{multicols}{2} \raggedcolumns
\begin{Exa} Matemaatikon mökki on kahden tien risteyksessä. Selvitä mökin sijainti, kun 
tiedetään, että kummallakin tiellä on oheinen viitta. 
\vspace{1cm}
\[ 
\boxed{ \rule[-2cm]{0mm}{2.5cm} } \boxed{ \quad x^y = y^x \quad } 
\]
\end{Exa}
\end{multicols}
\ratk Tie \,1\, on ilmeisesti puolisuora $y=x$, $x>0$. Tiellä \,2\, taas ovat esim.\ pisteet 
$(2,4)$ ja $(4,2)$. Teiden leikkauspisteen selvittämiseksi käytetään ensin kaavoja 
\eqref{exp-kaava 1} ja \eqref{exp-kaava 4} (ol.\ $x,y\in\R_+$):
\[
x^y=y^x\ \ekv\ e^{y\ln x} = e^{x\ln y}\ \ekv\ y\ln x = x\ln y\ 
                                        \ekv\ \ln y/y = \ln x/x.
\]
Säännöstä \eqref{exp-kaava 6} ja Luvun \ref{derivaatta} derivoimissäännöistä seuraa
\[ 
\dif \ln x/x = x^{-2}(1 - \ln x), \quad x>0, 
\]
joten funktio $f(x)=\ln x/x$ on aidosti kasvava välillä $(0,e]$ ja aidosti vähenevä välillä 
$[e,\infty)$. Näin ollen, jos $x\in\R_+$ ja $x>1$, $x\neq e$, niin yhtälöllä $f(y)=f(x)$ on 
kaksi ratkaisua $y$, joista toinen on $y=x$ (ks.\ kuvio). Kun $x=e$ tai $x \le 1$, on ainoa 
ratkaisu $y=x$. Päätellään, että mökki on pisteessä $(e,e)$. \loppu
\begin{multicols}{2} \raggedcolumns
\begin{figure}[H]
\setlength{\unitlength}{1cm}
\begin{center}
\begin{picture}(5,5)(0,-2)
\put(0,0){\vector(1,0){5}} \put(4.8,-0.4){$x$}
\put(0,-2){\vector(0,1){5}} \put(0.2,2.8){$y$}
\put(1,0){\line(0,-1){0.1}}
\put(0,2){\line(-1,0){0.1}}
\put(1.0,-0.3){$\scriptstyle{1}$}
\put(-0.4,1.9){$\scriptstyle{2}$}
\put(1.0,-1.0){$\scriptstyle{y=2e\,\ln x/x}$}
\dashline{0.1}(2.7183,0)(2.7183,2)
\dashline{0.1}(4.5,1.8171)(1.87,1.8171) \dashline{0.1}(4.5,1.8171)(4.5,0)
\put(2.64,-0.2){$\scriptstyle{e}$} \put(4.42,-0.2){$\scriptstyle{x}$}
%\curve(
%    0.5000,   -1.5367,
%    1.0000,         0,
%    1.5000,    1.4696,
%    2.0000,    1.8842,
%    2.5000,    1.9926,
%    3.0000,    1.9909,
%    3.5000,    1.9459,
%    4.0000,    1.8842,
%    4.5000,    1.8171,
%    5.0000,    1.7500)
\curve(
     0.7500,   -2.0853,
     1.0000,         0,
     1.2500,    0.9705,
     1.5000,    1.4696,
     1.7500,    1.7385,
     2.0000,    1.8842,
     2.2500,    1.9594,
     2.5000,    1.9926,
     2.7500,    1.9999,
     3.0000,    1.9909,
     3.2500,    1.9716,
     3.5000,    1.9459,
     3.7500,    1.9162,
     4.0000,    1.8842,
     4.2500,    1.8509,
     4.5000,    1.8171,
     4.7500,    1.7834,
     5.0000,    1.7500)
\end{picture}
%\caption{$y=(e/x)\ln x$}
\end{center}
\end{figure}
\begin{figure}[H]
\setlength{\unitlength}{1cm}
\begin{center}
\begin{picture}(5,5)(0,0)
\put(0,0){\vector(1,0){5}} \put(4.8,-0.4){$x$}
\put(0,0){\vector(0,1){5}} \put(0.2,4.8){$y$}
\put(1,0){\line(0,-1){0.1}}
\put(0,1){\line(-1,0){0.1}}
\put(0.93,-0.4){$\scriptstyle{1}$}
\put(-0.4,0.9){$\scriptstyle{1}$}
\put(0,0){\line(1,1){4}}
\put(1.4,1){$\scriptstyle{\text{tie 1}}$}
\put(4.7,2){$\scriptstyle{\text{tie 2}}$}
\curve(
   2.7183,             2.7183,
   3.2183,             2.3356,
   3.7183,             2.0981,
   4.2183,             1.9360,
   4.7183,             1.8182)
\curve(
   2.7183,             2.7183,
   2.3356,             3.2183,
   2.0981,             3.7183,
   1.9360,             4.2183,
   1.8182,             4.7183)
\end{picture}
%\caption{$y^x=x^y$}
\end{center}
\end{figure}
\end{multicols}
\begin{Exa}
$f(x)=\ln\left|\dfrac{1-\cos x}{\sin x}\right|, \quad f'(x)=\,$?
\end{Exa}
\ratk Säännön \eqref{exp-kaava 6} ja Lukujen \ref{derivaatta}--\ref{kaarenpituus}
derivoimissääntöjen perusteella
\begin{align*}
\dif f(x)&=\frac{\sin x}{1-\cos x}\cdot \left[1-\frac{(1-\cos x)\cos x}{\sin^2 x}\right] \\
         &=\frac{\sin x}{1-\cos x}\cdot\frac{\sin^2 x+\cos^2 x-\cos x}{\sin^2 x} \\
         &=\frac{1}{\sin x}\,. \loppu
\end{align*}

Esimerkin tulos, ja muunnoksella $x\hookrightarrow\pi/2-x$ saatava vastaava tulos, on syytä 
panna korvan taakse:
\[ \boxed{
\begin{aligned}
\ykehys\quad \dif\ln\left|\dfrac{1-\cos x}{\sin x}\right|\ 
                 &=\ \dfrac{1}{\sin x}\,, \quad\quad x\in\R,\ \ \sin x\neq 0, \\[1mm]
             \dif\ln\left|\dfrac{1-\sin x}{\cos x}\right|\ 
                 &=\ -\dfrac{1}{\cos x}\,, \quad x\in\R,\ \ \cos x\neq 0. \quad\akehys
\end{aligned} } \]
Tässä on itse asiassa
$\,\D \dfrac{1-\cos x}{\sin x} = \dfrac{\sin x}{1+\cos x}=\tan\dfrac{x}{2}\,$
(ks.\ Harj.teht.\,\ref{trigonometriset funktiot}:\ref{H-II-5: trigtuloksia}b).

\Harj
\begin{enumerate}

\item \label{H-exp-2: exp-dy}
Näytä, että differentiaaliyhtälön $y'=ay$ yleinen ratkaisu $\R$:ssä on $y(x)=Ce^{ax}$.\,
\kor{Vihje}: Tutki funktiota $u(x)=e^{-ax}y(x)$.

\item \label{H-exp-2: kaavat}
a)--d) Perustele laskukaavat (5)--(8).

\item \label{H-exp-2: raja-arvoja}
Määritä tai näytä oikeaksi raja-arvot
\begin{align*}
&\text{a)}\ \lim_{x \kohti 0} \frac{e^{2x}-1-2x}{x^2} \qquad\qquad
 \text{b)}\ \lim_{x \kohti 0} \frac{1-6x+18x^2-e^{-6x}}{x^3} \\
&\text{c)}\ \lim_{x \kohti 0} \frac{3e^x-e^{-x}-2e^{2x}}{x^2} \qquad\
 \text{d)}\ \lim_{x\kohti\infty} x\left(e^{4/x}-e^{1/x}\right) \\[2mm]
&\text{e)}\ \lim_{x \kohti 0^+} x^x=1 \qquad 
 \text{f)}\ \lim_{x \kohti 0^+} x^{x^x}=0 \qquad
 \text{g)}\ \lim_{x \kohti 0^+} x^{\sqrt{x}}=1
\end{align*}

\item
Derivoi seuraavat funktiot: \newline
a) \ $2^x \quad$ 
b) \ $e^{\sqrt{x}} \quad$ 
c) \ $\pi^{1/x} \quad$ 
d) \ $\ln(\ln x)\quad$
e) \ $x^{2x} \quad$ 
f) \ $x^{x^x}$

\item
Määritä seuraavien funktioiden paikalliset ääriarvokohdat ja absoluuttiset maksimi- ja 
minimarvot, sikäli kuin olemassa. Hahmottele myös kuvaajat. \newline
a) \ $x\ln x,\quad$ b) \ $\sqrt[10]{\abs{x}}\ln\abs{x},\quad$ c) \ $\abs{x}e^{1/x^2},\quad$
d) \ $x^{-\pi}\ln x,\quad$ e) \ $x^{1/x}$. 

\item
a) \ Millä $a$:n arvoilla funktio $f(x)=2\ln x +x^2-ax+1$ on aidosti kasvava välillä
$(0,\infty)\,$? \newline
b) \ Näytä, että jos $0<a<b$, niin 
$\,\displaystyle{1-\frac{a}{b}\,<\,\ln \frac{b}{a}\,<\,\frac{b}{a}-1}$. \newline
c) \ Todista laskukaavat $\ \log_a b\cdot\log_b a=1\ $ ja $\ \log_b a\cdot\log_c b=\log_c a$.

\item
Määritä funktion $f(t)=e^{-x}\sin x$ paikalliset ääriarvokohdat sekä absoluuttiset maksimi-
ja minimiarvot välillä $[0,\infty)$ sekä hahmottele $f$:n kuvaaja.

\item
Päättele, että kiintopisteiteraatio $x_{n+1}=e^{-x_n},\, n=0,1,\dots$
suppenee jokaisella $x_0\in\R$. Määritä kiintopiste iteroimalla ensin
kolme kertaa alkuarvauksesta $x_0=0$ ja kiihdyttämällä sen jälkeen Newtonin menetelmällä.

\item
Näytä, että differentiaaliyhtälön $y''-y'=0$ yleinen ($\R$:ssä kahdesti derivoituva) ratkaisu on
$y(x)=C_1 e^x +C_2\,,\ C_1\,,C_2\in\R$.

\item
a) Näytä, että funktiolle
\[
f(x) = \begin{cases}
       e^{-\frac{1}{x^2}}, &\text{kun}\ x \neq 0 \\ 0, &\text{kun}\ x=0
       \end{cases}
\]
pätee $f^{(k)}(0)=0\ \forall k\in\N$. Hahmottele funktion kuvaaja. \vspace{1mm}\newline
b) Halutaan määritellä funktio $f$ ehdoilla \vspace{2mm}\newline
(i) \   $\,\ f(x)>0\,$ kun $\,x\in(0,1)$ \newline
(ii)\   $\,\ f(x)=0\,$ kun $\,x \le 0\,$ tai $\,x \ge 1$ \newline
(iii) \ $f$ on koko $\R$:ssä sileä, eli mielivaltaisen monta kertaa derivoituva \vspace{2mm}
\newline
Näytä, että eräs vaatimukset täyttävä funktio on
\[
f(x)=\begin{cases} 
     e^{-\frac{1}{x(1-x)}}, &\text{kun}\ x\in(0,1) \\ 0, &\text{muulloin}
     \end{cases}
\]
Hahmottele tämän kuvaaja. \vspace{1mm}\newline
c) Näytä, että funktio
\[
f(x)=\begin{cases} 
     0,             &\text{kun}\ x/\pi\in\Z \\ e^{-\cot^2 x}, &\text{muulloin}
     \end{cases}
\]
on koko $\R$:ssä mielivaltaisen monta kertaa derivoituva (sileä) funktio. Hahmottele $f$:n
kuvaaja välillä $[0,\pi]$.

\item (*)
Millä $k$:n arvoilla käyrällä $y=\ln|x|$ ja suoralla $y=k(x-1)+1$ on täsmälleen yksi yhteinen
piste?

\item (*)
Millä $a$:n arvoilla yhtälöllä $\,e^x=x+ax^2\,$ on reaalisia ratkaisuja a) ei yhtään, 
b) vain yksi, c) täsmälleen kaksi, d) kolme tai enemmän? Määritä ratkaisut numeerisin keinoin
tapauksissa $a=1$ ja $a=2$.

\item (*)
a) Näytä, että
\[
1+x\,\le\,e^x\,\le\,1+x+(e-2)\,x^2 \quad \forall\,x\in[0,1].
\]
b) Olkoon $a \ge 1$ ja $n\in\N,\ n\ge\ln a$. Näytä, että
\[
1+\frac{\ln a}{n}\,\le\,\sqrt[n]{a}\,\le\,1+\frac{\ln a}{n}+(e-2)\left(\frac{\ln a}{n}\right)^2.
\]
Kuinka suuri on approksimaation $\,\sqrt[n]{a} \approx 1+\ln a/n\,$ virhe todellisuudessa, kun
$a=\pi$ ja $n=100\,$?

\item (*)
Näytä, että differentiaaliyhtälön $y'=2xy$ yleinen ratkaisu $\R$:ssä on
$y(x)=Ce^{x^2},\ C\in\R$.

\item (*) \index{Hermiten!a@polynomi}
\kor{Hermiten polynomi} $H_n(x)$ määritellään derivoimiskaavalla
\[
D^ne^{-x^2}=H_n(x)e^{-x^2}, \quad n\in\N\cup\{0\}.
\]
Näytä, että pätee palautuskaava
\[
H_{n+1}(x)=-2xH_n(x)-2nH_{n-1}(x), \quad n\in\N.
\]
 
\item (*) \label{H-exp-2: eksoottisia raja-arvoja}
Tarkastellaan funktioita
\[
f(x)=\left(1+\frac{1}{x}\right)^x,\,\ x\in(0,\infty), \qquad
g(x)=\left(1-\frac{1}{x}\right)^x,\,\ x\in[1,\infty).
\]
a) Näytä, että $f$ ja $g$ ovat määrittelyväleillään aidosti kasvavia. \newline
b) Näytä, että
\begin{align*}
&\lim_{x\kohti\infty}\left(1+\frac{1}{x}\right)^x
    =\lim_{n\kohti\infty}\left(1+\frac{1}{n}\right)^n=e, \\
&\lim_{x\kohti\infty}\left(1-\frac{1}{x}\right)^x
    =\lim_{n\kohti\infty}\left(1-\frac{1}{n}\right)^n=\frac{1}{e}\,.
\end{align*}
c) Näytä, että 
$\,\displaystyle{\lim_{n\kohti\infty}\left(1+\frac{x}{n}\right)^n=e^x\ \forall x\in\R}$.

\item (*) \index{zzb@\nim!Matemaatikon mökki}
(Matemaatikon mökki) Matemaatikon mökki on koordinaatistossa, jonka origo on Helsingissä,
positiivinen $x$-akseli osoittaa itään ja pituusyksikkö $=100$ km. Mökki on erään tien
varressa kohdassa, jossa tie on itä-länsi-suuntainen. Määritä mökin sijainti, kun tiedetään,
että tien yhtälö on
\[
(2x)^y=y^{3x} \quad (x,y>0)
\]
ja lisäksi tiedetään, että tien päätepiste a) on Helsinki, b) ei ole Helsinki.

\item (*) \label{H-exp-2: kosini ja sini} \index{zzb@\nim!Pieniä ihmeitä}
(Pieniä ihmeitä) Olkoon $A\in\R$. Halutaan löytää $\R$:ssä derivoituvat funktiot $u$ ja $v$,
jotka ratkaisevat alkuarvotehtävän
\[
\text{(P)} \quad \begin{cases} 
                 \,u'=-v,\,\ v'=u, \quad x\in\R, \\ \,u(0)=A,\,\ v(0)=0. 
                 \end{cases}
\]
a) Totea: Eräs ratkaisu on $u(x)=A\cos x,\ v(x)=A\sin x$. \newline
b) Todista: Jos $u$ ja $v$ ovat mikä tahansa (P):n ratkaisu, niin
\[
[u(x)]^2+[v(x)]^2\,=\,A^2\ \ \forall x\in\R.
\]
c) Näytä: Jos $A=0$, niin (P):n ainoa ratkaisu on $u(x)=v(x)=0$. \newline
d) Päättele: a-kohdan ratkaisu on ainoa (P):n ratkaisu. \newline
d) Näytä: (P):n ratkaisu on myös
\[
u(x)=A\sum_{k=0}^\infty (-1)^k\frac{x^{2k}}{(2k)!}\,, \qquad
v(x)=A\sum_{k=0}^\infty (-1)^k\frac{x^{2k+1}}{(2k+1)!}\,.
\]
e) Päättele: Jokaisella $x\in\R$ pätee
\[
\cos x\,=\,\sum_{k=0}^\infty (-1)^k\frac{x^{2k}}{(2k)!}\,, \qquad
\sin x\,=\,\sum_{k=0}^\infty (-1)^k\frac{x^{2k+1}}{(2k+1)!}\,.
\]

\end{enumerate} %Funktiot $e^x$ ja $\ln x$
\section{Kompleksinen eksponenttifunktio. \\ Hyperboliset funktiot} 
\label{kompleksinen eksponenttifunktio}
\sectionmark{Kompleksinen eksponenttifunktio}
\alku
\index{analyyttinen kompleksifunktio|vahv}
\index{funktio A!e@kompleksifunktio|vahv}

Peruseksponenttifunktio voidaan laajentaa kompleksimuuttujan funktioksi $\exp: \C\kohti\C$
niin, että sen keskeiset ominaisuudet säilyvät. Määritelmä kompleksialueella --- joka toimii
myös reaalimuuttujaan rajoitettuna --- on seuraava.
\begin{Def} \index{kompleksimuuttujan!c@eksponenttifunktio|emph}
\index{funktio C!f@$\exp$ ($e^x$, $e^z$)|emph}
Peruseksponenttifunktio $\exp (z)=e^z$ on funktio, jolla on ominaisuudet (aksioomat)
\index{eksponenttifunktio!a@aksioomat|emph}
\begin{itemize}
\item[(C1)] $\exp (z)$ on määritelty $\forall z\in\C$,
\item[(C2)] $\exp(x+i0)=e^x\ \forall x\in\R$,
\item[(C3)] $\exp(z)$ on derivoituva origossa,
\item[(C4)] $\exp (z_1+z_2)=\exp(z_1)\exp(z_2)\quad\forall z_1,z_2\in\C$.
\end{itemize}
\end{Def}
Aksiooman C2 mukaan $e^z$ on reaalisen eksponenttifunktion laajennus kompleksitasoon.
Aksiooma C3 (vrt.\ Luku \ref{analyyttiset funktiot}) vastaa edellisen luvun aksioomaa E3'.
Kompleksiselle eksponenttifunktille tämä aksiooman muoto on välttämätön, sillä pelkkä
jatkuvuusoletus (vastaten reaalisen eksponenttifunktion aksioomaa E3) ei takaisi funktion
$\exp(z)$ yksikäsitteisyyttä kompleksifunktiona
(ks.\ Harj.teht.\,\ref{H-VII-3: E(z) vaihtoehto}). Aksioomilla C1--C4 sen sijaan $\exp(z)$
tulee yksikäsitteisesti määritellyksi. Osoitetaan tässä ainoastaan, että aksioomien
mukainen funktio $\,\exp(z)\,$ on olemassa, ja annetaan samalla funktiolle koko $\C$:ssä
pätevä laskusääntö.
\begin{Lause} \label{funktio exp(z)}
Funktio
\begin{equation} \label{exp(z):n laskukaava}
\boxed{\kehys\quad e^{x+iy}=e^x(\cos y+i\sin y),\quad z=x+iy\in\C \quad}
\end{equation}
toteuttaa aksioomat C1--C4.
\end{Lause}
\tod (C4) \  Jos $z_1=x_1+iy_1$, $z_2=x_2+iy_2$, niin määritelmän \eqref{exp(z):n laskukaava} 
ja kompleksialgebran laskusääntöjen (ks.\ Luku \ref{kompleksiluvuilla laskeminen})
perusteella 
\begin{align*}
e^{z_1+z_2} &\,=\, e^{x_1+x_2}[1\vkulma(y_1+y_2)] \\
           &\,=\, e^{x_1}\cdot e^{x_2}\cdot (1\vkulma{y_1})\cdot (1\vkulma{y_2}) \\
           &\,=\, [e^{x_1}\cdot (1\vkulma{y_1})]\,[e^{x_2}\cdot (1\vkulma{y_2})] \\
           &\,=\, e^{z_1}\cdot e^{z_2}.
\end{align*}

(C3) \ Käytetään hajotelmia
\begin{align*}
e^x &= 1+x+ r_1(x), \\
\cos y &= 1+ r_2(y), \\
\sin y &= y + r_3(y),
\end{align*}
missä $\lim_{t \kohti 0} t^{-1}r_i(t)=0$ (vrt.\ Luvut \ref{kaarenpituus} ja 
\ref{exp(x) ja ln(x)}). Näiden ja säännön \eqref{exp(z):n laskukaava} perusteella on
\begin{align*}
e^z &= [1+x+r_1(x)]\,[1+iy+r_2(y)+ir_3(y)] \\
    &= 1 + x +iy + r(z) \\
    & = 1+z+r(z), \quad z=x+iy,
\end{align*}
missä jäännöstermille pätee $\,\lim_{z \kohti 0} z^{-1}r(z)=0$. Näin ollen
\[
\lim_{z \kohti 0} \frac{e^z-e^0}{z} = \lim_{z \kohti 0} \frac{e^z-1}{z} 
                                    = \lim_{z \kohti 0} \left(1+\frac{r(z)}{z}\right) = 1.
\]
Siis $e^z$ on origossa derivoituva ja derivaatan arvo $=1$.

(C2) ja (C1) \ Ilmeisiä. \loppu

Funktio $e^z$ on derivoituva, ei ainoastaan origossa (aksiooma C3), vaan koko
kompleksitasossa, sillä ominaisuudet C3--C4 yhdistämällä seuraa
\[
\lim_{\Delta z \kohti 0} \frac{e^{z+\Delta z}-e^z}{\Delta z} 
           = e^z \lim_{\Delta z \kohti 0} \frac{e^{\Delta z}-1}{\Delta z} = e^z, \quad z\in\C.
\]
Derivoimissääntö on siis sama kuin reaaliselle (perus)eksponenttifunktiolle:
\index{derivoimissäännöt!g@eksponenttifunktio}%
\begin{equation} \label{exp(z):n derivaatta}
\boxed{\kehys\quad \dif\,e^z = e^z.\quad}
\end{equation}
Tämän mukaan kompleksifunktio $e^z$ on koko kompleksitasossa analyyttinen eli kokonainen funktio 
(vrt.\ Luku \ref{analyyttiset funktiot}). Tähän ja aksioomaan C2 viitaten sanotaankin,
\index{analyyttinen jatko}%
että $e^z$ on reaalisen eksponenttifunktion \kor{analyyttinen jatko} koko kompleksitasoon.

Derivoimissäännöstä \eqref{exp(z):n derivaatta} (joka siis seuraa suoraan aksioomista 
C3--C4) voidaan myös päätellä, että aksioomien C1--C4 mukainen funktio on yksikäsitteinen.
Nimittäin jos funktio $E(z)$ toteuttaa nämä aksioomat, niin on $DE(z)=E(z)$, jolloin
derivoimalla funktiota $u(z)=E(z)e^{-z}$ todetaan, että $Du(z)=0$ 
(vrt.\ Lauseen \ref{exp-dy} todistus). Koska aksiooman C2 mukaan on $u(0)=1$, on
pääteltävissä (sivuutetaan yksityiskohdat), että $u(z)=1\ \ekv\ E(z)=e^z\ \forall z\in\C$.

Määritelmästä \eqref{exp(z):n laskukaava} nähdään, että
\[
\abs{e^z} = e^x > 0 \quad \forall\ z=x+iy \in \C,
\]
joten kompleksisellakaan eksponenttifunktiolla ei ole nollakohtia. Kun määritelmässä valitaan
$x=\text{Re}\,z=0$, saadaan 
\kor{Eulerin kaava} \index{Eulerin!a@kaava}
\begin{equation} \label{eulerin kaava}
\boxed{\kehys\quad e^{iy}=\cos y + i\sin y,\quad y\in\R. \quad}
\end{equation}
Eulerin kaavaa \eqref{eulerin kaava} käyttäen voidaan trigonometriset funktiot $\cos$ ja $\sin$
lausua kompleksisen eksponenttifunktion avulla:
\begin{equation} \label{sin ja cos exp(z):n avulla}
\boxed{\kehys\quad \cos x=\tfrac{1}{2}\,(e^{ix}+e^{-ix}),\quad 
                   \sin x=\tfrac{1}{2i}\,(e^{ix}-e^{-ix}),\quad x\in\R. \quad}
\end{equation}
Tällä perusteella voidaan $\cos$ ja $\sin$ myös laajentaa kompleksimuuttujan funktioiksi. 
Määritelmät ovat
\index{funktio C!a@$\sin$, $\cos$} \index{kompleksimuuttujan!d@sini ja kosini}%
\begin{equation} \label{sin z ja cos z}
\boxed{\kehys\quad \cos z=\tfrac{1}{2}\,(e^{iz}+e^{-iz})\,,\quad 
                   \sin z=\tfrac{1}{2i}\,(e^{iz}-e^{-iz})\,,\quad z\in\C. \quad}
\end{equation}
Kaikki trigonometrisisille funktioille ominaiset laskusäännöt 
(vrt. Luku \ref{trigonometriset funktiot}) ulottuvat myös kompleksialueelle. Esimerkiksi
säännöt
\begin{align*}
&\cos^2 z+\sin^2 z = 1, \\
&\cos 2z = \cos^2 z-\sin^2 z, \quad \sin 2z = 2\sin z \cos z, \\
&\dif\sin z = \cos z, \quad \dif\cos z = -\sin z
\end{align*}
ovat todennettavissa suoraan määritelmistä \eqref{sin z ja cos z} ja $e^z$:n
derivoimissäännöstä \eqref{exp(z):n derivaatta}. Nähdään myös, että $\sin z$ ja $\cos z$
ovat ($e^z$:n tavoin) analyyttisiä koko kompleksitasossa eli kokonaisia funktioita.
\begin{Exa} Etsi yhtälön $\,\sin z = 2\,$ kaikki ratkaisut kompleksitasosta. 
\end{Exa}
\ratk Kun merkitään $t=e^{iz}$, niin määritelmän \eqref{sin z ja cos z} mukaan
\begin{align*}
\sin z = 2\,\ \ekv\,\ \frac{1}{2i}\,(t-t^{-1}) = 2\,\ &\ekv\,\ t^2-4it-1=0 \\
                                                      &\ekv\,\ t = (2\pm\sqrt{3})i.
\end{align*}
Määritelmän \eqref{exp(z):n laskukaava} ja aksiooman C3 mukaan
\[
t=e^{iz} = e^{-y+ix} = e^{-y}e^{ix}, \quad z=x+iy,
\]
joten on oltava
\[
\begin{cases} \,e^{-y}=\abs{t}=2\pm\sqrt{3}, \\ \,e^{ix}=i  \end{cases} \ekv\quad
\begin{cases} \,y=-\ln(2\pm\sqrt{3}), \\ \,x=\frac{\pi}{2}+2k\pi,\,\ k\in\Z. \end{cases}
\]
Tässä on $\,\ln(2-\sqrt{3})=-\ln(2+\sqrt{3})$, joten ratkaisut ovat
\[ 
z= \frac{\pi}{2}+2k\pi\,\pm\,\ln(\sqrt{3}+2)i, \quad k\in\Z. \loppu
\]

\subsection{Hyperboliset funktiot}
\index{hyperboliset funktiot|vahv}
\index{funktio C!h@$\sinh$, $\cosh$, $\tanh$|vahv}
\index{funktio C!i@$\arsinh$, $\arcosh$, $\artanh$|vahv}

Hyberboliset funktiot ovat eksponenttifunktion johdannaisia, joilla on samantyyppisiä 
ominaisuuksia kuin trigonometrisillä funktioilla. \kor{Hyberbolinen kosini}, symboli $\cosh$
(cosinus hyperbolicus), ja \kor{hyberbolinen sini}, symboli $\sinh$ (sinus hyperbolicus) 
määritellään (vrt.\ trigonometristen funktioiden määritelmä \eqref{sin z ja cos z})
\begin{equation} \label{sinh ja cosh}
\boxed{\kehys\quad \cosh z=\tfrac{1}{2}(e^{z}+e^{-z}),\quad 
                   \sinh z=\tfrac{1}{2}(e^{z}-e^{-z}),\quad z\in\C. \quad}
\end{equation}
Esimerkiksi seuraavat laskulait ovat määritelmästä todennettavissa (vrt.\ trigonometristen
funktioiden vastaavat):
\index{derivoimissäännöt!i@hyperboliset funktiot}
\begin{align*}
&\cosh^2 z - \sinh^2 z = 1, \\
&\cosh 2z = \cosh^2 z + \sinh^2 z, \quad \sinh 2z = 2\sinh z\cosh z, \\
&\dif\sinh z = \cosh z, \quad \dif\cosh z = \sinh z.
\end{align*}
Reaalisilla muuttujan arvoilla $\cosh x=\frac{1}{2}(e^x+e^{-x})$ on parillinen ja 
$\sinh x=\frac{1}{2}(e^x-e^{-x})$ on pariton funktio. Edellinen on aidosti kasvava välillä 
$[0,\infty)$ ja bijektio kuvauksena $\cosh: [0,\infty)\kohti[1,\infty)$. Jälkimmäinen on 
aidosti kasvava koko $\R$:ssä ja bijektio kuvauksena $\sinh:\R\kohti\R$.
\begin{figure}[H]
\setlength{\unitlength}{1cm}
\begin{picture}(10,5)(-4,-1)
\put(-2,0){\vector(1,0){4}} \put(1.8,-0.4){$x$}
\put(0,-1){\vector(0,1){5}} \put(0.2,3.8){$y$}
\put(1,0){\line(0,-1){0.1}}
\put(0,1){\line(-1,0){0.1}}
\put(0.93,-0.4){$\scriptstyle{1}$}
\put(-0.4,0.83){$\scriptstyle{1}$}
\curve(
   -2.0000,    3.7622,
   -1.5000,    2.3524,
   -1.0000,    1.5431,
   -0.5000,    1.1276,
         0,    1.0000,
    0.5000,    1.1276,
    1.0000,    1.5431,
    1.5000,    2.3524,
    2.0000,    3.7622)
\put(5,1.5){\vector(1,0){4}} \put(8.8,1.1){$x$}
\put(7,-1){\vector(0,1){5}} \put(7.2,3.8){$y$}
\put(8,1.5){\line(0,-1){0.1}}
\put(7,2.5){\line(-1,0){0.1}}
\put(7.93,1.1){$\scriptstyle{1}$}
\put(6.6,2.6){$\scriptstyle{1}$}
\curve(
    5.5000,   -0.6293,
    5.7000,   -0.1984,
    5.9000,    0.1644,
    6.1000,    0.4735,
    6.3000,    0.7414,
    6.5000,    0.9789,
    6.7000,    1.1955,
    6.9000,    1.3998,
    7.1000,    1.6002,
    7.3000,    1.8045,
    7.5000,    2.0211,
    7.7000,    2.2586,
    7.9000,    2.5265,
    8.1000,    2.8356,
    8.3000,    3.1984,
    8.5000,    3.6293)
\end{picture}
\end{figure}

Reaalifunktioiden $\cosh$ ja $\sinh$ käänteisfunktioita ($\cosh$ rajoitettu välille 
$[0,\infty)$ tai $(-\infty,0]$) sanotaan
\index{area-funktiot}%
\kor{area-funktioiksi} ja merkitään arcosh, arsinh. Laskusäännöt saadaan
ratkaisemalla
\begin{align*}
\frac{1}{2}(e^y+e^{-y})=x \ \ekv \ y=\arcosh x, \\
\frac{1}{2}(e^y-e^{-y})=x \ \ekv \ y=\arsinh x.
\end{align*}
Nämä ovat toisen asteen yhtälöitä tuntemattoman $t=e^y$ suhteen, joten $y$ on ilmaistavissa 
logaritmien avulla:
\[ \boxed{
\begin{aligned}
\ykehys\quad y = \arcosh x &= \pm \ln\left(x+\sqrt{x^2-1}\,\right), \quad x \ge 1, \quad \\
             y = \arsinh x &= \ln\left(x+\sqrt{x^2+1}\,\right), \quad x\in\R. \akehys
\end{aligned} } \]
Tässä arcosh:n päähaara $\Arcosh$ saadaan etumerkillä +. Toisen haaran voi 
kirjoittaa myös muotoon
\[
y\ =\ -\ln\left(x+\sqrt{x^2-1}\,\right)\ =\ \ln\frac{1}{x+\sqrt{x^2-1}}\ 
                                         =\ \ln\left(x-\sqrt{x^2-1}\,\right).
\]

\kor{Hyberbolinen tangentti} $\tanh$ (tangens hyperbolicus) määritellään (ol.\ reaalimuuttuja)
\[
\boxed{\quad \tanh x\ =\ \frac{\ykehys\sinh x}{\akehys\cosh x}\ 
                      =\ \frac{e^x-e^{-x}}{e^x+e^{-x}}\ 
                      =\ \frac{1-e^{-2x}}{1+e^{-2x}}\,. \quad}
\]
\begin{figure}[H]
\setlength{\unitlength}{1cm}
\begin{center}
\begin{picture}(8,4)(-4,-2)
\put(-4,0){\vector(1,0){8}} \put(3.8,-0.4){$x$}
\put(0,-2){\vector(0,1){4}} \put(0.2,1.8){$y$}
\put(1,0){\line(0,-1){0.1}}
\put(0,1){\line(-1,0){0.1}}
\put(0.93,-0.4){$\scriptstyle{1}$}
\put(-0.4,0.9){$\scriptstyle{1}$}
\curve(
   -4.0000,   -0.9993,
   -3.5000,   -0.9982,
   -3.0000,   -0.9951,
   -2.5000,   -0.9866,
   -2.0000,   -0.9640,
   -1.5000,   -0.9051,
   -1.0000,   -0.7616,
   -0.5000,   -0.4621,
         0,         0,
    0.5000,    0.4621,
    1.0000,    0.7616,
    1.5000,    0.9051,
    2.0000,    0.9640,
    2.5000,    0.9866,
    3.0000,    0.9951,
    3.5000,    0.9982,
    4.0000,    0.9993)
\end{picture}
%\caption{$y=\tanh x$}
\end{center}
\end{figure}
Hyperbolinen tangentti on bijektio: $\ \tanh:\ \R \kohti (-1,1)$. Käänteisfunktio on
\[
\boxed{\kehys\quad \artanh\,x 
             = \frac{1}{2}\,\ln \left(\frac{1+x}{1-x}\right),\quad \abs{x}<1. \quad}
\]

Myöhempää käyttöä varten todettakoon vielä hyperbolisten käänteisfunktioiden derivoimiskaavat
\index{derivoimissäännöt!i@hyperboliset funktiot}%
\[ \boxed{ \begin{aligned}
\quad &\dif\ln(x+\sqrt{x^2+1}\,)\,=\,\frac{\ygehys 1}{\sqrt{x^2+1}}\,, \quad x\in\R, \\
      &\dif\,\bigl|\ln(x+\sqrt{x^2-1}\,\bigr|\,=\,\frac{\ykehys 1}{\sqrt{x^2-1}}\,, \quad 
                                                             \abs{x}>1, \quad \\[3mm]
      &\dif\,\frac{1}{2}\ln\left|\frac{1+x}{1-x}\right|\,
                               =\,\frac{1}{\akehys 1-x^2}\,, \quad \abs{x}\neq 1.
           \end{aligned} } \]

\subsection{Kompleksinen logaritmifunktio}
\index{kompleksimuuttujan!e@logaritmifunktio|vahv}
\index{logaritmifunktio|vahv}
\index{funktio C!g@$\log$, $\ln$|vahv}

Kompleksinen (perus)logaritmifunktio määritellään
\[
e^w=z \ \ekv \ w=\log z.
\]
Kun kirjoitetaan $w=u+iv$ ja otetaan $z$:lle polaariesitys $z=r(\cos\varphi+i\sin\varphi)$,
niin yhtälöstä
\[
e^w=z\ \ekv\ e^u(\cos v+i\sin v) = r(\cos\varphi+i\sin\varphi)
\]
nähdään, että on oltava
\[
\begin{cases}
\,e^u=r\ \ekv\ u = \ln r = \ln\abs{z}, \\
\,v=\varphi +2k\pi,\quad k\in\Z.
\end{cases}
\]
Kyseessä on siis äärettömän monihaarainen funktio. Logaritmifunktion $\log z$ määritelmäksi
sovitaan tämän funktion päähaara, jossa $k=0$\,:
\[
\boxed{\kehys\quad \log z = \ln\abs{z}+i\arg z, \quad 0 \le \arg z < 2\pi. \quad}
\]
Näin määritelty funktio ei ole jatkuva positiivisella reaaliakselilla, sillä jos $z_n=x+iy_n$,
$x>0$, niin $z_n\kohti x$, kun $y_n\kohti 0$, mutta
\begin{align*}
y_n\kohti 0^+ \ &\impl \ \log z_n\kohti \ln x, \\
y_n\kohti 0^- \ &\impl \ \log z_n\kohti \ln x+2\pi i.
\end{align*}
Muissa määrittelyjoukon pisteissä $\log z$ on paitsi jatkuva myös derivoituva
(ks.\ Harj.teht.\,\ref{H-exp-3: logaritmin derivaatta}), eli $\log z$ on analyyttinen nk.\
\index{aukileikattu kompleksitaso} \index{kompleksitaso!a@aukileikattu}%
\kor{aukileikatussa kompleksitasossa}, josta origo ja positiivinen reaaliakseli on poistettu.

Johtuen kompleksisen logaritmifunktion haaraisuudesta eivät reaalialueen laskusäännöt yleisty
sellaisenaan. Esimerkiksi tulon logaritmi on em.\ määritelmän perusteella
(vrt. Luku \ref{exp(x) ja ln(x)})
\[
\log (z_1z_2)=\begin{cases}
\,\log z_1+\log z_2\,,           &\text{jos}\,\ 0\leq\arg z_1+\arg z_2<2\pi, \\
\,\log z_1+\log z_2 -2\pi i,\  &\text{muulloin}.
\end{cases}
\]
\begin{Exa} Funktion $\log$ määritelmän mukaisesti
\begin{align*}
&\log(-1)    \,=\, \log(1\vkulma{\pi})           
             \,=\, \ln 1 + \pi i \,=\, \pi i, \\[2mm]
&\log (1+i)  \,=\, \log\left(\sqrt{2}\vkulma{\pi/4}\right)  
             \,=\, \frac{1}{2}\ln 2+\frac{\pi}{4}\,i, \\
&\log (-1-i) \,=\, \log\left(\sqrt{2}\vkulma{5\pi/4}\right) 
             \,=\, \frac{1}{2}\ln 2+\frac{5\pi}{4}\,i.\loppu
\end{align*}
\end{Exa}
Kompleksisen logaritmifunktion avulla voidaan määritellä (vrt.\ edellisen luvun kaava
\eqref{exp-kaava 5})
\[
\boxed{\kehys\quad z^y = e^{y\log z}, \quad z,y\in\C,\ z\neq 0. \quad}
\]
\begin{Exa} Edellisen esimerkin perusteella
\[
(-1)^\pi=e^{\pi\log(-1)}=e^{i\pi^2}=\cos\pi^2+i\sin\pi^2. \loppu
\]
\end{Exa}

\Harj
\begin{enumerate}

\item \label{H-VII-3: E(z) vaihtoehto}
Määritellään funktion $e^x$ laajennus kompleksitasoon kaavalla
\[
\exp(z)=\exp(x+iy)=e^x(\cos ay + i\sin ay),
\]
missä $a\in\R,\ a \neq 1$. Mitkä kompleksisen eksponenttifunktion aksioomista ovat voimassa 
tälle funktiolle?

\item
Tarkista seuraavien kaavojen pätevyys, kun $z\in\C$\,:
\begin{align*}
&\text{a)}\ \ \sin^2 z+\cos^2 z=1, \quad \sin 2z=2\sin z\cos z, \quad \cos 2z=2\cos^2 z-1 \\
&\text{b)}\ \ \dif\sin z=\cos z, \quad \dif\cos z=-\sin z
\end{align*}

\item
a) Lausu $\sinh\frac{x}{2}$ ja $\cosh\frac{x}{2}$ $\cosh x$:n avulla. \newline
b) Millaisilla luvuilla $n$ pätee $\,(\cosh x+\sinh x)^n=\cosh nx+\sinh nx\,$?

\item
a) Määritellään $\coth x=1/\tanh x\ (x\in\R,\ x \neq 0)$. Mikä on funktion 
$f(x)=4\tanh x+\coth x$ arvojoukko? \\
b) Missä pisteessä funktio
$\D f(x)=\frac{\sinh x}{1-a\cosh x}\,, \quad x\in\R$ \vspace{1mm}\newline
saavuttaa suurimman arvonsa, kun $a\in\R,\ a>1$\,?

\item
Näytä, että yhtälöllä $\,\cos x\cosh x+1=0\,$ on äärettömän monta reaalista ratkaisua.

\item
a) Näytä, että funktioilla $\Arcosh$ ja $\arsinh$ on asymptoottina funktio 
$\ln (2x)=\ln x + \ln 2$, kun $x\kohti\infty$. \newline
b) Laske funktioiden $\arsinh x$ ja $\Arcosh x$ derivaatat sekä implisiittisesti
derivoimalla että suoraan ko.\ funktioiden lausekkeista.

\item
Sievennä:
\[
\text{a)}\ \ \Arcosh(\cosh x) \qquad
\text{b)}\ \ \tanh(\Arcosh x) \qquad 
\text{c)}\ \ \artanh\frac{x}{\sqrt{x^2+1}}
\]

\item
Laske seuraavat kompleksiluvut perusmuodossa $x+iy$ (tarkat arvot!): \vspace{1mm} \newline
a) \ $e^z$, $\sinh z$, $\cosh z$ ja $\tanh z$, kun $z=2+3i$ \newline
b) \ $e^z$, $\sin z$, $\cos z$ ja $\tan z$, kun $z=-1-i$ \newline
c) \ $e^z$, $\sin z$, $\cos z$ ja $\tan z$, kun $z=3-2i$ \newline
d) \ $\cosh(\arsinh\frac{4}{3}),\ \tanh(\arsinh\pi),\ \Arcosh\sqrt{2},\ \artanh 1$ \newline 
e) \ $e^i,\ i^e,\ \log i,\ 5^{-i},\ i^{-\pi},\ i^i$

\item
Tietokoneohjelma laskee luvun $(-\pi)^\pi$ numeroarvoksi $-32.9139-15.6897i$. Tarkista lasku! 

\item
Määritä (tarkasti perusmuodossa $x+iy$) seuraavien yhtälöiden kaikki ratkaisut 
kompleksitasossa: \newline
a) \ $e^z=e,\quad$ b) \ $\cos z=-2,\quad$ c) $\cosh z=-1,\quad$ d) \ $\sin z=i$.  

\item
Määritä seuraaville funktioille suurin kompleksitason avoin osajoukko $G$, jossa funktio on 
analyyttinen:
\[
\text{a)}\ \ f(z)=\frac{1}{e^z+1} \qquad \text{b)}\ \ f(z)=\frac{z}{\cos z} \qquad
\text{c)}\ \ f(z)=\frac{1}{z\cosh z+2z}
\]

\item (*) \label{H-exp-3: logaritmin derivaatta}
Johda suoraan derivaatan määritelmästä derivoimssääntö
\[
\dif\log z = \frac{1}{z}\,, \quad z=r\vkulma\varphi,\,\ r>0,\,\ 0<\varphi<2\pi.
\]
\kor{Vihje}: Kirjoita $\Delta z=\Delta r\vkulma(\varphi+\psi)$.

\end{enumerate} %Kompleksinen eksponenttifunktio
\section{Eksponenttifunktion sovellusesimerkkejä}
\label{eksponenttifunktio fysiikassa}
\alku
\index{differentiaaliyhtälö!a@eksponenttifunktion|vahv}

Fysiikassa esiintyy koko joukko nk.\ eksponentiaalisia ilmiöitä, joissa fysikaalisen suureen 
muuttumista paikan ($x$) tai ajan ($t$) funktiona kuvaa eksponenttifunktio. Seuraavassa viisi
esimerkkiä.
\index{zza@\sov!Radioaktiivinen hajoaminen}%
\begin{Exa}: \label{radioaktiivisuus} \vahv{Radioaktiivinen hajoaminen}.\ Olkoon $A(t)$ tiettyä
lajia olevien radioaktiivisten ytimien lukumäärä hetkellä $t$. Jos merkitään
\[
A(t)=A_0 E(t),\quad t\geq 0,
\]
missä $A_0=A(0)$, niin $E(0)=1$. Jos oletetaan, että sama laki on sovellettavissa jokaisella 
ajan hetkellä, on oltava
\[
A(t_1+t_2) = A_0 E(t_1+t_2) = A(t_1)E(t_2) = A_0 E(t_1)E(t_2), \quad t_1,t_2 \ge 0.
\]
Siis funktiolla $E(t)$ on ominaisuudet
\[
E(0)=1, \quad E(t_1+t_2)=E(t_1)E(t_2), \quad t_1,t_2 \ge 0.
\]
Päätellään, että $E$ on eksponenttifunktio ja $A(t)$ siis esitettävissä muodossa
\[ A(t) = A_0 e^{-at}, \quad t \ge 0, \]
missä $a$ (mittayksikkö 1/s) on ytimille ominainen vakio (dimensiottomana positiivinen). 
\pain{Puoliintumisaika} $t_{1/2}$ määritellään ehdosta $A(t_{1/2}) = A_0/2$, jolloin on oltava
\[ 
e^{-a\,t_{1/2}} = \dfrac{1}{2} \qekv a\,t_{1/2} 
                = \ln 2 \qekv t_{1/2} = \dfrac{\ln 2}{a}\,. \loppu 
\]
\end{Exa} 
\index{zza@\sov!Szy@Säteilyn vaimeneminen}
\begin{Exa}: \label{säteilyvaimennus} \vahv{Säteilyn vaimeneminen}.\ Olkoon $I(x)$
radioaktiivisen säteilyn intensiteetti vaimentavassa (homogeenisessa) väliaineessa kuljetun
matkan $x$ funktiona. Jos merkitään
\[
I(x)=I_0E(x),\quad x \ge 0,
\]
missä $I_0=I(0)$, niin päätellään samoin kuin edellisessä esimerkissä, että $E$ on 
eksponenttifunktio, eli
\[ 
I(x) = I_0 e^{-ax}, \quad x \ge 0,
\]
missä $a$ on väliaineelle ominainen vaimennusvakio (mittayksikkö 1/m). Matka jonka kuluessa 
säteilyn intensiteetti on vaimentunut puoleen alkuintensiteetistä, eli nk.\ 
p\pain{uoliarvomatka} on
\[ 
d_{1/2} = \dfrac{\ln 2}{a}. \loppu 
\]
\end{Exa}
Em.\ esimerkeissä nojattiin suoraan eksponenttifunktion perusaksioomaan E4. Useammin
eksponenttifunktioon päädytään sovellustilanteissa niin, että tarkasteltavan fysikaalisen
suureen $y(x)$ (tai $y(t)$) todetaan toteuttavan eksponenttifunktion differentiaaliyhtälön
\begin{equation} \label{dy1}
y' = ay,
\end{equation}
missä $a$ on (dimensiollinen) vakio. Jos oletetaan, että muuttujan ($x$) fysikaalisesti 
relevantit arvot ovat (dimensiottomina) ei-negatiivisia ja että tunnetaan arvo $y(0) = y_0$, 
niin kyseessä on alkuarvotehtävä: On etsittävä funktio $y=y(x)$, joka on jatkuva välillä
$[0,\infty)$, derivoituva välillä $(0,\infty)$ ja toteuttaa
\[
\begin{cases} \,y' = ay, \quad x>0, \\ \,y(0) = y_0. \end{cases}
\]
Ratkaisu on $\,y(x)=y_0 e^{ax},\ x \ge 0$. 
\index{zza@\sov!Szyhkzza@Sähköpiiri: RC}%
\begin{Exa}: \vahv{Sähköpiiri: RC}.\ Kondensaattorin kapasitanssi on $\,C$ ja varaus $q_0$.
Hetkellä $t=0$ (aikayksikkö s) kondensaattoria ryhdytään purkamaan vastuksen $R$ läpi, jolloin
varaus hetkellä $t \ge 0$ on $q(t)$. Matemaattinen malli?
\end{Exa}
\ratk Jos vastuksen läpi kulkeva virta hetkellä $t$ on $i(t)$, niin piiriyhtälöt ovat
\[ 
q(t)/C = Ri(t), \quad i(t) = -q'(t), \quad t>0. 
\]
Eliminoimalla $i(t)$ päädytään alkuarvotehtävään
\[ 
\begin{cases} \,q' = -aq, \quad t>0, \\ \,q(0) = q_0, \end{cases} 
\]
missä $a = RC$. Ratkaisu on
\[
q(t) = q_0 e^{-t/\tau}, \quad t \ge 0, 
\]
missä $\tau = 1/(RC)$ on piirille ominainen nk.\ \pain{aikavakio} (mittayksikkö s). \loppu  

Sovelluksissa eksponenttifunktion differentiaaliyhtälöstä \eqref{dy1} esiintyy usein myös
variaatio
\begin{equation} \label{dy2}
y' = ay + b,
\end{equation}
missä $a$ ja $b$ ovat vakioita. Olettaen, että $a \neq 0$, tämän eräs ratkaisu on 
$y_0(x) = -b/a = \text{vakio}$. Kun yleisempää ratkaisua haetaan muodossa
$y(x) = y_0(x) + v(x)$, niin todetaan, että funktio $v$ toteuttaa eksponenttifunktion
differentiaaliyhtälön \eqref{dy1}. Differentiaaliyhtälön \eqref{dy2} yleinen ratkaisu,
kun $a \neq 0$, on siis
\[ 
y(x) = -b/a + C e^{ax}, \quad x\in\R.
\]
Sovellustilanteessa vakio $C$ määrätään alkuehdosta.


\begin{Exa}: \index{zza@\sov!Szyhkzzb@Sähköpiiri: LR}
\vahv{Sähköpiiri: LR}.\ Induktanssi $\,L$ ja vastus $\,R$ on kytketty sarjaan. Hetkellä $t=0$ s
piiri kytketään vakiojännitteeseen $E$, jolloin piiriin syntyy virta $i(t)$ ($i(0)=0$).
Matemaattinen malli vastuksen yli vaikuttavalle jännitteelle $y(t)\,$?
\end{Exa}
\ratk Piiriyhtälöt ovat
\[ 
L\,i'(t) + R\,i(t) = E, \quad R\,i(t) = y(t), \quad t>0. 
\]
Eliminoimalla $i(t)$ päädytään differentiaaliyhtälöön \eqref{dy2}, missä
$a = -R/L,\ b = ER/L$. Alkuarvo on $y(0)=0$. Alkuarvotehtävän ratkaisu saadaan valitsemalla
em.\ yleisessä ratkaisussa $C = b/a$, jolloin
\[ 
y(t) = (b/a)(e^{at}-1) = E(1-e^{-t/\tau}), \quad t \ge 0,
\]
missä $\tau = L/R$ (aikavakio, yksikkö s). \loppu
\index{zza@\sov!Jzyzy@Jäähtymislaki}%
\begin{Exa}: \vahv{Jäähtymislaki}.\ Hetkellä $t$ (yksikkö s) on kappaleen lämpötila $u(t)$ ja
lämpöenergia $U(t) = cu(t)$ ($c =$ lämpökapasiteetti). Miten kappale jäähtyy alkulämpötilasta
$u(0)=u_0$, jos oletetaan, että lämpövirta ympäröivään ilmaan on $Q(t) = k[u(t)-u_1]$, missä
$u_1 < u_0$ on ulkoilman lämpötila?
\end{Exa}
\ratk Energian säilymislaki on
\begin{align*}
U' = -Q &\qekv cu' = -k(u-u_1) \\
        &\qekv u' = au + b, \quad a = -k/c,\ b = ku_1/c.
\end{align*}
Differentiaaliyhtälön yleinen ratkaisu $u(t) = -b/a + C e^{at} = u_1 + C e^{at}$ toteuttaa 
alkuehdon $u(0)=u_0$, kun $C=u_0-u_1$, joten
\[ 
u(t) = u_1 + (u_0-u_1) e^{-t/\tau}, \quad t \ge 0, 
\]
missä $\,\tau = -a^{-1} = c/k\,$ on jäähtymisen aikavakio (yksikkö s). \loppu

\Harj
\begin{enumerate}

\item
Plutoniumin Pu$^{239}$ puolintumisaika on $25400$ vuotta. Paljonko $1000$ kg:sta plutoniumia
on jäljellä miljoonan vuoden kuluttua?

\item
Positiivisen $x$ akselin suuntaan etenevästä säteilystä pääsee välillä $[0,3]$ olevan
säteilysuojauksen läpi $3\%$ säteilyn intensiteetistä. Suojaus on rakennettu kahdesta
materiaalikerroksesta siten, että välillä $[0,1]$ olevassa materiaalissa 1 on säteilyn 
vaimennusvakio kolme kertaa niin suuri kuin välillä $[1,3]$ olevassa materiaalissa 2. Paljonko
jälkimmäistä materiaalikerrosta on vahvistettava, jotta säteilystä pääsisi läpi ainoastaan
$1\%$\,?
 
\item
Vuotuinen korkoprosentti on $5$ ja korko liitetään pääomaan a) jatkuvan koronkoron mukaisesti,
b) puolivuosittain. Mikä on eri tavoin saatujen pääomien suhde $50$ vuoden kuluttua?

\item
Keittokattilan alkulämpötila on $96\aste$C. Kattilan annetaan ensin jäähtyä huoneen lämmössä
($20\aste$C), kunnes sen lämpötila on $40\aste$C. Tämän jälkeen kattila sijoitetaan 
jääkaappiin, jossa lämpötila on $6\aste$C. Jos ensimmäisen jäähdytysvaiheen kesto on $32$ min,
niin kauanko kattilan on oltava jääkaapissa, jotta se on jäähtynyt lämpötilaan $7\aste$C? 
Oletetaan, että huoneessa ja jääkaapissa pätee sama eksponentiaalinen jäähtymislaki.

\item
Kupillinen kahvia, jonka lämpötila on aluksi $80\aste$C, jäähtyy ulkoilmassa siten, että $5$ 
minuutin kuluttua kahvin lämpötila on $60\aste$C ja $10$ minuutin kuluttua $44\aste$C. Mikä on
ulkoilman lämpötila?
 
\item
Vaakasuoralla maan pinnalla on kuution muotoinen vesisäiliö, jonka särmän pituus on $10$ m.
Säiliö on täynnä vettä. Hetkellä $t=0$ avataan säiliön pohjaventtiili, jolloin säiliö alkaa
tyhjentyä nopeudella $kH(t)$, missä $H(t)$ on veden korkeus säiliössä (yksikkö m) hetkellä
$t$ (yksikkö h=tunti) ja $k$ on venttiilin asennosta riippuva kerroin (yksikkö 
$\text{m}^2/\text{h}$). Tunnin kuluttua venttiilin avaamisesta havaitaan, että $8\%$
vedestä on virrannut pois. Tällöin venttiiliä avataan lisää, jolloin virtausnopeus
hetkellisesti kaksinkertaistuu. Venttiili jätetään tämän jälkeen uuteen asentoonsa. Määritä
säiliössä olevan veden määrä $m(t)$ (kuutiometreinä) ajan $t$ funktiona välillä $[0,\infty)$.

\item
Kappale on hetkellä $t=0$ s levossa veden pinnalla ja lähtee vajoamaan noudattaen liikelakia
\[
v'(t)=0.8g-kv(t),
\]
missä $v$ on vajoamisnopeus, $g=10\,\text{m}/\text{s}^2$ ja $k$ on vakio. Määritä
$v(t),\ t \ge 0$ (aikayksikkö s), kun tiedetään, että kappale vajoaa hyvin syvässä 
vedessä (asymptoottisella) nopeudella  $3.2\ \text{m}/\text{s}$.

\item (*) Edellisessä tehtävässä kappaleen vajoamissyvyys ennen pohjakosketusta toteuttaa
\[
\begin{cases} \,h'(t) = v(t), \quad t>0, \\ \,h(0)=0. \end{cases}
\]
Millä syvyydellä kappaleen vajoamisnopeus on $99.9\ \%$ nopeuden asymptoottisesta arvosta?

\end{enumerate}
 %Eksponenttifunktion sovellusesimerkkejä

\chapter{Yhden muuttujan differentiaalilaskenta} 
\label{yhden muuttujan differentiaalilaskenta}
\chaptermark{Differentiaalilaskenta}

Kun derivaattaa käytetään laskennassa välineenä, puhutaan \kor{differentiaalilaskennasta}
(kirjaimellisesti 'pienten erotusten laskennasta', engl.\ differential calculus). Derivaatan
toivat matematiikkaan toisistaan riippumatta englantilainen fyysikko-matemaatikko
\index{Newton, I.} \index{Leibniz, G. W.}%
\hist{Isaac Newton} (1642-1727) ja saksalainen filosofi-matemaatikko 
\hist{Gottfried Wilhelm Leibniz} (1646-1716) 1600-luvun lopulla. Leibniz on vaikuttanut
huomattavasti vielä nykyisinkin käytössä oleviin differentiaalilaskun
(myös myöhemmin tarkasteltavan integraalilaskun) merkintöihin. 

Koska derivaatta oli alunperinkin fysiikan motivoima käsite (etenkin Newtonin tutkimuksissa),
ei derivaatan soveltuvuudessa fysiikkaan ole ihmettelemistä. Pitkälle 1800-luvulle 
differentiaalilaskennan ja sen pohjalta nousevien matematiikan alojen kehitys olikin 
voimakkaasti sidoksissa fysiikkaan, ja vielä nykyisinkin on yhteys säilynyt vahvana monissa 
matematiikan lajeissa.

Tässä luvussa tarkastellaan yhden muuttujan differentiaalilaskennan soveltamista
käyräteoriassa, fysiikassa (mm.\ liikeopissa) ja funktioiden approksimoinnissa. Funktion
approksimoinnin keskeinen tulos on Luvussa \ref{taylorin lause} esitettävä ja todistettava
\kor{Taylorin lause}. Tässä on kyse linearisoivan approksimaation yleistämisestä
\kor{Taylorin polynomeihin} perustuvaksi yleisemmäksi polynomiapproksimaatioksi. Suotuisissa
oloissa funktio voidaan esittää myös tarkasti \kor{Taylorin sarjana}. Taylorin sarjat ovat
potenssisarjoja --- ja potenssisarjojen teorian kauniiksi lopuksi osoittautuukin, että
jokainen potenssisarja, jonka suppenemissäde on positiivinen, on itse asiassa sarjan summana
määritellyn funktion Taylorin sarja. Viimeisessä osaluvussa esitellään vielä numeerisissa
laskentamenetelmissä yleisesti käytettyjen \kor{interpolaatiopolynomien} teoriaa ja
käyttötapoja. %Yhden muuttujan differentiaalilaskenta
\section{Differentiaali ja muutosnopeus} \label{differentiaali}
\sectionmark{Differentiaali}
\alku

Jos $f$ on derivoituva pisteessä $x$, niin linearisoivan approksimaatioperiaatteen mukaisesti
on likimain
\[
\Delta f(x)=f(x+\Delta x)-f(x)\approx f'(x)\Delta x,
\]
kun $\abs{\Delta x}$ on pieni. Lauseketta $f'(x)\Delta x$ sanotaan $f$:n
\index{differentiaali}%
\kor{differentiaaliksi} pisteessä $x$ ja merkitään
\[
df(x,\Delta x) = f'(x)\Delta x.
\]
Differentiaalin avulla voi siis arvioida likimäärin funktion arvon muutoksen $\Delta f$, joka
vastaa muuttujan pientä muutosta $\Delta x$. Jos approksimaation virhettä merkitään
\[
r(x,\Delta x) = f(x+\Delta x)-f(x)-f'(x)\Delta x,
\]
niin derivaatan määritelmän mukaisesti pätee
\[
\lim_{\Delta x \kohti 0} \frac{r(x,\Delta x)}{\Delta x} = 0.
\]
\begin{Exa}
Neliön muotoista mökkiä, joka on ulkomitoiltaan $6\text{ m}\times 6\text{ m}$, kaupataan 
$36\text{ m}^2$:n mökkinä. Arvioi mökin todellinen lattiapinta-ala differentiaalin avulla
olettaen seinän paksuudeksi $20\text{ cm}$.
\end{Exa}
\ratk Jos $f(x)=x^2$, niin lattiapinta-ala on differentiaalin avulla arvioiden
\[
A=f(x-\Delta x) \approx f(x)-f'(x)\Delta x = x^2-2x\Delta x.
\]
Arvoilla $x=6\text{ m}$, $\Delta x=0.4\text{ m}$ (huom!) saadaan
\[
A\approx 31.2\text{ m}^2.
\]
Approksimaation virhe on tässä tapauksessa tarkasti
\[
r(x,\Delta x) = (\Delta x)^2 = 0.16\text{ m}^2,
\]
eli 'todellinen' lattiapinta-ala on $A\approx 31.4\text{ m}^2$. \loppu

Jäljempänä Luvussa \ref{taylorin lause} näytetään, että differentiaaliin perustuvan
approksimaation virhe on yleisesti suuruusluokkaa $\,\sim (\Delta x)^2\,$ silloin kun
funktio on riittävän säännöllinen (kuten esimerkissä).
\begin{Exa}
Ideaalikaasun adiabaattisissa (äkillisissä) paineen ja tiheyden vaihteluissa pätee tilanyhtälö
\[
p\rho^{-\gamma}=K=\text{vakio},
\]
missä $\gamma$ on kaasulle ominainen vakio ($\gamma>1$). Jos tiheys muuttuu 2\%, niin paljonko
muuttuu paine?
\end{Exa}
\ratk Koska $\,p=K\rho^\gamma=f(\rho)$, niin
\begin{align*}
\Delta p &\approx f'(\rho)\Delta\rho =\gamma K\rho^{\gamma-1}\Delta\rho 
                                    =\gamma p\inv{\rho}\Delta\rho \\
         &\impl \ \frac{\Delta p}{p} \approx \gamma\frac{\Delta\rho}{\rho}
                                     =\underline{\underline{2\gamma\,\%}}. \loppu
\end{align*}

\subsection{Muutosnopeus} 
\index{muutosnopeus|vahv}

Derivaatan tavallisin tulkinta fysiikan ym.\ sovelluksissa on
\[
\boxed{\kehys\quad \text{Derivaatta}=\text{(hetkellinen) muutosnopeus}. \quad}
\]
Jos $m$ on jokin fysikaalinen suure, joka muuttuu ajan $(t)$ (mittayksikkö s) mukana, eli 
$m=m(t)$, niin hetkellinen muutosnopeus on
\[
\lim_{\Delta t\kohti 0} \frac{m(t+\Delta t)-m(t)}{\Delta t} 
                 = \lim_{\Delta t \kohti 0} \frac{\Delta m}{\Delta t} = m'(t).
\]
Jos $m$:n yksikkö on M, niin muutosnopeuden yksikkö on M/s.
\begin{Exa}
Jos $A=\text{kansantalous}$ (yksikkö E), niin talouskasvu on $A$:n hetkellinen muutosnopeus
(yksikkö E/s). Kun sanotaan 'talouskasvu kiihtyy' tai 'talouskasvu hidastuu', tarkoitetaan 
$A'(t)$:n muutosnopeutta, eli toista derivaattaa $A''(t)$ (yksikkö E/s$^2$). \loppu
\end{Exa}
Jos kappale liikkuu siten, että sen sijainti hetkellä $t$ on $x(t)$ (yksiulotteinen liike),
niin tunnetusti
\[
x'(t)=\text{\pain{no}p\pain{eus}},\quad x''(t)=\text{\pain{kiiht}y\pain{v}yy\pain{s}}.
\]
\begin{Exa}
Kaksimetrinen mies kävelee neljän metrin korkeudella olevan katulampun ali hetkellä $t=0$. 
Määritä miehen varjon kärjen paikka, nopeus ja kiihtyvyys hetkellä $t\geq 0$, kun mies kävelee
vakionopeudella $v_0=2$ m/s. (Ol.\ katu vaakasuora).
\end{Exa}
\ratk Varjon kärjen paikka määräytyy ehdosta (ks.\ kuvio alla)
\[
\frac{x(t)-v_0t}{2}=\frac{x(t)}{4},
\]
%\begin{multicols}{2} \raggedcolumns
joten
\begin{align*}
x(t)&=2v_0t, \\
x'(t)&=2v_0=4\,\text{m}/\text{s}, \\
x''(t)&=0. \quad\loppu
\end{align*}
\begin{figure}[H]
\setlength{\unitlength}{1cm}
\begin{center}
\begin{picture}(7,4)(0,0)
\put(1,0){\vector(1,0){6}} \put(5.3,-0.5){$x(t)$}
\linethickness{0.05cm}
\put(0,0){\line(0,1){3.7}}
\curve(0,3.7,0.5,4,1,3.8)
\curve(1,3.8,0.9,3.75,0.8,3.6)
\put(0.8,3.6){\line(1,0){0.4}}
\curve(1,3.8,1.1,3.75,1.2,3.6)
\thinlines
\dashline{0.2}(1,0)(1,3.6)
\drawline(1,3.6)(5.6,0)
\drawline(3,0)(3,2.0) \put(2.9,1.83){$\bullet$}
\drawline(3,0.7)(2.7,0)
\path(3,1.8)(3.2,1.2)(3.3,1.3)
\path(3,1.8)(2.8,1.3)(2.9,1.2)
\multiput(1,-0.5)(2,0){2}{\line(0,-1){0.2}}
\put(1.5,-0.6){\vector(-1,0){0.5}} \put(2.5,-0.6){\vector(1,0){0.5}} \put(1.8,-0.7){$v_0t$}
\end{picture}
\end{center}
\end{figure}
\index{zza@\sov!Moottori}%
\begin{Exa}: \vahv{Moottori}. Männänvarren, jonka pituus = 2 (mittayksikkö = $10$ cm),
kampiakseliin kiinnitetty pää liikkuu pitkin yksikköympyrää siten, että napakulma hetkellä $t$
on $\varphi(t)=at$ ($a=$ vakio), ja toinen pää (johon mäntä on kiinnitetty) liukuu pitkin
$x$-akselia. a) Arvioi differentiaalin avulla, paljonko mäntä liikkuu kampiakselin pyöriessä
kulmasta $\varphi=30^\circ$ kulmaan $\varphi+\Delta\varphi=31^\circ$. b) Mikä on männän nopeuden 
(vauhdin) maksimiarvo, kun kampiakselin pyörimisnopeus on $3600$ kierrosta/min ? 
\end{Exa}
\begin{figure}[H]
\setlength{\unitlength}{1cm}
%\begin{center}
\begin{picture}(10,4)(0,0)
\put(4,2){\bigcircle{4}}
\put(4,2){\line(1,0){5.156}}
%\dashline{0.2}(4,2)(8.472,2)
\put(3.9,1.89){$\bullet$} \put(5.302,3.302){$\bullet$} \put(9.054,1.9){$\bullet$}
\put(4,2){\arc{0.7}{-0.785}{0}} \put(4.5,2.2){$\varphi(t)$}
\put(4.3,2.8){$1$} \put(7,3){$2$}
\put(9.956,2){\vector(1,0){2}} \put(12.2,1.9){$x$}
\put(9.156,1.8){\line(0,-1){0.4}} \put(8.8,1){$x(\varphi)$}
\thicklines
%\put(4,2){\line(1,2){0.894}} \put(4.894,3.789){\line(2,-1){3.578}}
\path(4,2)(5.414,3.414)(9.156,2)
\multiput(9.156,1.8)(0.8,0){2}{\line(0,1){0.4}}
\multiput(9.156,1.8)(0,0.4){2}{\line(1,0){0.8}}
\end{picture}
\end{figure}
\ratk a) Kampiakselin kiertokulman ollessa $\varphi$ on männän paikka
\[
x(\varphi) = \cos\varphi + \sqrt{4-\sin^2\varphi} 
           = \cos\varphi + \sqrt{\rule{0mm}{4mm}3+\cos^2\varphi}.
\]
Kulman muuttuessa $\Delta\varphi$:n verran on
\[
\Delta x(\varphi) \approx x'(\varphi)\Delta\varphi
= -\left(\sin\varphi + \frac{\cos\varphi\sin\varphi}{\sqrt{3+\cos^2\varphi}}\right)\Delta\varphi.
\]
Arvoilla $\varphi=30^\circ,\ \Delta\varphi=1^\circ \vastaa \pi/180\,$ saadaan
\[
\abs{\Delta x} 
   \approx \frac{1}{2}\left(1+\frac{1}{\sqrt{5}}\right)\frac{\pi}{180}\cdot 10\,\text{cm} 
   \approx  \underline{\underline{1.3\ \text{mm}}}.
\]

b) Kun kirjoitetaan $x(t)=x(\varphi(t))=x(at)$, niin
\begin{align*}
x'(t)  &= a\left(-\sin at - \frac{\cos at\sin at}{\sqrt{3+\cos^2 at}}\right), \\
x''(t) &= a^2\left(-\cos at + \frac{\sin^2 at-\cos^2 at}{\sqrt{3+\cos^2 at}}
                            - \frac{\cos^2 at\,\sin^2 at}{(3+\cos^2 at)^{3/2}}\right).
\end{align*}
Nopeuden $x'(t)$ ääriarvokohdissa on oltava
\[
x''(t)=0\ \ \ekv\ \ \cos at\left(3+\cos^2 at\right)^{3/2} = 3-6\cos^2 at-\cos^4 at.
\]
Neliöimällä puolittain ja merkitsemällä $u=\cos^2 at$ tämä sievenee polynomiyhtälöksi
\[
u^3+u^2-21u+3=0.
\]
Välillä $[0,1]$ tällä on yksikäsitteinen ratkaisu
\[
u = 0.143987..\ \impl\ \cos\varphi(t) = \sqrt{0.143987..} = 0.379456..\ 
                \impl\ \varphi(t) \approx 68\aste.
\]
Tätä vastaa vauhdin maksimiarvo
\[
\abs{x'(t)} = a\sqrt{1-u}\left(1+\sqrt{\frac{u}{3+u}}\right) \approx 1.12\,a.
\]
Annetulla pyörimisnopeudella on $a=60 \cdot 2\pi/$s, joten numeroarvoksi saadaan
\[
v_{max} \approx 1.12 \cdot 120\pi\,\text{s}^{-1} \cdot 10\,\text{cm} 
                          \approx \underline{\underline{42\,\text{m/s}}}. \loppu
\]

\subsection{Differentiaali ja differentiaaliyhtälöt}
\index{differentiaaliyhtälö!a@eksponenttifunktion|vahv}

Differentiaaleihin perustuva ajattelu on hyvin tavallista silloin, kun luonnonilmiötä 
(tai muuta ilmiötä) kuvaava matemaattinen malli on differentiaaliyhtälö, ja halutaan 
j\pain{ohtaa} tämä yhtälö. Seuraavissa esimerkeissä päädytään eksponenttifunktion
differentiaaliyhtälöön $y'=ay$ fysikaalisesta (tai muusta) lainalaisuudesta muotoa
\[ 
\Delta y(x)\ =\ y(x+\Delta x)-y(x)\ \approx\ ay(x)\Delta x, \quad 
                                       \text{kun}\ \abs{\Delta x}\ \text{pieni}. 
\]
Kun tulkitaan tämän tarkoittavan, että
\[ 
\Delta y(x) = ay(x)\Delta x + r(x,\Delta x), 
\]
missä $\lim_{\Delta x \kohti 0} r(x,\Delta x)/\Delta x = 0$, niin jakamalla $\Delta x$:llä ja 
antamalla $\Delta x \kohti 0$ saadaan 'äärettömän pienille' muutoksille pätevä laki
\[ 
\lim_{\Delta x \kohti 0} \frac{\Delta y}{\Delta x} = \frac{dy}{dx} = ay(x). 
\]
\index{zza@\sov!Radioaktiivinen hajoaminen}%
\begin{Exa}: \vahv{Radioaktiivinen hajoaminen}. Radioaktiivisessa aineessa on yksittäisen
ytimen todennäköisyys hajota (ja niinmuodoin 'kadota') aikavälillä $[t,t+\Delta t]$
verrannollinen aikavälin pituuteen $\Delta t$, kun $\Delta t$ on pieni. Jos radioaktiivisten
(toistaiseksi hajoamattomien) ytimien lukumäärä hetkellä $t$ on $A(t)$, niin saadaan
likimääräiseksi hajoamislaiksi
\[
\frac{A(t+\Delta t)-A(t)}{A(t)}\ =\ \frac{\Delta A}{A(t)}\ \approx\ -a\Delta t,
\]
missä $a$ on ytimille ominainen vakio (dimensiottomana positiivinen, mittayksikkö 1/s). Rajalla
$\Delta t\kohti 0$ saadaan tarkka hajoamislaki
(vrt.\ Esimerkki \ref{eksponenttifunktio fysiikassa}:\,\ref{radioaktiivisuus})
\[
A'(t)=-aA(t) \qimpl A(t) = A(0)\,e^{-at}, \quad t \ge 0. \loppu
\]
\end{Exa}
\index{zza@\sov!Szy@Säteilyn vaimeneminen}%
\begin{Exa}: \vahv{Säteilyn vaimeneminen}. Säteilyn kulkiessa homogeenisessa väliaineessa on
yksityisen hiukkasen todennäköisyys törmätä väliaineen atomiytimeen verrannollinen kuljettuun
matkaan $\Delta x$, kun $\abs{\Delta x}$ on pieni. Jos säteilyn intensiteetti on $I(x)$
(hiukkasta/m$^2$/s), niin säteily vaimenee tällöin törmäysten johdosta määrällä
\[
I(x+\Delta x)-I(x)\ =\ \Delta I\ \approx\ -aI(x)\Delta x,
\]
missä $a$ on aineelle ominainen vakio (dimensiottomana positiivinen, mittayksikkö 1/m). Jos 
tässä oletetaan, että approksimaation virhe on $r(x,\Delta x)$, missä 
$\lim_{\Delta x \kohti 0} r(x,\Delta x)/\Delta x = 0$, niin jakamalla $\Delta x$:llä ja 
antamalla $\Delta x \kohti 0$ saadaan tarkka säteilyn vaimenemislaki
(vrt.\ Esimerkki \ref{eksponenttifunktio fysiikassa}:\,\ref{säteilyvaimennus})
\[
I'(x)=-aI(x) \qimpl I(x) = I(0)\,e^{-ax}, \quad x \ge 0. \loppu 
\]
\end{Exa}
\index{zza@\sov!Koronkorko}%
\begin{Exa}: \vahv{Koronkorko}. Pääoman kasvaessa jatkuvaa (koron)korkoa noudattaa pääoman
määrä $A(t)$ lakia
\[ 
A(t+\Delta t) - A(t)\ =\ \Delta A\ \approx\ aA(t)\Delta t \quad (\Delta t\ \text{pieni}), 
\]
missä $a = \tfrac{p}{100T}$, korkoprosentin ollessa $p$ ajassa $T$ (esim.\ $\,T = 1$ vuosi).
Rajalla $\Delta t \kohti 0$ päädytään pääoman kasvulakiin
\[ 
A'(t) = aA(t) \qimpl A(t) = A(0)\,e^{at}, \quad t \ge 0. \loppu 
\] 
\end{Exa}
\index{zza@\sov!Ilmanpaine}%
\begin{Exa}: \vahv{Ilmanpaine}.\ Määritä ilmanpaine korkeudella $x$ maan pinnasta. Oletetaan 
vakiolämpötila. \end{Exa}
\ratk Jos ilman tiheys korkeudella $x$ on $\rho(x)$, niin voimatasapainon perusteella saadaan
paineen muutoksen ja tiheyden välille yhteys
\[ 
\Delta p = p(x+\Delta x) - p(x) = - g\rho(x) \Delta x + r(x,\Delta x), 
\]
missä $g \approx 10\ \text{m/s}^2$ on maan vetovoiman kiihtyvyys ja 
$r(x,\Delta x)/\Delta x \kohti 0$, kun $\Delta x \kohti 0$. 
\begin{figure}[H]
\setlength{\unitlength}{1cm}
\begin{center}
\begin{picture}(6,4)
\put(0,2){\line(1,0){4} $ \quad x+\Delta x$}
\put(0,1){\line(1,0){4} $\quad x$}
\put(2,0){\vector(0,1){1} $\ p(x)$}
\put(2,3){\vector(0,-1){1} $\ p(x+\Delta x)$}
\put(1.75,1.4){$\rho(x)$}
\end{picture}
%\caption{Paineen differentiaalinen muutos}
\end{center}
\end{figure}
Kaasun tilanyhtälö vakiolämpötilassa on
\[ 
p/\rho = K = \text{vakio}. 
\]
Sijoittamalla tästä $\rho(x) = p(x)/K$ tasapainoyhtälöön ja antamalla $\Delta x \kohti 0$ 
päädytään paineen vähenemislakiin
\[ 
p' = - p/a \qimpl p(x) = p(0) e^{-x/a}, 
\]
missä
\[ 
a\ =\ \dfrac{K}{g}\ =\ \dfrac{p(0)}{\rho(0)g}\ 
      \approx\ \dfrac{10^5\ \text{N/m}^2}{1\ \text{kg/m}^3 \cdot 10\ \text{m/s}^2}\ 
   =\ 10\ \text{km}. 
\]
Paineen puoliarvokorkeus on $\,h_{1/2} = (\ln 2)a \approx 6.9$ km. \loppu

\Harj
\begin{enumerate}

\item
Laske $df(x,\Delta x)=f'(x)\Delta x$ ja $f(x+\Delta x)-f(x)$ kuudella desimaalilla, kun
$\,x=1$, $\Delta x=0.02\,$ ja \ a) \ $f(x)=x^2+200x+700,\quad$ b) \ $f(x)=x^{-2}-x^{-1},$ 
\newline
c) \ $f(x)=e^x,\quad$ d) $f(x)=\ln(1+x),\quad$ e) \ $f(x)=\tan\tfrac{\pi x}{2},\quad$
f) \ $f(x)=x^x$.

\item
Arvioi differentiaalin avulla, paljonko kuution a) tilavuus kasvaa, kun särmä pitenee $p\%$, \ 
b) pinta-ala pienenee, kun tilavuus pienenee $p\%$.

\item
Lentokone lentää suoraan maassa olevan katsojan yli $10$ km:n korkeudella. Lentokoneen
näkyessä vaakatasoon nähden kulmassa $60\aste$ havaitaan, että kulma muuttuu $4.3\aste$
viidessä sekunnissa. Arvioi koneen nopeus käyttäen differentiaalia.

\item
Hiihtäjä, jonka paino on $80$ kg, laskee mäkeä, jonka kaltevuus on vakio. Hiihtäjän nopeus $v$
nousee arvoon $50$ km/h, jolloin painovoima, kitka ja ilmanvastus saavuttavat tasapainon.
Painovoima ja kitkavoima ovat verrannollisia hiihtäjän massaan $m$ ja ilmanvastusvoima on
$F_v=\text{vakio}\times Av^\gamma$, missä $A$ on hiihtäjän efektiivinen poikkipinta-ala ja 
$\gamma \ge 1$ on vakio. \newline
Oletetaan lisäksi, että $m=\text{vakio}\times L^{2.4}$ ja $A=\text{vakio}\times L^{1.2}$, missä 
$L$ on hiihtäjän vatsanympärys. Arvioi tapauksissa a) $\gamma =1$, b) $\gamma=2$, kuinka paljon
prosentteina hiihtäjän laskuvauhti kasvaa tai pienenee hänen lihottuaan $2$ kg. Käytä 
differentiaalia! (Muut olosuhteet, kuten keli, oletetaan vakioksi.)

\item
a) Valmistettaessa $x$ kpl jääkaappeja on tuotantokustannus
\[
K(x)=8000+200x+\max\{0,\,200x-0.5x^2\}.
\]
Arvioi differentiaalin avulla yhden jääkaapin nk.\ marginaalinen tuotantokustannus, eli luku
$K(x+1)-K(x)$, kun $x=200$. \vspace{1mm}\newline
b) Erään valtion verotuksessa on marginaalinen (eli pienen lisätulon) veroprosentti vuositulon
$x$ (yksikkö $10^5$ euroa) funktiona
\[
p(x)=20+10\,\min\{x+3x^2,\,4\}.
\]
Millä vuositulon arvolla marginaalinen veroprosentti $=50$\,? Entä millä vuositulon arvolla
vero on puolet vuositulosta?

\item
a) Kuution pinta-ala kasvaa $50\ \text{cm}^2/\text{s}$. Mikä on tilavuuden kasvunopeus, kun
särmän pituus $=20$ cm? \vspace{1mm}\newline
b) Vesisäiliön tilavuus on $400$ l. Säiliöstä lasketaan vettä niin, että veden määrä
(yksikkö l) säiliössä hetkellä $t$ (yksikkö min) on
\[
V(t)=(20-t)^2, \quad 0 \le t \le 20.
\]
Mikä on veden hetkellinen virtausnopeus säiliöstä silloin, kun säiliö on vettä puolillaan?

\item
Piste $P$ liikkuu pitkin $x$-akselia negatiiviseen suuntaan vakionopeudella $v_1$ ja piste $Q$
liikkuu $y$-akselia pitkin negatiiviseen suuntaan vakionopeudella $v_2$. Hetkellä $t=0$ on
$P=(a,0)$ ja $Q=(0,b)$, missä $a,b>0$. Jos $s(t)$ janan $PQ$ pituus hetkellä $t$, niin millä 
hetkellä on $s'(t)=0$, ja mikä on $s$:n pienin arvo?

\item \index{zzb@\nim!IKEA}
(IKEA) Kaupan pysäköintialueelle tulee aikavälillä $[0,T]$ $Q_{in}(t)$ autoa/h ja alueelta
lähtee $Q_{out}(t)$ autoa/h (hetkellisiä arvoja). Johda differentiaaliyhtälö
(= autojen säilymislaki!) pysäköintialueella olevien autojen määrälle $m(t)$ hetkellä $t$, ja
ratkaise $m(t)$ aikavälillä $\,[0,T]\,$, kun $\,T=6$h, pysäköinti\-alueella on $120$ autoa
hetkellä $t=0$ ja
\[
Q_{in}(t)=200(1-t/T), \quad Q_{out}(t)=270(t/T)^2.
\]
Jos halutaan vain selvittää, millä ajan hetkellä pysäköintialueella on eniten autoja, niin
miten tämä saadaan yksinkertaisimmin selville?

\item (*) \index{zzb@\nim!Lusi}
(Lusi) Moottoritie kapenee kaksikaistaiseksi tieksi L:ssä. Moottoritietä pitkin tulee L:ään
$Q(t)$ autoa/h (hetkellinen arvo), ja L:stä alkavalle kapeammalle tielle mahtuu ajamaan
enintään $Q_0=1800$ autoa/h. Olkoon $m(t)=$ L:ään ruuhkautuneiden autojen määrä hetkellä
$t$ (aikayksikkö =h). Muodosta ruuhkautumisen matemaattinen malli ja ratkaise sen avulla
$m(t)$ aikavälillä $t\in[0,20]$, kun tiedetään, että $m(0)=0$ ja
\[
Q(t)=540+\max\,\{0,\,1440t-180t^2\}.
\]

\item (*) \index{zzb@\nim!Puimakone} 
(Puimakone) Hihnapyörän keskipiste on origossa ja säde on $R=0.3$ (yksikkö m). Hihnapyörää
pyörittää myötäpäivään hihna, joka koskettaa pyörää napakulmissa $\varphi\in[0,\pi]$. Hihnaa
liikutetaan toisella samansäteisellä, moottorikäyttöisellä hihnapyörällä, joka sijaitsee
negatiivisella $y$-akselilla.
\begin{multicols}{2} \raggedcolumns
Olkoon $F(\varphi)$ hihnaa jännittävä voima napakulmassa $\varphi,\ \varphi\in[0,\pi]$ ja
olkoon $\mu$ hihnan ja hihnapyörän välinen kitkakerroin. Tarkastelemalla välillä 
$[\varphi-\Delta\varphi/2,\varphi+\Delta\varphi/2]$ olevan hihnan palan voimatasapainoa johda 
differentiaaliyhtälö
\[
F'(\varphi) = -\mu F(\varphi), \quad \varphi\in(0,\pi)
\]
ja laske hihnaa jännittävät voimat $F_1$ ja $F_2$ hihnapyörien välisissä hihnan osissa
(ks.\ kuva), kun $\mu=0.2$ ja hihnapyörää vääntävä momentti on $M=(F_1-F_2)R=1200$ Nm. 
\begin{figure}[H]
\setlength{\unitlength}{1cm}
\begin{picture}(5,6)(0,0)
\thicklines
\put(3,3){\bigcircle{3}}
\put(4.5,3){\line(0,-1){3}} \put(1.5,3){\line(0,-1){3}}
\put(3,1.5){\vector(-1,0){0.2}}
\put(4.5,0.1){\vector(0,-1){0.1}} \put(1.5,0.1){\vector(0,-1){0.1}}
\put(3,3){\line(1,1){1.061}} \put(3,3){\vector(1,0){2.5}} \put(3,3){\vector(0,1){2.5}}
\thinlines
\put(3.3,3.8){$R$}
\put(3,3){\arc{0.7}{-0.785}{0}} \put(3.6,3.2){$\varphi$}
\put(5.7,2.9){$x$} \put(3.2,5.5){$y$}
\put(1.65,0){$F_2$} \put(4.65,0){$F_1$}
\end{picture}
\end{figure}
\end{multicols}

\end{enumerate} %Differentiaali ja muutosnopeus
\section{Käyrän tangentti ja normaali} \label{derivaatta geometriassa}
\alku \sectionmark{Tangentti ja normaali}
\index{tangentti (käyrän)|vahv} \index{normaali(vektori)!a@suoran, käyrän|vahv}
\index{kzyyrzy@käyrä|vahv} \index{parametrinen käyrä|vahv}

Derivaatan tavanomainen geometrinen tulkinta euklidisessa tasossa on:
\index{kulmakerroin}%
\[
\boxed{\kehys\quad \text{Derivaatta}=\text{tangentin kulmakerroin}. \quad}
\]
Käyrän $y=f(x)$ sivuaja eli \kor{tangentti} pisteessä $(c,f(c))$ on pisteiden $(c,f(c))$ ja 
$(c+\Delta x,f(c+\Delta x))$ kulkevan suoran eli
\index{sekantti (käyrän)}%
\kor{sekantin} 'raja-arvo', kun 
$\Delta x\kohti 0$. Pisteen $(c,f(c))$ kautta kulkeva suora, joka on tangenttia vastaan 
kohtisuora, on käyrän $y=f(x)$ \kor{normaali} ko. pisteessä.
\begin{figure}[H]
\setlength{\unitlength}{1cm}
\begin{center}
\begin{picture}(10,8)(-1,-1)
\put(-1,0){\vector(1,0){10}} \put(8.8,-0.4){$x$}
\put(0,-1){\vector(0,1){8}} \put(0.2,6.8){$y$}
\Thicklines
\curve(
   -1.0000,    0.4586,
   -0.5000,    0.7987,
         0,    1.0117,
    0.5000,    1.1249,
    1.0000,    1.1657,
    1.5000,    1.1614,
    2.0000,    1.1394,
    2.5000,    1.1269,
    3.0000,    1.1514,
    3.5000,    1.2402,
    4.0000,    1.4206,
    4.5000,    1.7200,
    5.0000,    2.1657,
    5.5000,    2.7851,
    6.0000,    3.6054,
    6.5000,    4.6542,
    7.0000,    5.9586)
\thinlines
\put(3.28,0.9){\line(1,1){3.8}}
\put(1.8,0.66){\line(3,1){6}}
\put(2.6,4.5){\line(1,-3){1.3}}
\put(2,4.7){normaali}
\put(7,3){tangentti}
\put(7,5){sekantti}
\dashline{0.1}(3.67,0)(3.67,1.25)
\dashline{0.1}(6,0)(6,3.6)
\put(3.57,-0.5){$c$} \put(5.9,-0.5){$c+\Delta x$}
\put(6.5,6.1){$y=f(x)$}
\end{picture}
%\caption{Derivaatan geometrinen tulkinta}
\end{center}
\end{figure}
Tangentin ja normaalin yhtälöt ovat
\begin{align*}
\text{Tangentti}\,: &\qquad y-f(c) = f'(c)(x-c). \\
\text{Normaali}\,:  &\qquad x-c = -f'(c)(y-f(c)).
\end{align*}
\index{kohtisuora leikkaus!a@käyrien}%
Sanotaan, että kaksi käyrää \kor{leikkaavat kohtisuorasti}, jos leikkauspisteessä tangentit
ovat kohtisuorassa toisiaan vastaan.
\begin{Exa}
Suora kulkee pisteen $(4,0)$ kautta ja leikkaa käyrän $y=x^2$ kohtisuorasti. Suoran yhtälö?
\end{Exa}
\ratk Suora on käyrän normaali leikkauspisteessä. Jos leikkauspiste on $(t,t^2)$, niin suoran 
yhtälö on siis
\[
x-t=-2t(y-t^2).
\]
Tämä kulkee pisteen $(4,0)$ kautta ehdolla
\[
2t^3+t-4=0 \ \impl \ t\approx 1.12817390.
\]
Suoran yhtälö: $\ y=k(x-4)$, $\ k\approx -0.44319409$. \loppu
\begin{Exa} \label{kohtisuora leikkaus} \index{kohtisuora leikkaus!b@käyräparvien}
Näytä, että \kor{käyräparvet} \index{kzyyrzy@käyräparvi}
\[
xy=a,\quad x^2-y^2=b,
\]
missä $a,b\in\R$, $a\neq 0$, leikkaavat toisensa kohtisuorasti.
\end{Exa}
\ratk Yhteisissä pisteissä $(x,y)$ on oltava $x\neq 0$, $y\neq 0$, koska $xy=a\neq 0$. Kun 
merkitään
\[
xy=a \ \ekv \ y=y(x)\quad (x\neq 0),
\]
niin implisiittisesti derivoimalla saadaan
\begin{align*}
\frac{d}{dx}[xy(x)]&=y(x)+xy'(x)=\frac{d}{dx}a=0 \\
&\impl \ y'(x)=-y(x)/x=-y/x.
\end{align*}
Vastaavasti voidaan merkitä
\[
x^2-y^2=b \ \ekv \ y=y(x),
\]
jolloin kyse on kaksihaaraisesta implisiittifunktiosta. Molemmille haaroille pätee 
derivoimissääntö
\begin{align*}
\frac{d}{dx}\left(x^2-[y(x)]^2\right)&= 2x-2y(x)y'(x)=\frac{d}{dx}b=0 \\
&\impl \ y'(x)=x/y(x)=x/y.
\end{align*}
Käyrien yhteisissä pisteissä $(x,y)$ tangenttien kulmakertoimien tulo on siis 
\[
-\frac{y}{x}\cdot\frac{x}{y} = -1,
\]
joten tangentit ovat kaikissa leikkauspisteissä kohtisuorat. \loppu
\index{zza@\sov!Heijastuslaki}%
\begin{Exa}: \vahv{Heijastuslaki}. Halutaan määrittää toistaiseksi tuntemattomalla välillä
$\,A\subset\R\,$ funktio $\,x \in A \map y(x) \ge 0\,$ ja vastaava käyränkaari
\[
S = \{\,(x,y) \in \Rkaksi \mid x \in A\ \ja\ y = y(x) \ge 0\,\}
\]
siten, että jokainen pisteestä $(-c,0)$ lähtenyt ja käyrästä heijastunut valonsäde kulkee
pisteen $(c,0)$ kautta ($c>0$). Muotoile ongelma differentiaaliyhtälöksi muotoa
\[ 
F(x,y,y') = 0, 
\]
\index{ellipsi}%
ja etsi ratkaisuja \kor{ellipsin} kaarien
\[
y=y(x) \ \ekv \ \frac{x^2}{a^2}+\frac{y^2}{b^2}=1
\]
joukosta ($a,b\in\R_+$).
\end{Exa}
\ratk Valonsäteen heijastuessa pisteessä $P=(x,y)$ on tulokulma sama kuin heijastuskulma
käyrän tangenttiin nähden.
\begin{figure}[H]
\setlength{\unitlength}{1cm}
\begin{center}
\begin{picture}(10,5)(-5,-1)
\put(-5,0){\vector(1,0){10}} \put(4.8,-0.4){$x$}
\put(0,0){\vector(0,1){4}} \put(0.2,3.8){$y$}
\path(-4,0)(1,3)(4,0)
\dashline{0.1}(1,3)(4,3)
\put(1,3){\line(5,-1){3}}\put(1,3){\line(-5,1){3}}
\put(4,2.4){\line(-2,3){0.1}}
\put(4,2.4){\line(-3,-2){0.15}}
\put(0,0){\vector(1,0){1}} \put(0,0){\vector(0,1){1}} \put(0.9,-0.5){$\vec i$}
\put(0.2,0.8){$\vec j$}
\multiput(-4,0)(8,0){2}{\line(0,-1){0.1}} \put(-4.4,-0.5){$-c$} \put(3.9,-0.5){$c$}
\put(-4.2,0.2){$A$} \put(3.9,0.2){$B$} \put(0.8,3.3){$P=(x,y)$} \put(4.2,2.2){$\vec t$}
\put(0.9,2.9){$\bullet$}
\put(1,3){\arc{3}{0}{0.197}}
\put(1,3){\arc{1}{0.197}{0.785}}
\put(1,3){\arc{1}{2.6}{3.35}}
\put(2.65,2.75){$\scriptstyle{\beta}$} \put(1.6,2.55){$\alpha$} \put(0.2,2.8){$\alpha$}
\end{picture}
%\caption{Valonsäteen heijastuminen}
\end{center}
\end{figure}
Käyrän tangentin suuntainen vektori heijastuspisteessä on (vrt.\ kuvio)
\[
\vec t=\vec i +y'\vec j,
\]
ja vektorit $\overrightarrow{AP}$ ja $\overrightarrow{PB}$ ovat (kuvio)
\[
\overrightarrow{AP}=(x+c)\vec i+y\vec j,\quad \overrightarrow{PB}=(c-x)\vec i -y\vec j.
\]
Vaadittu heijastusehto on
\begin{align*}
&\frac{\overrightarrow{AP}\cdot\vec t}{\abs{\overrightarrow{AP}}}=
\frac{\overrightarrow{PB}\cdot\vec t}{\abs{\overrightarrow{PB}}} \\
&\ekv \ \frac{(c+x)+yy'}{\sqrt{(c+x)^2+y^2}}=\frac{(c-x)-yy'}{\sqrt{(c-x)^2+y^2}}\,.
\end{align*}
Neliöön korottamalla ja sieventämällä tämä yksinkertaistuu yhtälöksi
\[
F(x,y,y') = xy^2(y')^2-(c^2-x^2+y^2)yy'-xy^2=0.
\]
Kokeillaan, toteutuuko tämä, kun
\[
y^2=b^2-\frac{b^2}{a^2}x^2 \ \impl \ y'=-\frac{b^2}{a^2}\,x/y.
\]
Näillä sijoituksilla differentiaaliyhtälö pelkistyy yhtälöksi
\[
\left(\frac{c^2+b^2}{a^2}-1\right) b^2x=0,
\]
joka toteutuu $\forall x$ ehdolla $\,a^2=c^2+b^2$. Tässä $b\in\R_+$ on vapaasti valittavissa,
joten ongelman ratkaisuja ovat
\[
y=y(x) = b\,\sqrt{1-\frac{x^2}{a^2}}\,,\quad x\in[-a,a]=A,\,\ a=\sqrt{c^2+b^2},\,\ b\in\R_+.
\]
\index{polttopiste (ellipsin)}%
Nämä ovat erimuotoisia ellipsin puolikaaria, joiden \kor{polttopisteinä} ovat $(\pm c,0)$. 
Kuvassa on $c=1$. \loppu
\begin{figure}[H]
\setlength{\unitlength}{1cm}
\begin{center}
\begin{picture}(10,4)(-5,0)
\put(-5,0){\vector(1,0){10}} \put(4.8,-0.4){$x$}
\put(0,0){\vector(0,1){4}} \put(0.2,3.8){$y$}
\curve(
   -4.26,    0,
   -4.2000,    0.4243,
   -4.1000,    0.7714,
   -4.0000,    1.0000,
   -3.9000,    1.1811,
   -3.6000,    1.5875,
   -3.3000,    1.8855,
   -3.0000,    2.1213,
   -2.7000,    2.3141,
   -2.4000,    2.4739,
   -2.1000,    2.6067,
   -1.8000,    2.7166,
   -1.5000,    2.8062,
   -1.2000,    2.8775,
   -0.9000,    2.9317,
   -0.6000,    2.9698,
   -0.3000,    2.9925,
         0,    3.0000,
    0.3000,    2.9925,
    0.6000,    2.9698,
    0.9000,    2.9317,
    1.2000,    2.8775,
    1.5000,    2.8062,
    1.8000,    2.7166,
    2.1000,    2.6067,
    2.4000,    2.4739,
    2.7000,    2.3141,
    3.0000,    2.1213,
    3.3000,    1.8855,
    3.6000,    1.5875,
    3.9000,    1.1811,
    4.0000,    1.0000,
    4.1000,    0.7714,
    4.2000,    0.4243,
    4.26,    0)
\curve(
  -3.3541,                  0,
  -3.3300,             0.1795,
  -3.3000,             0.2683,
  -3.2000,             0.4494,
  -3.1000,             0.5727,
  -3.0000,             0.6708,
  -2.7000,             0.8899,
  -2.4000,             1.0479,
  -1.8000,             1.2657,
  -1.2000,             1.4007,
  -0.6000,             1.4758,
        0,             1.5000,
   0.6000,             1.4758,
   1.2000,             1.4007,
   1.8000,             1.2657,
   2.4000,             1.0479,
   2.7000,             0.8899,
   3.0000,             0.6708,
   3.1000,             0.5727,
   3.2000,             0.4494,
   3.3000,             0.2683,
   3.3300,             0.1795,
   3.3541,                  0)
\multiput(-3,0)(6,0){2}{\line(0,-1){0.1}} \put(-3.4,-0.5){$-1$} \put(2.9,-0.5){$1$}
\put(-0.075,2.925){$\scriptstyle{\bullet}$} \put(0.2,2.6){$1$}
\put(0.5,1.6){$b=\tfrac{1}{2}$} \put(1,3){$b=1$}
\end{picture}
%\caption{$c=1$}
\end{center}
\end{figure}

\subsection{Parametrisen käyrän tangentti}

Parametrisen käyrän, eli vektoriarvoisen funktion $t \in A\ \map \vec r\,(t) \in \R^d$, missä
$A \subset \R$ ja $d=2$ tai $d=3$ (vrt.\ Luku \ref{parametriset käyrät}), derivaatta $\dvr(t)$ 
määritellään tavanomaiseen tapaan, eli
\[ 
\dvr(t) = \lim_{\Delta t \kohti 0}\dfrac{1}{\Delta t}\left[\vec r\,(t+\Delta t)-\vec r\,(t)\right]
        = \begin{cases} \begin{aligned}
          x'(t)\vec i + y'(t)\vec j \quad\quad\quad\quad\ \ &\text{(tasokäyrä)} \\
          x'(t)\vec i + y'(t)\vec j + z'(t)\vec k \quad     &\text{(avaruuskäyrä)}
          \end{aligned} \end{cases} \]
sellaisissa pisteissä, joissa funktiot $x(t)$, $y(t)$ ja $z(t)$ ovat derivoituvia. Määritelmän
mukaan derivaattavektori on pisteiden $P \vastaa \vec r\,(t)$ ja
$Q \vastaa \vec r\,(t + \Delta t)$ kulkevan suoran suuntavektorin raja-arvo, sikäli kuin pisteet
$P$ ja $Q$ ovat erillisiä $\abs{\Delta t}$:n ollessa riittävän pieni. Viimeksi mainittu ehto on
voimassa ainakin, jos
\begin{equation} \label{pysähtymättömyysehto}
 \abs{\dvr(t)} \neq 0,
\end{equation}
sillä linearisoivan approksimaatioperiaatteen mukaisesti pätee
\[
\abs{\vec r\,(t+\Delta t)-\vec r\,(t)}\ =\ \abs{\dvr(t)}\abs{\Delta t}+g(t,\Delta t), \quad 
                                    \lim_{\Delta t \kohti 0} \frac{g(t,\Delta t)}{\Delta t}=0,
\]
jolloin ehdolla \eqref{pysähtymättömyysehto} on tässä vasen puoli $\neq 0$, kun
$\Delta t \in (-\delta,\delta)$ jollakin (riittävän pienellä) $\delta>0$. Ehdolla 
\eqref{pysähtymättömyysehto} voidaan siis vektori $\dvr(t) \neq \vec 0$ tulkita käyrän 
\index{tangenttivektori (käyrän)}%
$S = \{P(t) \vastaa \vec r\,(t) \mid t \in A\}$ \kor{tangenttivektoriksi} pisteessä 
$P(t) \vastaa \vec r\,(t)$. Vektorin $\dvr(t)$ suuntainen yksikkövektori $\tv(t)$ on nimeltään 
\kor{yksikkötangenttivektori}: \index{yksikkövektori!a@yksikkötangenttivektori}%
\begin{equation} \label{yksikkötangenttivektori} 
\boxed{\kehys\quad \tv(t) = \dfrac{\dvr(t)}{\abs{\dvr(t)}} \quad 
                     (\text{yksikkötangenttivektori}, \,\ \abs{\dvr(t)} \neq 0\,). \quad} 
\end{equation}
Huomattakoon, että yksikkötangenttivektori annetussa käyrän pisteessä $P$ on puhtaasti
g\pain{eometrinen} käsite: Mahdollisia $\tv$:n arvoja (sikäli kuin $\tv$ yleensä on 
määritelty) on vain kaksi. Ehdon \eqref{pysähtymättömyysehto} ollessa voimassa parametrisointi 
valitsee näistä toisen kaavan \eqref{yksikkötangenttivektori} mukaisesti, muuten $\tv$ 
\pain{ei} \pain{rii}p\pain{u} p\pain{arametrisoinnista}. 

Yksikkötangenttivektori voidaan tulkita selkeämmin geometrisesti, kun merkitään
\[
\Delta\vec r=\vec r\,(t+\Delta t)-\vec r\,(t).
\]
Jos pisteet $P \vastaa \vec r\,(t)$ ja $Q \vastaa \vec r\,(t + \Delta t)$ ovat erillisiä, niin 
$\Delta\vec r$ on pisteiden $P$ ja $Q$ kautta kulkevan käyrän sekantin suuntavektori. Ehdon 
\eqref{pysähtymättömyysehto} voimassa ollessa tarkkenee approksimaatio 
$\abs{\Delta\vec r\,} \approx \abs{\dvr(t)}\abs{\Delta t}$ rajalla $\Delta t \kohti 0$, jolloin 
yksikkötangenttivektorin molemmat arvot saadaan toispuolisina raja-arvoina
\begin{multicols}{2} \raggedcolumns
\[
\tv_\pm = \lim_{\Delta t\kohti 0^\pm} \frac{\Delta\vec r}{\abs{\Delta\vec r\,}}\,.
\]
\begin{figure}[H]
\setlength{\unitlength}{1cm}
\begin{center}
\begin{picture}(8,3)(0,1)
\put(0,3){\vector(3,-2){3}}
\put(0,3){\vector(1,0){7}}
\put(3,1){\vector(2,1){4}}
\curve(1,0.8,3,1,5,1.7,7,3,8,4)
\put(3,1){\vector(4,1){3.5}}
\put(6,3.2){$\scriptstyle{\vec r(t+\Delta t)}$}
\put(2.6,1.4){$\scriptstyle{\vec r(t)}$}
\put(6.3,1.5){$\scriptstyle{\tv}$}
\put(5.5,2.5){$\scriptstyle{\Delta\vec r}$}
\put(2.95,0.95){$\scriptstyle{\bullet}$}
\put(6.94,2.93){$\scriptstyle{\bullet}$}
\end{picture}
\end{center}
\end{figure}
\end{multicols}
Tutkittaessa käyrän geometriaa parametrisoinnin avulla asetetaan ehto
\eqref{pysähtymättömyysehto} usein lähtökohtaisesti 'mukavuusehtona' koko tarkasteltavalla 
välillä, jolloin kaava \eqref{yksikkötangenttivektori} on käytettävissä. Tyypillisesti 
oletetaan, että $\vec r\,(t)$:n koordinaattifunktiot $x(t)$, $y(t)$, $z(t)$ ovat
j\pain{atkuvasti} \pain{derivoituvia} suljetulla välillä $A = [a,b]$ ja että ehto
\eqref{pysähtymättömyysehto} on voimassa koko välillä, mukaanlukien päätepisteet, joissa
derivaatta $\dvr$ tulkitaan toispuoliseksi.
\begin{Exa} Ruuviviivan 
$\,\vec r\,(\varphi)=R\cos\varphi\vec i+R\sin\varphi\vec j+a\varphi\vec k\ (R>0,\ a \neq 0)$
yksikkötangenttivektori pisteessä $\,P(\varphi) = (x(\varphi),y(\varphi),z(\varphi))\,$ on
\[ 
\tv(\varphi) = \pm \dfrac{-R\sin\varphi\vec i + R\cos\varphi\vec j + a\vec k}{\sqrt{R^2+a^2}}
                = \pm \dfrac{1}{\sqrt{R^2+a^2}}\,(-y\vec i + x\vec j + a\vec k).  
\]
Tässä on $\,\abs{\dvr(\varphi)} = \sqrt{R^2+a^2} =$ vakio. \loppu 
\end{Exa}
\begin{Exa} Käyrä $\,y=x^2,\,z=x^3\,$ leikkaa tason $T$ kohtisuorasti pisteessä $(1,1,1)$. 
Tason yhtälö? 
\end{Exa}
\ratk Käyrän eräs parametrisointi on 
\[ 
x(t)=t,\ y(t)=t^2,\ z(t)=t^3\ \ekv\ \vec r\,(t) = t\vec i + t^2\vec j + t^3\vec k. 
\]
Taso $T$ kulkee pisteen $(1,1,1) \vastaa \vec r\,(1)$ kautta ja sen normaalivektori on 
$\vec n = \dvr(1) = \vec i + 2\vec j + 3\vec k$, joten tason yhtälö on 
(vrt.\ Luku \ref{suorat ja tasot})
\[ 
(x-1)+2(y-1)+3(z-1) = 0 \qekv x+2y+3z-6 = 0. \loppu 
\]

\subsection{Nopeusvektori}
\index{nopeusvektori|vahv}

Jos parametrisessa käyrässä $t \map \vec r\,(t)$ parametri $t$ on fysikaalinen aikamuuttuja ja 
$\vec r\,(t)$ on avaruudessa liikkuvan pisteen (partikkelin yms.) paikka hetkellä $t$, niin
ko.\ pisteen \pain{no}p\pain{eus}(vektori) ja \pain {ratano}p\pain{eus} eli \pain{vauhti} 
määritellään:
\[ \boxed{\begin{aligned} 
  \ykehys\quad \text{\pain{No}p\pain{eusvektori}:} \quad   
           &\vec v(t) = \dvr(t) = x'(t)\vec i + y'(t)\vec j + z'(t)\vec k. \quad \\
  \text{\pain{Vauhti}:} \quad\quad\quad\quad\ 
           &v(t) = \abs{\dvr(t)}. \akehys
          \end{aligned} } \]
Aikaparametrisoinnissa ehto \eqref{pysähtymättömyysehto} on siis 'pysähtymiskielto'.
\begin{Exa} \label{ympyräliike} Partikkeli liikkuu pitkin origokeskistä $R$-säteistä 
ympyräviivaa siten, että napakulma hetkellä $t$ on $\varphi(t)$. Nopeus ja vauhti hetkellä 
$t$\,? 
\end{Exa}
\ratk Partikkelin paikkavektori hetkellä $t$ on 
$\vec r\,(t) = R\cos\varphi(t)\vec i + R\sin\varphi(t)\vec j$, joten
\[ 
\vec v(t) = R\varphi'(t)\,[-\sin\varphi(t)\vec i + \cos\varphi(t)\vec j\,] 
          = \begin{cases} \begin{aligned} 
            v(t)\tv_+(t), \quad &\text{jos}\ \varphi'(t) \ge 0, \\
            v(t)\tv_-(t), \quad &\text{jos}\ \varphi'(t) < 0,
            \end{aligned} \end{cases} 
\]
missä $v(t) = R\abs{\varphi'(t)}$ on vauhti ja 
$\tv_\pm(t) = \pm[-\sin\varphi(t)\vec i + \cos\varphi(t)\vec j\,]$ on liikkeen suuntainen
yksikkövektori = liikeradan yksikkötangenttivektori. \loppu
\begin{Exa}: \vahv{Sykloidi}. \index{sykloidi} \index{zza@\sov!Sykloidi}
$R$-säteinen pyörä vierii liukumatta pitkin $x$-akselia siten, että pyörän keskipisteen
liikenopeus on $v_0\vec i$, $v_0=\text{vakio}$. Pyörän ulkokehän piste $P$ on hetkellä $t=0$
origossa. Mikä on $P$:n nopeus ajan funktiona?
\end{Exa}
\ratk 
\begin{multicols}{2} \raggedcolumns
Pisteen $P$ liikerata on sykloidi, jonka aikaparametrisaatio on 
(vrt.\ Luku \ref{parametriset käyrät})
\begin{align*}
x(t) &= v_0t-R\sin \varphi(t), \\
y(t) &= R-R\cos \varphi(t),
\end{align*}
missä
\[
R\varphi(t)=v_0t,
\]
\begin{figure}[H]
\setlength{\unitlength}{1cm}
\begin{center}
\begin{picture}(5,2)(0,1)
\put(0,0){\vector(1,0){4}} \put(3.8,-0.5){$x$}
\put(0,0){\vector(0,1){3}} \put(0.2,2.8){$y$}
\put(2,1.25){\circle{2.5}}
\dashline{0.2}(0,2.5)(4,2.5) \put(-0.5,2.4){$\scriptstyle{2R}$}
\dashline{0.1}(2,1.25)(0.8,1.6)
\put(2,1.25){\vector(1,0){1}} \put(2.6,1.4){$\scriptstyle{v_0\vec i}$}
\dashline{0.1}(2,0)(2,1.25)
\put(2,1.25){\arc{0.6}{1.59}{3.43}}
\put(1.2,0.85){$\scriptstyle{\varphi(t)}$}
\put(1.93,1.18){$\scriptstyle{\bullet}$} \put(0.73,1.53){$\scriptstyle{\bullet}$} 
\put(0.5,1.7){$\scriptstyle{P}$}
\end{picture}
\end{center}
\end{figure}
\end{multicols}
joten $P$:n paikkavektori hetkellä $t$ on
\[
\vec r\,(t)=\left[v_0t-R\sin\frac{v_0t}{R}\right]\,\vec i 
                + R\left[1-\cos\frac{v_0t}{R}\right]\,\vec j.
\]
Nopeus ja vauhti hetkellä $t$ ovat
\begin{align*}
&\vec v(t) = \dvr(t) = v_0\left[\left(1-\cos\frac{v_0t}{R}\right)\vec i 
                                   + \sin\frac{v_0t}{R}\vec j\right], \\
&v(t) = \abs{\dvr(t)} = \abs{v_0}\sqrt{2-2\cos(\frac{v_0t}{R})}
                      = 2\abs{v_0}\Bigl|\sin\dfrac{v_0t}{2R}\Bigr|.
\end{align*}
Vauhdin minimiarvo $v_{\text{min}} = 0$ saavutetaan aina kun piste $P$ koskettaa $x$-akselia ja
maksimiarvo $v_{\text{max}} = 2\abs{v_0}$ aina kun $P$ on korkeudella $2R$. \loppu
\begin{figure}[H]
\setlength{\unitlength}{1cm}
\begin{center}
\begin{picture}(8,3)(-0.5,0)
\put(0,0){\vector(1,0){7.5}} \put(7.3,-0.5){$x$}
\put(0,0){\vector(0,1){3}} \put(0.2,2.8){$y$}
\dashline{0.2}(0,2)(7.5,2) \put(-0.5,1.9){$\scriptstyle{2R}$}
\curve(
      0,         0,
    0.0206,    0.1224,
    0.1585,    0.4597,
    0.5025,    0.9293,
    1.0907,    1.4161,
    1.9015,    1.8011,
    2.8589,    1.9900,
    3.8508,    1.9365,
    4.7568,    1.6536,
    5.4775,    1.2108,
    5.9589,    0.7163,
    6.2055,    0.2913,
    6.2794,    0.0398,
    6.2849,    0.0234,
    6.3430,    0.2461,
    6.5620,    0.6534,
    7.0106,    1.1455)
\put(6.05,-0.4){$\scriptstyle{2\pi R}$}
\end{picture}
\end{center}
\end{figure}

\pagebreak

\Harj
\begin{enumerate}

\item
Määritä seuraavien tasokäyrien tangentin ja normaalin yhtälöt perusmuodossa $ax+by+c=0$
annetussa pisteessä: \newline
a) \ $y=x^3-2x^2+x+1, \quad (x,y)=(2,3)$ \newline
b) \ $y=\ln x, \quad (x,y)=(e,1)$ \newline
c) \ $x^2y^3-x^3y^2=12, \quad (x,y)=(-1,2)$ \newline
d) \ $x\sin(xy-y^2)=x^2-1, \quad (x,y)=(1,1)$ \newline
e) \ $x^y=y^x, \quad (x,y)=(2,4)$

\item
a) Määritä $a,b,c$ siten, että käyrät $y=x^2+ax+b$ ja $y=cx-x^2$ sivuavat toisiaan (eli niillä
on yhteinen tangentti) pisteessä $(1,3)$. \vspace{1mm}\newline
b) Määritä pisteen $(-4,23/2)$ kautta kulkevien käyrän $9y=x^2$ normaalien yhtälöt.
\vspace{1mm}\newline
c) Mihin käyrän $y=x^3$ pisteeseen asetettu normaali leikkaa $x$-akselin pisteessä $(4,0)$?
\vspace{1mm}\newline
d) Määritä käyrien $y=\sinh x$ ja $y=\cosh x$ pisteseen $x=a$ asetettujen tangenttien ja
normaalien leikkauspisteet. \vspace{1mm}\newline
e) Suora kulkee pisteen $(3,0)$ kautta ja leikkaa käyrän $y=e^x$ kohtisuorasti. Määritä suoran
yhtälö, tarvittaessa numeerisin apukeinoin. \vspace{1mm}\newline
f) Mikä suora leikkaa kohtisuorasti käyrät $y=e^x$ ja $y=\ln x$\,? Mikä on käyrien lyhin
etäisyys?

\item \index{logaritminen spiraali}
a) Määritä \kor{logaritmisen spiraalin} $\,r=e^\varphi,\ \varphi\in\R\,$ (napakoordinaatit)
tangentin ja normaalin yhtälöt pisteessä $(r,\varphi)=(1,0)$. \vspace{1mm}\newline
b) Näytä implisiittisellä derivoinnilla, että käyrällä $\,S:\ x^5+x^2y^3+y^5=1\,$
on ainakin yksi vaakasuora (eli $x$-akselin suuntainen) ja yksi pystysuora
($y$-akselin suuntainen) tangentti. Missä käyrän pisteissä nämä sijaitsevat?
 
\item
a) Millä ehdolla käyrät $y=e^{ax}$ ja $y=e^{bx}$ leikkaavat kohtisuorasti? Missä kulmassa
käyrät leikkaavat, jos $a=1$ ja $b=2$\,? \vspace{1mm}\newline
b) Laske käyrien $y=\Arcsin x$ ja $y=\Arccos x$ (tangenttien) välinen kulma käyrien 
leikkauspisteessä. \vspace{1mm}\newline
c) Näytä, että sikäli kuin käyrät $y=ae^x$ ja $y=\sqrt{b-2x}\ $ ($a,b\in\R$) leikkaavat, niin
ne leikkaavat kohtisuorasti. Millä ehdolla käyrät leikkaavat?

\item
Avaruussuora $S$ on avaruuskäyrän
\[
\vec r\,(t)=7\sqrt{2}\cos t\,\vec i+\sqrt{2}\sin t(3\vec i-2\vec j+6\vec k)
\]
tangentti pisteessä $(4,2,-6)$. Määritä $S$:n yhtälö parametrimuodossa.

\item
Määritä seuraavien avaruuskäyrien yksikkötangenttivektori annetussa käyrän pisteessä $P$ sekä
sen avaruustason yhtälö, jonka käyrä leikkaa kohtisuorasti $P$:ssä: \newline
a) \ $\vec r=t^3\,\vec i+(2t-t^2)\,\vec j+(3t-2t^4)\,\vec k,\quad P=(1,1,1)$ \newline
b) \ $\vec r=e^t\,\vec i-\ln(t+1)\,\vec j-\cos t\,\vec k,\quad P=(1,0,-1)$ \newline
c) \ $x=2\sqrt{2}\cos t,\ y=\sqrt{2}\sin t,\ z=4t,\quad P=(-2,1,3\pi)$

\item
Näytä, että ruuviviivan $\vec r\,(t)=\cos t\,\vec i+\sin t\,\vec j+t\,\vec k,\ t\in\R$
tangenttien ja $xy$-tason leikkauspisteet muodostavat tasokäyrän
\[
x=\cos t+t\sin t, \quad y=\sin t-t\cos t, \quad t\in\R.
\]

\item
Tasolla liikuu pistemäinen kappale siten, että hetkellä $t$ kappale on pisteessä 
$P(t)=(x(t),y(t))$, missä $x(t)=\sin f(t)$, $y(t)=1-\cos f(t)$, ja edelleen 
\[
f(t)=\begin{cases} \,t^2(3-t)^2, &\text{kun}\ t\in[0,3], \\ 0, &\text{muulloin}. \end{cases}
\] 
Millä ajan hetkillä kappaleen vauhti (=nopeuden itseisarvo) on suurimmillaan, ja millainen
geometrinen jälki kappaleen liikkeestä jää?

\item (*)
Määritä (tarvittaessa numeerisin keinoin) suoran $S$ yhtälö tiedoista: \newline
a) \ $S$ sivuaa käyriä $y=x^2$ ja $y=\ln x$. \newline
b) \ $S$ leikkaa kohtisuorasti käyrät $y=x^2$ ja $y=\ln x$.

\item (*)
Avaruuskäyrä
\[
S: \quad \begin{cases} \,x^3+2y^3=3x^2yz, \\ \,y^3+z^3=2xy \end{cases}
\]
leikkaa kohtisuorasti avaruustason $T$ pistessä $(1,1,1)$. Määritä $T$:n yhtälö
perusmuodossa $ax+by+cz+d=0$.

\item (*) \index{zzb@\nim!Palloaberraatio}
(Palloaberraatio) Avaruudessa $xy$-tasoa pitkin negatiivisen $y$-akselin suuntaan etenevä
valonsäde heijastuu pisteessä $P$ pallopeilistä 
\[
S:\quad x^2+(y-R)^2+z^2=R^2, \quad 0 \le y \le R, 
\]
jolloin heijastunut säde leikkaa $y$-akselin pisteessä $y=c$. Laske $c$:n lauseke 
$P$:n $x$-koordinaatin funktiona. Totea, että pienillä suhteen $|x|/R$ arvoilla on likimain
$c(x) \approx R/2$ ja että poikkeama (nk.\ palloaberraatio) on likimain
\[
c(x)-\frac{R}{2} \,\approx\, -\frac{3R}{4}\left(\frac{x}{R}\right)^2.
\]

\item (*) \label{H-dif-2: tutka} \index{zzb@\nim!Tutka}
(Tutka) Origosta lähtevät radioaallot  heijastuvat pyörähdyspinnasta, jonka symmetria-akseli on
$y$-kaseli ja profiili $xy$-tasossa on käyrä $y=y(x)$, $x\in \R$. Profiili on valittu siten,
että  heijastuneet aallot kulkevat positiivisen $y$-akselin suuntaan heijastuspisteestä
riippumatta. Näytä, että tämä ehto voidaan esittää differentiaaliyhtälönä 
\[
(xy'-y)^2=x^2+y^2.
\]
Etsi mahdolliset polynomiratkaisut muotoa $y(x)=ax^2+b$.

\item (*) \index{zzb@\nim!Polttolasi}
(Polttolasi) Lasista on valmistettu linssi, jonka optinen akseli (symmetria-akseli) on
$y$-akseli ja profiili $xy$-tasossa on
\[
A=\{(x,y)\in\Rkaksi \mid x\in[-a,a]\ \ja\ f(x) \le y \le b\},
\]
\begin{multicols}{2} \raggedcolumns
missä $f$ on parillinen funktio ja $f(0)=0$. Käyrän $y=f(x)$ muoto halutaan sellaiseksi, että
linssi kokoaa kaikki negatiivisen $y$-akselin suuntaan kulkevat, linssin läpäisseet valonsäteet
pisteeseen $(0,-c)$. Taittumislaki linssin kaarevalla pinnalla on $k\sin\alpha_1=\sin\alpha_2$,
missä $\alpha_1$ on säteen tulokulma (lasissa) pinnan normaalin suhteen, $\alpha_2$ on 
lähtökulma, ja $k>1$ on lasin ja ilman välinen taitekerroin. 
\begin{figure}[H]
\setlength{\unitlength}{1cm}
\begin{picture}(5,5.5)(0,0.5)
\thicklines
\put(3,6){\arc{6}{0.785}{2.356}}
\path(0.879,3.879)(5.121,3.879)
\thinlines
\path(4,5.5)(4,3.172)(3,1)
\path(3.333,5.057)(4.667,1.286)
\put(4,3.172){\arc{1}{1.231}{2.002}} \put(4,3.172){\arc{2.1}{-1.911}{-1.571}}
\put(3.6,4.5){$\alpha_1$} \put(3.75,2.2){$\alpha_2$} 
\put(2.93,0.93){$\scriptstyle{\bullet}$} \put(2.3,0.9){$-c$}
\put(3,3){\vector(1,0){2.5}} \put(3,0.5){\vector(0,1){5}}
\put(5.7,2.9){$x$} \put(3.2,5.5){$y$}
\end{picture}
\end{figure}
\end{multicols}
Näytä, että funktion $y=f(x)$ on toteutettava differentiaaliyhtälö
\[
\left[k^2x^2+(k^2-1)(y+c)^2\right](y')^2-2x(y+c)y'=x^2.
\]

\end{enumerate} %Käyrän tangentti ja normaali
\section{Käyrän kaarevuus} \label{käyrän kaarevuus}
\alku
\index{kaarevuus (käyrän)|vahv}
\index{kzyyrzy@käyrä|vahv}
\index{parametrinen käyrä|vahv}

Tarkastellaan tason tai avaruuden (parametrisoitua) käyrää
\[
S=\{P(t) \in \R^d \mid P(t) \vastaa \vec r\,(t),\,\ t\in [a,b]\},
\]
missä
\[
\vec r\,(t) = \begin{cases}
x(t)\vec i + y(t)\vec j, &(\Rkaksi) \\
x(t)\vec i + y(t)\vec j + z(t)\vec k. &(\Rkolme)
\end{cases}
\]
Jatkossa oletetaan, että $x(t)$, $y(t)$ ja $z(t)$ ovat kahdesti jatkuvasti derivoituvia
välillä $[a,b]$. Merkitään lisäksi $\,\abs{\dvr(t)}=v(t)\,$ ja asetetaan 'pysähtymiskielto'
(vrt.\ edellinen luku)
\begin{equation} \label{derivaattaehto}
v(t)=\abs{\dvr(t)}>0 \quad\forall t\in [a,b].
\end{equation}
Käyttöön tulevat myös seuraavat derivoimissäännöt vektoriarvoisille funktioille.
\begin{Prop} \label{vektorifunktioiden tulon derivointi} Jos $f$ on derivoituva pisteessä
$t$ ja
\[ 
\vec u(t) = u_1(t)\vec i + u_2(t)\vec j + u_3(t)\vec k, \quad 
\vec v(t) = v_1(t)\vec i + v_2(t)\vec j + v_3(t)\vec k, 
\]
missä funktiot $u_i$ ja $v_i$ ovat derivoituvia pisteessä $t$, niin pätee
\[ \boxed{ \begin{aligned}
\quad \frac{\ykehys d}{dt}\left[f(t)\vec u(t)\right] \quad\, 
            &=\,f'(t)\,\vec u(t) + f(t)\,\vec u\,'(t), \\
\frac{d}{dt}\left[\vec u(t)\cdot\vec v(t)\right]\,\          
            &=\,\vec u\,'(t)\cdot\vec v(t) + \vec u(t)\cdot\vec v\,'(t), \\
\frac{d}{\akehys dt}\left[\vec u(t)\times\vec v(t)\right]    
            &=\,\vec u\,'(t)\times\vec v(t) + \vec u(t)\times\vec v\,'(t). \quad
\end{aligned} } \]
\end{Prop}
\tod Kyse on tulon derivoimissäännön (Luku \ref{derivaatta}) yleistyksistä. Esimerkkinä
johdettakoon säännöistä toinen (muut perustellaan vastaavalla tavalla):
\begin{align*} 
\frac{d}{dt}\,\vec u(t)\cdot\vec v(t)\ 
               &=\ \frac{d}{dt}\,\sum_{i=1}^3 u_i(t)\,v_i(t)\ 
                =\ \sum_{i=1}^3 \frac{d}{dt}\,u_i(t)\,v_i(t) \\
               &=\ \sum_{i=1}^3\left[\,u_i'(t)v_i(t)+u_i(t)v_i'(t)\,\right] \\
               &=\ \sum_{i=1}^3 u_i'(t)\,v_i(t) + \sum_{i=1}^3 u_i(t)\,v_i'(t)
                =\ \vec u\,'(t)\cdot\vec v(t) + \vec u(t)\cdot\vec v\,'(t). \loppu
\end{align*}

Kerrattakoon edellisestä luvusta, että kun merkitään
\[
\Delta\vec r=\vec r\,(t+\Delta t)-\vec r\,(t) \approx \dvr(t)\Delta t,
\]
niin ehdon \eqref{derivaattaehto} ollessa voimassa voidaan käyrän $S$ yksikkötangenttivektori
$\tv(t)$ pisteessä $t \in [a,b]$ määrätä toispuolisina raja-arvoina
\[
\tv_\pm = \lim_{\Delta t\kohti 0^\pm} \frac{\Delta\vec r}{\abs{\Delta\vec r\,}}, 
                                                             \quad t \in [a,b].
\]
(Välin päätepisteissä on vain toinen raja-arvoista mahdollinen.) Funktion $t \map \vec r\,(t)$ 
ollessa kahdesti jatkuvasti derivoituva välillä $[a,b]$ voidaan edelleen laskea 
yksikkötangenttivektorin derivaatta $\tv\,'(t)$, kun $t \in [a,b]$ (toispuolinen derivaatta
välin päätepisteissä). Jos käyrä $S$ on jana, niin $\tv(t)$ on vakio, jolloin 
$\tv\,'(t) = \vec 0,\ t \in [a,b]$. Yleisemmin itseisarvo $\abs{d\tv/dt}$ kertoo, kuinka
\pain{kaareva} käyrä on ko.\ pisteessä. Kun tangenttivektorin $\tv$  muutosta merkitään
\[
\Delta\tv=\tv(t+\Delta t)-\tv(t),
\]
niin käyrän \kor{kaarevuus} (engl.\ curvature) määritellään raja-arvona
\begin{equation} \label{kaarevuuden peruskaava} \boxed{
\quad \kappa(t)=\frac{\ykehys 1}{\akehys R(t)}
               =\lim_{\Delta t\kohti 0} \frac{\abs{\Delta\tv\,}}{\abs{\Delta\vec r\,}}
               =\frac{1}{v(t)} \left|\frac{d\tv}{dt}\right| \quad \text{(kaarevuus).} \quad }
\end{equation}
Tässä siis $\,v(t)=\abs{\dvr(t)}$, ja $\,R=1/\kappa\,$ on nimeltään 
\index{kaarevuussäde}%
\kor{kaarevuussäde} (engl.\ radius of curvature).
\begin{figure}[H]
\setlength{\unitlength}{1cm}
\begin{center}
\begin{picture}(8,5)
\put(0,3){\vector(3,-2){3}}
\put(0,3){\vector(1,0){7}}
\put(3,1){\vector(2,1){4}}
\curve(1,0.8,3,1,5,1.7,7,3,8,4)
\put(3,1){\vector(4,1){3.5}}
\put(7,3){\vector(4,3){3}}
\put(6,3.2){$\scriptstyle{\vec r(t+\Delta t)}$}
\put(2.6,1.4){$\scriptstyle{\vec r(t)}$}
\put(6.3,1.5){$\scriptstyle{\tv}$}
\put(9.9,4.7){$\scriptstyle{\tv + \Delta\tv}$}
\put(5.5,2.5){$\scriptstyle{\Delta\vec r}$}
\put(2.95,0.95){$\scriptstyle{\bullet}$}
\put(6.94,2.93){$\scriptstyle{\bullet}$}
\end{picture}
\end{center}
\end{figure}
%\end{multicols}
Samoin kuin yksikkötangenttivektori, myös kaarevuus on vain käyrän geometriasta, ei 
parametrisoinnista riippuva.
\begin{Exa} Partikkeli liikkuu tasossa pitkin origokeskistä $R$-säteistä ympyrärataa siten, 
että napakulma hetkellä $t$ on $\varphi(t)$. Määritä liikeradan kaarevuus pisteessä 
$P(t) \vastaa \vec r\,(t)$, jossa $\varphi'(t) \neq 0$. \end{Exa}
\ratk Tässä on (vrt.\ Esimerkki \ref{ympyräliike} edellisessä luvussa)
\[ 
\vec r\,(t) = R[\cos\varphi(t)\vec i + \sin\varphi(t)\vec j\,], \quad  
\tv(t) = \pm[-\sin\varphi(t)\vec i + \cos\varphi(t)\vec j\,], 
\]
joten $v(t)=\abs{\dvr(t)}=R\abs{\varphi'(t)}$ ja $\abs{d\tv/dt}=\abs{\varphi'(t)}$. Kaavan
\eqref{kaarevuuden peruskaava} mukaan kaarevuus on $\kappa(t)=1/R=$ vakio, ja kaarevuussäde
siis $=R$, kuten odottaa sopikin. \loppu

Vektorin $d\tv/dt$ suuntaista yksikkövektoria $\vec n$ sanotaan käyrän 
\index{pzyzy@päänormaalivektori}%
\kor{päänormaalivektoriksi}. Tämä on todella käyrän normaalivektori, eli tangenttivektoria 
vastaan kohtisuora. Nimittäin koska $\abs{\tv(t)} = 1\ \forall t$, niin 
Proposition \ref{vektorifunktioiden tulon derivointi} säännöillä päätellään
\[ 
\tv(t)\cdot\tv(t) = 1 \qimpl \frac{d}{dt}\,\tv(t)\cdot\tv(t) 
                              = 2\,\tv(t)\cdot\tv\,'(t) = \frac{d}{dt}\,1 = 0. 
\]
(Yleisemmin on y\pain{ksikkövektorin} derivaatta aina vektoria vastaan kohtisuora.) Kaava 
\eqref{kaarevuuden peruskaava} huomioiden on siis päädytty vektorimuotoiseen kaarevuuden
määritelmään
\begin{equation} \label{kaarevuuskaava a}
\boxed{\quad \frac{\ykehys 1}{\akehys v(t)}\frac{d\tv}{dt}
              =\frac{1}{R}\vec n,\qquad \begin{cases} 
                                       \,1/R=\text{kaarevuus}, \\
                                       \,\vec n=\text{päänormaalivektori}. \quad 
                                       \end{cases}}
\end{equation}
\jatko\begin{Exa} (jatko) Tässä saadaan kaavasta \eqref{kaarevuuskaava a} odotusten
mukaisesti
\[
\vec n(t) = \frac{R}{v(t)}\frac{d\tv}{dt} 
          = -\cos\varphi(t)\vec i-\sin\varphi(t)\vec j = -\vec r\,(t)/R. \loppu
\]
\end{Exa}
Päänormaalivektori $\vec n$ on siis käyrän normaalivektoreista (kaksi vaihtoehtoa!) se, joka
osoittaa käyrän kaareutumissuuntaan. Kun derivoidaan puolittain yhtälöt
$\vec n(t)\cdot\vec n(t)=1$ ja $\vec n(t)\cdot\tv(t)=0$ ja sovelletaan kavaa
\eqref{kaarevuuskaava a}, niin seuraa
\[
\frac{d\vec n(t)}{dt}\cdot\vec n(t)=0, \quad
\frac{d\vec n(t)}{dt}\cdot\tv(t) = -\vec n(t)\cdot\frac{d\tv(t)}{dt}=-\frac{v(t)}{R}\,.
\]
Tämän mukaan kaarevuuden määritelmän \eqref{kaarevuuskaava a} voi esittää myös muodossa
\begin{equation} \label{kaarevuuskaava b}
\boxed{\quad \frac{\ykehys 1}{\akehys v(t)}\frac{d\vec n}{dt}
              =-\frac{1}{R}\tv. \quad }
\end{equation}
Johdetaan vielä kaarevuudelle lauseke, jonka arvo määräytyy suoraan derivaatoista $\dvr$ ja 
$\ddvr$. Määritelmän ja Proposition \ref{vektorifunktioiden tulon derivointi} ensimmäisen
säännön mukaan on
\[
\frac{1}{R}\vec n\ =\ \frac{1}{\abs{\dvr}}\frac{d}{dt}\left(\frac{1}{\abs{\dvr}}\,\dvr\right)\
             =\ \frac{1}{\abs{\dvr}^2}\,\ddvr+\frac{d}{dt}\left(\frac{1}{\abs{\dvr}}\right)\tv.
\]
Kertomalla tämä ristiin vektorilla $\tv$ ja ottamalla puolittain itseiarvot saadaan
\[
\frac{1}{R}\ =\ \frac{\abs{\ddvr\times\tv\,}}{\abs{\dvr}^2}\,.
\]
Kun tähän vielä sijoitetaan $\tv=\dvr/\abs{\dvr}$, tulee laskukaavaksi
\begin{equation} \label{kaarevuuden laskukaava}
\boxed{\quad \kappa = \frac{\ykehys 1}{\akehys R}
                    =\frac{\abs{\dvr\times\ddvr}}{\abs{\dvr}^3}\,. \quad}
\end{equation}
Tulkitaan tämä vielä tasokäyrälle $y=f(x)$. Kun parametrina on $t=x$, niin
$\,\vec r\,(x) = x\vec i + f(x)\vec j$, $\,\dvr = \vec i + f'(x)\vec j$,
$\,\ddvr=f''(x)\vec j$, joten $\,\dvr\times\ddvr=f''(x)\vec k$. Sijoittamalla tämä kaavaan
\eqref{kaarevuuden laskukaava} todetaan, että tasokäyrän kaarevuus pisteessä $(x,f(x))$ on
\begin{equation} \label{tasokäyrän kaarevuus}
\boxed{\quad \kappa=\frac{\ykehys 1}{\akehys R}
     =\frac{\abs{f''(x)}}{[1+(f'(x))^2\,]^{3/2}} \quad (\text{tasokäyrä $y=f(x)$}). \quad}
\end{equation}
\index{merkkinen kaarevuus}%
Ilman itseisarvomerkkejä tätä sanotaan \kor{merkkiseksi} kaarevuudeksi.
\begin{Exa} Käyrän $S:\ y = x^2$ kaarevuussäde pisteessä $(x,x^2)$ on kaavan 
\eqref{tasokäyrän kaarevuus} mukaan
\[ R(x) = \frac{1}{2}\,(1+4x^2)^{3/2}. \loppu \]
\end{Exa}

\subsection{Kaarevuuskeskiö. Evoluutta}
\index{kaarevuuskeskiö|vahv} \index{evoluutta|vahv}

Kun parametrisen käyrän pisteestä $P(t)\vastaa\vec r\,(t)$ kuljetaan kaarevuussäteen $R$
pituinen matka päänormaalivektorin $\vec n$ suuntaan, tullaan \kor{kaarevuuskeskiöön}. Tämän
paikkavektori on siis
\[
\boxed{\kehys\quad \vec r_0(t)=\vec r\,(t)+R(t)\vec n(t) \quad\text{(kaarevuuskeskiö)}. \quad}
\]
Geometrisesti voidaan tulkita niin, että pisteen $P(t)\vastaa\vec r\,(t)$ ympäristössä käyrä 
on likimain ympyräviiva, jonka säde $=\text{kaarevuussäde}\ R$ ja keskipiste =
kaarevuuskeskiö. Lisäksi tämä nk.\
\index{kaarevuusympyrä}%
\kor{kaarevuusympyrä} on tasossa, jonka suuntavektoreina
ovat $\tv$ ja $\vec n$. Kaarevuusympyrän 'vieriessä' pitkin käyrää piirtää kaarevuuskeskiö
toisen käyrän, jota sanotaan alkuperäisen käyrän \kor{evoluutaksi}.
\begin{Exa} \label{evoluutta} Määritä käyrän $\,y=1/x$, $x>0\,$ evoluutta.
\end{Exa}
\ratk Koska $f'(x)=-1/x^2$, niin käyrän (pää)normaalivektori pisteessä $(x,1/x)$ on
\[
\vec n \,=\, (1+x^{-4})^{-1/2}(x^{-2}\vec i + \vec j)
       \,=\, (x^4+1)^{-1/2}(\vec i + x^2\vec j).
\]
Kaarevuus ko. pisteessä on
\[
\frac{1}{R}=\frac{2x^{-3}}{(1+x^{-4})^{3/2}}=\frac{2x^3}{(x^4+1)^{3/2}}.
\]
Käyrän pistettä $(t,1/t)$ vastaava kaarevuuskeskiö $(x(t),y(t))$ on näin ollen
\begin{align*}
x(t) &\,=\, t+\frac{(t^4+1)^{3/2}}{2t^3}\cdot (t^4+1)^{-1/2} 
      \,=\, \frac{1}{2}(3t+t^{-3}), \\
y(t) &\,=\, t^{-1}+\frac{(t^4+1)^{3/2}}{2t^3}\cdot t^2(t^4+1)^{-1/2} 
      \,=\,\frac{1}{2}(3t^{-1}+t^3).
\end{align*}
Tämä on evoluutan parametriesitys ($t>0$). Suurilla ja pienillä $t$:n arvoilla on likimain
\[ 
\begin{cases}
\,t\gg 1\ \impl\ y\approx\frac{4}{27}x^3, \\
\,t\ll 1\ \impl\ x\approx\frac{4}{27}y^3.
\end{cases}
\]
Pisteessä $(x,y)=(2,2)$ ($t=1$) evoluutalla on nk.\
\index{kzyzy@kääntymispiste}%
\kor{kääntymispiste} (engl.\ turn\-ing point). Kääntymispisteessä on $x'(t)=y'(t)=0$, joten
kyseessä on myös 'pysähtymispiste'. \loppu
\begin{figure}[H]
\setlength{\unitlength}{1cm}
\begin{center}
\begin{picture}(8,8)(-0.5,0)
\put(-0.5,0){\vector(1,0){8}} \put(7.3,-0.5){$x$}
\put(0,-0.5){\vector(0,1){8}} \put(0.2,7.3){$y$}
\curve(
     2,      2,
2.0894,  2.114,
2.2822,2.44343,
2.5221, 2.9855,
2.7857,3.74933,
3.0625,   4.75,
 3.347,6.00582,
3.6362,  7.537)
\curve(
      2,     2,
  2.114,2.0894,
2.44343,2.2822,
 2.9855,2.5221,
3.74933,2.7857,
   4.75,3.0625,
6.00582, 3.347,
  7.537,3.6362)
\curvedashes[0.1cm]{0,1,2}
\curve(
  1,      1,
1.2,0.83333,
1.4,0.71429,
1.6,  0.625,
1.8,0.55556,
  2,    0.5,
2.2,0.45455,
2.4,0.41667,
2.6,0.38462,
2.8,0.35714,
  3,0.33333,
3.2, 0.3125,
3.4,0.29412,
3.6,0.27778,
3.8,0.26316,
  4,   0.25)
\curve(
      1,  1,
0.83333,1.2,
0.71429,1.4,
  0.625,1.6,
0.55556,1.8,
    0.5,  2,
0.45455,2.2,
0.41667,2.4,
0.38462,2.6,
0.35714,2.8,
0.33333,  3,
 0.3125,3.2,
0.29412,3.4,
0.27778,3.6,
0.26316,3.8,
   0.25,  4)
\put(1.1,1.1){$y=1/x$}
\put(1,0){\line(0,-1){0.1}}
\put(0,1){\line(-1,0){0.1}}
\put(0.93,-0.4){$\scriptstyle{1}$}
\put(-0.4,0.9){$\scriptstyle{1}$}
\end{picture}
\end{center}
\end{figure}

\pagebreak
\subsection{Kaarevuus fysiikassa: kiihtyvyys}
\index{kiihtyvyys|vahv}

Kun parametrisessa käyrässä on kyse liikkuvan pisteen $P(t) \vastaa \vec r(t)$ paikasta ajan 
funktiona, niin $\vec v(t)=\dvr\,(t)=$ nopeusvektori. Tällöin vektoria
\[ 
\vec a(t) = \vec v\,'(t) = \ddvr(t) = x''(t)\vec i + y''(t)\vec j + z''(t)\vec k 
\]
sanotaan \pain{kiiht}y\pain{v}yy\pain{deksi} hetkellä $t$. Kiihtyvyysvektori voidaan kaartuvalla
ratakäyrällä aina esittää muodossa
\[ 
\vec a(t) = a_\tau\,\tv + a_n\,\vec n, 
\]
missä $a_\tau$ on liikeradan suuntainen \pain{tan}g\pain{entiaalikiiht}y\pain{v}yy\pain{s} ja $a_n$
on päänormaalivektorin suuntainen \pain{normaalikiiht}y\pain{v}yy\pain{s}. Tähän tulokseen 
päädytään, kun kirjoitetaan
\[
\vec v(t)=v(t)\,\tv(t),\quad v(t)=\abs{\vec v(t)}=\abs{\dvr(t)}
\]
ja käytetään Proposition \ref{vektorifunktioiden tulon derivointi} ensimmäistä sääntöä sekä
kaavaa \eqref{kaarevuuskaava a}\,:
\begin{align*}
\vec a(t) &= v'(t)\,\tv(t)+v(t)\,\frac{d\tv}{dt} \\
          &=v'(t)\,\tv(t)+\frac{[v(t)]^2}{R}\,\vec n(t).
\end{align*}
Tämän mukaisesti on siis
\[ \boxed{\kehys\quad a_\tau(t) 
          = v'(t), \quad a_n(t) 
          = \frac{[v(t)]^2}{R} \quad \text{(tangentiaali- ja normaalikiihtyvyys)}. \quad} 
\]
Kiihtyvyyden kannalta voidaan siis liike ajatella hetkellisesti ympyräliikkeeksi, jossa 
liikerata yhtyy kaarevuusympyrään.
\index{zza@\sov!Vapaa putoamisliike}%
\begin{Exa}:\ \vahv{Vapaa putoamisliike}. Maan vetovoimakentässä hetkellä $t=0$ käynnistyvän
vapaan putoamisliikkeen (Newtonin) liikeyhtälö on
\[ 
\vec a(t) = \ddvr(t) = -g\vec k, \quad t>0, 
\]
missä $g \approx 9.81$ m/s$^2$. Yhtälö sisältää kolme skalaarista differentiaaliyhtälöä:
\[ 
x''(t) = 0, \quad y''(t) = 0, \quad z''(t) = -g. 
\]
Näiden mukaisesti on ensinnäkin oltava
\[ 
x'(t) = \alpha, \quad y'(t) = \beta, \quad z'(t) = -gt + \gamma, 
\]
missä $\alpha,\beta,\gamma$ ovat (määräämättömiä) vakioita. Tästä nähdään, että on edelleen
oltava (vrt.\ Luku \ref{väliarvolause 2})
\[ 
x(t) = \alpha\,t + x_0, \quad y(t) = \beta\,t + y_0, \quad 
                              z(t) = -\tfrac{1}{2}\,gt^2 + \gamma\,t + z_0, 
\]
missä $x_0,y_0,z_0$ ovat jälleen määräämättömiä vakioita. Siis
\begin{align*}
\vec r\,(t)&= \vr_0 + \vec v_0\,t - \tfrac{1}{2}\,gt^2\,\vec k, \quad t>0, \quad \text{missä} \\
\vec r_0   &= x_0\vec i + y_0\vec j + z_0\vec k = \vr(0^+), \quad 
   \vec v_0 = \alpha\vec i + \beta\vec j + \gamma\vec k = \dvr(0^+) = \vec v(0^+). 
\end{align*}
Jos raja-arvot $\vec r\,(0^+)=\vr_0$ ja $\vec v(0^+) = \vec v_0$ tunnetaan alkuehtoina, on
$\vec r\,(t)$ yksikäsitteisesti määrätty, kun $t \ge 0$. Ratakäyrä on tällöin paraabeli
avaruustasossa, jonka suuntavektorit ovat $\vec v_0$ ja $\vec k$
(vrt.\ Esimerkki \ref{parametriset käyrät}:\,\ref{heittoparaabeli}). \loppu 
\end{Exa}
\index{zza@\sov!Irtoaminen}%
\begin{Exa}:\ \vahv{Irtoaminen}. Kappale, johon vaikuttaa painovoima
\[
\vec G=-mg\vec j \quad (\text{$m=$ massa, $g=$ maan vetovoiman kiihtyvyys})
\]
on levossa origossa ja lähtee siitä kitkattomaan liukuun pitkin käyrää 
\[
y=-\frac{1}{3}\,x^3,\quad x\geq 0.
\]
Määritä kappaleen liikerata muodossa $y=f(x),\ x\ge0$.
\end{Exa}
\ratk Rata noudattaa aluksi käyrää $y=-x^3/3$, mutta irtoaa siitä, kun tämän radan mukainen 
normaalikiihtyvyys ylittää maan vetovoiman kiihtyvyyden normaalin suunnalla. Irtoamisehto on 
siis
\[
\frac{v^2}{R}=-g\vec j\cdot\vec n,
\]
missä $\vec n$ on (pää)normaalivektori
\[
\vec n=-\frac{1}{\sqrt{1+x^4}}(x^2\vec i + \vec j).
\]
Energiaperiaatteen mukaan on
\[
\frac{1}{2}m[v(x)]^2 = mg\cdot \frac{1}{3}\,x^3,
\]
ja kaarevuus pisteessä $x$ on
\[
\frac{1}{R}=\frac{2x}{(1+x^4)^{3/2}}\,,
\]
joten irtoaminen tapahtuu, kun
\begin{align*}
\frac{2x}{(1+x^4)^{3/2}}\cdot\frac{2}{3}gx^3 &= \frac{g}{(1+x^4)^{1/2}} \\
\ekv \ x^4=3 \ &\ekv \ \underline{\underline{x=\sqrt[4]{3}=1.316074..}}
\end{align*}
Irtoamisen jälkeen lentorata on paraabeli (vapaa putoamisliike), joten koko liikerata on
muotoa
\[
y=f(x)=\begin{cases}
-x^3/3,     &\text{kun}\ 0\leq x\leq\sqrt[4]{3}, \\
Ax^2+Bx+C,  &\text{kun}\ x>\sqrt[4]{3}.
\end{cases}
\]
Irtoamiskohdassa on maan vetovoima ainoa kappaleeseen vaikuttava ulkoinen voima. Koska tämä 
voima on jatkuva, ja myös kappaleeseen vaikuttavan tukivoiman voi olettaa olevan jatkuva 
irtoamiskohdassa (irtoamisesta eteenpäin tukivoima $=0$), niin päätellään, että 
kiihtyvyysvektori on jatkuva. Tämä merkitsee, että funktio $f$ on kahdesti jatkuvasti 
derivoituva myös irtoamiskohdassa, eli on oltava
\[
f^{(k)}(x_0^+)=f^{(k)}(x_0^-),\quad k=0,1,2 \ , \ x_0=\sqrt[4]{3}.
\]
Näistä ehdoista voidaan ratkaista vakiot $A,B,C$. Tulos (liikerata) on
\[
y=f(x)=\begin{cases}
-x^3/3,                                         &x\in [0,\sqrt[4]{3}], \\
-\sqrt[4]{3}\,x^2+\sqrt{3}\,x-1/\sqrt[4]{3},\ \ &x\in (\sqrt[4]{3},\infty).
\end{cases}
\]
Kuvassa liuku/lentorata on piirretty yhtenäisellä viivalla. Toisen asteen käyrä ennen
irtoamiskohtaa ja käyrä $y=-x^3/3$ irtoamiskohdan jälkeen on merkitty katkoviivalla. \loppu
\begin{figure}[H]
\setlength{\unitlength}{1cm}
\begin{center}
\begin{picture}(8,7)(-0.5,-6)
\put(-0.5,0){\vector(1,0){8}} \put(7.3,-0.5){$x$}
\put(0,-6){\vector(0,1){7}} \put(0.2,0.8){$y$}
\curve(
  0,        -0,
0.4,-0.0026667,
0.8, -0.021333,
1.2,    -0.072,
1.6,  -0.17067,
  2,  -0.33333,
2.4,    -0.576,
2.8,  -0.91467)
\curvedashes[0.1cm]{0,1,2}
\curve(
2.8,  -0.91467,
3.2,   -1.3653,
3.6,    -1.944,
  4,   -2.6667,
4.4,   -3.5493,
4.8,    -4.608,
5.2,   -5.8587)
%5.6,   -7.3173,
%  6,        -9)
\curvedashes[0.1cm]{0,1,2}
\curve(
  0,-0.75984,
0.4,-0.46607,
0.8,-0.27759,
1.2,-0.19439,
1.6,-0.21648,
  2,-0.34386,
2.4,-0.57652,
2.8,-0.91447)
\curvedashes{}
\curve(
2.8,-0.91447,
3.2, -1.3577,
3.6, -1.9062,
  4,   -2.56,
4.4, -3.3191,
4.8, -4.1835,
5.2, -5.1532)
%5.6, -6.2281,
%  6, -7.4083)
\put(2,0){\line(0,1){0.1}}
\put(0,-1){\line(-1,0){0.1}}
\put(1.9,0.2){$1$}
\put(-0.7,-1.1){$-1$}
\put(2.62,-0.885){$\scriptstyle{\bullet}\,\ \text{Irtoamiskohta}$}
\end{picture}
\end{center}
\end{figure}

\Harj
\begin{enumerate}

\item
Olkoon $\vec u(t)$ ja $\vec v(t)$ derivoituvia vektoriarvoisia funktioita. Todista: \newline
a) \ $\vec u(t)\ ||\ \vec v(t)\ \forall t \,\qimpl \vec u\,'\times\vec v=\vec v\,'\times\vec u$ 
\newline
b) \ $\vec u(t) \perp \vec v(t)\ \forall t  \qimpl \vec u\,'\cdot\vec v=-\vec v\,'\cdot\vec u$
 
\item
Määritä seuraavien tasokäyrien karevuusympyrä (säde $R$ sekä kaarevuuskeskiö) annetussa
käyrän pisteessä $P$: \newline
a) \ $y=x^2,\quad P=(1,1)$ \newline
b) \ $y=x^3-2x^2,\quad P=(2,0)$ \newline
c) \ $y=e^x,\quad P=(0,1)$ \newline
d) \ $\vec r=t\cos t\,\vec i+t\sin t\,\vec j,\quad P=(-\pi,0)$ \newline
e) \ $x=t-\sin t,\ y=1-\cos t,\quad P=(\pi,2)$ \newline
f) \ $2x^2+3y^2=5,\quad P=(1,-1)$ \newline
g) \ $x^3-y^3+y^2-3x+2=0,\quad P=(2,2)$ \newline
h) \ $r=e^\varphi,\quad P=(r,\varphi)=(1,0)$

\item
Määritä seuraavien avaruuskäyrien kaarevuusssäde käyrän pisteessä, joka vastaa annettua
parametrin arvoa. Määritä myös sen avaruustason yhtälö, jossa kaarevuusympyrä sijaitsee, sekä
kaarevuuskeskiö. \newline
a) \ $\vec r=t\,\vec i+t^2\,\vec j+t^3\,\vec k,\quad t=1$ \newline
b) \ $x=\cos t,\ y=\sin t,\ z=(4/\pi)\,t,\quad t=\pi/4$ \newline
c) \ $x=e^t\cos t,\ y=e^t\sin t,\ z=e^t,\quad t=0$

\item \index{oskuloivat käyrät}
Sanotaan, että tasokäyrät $S_1$ ja $S_2$ \kor{oskuloivat} (suom.\ suutelevat) pisteessä 
$(x_0,y_0)$, jos käyrillä on yhteinen kaarevuusympyrä ko.\ pisteessä. \ a) Päättele, että jos 
käyrien yhtälöt ovat $S_1: y=f(x)$ ja $S_2: y=g(x)$, niin oskulointiehdot ovat
\[
f(x_0)=g(x_0)=y_0\,, \quad f^{(k)}(x_0)=g^{(k)}(x_0),\,\ k=1,2.
\]
b) Etsi sellainen toisen asteen polynomikäyrä $S_1:\ y=ax^2+bx+c$, joka oskuloi käyrää
$S_2:\ 2x^3+6y^3+xy=0$ pisteessä $(3,-2)$.

\item
Määritä käyrän $y=x^2$ evoluutta. Piirrä kuva!

\item
Pistemäinen kappale liikkuu pitkin ruuviviivaa
\[
x=\cos \varphi,\ \ y=\sin \varphi,\ \ z=\varphi, \quad\varphi \in [0,\infty),
\]
siten, että sen ratanopeus (vauhti) on vakio $v_0$. Määritä kappaleen nopeus
$\vec{v}$ ja kiihtyvyys $\vec{a}$ ajan $t$ funktiona ($t\geq 0$), kun kappale on pisteessä 
$(1,0,0)$ hetkellä $t=0$. Määritä myös radan kaarevuussäde $R$ ja tarkista, että pätee:
$|\vec{a}|=|\vec{v}|^2/R$.

\item
Maaston korkeusprofiili tunturimaastossa on
\[
h(x,y)=40-0.005xy,
\]
missä pituusyksikkö = m. Retkeilijä heittää pisteessä $(x,y)=(0,0)$ olevasta leiripaikastaan
pilaantuneen tomaatin siten, että tomaatin lähtönopeus on 
\[
\vec v_0=(10\,\text{m}/\text{s})(\vec i-2\vec j+2\vec k).
\]
Missä pisteessä tomaatti törmää maahan ja mikä on tällöin sen vauhti? 
(Oletetaan $\,g=10\text{m}/\text{s}^2$, ei ilmanvastusta eikä tuulikorjausta.)

\item (*)
Näytä, että sykloidin $S: x=R(t-\sin t),\ y=R(1-\cos t),\ t\in\R$ evoluutta on toinen, $S$:n
kanssa yhtenevä sykloidi, joka saadaan siirtämällä $S\,$ vektorin
$-\pi R\vec i-2R\vec j$ verran. Kuva!

\item (*)
Origossa oleva kappale lähtee levosta liukumaan kitkattomasti pitkin käyrää 
$y=1-\cosh x,\ x \ge 0$, painovoiman vaikuttaessa suunnassa $-\vec j$. Määritä kappaleen 
liikerata.

\item (*) \label{H-dif-3: kierevyys}
\index{Frenet'n kanta} \index{kierevyys (käyrän)} \index{sivunormaalivektori}
Avaruuskäyrän pisteeseen $P(t)\vastaa\vec r\,(t)$ liittyvä \kor{Frenet'n kanta} on
vektorisysteemi $\{\tv,\vec n,\vec\nu\}$, missä $\tv$ ja $\vec n$ ovat yksikkö\-tangentti- ja 
päänormaalivektorit ko.\ pisteessä ja $\vec\nu=\tv\times\vec n$ on \kor{sivunormaalivektori}. 
Avaruuskäyrän \kor{kierevyys} $\omega$ (engl.\ torsion) määritellään tällöin kaavalla
\[
\frac{1}{v(t)}\frac{d\vec\nu}{dt}=-\omega(t)\vec n, \quad v(t)=\abs{\dvr(t)}.
\]
a) Näytä, että $d\vec\nu/dt$ todella on päänormaalivektorin suuntainen. \newline
b) Päättele, että jos avaruuskäyrä on tasokäyrä jollakin avaruustasolla, niin sen kierevyys 
$=0$. \newline
c) Laske ruuviviivan $S:\ x=a\cos t,\ y=a\sin t,\ z=bt$ kierevyys käyrän pisteessä $P(t)$.

\item (*) \index{zzb@\nim!Sotaharjoitus 2}
(Sotaharjoitus 2) Tykinammus laukaistaan origosta lähtönopeudella 
$\vec v_0=(150\,\text{m}/\text{s})(\vec i+2\vec j+\vec k)$. Lentoradalla ammukseen vaikuttaa 
painovoiman lisäksi tuuli ja nopeuteen verrannollinen vastusvoima siten, että liikeyhtälöt
ovat
\[
\vec v\,'=c\vec i-g\vec k-k\vec v, \quad \dvr=\vec v,
\]
missä $\vec r\,(t)$ ja $\vec v(t)$ ovat ammuksen paikka- ja nopeusvektorit hetkellä $t$,
$\,g=10$ m/s$^2$, $k=0.01$ s$^{-1}$ ja $c=0.10$ m/s$^2$. Laske, mihin $xy$-tason pisteeseen
($10$ metrin tarkkuus!) ammus putoaa. \kor{Vihje}: Ratkaise liikeyhtälöt erikseen suunnissa
$\vec i,\,\vec j,\,\vec k$ (ensin $\vec v$, sitten $\vec r\,$). Aloita pystysuunnasta lentoajan
selville saamiseksi! 

\end{enumerate} %Käyrän kaarevuus
\section{Taylorin polynomit ja Taylorin lause} \label{taylorin lause}
\sectionmark{Taylorin lause}
\alku

Tässä luvussa tarkastellaan funktioita, jotka ovat annetun pisteen $x_0$ ympäristössä
riittävän sileitä, eli riittävän monta kertaa (jatkuvasti) derivoituvia.
\begin{Def} \index{Taylorin polynomi|emph}
Funktion $f:\DF_f\kohti\R$, $\DF_f\subset\R$, joka on $n$ kertaa derivoituva pisteessä
$x_0\in\DF_f$, \kor{Taylorin polynomi astetta $n$ pisteessä $x_0$} on
\[
\boxed{\kehys\quad T_n(x,x_0)=\sum_{k=0}^n\frac{f^{(k)}(x_0)}{k!}(x-x_0)^k. \quad}
\]
\end{Def}
Määritelmän mukaiset kolme ensimmäistä Taylorin polynomia ovat
\begin{align*}
T_0(x,x_0)\ &=\ f(x_0), \\[2mm]
T_1(x,x_0)\ &=\ f(x_0) + f'(x_0)(x-x_0), \\
T_2(x,x_0)\ &=\ f(x_0) + f'(x_0)(x-x_0) + \frac{1}{2}f''(x_0)(x-x_0)^2.
\end{align*}
Erityisesti siis $\,T_1(x,x_0) = f$:n linearisoiva approksimaatio pisteessä $x_0$.
\begin{Prop}
Funktion $f$ Taylorin polynomi $T_n(x,x_0)$ määräytyy yksikäsitteisesti ehdoista
\[
\frac{d^k}{dx^k}T_n(x,x_0)_{|\,x=x_0}\ =\ f^{(k)}(x_0),\quad k=0\ldots n.
\]
\end{Prop}
\tod Helposti nähdään, että $T_n(x,x_0)$ toteuttaa mainitut ehdot. Jos jokin toinen polynomi 
$p(x)$ toteuttaa samat ehdot, eli
\[
p^{(k)}(x_0)=f^{(k)}(x_0),\quad k=0\ldots n,
\]
niin silloin polynomi
\[
q(x)=T_n(x,x_0)-p(x)=\sum_{k=0}^n a_kx^k
\]
toteuttaa
\[
q^{(k)}(x_0)=0,\quad k=0\ldots n.
\]
Tällöin koska $q^{(n)}(x_0)=n!\,a_n$, seuraa $a_n=0$, jolloin 
$q^{(n-1)}(x_0)=(n-1)!\,a_{n-1} \ \impl \ a_{n-1}=0$, jne. Siis $q=0$, ja näin ollen em.\ 
ehdoista määräytyvä polynomi on yksikäsitteinen. \loppu
\begin{Exa} Määritä seuraavat Taylorin polynomit pisteessä $x_0=0\,$:
\begin{align*} 
&\text{a)}\ \ f(x)=\sqrt[3]{1+x},\ \ n=2 \qquad \text{b)}\ \ f(x)=\tan x,\ \ n=5 \\ 
&\text{c)}\ \ f(x)=\ln (1+x),\ \ n \in \N 
\end{align*}
\begin{align*} 
\text{\ratk} \text{a)} 
&\quad f'(x)=\frac{1}{3}(1+x)^{-2/3}, \quad f''(x)=-\frac{2}{9}(1+x)^{-5/3} \\
&\quad\quad\impl\ f(0)=1,\quad f'(0)=\frac{1}{3},\quad f''(0)=-\frac{2}{9} \\
&\quad\quad\impl\ T_2(x,0)=\underline{\underline{1+\frac{x}{3}-\frac{1}{9}x^2.}}\\ \\
&\text{b)} \quad f'(x)=1/\cos^2 x, \quad f''(x)=2\sin x/\cos^3 x, \\
&\quad\quad f'''(x)= 2/\cos^2 x+6\sin^2 x/\cos^4 x=-4/\cos^2 x + 6/\cos^4 x \\
&\quad\quad f^{(4)}(x)=-8\sin x/\cos^3 x+24\sin x/\cos^5 x \\
&\quad\quad f^{(5)}(x)=-16/\cos^2 x+120\sin^2 x/\cos^6 x \\ 
&\quad\quad\quad\impl \begin{cases} 
                      \,f(0)=f''(0)=f^{(4)}=0, \\ 
                      \,f'(0)=1,\quad f'''(0)=2,\quad f^{(5)}(0)=16 
                      \end{cases} \\
&\quad\quad\quad\impl\ T_5(x,0)=\underline{\underline{x+\frac{1}{3}x^3+\frac{2}{15}x^5.}} \\ \\
&\text{c)} \quad f'(x)=(1+x)^{-1}, \quad f''(x)=-(1+x)^{-2},\quad f'''(x)=2(1+x)^{-3},\\
&\quad\quad\ldots,\quad f^{(k)}=(-1)^{k-1}(k-1)!(1+x)^{-k} \\
&\quad\quad\impl \ f(0)=0,\quad f^{(k)}(0)= (-1)^{k-1}(k-1)!,\quad k=1,2,\ldots, \\
&\quad\quad\impl \ T_n(x,0)
  =\underline{\underline{x-\frac{1}{2}x^2+\frac{1}{3}x^3+\cdots +(-1)^{n-1}\frac{x^n}{n}\,.}} 
                                                                                       \loppu
\end{align*}
\end{Exa}
Taylorin polynomin derivaatta on
\[
\frac{d}{dx} T_n(x,x_0) = f'(x_0) + f''(x_0)(x-x_0) + \ldots 
                                  + \frac{f^{(n-1)}(x_0)}{(n-1)!}(x-x_0)^{n-1}.
\]
Derivoinnin tulos = $f'$:n Taylorin polynomi astetta $n-1$ pisteessä $x_0$, eli lyhyesti:
Taylorin polynomin derivaatta = derivaatan Taylorin polynomi (astetta alempi).

\pagebreak

\jatko\begin{Exa}
(jatko). Esimerkin tuloksista saadaan derivoimalla seuraavat Taylorin polynomit: \vspace{0.5cm}
\newline
\begin{tabular}{rlcl}
a) & $f(x)=(1+x)^{-2/3}$ & : & $T_1(x,0)=1-\frac{2}{3}x$. \\
b) & $f(x)=1/\cos^2 x$ & : & $T_4(x,0)=1+x^2+\frac{2}{3}x^4$. \\
c) & $f(x)=1/(1+x)$ & : & $T_{n-1}(x,0)=1-x+\cdots +(-1)^{n-1}x^{n-1}$. \loppu
\end{tabular}
\end{Exa}

\subsection{Taylorin lause}
\index{Taylorin lause|vahv}

Taylorin polynomeihin perustuu seuraava huomattava approksimaatiolause. Todistus esitetään
luvun lopussa.
\begin{Lause} (\vahv{Taylorin lause}) \label{Taylor}
Jos $f$ on jatkuva välillä $[a,b]$ ja $n+1$ kertaa derivoituva välillä $(a,b)$, niin jokaisella
$x_0\in(a,b)$ ja $x\in[a,b],\ x \neq x_0$ pätee
\[
f(x)=T_n(x,x_0)+R_n(x),
\]
missä
\[
R_n(x)=\frac{f^{(n+1)}(\xi)}{(n+1)!}\,(x-x_0)^{n+1} \quad 
                \text{jollakin}\,\ \xi\in (x_0,x)\ \text{ tai }\ \xi\in (x,x_0).
\]
\end{Lause}
Taylorin lauseen mukaan funktiota, joka on tietyn pisteen $x_0$ ympäristössä
säännöllinen, voi tässä ympäristössä approksimoida polynomilla --- nimittäin Taylorin
polynomilla --- ja approksimaatio on yleisesti ottaen sitä tarkempi, mitä korkeampi on
polynomin asteluku, ja mitä lähempänä ollaan pistettä $x_0$. Tuloksen voi esittää
kvalitatiivisesti muodossa:
\[
\boxed{\kehys\quad \text{\pain{Sileä} funktio}\ 
               \approx\ \text{polynomi \pain{l}y\pain{h}y\pain{ellä} välillä}. \quad}
\]

Taylorin lauseen tuloksella on perustavaa laatua oleva merkitys lähes kaikessa numeerisessa 
laskennassa, johon sisältyy funktioiden approksimointia. Virhetermille $R_n(x,x_0)$, eli 
\index{jzyzy@jäännöstermi (Lagrangen)}%
Taylorin polynomiapproksimaation nk.\ \kor{jäännöstermille} (engl.\ remainder), tunnetaan
monia muotoja. Lauseen \ref{Taylor} esittämää sanotaan jäännöstermin
\index{Lagrangen!a@jäännöstermi}% 
\kor{Lagrangen}\footnote[2]{Italialais-ranskalainen \hist{Joseph Louis}
(synt.\ Giovanni Luigi) \hist{Lagrange} (1736-1813) oli aikansa huomattavimpia
matemaatikkoja. Erityisesti differentiaalilaskennan (myös integraalilaskennan) kehittämisessä
Lagrangen panos oli merkittävä. Matematiikan ohella Lagrange tutki mekaniikkaa ja saavutti
silläkin alalla pysyvän nimen. \index{Lagrange, J. L.|av} \index{Taylor, B.|av} 

Taylorin polynomit, Taylorin lause, ja erityisesti jäljempämä esitettävät
\kor{Taylorin sarjat} viittaavat englantilaiseen matemaatikkoon \hist{Brook Taylor}iin 
(1685-1731). Nimeään kantavaa lausetta ei Taylor todellisuudessa tuntenut.} muodoksi.

\begin{Exa} \label{Taylor ja exp,cos,sin} Soveltamalla derivointisääntöjä
\[ \begin{cases}
\,D^ke^x=e^x,\quad k=0,1,2,\ldots \\
\,D^{2k}\cos x=(-1)^k\cos x,\quad D^{2k+1}\cos x=(-1)^{k+1}\sin x, \quad k=0,1,2,\ldots \\
\,D^{2k}\sin x=(-1)^k\sin x,\quad D^{2k+1}\sin x=(-1)^k\cos x, \quad k=0,1,2,\ldots
\end{cases} \]
ja Lausetta \ref{Taylor} nähdään, että jos $x\in\R,\ x \neq 0$ ja $n\in\N$, niin funktioille 
$e^x,\ \cos x,\ \sin x$ pätee jollakin $\xi\in(0,x)\ (x>0)$ tai $\xi\in(x,0)\ (x<0)$ 
\newline
\[ \boxed{ \begin{aligned}
         e^x\ &=\ \Bigl(\,1+x+\frac{x^2}{2!}+\cdots +\frac{\ykehys x^n}{n!}\,\Bigr)
                             \,+\,\frac{e^\xi}{(n+1)!}\,x^{n+1}, \\
      \cos x\ &=\ \left(\,1-\frac{x^2}{2!}+\cdots +(-1)^n\frac{x^{2n}}{(2n)!}\,\right) 
                             \,+\,(-1)^{n+1}\frac{\cos\xi}{\akehys (2n+2)!}\,x^{2n+2}, \quad \\ 
\quad \sin x\ &=\ \left(\,x-\frac{x^3}{3!}+\cdots +(-1)^n\frac{x^{2n+1}}{(2n+1)!}\,\right) 
                             \,+\,(-1)^{n+1}\frac{\cos\xi}{(2n+3)!}\,x^{2n+3}. \quad
           \end{aligned} } \]
\newline
Tässä on sulkeilla ympäröity Taylorin polynomit 
\begin{align*}
e^x\,:    \quad &T_n(x,0), \\
\cos x\,: \quad &T_{2n}(x,0)=T_{2n+1}(x,0), \\
\sin x\,: \quad &T_{2n+1}(x,0)=T_{2n+2}(x,0). \loppu
\end{align*}
\end{Exa}

Taylorin lause on muotoiltavissa myös niin, että funktion $(n+1)$-kertaisen derivoituvuuden
sijasta oletetaan ainoastaan $n$-kertainen derivoituvuus ja derivaatan $f^{(n)}$ jatkuvuus.
\begin{Lause} \label{Taylorin approksimaatiolause} Jos $f$ on $n$ kertaa derivoituva välillä
$(a,b)$ ja $f^{(n)}$ on jatkuva ko.\ välillä, niin jäännöstermille $R_n(x)=f(x)-T_n(x,x_0)$
pätee jokaisella $x_0\in(a,b)$
\[
\lim_{x \kohti x_0} \frac{R_n^{(k)}(x)}{(x-x_0)^{n-k}} = 0, \quad k = 0 \ldots n.
\]
\end{Lause}
\tod Tapauksessa $n=0$ väittämä on tosi jatkuvuuden määritelmän nojalla. Jos $n \ge 1$, niin 
oletusten ja Taylorin lauseen perusteella on
\begin{align*}
f(x)\ &=\ T_{n-1}(x,x_0) + \frac{f^{(n)}(\xi)}{n!}(x-x_0)^n \\
      &=\ \left[T_{n-1}(x,x_0) + \frac{f^{(n)}(x_0)}{n!}(x-x_0)^n\right] 
                               + \frac{1}{n!}\,[f^{(n)}(\xi)-f^{(n)}(x_0)]\,(x-x_0)^n \\
      &=\ T_n(x,x_0) + \frac{1}{n!}\,[f^{(n)}(\xi)-f^{(n)}(x_0)]\,(x-x_0)^n, \quad x\in(a,b),
\end{align*} 
missä $\xi=\xi(x)=x_0$, jos $x=x_0$, muulloin $\xi(x)\in(x_0,x)$ tai $\xi\in(x,x_0)$. Siis
\[
R_n(x)\ =\  \frac{1}{n!}\,[f^{(n)}(\xi(x))-f^{(n)}(x_0)]\,(x-x_0)^n,
\]
missä $\xi(x) \kohti x_0$ kun $x \kohti x_0$. Koska $f^{(n)}$ on jatkuva $x_0$:ssa, niin
seuraa
 \[
\lim_{x \kohti x_0} \frac{R_n(x)}{(x-x_0)^n} 
              = \lim_{x \kohti x_0} \frac{1}{n!}\,[f^{(n)}(\xi(x))-f^{(n)}(x_0)] = 0.
\]
Muut väitetyt raja-arvotulokset seuraavat tästä derivoimalla: Koska
\[
f^{(k)}(x) = \left(\frac{d}{dx}\right)^k T_n(x,x_0) + R_n^{(k)}(x), \quad 
                                  x \in (x_0,x_0+a),\ \ k=1 \ldots n,
\]
ja koska tässä $(d/dx)^k\,T_n(x,x_0) = f^{(k)}$:n Taylorin polynomi astetta $n-k$, niin jo 
todistetun perusteella
\[
\lim_{x \kohti x_0} \frac{R_n^{(k)}(x)}{(x-x_0)^{n-k}} = 0, \quad k=1 \ldots n. \loppu
\]

\subsection{Taylorin polynomien nopea laskeminen}

Joskus $f$:n derivaatat ovat niin hankalia laskea, että Taylorin polynomin saa määrätyksi 
suoremmin muilla menetelmillä, jolloin polynomin avulla voi päinvastoin määrittää derivaatat 
$f^{(k)}(x_0)$, $k=0\ldots n$ (!). Polynomia muilla keinoin määrättäessä riittää, että 
jäännöstermi saadaan riittävän pieneksi, sillä tälläkin kriteerillä polynomi on
yksikäsitteinen:
\begin{Prop} \label{Taylor-prop}
Olkoon funktio $f$ $\,n$ kertaa derivoituva välillä $(a,b)$ ja olkoon $f^{(n)}$ jatkuva
välillä $(a,b)$. Tällöin jos $p$ on polynomi astetta $\le n$ siten, että jollakin
$x_0\in(a,b)$ pätee
\[
\lim_{x \kohti x_0} \frac{f(x)-p(x)}{(x-x_0)^n}=0,
\]
niin $p(x) = f$:n Taylorin polynomi $T_n(x,x_0)$.
\end{Prop}
\tod Kun merkitään $q(x)=p(x)-T_n(x,x_0)$, niin raja-arvojen yhdistelysääntöjen
(Lause \ref{funktion raja-arvojen yhdistelysäännöt}), Lauseen
\ref{Taylorin approksimaatiolause} ja oletuksen mukaan
\[
\lim_{x \kohti x_0} \frac{q(x)}{(x-x_0)^n} 
     \,=\, \lim_{x \kohti x_0} \left[\frac{f(x)-T_n(x,x_0)}{(x-x_0)^n}
                                 - \frac{f(x)-p(x)}{(x-x_0)^n}\right] = 0-0 = 0.
\]
Tämä on mahdollista vain kun $q(x)=0$, koska $q$ on polynomi astetta $\le n$. \loppu

Seuraavissa esimerkeissä käytetään lyhennysmerkintää $[x^m]$ funktiosta muotoa $x^m g(x)$, 
missä $g$ on rajoitettu pisteen $x=0$ ympäristössä.
\begin{Exa} \label{nopea Taylor 1}
$f(x)=(x+1)/\cos x,\quad$ $T_5(x,0)=\ ?$
\end{Exa}
\ratk Koska $\,\cos x=1-\frac{x^2}{2}+\frac{x^4}{24}+[x^6]\,$ ja
$\,1/(1-t)=1+t+t^2+[t^3]$, niin
\begin{align*}
f(x) &= (x+1)\left(1-\frac{x^2}{2}+\frac{x^4}{24}\right)^{-1}(1+[x^6])^{-1} \\
     &= (x+1)\left[1-\left(\frac{x^2}{2}-\frac{x^4}{24}\right)\right]^{-1}+[x^6] \\
     &= (x+1)\left[1+\left(\frac{x^2}{2}-\frac{x^4}{24}\right)
                    +\left(\frac{x^2}{2}-\frac{x^4}{24}\right)^2\right]+[x^6] \\ 
     &= (x+1)\left(1+\frac{1}{2}x^2+\frac{5}{24}x^4\right)+[x^6] \\
     &= \Bigl(1+x+\frac{1}{2}x^2+\frac{1}{2}x^3+\frac{5}{24}x^4+\frac{5}{24}x^5\Bigr)+[x^6].
\end{align*}
Proposition \ref{Taylor-prop} mukaan sulkeissa oleva polynomi $=f$:n Taylorin polynomi 
$T_5(x,0)$. \loppu
\begin{Exa} \label{nopea Taylor 2}
$f(x)=1/(1+x^4e^{x^2}),\quad$ $f^{(8)}(0)=\ ?$
\begin{align*}
\text{\ratk} \quad f(x) &= 1-(x^4e^{x^2})+(x^4e^{x^2})^2+[x^{12}] \hspace{6cm} \\
                        &= 1-x^4\Bigl(1+x^2+\frac{1}{2}x^4+[x^6]\Bigr)
                                           +x^8\left(1+[x^2]\right)^2+[x^{12}] \\
                        &= 1-x^4-x^6+\frac{1}{2}x^8+[x^{10}] = T_8(x,0)+[x^{10}] \\
                        &\impl \ f^{(8)}(0) = \frac{1}{2}\cdot8!=\underline{\underline{20160}}.
                                                                                       \loppu
\end{align*}
\end{Exa}

\subsection{Taylorin sarjat}
\index{Taylorin sarja|vahv}

Kun jäännöstermi Taylorin lauseessa \ref{Taylor} arvioidaan funktioille $\cos x$ ja $\sin x$, 
niin nähdään, että (ks.\ Esimerkki \ref{Taylor ja exp,cos,sin} edellä)
\begin{alignat*}{2}
\abs{\cos x -T_{2n}(x,0)}   &\leq \frac{1}{(2n+2)!}\abs{x}^{2n+2},\quad &x\in\R, \\
\abs{\sin x -T_{2n+1}(x,0)} &\leq \frac{1}{(2n+3)!}\abs{x}^{2n+3},\quad &x\in\R.
\end{alignat*}
Koska $\abs{x}^n/n!\kohti 0 \ \forall x\in\R$, kun $n\kohti\infty$, niin jokaisella $x\in\R$
pätee
\[ 
\cos x = \lim_{n\kohti\infty}T_{2n}(x,0), \qquad \sin x = \lim_{n\kohti\infty} T_{2n+1}(x,0),
\] 
eli (vrt.\ Harj.teht.\,\ref{exp(x) ja ln(x)}:\ref{H-exp-2: kosini ja sini})
\[
\boxed{\begin{aligned}
\quad\cos x\ &=\ \sum_{k=0}^\infty (-1)^k\frac{x^{2k}}{(2k)!}, \quad x\in\R, \\
     \sin x\ &=\ \sum_{k=0}^\infty (-1)^k\frac{x^{2k+1}}{(2k+1)!}, \quad x\in\R. \quad
\end{aligned}}
\]
\begin{Def}
Jos $f$ on mielivaltaisen monta kertaa derivoituva $x_0$:ssa, niin sarja
\[
\sum_{k=0}^\infty \frac{f^{(k)}(x_0)}{k!}(x-x_0)^k 
\]
on $f$:n \kor{Taylorin sarja} $x_0$:ssa.\footnote[2]{Tapauksessa $x_0=0$ käytetään Taylorin
sarjasta myös nimitystä \kor{Maclaurinin sarja}. \index{Maclaurinin sarja|av}}
\end{Def}
Taylorin sarjojen teoriassa aivan ilmeisesti keskeisin kysymys on: Milloin sarja suppenee
kohti $f(x)$:ää, ts. milloin pätee
\[
f(x)=\lim_{n\kohti\infty} T_n(x,x_0)=\sum_{k=0}^\infty \frac{f^{(k)}(x_0)}{k!}(x-x_0)^k \ ?
\]
Funktioiden $\sin x$ ja $\cos x$ kohdalla vastaus on: Aina, eli jokaisella $x\in\R$ 
(myös jokaisella $x_0 \in \R$). Myös eksponenttifunktion $e^x$ kohdalla vastaus on sama; tälle
Taylorin lause vahvistaa ennestään tunnetun tuloksen (vrt. Luku \ref{exp(x) ja ln(x)})
\[
\boxed{\quad e^x=\sum_{k=0}^\infty \frac{x^k}{k!}, \quad x\in\R. \quad}
\]
Tarkasti ottaen ym. kysymys Taylorin sarjan suppnemisesta sisältää kaksi erillistä
kysymystä, kuten nähdään seuraavasta esimerkistä.
\begin{Exa} \label{outo Taylorin sarja} Funktio
\[
f(x)=\begin{cases}
e^{-1/x^2}, \quad   &\text{kun}\ x \neq 0, \\
0\ ,                &\text{kun}\ x = 0
\end{cases}
\]
on mielivaltaisen monta kertaa derivoituva pisteessä $x=0$ (myös muualla) ja
\[
f^{(k)}(0)=0,\quad k=0,1,2,\ldots\,,
\]
joten $T_n(x,0)=0 \ \forall n$. Tässä tapauksessa siis Taylorin sarja suppenee $\forall x\in\R$,
mutta $\lim_n T_n(x,0) = 0 \neq f(x)$, kun $x \neq 0 $. \loppu
\end{Exa}
Esimerkin mukaan kahdella eri funktiolla voi olla sama Taylorin sarja (esimerkissä funktioilla 
$f$ ja $g=0$), joten Taylorin sarjan kertoimista (tai sarjan summasta) ei voi päätellä 
funktiota, josta sarja on johdettu. Useille 'normaaleille' funktioille $f$ kuitenkin pätee,
että $f$:n Taylorin sarjan summa $=f(x)$ aina kun sarja suppenee. Tällaisia normaalitapauksia
ovat esim.\ rationaalifunktiot.
\begin{Exa}
Funktion $f(x)=1/(1+4x^2)$ Taylorin polynomit origossa ovat
\[
T_{2n}(x,0)=T_{2n+1}(x,0)=\sum_{k=0}^n (-4)^kx^{2k},\quad n=0,1,\ldots
\]
Taylorin sarja eli potenssisarja $\{T_n(x,0), \ n=0,1,2,\ldots\}$ suppenee tässä tapauksessa
täsmälleen kun $\abs{x}<1/2$ (vrt.\ Luku \ref{potenssisarja}), ja tällöin summa $=f(x)$\,:
\[
\sum_{k=0}^\infty (-4)^kx^{2k} = \frac{1}{1+4x^2} 
                               = f(x), \quad x \in (-\tfrac{1}{2},\,\tfrac{1}{2}\,). \loppu
\]
\end{Exa}

Taylorin sarjojen suppenemista tutkittaessa voidaan aina tehdä muuttujan vaihdos 
$x-x_0\hookrightarrow x$, jolloin riittää tarkastella yleistä potenssisarjaa muotoa
\[ 
f(x) = \sum_{k=0}^n a_k x^k. 
\]
Tällaisen sarjan suppenemiskysymys on ratkaistu Luvussa \ref{potenssisarja}: Lauseen 
\ref{suppenemissäde} mukaan sarja suppenee joko (a) vain kun $x=0$ tai (b) välillä 
$(-\rho,\rho)$ (mahdollisesti myös kun $x = \pm \rho$), missä $\rho$ on sarjan suppenemissäde
($\rho \in \R_+$ tai $\rho = \infty$). Luvussa \ref{derivaatta} osoitettiin, että 
potenssisarjan summana määritelty funktio on mielivaltaisen monta kertaa derivoituva välillä 
$(-\rho,\rho)$ ja että derivaatat voidaan laskea derivoimalla sarja termeittäin
(Lause \ref{potenssisarja on derivoituva}). Näin ollen jos potenssisarjan
$\sum_{k=0}^\infty a_k x^k$ suppenemissäde on $\rho>0$ ja $x_0 \in \R$, niin funktio
\[ 
f(x) = \sum_{k=0}^\infty a_k\,(x-x_0)^k 
\]
on määritelty ja mielivaltaisen monta kertaa derivoituva välillä $(x_0-\rho,x_0+\rho)$ 
(koko $\R$:ssä, jos $\rho=\infty$) ja $f$:n derivaatat voidaan laskea derivoimalla sarja 
termeittäin. Kun derivoimispisteeksi valitaan erityisesti $x_0$, saadaan tulos
\[
f^{(k)}(x_0) = k!\,a_k \qekv a_k = \frac{f^{(k)}(x_0)}{k!}, \quad k = 0,1,2, \ldots 
\]
Siis $f(x)$ on esitettävissä muodossa
\[
f(x) = \sum_{k=0}^\infty \frac{f^{(k)}(x_0)}{k!}\,(x-x_0)^k.
\]
On tultu seuraavaan huomionarvoiseen tulokseen:
\begin{Lause} Jos sarja $\,\sum_{k=0}^\infty a_k\,(x-x_0)^k\,$ suppenee välillä 
$(x_0-\rho,x_0+\rho)$, $\rho>0$, niin ko.\ sarja = sarjan summana määritellyn funktion
Taylorin sarja pisteessä $x_0$.
\end{Lause}
\begin{Exa} \label{sinx/x}
Funktion $\,\sin x\,$ Taylorin sarjasta nähdään, että
\[
\sum_{k=0}^\infty (-1)^k\frac{x^{2k}}{(2k+1)!} = f(x) = \begin{cases}
                                                        \ \sin x/x\ , \ \ &x\neq 0, \\
                                                        \ 1\ ,            &x=0.
                                                        \end{cases}
\]
Koska sarja suppenee $\forall x\in\R$, niin kyseessä on sarjan summana määritellyn funktion 
Taylorin sarja origossa. Funktio $f$ on siis mielivaltaisen monta kertaa derivoituva
jokaisessa pisteessä $x\in\R$, origo mukaan lukien (!). \loppu
\end{Exa}
\begin{Exa} Ratkaise Taylorin sarjoilla alkuarvotehtävä
\[ \begin{cases} 
    \,y' = e^{-x^2}, \quad x \in \R, \\
    \,y(0) = 0.
\end{cases} \]
\end{Exa}
\ratk Koska eksponenttifunktion $e^x$ Taylorin sarja suppenee kaikkialla, niin
\[ 
e^{-x^2}\ =\ \sum_{k=0}^\infty \frac{(-x^2)^k}{k!}\ 
          =\ \sum_{k=0}^\infty \frac{(-1)^k}{k!}\,x^{2k}\,, \quad x \in \R. 
\]
Kun valitaan
\[ 
y(x)\ =\ \sum_{k=0}^\infty \frac{(-1)^k}{(2k+1)\,k!}\,x^{2k+1}\ 
      =\ x - \frac{x^3}{18} + \frac{x^5}{600} - \ldots, 
\]
niin nähdään termeittäin derivoimalla, että $y'(x)=e^{-x^2},\ x\in\R$. Koska on myös $y(0)=0$,
niin ratkaisu on tässä. \loppu

\subsection{Taylorin lauseen todistus}
\index{Taylorin lause|vahv}

Taylorin lauseelle on monia erilaisia todistustapoja. Seuraavassa
'lyhyen kaavan' mukaisessa todistuksessa päättelyn kulmakivi on Rollen lause
(Lause \ref{Rollen lause}).

Olkoon $x_0<x$ (tapaus $x_0>x$ käsitellään vastaavasti) ja tarkastellaan välillä $[x_0,x]$
funktiota $g(t)$, joka määritellään
\[
g(t) = f(t) - T_n(t,x_0) - H(t-x_0)^{n+1}, \quad H = \frac{f(x)-T_n(x,x_0)}{(x-x_0)^{n+1}}\,. 
\]
Tälle pätee $g^{(k)}(x_0)=0,\ k=0 \ldots n$ ja $g(x)=0$. Koska siis $g(x_0)=g(x)=0$, niin
Rollen lauseen mukaan on $g'(\xi_1)=0$ jollakin $\xi_1 \in (x_0,x)\subset(a,b)$. Jos $n \ge 1$,
on myös $g'(x_0)=0$, jolloin saman lauseen mukaan on $g''(\xi_2)=0$ jollakin
$\xi_2 \in (x_0,\xi_1)$. Jos edelleen $n \ge 2$, on myös $g''(x_0)=0$, joten saman lauseen
mukaan on $g'''(\xi_2)=0$ jollakin $\xi_2\in(x_0,\xi_1)$. Jatkamalla samalla tavoin päätellään,
että yleisesti on $g^{(n)}(x_0)=g^{(n)}(\xi_n)=0$ jollakin $\xi_n \in (x_0,\xi_{n-1})$, jolloin
Rollen lauseen mukaan on $g^{(n+1)}(\xi_{n+1})=0$ jollakin 
$\xi_{n+1} \in (x_0,\xi_n) \subset (x_0,x) \subset (a,b)$. Mutta 
$g^{(n+1)}(t) = f^{(n+1)}(t)-H(n+1)!$ --- Siis $f^{(n+1)}(\xi_{n+1})-H(n+1)!=0\ \impl$ väite.
\loppu

\Harj
\begin{enumerate}

\item
Laske funktion $\,f(x)=x^4+3x^3+x^2+2x+8,$ kaikki Taylorin polynomit $T_n(x,2),\ n=0,1,2,\ldots$
ja saata ne polynomin perusmuotoon ($x$:n potenssien mukaan). Piirrä ko.\ polynomien kuvaajat
ja vertaa funktioon $f$.

\item
Laske funktion $f(x)=(x-1)/(x-2)$ Taylorin polynomi $T_3(x,0)$ ja piirrä funktion, Taylorin
polynomin ja jäännöstermin kuvaajat.
 
\item
Määritä seuraaville funktioille Taylorin polynomi $T_n(x,0)$ annettua astetta $n$ ja arvioi 
jäännöstermin Lagrangen muodosta mahdollisimman tarkasti luku 
$\displaystyle{r_n=\max_{x\in[-1,1]}\abs{R_n(x)}}$\,: \newline
a) \ $f(x)=\cosh x,\,\ n=4\qquad\qquad\quad\,$ b) \ $f(x)=e^{-0.2x},\,\ n=3$ \newline
c) \ $f(x)=\sqrt{5+x},\,\ n=3\qquad\qquad\ \ $ d) \ $f(x)=\sqrt[5]{5-x},\,\ n=3$ \newline
e) \ $f(x)=2^x,\,\ n=5\qquad\qquad\qquad\quad$ f) \  $f(x)=\ln(e+x),\,\ n=6$ \newline
g) \ $f(x)=\cot(x+\pi/3),\,\ n=4\qquad$ \      h) \ $f(x)=\sin x-1/\cos x,\,\ n=3$

\item
a) Funktioiden $\cosh x$ ja $\sinh x$ Taylorin polynomit $T_n(x,0)$ voidaan laskea joko suoraan
määritelmästä tai funktioiden $e^x$ ja $e^{-x}$ Taylorin polynomien avulla. Varmista, että
kummallakin tavalla tulos on sama. \vspace{1mm}\newline
b) Näytä, että parillisen (vastaavasti parittoman) funktion Taylorin polynomissa $T_n(x,0)$ on
vain parillisia (parittomia) potensseja.

\item
Todista Taylorin lauseen avulla:
\[
1-\frac{1}{2}x^2\,\le\,\cos x\,\le\,1-\frac{1}{2}x^2+\frac{1}{24}x^4\,, \quad
                                 \text{kun}\ x\in\left[-\frac{\pi}{2}\,,\,\frac{\pi}{2}\right].
\]
Ovatko nämä epäyhtälöt tosia myös välin $[-\pi/2,\,\pi/2]$ ulkopuolella?

\item
Harjoitustehtävässä \ref{ääriarvot}:\ref{H-V-5: jatkaminen polynomeilla}b oletetaan, että
funktio $f$ on $m$ kertaa jatkuvasti derivoituva välillä $[a,b]$. Näytä, että tehtävän
ratkaisu on $p_1(x)=T_m^+(a,x)$, $p_2(x)=T_m^-(b,x)$, missä $T_m^+(a,x)$ ja
$T_m^-(b,x)$ ovat toispuolisten derivaattojen $\dif_+^kf(a)$ ja $\dif_-^kf(b)$ avulla
määritellyt $f$:n Taylorin polynomit.

\item
Laske seuraavien funktioiden Taylorin polynomi $T_n(x,0)$ annettua astetta käyttäen 
mahdollisimman nopeita oikoteitä:
\begin{align*}
&\text{a)}\ \ f(x)=2/(4+x^3), \quad n=12 \qquad
 \text{b)}\ \ f(x)=\cos x^4, \quad n=16 \\
&\text{c)}\ \ f(x)=\Arcsin x^3,\quad n=6 \qquad
 \text{d)}\ \ f(x)=(x^3-x^5)\Arctan x^2, \quad n=7
\end{align*}

\item
Laske seuraavat derivaatat käyttäen hyväksi Taylorin polynomeja:
\newline
a) \ $f(x)=\sin x^8,\,\ f^{(40)}(0)\qquad\qquad\quad$ 
b) \ $f(x)=e^{-x^4},\,\ f^{(20)}(0)$ \newline
c) \ $f(x)=x^2/(1+x^4),\,\ f^{(100)}(0)\qquad\,$
d) \ $f(x)=x^3\ln(2+x^2),\,\ f^{(87)}(0)$ \newline
e) \ $f(x)=e^{x^3}/(1+e^{x^3}),\,\ f^{(9)}(0)\qquad\ $
f) \ $f(x)=\cos(x^2\sin^2 x),\,\ f^{(12)}(0)$ 

\item
Seuraavilla käyrillä on kääntymispiste (vrt.\ edellisen luvun Esimerkki \ref{evoluutta})
annetussa pisteessä $P$. Mihin suuntaan käyrä lähtee pisteestä $P$? \vspace{1mm}\newline
a)\ $x(t)=4t+1/t,\ y(t)=t^2+16/t,\ \ P=(17/2,12)$ \newline
b)\ $\vec r(t) = t^4\vec i + (2-2\cos t - t^2)\vec j + (t^2-t\sinh t)\vec k,\ \ P=(0,0,0)$

\item
Mikä on sarjan
\[
\text{a)}\,\ 1+4+\frac{16}{2!}+\frac{64}{3!}+\frac{256}{4!}+ \ldots \quad\
\text{b)}\,\ 1+\frac{4}{3!}+\frac{16}{5!}+\frac{64}{7!}+ \ldots  
\]
summa?

\item
Esitä seuraavien funktioiden funktioiden Taylorin sarjat
pisteessä $x_0=0\,$: \vspace{1mm}\newline
a) \ $e^{3x+1}\quad$ 
b) \ $\cos(2x^3)\quad$ 
c) \ $\sin(x-\frac{\pi}{4})\quad$ 
d) \ $\cos(2x-\pi)\quad$ 
e) \ $x^2\sin 3x$\vspace{1mm}\newline
f) \ $\sin x\cos x\quad$ 
g) \ $(1+x^3)/(1+x^2)\quad$ 
h) \ $\ln(2+x^2)\quad$ 
i) \ $x^2\ln(1+x)$

\item
Määritä seuraavien funktioiden Taylorin sarja annetussa pisteessä sekä sarjan
suppenemisväli: \vspace{1mm}\newline
a) \ $e^{-2x},\,\ x_0=-1 \qquad$
b) \ $\sin x,\,\ x_0=\tfrac{\pi}{2} \qquad\,\ $
c) \ $\ln x,\,\ x_0=1$ \newline
d) \ $\cos x,\,\ x_0=\pi \qquad\,\ $
e) \ $\cos^2 x,\,\ x_0=\tfrac{\pi}{8} \qquad$ 
f) \ $x/(4+x),\,\ x_0=3$

\item
Olkoon
\[
f(x)= \begin{cases} 
      \dfrac{2\cos x-2}{x^2}\,, &\text{kun}\ x \neq 0, \\[2mm] -1, &\text{kun}\ x=0.
      \end{cases}
\]
Laske $f'(0)$ ja $f''(0)$ \ a) suoraan derivaatan määritelmästä, \ b) $f$:n Taylorin sarjan
avulla.

\item
Seuraavat funktiot $f$ määritellään kukin jatkuvaksi pisteessä $x=0$, jolloin funktiot ovat
tässä pisteessä (ja siis koko $\R$:ssä) mielivaltaisen monta kertaa derivoituvia. Määritä
funktioiden Taylorin sarjat pisteesä $x_0=0$ ja näiden avulla $f(0)$, $f'(0)$ ja $f''(0)$.
\begin{align*}
&\text{a)}\ \ \frac{e^x-1}{x} \qquad 
 \text{b)}\ \ \frac{e^{-2x}-1+2x-2x^2}{x^3} \qquad
 \text{c)}\ \ \frac{\sinh x-x}{x^3} \\
&\text{d)}\ \ \frac{\ln(1+x)-\sin x}{x^2} \qquad
 \text{e)}\ \ \frac{e^{2x}-4e^{-x}-6x+3}{x^2}
\end{align*}

\item
Tiedetään, että
\[
\sum_{k=0}^\infty a_k(x+\pi)^k\,=\,x\cos x, \quad x\in\R.
\]
Laske sarjan kertoimet $a_0$, $a_1$ ja $a_2$.

\item (*) \label{H-dif-4: Newtonin konvergenssi}
Todista Taylorin lauseen avulla Newtonin menetelmän konvergenssia koskeva Lause
\ref{Newtonin konvergenssi} .

\item (*) Potenssisarja $\sum_{k=0}^\infty a_k (x+1)^k$ suppenee pisteen $x=-1$ lähellä,
jolloin sarjan summa on
\[
\sum_{k=0}^\infty a_k (x+1)^k\,=\,\frac{x+2}{x^2+2x+3}\,.
\]
Mitkä ovat kertoimien $a_0$, $a_1$ ja $a_2$ arvot ja mikä on sarjan suppenemisväli?

\end{enumerate} %Taylorin polynomit ja Taylorin lause
\section{Taylorin polynomien sovelluksia} \label{taylorin polynomien sovelluksia}
\alku
\index{Taylorin polynomi|vahv}

Tässä luvussa tarkastellaan eräitä tavallisia Taylorin polynomien ja Taylorin lauseen 
käyttötapoja funktiotutkimuksessa ja funktioiden approksimoimisessa. Taylorin polynomeja 
laskettaessa tai jäännöstermiä arvioitaessa voidaan laskuja usein huomattavasti lyhentää ja 
selkiyttää käyttämällä nk.\ \kor{suuruusluokkamerkintöjä}. Esimakua tällaisista merkinnöistä 
on saatu jo edellisen luvun Esimerkeissä \ref{nopea Taylor 1} ja \ref{nopea Taylor 2}, joissa
käytettiin lyhennysmerkintää $[x^m]$. Tämän tilalla on tavallisempaa käyttää merkintää
$\ordoO{\abs{x}^m}$, joka luetaan 'suuruusluokkaa $x^m$' tai vain 'oo $x^m$'. Toinen,
merkitykseltään hiukan voimakkaampi suuruusluokkamerkintä on $o(\abs{x}^m)$, joka luetaan
'pikku oo $x^m$'. Näissä merkinnöissä on $\abs{x}^m$ nk.\ vertailufunktio. Tämän tilalla voi
olla yleisempi, ei-negatiivisia arvoja saava funktio, esim.\ 
$\abs{x-x_0}^\alpha,\ \alpha\in\Q$.
\begin{Def} \label{iso oo ja pikku oo} \vahv{(Suuruusluokkamerkinnät $\mathcal{O}$ ja $o$)}
\index{suuruusluokkamerkinnät|emph} 
Jos $f$ ja $g$ on määritelty välillä $[x_0-a,x_0+a]$ ($a>0$) ja 
$g(x) \ge 0\ \forall\,x\in[x_0-a,x_0+a]$, niin käytetään merkintää
\[
f(x)=\mathcal{O}(g(x)), \quad x\in[x_0-a,x_0+a],
\]
jos on olemassa vakio $C\in\R_+$, siten että pätee
\[
\abs{f(x)}\leq Cg(x) \quad\forall\ x\in[x_0-a,x_0+a].
\]
Jos $g(x)>0\,\ \forall\,x \neq x_0$ ja pätee $\,\lim_{x\kohti x_0} (f/g)(x)=0$, niin käytetään
merkintää
\[
f(x)=o(g(x)), \quad \text {kun}\ x\kohti x_0.
\]
\end{Def}

Jos merkinnällä $\ordoO{..}$ halutaan ainoastaan kertoa, että arvio on pätevä jossakin $x_0$:n 
ympäristössä, niin tämä voidaan ilmaista kirjoittamalla, kuten vastaavassa merkinnässä $o(..)$,
\[
f(x)=\ordoO{g(x)}, \quad \text{kun}\ x\kohti x_0.
\]
Näissä merkinnöissä voi raja-arvon $x_0$ tilalla olla $\pm\infty$, jolloin kyse on $f$:n 
arvioimisesta suurilla $\abs{x}$:n arvoilla. Vertailufunktio on tällöin tyypillisesti 
$g(x)=\abs{x}^\alpha$ jollakin $\alpha\in\Q$.

Laskuissa suuruusluokkamerkintöjä on kätevä käsitellä kuten funktioita, jolloin niitä voi 
yhdistellä funktioiden tavoin. Esimerkiksi $\,f(x)+o(x^2)$ tarkoittaa funktiota $\,f(x)+g(x)$,
missä $\,g(x)=o(x^2)$, ja $\,\ordoO{x^2}+\ordoO{y^2}$ tarkoittaa samaa kuin $\ordoO{x^2+y^2}$.
Määritelmästä \ref{iso oo ja pikku oo} on helposti todennettavissa seuraavat
suuruusluoka-algebran säännöt (Harj.teht.\,\ref{H-VIII-5: suuruusluokka-algebra}). Säännöissä
oletetaan, että $x,y \ge 0$.
\[ \boxed{ \begin{aligned} 
\ykehys\quad \Ord{x}+\Ord{y} &= \Ord{\max\{x,y\}}, \quad \\
\Ord{x}\Ord{y}               &= \Ord{xy}, \quad \\
\ord{x}\Ord{y}               &= \ord{xy}. \quad
           \end{aligned} } \]
Sovelluksissa kahta jälkimmäistä sääntöä käytetään tavallisimmin muodossa
\[
\Ord{x^\alpha}\Ord{x^\beta}=\Ord{x^{\alpha+\beta}}, \quad
\ord{x^\alpha}\Ord{x^\beta}=\ord{x^{\alpha+\beta}}, \quad x \ge 0,\ \alpha,\beta\in\R.
\]
\begin{Exa} Muuttujan vaihdon $x^2=t$ ja Taylorin lauseen avulla päätellään, että pisteen
$x=0$ lähellä (esim.\ välillä $[-1,1]$) on
\[
\sqrt{x^2+1} = \sqrt{1+t} = 1+\frac{1}{2}\,t + \Ord{t^2} = 1+\frac{1}{2}\,x^2+\Ord{x^4}.
\]
Suurilla $x$:n arvoilla ($x\kohti\infty$) saadaan muuttujan vaihdolla $x^{-2}=t$ vastaavasti
arvio
\[
\sqrt{x^2+1} = x\sqrt{1+x^{-2}} = x\left(1+\frac{1}{2x^2}+\Ord{x^{-4}}\right)
                                = x+\frac{1}{2x}+\Ord{x^{-3}}. \loppu
\]
\end{Exa}
\begin{Exa} Näytä, että pienillä $|x|$:n arvoilla on
\[
f(x) \,=\, \frac{1-x}{1+2x-x^2+\Ord{|x|^3}} \,=\,1-3x+7x^2+\Ord{|x|^3}.
\]
\ratk Koska pienillä $|t|$:n arvoilla on $1/(1+t)=1-t+t^2+\Ord{|t|^3}$, niin
suuruusluokka-algebra antaa
(vrt.\ edellisen luvun Esimerkit \ref{nopea Taylor 1}--\ref{nopea Taylor 2})
\begin{align*}
f(x)\,&=\, (1-x)\left\{1-\left[2x-x^2+\Ord{|x|^3}\right]
                        +\left[2x-x^2+\Ord{|x|^3}\right]^2+\Ord{|x|^3}\right\} \\
    \,&=\, (1-x)[1-(2x-x^2)+4x^2]+\Ord{|x|^3} \,=\, 1-3x+7x^2+\Ord{|x|^3}. \loppu
\end{align*}
\end{Exa}

\subsection{Funktion approksimointi}
%\index{funktion approksimointi!b@Taylorin polynomilla|vahv}

Taylorin polynomien avulla voidaan differentiaaliin perustuva approksimaatio
\[
f(x+\Delta x)\approx f(x)+f'(x)\Delta x
\]
(vrt. Luku \ref{differentiaali}) sekä yleistää että tarkentaa: Jos $f$ on $n+1$ kertaa 
jatkuvasti derivoituva pisteen $x$ ympäristössä, niin
\[
\boxed{\begin{aligned}
\quad f(x+\Delta x)=f(x)
         &+f'(x)\Delta x+\frac{1}{2!}f''(x)(\Delta x)^2 \\
         &+ \cdots + \frac{1}{n!}f^{(n)}(x)(\Delta x)^n+\Ord{\abs{\Delta x}^{n+1}}. \quad
\end{aligned}}
\]
Hyvä arvio approksimaation virheelle, tai ainakin sen suuruusluokalle, on yleensä
\[
\text{virhe}\ =\ \text{tarkka}-\text{approksimaatio}\ 
                     \approx\ \frac{1}{(n+1)!}f^{(n+1)}(x)(\Delta x)^{n+1}.
\]
\begin{Exa} $\sqrt[10]{1000}\ \approx\ ?$ \end{Exa}
\ratk
\[
\sqrt[10]{1000}\ =\ \sqrt[10]{1024-24}\ =\ 2\sqrt[10]{1-\tfrac{3}{128}}\,.
\]
Kun merkitään $f(x)=\sqrt[10]{1-x}$, niin
\begin{alignat*}{2}
f'(x)&=-\tfrac{1}{10}(1-x)^{-9/10},\quad &f''(x)&=-\tfrac{9}{100}(1-x)^{-19/10}, \\
f'''(x)&=-\tfrac{171}{1000}(1-x)^{-29/10},\quad &f^{(4)}(x)&=-\tfrac{4959}{10000}(1-x)^{-39/10}
\end{alignat*}
\begin{align*}
\impl \ &\sqrt[10]{1000} =2f(\tfrac{3}{128}) \\
&\approx 2\left[1+\left(-\tfrac{1}{10}\right)\cdot\tfrac{3}{128}
                 +\tfrac{1}{2}\cdot\left(-\tfrac{9}{100}\right)\left(\tfrac{3}{128}\right)^2 +
       \tfrac{1}{6}\cdot\left(-\tfrac{171}{1000}\right)\left(\tfrac{3}{128}\right)^3\right] \\
&\approx 2\,(\,1-0.00234275-0.0000247192-0.000000366926\,) \\[2mm]
&= 1.995262327671.
\end{align*}
Virhe (tarkka arvo $-$ likiarvo) on suuruusluokkaa
\[
+\frac{1}{4!}f^{(4)}(0)\cdot (\Delta x)^4 
             = -\tfrac{1653}{80000}\cdot\left(\tfrac{3}{128}\right)^4\approx -6\cdot 10^{-9}.
\]
Oikea 12-desimaalinen arvo on
\[
\sqrt[10]{1000}\approx 1.995262314969. \loppu
\]

\begin{Exa}\ Arvioi $\,\sin 31^\circ\,$ lähtien arvosta $\,\sin 30^\circ = \tfrac{1}{2}$.
\end{Exa}
\ratk Kun $f(x)=\sin x$,\ $x=\tfrac{\pi}{6}$,\ $\Delta x=\tfrac{\pi}{180}$, on
\begin{align*}
f(x+\Delta x) \,&\approx\, \sin x + \cos x\,\Delta x-\tfrac{1}{2}\sin x\,(\Delta x)^2 \\
              \,&=\, \tfrac{1}{2}+\tfrac{\sqrt{3}}{2}\cdot\tfrac{\pi}{180}
                  -\tfrac{1}{2}\cdot\tfrac{1}{2}\cdot\left(\tfrac{\pi}{180}\right)^2\ 
                 \approx\ 0.51503882.
\end{align*}
Virhe (tarkka arvo $-$ likiarvo) on luokkaa
\[
-\frac{1}{3!}\cos x\,(\Delta x)^3\approx -8\cdot 10^{-7}.
\]
Oikea 8-desimaalinen arvo on: $\,\sin 31^\circ\approx 0.51503807$. \loppu

\subsection{Paikalliset ääriarvot}

Lauseen \ref{ääriarvolause} mukaan derivoituvan funktion paikallisessa ääriarvokohdassa on
myös derivaatan nollakohta. Jos funktio on derivaatan nollakohdan ympäristössä riittävän 
säännöllinen, niin Taylorin polynomien avulla voidaan selvittää, onko kyseessä todella 
ääriarvokohta ja jos, niin minkälaatuinen.
\begin{Lause} \label{Taylorin ääriarvolause}
\index{paikallinen maksimi, minimi, ääriarvo|emph}
\index{maksimi (funktion)!a@paikallinen|emph} 
\index{minimi (funktion)!a@paikallinen|emph}
\index{zyzy@ääriarvo (paikallinen)|emph}
\index{kriittinen piste!a@luokittelu|emph}
Jos $f$ on pisteen $c$ ympäristössä $n$ kertaa jatkuvasti derivoituva, $n\geq 2$, ja pätee
$\,f^{(k)}(c)=0,\quad k=1\ldots n-1\,$ ja $\,f^{(n)}(c)\neq 0$, niin
\begin{itemize}
\item[a)] jos $n$ on pariton, niin $f$:llä ei ole $c$:ssä paikallista ääriarvoa,
\item[b)] jos $n$ on parillinen, niin $f$:llä on $c$:ssä
\begin{itemize}
\item[-] paikallinen minimi, jos $f^{(n)}(c)>0$,
\item[-] paikallinen maksimi, jos $f^{(n)}(c)<0$.
\end{itemize}
\end{itemize}
\end{Lause}
\tod Lauseen \ref{Taylorin approksimaatiolause} ja oletuksien mukaan
\[
f(x)=f(c)+\frac{1}{n!}f^{(n)}(c)(x-c)^n+o(\abs{x-c}^n), \quad \text{kun}\ x \kohti c.
\]
Koska $f^{(n)}(c)\neq 0$, tämä voidaan kirjoittaa myös muotoon
\[
f(x)=f(c)+\frac{1}{n!}f^{(n)}(c)\,[\,1+o(1)\,]\,(x-c)^n, \quad \text{kun}\ x\kohti c,
\]
jolloin nähdään, että riittävän pienellä $\delta$ ($\delta>0$) on oltava
\[
f(x)=f(c)+\frac{1}{n!}f^{(n)}(c)k(x)(x-c)^n,\quad x\in [c-\delta,c+\delta],
\]
missä esimerkiksi
\[
\frac{1}{2}\leq k(x)\leq \frac{3}{2}\,.
\]
Näin ollen jos $n$ on parillinen ja $f^{n)}(c)>0$, on $f(x)>f(c)$ kun $x\in[c-\delta,c)$ tai
$x\in(c,c+\delta]$, jolloin $c$ on paikallinen minimikohta
(Määritelmä \ref{paikallinen ääriarvo}). Muissa tapauksissa on päättely vastaava
(vrt.\ kuvio). \loppu
\begin{figure}[H]
\setlength{\unitlength}{1cm}
\begin{center}
\begin{picture}(11,4)(0,-2)
\multiput(0,0)(3,0){4}{\vector(1,0){2}}
\multiput(1.8,-0.5)(3,0){4}{$x$}
\multiput(0.9,-0.4)(3,0){4}{$c$}
\multiput(1,0)(3,0){4}{\dashline{0.1}(0,0)(0,1)}
\put(0,-2){\parbox{2cm}{\small $n$ pariton, $f^{(n)}(c)>0$}}
\put(3,-2){\parbox{2cm}{\small $n$ pariton, $f^{(n)}(c)<0$}}
\put(6,-2){\parbox{2cm}{\small $n$ parillinen $f^{(n)}(c)>0$}}
\put(9,-2){\parbox{2cm}{\small $n$ parillinen, $f^{(n)}(c)<0$}}
\curve(
  0,  0.5,
0.2,0.744,
0.4,0.892,
0.6,0.968,
0.8,0.996,
  1,    1,
1.2,1.004,
1.4,1.032,
1.6,1.108,
1.8,1.256,
  2,  1.5)
\curve(
  3,  1.5,
3.2,1.256,
3.4,1.108,
3.6,1.032,
3.8,1.004,
  4,    1,
4.2,0.996,
4.4,0.968,
4.6,0.892,
4.8,0.744,
  5,  0.5)
\curve(
  6,   1.5,
6.2,1.2048,
6.4,1.0648,
6.6,1.0128,
6.8,1.0008,
  7,     1,
7.2,1.0008,
7.4,1.0128,
7.6,1.0648,
7.8,1.2048,
  8,   1.5)
\curve(
  9,   0.5,
9.2,0.7952,
9.4,0.9352,
9.6,0.9872,
9.8,0.9992,
  10,     1,
10.2,0.9992,
10.4,0.9872,
10.6,0.9352,
10.8,0.7952,
  11,   0.5)
\end{picture}
\end{center}
\end{figure}
\begin{Exa}
Tutki mahdollisen ääriarvokohdan laatu seuraavissa tapauksissa: \vspace{2mm}\newline
a) \   $f(x)=\sin x + \cos x, \quad c=\pi/4$ \newline
b) \   $f(x)=e^{-x}+\sin x + \sqrt{1-x^2}, \quad c=0$ \newline
c) \,\ $f(x)=1-\cos x-\sqrt{1+x^2}, \quad c=0$
\begin{align*} 
\text{\ratk} \quad &\text{a)} \quad f'(\pi/4)=0,\quad f''(\pi/4)=-\sqrt{2} \qquad
                                    \qimpl \text{paik.\ maksimi.} \hspace{1cm} \\[1mm]
                   &\text{b)} \quad f'(0)=f''(0)=0,\quad f'''(0)=-2 \quad\,
                                    \qimpl \text{ei ääriarvoa.} \\
                   &\text{c)} \quad f^{(k)}(0)=0,\ k=1 \ldots 3,\ f^{(4)}(0)=5 \,
                                    \qimpl \text{paik.\ minimi.} \loppu
\end{align*}
\end{Exa}

\subsection{Funktio $\tfrac{f(x)}{g(x)}=\tfrac{0}{0}\,$, kun $x=a$}
%\index{funktion approksimointi!b@Taylorin polynomilla|vahv}

Jos funktiot $f$ ja $g$ ovat pisteen $a$ ympäristössä säännöllisiä ja $f(a)=g(a)=0$, niin
Taylorin lauseen avulla voidaan tutkia, millainen funktio $F(x)=f(x)/g(x)$ on pisteen $a$
ympäristössä. Ensinnäkin jos halutaan ratkaista raja-arvokysymys
\[
\lim_{x\kohti a} \frac{f(x)}{g(x)}=\,?
\]
niin Taylorin lauseeseen perustuva menettely on seuraava: Oletetaan, että $f$ ja $g$ ovat $n$
kertaa jatkuvasti derivoituvia pisteen $a$ ympäristössä ja että pätee
\[
\frac{f(a)}{g(a)}=\frac{0}{0}\,,\quad \frac{f'(a)}{g'(a)}=\frac{0}{0}\,,\ \ldots,\ 
                                      \frac{f^{(n-1)}(a)}{g^{(n-1)}(a)}=\frac{0}{0}\,, \quad 
                                      \frac{f^{(n)}(a)}{g^{(n)}(a)}=\frac{A}{B}\,,
\]
missä $A \neq 0$ tai $B \neq 0$. Tällöin Lauseen \ref{Taylorin approksimaatiolause} mukaan
\[
\frac{f(x)}{g(x)} = \frac{\frac{A}{n!}\,(x-a)^n + \ord{\abs{x-a}^n)}}{\frac{B}{n!}\,(x-a)^n 
                                                + \ord{\abs{x-a}^n)}}
                  = \frac{A+\ord{1}}{B+\ord{1}}, \quad \text{kun}\ x \kohti a.
\]
Päätellään, että jos $A \neq 0$ ja $B=0$, niin $\abs{f(x)/g(x)}\kohti\infty$, kun
$x \kohti a$, jolloin raja-arvoa ei ole (reaalilukuna --- voi olla $\lim=\pm\infty$). Muussa
tapauksessa, eli jos $B \neq 0$, on
\[
\lim_{x \kohti a} \frac{f(x)}{g(x)} = \frac{A}{B} = \frac{f^{(n)}(a)}{g^{(n)}(a)}\,,
\]
ja tällöin on yleisemminkin
\[
\lim_{x \kohti a} \frac{f(x)}{g(x)} 
               = \lim_{x \kohti a} \frac{f'(x)}{g'(x)} 
               =\ \ldots\ = \lim_{x \kohti a} \frac{f^{(n)}(x)}{g^{(n)}(x)}\,.
\]
Tämä tulos on siis voimassa (tehtyjen säännöllisyysoletusten puitteissa), kunhan ketjun 
\pain{viimeinen} raja-arvo on olemassa. --- Tulos on jo ennestään tuttu l'Hospitalin
sääntönä (Lause \ref{Hospital}).
\begin{Exa} Määritä raja-arvo 
$\ \displaystyle{\lim_{x \kohti 0} \frac{\sin 2x-2x}{\sinh x-x}\,}$. 
\end{Exa}
\ratk \,Funktiot $f(x)=\sin 2x-2x$ ja $g(x)=\sinh x-x$ ovat sileitä $\R$:ssä. Derivoimalla 
todetaan
\[
\frac{f'(x)}{g'(x)}=\frac{2\cos 2x-2}{\cosh x-1}\,, \quad 
\frac{f''(x)}{g''(x)}=\frac{-4\sin 2x}{\sinh x}\,, \quad
\frac{f'''(x)}{g'''(x)}=\frac{-8\cos 2x}{\cosh x}\,,\ \ldots
\]
Havaitaan, että $f^{(k)}(0)=g^{(k)}(0)=0$, kun $k=0,1,2$, ja $f'''(0)=-8,\ g'''(0)=1$. Siis
\[
\lim_{x \kohti 0} \frac{\sin 2x-2x}{\sinh x-x} = \frac{-8}{1} = -8. \loppu
\]

Taylorin lauseen perusteella saadaan siis raja-arvolle $\lim_{x \kohti a} f(x)/g(x)$ sama
laskukaava kuin l'Hospitalin säännöllä. Funktiosta $f(x)/g(x)$ saadaan kuitenkin Taylorin
polynomien avulla selville paljon muutakin. Seuraavassa esimerkki vaativammasta tehtävän
asettelusta.
\begin{Exa} Määritä polynomit $p$ ja $q$ siten, että väleillä $[-1,0)$ ja $(0,1]$ pätee
\[
F(x)=\frac{\sin 2x-2x}{\sinh x-x} \,=\, p(x)+\Ord{x^4},\quad\ 
G(x)=\frac{\sin x}{2x^2}+\frac{\cos x -1}{x^3}=q(x)+\Ord{\abs{x}^5}.
\]
\begin{align*}
\text{\ratk} \quad  \sin 2x -2x\
         &=\ \Bigl(2x-\frac{1}{6}(2x)^3+\frac{1}{120}(2x)^5+\Ord{\abs{x}^7}\Bigr)-2x 
                                                                           \hspace{2cm} \\
         &=\ -\frac{4}{3}x^3+\frac{4}{15}x^5+\Ord{\abs{x}^7}, \\
\sinh x -x \,\ 
         &=\ \Bigl(x+\frac{1}{6}x^3+\frac{1}{120}x^5+\Ord{\abs{x}^7}\Bigr)-x \\
         &=\ \frac{1}{6}x^3+\frac{1}{120}x^5+\Ord{\abs{x}^7} \\
\qimpl \frac{\sin 2x-2x}{\sinh x-x}\ 
         &=\ \frac{-8+\frac{8}{5}x^2+\ordoO{x^4}}{1+\frac{1}{20}x^2+\Ord{x^4}} \\[2mm]
         &=\ -8+2x^2+\Ord{x^4} \,=\, p(x)+\Ord{x^4}. \\[5mm]
\frac{\sin x}{2x^2}+\frac{\cos x -1}{x^3}\ 
         &=\ \frac{1}{2x^2}\Bigl(x-\frac{x^3}{6}+\frac{x^5}{120}+\Ord{\abs{x}^7}\Bigr) \\
         & \quad\ + \frac{1}{x^3}\Bigl(-\frac{1}{2}x^2 +\frac{1}{24}x^4-\frac{1}{720}x^6 + 
                                                      \Ord{x^8}\Bigr) \\
         &=\ -\frac{1}{24}x+\frac{1}{360}x^3+\Ord{\abs{x^5}} \,=\, q(x)+\Ord{\abs{x}^5}.
\end{align*}

\vspace{2mm}

Tuloksista voi päätellä esimerkiksi raja-arvot
\[
\lim_{x\kohti 0}\left(\frac{\sin 2x-2x}{x^2\sinh x-x^3}+\frac{8}{x^2}\right) = 2, \qquad 
\lim_{x\kohti 0} \left(\frac{\sin x}{2x^3}+\frac{\cos x -1}{x^4}\right)=-\frac{1}{24}\,.
\]
Nähdään myös, että jos asetetaan (jatkamisperiaatteella) $F(0)=-8$ ja $G(0)=0$, niin $F$:llä
on pisteessä $x=0$ paikallinen minimi ja $G$ on origon ympäristössä aidosti vähenevä. \loppu
\end{Exa}

\subsection{Differenssiapproksimaatiot}
\index{differenssiapproksimaatio|vahv}

Taylorin lauseella on usein keskeinen rooli, kun halutaan johtaa virhearvioita numeerisille
(likimääräisille) laskentamenetelmille, jotka perustuvat oletettuun funktion säännöllisyyteen.
Tarkastellaan esimerkkinä derivaattojen numeerisessa laskemisessa käytettäviä nk.\
\kor{differenssiapproksimaatioita}. Näissä ei funktiota oleteta tunnetuksi lausekkeena vaan
riittää tuntea funktion arvo äärellisen monessa (tyypillisesti vain muutamassa) pisteessä
tarkasteltavan pisteen lähellä. Seuraavat kolme ovat differenssiapproksimaatioista
tavallisimmat (ol.\ $h>0$):
\begin{align}
f'(a)  &\,\approx\, \frac{f(a+h)-f(a)}{h}\,, \label{dif1} \tag{\text{a}} \\
f'(a)  &\,\approx\, \frac{f(a+h)-f(a-h)}{2h}\,, \label{dif2} \tag{\text{b}} \\
f''(a) &\,\approx\, \frac{f(a+h)-2f(a)+f(a-h)}{h^2}\,. \label{dif3} \tag{\text{c}}
\end{align}
Näistä (a) perustuu suoraan derivaatan määritelmään. Approksimaatioita (b) ja (c) sanotaan 
\index{keskeisdifferenssiapproksimaatio}%
\kor{keskeis}differenssiapproksimaatioiksi, syystä että funktio evaluoidaan näissä $a$:n
suhteen symmetrisessä pisteistössä.

Em.\ approksimaatioille voidaan johtaa virhearvio Taylorin lauseesta olettaen, että $f$ on
riittävän säännöllinen. Tarkastellaan esimerkkinä approksimaatiota (b), muut jätetään
harjoitustehtäviksi (Harj.teht.\,\ref{H-dif-5: virhearvio (a)},\ref{H-dif-5: virhearvio (c)}).
\begin{Prop} \label{keskeisdifferenssin tarkkuus} Jos $f$ on kolmesti jatkuvasti derivoituva
välilä $[a-h,a+h]$, niin differenssiapproksimaatiolle (b) pätee virhearvio
\[
\left|\,f'(a)-\frac{f(a+h)-f(a-h)}{2h}\,\right| \,\le\,\frac{1}{6}M\,h^2, \quad
                                          M=\max_{x\in[a-h,a+h]}|f'''(x)|.
\]
\end{Prop}
\tod Oletuksien perusteella on joillakin $\,\xi_1\in(a,a+h)\,$ ja $\,\xi_2\in(a-h,a)$ 
\begin{align*}
f(a+h) &= f(a)+f'(a)h+\frac{1}{2}f''(a)\,h^2+\frac{1}{6}f'''(\xi_1)\,h^3, \\
f(a-h) &= f(a)-f'(a)h+\frac{1}{2}f''(a)\,h^2-\frac{1}{6}f'''(\xi_2)\,h^3
\end{align*}
(Lause \ref{Taylor}: välin $[a,b]$ tilalla väli $[a-h,a+h]$, $x_0=a$, $x=a \pm h$, $n=2$).
Näistä seuraa vähennyslaskulla
\[
f'(a)-\frac{f(a+h)-f(a-h)}{2h} = -\frac{1}{12}\,[f'''(\xi_1)+f'''(\xi_2)]\,h^2.
\]
Käyttämällä oikealla arviota $|f'''(\xi_1)+f'''(\xi_2)| \le |f'''(\xi_1)|+|f'''(\xi_2)| \le 2M$
seuraa väite. \loppu

Proposition \ref{keskeisdifferenssin tarkkuus} perusteella approksimaatio (b) on tarkka
toisen asteen polynomeille (joille $M=0$). Valitsemalla $f(x)=\frac{1}{6}M(x-a)^3$ nähdään
myös, että virhearvio on tehdyin oletuksin tarkin mahdollinen.

Yleisesti sanotaan, että differenssiapproksimaation (tarkkuuden)
\index{kertaluku!b@tarkkuuden}%
\kor{kertaluku} on $r$, jos
sen virhe on $\mathcal{O}(h^r)$ mutta ei $o(h^r)$ yleiselle, riittävän säännölliselle
funktiolle. Approksimaation (b) kertaluku on siis $r=2$, eli kyseessä on
\kor{toisen kertaluvun} approksimaatio. Approksimaation (c) kertaluku on samoin $r=2$
(Harj.teht.\,\ref{H-dif-5: virhearvio (c)}), sen sijaan (a) on ensimmäisen kertaluvun ($r=1$)
approksimaatio (Harj.teht.\,\ref{H-dif-5: virhearvio (a)}).

\Harj
\begin{enumerate}

\item \label{H-VIII-5: suuruusluokka-algebra}
Perustele suuruusluokka-algebran säännöt (ol.\ $x,y \ge 0$)
\[
\Ord{x}+\Ord{y} = \Ord{\max\{x,y\}}, \quad 
\Ord{x}\Ord{y} = \Ord{xy}, \quad
\ord{x}\Ord{y} = \ord{xy}.
\]

\item
Mitä funktion $f$ ominaisuuksia tarkoitetaan seuraavilla merkinnöillä? \newline
a) \ $f(x)=f(a)+\ord{1}, \quad \text{kun}\,\ x \kohti a$. \newline
b) \ $f(x_1)-f(x_2)=\Ord{\abs{x_1-x_2}}, \quad \text{kun}\,\ x_1,x_2\in[a,b]$. \newline
c) \ $f(x)=f(a)+k(x-a)+o\,(\abs{x-a}), \quad \text{kun}\,\ x \kohti a$.

\item
Anna suuruusluokka-arviot seuraavien approksimaatioiden virheille: \vspace{1mm}\newline
a) \ $\sqrt{x^4+2} \approx \sqrt{2}, \quad \text{kun}\ x \kohti 0$ \vspace{2mm}\newline
b) \ $\sqrt{x^4+3x} \approx \sqrt{3x}, \quad \text{kun}\ x \kohti 0^+$ \vspace{1mm}\newline
c) \ $\sqrt{x^4+4x} \approx x^2, \quad \text{kun}\ x\kohti\pm\infty$ \vspace{1mm}\newline
d) \ $\sqrt{x^4+4x} \approx x^2+2x^{-1} \quad \text{kun}\ x\kohti\infty$ \vspace{0.5mm}\newline
e) \ $x/\sin x \approx 1, \quad \text{kun}\ x \kohti 0$ \vspace{1mm}\newline
f) \ $\ln(1+e^x) \approx x, \quad \text{kun}\ x\kohti\infty$

\item
Arvioi approksimaation $f(x) \approx T_2(x,0)$ virheen itseisarvo välillä $[-1/2,\,1/2]$
seuraaville funktioille: \vspace{1mm}\newline
a) \, $\sqrt{1+x}\quad\,\ $ 
b) \, $\sqrt[10]{1+x}\quad\,\ $ 
c) \, $\tan x\quad\,\ $ 
d) \, $\ln(1+x)\quad\,\ $ 
e) \, $e^{\sin x}$   

\item
Arvioi seuraavien funktioapproksimaatioiden virhe:
\begin{align*}
&\text{a)}\ \ \sqrt[3]{8+x} \approx 2+\frac{x}{12}-\frac{x^2}{288}\,, \quad 
                                         \text{kun}\ \abs{x} \le 1 \\
&\text{b)}\ \ \sin x \approx x-\frac{1}{6}x^3, \quad 
                                         \text{kun}\ x\vastaa\alpha\in[0\aste,10\aste] \\
&\text{c)}\ \ \frac{1}{x^4}\left[\frac{1}{\sqrt{1+x^2}}-\sqrt{1-x^2}\right] 
            \approx \frac{1}{2}\,, \quad \text{kun}\ \abs{x} \le 0.15\,\ \text{ja}\,\ x \neq 0
\end{align*}

\item
Jos $T_n(x,0)$ on funktion $f(x)=\cos x$ Taylorin polynomi, niin millä $a$:n ja $n$:n arvoilla
voidaan taata, että \newline
a) \ $\abs{\cos x-T_2(x,0)} \le 10^{-4}$ välillä $[-a,a]$, \newline
b) \ $\abs{\cos x-T_n(x,0)} \le 10^{-4}$ välillä $[-\pi/2,\,\pi/2]$\,?

\item
a) Laske luvulle $1/\sqrt[4]{e}$ rationaalinen likiarvo approksimoimalla funktiota $e^x$ toisen
asteen Taylorin polynomilla. Arvioi approksimaation virhe Taylorin lauseen avulla ja vertaa
virheen tarkkaan arvoon. \vspace{1mm}\newline
b) Laske luvulle $a=\sqrt[12]{4000}$ rationaalinen likiarvo kirjoittamalla $4000=4096(1-x)$ ja
approksimoimalla funktiota $f(x)=\sqrt[12]{1-x}$ toisen asteen Taylorin polynomilla. Arvioi
approksimaation virhe Taylorin lauseen avulla ja vertaa virheen tarkkaan arvoon.

\item
Suhteellisuusteorian mukaan vauhdilla $v$ liikkuvan partikkelin liike-energia on
\[
E_k = \frac{mc^2}{\sqrt{1-\frac{v^2}{c^2}}}-mc^2,
\]
missä $m$ on partikkelin massa ja $c \approx 3 \cdot 10^8$ m/s on valon nopeus. Millaisilla 
vauhdin $v$ arvoilla approksimaation $E_k \approx \frac{1}{2} mv^2$ suhteellinen virhe on
enintään $0.01\%$\,? 

\item 
Seuraavilla funktioilla piste $x=0$ on mahdollinen paikallinen minimi- tai maksimikohta.
Tutki asia derivoimalla!
\[
\text{a)}\ (1-x)e^x \quad\
\text{b)}\ (2-x^2)\cos x \quad\
\text{c)}\ (x-x^4)\sin x \quad\
\text{d)}\ \sqrt{1-x^2}-1/\sqrt{1+x^2}
\]

\item
Millä $a$:n ja $b$:n arvoilla funktiolla
\[
\text{a)}\ e^x+ax+bx^2 \quad\ 
\text{b)}\ x\sin x+ax^2+bx^4 \quad\ 
\text{c)}\ \sin x+\cos x+ax+bx^2
\]
on paikallinen minimi pisteessä $x=0$\,?

\item
Yhtälö $\,y(\cos y-\sin y)=2\sin x+\cos x+ax+b\,$ määritelee pisteen $x=0$ ympäristössä
funktion $y(x)$. Määritä vakiot $a$ ja $b$, kun tiedetään, että $y(x)$ saavuttaa pisteessä
$x=0$ paikallisen ääriarvon $y(0)=0$. Onko kyseessä maksimi vai minimi?

\item
Määritä seuraavat raja-arvot:
\begin{align*}
&\text{a)}\ \ \lim_{x \kohti 0}\frac{\sqrt[7]{1+x}-1}{x} \qquad
 \text{b)}\ \ \lim_{x \kohti 0}\frac{3x}{\tan 4x} \qquad
 \text{c)}\ \ \lim_{x \kohti 0}\frac{ 2\cos x-2+x^2 }{x^4 } \\
&\text{d)}\ \ \lim_{x \kohti 0}\frac{x-\sin x}{x-\tan x} \qquad
 \text{e)}\ \ \lim_{x \kohti 0}\frac{10^x-e^x}{x} \qquad
 \text{f)}\ \ \lim_{x \kohti 0}\frac{1-\cos ax}{1-\cos bx} \\
&\text{g)}\ \ \lim_{x\kohti\pi/2}\frac{\cos 3x}{\pi-2x} \qquad
 \text{h)}\ \ \lim_{t\kohti\pi/2}\frac{\ln\sin t}{\cos t} \qquad
 \text{i)}\ \ \lim_{x \kohti 0^+}\frac{\sin^2 x}{\tan x-x} \\
&\text{j)}\ \ \lim_{x \kohti 1}\frac{\ln(ex)-1}{\sin\pi x} \qquad
 \text{k)}\ \ \lim_{x \kohti -\infty} x\sin\frac{1}{x} \qquad
 \text{l)}\ \ \lim_{x\kohti\infty}x^2\ln(\cos\frac{\pi}{x}) \\[1mm]
&\text{m)}\ \ \lim_{x\kohti\infty} \left[\sqrt{x^2+154x}-\ln(5+e^x)\right] \qquad
 \text{n)}\ \ \lim_{x\kohti\infty}\left(\sqrt[6]{x^6+42x^5+77x}-x\right) \qquad
\end{align*}

\item
Määritä seuraaville funktioille $f$ mahdollisimman alhaista astetta oleva polynomi $p$
siten, että annetulla $m$ ja jollakin $\delta>0$ on $f(x)=p(x)+\Ord{|x|^m}$, kun
$0<|x|<\delta$. Tutki myös pisteen $x=0$ laatu mahdollisena $f$:n paikallisena
ääriarvokohtana, kun asetetaan $f(0)=\lim_{x \kohti 0} f(x)$.
\begin{align*}
&\text{a)}\,\ \frac{1-\cos x}{x^2}\,,\,\ m=7 \qquad
 \text{b)}\,\ \frac{\sin x-x}{x^3}\,,\,\ m=8 \qquad
 \text{c)}\,\ \frac{x^3}{\sin x-x}\,,\,\ m=3 \\
&\text{d)}\,\ \frac{x}{e^x-1}\,,\,\ m=3 \qquad
 \text{e)}\,\ \frac{\ln(1+x)}{x}\,,\,\ m=5 \qquad
 \text{f)}\,\ \frac{\sqrt{1+x}-1}{\sqrt[3]{1-x}-1}\,,\,\ m=3
\end{align*}

\item \label{H-dif-5: virhearvio (a)}
Näytä, että jos $f$ on jatkuva välillä $[a,a+h]$ ja
kahdesti derivoituva välillä $(a,a+h)$, niin jollakin $\xi\in(a,a+h)$ pätee
\[
f'(a)-\frac{f(a+h)-f(a)}{h} \,=\, -\frac{1}{2}\,f''(\xi)\,h.
\]

\item (*) \label{H-dif-5: virhearvio (c)}
Näytä Taylorin lauseen avulla, että jos $f$ on neljä
kertaa jatkuvasti derivoituva välillä $[a-h,a+h]$, niin pätee
\[
\left|\,f''(a)-\frac{f(a+h)-2f(a)+f(a-h)}{h^2}\,\right| 
               \,\le\,\frac{1}{12}\left(\max_{x\in[a-h,a+h]}|f^{(4)}(x)|\right) h^2.
\]
Miten arvio toteutuu funktiolle $f(x)=(x-a)^4$\,?

\item (*)
a) Näytä, että eräillä (millä?) $A$:n ja $B$:n arvoilla pätee
\[
\Arccos x\,=\,A\sqrt{1-x}+B(1-x)^{3/2}+\mathcal{O}\left((1-x)^{5/2}\right),
\quad \text{kun}\ x \kohti 1^-.
\]
b) Näytä, että jos sykloidin yhtälöt $x=R(t-\sin t),\ y=R(1-\cos t)$ eliminoidaan muotoon
$y=y(x)$ (ratkaisemalla $t=t(x)$ ensimmäisestä yhtälöstä), niin origon ympäristössä pätee
\[
y(x) = 3\sqrt[3]{\frac{Rx^2}{6}}+\mathcal{O}\left(\sqrt[3]{\frac{x^4}{R}}\,\right).
\]

\item (*)
Olkoon $c$ funktion $f$ yksinkertainen nollakohta ja olkoon $f$ kolme kertaa jatkuvasti
derivoituva $c$:n ympäristössä. Halutaan määrätä $c$ Newtonin menetelmästä muunnellulla
iteraatiolla $\,x_{n+1}=F(x_n)$, missä
\[
F(x)=x-\frac{f(x)}{f'(x)}+[f(x)]^2g(x).
\]
Miten $g(x)$ on valittava, jotta iteraation suppeneminen kohti $c$:tä on vähintään
kuutiollista? Millä ehdolla suppeneminen on täsmälleen kuutiollista? Sovella menetelmää 
funktioon $f(x)=x^2-a,\ a>0$
(vrt.\ Harj.teht.\,\ref{kiintopisteiteraatio}:\ref{H-V-7: kuutiollisia iteraatioita}b).

\end{enumerate}
 %Taylorin polynomien sovelluksia
\section{Interpolaatiopolynomit} \label{interpolaatiopolynomit}
\alku
\index{interpolaatiopolynomi|vahv}

Tavallisessa \kor{polynomi-interpolaatiossa} on lähtöajatuksena, että funktiosta $f$ tunnetaan
vain äärellinen määrä pistearvoja:
\[
f(x_i)=f_i,\quad i=0 \ldots n.
\]
Näiden tietojen perusteella halutaan esittää $f$ likimäärin polynomina muuallakin kuin
pisteissä $x_i$, esim.\ jollakin välillä. Interpolaatio on kyseessä erityisesti silloin, kun
$f(x)$ halutaan laskea 'välipisteissä' $x \in [\min\{x_i\},\max\{x_i\}]$, muussa tapauksessa
(eli kun $x < \min\{x_i\}$ tai $x > \max\{x_i\}$) sanotaan, että kyseessä on
\kor{ekstrapolaatio}. Annetut tiedot voivat olla esim.\ mitattua 'dataa'. Toinen yleinen
sovellustilanne on sellainen, jossa funktio $f$ tunnetaan epäsuorasti, esim.\
differentiaaliyhtälön ratkaisuna, ja halutaan laskea arvot $f_i$ valituissa pisteissä $x_i$.
Tällöin polynomi-interpolaatioista on hyötyä itse laskenta-algoritmin suunnittelussa.

Funktion polynomiapproksimaatiossa on yleensä perusoletuksena (tai ainakin toivomuksena), että
funktio on riittävän säännöllinen, jolloin Taylorin lauseen perusteella tiedetään, että 
funktiota voi (ainakin lyhyellä välillä) approksimoida hyvin polynomilla, nimittäin Taylorin
polynomilla. Koska itse Taylorin polynomia ei ym.\ tiedoista voi suoraan määrätä, on luontevaa
valita approksimoivaksi polynomiksi $p$ sellainen, joka sopii annettuihin tietoihin, eli
toteuttaa
\begin{equation} \label{interpolaatioehdot}
p(x_i)=f(x_i),\quad i=0 \ldots n.
\end{equation}
Koska tässä on ehtoja $n+1$ kpl, on edelleen luonnollista valita $p$:n asteluvuksi $n$, jolloin
$p$:ssä on vapaita kertoimia myöskin $n+1$ kpl. Näin määriteltyä polynomia $p$ sanotaan $f$:n 
\index{interpolaatiopolynomi!a@Lagrangen} \index{Lagrangen!b@interpolaatio(polynomi)}%
\kor{Lagrangen interpolaatiopolynomiksi} pisteissä $x_i$. Pisteitä $x_i$ sanotaan tässä 
yhteydessä \kor{interpolaatiopisteiksi} ja ehtoja \eqref{interpolaatioehdot} 
\kor{interpolaatioehdoiksi}. Nämä  ehdot todella määrittelevät yksikäsitteisen polynomin
astetta $n$:
\begin{Prop} \label{interpolaatiopolynomin yksikäsitteisyys}
Lagrangen interpolaatiopolynomi on yksikäsitteinen.
\end{Prop}
\tod Ehdot \eqref{interpolaatioehdot} toteuttavan interpolaatiopolynomin olemassaolo seuraa 
jäljempänä esitettävästä laskukaavasta \eqref{Lagrangen interpolaatiokaava}, joten riittää
näyttää yksikäsitteisyys. Olkoot siis $p_1$ ja $p_2$ molemmat ehdot \eqref{interpolaatioehdot}
täyttäviä polynomeja astetta $n$. Tällöin $q(x)=p_1(x)-p_2(x)$ on polynomi enintään astetta $n$
ja pätee
\[
q(x_i)=0, \quad i=0 \ldots n,
\]
eli $q$:lla on $n+1$ reaalista nollakohtaa. Algebran peruslauseen mukaan on tällöin oltava
$q=0$. Siis $p_1=p_2$, eli ehdot \eqref{interpolaatioehdot} täyttävä polynomi (sikäli kuin
olemassa) on yksikäsitteinen. \loppu

Lagrangen interpolaatiopolynomin approksimaatiovirheelle pätee seuraava tulos, joka muistuttaa 
Taylorin polynomien virhekaavaa (Lause \ref{Taylor}). Jäljempänä nähdäänkin
(Lause \ref{usean pisteen Taylor}), että nämä kaksi tulosta ovat erikoistapauksia yleisemmästä
interpolaatiopolynomien virhekaavasta. Jatkossa sanotaan äärellisen pisteistön $X$ 
\index{virittää (väli)}%
\kor{virittämäksi} väliksi suljettua väliä $[a,b]$, missä $a = \min \{x \mid x \in X\}$ ja 
$b = \max \{x \mid x \in X\}$.
\begin{Lause} \label{Lagrangen interpolaatiovirhe}
Olkoon $f\,$ jatkuva välillä $[a,b]$ ja $n+1$ kertaa derivoituva välillä $(a,b)$, missä $[a,b]$
on pisteen $x$ ja erillisten pisteiden $x_0,\ldots,x_n$ virittämä väli, $n\in\N$. Tällöin jos
$p$ on $f$:n Lagrangen interpolaatiopolynomi pisteissä $x_0,\ldots,x_n$, niin jollakin
$\xi \in (a,b)$ pätee virhekaava
\[
f(x)-p(x)=\frac{1}{(n+1)!}\,f^{(n+1)}(\xi) \prod_{i=0}^n (x-x_i).
\]
\end{Lause}
\tod perustuu samaan ideaan kuin Taylorin lauseen todistus Luvussa \ref{taylorin lause}.
Ensinnäkin jos $x\in\{x_0,\ldots,x_n\}$, on väittämä tosi jokaisella $\xi \in (a,b)$, joten
voidaan olettaa, että $x\notin\{x_0,\ldots,x_n\}$. Merkitään
\[
w(t)=\prod_{i=0}^n (t-x_i)
\]
ja tutkitaan funktiota
\[
g(t)=f(t)-p(t)- Hw(t),\quad H=[f(x)-p(x)]/w(x).
\]
Funktion $g$ nollakohtia ovat interpolointipisteet $x_0,\ldots,x_n$ ja lisäksi piste $x$. Koska
nollakohtia välillä $[a,b]$ on siis ainakin $n+2$ kpl, ja koska näiden välissä on aina 
derivaatan nollakohta (Lause \ref{Rollen lause}), on $g'(t)$:llä ainakin $n+1$ nollakohtaa
avoimella välillä $(a,b)$. Tällöin $g''$:lla on ainakin $n$ nollakohtaa tällä välillä, ja
lopulta $g^{(n+1)}$:lla ainakin yksi nollakohta $\xi\in (a,b)$. Mutta tällöin
\begin{align*}
0 = g^{(n+1)}(\xi) &= f^{(n+1)}(\xi)-p^{(n+1)}(\xi)-H w^{(n+1)}(\xi) \\
                   &= f^{(n+1)}(\xi)-H (n+1)! \\[2mm]
      \impl\quad H &= \frac{f(x)-p(x)}{w(x)}=\frac{1}{(n+1)!}\,f^{(n+1)}(\xi). \loppu
\end{align*}

Interpolaatioperiaatteista on astelukuun $n=1$ perustuva
\index{lineaarinen interpolaatio}%
\kor{lineaarinen interpolaatio} 
yksinkertaisuutensa vuoksi hyvin yleisesti käytetty. Esimerkiksi käyrän $y=f(x)$ approksimointi
pisteiden $(x_i,f(x_i))$ kautta kulkevalla murtoviivalla (vaikkapa kuvan piirtämiseksi tai
kaarenpituuden arvioimiseksi) tarkoittaa $f$:n \kor{paloittain lineaarista}
interpolaatiota.
\begin{figure}[H]
\begin{center}
\import{kuvat/}{kuvaipol-1.pstex_t}
\end{center}
\end{figure}
Jos $f$ on kahdesti jatkuvasti derivoituva interpolointivälillä $[x_0,x_1]$, niin lineaarisen
interpolaation virhe ko.\ välillä on Lauseen \ref{Lagrangen interpolaatiovirhe} mukaan enintään
\begin{align*}
\max_{x\in [x_0,x_1]} \abs{f(x)-p(x)} 
         &\le \max_{x\in[x_0,x_1]} \left|\frac{1}{2}(x-x_0)(x-x_1)\right|
              \max_{\xi\in[x_0,x_1]}\abs{f''(\xi)} \\
         &= \frac{1}{8}\,(x_1-x_0)^2\max_{x\in [x_0,x_1]} \abs{f''(x)}.
\end{align*}
Jos interpolointi tapahtuu paloittain välillä $[a,b]$ ja interpolointipisteiden väli on
enintään $h$, niin ko.\ approksimaation maksimivirhe on siis enintään
\[
\max_{x\in [a,b]} \abs{f(x)-\tilde{f}(x)}\leq\frac{1}{8}\,h^2\max_{x\in [a,b]} \abs{f''(x)}.
\]
Tämä arvio ei ole parannettavissa, sillä jos $f''(x)=$ vakio (eli $f$ on toisen asteen
polynomi) ja interpolointipisteet ovat tasaväliset, niin virhearvion yläraja totetuu
peräkkäisten interpolointipisteiden puolivälissä.

\index{kvadraattinen!b@interpolaatio}%
Jos \kor{kvadraattisessa} (= toisen asteen) interpolaatiossa pisteet $x_0,x_1,x_2$ ovat 
tasavälein ja väli $=h$, niin
\[
\max_{x\in [x_0,x_2]} \abs{(x-x_0)(x-x_1)(x-x_2)}
     = \max_{x\in [-h,h]} \abs{x}(h^2-x^2)=\frac{2}{3\sqrt{3}}\,h^3,
\]
joten kvadraattinen interpolaatiovirhe on tasavälisten interpolointipisteiden virittämällä 
välillä enintään
\[
\max_{x\in [x_0,x_2]} \abs{f(x)-p(x)}
   \le \frac{1}{9\sqrt{3}}\,h^3\max_{x\in [x_0,x_2]} \abs{f'''(x)} \quad (h=x_1-x_0=x_2-x_1).
\]
\begin{Exa}
Välillä $(0,\infty)$ määritelty funktio $E$ toteuttaa ehdot
\[
E'(x) = -\frac{e^{-x}}{x}\,, \quad x>0, \qquad \lim_{x\kohti\infty} E(x)=0.\footnote[2]{Funktio
on nimeltään \kor{eksponentiaali-integraalifunktio}.
\index{eksponentiaali-integraalifunktio|av}}
\]
Halutaan määrittää $E$:n arvot likimäärin välillä $[1,2]$ laskemalla ensin $E$ pisteissä 
$x_i=1+(i-1)h,\ i=0\ldots n,\ h=1/n$ (oletetetaan, että tämä onnistuu hyvin tarkasti) ja 
käyttämällä interpolaatiota muissa pisteissä. Kuinka suuri on $n$:n oltava, jos käytetään \ 
a) lineaarista, \ b) kvadraattista interpolaatiota ja halutaan, että interpolaatiovirhe on 
enintään $5\cdot 10^{-9}$\,?
\end{Exa} 
\ratk Koska $E'(x) = -e^{-x}/x,\ x>0$, niin
\begin{align*}
E''(x)  &= e^{-x}(\tfrac{1}{x}+\tfrac{1}{x^2}) 
              \qimpl \abs{E''(x)}\leq 2/e,\quad x\in [1,2], \\
E'''(x) &= -e^{-x}(\tfrac{1}{x}+\tfrac{2}{x^2}+\tfrac{2}{x^3}) 
              \qimpl \abs{E'''(x)}\leq 5/e,\quad x\in [1,2].
\end{align*}
Näin ollen riittää valita $N=n+1$ siten, että pätee
\begin{align*}
&\text{a)} \quad \frac{1}{8}\cdot\frac{2}{e}\cdot\left(\frac{1}{N-1}\right)^2
                \le 5\cdot 10^{-9} \qekv \underline{\underline{N \ge 4290}}, \\
&\text{b)}\quad \frac{1}{9\sqrt{3}}\cdot\frac{5}{e}\cdot\left(\frac{1}{N-1}\right)^3
                \le 5\cdot 10^{-9} \qekv \underline{\underline{N \ge 288}}. \loppu
\end{align*}

\subsection{Lagrangen kantapolynomit}
\index{interpolaatiopolynomi!a@Lagrangen|vahv}
\index{Lagrangen!b@interpolaatio(polynomi)|vahv}

\index{polynomisovitus}%
Korkeampiasteisissa polynomi-interpolaatioissa on usein kyse \kor{polynomisovituksesta} 
pisteisiin $(x_i,f(x_i))$, jolloin arvot $f(x_i)$ voivat olla esimerkiksi mitattuja. 
Kehittyneissä numeerisen ja symbolisen analyysin ohjelmistoissa on tähän tehtävään omat 
komentonsa (esim.\ Mathematica: \verb|Fit|, Matlab: \verb|Polyfit|). Käsin laskettaessa, tai
etenkin haluttaessa tutkia interpolaatiopolynomin ominaisuuksia teoreettiselta kannalta,
voidaan käyttää esitysmuotoa
\begin{equation} \label{Lagrangen interpolaatiokaava}
p(x)=\sum_{i=0}^n f(x_i)L_i(x),
\end{equation}
\index{kantapolynomi (Lagrangen)}%
missä nk.\ (Lagrangen) \kor{kantapolynomit} $L_i(x)$ (astetta $n$) määräytyvät 
interpolaatioehdoista
\begin{equation} \label{Lagrangen kantaehdot}
L_i(x_j)=\begin{cases} 
         1, &\text{kun}\  j=i, \\ 0, &\text{kun}\ j\neq i \quad (j \in \{0, \ldots, n\}). 
         \end{cases}
\end{equation}
Helposti on tarkistettavissa, että nämä ehdot toteuttava polynomi on (vrt.\ kuvio)
\[
L_i(x) \,=\, \frac{(x-x_0)\,\cdots\,(x-x_{i-1})(x-x_{i+1})\,\cdots\,(x-x_n)}
               {\,(x_i-x_0)\,\cdots\,(x_i-x_{i-1})(x_i-x_{i+1})\,\cdots\,(x_i-x_n)}
       \,=\, \frac{\prod_{j \neq i} (x-x_j)}{\prod_{j \neq i} (x_i-x_j)}\,.
\]
\begin{figure}[H]
\begin{center}
\import{kuvat/}{kuvaipol-2.pstex_t}
\end{center}
\end{figure}
Ehdoista \eqref{Lagrangen kantaehdot} nähdään myös välittömästi, että polynomi 
\eqref{Lagrangen interpolaatiokaava} täyttää asetetut interpolaatioehdot 
\eqref{interpolaatioehdot}. Lagrangen interpolaatio-ongelman ratkeavuus on siis näin tullut 
todetuksi.
\begin{Exa} \label{interpolaatioesimerkki}
Säännöllisestä funktiosta tiedetään mittaustuloksina
\[
f_1=f(0.1)=1.4491,\quad f_2=f(0.2)=1.4832,\quad f_3=f(0.3)=1.5166.
\]
Arvioi $f(0)$.
\end{Exa}
\ratk Sovitetaan mittaustuloksiin toisen asteen Lagrangen interpolaatiopolynomi $p$ ja
arvioidaan $f(0)\approx p(0)$. Tässä on $\{x_0,x_1,x_2\}=\{0.1,0.2,0.3\}$, joten
\begin{align*}
p(0) &= 1.4491\cdot\frac{(0-0.2)\cdot (0-0.3)}{(0.1-0.2)\cdot (0.1-0.3)} \\
&+ 1.4832\cdot\frac{(0-0.1)\cdot (0-0.3)}{(0.2-0.1)\cdot (0.2-0.3)} \\
&+ 1.5166\cdot\frac{(0-0.1)\cdot (0-0.2)}{(0.3-0.1)\cdot (0.3-0.2)} \\
&= 3\cdot 1.4491-3\cdot 1.4832+1\cdot 1.5166=\underline{\underline{1.4143}}.
\end{align*}
Tässä 'mittaustulokset' on itse asiassa saatu funktiosta $f(x)=\sqrt{2+x}$, jolle $f(0)=1.4142$.
\loppu

Kantapolynomeihin perustuva esitysmuoto \eqref{Lagrangen interpolaatiokaava} on erityisen
kätevä silloin, kun halutaan arvioida funktion \pain{evaluoinnin} \pain{virheiden} 
(esim.\ mittausvirheiden) vaikutus interpolointitulokseen. Nimittäin jos virheellisten arvojen
$\tilde{f}_i$ (oikea arvo $=f_i$) perusteella lasketaan interpolaatiopolynomi $\tilde{p}$, niin
kaavan \eqref{Lagrangen interpolaatiokaava} mukaan
\[ 
p(x)-\tilde{p}(x) = \sum_{i=0}^n (f_i-\tilde{f}_i) L_i(x). 
\]
Jos erityisesti tiedetään, että $\abs{f_i-\tilde{f}_i} \le \delta,\ i=0,\ldots n$, niin
\[ 
\abs{p(x)-\tilde{p}(x)} \le \delta \sum_{i=0}^n \abs{L_i(x)} = \delta K(x). 
\]
Tässä määritelty virheen vahvistuskerroin $K(x)=\sum_{i=0}^n \abs{L_i(x)}$ siis kertoo, kuinka
paljon evaluointivirheet voivat pahimmillaan vahvistua pisteessä $x$.
\jatko \begin{Exa} (jatko) Esimerkissä on $L_1(0)=3$, $L_2(0)=-3$ ja $L_3(0)=1$. Siis
$K(0)=3+3+1=7$, eli virheiden vaikutus pisteessä $x=0$ on pahimmassa tapauksessa $7$-kertainen
yksittäisiin evaluointivirheisiin verrattuna. Jos oletetaan, että virheet ovat enintään
$10^{-4}$ itseisarvoltaan, niin pahin vaihtoehto toteutuu, kun
$f_1-\tilde{f}_1=-(f_2-\tilde{f}_2)=f_3-\tilde{f}_3=\pm 10^{-4}$. \loppu
%\[ 
%\abs{p(0) - \tilde{p}(0)} \le 10^{-4} \cdot (3+3+1) = 7 \cdot 10^{-4}. 
%\]
\end{Exa}

\subsection{Ekstrapolaatio}
\index{ekstrapolaatio|vahv}

Esimerkissä \ref{interpolaatioesimerkki} funktiota approksimoitiin interpolaatiopisteiden 
virittämän välin ulkopuolella, jolloin sanotaan että kyse on \kor{ekstrapolaatiosta}. 
Ekstrapolaatio on vanhastaan hyvin suosittu ja melko yleispätevä tapa parantaa numeeristen 
laskujen tarkkuutta. Ekstrapolaatiota voidaan käyttää aina, kun laskettavan suureen voidaan 
otaksua riippuva säännöllisellä (eli sileällä) tavalla jostakin laskentaan liittyvästä 
parametrista. Olkoon esimerkiksi laskettava suure reaaliluku $a$, joka määräytyy raja-arvona
\[
a=\lim_{h\kohti 0^+} f(h),
\]
missä jokainen $f(h)$, $h>0$, on laskettavissa, mutta laskenta tulee yhä työläämmäksi $h$:n 
pienetessä. Jos nyt voidaan olettaa, että funktio $f(x)$ on säännöllinen jollakin välillä 
$[0,b]$, $b>0$, voidaan numeerisen algoritmin antamia approksimaatioita
\[
a\approx a_n=f(x_n),\quad n=1,2,\ldots\quad (x_n\kohti 0^+)
\]
parantaa ekstrapolaatiolla. Näin syntyy nk.\ \kor{ekstrapolaatiotaulukko}, jossa laskettuihin 
tuloksiin sovitetaan yhä korkeampiasteisia polynomeja $p(x)$, ja arvioidaan kunkin polynomin
avulla $a\approx p(0)$\,:

\begin{center}
\begin{tabular}{lllll}
 & $\text{aste}=0$ & $\text{aste}=1$ & $\text{aste}=2$ & $\text{aste}=3$ \\ \hline \\
$x_1$ & $f(x_1)$ \\
$x_2$ & $f(x_2)$ & $p^{(1,2)}(0)$ \\
$x_3$ & $f(x_3)$ & $p^{(2,3)}(0)$ & $p^{(1,3)}(0)$ \\
$x_4$ & $f(x_4)$ & $p^{(3,4)}(0)$ & $p^{(2,4)}(0)$ & $p^{(1,4)}(0)$
\end{tabular}
\end{center}
Tässä $p^{(i,j)}(x)$ tarkoittaa pisteisiin $\,x_i\ldots x_j\,$ sovitettua interpolaatiopolynomia
astetta $j-i$ ($p^{(i,i)}(x)=f(x_i)=$ vakio). Osoittautuu, että taulukon sarakkeet määräytyvät
palautuvasti edellisen sarakkeen avulla. Nimittäin
\begin{equation} \label{Nevillen kaava}
\boxed{\ p^{(i,j)}(x)=\frac{(x_j-x)p^{(i,j-1)}(x)+(x-x_i)p^{(i+1,j)}(x)}{x_j-x_i}\,. \quad}
\end{equation}
Tämä palautuskaava (perustelu induktiolla: Harj.teht.\,\ref{H-dif-6: Nevillen kaava}) helpottaa
taulukon muodostamista huomattavasti.\footnote[2]{Kaavaan \eqref{Nevillen kaava} perustuvaa
ekstrapolaatiotaulukkoa sanotaan \kor{Nevillen kaavioksi}. Ennen tietokoneiden aikaa
tällaisilla (käsinlaskua helpottavilla) algoritmisilla keksinnöillä oli huomattava käytännön
merkitys. \index{Nevillen kaavio|av}} 
Seuraavassa esimerkki kaavaan \eqref{Nevillen kaava} perustuvasta 'laskemisen taiteesta'.
\begin{Exa} \label{Neville} \index{Stirlingin kaava}
\kor{Stirlingin kaavan} mukaan $\phi(n)=n!/(\sqrt{2\pi n}\,e^{-n}n^n) \approx 1$ suurilla $n$:n
arvoilla. Ekstrapoloi $\phi(100)$ arvoista $\phi(n)$, $n=5\ldots 9$, kun tiedetään lisäksi,
että $\phi(n)=f(1/n)$, missä $f(x)$ on sileä funktio välillä $[0,1]$.
\end{Exa}
\ratk Muodostetaan ekstrapolaatiotaulukko

\begin{center}
\begin{tabular}{llllll}
$x_i$  & $f(x_i)$&$\text{aste}=1$&$\text{aste}=2$&$\text{aste}=3$&\text{aste}=4 \\ \hline \\
$1/5$  & $1.0167..$ \\
$1/6$  & $1.0139..$ & $1.00076..$ \\
$1/7$  & $1.0119..$ & $1.00078..$ & $1.000823..$ \\
$1/8$  & $1.0104..$ & $1.00079..$ & $1.000827..$ & $1.000833565..$ \\
$1/9$  & $1.0092..$ & $1.00080..$ & $1.000830..$ & $1.000833632..$ & $1.000833708..$ \\
\end{tabular}
\end{center}
Oletettavasti neljännen asteen interpolaatio antaa tarkimman tuloksen, joten
$\phi(100)\approx\underline{\underline{1.00083371}}$. (Oikea arvo on on $1.000833677..\,$)
\loppu

Esimerkissä olennaista oli lisätieto, joka mahdollisti oikean muuttujan valinnan ($x_i=1/n_i$)
ekstrapoloinnissa.

\subsection{*Yleistetty polynomi-interpolaatio}
\index{interpolaatiopolynomi!b@yleistetty|vahv}

Jos $p$ on polynomi astetta $n$ ja toteuttaa ehdot
\begin{equation} \label{yleiset interpolaatioehdot} 
\begin{aligned}
p^{(k)}(x_i)&= f^{(k)}(x_i),\quad k=0\ldots\nu_i-1,\ \ i=1\ldots m, \\ 
           &\qquad\ \text{missä}\quad \nu_i \in \N \quad \text{ja} \quad \sum_{i=1}^m \nu_i=n+1,
\end{aligned} \end{equation}
niin sanotaan, että $p$ on funktion $f$ \kor{yleistetty interpolaatiopolynomi} pisteissä
$x_i$. Tässä ja jatkossa oletetaan, että ehdoissa \eqref{yleiset interpolaatioehdot}
esiintyvät $f$:n derivaatat (jos $\nu_i \ge 2$) ovat olemassa pisteissä $x_i$. Ehtojen
\eqref{yleiset interpolaatioehdot} mukaisesti interpolaatiopisteiden lukumäärä $m$ voi
yleistetyssä polynomi-interpolaatiossa astetta $n$ olla mikä tahansa välillä $1 \le m \le n+1$.
Tapauksessa $m=n+1$ on oltava $\nu_i=1\ \forall i$, jolloin kyseessä on Lagrangen
interpolaatio. Toisessa ääripäässä ($m=1$) on taas oltava $\nu_1=n+1$, jolloin $p(x)=f$:n
Taylorin polynomi $T_n(x,x_1)$.
\begin{Prop} Jos $p$ on polynomi astetta $n$, niin $p$ määräytyy ehdoista 
\eqref{yleiset interpolaatioehdot} yksikäsitteisesti. 
\end{Prop}
\tod Interpolaatio-ongelman \eqref{yleiset interpolaatioehdot} ratkeavuus voidaan todeta
samaan tapaan kuin Lagrangen interpolaation tapauksessa, ks.\
Harj.teht.\,\ref{H-dif-6: yleinen interpolaatio-ongelma}. Yksikäsitteisyyden toteamiseksi
riittää osoittaa, että jos $f^{(k)}(x_i)=0\ \forall i,k$, niin on oltava $p=0$
(vrt.\ Proposition \ref{interpolaatiopolynomin yksikäsitteisyys} todistus). Tässä tapauksessa
on $x_i$  polynomin $p$ $\,\nu_i$-kertainen nollakohta interpolaatioehtojen
\eqref{yleiset interpolaatioehdot} mukaan, joten on oltava
\[ 
p(x) = q(x)\,\prod_{i=1}^m (x-x_i)^{\nu_i} = q(x)\,w(x), 
\]
missä $q$ on polynomi. Mutta $p$ on astetta $n$, ja ehtojen \eqref{yleiset interpolaatioehdot}
perusteella $w$ on astetta $n+1$, joten ainoa mahdollisuus on $q=0$, jolloin myös $p=0$.
\loppu

Seuraava yleinen polynomiapproksimaatiotulos, jonka täydellistä todistusta ei esitetä,
sisältää erikoistapauksina sekä Lauseen \ref{Lagrangen interpolaatiovirhe} että Taylorin
lauseen \ref{Taylor}.
\begin{Lause} \label{usean pisteen Taylor} \vahv{(Usean pisteen Taylorin lause)} 
\index{Taylorin lause!b@usean pisteen|emph}
Olkoon $f$ jatkuva välillä $[a,b]$ ja $n+1$ kertaa derivoituva välillä $(a,b)$, missä $[a,b]$
on pisteen $x$ ja erillisten pisteiden $x_0,\ldots,x_m$ virittämä väli. Tällöin jos $p$ on
ehdoilla \eqref{yleiset interpolaatioehdot} määritelty yleistetty interpolaatiopolynomi
astetta $n$, niin jollakin $\xi \in (a,b)$ pätee virhekaava
\[
f(x)-p(x) = \frac{1}{(n+1)!}\,f^{(n+1)}(\xi)\,\prod_{i=1}^m (x-x_i)^{\nu_i}.
\]
\end{Lause}
\tod (idea) Jos $x \in \{x_1,\ldots, x_m\}$, on väite tosi $\forall\xi$. Olkoon siis 
$x \neq x_i\ \forall i$, ja otetaan tarkastelun kohteeksi funktio 
\[
g(t)=f(t)-p(t)- Hw(t), \quad H=[f(x)-p(x)]/w(x), \quad w(x)=\prod_{i=1}^m (x-x_i)^{\nu_i}.
\]
Koska $w(t)=t^{n+1}+(\text{polynomi astetta}\ n)$, niin $w^{(n+1)}(t)=(n+1)!\ \forall t$, joten
virhekaava väittää, että $g^{(n+1)}(\xi)=0$ jollakin $\xi \in (a,b)$. Aiemmin tämä on osoitettu
tapauksissa $m=1,\ \nu_1=n+1$  (Taylorin lause) ja $m=n+1,\ \nu_i=1$
(Lause \ref{Lagrangen interpolaatiovirhe}). Yleisemmässäkin tapauksessa on päättely vastaava.
Esimerkiksi olkoon 
\[ 
\nu_i=2,\ i=1\ldots m, \quad m \ge 2, \quad n=2m-1. 
\]
Tällöin koska $g(x_i)=0,\ i=1\ldots m$, ja $g(x)=0$, on pisteiden $x$ ja $x_i,\ i=1\ldots m$,
välisillä avoimilla väleillä kullakin $g'$:n nollakohta (yhteensä $m$ kpl). Toisaalta on myös 
$g'(x_i)=0,\ i=1\ldots m$, joten $g'$:lla on välillä $[a,b]$ ainakin $2m$ nollakohtaa. Tästä 
seuraa (vrt.\ Lauseen \ref{Lagrangen interpolaatiovirhe} todistus), että ainakin yhdessä
pisteessä $\xi \in (a,b)$ on $g^{(2m)}(\xi)=g^{(n+1)}(\xi)=0$, eli väite on tosi oletetussa
tapauksessa. (Yleinen tapaus sivuutetaan.) \loppu

Edellä tarkastellussa tapauksessa, jossa $\nu_i=2,\ i=1\ldots m$, sanotaan polynomia $p$ 
funktion $f$
\index{interpolaatiopolynomi!c@Hermiten} \index{Hermiten!b@interpolaatio(polynomi)}%
\kor{Hermiten} interpolaatiopolynomiksi pisteissä $x_i$.
\setcounter{Exa}{0}
\begin{Exa} (jatko) Montako tasavälistä jakopistettä $x_i$ tarvitaan, jos pisteiden välissä
käytetään kolmannen asteen Hermiten interpolaatiota?
\end{Exa}
\ratk Lauseen \ref{usean pisteen Taylor} mukaan interpolaatiovirhe välillä $[x_i,x_{i+1}]$ on 
enintään
\begin{align*}
\abs{E(x)-p(x)} &\leq \max_{x\in [x_i,x_{i+1}]} \frac{1}{4!} (x-x_i)^2(x-x_{i+1})^2
                      \max_{\xi\in[x_i,x_{i+1}]}\abs{E^{(4)}(\xi)} \\
                &=    \frac{1}{384}\,h^4\max_{x\in[x_i,x_{i+1}]}\abs{E^{(4)}(x)}.
\end{align*}
Välillä $[1,2]$ on $\abs{E^{(4)}(x)} \le 16/e$, joten vaadittuun tarkkuuteen riittää:
\[
\frac{1}{384}\cdot\frac{16}{e}\cdot\left(\frac{1}{N-1}\right)^4\leq 5\cdot 10^{-9} 
           \qekv \underline{\underline{N\geq 43}}. \loppu
\]

\Harj
\begin{enumerate}

\item
Funktiosta $f(x)$ tiedetään: $f(-0.1)=1.70 \pm 0.05$, $f(0.2)=1.80 \pm 0.03$ ja
$-1 \le f''(x) \le 0$ välillä $[-0.1,\,0.2]$. Määritä mahdollisimman ahdas väli, jolla $f(0)$
varmasti sijaitsee.

\item
Funktion $e^x$ arvot on laskettu viiden merkitsevän numeron tarkkuudella (normaalipyöristys)
välin $[0,2]$ pisteissä $x_i=i/100,\ i=0 \ldots 200$. Arvioi, kuinka suuri on näistä
arvoista lasketun a) lineaarisen,\ b) kvadraattisen interpolaation virhe enintään välillä 
$[0,2]$. Arvioi erikseen pyöristysvirheiden vaikutus.

\item
Eräästä välillä $(0,\infty)$ määritellystä, säännöllisestä funktiosta $F$ tiedetään, että
$F$ saavuttaa absoluuttisen minimiarvonsa välillä $[1,2]$. Lisäksi tiedetään, että
$F(0.5)=\sqrt{\pi}$, $F(1)=1$, $F(1.5)=\frac{1}{2}\sqrt{\pi}$ ja $F(2)=1$. Laske $F$:n
minimikohta ja -arvo likimäärin käyttäen a) kvadraattista interpolaatiota pisteissä
$\{0.5,1,1.5\}$, \ b) kvadraaattista interpolaatiota pisteissä $\{1,1.5,2\}$, \ c) kolmannen
asteen interpolaatiota kaikissa neljässä pisteessä. (Oletukset täyttää $\Gamma$-funktio, ks.\
Harj.teht.\,\ref{integraalin laajennuksia}:\ref{H-int-7: Gamma} ja Propositio \ref{Gamma(1/2)}.)

\item
Parametrista käyrää $\vec r=\vec u(t)=x(t)\vec i+y(t)\vec j,\ t\in[a,b]$, approksimoidaan
pisteiden $(x(t_i),y(t_i)),\ i=0 \ldots n$ kautta kulkevalla murtoviivalla
$\vec r = \vec v(t)=\hat{x}(t)\vec i+\hat{y}(t)\vec j\ (a=t_0 < t_1 < \cdots < t_n = b)$. 
Jos $x(t)$ ja $y(t)$ ovat välillä $[a,b]$ kahdesti jatkuvasti derivoituvia, 
$\abs{x''(t)} \le M$, $\abs{y''(t)} \le M$ ja $t_i-t_{i-1} \le h$, niin kuinka suuri on
enintään $\delta_h=\max_{t\in[a,b]}\abs{\vec u(t)-\vec v(t)}$\,? Vertaa arviota todellisuuteen 
tapauksessa $x(t)=R\cos t,\ y(t)=R\sin t,\ t\in[0,\pi]$.

\item %\index{zzb@\nim!Rukkaus}
Heilurilla varustettu seinäkello jätättää vuorokaudessa $5$ min $24$ s. Kelloa
rukataan kiertämällä heilurin päässä olevaa ruuvia kiinni $5$ täyttä kierrosta (jolloin
heilurin varsi hieman lyhenee). Rukkauksen jälkeen havaitaan kellon edistävän $3$ min $36$ s
vuorokaudessa. \ a) Miten kelloa kannattaa seuraavaksi rukata? \ b) Olkoon kellon suhteellinen
edistämä vuorokaudessa $f(x)$, missä $x=$ ruuvin kiertymä (kierroksina) oikeasta säätöasennosta
kiinni päin. Arvioi, montako sekuntia kello edistää tai jätättää vuorokaudessa a-kohdan
rukkauksen jälkeen, jos oletetaan, että $f''(0)=+2 \cdot 10^{-5}$.

\item
Ekstrapoloi luvuista $7! \ldots 10!$ raja-arvo $\,a=\lim_{n\kohti\infty} n!e^nn^{-n-\frac{1}{2}}$
ja vertaa tarkkaan arvoon $a=\sqrt{2\pi}$. (Vrt.\ Esimerkki \ref{Neville}.)

\item 
Todista Lause \ref{usean pisteen Taylor} tapauksessa $m=n=3,\ \nu_1=\nu_3=1,\ \nu_2=2$.

\item
Funktioiden $\sin x$ ja $\cos x$ arvot halutaan määrätä välillä $[0,\pi/4]$ siten, että
funktioiden arvot lasketaan ensin riittävän tarkasti ko.\ välin tasavälisessä pisteistössä ja
sen jälkeen käytetään kolmannen asteen Hermiten interpolaatiota pisteiden välillä. Montako
interpolointipistettä tarvitaan, jos virheen sallitaan olevan enintään $5 \cdot 10^{-9}$\,?

\pagebreak

\item \label{H-dif-6: interpolaatiot ja differessikaavat}
Halutaan laskea numeerisesti funktion $f$ derivaatta $f^{(k)}(a)$ approksimaatiolla
$f^{(k)}(a) \approx p^{(k)}(a)$, missä $p(x)$ on $f$:n interpolaatiopolynomi. Näytä, että
seuraavissa tapauksissa vaihtoehtoiset interpolaatiot johtavat samaan
differenssiapproksimaatioon --- millaiseen? (Vrt.\ edellinen luku.) \vspace{1mm}\newline
a) $k=1$: Lineaarinen interpolaatio pisteissä $a \pm h$ tai kvadraattinen interpolaatio
pisteissä $a$ ja $a \pm h$. \newline
b) $k=2$: Kvadraattinen interpolaatio pisteissä $a$ ja $a \pm h$ tai yleistetty kolmannen
asteen interpolaatio samoissa pisteissä lisäehdolla $p'(a)=f'(a)$. 

\item (*) \label{H-dif-6: Nevillen kaava}
Todista palautuskaava \eqref{Nevillen kaava} induktiolla.

\item (*) \index{zzb@\nim!Rajatieto}
(Rajatieto) Lukujonosta $\seq{a_n}$, missä $a_n = \sum_{k=1}^n k^{-5/4},\ n=1,2, \ldots\,$
tiedetään, että on olemassa (tuntemattomat) $a,b,c\in\R$ siten, että suurilla $n$:n arvoilla
pätee
\[
a_n=a+n^{-1/4}(b+c n^{-1})+\Ord{n^{-9/4}}. 
\]
Ekstrapoloi raja-arvo $\lim_n a_n=a$ tämän tiedon perusteella mahdollisimman tarkasti tuloksista
$a_{100}=3.331779$, $a_{200}=3.532117$, $a_{400}=3.700964$. (Tarkka arvo kuudella
desimaalilla: $a=4.595112$.)

\item (*)  \label{H-dif-6: yleinen interpolaatio-ongelma}
Näytä, että jos interpolaatio-ongelmassa \eqref{yleiset interpolaatioehdot} on
$f^{(k)}(x_i)=1$ kun $i=j$ ja $k=l$ $(1 \le j \le m,\ 0 \le l \le \nu_j-1)$ ja
$f^{(k)}(x_i)=0$ muulloin, niin ongelmalla on ratkaisu
\[
p(x) = L_{j,l}(x) = q(x) \prod_{\substack{i=1 \\ i \neq j}}^m (x-x_i)^{\nu_i},
\]
missä $q$ on polynomi muotoa $q(x)=\sum_{r=l}^{\nu_j-1} c_r (x-x_j)^r$. Päättele edelleen, että
interpolaatio-ongelman \eqref{yleiset interpolaatioehdot} ratkaisu yleisessä tapauksessa on
\[
p(x) = \sum_{i=1}^m \sum_{k=0}^{\nu_i-1} f^{(k)}(x_i) L_{i,k}(x).
\] 

\end{enumerate} %Interpolaatiopolynomit


\chapter{Integraali}
\label{Integraali}

Yhden reaalimuutujan analyysissä on \kor{integraalin} käsitteellä kaksi olomuotoa:
\kor{määräämätön} ja \kor{määrätty} integraali. Edellisellä tarkoitetaan
annetun funktion \kor{integraalifunktiota}, eli kyse on derivoinnin käänteisoperaatiosta,
johon on jo alustavasti tutustuttu Luvussa \ref{väliarvolause 2}. Tässä luvussa, tarkemmin
luvuissa \ref{integraalifunktio}--\ref{osamurtokehitelmät}, esitellään aiempaa
systemaattisemmin integraalifunktioiden etsimisessä käytetyt  menetelmät ja funktiotyypit,
jotka näillä menetelmillä ovat hallittavissa. Kyseessä on hyvin perinteinen matematiikan
taitolaji, jossa keskeistä roolia näyttelevät erilaiset funktioalgebran keinot, kuten
\kor{osittaisintegrointi}, muuttujan vaihto eli \kor{sijoitus} ja \kor{osamurtokehitelmät}.

Luvuissa \ref{määrätty integraali}--\ref{analyysin peruslause} tarkastellaan määrätyn
integraalin, tarkemmin \kor{Riemann\-integraalin}, eri määrittelytapoja ja ominaisuuksia
sekä liittymistä määräämättömään integraaliin \kor{Analyysin peruslauseen} kautta.
Luvussa \ref{integraalin laajennuksia} tarkastellaan Riemannin integraalin käsitteen
laajennuksia, \kor{epäoleellisia} integraaleja, ja näihin liittyen integraalien ja sarjojen
vertailua. Kahdessa viimeisessä osaluvussa tarkastelun kohteena ovat määrättyyn integraaliin
perustuvat pinta-alan ja kaarenpituuden yksinkertaisimmat laskukaavat sovelluksineen
(Luku \ref{pinta-ala ja kaarenpituus}) ja lopuksi \kor{numeerisen integroinnin} menetelmät
(Luku \ref{numeerinen integrointi}).  %Integraali
\section{Integraalifunktio} \label{integraalifunktio}
\alku
\index{integraalifunktio|vahv}

Palautettakoon mieliin Luvusta \ref{väliarvolause 2}, että funktiota $F(x)$ sanotaan
funktion $f(x)$ \kor{integraalifunktioksi} avoimella välillä $(a,b)$, jos $F$ on derivoituva
välillä $(a,b)$ ja
\[
F'(x)=f(x),\quad x\in (a,b).
\]
Integraalifunktiota merkitään
\[
F(x)=\int f(x)\, dx.
\]
Hieman erikoinen merkintä viittaa integraalifunktion laskennalliseen määritelmään, joka 
esitetään myöhemmin (Luku \ref{määrätty integraali}). Integraalifunktiota sanotaan myös
\index{mzyzyrzyzy@määräämätön integraali}%
\kor{määräämättömäksi integraaliksi} (engl.\ indefinite integral, antiderivative). Tässä
'määräämätön' viittaa siihen, että integraalifunktio ei ole yksikäsitteinen, vaan siihen 
voidaan lisätä mielivaltainen vakio: Jos yksi integraalifunktio $F$ tunnetaan, niin kaikki 
\index{integroimisvakio}%
integraalifunktiot ovat esitettävissä nk.\ \kor{integroimisvakion} $C$ avulla muodossa
\[
\int f(x)\, dx=F(x)+C.
\]
Määrämätön integraali on siis itse asiassa funktiojoukko. Kyseessä on differentiaaliyhtälön
$\,y'=f(x)\,$ yleinen ratkaisu välillä $(a,b)$ (ks.\ Korollaari 
\ref{toiseksi yksinkertaisin dy}).
\begin{Exa} Määritä funktion
\[
f(x)=\begin{cases} 
     \,1, &\text{kun}\ x \le 2, \\[2mm] 
     \,\dfrac{10}{x}-2x, &\text{kun}\ x>2
     \end{cases}
\]
integraalifunktio $\R$:ssä. \end{Exa}
\ratk Koska $f(2^-)=f(2^+)=1=f(2)$, niin $f$ on jatkuva $\R$:ssä. Tunnettujen 
derivoimissääntöjen (Luku \ref{derivaatta}) perusteella on oltava
\[
F(x) = \begin{cases} 
       \,x+C_1\,, &\text{kun}\ x\in(-\infty,2), \\ 
       \,10\ln x-x^2+C_2\,, &\text{kun}\ x\in(2,\infty),
        \end{cases}
\]
jolloin on $F'(x)=f(x)$, kun $x \neq 2$. Jotta $F$ olisi derivoituva myös pisteessä $x=2$, on
$F$:n oltava tässä pisteessä ainakin jatkuva, eli on oltava
\[
2+C_1 = 10\ln 2-4+C_2 \qimpl C_2 = C_1-10\ln 2+6.
\]
Kirjoittamalla $C_1=C$ on saatu
\[
F(x)=\begin{cases} 
     \,x+C, &\text{kun}\ x \le 2, \\ 
     \,10\ln(x/2)-x^2+6+C, &\text{kun}\ x \ge 2. 
     \end{cases}
\]            
Funktion $F$ toispuoliset derivaatat pisteessä $x=2$ ovat
\[
D_-F(2)=f(2^-), \quad D_+F(2)=f(2^+).
\]
Koska $f(2^-)=f(2^+)$, niin $F$ on derivoituva myös pisteessä $x=2$ ja $F'(2)=f(2)$.
Integraalifunktio on siis löydetty. Kuvassa $f$ ja $F$, kun $C=0$. \loppu
\begin{figure}[H]
\setlength{\unitlength}{1.5cm}
\begin{center}
\begin{picture}(10,3)(-0.5,-0.7)
\put(-0.5,0){\vector(1,0){4}} \put(3.3,-0.35){$x$}
\put(0,-1){\vector(0,1){3}} \put(0.2,1.8){$f(x)$}
\put(0,1){\line(1,0){2}}
\curve(
2.0, 1.0,
2.1, 0.5619,
2.2, 0.1455,
2.3, -0.2522,
2.4, -0.6333,
2.5, -1.0)
\put(-0.3,0.9){$1$}
\put(2,0){\line(0,-1){0.1}} \put(1.9,-0.4){$2$}
\put(4.5,0){\vector(1,0){4}} \put(8.3,-0.35){$x$}
\put(5,-1){\vector(0,1){3}} \put(5.2,1.8){$F(x)$}
\put(5,0){\line(1,1){2}}
\curve(
7.0, 2.0,
7.1, 2.0779,
7.2, 2.1131,
7.3, 2.1076,
7.4, 2.0632,
7.5, 1.9814,
7.6, 1.8636,
7.7, 1.7110,
7.8, 1.5247,
7.9, 1.3056,
8.0, 1.0546,
8.1, 0.7725,
8.2, 0.4600,
8.3, 0.1177)
\put(5,1){\line(-1,0){0.1}} \put(4.7,0.9){$1$}
\put(7,0){\line(0,-1){0.1}} \put(6.9,-0.4){$2$}
\end{picture}
\end{center}
\end{figure}

'Integrointi', ymmärrettynä integraalifunktion etsimisenä, on siis derivoinnin käänteistoimi.
Sekä yksittäisten funktioiden integroimiskaavat että integraalifunktioiden yleisemmät
laskusäännöt ovatkin derivoimissääntöjä 'nurinpäin' luettuna. 
\begin{Exa} Koska (vrt.\ Luku \ref{kompleksinen eksponenttifunktio})
\[
D\ln(x+\sqrt{x^2+1}) = \frac{1}{\sqrt{x^2+1}}, \quad x\in\R,
\]
niin saadaan (koko $\R$:ssä pätevä) integroimissääntö
\[
\int \frac{1}{\sqrt{x^2+1}}\,dx \ =\ \ln(x+\sqrt{x^2+1})+C. \loppu
\]
\end{Exa}

Integrointimenetelmistä yksinkertaisin on 'taulukkomenetelmä' eli kaavakokoelma, joka perustuu
suoraan tunnettuihin derivoimissääntöihin. Seuraavaan taulukkoon on koottu tällaisista
integroimiskaavoista keskeisimmät (ml.\ esimerkin tulos). Nämä voidaan suoraan lukea Lukujen 
\ref{derivaatta}, \ref{kaarenpituus}, \ref{exp(x) ja ln(x)} ja
\ref{kompleksinen eksponenttifunktio} derivoimissäännöistä.

\[ \boxed{\begin{alignedat}[l]{2} \quad
&(1)  \quad \int x^\alpha\,dx & \ = \ 
     & \frac{\ykehys 1}{\alpha +1}\,x^{\alpha+1}+C,\quad \alpha\in\R, \ \alpha\neq -1. \quad \\
&(2)  \quad \int \frac{1}{x}\,dx & \ = \ &\ln\abs{x}+C. \\
&(3)  \quad \int \cos x\,dx & \ = \ & \sin x+C. \\
&(4)  \quad \int \sin x\,dx & \ = \ & -\cos x+C. \\
&(5)  \quad \int e^x \, dx & \ = \ & e^x+C. \\
&(6)  \quad \int \frac{1}{\cos^2 x}\,dx & \ = \ & \tan x+C. \\
&(7)  \quad \int \frac{1}{\sin^2 x}\,dx & \ = \ & -\cot x+C. \\
&(8)  \quad \int \frac{1}{\sqrt{1-x^2}}\,dx & \ = \ & \Arcsin x+C. \\
&(9)  \quad \int \frac{1}{\sqrt{x^2+1}}\,dx & \ = & \ln(x+\sqrt{x^2+1})+C. \quad \\
&(10) \quad \int \frac{1}{\sqrt{x^2-1}}\,dx & \ = \ & \ln\abs{x+\sqrt{x^2-1}}+C. \\
&(11) \quad \int \frac{1}{1+x^2}\,dx & \ = \ & \Arctan x+C. \\
&(12) \quad \int \frac{1}{\akehys 1-x^2}\,dx  & \ = \ & 
                 \frac{1}{2}\ln\left|\frac{1+x}{1-x}\right|+C.
\end{alignedat}} \]

Taulukkoa voitaisiin helposti jatkaa tunnettujen derivoimissääntöjen perusteella. Esimerkiksi
pätee (ks.\ Luku \ref{exp(x) ja ln(x)})\,:

\[ \boxed{\begin{alignedat}[l]{2} \quad
&(13) \quad \int\dfrac{\ykehys 1}{\sin x}\,dx & \ = \ & 
                     \ln\left|\dfrac{1-\cos x}{\sin x}\right|+C. \\
&(14) \quad \int\dfrac{1}{\cos x}\,dx & \ = \ & 
                    -\ln\left|\dfrac{1-\sin x}{\akehys\cos x}\right|+C. \hspace{35mm}
\end{alignedat}} \]

Sääntöjä (1)--(14) voidaan tarkemmin käyttää sellaisilla \pain{avoimilla} \pain{väleillä},
joilla annettu funktio $f$ ja integraalifunktio $F$ ovat molemmat määriteltyjä ja jälkimmäinen
on derivoituva. Esimerkiksi säännöt (3)--(5) ovat päteviä koko $\R$:ssä, sääntö (2) on pätevä 
väleillä $(-\infty,0)$ ja $(0,\infty)$, sääntö (8) välillä $(-1,1)$, ja sääntö (7) väleillä
$(k\pi,(k+1)\pi),\ k\in\Z$. Silloin kun integroimissäännön pätevyys on rajoitettu useammalle
pistevieraalle $\R$:n osavälille (kuten sännöissä (2) ja (7)), ei integroimisvakion tarvitse
olla eri osaväleillä sama.

Taulukkokaavat (9), (10) ja (12) voidaan ilmaista myös hyperbolisten käänteisfunktioiden
$\arsinh$, $\Arcosh$ ja $\artanh$ avulla, sillä pätee
(ks.\ Luku \ref{kompleksinen eksponenttifunktio})
\begin{align*}
&\ln(x+\sqrt{x^2+1}) \,=\, \arsinh x, \quad x\in\R, \\
&\ln|x+\sqrt{x^2-1}|\, = \begin{cases}
                         \,\Arcosh x, &\text{kun}\ x>1, \\ -\Arcosh|x|, &\text{kun}\ x<-1,
                         \end{cases} \\
&\frac{1}{2}\ln\left|\frac{1+x}{1-x}\right|\, = \begin{cases}
                                                \,\artanh x,      &\text{kun}\ |x|<1, \\
                                                \,\artanh\,(1/x), &\text{kun}\ |x|>1.
                                                \end{cases}
\end{align*}  

Integraalifunktion etsiminen --- silloin kun funktio ei löydy suoraan yksinkertaisimpien
derivoimissääntöjen perusteella --- on taitolaji, jota perinteisessä matematiikassa on 
harjoiteltu varsin paljon. Lajilla on urheilullista mielenkiintoa mm.\ siksi, että monien
varsin yksinkertaisten funktioiden, kuten 
\[
\frac{1}{x^4+1}\,, \quad \frac{\sqrt{x^2+1}}{x}
\]
integraalifunktioiden määrittäminen on työlästä. On myös paljon funktioita, esim.\ 
\[
e^{x^2}, \quad
\cos x^2, \quad
\frac{e^x}{x}\,, \quad
\frac{\sin x}{x}\,, \quad
\frac{1}{\ln x}\,, \quad
\sqrt{x}\sin x, \quad
\sqrt{x^4+1}, \quad
\sqrt{1+\sin^2 x},
\]
joiden integraalifunktiot eivät ole nk.\ 
\index{alkeisfunktio}%
\kor{alkeisfunktioita}, eli ne eivät ole 
polynomien tai rationaalifunktioiden, juurilausekkeiden, eksponentti- tai 
logaritmifunktioiden tai trigonometristen funktioiden tai niiden käänteisfuntioiden avulla
ilmaistavissa olevia lausekkeita. Sanotaan tällöin, että funktio ei ole integroitavissa
\index{suljettu muoto (integroinnin)}%
\kor{suljetussa muodossa} (engl.\ in closed form). Tällaisista integraalifunktioista on ennen
puhuttu kunnioittavaan sävyyn, kutsuen niitä mm. 'korkeammiksi transkendenttifunktioiksi'.
Nykyisin laskimet ja tietokoneet, ja niiden myötä numeerisen laskennan helppous, ovat
pudottaneet tällaisetkin funktiot tavallisten kuolevaisten joukkoon.

Tietokoneiden myötä on tullut myös symbolisen laskennan mahdollisuus, mikä on vähentänyt
perinteisen käsityötaidon merkitystä integraalifunktioiden etsinnässä. Tavallisimmat 
integroimissäännöt ja -menetelmät on silti edelleen syytä tuntea, mm.\ koska tällä 
laskutekniikalla on yleisempääkin käyttöä. Tässä ja kahdessa seuraavassa luvussa käydään läpi
integroimistekniikan yleisimmät menetelmät.

\subsection{Kolme yleistä integroimissääntöä}

Yleisiä derivoimissääntöjä 'nurinpäin' lukemalla saadaan yleisiä integroimissääntöjä. 
Seuraavista kolmesta säännöstä ensimäinen kertoo, että integrointi on derivoinnin tavoin 
\pain{lineaarinen} toimitus. Toinen ja kolmas sääntö ovat yhdistetyn funktion
derivoimissääntöön suoraan perustuvia.
\begin{align*}
&\int f(x)\, dx=F(x)+C\ \ \ja\,\ \int g(x)\, dx=G(x)+C \\
&\impl \quad \left\{ \begin{array}{lr}
     \int\, [\alpha f(x)+\beta g(x)]\, dx
                =\alpha F(x) +\beta G(x) +C,\quad \alpha,\beta\in\R, &\qquad \text{(I-1)} \\
     \int f(ax+b)\, dx 
                =\dfrac{1}{a}\,F(ax+b)+C,\quad a,b\in\R, \ a\neq 0, &\qquad \text{(I-2)} \\[2mm]
     \int f(g(x))g'(x)\, dx = F(g(x))+C.     &\qquad \text{(I-3)}
                     \end{array} \right.
\end{align*}
Näissä säännöissä (kuten säännöissä (1)--(14)) on huomioitava, että integroimisvakio $C$ on nk.\
\index{geneerinen vakio}%
\kor{geneerinen} vakio, joka voi eri yhteyksissä (kuten lausekkeissa $F(x)+C$ ja $G(x)+C$) aina
saada erilaisia arvoja, ellei toisin sovita.
\begin{Exa} \label{E10.1.2}
$\D\int\frac{1}{x^2+x}\, dx=\,?$
\end{Exa}
\ratk Hajotetaan integroitava funktio ensin kahden yksinkertaisemman funktion summaksi:
\begin{align*}
&\frac{1}{x^2+x} \,=\, \frac{1}{x(x+1)} \,=\, \frac{A}{x}+\frac{B}{x+1}
                                         \,=\, \frac{(A+B)x+A}{x(x+1)} \\ 
&\qquad\ \qimpl \begin{cases} \,A+B=0 \\ \,A=1 \end{cases}
         \ekv\quad \begin{cases} \,A=1 \\ \,B=-1  \end{cases} 
         \impl\quad \frac{1}{x^2+x} \,=\, \frac{1}{x}-\frac{1}{x+1} \\[2mm]
&\qquad\ \qimpl \int \frac{1}{x^2+x}\,dx\ 
                \overset{\text{(I-1)}}{=}\ \int \frac{1}{x}\, dx - \int\frac{1}{x+1}\, dx \\
&\hspace{48mm}\ \overset{(2),\,\text{(I-2)}}{=}\ \ln\abs{x}-\ln\abs{x+1}+C\
                    =\ \underline{\underline{\ln\left|\frac{x}{x+1}\right|+C}}. \loppu
\end{align*}
\begin{Exa} Johda taulukkokaava (12). \end{Exa}
\ratk Edellistä esimerkkiä mukaillen lasketaan
\begin{align*}
&\frac{1}{1-x^2} \,=\, \frac{1}{(1+x)(1-x)}
                \,=\, \frac{1}{2}\,\frac{1}{1+x}+\frac{1}{2}\,\frac{1}{1-x} \\
&\ \qimpl \int\frac{1}{1-x^2}\,dx \,=\, \frac{1}{2}\ln|1+x|-\frac{1}{2}\ln|1-x|+C
                                  \,=\, \frac{1}{2}\ln\left|\frac{1+x}{1-x}\right|+C. \loppu
\end{align*}

Esimerkeissä integraalifunktio löydettiin nk.
\index{osamurtokehitelmä}%
\kor{osamurtokehitelmän} avulla. Tämä on
yleisempiinkin rationaalifunktiohin soveltuva menetelmä, kuten nähdään jäljempänä
Luvussa \ref{osamurtokehitelmät}.
\begin{Exa} Säännön (I-3), taulukkokaavan (1) ja derivoimissääntöjen $\dif\ln|x|=1/x$ ja
$\dif\sin x=\cos x$ perusteella
\begin{align*}
&\int \frac{1}{x}\,\ln\abs{x}\,dx = \frac{1}{2}\,(\ln\abs{x})^2 + C, \\
&\int \sin^4 x \cos x\, dx = \frac{1}{5}\,\sin^5 x + C. \loppu
\end{align*}
\end{Exa}
\begin{Exa}
$\D\int\frac{1}{x^2+x+1}\, dx=\,?$
\end{Exa}
\ratk Koska
\[
\frac{1}{x^2+x+1}=\frac{1}{(x+\frac{1}{2})^2+\frac{3}{4}}
                 =\frac{4}{3}\,\frac{1}{(\frac{2x+1}{\sqrt{3}})^2+1}\,,
\]
niin
\begin{align*}
\int\frac{1}{x^2+x+1}\,dx\ 
&\,\ \ \overset{\text{(I-1)}}{=}\ \ 
          \frac{4}{3}\int \frac{1}{(\frac{2x+1}{\sqrt{3}})^2+1}\,dx \\
&\overset{\text{(I-2)},(11)}{=}\ 
          \frac{4}{3}\cdot\frac{\sqrt{3}}{2}\Arctan\left(\frac{2x+1}{\sqrt{3}}\right)+C \\
&\quad\, =\ \ 
 \underline{\underline{\frac{2}{\sqrt{3}}\Arctan\left(\frac{2x+1}{\sqrt{3}}\right)+C}}. \loppu
\end{align*}

\begin{Exa} \label{cos-integraaleja}
$\text{a)}\ \D\int\cos^2 x\, dx=\,? \quad\text{b)}\ \int\cos^3 x\, dx=\,?$
\end{Exa}
\ratk Koska
\[
\cos^2 x=\frac{1}{2}+\frac{1}{2}\cos 2x, \quad \cos^3 x=(1-\sin^2 x)\cos x,
\]
niin
\begin{align*}
&\text{a)}\ \int \cos^2 x\, dx\,=\,\frac{1}{2}\,x+\frac{1}{4}\sin 2x+C\,
                     =\, \underline{\underline{\frac{1}{2}\,(x+\cos x\sin x)+C}}, \\
&\text{b)}\ \int \cos^3 x\, dx\,=\,\int\left(\cos x-\sin^2 x\cos x\right)\,dx\,
                     =\, \underline{\underline{\sin x-\frac{1}{3}\sin^3 x+C}}. \loppu
\end{align*}

\Harj
\begin{enumerate}

\item 
Määritä se funktion $2 x - 3$ integraalifunktio, jonka kuvaaja sivuaa suoraa $x+y = 0$.

\item 
Ratkaise seuraavat alkuarvotehtävät välillä $(-\infty,\infty)$\,: \newline
a) \ $F'(x)=2x^2-\abs{x},\ F(0) = 1$. \newline
b) \ $F'(x)=\abs{x^2+4x},\ F(-4)=0$. \newline
c) \ $F'(x)=\max\{x,\,8x-x^2\},\ F(4)=-1$. \newline
d) \ $F'(x)=\min\{4-4x+x^2,\,40+2x-x^2\},\ F(0)=180$.

\item 
Määritä seuraavien funktioiden integraalifunktiot taulukkokaavoihin (1)--(12) ja sääntöihin
(I-1)--(I-3) vedoten. Muunna funktio tarvittaessa ensin integroinnin kannalta
soveliaampaan muotoon.
\begin{align*}
&\text{a)}\ \ (2x^2 - 5 x + 7)(4x-4) \qquad 
 \text{b)}\ \ \frac{1}{\sqrt{2x + 3}} \qquad
 \text{c)}\ \ \frac{3}{3-\pi x} \qquad
 \text{d)}\ \ \frac{x^2}{4x^3-1} \\
&\text{e)}\ \ \frac{2x^3+x}{(x^4+x^2+1)^3} \qquad
 \text{f)}\ \ \frac{x}{(x^2+1)\sqrt{x^2+1}} \qquad
 \text{g)}\ \ \frac{1}{x\ln\abs{x}} \qquad 
 \text{h)}\ \ \frac{e^x}{e^x+1} \\
&\text{i)}\ \ \frac{1}{x(\ln\abs{x})^2} \qquad
 \text{j)}\ \ \frac{1}{\sqrt{x^2-3}} \qquad
 \text{k)}\ \ \frac{1}{\sqrt{x^2+3}} \qquad 
 \text{l)}\ \ \frac{x+2}{\sqrt{x^2-1}} \\
&\text{m)}\ \ \frac{3x-2}{\sqrt{2x^2+5}} \qquad
 \text{n)}\ \ \frac{1}{x^2+14x+50} \qquad
 \text{o)}\ \ \frac{1}{x^2+14x+48} \\
&\text{p)}\ \ \frac{1}{x^2+2x+10} \qquad
 \text{q)}\ \ \frac{1}{x^2+3x-10} \qquad
 \text{r)}\ \ \frac{2x+1}{x^2+3x-10} \\
&\text{s)}\ \ \sin 2x\,\cos^4 x \qquad
 \text{t)}\ \ \frac{\sin x}{\cos^3 x} \qquad
 \text{u)}\ \ \tan x \qquad
 \text{v)}\ \ \tan^2 x \qquad
 \text{x)}\ \ \tan^3 x \\
&\text{y)}\ \ \frac{e^{\tan x}}{\cos^2 x} \qquad
 \text{z)}\ \ \sin^5 x \qquad
 \text{å)}\ \ \cosh^2 x \qquad 
 \text{ä)}\ \ \tanh x \quad\ \
 \text{ö)}\ \ \frac{\arsinh x}{\sqrt{x^2+1}}
\end{align*}

\item (*)
Määritä $a\in\R$ ja $\R$:ssä jaksollinen funktio $u$ siten, että $y(x)=ax+u(x)$ on 
alkuarvotehtävän $\,y'=\abs{\sin x},\ y(0)=0\,$ ratkaisu.

\item (*)
Olkoon $f(0)=0$ ja $f$:n määritelmä muualla kuin origossa
\[
\text{a)}\ \ f(x)=\frac{x}{\sqrt{\abs{x}}}\,, \qquad
\text{b)}\ \ f(x)=\frac{x}{\abs{x}}\,, \qquad
\text{c)}\ \ f(x)=\frac{1}{x}\left(\sin\frac{1}{x^2}+x^2\cos\frac{1}{x^2}\right).
\]
Konstruoi $f$:n integraalifunktio välillä $(-1,1)$ tai näytä, että ko.\ välillä $f$:llä
ei ole integraalifunktiota.

\end{enumerate} %Integraalifunktio
\section{Osittaisintegrointi ja sijoitus} \label{osittaisintegrointi}
\alku
\index{osittaisintegrointi|vahv}

Integroimistekniikan, eli integraalifunktioiden etsimisen, kaksi keskeisintä yleistä metodia
ovat \kor{osittaisintegrointi} (engl. integration by parts, partial integration) ja 
\kor{sijoitus}(menettely) (engl.\ substitution). Näillä menetelmillä on 'matematiikan kaavoina'
yleisempääkin käyttöä.

Osittaisintegroinnin kaava on yksinkertaisesti tulon derivoimissääntö toisin kirjoitettuna:
\[
\frac{d}{dx}(fg)=f'g+fg' \qekv f'g = \frac{d}{dx}(fg)-fg'
\]
\[
\impl \quad \boxed{\quad \int f'(x)g(x)\, dx=f(x)g(x)-\int f(x)g'(x)\, dx. \quad}
\]
\begin{Exa}\ a) \ $\D\int xe^x\, dx=\,? \quad$ b) \ $\D\int e^x\sin x\, dx=\,?$
\end{Exa}
\ratk
\begin{align*}
\text{a)} \quad   \int xe^x\, dx\, &=\,\int\underbrace{e^x}_{f'(x)}\underbrace{x}_{g(x)}\, dx\,
                                    =\,e^x\cdot x-\int e^x\, dx
                                    =\,\underline{\underline{(x-1)e^x+C}}. \\
     \text{\underline{Tarkistus}:} &\quad \frac{d}{dx}((x-1)e^x)
                                              =e^x+(x-1)e^x=xe^x. \quad\text{OK!} \loppu
\intertext{$\qquad\ $b) \ Valitaan $f'(x)=f(x)=e^x$ ja integroidaan kahdesti osittain:}
F(x)        &= \int e^x\sin x\, dx=e^x\sin x-\int e^x\cos x\, dx \\
            &=e^x\sin x-e^x\cos x-\int e^x\sin x\, dx \\[2mm]
            &=e^x(\sin x -\cos x)-F(x) \\[2mm]
\impl \ F(x)&=\underline{\underline{\frac{1}{2}\,e^x(\sin x-\cos x)+C}}. \loppu
\end{align*}
\begin{Exa} \label{paha integraali} Määritä integraalifunktio 
$\displaystyle{\int \sqrt{a^2+x^2}\,dx}$, \ kun $a>0$. 
\end{Exa}
\ratk Osittain integroimalla \,($f'(x)=1,\ g(x)=\sqrt{a^2+x^2}$\,)\, saadaan ensin
\[
\int \sqrt{a^2+x^2}\,dx = x\sqrt{a^2+x^2} - \int \frac{x^2}{\sqrt{a^2+x^2}}\,dx.
\]
Tässä on
\begin{align*}
-\int\frac{x^2}{\sqrt{a^2+x^2}}\,dx\ 
               &=\ -\int\frac{(x^2+a^2)-a^2}{\sqrt{a^2+x^2}}\,dx \\
               &=\ -\int\sqrt{a^2+x^2}\,dx + \int\frac{a^2}{\sqrt{a^2+x^2}}\,dx,
\end{align*}
joten seuraa (ks.\ edellisen luvun kaavat (9) ja (I-2))
\begin{align*}
2\int \sqrt{a^2+x^2}\,dx\ &=\ x\sqrt{a^2+x^2} + \int \frac{a}{\sqrt{(x/a)^2+1}}\,dx \\
                          &=\ x\sqrt{a^2+x^2} + a^2\ln\bigl[(x/a)+\sqrt{(x/a)^2+1}\,\bigr]+C.
\end{align*}
Kun tässä kirjoitettaan vakion $C$ tilalle $a^2\ln a+C$ (mahdollista, koska $C$ on joka 
tapauksessa määräämätön), saadaan tulos muotoon
\[
\int \sqrt{a^2+x^2}\,dx\ 
  =\ \underline{\underline{\frac{1}{2}\bigl[x\sqrt{a^2+x^2} 
                + a^2\ln\bigl(x + \sqrt{a^2+x^2}\bigr)\bigr]+C}}. \loppu
\]

\subsection{Reduktiokaavat}
\index{reduktiokaavat (integraalien)|vahv}

Osittain integroimalla voidaan johtaa \kor{reduktiokaavoja} (palautuskaavoja) koko joukolle
integraaleja, joissa on kokonaislukuparametri. Esimerkiksi seuraavat integraalit saadaan tällä
tavoin lasketuksi suljetussa muodossa:
\begin{align*}
&\int x^ne^x\,dx,\quad \int x^n\cos x\,dx,\quad x^n\sin x\,dx,\quad
 \int (1+x^2)^{-n}\, dx,\quad n\in\N, \\
&\int \cos^n x\, dx,\quad \int\sin^n x\, dx,\quad n\in\Z. \\
\end{align*}
Tarkastellaan esimerkkinä integraalia 
\[
I_n(x)=\int \cos^n x\,dx.
\]
Osittain integroimalla saadaan
\begin{align*}
I_n(x) &= \int \cos x\cdot \cos^{n-1} x\, dx \qquad[\,f'(x)=\cos x,\ g(x)=\cos^{n-1}x\,] \\
&=\sin x\cos^{n-1} x + (n-1)\int\sin^2 x\cos^{n-2} x\, dx \\
&=\sin x\cos^{n-1} x + (n-1)\int(1-\cos^2 x)\cos^{n-2}x\,dx \\[1mm]
&=\sin x\cos^{n-1} x + (n-1)I_{n-2}(x)-(n-1)I_n(x).
\end{align*}
Tämän perusteella integraalille $I_n(x)$ pätee reduktiokaava
\begin{equation} \label{reduktio 1}
nI_n(x)-(n-1)I_{n-2}(x)=\sin x\cos^{n-1} x.
\end{equation}
Kaava pätee itse asiassa kun $n\hookrightarrow\alpha$, $\alpha\in\R$, mutta siitä on hyötyä
lähinnä kun $n\in\Z$, jolloin $I_n$ voidaan palauttaa kaavan avulla tapauksiin $n=0,\pm 1$ ja
näin ollen integroida suljetussa muodossa (tapaukseen $n=-1$ soveltuu edellisen luvun 
taulukkokaava (14)).
\begin{Exa}
$\text{a)}\ \D\int\cos^3 x\, dx=\,? \quad\text{b)}\ \int\cos^{-3} x\, dx=\,?$
\end{Exa}
\ratk Indekseillä $n=3$ ja $n=-1$ reduktioaava \eqref{reduktio 1} antaa
\begin{align*}
I_3(x)    &= \frac{1}{3}\sin x\cos^2 x + \frac{2}{3}I_1(x), \\
I_{-3}(x) &= \frac{1}{2}\sin x\cos^{-2} x + \frac{1}{2}I_{-1}(x),
\end{align*}
joten (vrt.\ edellisen luvun Esimerkki \ref{cos-integraaleja})
\begin{align*}
\text{a)}\,\ \int \cos^3 x\, dx 
        &= \frac{1}{3}\sin x\cos^2 x + \frac{2}{3}\int\cos x\, dx \\
        &= \underline{\underline{-\frac{1}{3}\sin^3 x + \sin x+C}}
\intertext{ja edellisen luvun kaavan (14) perusteella}
\text{b)}\,\ \int \cos^{-3} x\, dx 
        &= \frac{1}{2}\sin x\cos^{-2} x + \frac{1}{2}\int\cos^{-1} x\,dx \\
        &=\underline{\underline{\frac{1}{2}\ln\left|
              \frac{\cos x}{1-\sin x}\right|+ \frac{1}{2}\,\frac{\sin x}{\cos^2 x}+C}}. \loppu
\end{align*}

Toisena esimerkkinä tarkasteltakoon integraalia
\[
I_n(x)=\int\frac{1}{(1+x^2)^n}\,dx,\quad n \ge 2.
\]
Pyritään palauttamaan tämä tapaukseen $n=1$, jossa integraalifunktio tunnetaan (edellisen
luvun taulukkokaava (11)). Valitaan $f'(x)=1,\ g(x)=(1+x^2)^{-n}$ ja integroidaan osittain:
\begin{align*}
I_n(x) &= \frac{x}{(1+x^2)^n}+2n\int\frac{x^2}{(1+x^2)^{n+1}}\, dx \quad [x^2=(x^2+1)-1] \\
&= \frac{x}{(1+x^2)^n}+2n I_n(x) - 2n I_{n+1}(x).
\end{align*}
Kirjoittamalla $n$:n tilalle $n-1$ päädytään reduktiokaavaan
\begin{equation} \label{reduktio 2}
I_n(x)=\frac{1}{2n-2}\,\frac{x}{(1+x^2)^{n-1}}+\frac{2n-3}{2n-2}\,I_{n-1}(x).
\end{equation}
\begin{Exa}
Kaavaa \eqref{reduktio 2} kahdesti soveltamalla saadaan
\begin{align*}
\int\frac{1}{(1+x^2)^3}\,dx 
&= \frac{1}{4}\frac{x}{(1+x^2)^2}+\frac{3}{4}\int\frac{1}{(1+x^2)^2}\,dx \\
&= \frac{1}{4}\,\frac{x}{(1+x^2)^2}+\frac{3}{4}
        \left(\frac{1}{2}\,\frac{x}{1+x^2}+\frac{1}{2}\int \frac{1}{1+x^2}\,dx\right) \\
&= \underline{\underline{\frac{1}{4}\,
         \frac{x}{(1+x^2)^2}+\frac{3}{8}\,\frac{x}{1+x^2}+\frac{3}{8}\Arctan x+C}}. \loppu
\end{align*}
\end{Exa}

\subsection{Sijoitusmenettely}
\index{muuttujan vaihto (sijoitus)!b@integraalissa|vahv}

Sijoituksella tarkoitetaan integroinnissa \pain{muuttu}j\pain{an} \pain{vaihtoa}
(kuten raja-arvoja laskettaessa, vrt.\ Luku \ref{funktion raja-arvo}). Integroinnin 
sijoitusmenettely perustuu seuraavaan tulokseen, joka puolestaan perustuu yhdistetyn funktion
ja käänteisfunktion derivoimissääntöihin.
\begin{Prop} \label{sijoituspropositio}
Olkoon $u:(c,d)\Kohti (a,b)$ bijektio ja olkoon $u$ derivoituva välillä $(c,d)$ ja
käänteisfunktio $v=\inv{u}$ derivoituva välillä $(a,b)$. Tällöin, jos $G(t)$ on funktion
$g(t)=f(u(t))u'(t)$ integraalifunktio välillä $(c,d)$, ts.\
\[
\int f(u(t))u'(t)\, dt=G(t)+C,\quad t\in (c,d),
\]
niin funktion $f(x)$ integraalifunktio välillä $(a,b)$ on
\[
\int f(x)\, dx=G(v(x))+C,\quad x\in (a,b).
\]
\end{Prop}
\tod Jos $x\in (a,b)$ ja $t=v(x) \ \ekv \ x=u(t)$, niin yhdistetyn funktion derivoimissäännön
ja oletuksien perusteella
\begin{align*}
\frac{d}{dx}G(v(x)) &= G'(v(x))v'(x)=G'(t)v'(x) \\
&= f(u(t))u'(t)v'(x).
\end{align*}
Tässä on $u(t)=x$, joten $f(u(t))=f(x)$ ja käänteisfunktion derivoimissäännön
(Luku \ref{derivaatta}, kaava \eqref{D5}) mukaan
\[ 
u'(t)v'(x)=1 \qimpl \text{väite}. \loppu 
\]

Sovelluksissa sijoitus tehdään usein (ehkä useammin) muodossa $v(x)=t$. Mekaanisesti
sijoitusmenettely toimii siis seuraavalla tavalla:

\kor{Ongelma:} $\quad \D\int f(x)\,dx=\,?$
\begin{enumerate}
\item Tehdään sijoitus: $\quad x=u(t) \quad \text{tai} \quad v(x)=t$.
\item Ratkaistaan
      \[
      x=u(t)\ \impl\ t=v(x) \quad \text{tai} \quad v(x)=t \ \impl \ x=u(t).
      \]
\item Sijoitetaan integraaliin
      \[
      x=u(t)\,\ (\text{tai}\ v(x)=t) \quad \text{ja} \quad dx=u'(t)\,dt.
      \]
\item Kirjoitetaan $f(u(t))u'(t)=g(t)$ (mahdollinen sievennys) ja ratkaistaan muunnettu
      ongelma:
      \[
      \int g(t)\,dt = G(t)+C.
      \]
\item Sijoitetaan ratkaisuun $t=v(x)$, jolloin saadaan alkuperäisen ongelman ratkaisu:
      \[
      \int f(x)\, dx=G(v(x))+C.
      \]
\end{enumerate}
Vaiheessa 2 ei haittaa, vaikka käänteisfunktio $v=\inv{u}$ tai $u=\inv{v}$ olisi monihaarainen
(yhtälön $u(t)=x$ tai $v(x)=t$ ratkaisu monikäsitteinen), kunhan Proposition
\ref{sijoituspropositio} oletukset toteutuvat valitulla haaralla. On ainoastaan pidettävä
huolta, ettei laskun eri vaiheissa hypellä haaralta toiselle.
\begin{Exa} $\displaystyle{\int \frac{1}{\sqrt{x}+1}\, dx=\,?}$ 
\end{Exa}
\ratk Tässä on $(a,b) = (0,\infty)$.
\begin{enumerate}
\item Tehdään sijoitus: $\quad v(x) = \sqrt{x} = t \in (0,\infty)$.
\item Ratkaistaan: $\quad \sqrt{x}=t\ \impl\ x=u(t)=t^2$.
\item Sijoitetaan integraaliin: $\quad \sqrt{x}=t,\ dx=2t\,dt$.
\item Ratkaistaan muunnettu ongelma:
\[
\int \frac{2t}{t+1}\,dt \,=\, \int \left(2-\frac{2}{t+1}\right)dt \,=\, 2t - 2\ln(t+1) + C.
\]
\item Alkuperäisen ongelman ratkaisu sijoituksella $t=\sqrt{x}\,$:
\[
\int \frac{1}{\sqrt{x}+1}\,dx \,=\, \underline{\underline{2\sqrt{x} - 2\ln(\sqrt{x}+1) + C}}.
\]
\end{enumerate}
\begin{Exa} \label{rat-palautuva integraali 1}
$\D\int\frac{1}{e^x+1}\, dx=\,?$
\end{Exa}
\ratk Sijoittamalla $\ \D{e^x=t\in(0,\infty)\ \impl\ x=\ln t,\ dx=\frac{1}{t}\,dt}\,$ saadaan
\begin{align*}
\int\frac{1}{e^x+1}\,dx\,
&=\,\int\frac{1}{(t+1)t}\,dt\,
 =\,\ln\left|\frac{t}{t+1}\right|+C\quad 
               (\text{Esimerkki \ \ref{integraalifunktio}:\,\ref{E10.1.2}}) \\
&\impl \ \int\frac{1}{e^x+1}\,dx\,
 =\,\ln\left(\frac{e^x}{e^x+1}\right)+C\,
 =\,\underline{\underline{x-\ln(e^x+1)+C}}. \loppu
\end{align*}

\begin{Exa}
$\D\int\frac{1}{(1+x^2)^2}\,dx=\,?$
\end{Exa}
\ratk Tämän voi laskea reduktiokaavalla \eqref{reduktio 2}. Toinen vaihtoehto on
trigonometrinen sijoitus
\[
x=\tan t, \quad dx=\,\frac{1}{\cos^2 t}\,dt, \quad t\in (-\pi/2,\pi/2).
\]
Koska $\,1+\tan^2 t=1/\cos^2 t$, saadaan muunnetuksi integraaliksi
\begin{align*}
\int\frac{1}{(1+\tan^2 t)^2}\cdot\frac{1}{\cos^2 t}\,dt
       &= \int\cos^2 t\, dt = \frac{1}{2}(t+\sin t\cos t)+C \\
       &= \frac{1}{2}\left(t+\frac{\sin t}{\cos t}\cdot\cos^2 t\right)+C 
        = \frac{1}{2}\left(t+\frac{\tan t}{1+\tan^2 t}\right)+C.
\end{align*}
Ratkaisemalla lopuksi
\[
x=\tan t \ \ja \ t\in (-\pi/2,\pi/2) \qekv t=\Arctan x
\]
saadaan tulos
\[
\int\frac{1}{(1+x^2)^2}\, dx
        =\underline{\underline{\frac{1}{2}\left(\Arctan x + \frac{x}{1+x^2}\right) + C}}.
\]
\underline{Vertailu}: Reduktiokaava \eqref{reduktio 2} on menetelmänä suoraviivaisempi. \loppu
\begin{Exa} \label{rat-palautuva integraali 2} $\D\int\frac{1}{x+1+\sqrt[3]{x+1}}\, dx=\,?$ 
\end{Exa}
\ratk Sijoituksella
\begin{align*}
\sqrt[3]{x+1}=t \ \ekv \ x=t^3-1,\quad dx 
                     &= 3t^2\, dt\quad (x>-1,\ t>0)
\intertext{saadaan}
\int\frac{1}{x+1+\sqrt[3]{x+1}}\, dx = \int\frac{3t^2}{t^3+t}\,dt 
                     &=\int\frac{3t}{t^2+1}\, dt \\
                     &=\frac{3}{2}\ln (t^2+1)+C \\
                     &=\underline{\underline{\frac{3}{2}\ln[(x+1)^{2/3}+1]+C}}.
\end{align*}
Jos välillä $x\in(-\infty,-1)$ tulkitaan $\sqrt[3]{x+1}=-\sqrt[3]{\abs{x+1}}$, niin tulos on
pätevä myös tällä välillä. \loppu

\begin{Exa}
$\D\int \frac{e^x}{\sqrt{x}}\,dx\ =\ ?$ 
\end{Exa}
\ratk Sijoituksella $\ \D{\sqrt{x} = t,\ x = t^2,\ dx = 2t\,dt}\ $ tehtävä muuntuu muotoon
\[ 
\int 2e^{t^2}\,dt\ =\ ? 
\]
Tällä ei ole alkeisfunktioratkaisua (ainoastaan sarjaratkaisu, ks. seuraava luku), joten 
tyydytään tulokseen
\[ 
\int \frac{e^x}{\sqrt{x}}\,dx = 2G(\sqrt{x}), \quad G(t) = \int e^{t^2}\,dt. \loppu 
\]

\Harj
\begin{enumerate}

\item
Laske osittain integroimalla
\begin{align*}
&\text{a)}\ \ \int x\cos x\,dx \qquad
 \text{b)}\ \ \int x^3 e^{-x^2}\,dx \qquad
 \text{c)}\ \ \int x^3\cos(x^2)\,dx \\
&\text{d)}\ \ \int x^2 e^{ax}\,dx,\,\ a\in\R \qquad
 \text{e)}\ \ \int x^\alpha\ln x\,dx,\,\ \alpha\in\R \\
&\text{f)}\ \ \int \Arctan x\,dx \qquad
 \text{g)}\ \ \int \Arcsin x\,dx \qquad
 \text{h)}\ \ \int x\Arctan x\,dx
\end{align*}

\item
Johda osittain integroimalla reduktiokaava annetulle integraalille $I_n(x)$ ja laske kaavan
avulla $I_n(x)$ annetuilla $n$:n arvoilla. \vspace{1mm}\newline
a) \ $\int x^n e^x\,dx,\ n\in\N\cup\{0\},\ \ n=1,2,3$. \newline
b) \ $\int x^n \sin x\,dx,\ n\in\N\cup\{0\},\ \ n=1,2,3$. \newline
c) \ $\int x^n \cos x\,dx,\ n\in\N\cup\{0\},\ \ n=1,2,3$. \newline
d) \ $\int \sin^n x\,dx,\ n\in\Z,\ \ n=-3,4$. \newline
e) \ $\int (\ln x)^n\,dx,\ n\in\N\cup\{0\},\ \ n=1,2$. \newline
f) \ $\int (x^2+1)^{n-1/2}\,dx,\ n\in\Z,\ \ n=-1,2$. \newline
g) \ $\int (x^2-1)^{n-1/2}\,dx,\ n\in\Z,\ \ n=-1,2$. \newline
h) \ $\int (1-x^2)^{n-1/2}\,dx,\ n\in\Z,\ \ n=-1,2$.

\item
Laske annettua sijoitusta käyttäen:
\begin{align*}
&\text{a)}\ \ \int \frac{1}{\sqrt{5-x^2}}\,dx,\ \ x=\sqrt{5}\,t \qquad
 \text{b)}\ \ \int \frac{1}{x^2+2x+3}\,dx,\ \ x+1=\sqrt{2}\,t \\
&\text{c)}\ \ \int e^{\sqrt{x}}\,dx,\ \ x=t^2 \qquad\qquad\quad\,\
 \text{d)}\ \ \int e^{\sqrt[3]{x}}\,dx,\ \ \sqrt[3]{x}=t
\end{align*}

\item
Muunna seuraavat integraalit toiseen muotoon annetulla sijoituksella.
\begin{align*}
&\text{a)}\ \ \int \frac{1}{\sqrt{1+x^4}}\,dx,\ \ x^4=t \qquad
 \text{b)}\ \ \int e^{x^2}\,dx,\ \ x^2=t \\
&\text{c)}\ \ \int \sin(\ln x)\,dx,\ \ x=e^t \qquad\,
 \text{d)}\ \ \int \ln(\tan x)\,dx,\ \ \tan x=t
\end{align*}

\item
Funktioiden $x^\alpha\sin\beta x,\ x^\alpha\cos\beta x,\ x^\alpha e^{\beta x}$
($\alpha,\beta\in\R$) integraalifunktiot ovat alkeisfunktioita vain kun joko $\beta=0$ tai 
$\alpha\in\N\cup\{0\}$. Mitkä seuraavista ovat tämän tiedon perusteella alkeisfunktioita ja 
mitkä eivät?
\begin{align*}
&\text{a)}\ \ \int \cos(x^2)\,dx \qquad
 \text{b)}\ \ \int x^{13}\sqrt{x}\cos\sqrt{x}\,dx \qquad
 \text{c)}\ \ \int x^5 e^{x^4}\,dx \\
&\text{d)}\ \ \int x^7 e^{x^4}\,dx \qquad\,\ \
 \text{e)}\ \ \int \sqrt{\ln x}\,dx \qquad\
 \text{f)}\ \ \int \frac{1}{\ln x}\,dx \\
&\text{g)}\ \ \int \frac{1}{x\sqrt{\ln x}}\,dx \quad\,\ \ 
 \text{h)}\ \ \int \sin(e^x)\,dx \qquad
 \text{i)}\ \ \int \cos x\,\ln x\,dx
\end{align*}

\item (*)
Olkoon $n\in\N\cup\{0\}$ ja $a,b\in\R$. Johda reduktiokaavat integraaleille
\[
I_n(x)=\int x^n e^{ax}\cos bx, \quad J_n(x)=\int x^n e^{ax}\sin bx.
\]

\end{enumerate} %Osittaisintegrointi ja sijoitus
\section{Osamurtokehitelmät. Sarjamenetelmä} \label{osamurtokehitelmät}
\alku
\index{osamurtokehitelmä|vahv}

Yleisen rationaalifunktion integraalifunktio on löydettävissä käyttäen nk. 
\kor{osamurtokehitelmää} yhdessä edellisen luvun tekniikoiden kanssa. Tarkastellaan yleistä
(reaalikertoimista) rationaalifunktiota
\[
f(x)=\frac{p(x)}{q(x)}=\frac{a_mx^m+\cdots +a_0}{b_nx^n+\cdots b_0},
\]
missä $m\geq 0$, $n\geq 1$ ja $a_m,b_n\neq 0$. Jos $m\geq n$, suoritetaan ensiksi jakolasku:
\[
\frac{p(x)}{q(x)}=p_0(x)+\frac{p_1(x)}{q(x)}\,,
\]
missä $p_0$ on polynomi astetta $m-n$,
\[
p_0(x)=c_{m-n}x^{m-n}+\cdots +c_0,
\]
ja jakojäännöksen $p_1(x)$ aste on $n-1$:
\[
p_1(x)=d_{n-1}x^{n-1}+\cdots +d_0.
\]
Jakolasku on suoritettavissa esimerkiksi yksinkertaisesti identifioimalla polynomien kertoimet
yhtälössä
\[
p(x)=p_0(x)q(x)+p_1(x),
\]
jolloin tuntemattomille kertoimille $c_{m-n},\ldots,c_0$ ja $d_{n-1},\ldots,d_0$ saadaan
(ratkeava) lineaarinen yhtälöryhmä. Seuraavassa esimerkissä käytetään vaihtoehtoista
\index{jakoalgoritmi (polynomien)} \index{polynomien jakoalgoritmi}%
(myös yleispätevää) menetelmää, \kor{polynomien jakoalgoritmia}, joka on etenkin käsinlaskussa
kätevä. Algoritmin yleisperiaate on esimerkistä arvattavissa.
\begin{Exa} Laske $p_0$ ja $p_1$, kun $\,\D{f(x)=\frac{x^3+2x^2}{x^2-2x+3}}\,$.
\end{Exa}
\ratk
\begin{align*}
x^3+2x^2                          &= [x(x^2-2x+3)+2x^2-3x]+2x^2 \\
                                  &= x(x^2-2x+3)+4x^2-3x \\
                                  &= x(x^2-2x+3)+[4(x^2-2x+3)+8x-12]-3x \\
                                  &= (x+4)(x^2-2x+3)+5x-12 \\[3mm]
\impl \ \frac{x^3+2x^2}{x^2-2x+3} &= x+4+\frac{5x-12}{x^2-2x+3}
                                   =p_0(x)+\frac{p_1(x)}{q(x)}\,. \loppu
\end{align*}

Kun jakolasku on suoritettu, saadaan integraalin lineaarisuuden nojalla
\[
\int\frac{p(x)}{q(x)}\, dx=\int p_0(x)\, dx+\int\frac{p_1(x)}{q(x)}\, dx.
\]
Tässä polynomiosan integraalifunktio on heti määrättävissä (polynomina), joten jatkossa
voidaan rajoittua tutkimaan jälkimmäistä termiä, jossa osoittajan aste on pienempi kuin
nimittäjän aste. Oletetaan, että tunnetaan kaikki $q$:n (reaaliset ja kompleksiset)
nollakohdat --- nämähän ovat ainakin numeerisesti laskettavissa. Jos nollakohdat
$x_1\ldots x_n$ ovat kaikki reaalisia ja yksinkertaisia, niin saadaan osamurtokehitelmä
(ks.\ Lause \ref{osamurtolause} alla)
\[
\frac{p_1(x)}{q(x)}=\frac{A_1}{x-x_1}+\cdots +\frac{A_n}{x-x_n}\,.
\]
Kertoimet $A_i$ voidaan määrätä lineaarisesta yhtälöryhmästä, joka syntyy, kun tämä yhtälö
kerrotaan puolittian $q(x)$:llä ja identifioidaan polynomien kertoimet vasemmalla ja oikealla.
Helpommin kertoimet kuitenkin lasketaan kirjoittamalla $q(x)=(x-x_i)q_i(x)$, jolloin
kertomalla yhtälö puolittain $(x-x_i)$:llä seuraa
\[
A_i=\lim_{x\kohti x_i} \left[(x-x_i)\frac{p_1(x)}{q(x)}\right] =\, \frac{p(x_i)}{q_i(x_i)}\,.
\]
Jos $q(x)$:llä on tekijänä $(x-a)^k$ vastaten $k$-kertaista reaalijuurta $x_i=a$, niin
osamurtokehitelmässä tätä tekijää vastaavat yleisemmin termit
\[
(x-a)^k\ \ext\ \frac{A_1}{x-a}+\frac{A_2}{(x-a)^2}+\cdots +\frac{A_k}{(x-a)^k}\,.
\]
Näistä kukin termi on suoraan integroitavissa. Jos lopuksi $q(x)$:llä on tekijänä
$(x^2+ax+b)^k$ vastaten $k$-kertaista kompleksista konjugaattijuuriparia 
(jolloin on $b-a^2/4>0$), niin tämä tekijä tuottaa osamurtokehitelmään termit
\[
(x^2+ax+b)^k\ \ext\ \frac{B_1 x+C_1}{x^2+ax+b}
             +\frac{B_2 x+C_2}{(x^2+ax+b)^2}+\cdots + \frac{B_k x+C_k}{(x^2+ax+b)^k}\,.
\]
Näiden integroimiseksi huomataan ensinnäkin, että
\[
\frac{Ax+B}{(x^2+ax+b)^l}\,
          =\,\frac{A}{2}\,\frac{2x+a}{(x^2+ax+b)^l}+\frac{B-\frac{1}{2}\,aA}{(x^2+ax+b)^l}\,,
\]
joten
\begin{align*}
\int\frac{Ax+B}{(x^2+ax+b)^l}\,dx\, 
        &=\,\frac{A}{2}\cdot\begin{cases}
                            \ln (x^2+ax+b),                                &\text{jos}\ l=1, \\
                            -\dfrac{1}{l-1}\,\dfrac{1}{(x^2+ax+b)^{l-1}}\,, &\text{jos}\ l>1
                            \end{cases} \\
        &+\,(B-\frac{1}{2}\,aA)\int\frac{1}{(x^2+ax+b)^l}\ dx.
\end{align*}
Tässä jäljelle jäänyt integraali on käsiteltävissä edellisen luvun menetelmin, sillä kun
kirjoitetaan
\[
x^2+ax+b = \left(x+\frac{a}{2}\right)^2 + D^2, \quad D=\sqrt{b-\frac{a^2}{4}}\,,
\]
niin sijoituksella 
\[
x+\frac{a}{2} = Dt, \quad dx = D dt
\]
saadaan
\[
\int\frac{1}{(x^2+ax+b)^l}\ dx 
        = D^{-2l+1}\int \frac{1}{(t^2+1)^l}\,dt, \quad t=D^{-1}\left(x+\frac{a}{2}\right).
\]

Kun em.\ tulokset yhdistetään ja huomioidaan edellisen luvun tulokset, voidaan päätellä, että
jokaisen rationaalifunktion integraalifunktio on muotoa
\[
\int \frac{p(x)}{q(x)}\, dx=R(x)+\sum A_iF_i(x)+C,
\]
missä $R$ on rationaalifunktio (tai polynomi), kertoimet $A_i$ ovat reaalilukuja ja kukin
$F_i$ (enintään $n$ kpl, $n=q$:n aste) on jokin seuraavista funktiomuodoista:
\[
\ln\abs{x+a_i},\quad\ln(x^2+a_ix+b_i),\quad\Arctan(a_ix+b_i).
\]

On siis päätelty, että rationaalifunktio voidaan aina integroida suljetussa muodossa, kun
nimittäjäpolynomin kaikki nollakohdat (kertalukuineen) tunnetaan. Päätelmä perustui
oletukseen, että osamurtokehitelmä edellä kuvatulla tavalla on aina mahdollinen, ts. että
pätee
\begin{Lause} (\vahv{Rationaalifunktion osamurtohajotelma}) \label{osamurtolause}
\index{rationaalifunktio!a@osamurtohajotelma|emph}
Olkoon $f(x)=p(x)/q(x)$ reaalinen rationaalifunktio, jossa osoittajapolynomin $p$ aste
$<q$:n aste. Olkoon nimittäjäpolynomin $q$ reaalijuuret $x_i,\ i=1 \ldots n_1$ ja
kompleksijuuret $c_j=a_j \pm ib_j,\ j=1 \ldots n_2\,$ ($a_j,b_j\in\R,\ b_j \neq 0$). Edelleen
olkoon näiden juurien kertaluvut $m_i$, $i=1 \ldots n_1$ ja $\nu_j$, $j=1 \ldots n_2$.
Tällöin on olemassa kertoimet $A_{ik},B_{jk},C_{jk}\in\R$ siten, että jokaisella $x\in\DF_f$
pätee
\[
f(x) = \sum_{i=1}^{n_1}\sum_{k=1}^{m_i} \frac{A_{ik}}{(x-x_i)^k}
     + \sum_{j=1}^{n_2}\sum_{k=1}^{\nu_j} \frac{B_{jk}x+C_{jk}}{[(x-a_j)^2+b_j^2]^k}\,.
\]
\end{Lause}
\tod Sivuutetaan, ks.\ Harj.teht.\,\ref{H-int-3: osamurtohajotelma}.
\begin{Exa}
$\D\int\frac{x^2}{x^3+x^2+x-3}\, dx\ =\ ?$
\end{Exa}
\ratk Nimittäjäpolynomin eräs juuri on $x_1=1:\ q(x)=x^3+x^2+x-3=(x-1)(x^2+2x+3)$. Muut kaksi
juurta ovat kompleksisia (konjugaattipari), joten osamurtokehitelmä on muotoa
\[
\frac{x^2}{x^3+x^2+x-3}=\frac{A}{x-1}+\frac{Bx+C}{x^2+2x+3}\,.
\]
Kertomalla tässä molemmat puolet $q(x)$:llä saadaan ensin määrätyksi kertoimet $A,B,C$, minkä
jälkeen integrointi onnistuu:
\begin{align*}
&\qquad\qquad x^2= A(x^2+2x+3)+(Bx+C)(x-1)\quad\forall x\in\R \\
&\qekv       (A+B-1)x^2+(2A-B+C)x+(3A-C)=0\quad\forall x\in\R \\
&\qekv       \begin{cases} A+B-1 &= 0 \\ 2A-B+C &= 0 \\ 3A-C &= 0 \end{cases} 
             \qekv \begin{cases} A &= 1/6 \\ B &= 5/6 \\ C &= 1/2 \end{cases} \\[1mm]
&\qimpl\int\frac{x^2}{x^3+x^2+x-3}\,dx \,=\, 
                \frac{1}{6}\int\frac{1}{x-1}\, dx+\frac{1}{6}\int\frac{5x+3}{x^2+2x+3}\, dx \\
&\qquad\qquad=\,\frac{1}{6}\ln\abs{x-1}+\frac{1}{6}\int\left(
                \frac{5}{2}\,\frac{2x+2}{x^2+2x+3}-\frac{2}{x^2+2x+3}\right)dx \\
&\qquad\qquad=\,\frac{1}{6}\ln\abs{x-1}+\frac{5}{12}\int\frac{2x+2}{x^2+2x+3}\,dx
               -\frac{1}{6}\int\frac{1}{(\frac{x+1}{\sqrt{2}})^2+1}\,dx \\
&\qquad\qquad=\,\underline{\underline{\frac{1}{6}\ln\abs{x-1}+\frac{5}{12}\ln(x^2+2x+3)
                            -\frac{\sqrt{2}}{6}\Arctan\left(\frac{x+1}{\sqrt{2}}\right)+C}}.
\end{align*}
Kerroin $A$ olisi voitu määrätä myös suoremmin laskemalla
\[
A\ =\ \lim_{x\kohti 1}\left[(x-1)\frac{x^2}{x^3+x^2+x-3}\right]\ 
   =\ \lim_{x\kohti 1}\frac{x^2}{x^2+2x+3}\ =\ \frac{1}{6}\,. \loppu
\]

\subsection{Rationaalisiksi palautuvia integraaleja}

Jos $R(x,y)$ on kahden muuttujan rationaalifunktio, niin mm. seuraavat integraalit ovat
palautettavissa rationaalisiksi sopivilla sijoituksilla:
\begin{itemize}
\item[] a)\ $\int R(\sin x,\cos x)\,dx, \quad$ b)\ $\int R(e^x)\,dx$,
\item[] c)\ $\int R(x,\sqrt{1-x^2}\,)\,dx, \quad$ d)\ $\int R(x,\sqrt{x^2+1}\,)\,dx, \quad$
        e)\ $\int R(x,\sqrt{x^2-1}\,)\,dx$,
\item[] f)\ $\int R(x,\sqrt[m]{ax+b}\,)\,dx \quad (a\neq 0,\ m \in \N).$
\end{itemize}

\underline{Tapaus a)}\ \ Tässä toimii trigonometrinen 'gurusijoitus' 
(vrt.\ Luku \ref{trigonometriset funktiot})
\begin{align*}
\tan\frac{1}{2}x=t \qimpl &\begin{cases}
                        \,\sin x=\dfrac{2t}{1+t^2},\quad \cos x=\dfrac{1-t^2}{1+t^2}\,, \\[5mm] 
                        \,\dfrac{dt}{dx} = \dfrac{1}{2}\left(1+\tan^2\dfrac{1}{2}x\right) 
                                               \qimpl dx = \dfrac{2\, dt}{1+t^2}
                           \end{cases} \\[5mm]
\qimpl \int R(\sin x,\cos x)\,dx\, 
                          &=\, \int R\left(\frac{2t}{1+t^2},\frac{1-t^2}{1+t^2}\right)
                                                                   \frac{2}{1+t^2}\,dt \\
                          &=\, \int Q(t)\, dt,
\end{align*}
missä $Q$ on rationaalifunktio.

\underline{Tapaus b)}\ \ Sijoituksella (vrt.\ edellisen luvun Esimerkki 
\ref{rat-palautuva integraali 1})
\[
e^x = t \qimpl x=\ln t, \quad dx=\frac{1}{t}\,dt
\]
saadaan integraali muotoon 
\[
\int R(e^x)\,dx \,=\, \int R(t)\frac{1}{t}\,dt \,=\, \int Q(t)\,dt,
\]
missä $Q$ on jälleen rationaalifunktio. \\[5mm]
\underline{Tapaus c)}\ \ Tämä palautuu tapaukseen a) sijoituksilla
\[
x=\cos t \quad \text{tai} \quad x=\sin t.
\]

\underline{Tapaukset d) ja e)}\ \ Nämä palautuvat tapaukseen b) sijoituksilla
\[
\text{d)}\ x=\sinh t, \qquad \text{e)}\ x=\cosh t.
\]

\underline{Tapaus f)}\ \ Tässä toimii sijoitus 
(vrt.\ edellisen luvun Esimerkki \ref{rat-palautuva integraali 2}) 
\begin{align*}
&\sqrt[m]{ax+b}=t \qimpl x = \frac{1}{a}(t^m-b),\quad dx = \frac{m}{a}\,t^{m-1}\, dt \\[3mm]
&\impl \ \int R(x,\sqrt[m]{ax+b}\,)\, dx 
     = \int R\left(\frac{t^m-b}{a},\,t\right)\frac{m}{a}\,t^{m-1}\, dt=\int Q(t)\, dt.
\end{align*}
\begin{Exa}
$\D\int\frac{1}{\sin x}\, dx\ =\ ?$
\end{Exa}
\ratk Sijoituksella $\,\tan\frac{1}{2} x = t\,$ saadaan
\begin{align*}
\int\frac{1}{\sin x}\, dx 
  &= \int\frac{1+t^2}{2t}\cdot\frac{2}{1+t^2}\, dt =\int\frac{1}{t}\, dt \\
  &= \ln\abs{t}+C \\
  &= \ln\left|\tan\frac{1}{2}x\right|+C 
   = \underline{\underline{\ln\left|\frac{1-\cos x}{\sin x}\right|+C}}. \loppu
\end{align*}
\begin{Exa} $\D\int\frac{\sqrt{x^2+1}}{x}\,dx\ =\ ?$
\end{Exa}
\ratk Tehdään ensin sijoitus
\[
x=\sinh t,\quad dx=\cosh t\, dt,\quad \sqrt{x^2+1}=\cosh t
\]
\begin{align*}
\impl \ \int\frac{\sqrt{x^2+1}}{x}\,dx 
     &= \int\frac{\cosh^2 t}{\sinh t}\,dt = \int\frac{\sinh^2 t + 1}{\sinh t}\,dt \\
     &= \int \sinh t\,dt + \int\frac{1}{\sinh t}\,dt = \cosh t + \int \frac{2}{e^t-e^{-t}}\,dt.
\end{align*}
Tehdään uusi sijoitus 
\[
e^{t}=u,\quad t=\ln u,\quad \, dt=\frac{1}{u}\,du
\]
\begin{align*}
\impl \ \int \frac{2}{e^t-e^{-t}}\,dt 
               &= \int\frac{2}{u-u^{-1}}\cdot\frac{1}{u}\,du = \int\frac{2}{u^2-1}\,du \\
               &= \ln\left|\frac{u-1}{u+1}\right|+C = \ln\left|\frac{e^t-1}{e^t+1}\right|+C.
\end{align*}
Koska $\,x=\sinh t\ \ekv\ e^t = x+\sqrt{x^2+1}\,$, niin todetaan, että
\[
\int\frac{\sqrt{x^2+1}}{x}\,dx 
          = \sqrt{x^2+1}+\ln\left|\frac{x-1+\sqrt{x^2+1}}{x+1+\sqrt{x^2+1}}\right|+C.
\]
Huomioimalla vielä, että
(vrt.\ Harj.teht.\,\ref{jatkuvuuden käsite}:\ref{H-V-1: pelkistyvä funktio})
\[
\frac{x-1+\sqrt{x^2+1}}{x+1+\sqrt{x^2+1}} \,=\, \frac{x}{1+\sqrt{x^2+1}}\,, \quad x\in\R,
\]
saadaan lopputulokselle hiukan sievempi muoto:
\[
\int\frac{\sqrt{x^2+1}}{x}\,dx 
          = \underline{\underline{\sqrt{x^2+1}+\ln\frac{|x|}{1+\sqrt{x^2+1}}+C}}. \loppu
\]

\subsection{Integroinnin sarjamenetelmä}
\index{sarjamenetelmä (integroinnin)|vahv}

Silloin kun funktio on esitettävissä suppenevan potenssisarjan (Taylorin sarjan, ks.\ Luku
\ref{taylorin lause}) summana, voidaan myös integraalifunktio esittää tässä muodossa.
Oletetaan, että
\[
f(x)=\sum_{k=0}^\infty a_k(x-x_0)^k,\quad \abs{x-x_0}<\rho \quad (\rho>0).
\]
Tällöin myös sarja
\[
F(x)=\sum_{k=0}^\infty \frac{a_k}{k+1}\,(x-x_0)^{k+1}
\]
suppenee, kun $\abs{x-x_0}<\rho$ (Lause \ref{potenssisarjan skaalaus}) ja sarja termeittäin 
derivoimalla (mikä Lauseen \ref{potenssisarja on derivoituva} mukaan on sallittua) todetaan,
että
$F$ on $f$:n integraalifunktio välillä $(x_0-\rho,x_0+\rho)$.
\begin{Exa}
$\D\int e^{x^2}\, dx\ =\ ?$
\end{Exa}
\ratk
\[
e^{x^2}=\sum_{k=0}^\infty \frac{1}{k!}\,x^{2k},\quad x \in (-\infty,\infty),
\]
joten
\begin{align*}
\int e^{x^2}\, dx 
      &= \sum_{k=0}^\infty \frac{1}{(2k+1)k!}\,x^{2k+1}+C \\
      &= C+x+\frac{1}{3}\,x^3+\frac{1}{10}\,x^5+\frac{1}{42}\,x^7+..\,,\quad x\in\R. \loppu
\end{align*}
\begin{Exa} Määritä sarjoja hyväksi käyttäen
\[
\text{a)}\ \int \frac{e^{-x}}{x}\,dx, \qquad
\text{b)}\ \int \frac{\sin x}{x^4}\,dx.
\]
\end{Exa} 
\ratk Lähtien Taylorin sarjoista
\[
e^{-x} = \sum_{k=0}^\infty \frac{(-1)^k}{k!}\,x^k, \quad\
\sin x = \sum_{k=0}^\infty \frac{(-1)^k}{(2k+1)!}\,x^{2k+1}
\]
saadaan
\begin{align*}
\int\frac{e^{-x}}{x}\,dx   
  &= \int \frac{1}{x}\left(1-x+\frac{1}{2!}\,x^2-\frac{1}{3!}\,x^3+\ldots\right)\,dx \\
  &= \int \left(x^{-1}-1+\frac{1}{2!}\,x-\frac{1}{3!}\,x^2+\ldots\right)\,dx \\
  &= \left(\ln\abs{x}-x+\frac{1}{2\cdot 2!}\,x^2-\frac{1}{3\cdot 3!}\,x^3+\ldots\right) + C \\
  &= \ln\abs{x} + C + \sum_{k=1}^\infty \frac{(-1)^k}{k\cdot k!}\,x^k,
\end{align*}
\begin{align*}
\int\frac{\sin x}{x^4}\,dx 
  &= \int \frac{1}{x^4}\left(x-\frac{1}{3!}\,x^3+\frac{1}{5!}\,x^5-\ldots\right)\,dx \\ 
  &= \int \left(x^{-3}-\frac{1}{3!}\,x^{-1}+\frac{1}{5!}\,x-\ldots\right)\,dx \\
  &= \left(-\frac{1}{2}\,x^{-2}-\frac{1}{6}\ln\abs{x}
                               +\frac{1}{2\cdot 5!}\,x^2-\ldots\right) +C \\
  &= -\frac{1}{2x^2}-\frac{1}{6}\ln\abs{x}+C
                    +\sum_{k=1}^\infty \frac{(-1)^{k+1}}{2k(2k+3)!}\,x^{2k}.
\end{align*}
Molemmat tulokset ovat päteviä väleillä $(-\infty,0)$ ja $(0,\infty)$. \loppu

\Harj
\begin{enumerate}

\item
Integroi seuraavat rationaalifunktiot:
\begin{align*}
&\text{a)}\ \ \frac{x^3+2x^2-x+5}{x+2} \qquad
 \text{b)}\ \ \frac{x^2}{x^2+x-2} \qquad
 \text{c)}\ \ \frac{1}{x^3-4x^2+3x} \\
&\text{d)}\ \ \frac{24}{x(x^2-1)(x^2-4)} \qquad
 \text{e)}\ \ \frac{1}{(1-x)^2(1+x)} \qquad
 \text{f)}\ \ \frac{3+x-2x^3}{1-x^3} \\
&\text{g)}\ \ \frac{x^8+1}{x^6+x^4} \qquad
 \text{h)}\ \ \frac{3x^2+1}{(x^2-1)^2} \qquad
 \text{i)}\ \ \frac{1}{x^4+3x^2+2} \qquad
 \text{j)}\ \ \frac{1}{x^4+1} \\
&\text{k)}\ \ \frac{1}{x(x^2+a^2)} \qquad
 \text{l)}\ \ \frac{1}{x^2(x^2-a^2)} \qquad
 \text{m)}\ \ \frac{x^3}{x^3-a^3} \qquad
 \text{n)}\ \ \frac{1}{x^4-a^4}
\end{align*}

\item
Muunna integraali 
\[
\int \frac{1}{x^4}{\sqrt{x^2+a^2}}\,dx, \quad a>0
\]
sijoituksilla \ a) $x=a\sinh t$, \ b) $x=a\tan t$, \ c) $x=a/t$. Valitse vaihtoehdoista
helpoin ja laske integraali suljetussa muodossa.

\item
Integroi sopivalla sijoituksella:
\begin{align*}
&\text{a)}\ \ \frac{e^{3x}}{e^x+2} \qquad
 \text{b)}\ \ \frac{1}{\cosh x} \qquad
 \text{c)}\ \ \frac{1}{1+\sqrt[3]{x}} \qquad
 \text{d)}\ \ \frac{1}{\sqrt{x+1}+\sqrt[4]{x+1}} \\
&\text{e)}\ \ x^2\sqrt{a^2-x^2} \qquad
 \text{f)}\ \ \frac{1}{x\sqrt{1-x^2}} \qquad
 \text{g)}\ \ \frac{1}{x(x^2+1)^{3/2}} \\
&\text{h)}\ \ \frac{1}{5+4\sin x} \qquad
 \text{i)}\ \ \frac{2-\sin x}{2+\cos x} \qquad
 \text{j)}\ \ \frac{1}{5-4\sin x+3\cos x}
\end{align*}

\item
Laske seuraavat integraalit sijoituksella $\,\tan x=t$\,:
\[
\text{a)}\ \ \int \frac{\tan x}{1+2\tan x}\,dx \qquad
\text{b)}\ \ \int \frac{1}{a\cos^2 x+b\sin^2 x}\,dx,\,\ (a,b) \neq (0,0)
\]

\item
Jos $R(x,y)$ on kahden muuttujan rationaalinen lauseke, niin millaisella sijoituksella
integraali
\[
\int R\left(x,\sqrt[m]{\frac{x+a}{x+b}}\right)dx, \quad m\in\N,\ m \ge 2,\ \ a,b\in\R
\]
muuntuu rationaaliseksi? Laske tällä tavoin $\D\int\frac{1}{x}\sqrt{\frac{x+1}{x-1}}\,dx$.


\item
Laske seuraavien funktioiden integraalifunktiot potenssisarjojen avulla:
\begin{align*}
&\text{a)}\ \ \frac{\sin x}{x} \qquad
 \text{b)}\ \ \frac{\cosh x}{x} \qquad\,\,
 \text{c)}\ \ \frac{1-\cos x}{x^3} \qquad
 \text{d)}\ \ \frac{\sin x}{x^6} \\
&\text{e)}\ \ x^3 e^{x^3} \qquad
 \text{f)}\ \ \sin x^2 \qquad\ \
 \text{g)}\ \ \frac{\sin x^2}{x^3} \qquad\quad\,\
 \text{h)}\ \ \frac{\cos x^3}{\sqrt{x}}
\end{align*}

\item (*)
Esitä alkeisfunktioina (jos mahdollista) tai potenssisarjoina funktiot $F$ ja $G$ siten,
että pätee
\begin{align*}
&\text{a)}\ \ \int e^x\ln x\,dx = F(x)\ln x + G(x)+C, \quad x\in(0,\infty), \\
&\text{b)}\ \ \int \frac{1}{\sqrt{x}}\,\ln x\,\sin x\,dx
                   = x\sqrt{x}\,[\,F(x)\ln x + G(x)\,]+C, \quad x\in(0,\infty), \\
&\text{c)}\ \ \int \frac{\cos x}{x^2+1}\,dx = F(x)\Arctan x+G(x), \quad x\in\R, \\
&\text{d)}\ \ \int \frac{\sin x}{x^2+1}\,dx = F(x)\ln\,(x^2+1)+G(x), \quad x\in\R.
\end{align*}

\item (*) \label{H-int-3: osamurtohajotelma} \index{rationaalifunktio!a@osamurtohajotelma}
Lauseen \ref{osamurtolause} todistamiseksi tarkastellaan kompleksimuuttujan
rationaalifunktiota $f(z)=p(z)/q(z)$, missä $q$ on reaalikertoiminen polynomi astetta
$n\in\N$ ja $p$ on reaalikertoiminen polynomi astetta $\le n-1$. \vspace{1mm}\newline
a) Olkoon $c\in\C$ polynomin $q$ yksinkertainen nollakohta ja olkoon 
$q(z)=(z-c)\,q_1(z)$ ja $g(z)=p(z)/q_1(z)$. Näytä, että
\[
f(z) \,=\, \frac{g(c)}{z-c}+\frac{r(z)}{q_1(z)}\,, \quad z\in\DF_f,
\]
missä $r$ on polynomi astetta $\le n-2$. Päättele, että jos $c=a\in\R$, niin $r$ on
reaalikertoiminen. \newline
b) Olkoon $c=a+ib$ ja $\overline{c}=a-ib$ polynomin $q$ yksinkertaiset kompleksijuuret
($a,b\in\R,\ b \neq 0$) ja kirjoitetaan $q(z)=(z-c)(z-\overline{c})\,q_1(z)$. Näytä, että on
olemassa $A\in\C$ ja $B,C\in\R$ siten, että pätee
\[
f(z) \ =\ \frac{A}{z-c}+\frac{\overline{A}}{z-\overline{c}} +\frac{r(z)}{q_1(z)}
     \ =\ \frac{Bz+C}{(z-a)^2+b^2}+\frac{r(z)}{p_1(z)}, \quad z\in\DF_f,
\]
missä $r$ on reaalikertoiminen polynomi astetta $\le n-3$. \newline
c) Olkoon $c\in\C$ $q$:n $m$-kertainen juuri ja olkoon $q(z)=(z-c)^mq_1(z)$ ja
$g(z)=p(z)/q_1(z)$. Näytä, että pätee
\[
f(z) \,=\, \sum_{k=1}^m \frac{A_k}{(z-c)^k}+\frac{r(z)}{p_1(z)}, \quad z\in\DF_f,
\]
missä $A_k=g^{(m-k)}(c)/(m-k)!$ ja $r$ on polynomi astetta $\le n-m-1$. Päättele, että jos
$c=a\in\R$, niin kertoimet $A_k$ ovat reaaliset ja $r$ on reaalikertoiminen. \newline
d) Olkoon $c=a+ib$ ja $\overline{c}=a-ib$ polynomin $q$ $m$-kertaiset kompleksijuuret
($a,b\in\R,\ b \neq 0$) ja kirjoitetaan $q(z)=(z-c)^m(z-\overline{c})^mq_1(z)$. Näytä, että
on olemassa kertoimet $A_k\in\C$ ja $B_k,C_k\in\R$ siten, että pätee
\begin{align*}
f(z) \,&=\, \sum_{k=1}^m \frac{A_k}{(z-c)^k}
           +\sum_{k=1}^m \frac{\overline{A}_k}{(z-\overline{c})^k}+\frac{r(z)}{q_1(z)} \\
       &=\, \sum_{k=1}^m \frac{B_kz+C_k}{[(z-a)^2+b^2]^k}+\frac{r(z)}{q_1(z)}, \quad z\in\DF_f,
\end{align*}
missä $r$ on reaalikertoiminen polynomi astetta $\le n-2m-1$. \newline
e) Perustuen kohtien a) ja b) tuloksiin ja Algebran peruslauseeseen, johda Lauseen
\ref{osamurtolause} väittämä siinä tapauksessa, että $q$:n kaikki juuret ovat
yksinkertaisia. \newline
f) Todista Lause \ref{osamurtolause} yleisessä tapauksessa perustuen kohtien c) ja d)
tuloksiin ja Algebran peruslauseeseen.


\end{enumerate} %Osamurtokehitelmät. Sarjamenetelmä
\section{Integraalifunktion numeerinen laskeminen. \\ Määrätty integraali}
\label{määrätty integraali}
\sectionmark{Määrätty integraali}
\alku

Annetun funktion integraalifunktion määräämiseksi on toistaiseksi tarkasteltu erilaisia
funktioalgebran keinoja. Näissä funktio on tunnettava laskusääntönä eli jonakin
algebrallisena lausekkeena tarkasteltavalla välillä tai sen osaväleillä. Sovelluksia
ajatellen tällainen oletus on melko rajoittava, sillä funktio voi yhtä hyvin olla pelkkä
pisteittäin määritelty 'luettelo'. Näin on esimerkiksi, jos funktion arvoja voidaan
käytännössä vain mitata tai jos funktion määritelmä on epäsuora niin, että funktioevaluaatio
$x \map f(x)$ onnistuu vain numeerisesti. Jos tällaisella funktiolla on integraalifunktio,
niin luonnollisesti integraalifunktionkaan arvoja ei voida käytännössä laskea muuten kuin
numeerisesti. Mutta miten? --- Jatkossa ratkaistaan tämä ongelma johtamalla yleispätevä
laskukaava numeeriselle funktioevaluaatiolle $x \map F(x)$, missä $F$ on annetun funktion
integraalifunktio. Laskukaava johdattaa samalla integraalifunktiosta riippumattomaan
\kor{määrätyn integraalin} käsitteeseen.

Oletetaan, että $f$ on määritelty (reaalifunktiona) välillä $[a,b]$, eli jokaiseen
$x\in[a,b]$ liittyy yksikäsitteinen luku $f(x)\in\R$. Täsmennetään $f$:n
integraalifunktion määrittämisen ongelma välillä $(a,b)$ seuraavasti:
\begin{itemize}
\item[(P)] Etsi funktio $y(x)$, joka on jatkuva välillä $[a,b]$, derivoituva välillä $(a,b)$
ja toteuttaa
\[
\begin{cases}
\,y'(x)=f(x),\quad x\in (a,b), \\
\,y(a)=0.
\end{cases}
\]
\end{itemize}
Tämän mukaan etsitään siis funktiota, joka on $f$:n integraalifunktio välillä $(a,b)$ ja
toteuttaa lisäksi asetetun jatkuvuusehdon ja alkuehdon. Koska derivoituvuus jo takaa
jatkuvuuden välillä $(a,b)$, niin jatkuvuusehto merkitsee ainoastaan lisäehtoja:
$y(x)$ on oikealta jatkuva $a$:ssa ja vasemmalta jatkuva $b$:sä. 

Sikäli kuin probleemalla (P) on ratkaisu, takaa asetettu alkuehto ratkaisun
yksikäsitteisyyden (vrt.\ Korollaari \ref{toiseksi yksinkertaisin dy}). Vaikeampaan
kysymykseen, joka koskee ratkaisun olemassoloa, antaa erään vastauksen seuraava huomattava
lause.
\begin{*Lause} \label{P:n ratkeavuus} Jos $f$ on jatkuva välillä $[a,b]$, niin probleemalla
(P) on ratkaisu.
\end{*Lause}
Lause \ref{P:n ratkeavuus} on keskeinen osaväittämä yleisemmässä lauseessa, joka tunnetaan
\kor{Analyysin peruslauseen} nimellä. Tämä muotoillaan ja (osittain) todistetaan jäljempänä
Luvussa \ref{analyysin peruslause}. Tässä luvussa asetetaan kysymys toisin: Jos yksinkertaisesti
\pain{oletetaan}, että (P) ratkeaa, niin miten luku $y(x)\in\R$ voidaan käytännössä
\pain{laskea}, kun $x\in(a,b]$ on annettu? Jatkossa tarkastellaan tätä kysymystä ensin
tapauksessa $x=b$, jolloin tehtävänä on siis laskea luku $y(b)$, kun $y(x)$ on määritelty
epäsuorasti (P):n ratkaisuna. Koska $y(b)$ on reaaliluku, niin ko.\ lukua ei yleisesti voi
laskea 'tarkasti', vaan käytännössä on tyydyttävä konstruoimaan $y(b)$ jonkin (viime kädessä
rationaalisen) lukujonon $\seq{Y_n}$ raja-arvona. Tehtävänä on siis määrittää algoritmi, jolla
on mahdollista laskea jokaisella $n\in\N$ luku $Y_n$ siten, että $\,\lim_nY_n=y(b)$.

Olkoon $n\in\N$ annettu. Lukua $Y_n$ määrättäessä otetaan lähtökohdaksi välin $[a,b]$
\index{jako (osaväleihin)}%
\kor{jako osaväleihin} $[x_{k-1},x_k]$, $k=1 \ldots n$, missä 
$a = x_0 < x_1 < \ldots < x_n = b$. Jako voi olla esim.\ 
\index{tasavälinen jako}%
\kor{tasavälinen}, jolloin on
$x_k=a+kh_n$, missä $h_n=(b-a)/n$. Yleisemmin osavälijakoja rajoitetaan ainoastaan ehdolla
\begin{equation} \label{int-4: tiheysehto}
\lim_nh_n=0, \quad \text{missä}\,\ h_n = \max_{k=1 \ldots n} (x_k-x_{k-1}),\,\ n\in\N.
                                                                       \tag{$\star$}
\end{equation}
Tässä $h_n$ on osavälijaon nk.\ 
\index{tiheysparametri}%
\kor{tiheysparametri}. Tasavälisille jaoille raja-arvoehto
\eqref{int-4: tiheysehto} luonnollisesti toteutuu.

Kun pisteet $x_k,\ k = 0 \ldots n$ on valittu em.\ tavalla, niin (P):n ratkaisulle $y(x)$
voidaan ensinnäkin kirjoittaa
\[
y(b) \,=\, [y(x_n)-y(x_{n-1})]+ \ldots +[y(x_1)-y(x_0)]+y(x_0)
\]
ja alkuehdon $y(x_0)=y(a)=0$ perusteella siis
\[
y(b)=\sum_{k=0}^n [y(x_k)-y(x_{k-1})].
\]
Funktio $y(x)$ toteuttaa Differentiaalilaskun väliarvolauseen ehdot osaväleillä $\quad$
$[x_{k-1},x_k]$ ja lisäksi $y'(x)=f(x),\ x\in(x_{k-1},x_k)$. Väliarvolauseen mukaan on tällöin
olemassa pisteet $\xi_k\in(x_{k-1},x_k),\ k=1 \ldots n$ siten, että
\[
y(x_k)-y(x_{k-1})=f(\xi_k)(x_k-x_{k-1}), \quad k=1\ldots n.
\]
Yhdistämällä kaksi viimeistä tulosta on päätelty, että (P):n ratkaisulle pätee
\[
y(b)=\sum_{k=1}^nf(\xi_k)(x_k-x_{k-1}).
\]
Tämä ei toimi suoraan laskukaavana, koska pisteitä $\xi_k$ ei tunneta, mutta kaava on
luonteva lähtökohta approksimatiolle: Korvataan tuntemattomat pisteet $\xi_k$ joillakin
\pain{valituilla} pisteillä $\xi_k\in[x_{k-1},x_k]$, esim.\ $\xi_k=x_{k-1}$ tai
$\xi_k=(x_{k-1}+x_k)/2$. Kun valinta on jokaisella osavälillä tehty, niin lasketaan
\[
Y_n = \sum_{k=1}^nf(\xi_k)(x_k-x_{k-1}).
\]
Näin muodostuu lukujono $\seq{Y_n}$, jonka $n$:s termi siis lasketaan seuraavasti:
\begin{itemize}
\item[A1.] Valitaan \kor{jakopisteet} $x_k,\ k=0 \ldots n$ niin, että
           $a=x_0 < x_1 < \ldots < x_n=b$. \index{jakopiste}%
\item[A2.] Valitaan \kor{välipisteet} $\xi_k\in[x_{k-1},x_k],\ k=1 \ldots n$.
           \index{vzy@välipiste(istö)}%
\item[A3.] Lasketaan $Y_n=\sum_{k=1}^n f(\xi_k)(x_k-x_{k-1})$.
\end{itemize}
Tämän mukaisesti $Y_n$:n laskemiseksi riittää suorittaa äärellinen määrä laskuoperaatioita:
$n$ kpl (käytännössä yleensä likimääräisiä) funktioevaluaatioita $x \map f(x)$ ja lisäksi
$n$ kertolaskua ja $n-1$ yhteenlaskua.

Verrattaessa algoritmilla A1--A3 laskettua lukua $Y_n$ lukuun $y(b)$ on em.\ tarkkaa
$y(b)$:n lauseketta verrattava $Y_n$:n vastaavaan lausekkeeseen (A3). Näissä välipisteet
$\xi_k$ eivät ole samat, joten kirjoitettakoon lausekkeessa (A3) $\xi_k$:n tilalle $\eta_k$.
Approksimaatiolle $\,Y_n \approx y(b)$ saadaan tällöin virhekaava
\[
y(b)-Y_n \,=\, \sum_{k=1}^n[f(\xi_k)-f(\eta_k)](x_k-x_{k-1}).
\]
Virheen arvioimiseksi on siis pystyttävä arvioimaan erotuksia $f(\xi_k)-f(\eta_k)$, kun
pisteistä $\xi_k$ ja $\eta_k$ tiedetään ainoastaan, että ne ovat samalla osavälillä
$[x_{k-1},x_k]$. Arvion onnistumiseksi toivotulla tavalla on ilmeisesti tehtävä jokin
säännöllisyys\-oletus funktiosta $f$. --- Yleisesti jos $f$ on riittävän säännöllinen, on
$\abs{f(\xi)-f(\eta)}$ enintään verrannollinen lukuun $\abs{\xi-\eta}$, ts.\
$f(\xi)-f(\eta)=\Ord{\abs{\xi-\eta}}$. Tämän takaava minimioletus on, että $f$ on välillä
$[a,b]$ Lipschitz-jatkuva, ts.\ $\exists\,L\in\R_+$ (= $f$:n Lipschitz-vakio) siten, että
\[
\abs{f(\xi)-f(\eta)} \le L \abs{\xi-\eta} \quad \forall\ \xi,\eta\in[a,b]
\]
(ks.\ Määritelmä \ref{funktion l-jatkuvuus} ja myös Lause \ref{Lipschitz-kriteeri}).

Kun oletetaan $f$:n Lipschitz-jatkuvuus, niin pätee
\[
\abs{f(\xi_k)-f(\eta_k)} \le L\abs{\xi_k-\eta_k}, \quad k=1 \ldots n.
\]
Tässä on $\,\abs{\xi_k-\eta_k} \le x_k-x_{k-1}$, koska $\xi_k,\eta_k\in[x_{k-1},x_k]$ ja
edelleen $x_k-x_{k-1} \le h_n$, joten $\,\abs{f(\xi_k)-f(\eta_k)} \le Lh_n,\ k=1 \ldots n$.
Kun käytetään näitä arvioita yhdessä kolmioepäyhtälön kanssa em.\ virhekaavassa, niin seuraa
\begin{align*}
\abs{y(b)-Y_n)} \,&\le\, \sum_{k=1}^n \abs{f(\xi_k)-f(\eta_k)}(x_k-x_{k-1}) \\
                  &\le\, Lh_n\sum_{k=1}^n (x_k-x_{k-1}) \,=\, L(b-a)h_n.
\end{align*}

On päädytty seuraavaan tulokseen.
\begin{Lause} \label{summakaavalause} Olkoon $f$ Lipschitz-jatkuva välillä $[a,b]$ vakiolla
$L$ ja olkoon $y(x)$ probleeman (P) ratkaisu. Tällöin jos $\seq{Y_n}$ on mikä tahansa
algoritmilla A1--A3 laskettu lukujono raja-arvoehdolla \eqref{int-4: tiheysehto}, niin
$\,\lim_nY_n=y(b)$, tarkemmin 
$\abs{y(b)-Y_n} \le L(b-a)h_n\ \forall n\in\N$.\footnote[2]{Kuten nähdään jäljempänä
Luvussa \ref{analyysin peruslause}, Lauseen \ref{summakaavalause} raja-arvoväittämä
(ilman tarkennusta) on tosi myös, jos $f$ oletetetaan ainoastaan jatkuvaksi välillä $[a,b]$.
Mitään kvantitatiivista virhearviota approksimaatiolle $Y_n \approx y(b)$ ei pelkän
jatkuvuusoletuksen perusteella saada.}
\end{Lause}

Edellä laskettiin toistaiseksi (P):n ratkaisu vain pisteessä $x=b$. Tulos on kuitenkin
helposti yleistettävissä. Nimittäin jos $y(x)$ on (P):n ratkaisu ja kiinnitetään $x\in(a,b]$,
niin funktio $y(t),\ t\in[a,x]$ on jatkuva välillä $[a,x]$ (koska on jatkuva välillä
$[a,b]$) ja ratkaisu alkuarvotehtävälle
\[
\begin{cases}
\,y'(t)=f(t),\,\ t\in(a,x), \\ \,y(a)=0.
\end{cases}
\]
Kyseessä on siis probleema (P), missä muuttujana on $x$:n sijasta $t$ ja $b$:n tilalla on $x$.
Näin muodoin kun tehdään samat vaihdokset algoritmissa A1--A3, niin saadaan lasketuksi $y(x)$
missä tahansa halutussa pisteessä $x\in(a,b]$.
\begin{Exa} Ratkaise alkuarvotehtävä $\,y'(x)=x,\ x>0,\ y(0)=0$ algoritmilla A1--A3.
\end{Exa}
\ratk  Kiinnitetään $x>0$ ja tarkastellaan väliä $[0,x]$. Asetetaan jakopisteet $t_k$
tasavälisesti, eli $t_k=kx/n,\ k=0 \ldots n$, jolloin laskukaavan (A3) mukaan on
\[
Y_n \,=\, \sum_{k=1}^n f(\xi_k)(t_k-t_{k-1}) 
    \,=\, \sum_{k=1}^n \xi_k\,\frac{x}{n}\,.
\]
Välipisteiden valinnassa kokeiltakoon vaihtoehtoja \ a) $\xi_k=t_k$, \ b) $\xi_k=t_{k-1}$ ja
\ c) $\xi_k=(t_{k-1}+t_k)/2$, \ $k=1 \ldots n$, jolloin saadaan:
\begin{align*}
\text{a)}\ \ Y_n &\,=\, \sum_{k=1}^{n}k\,\frac{x}{n}\cdot\frac{x}{n}
                  \,=\, \frac{x^2}{n^2}\sum_{k=1}^{n} k
                  \,=\, \frac{x^2}{n^2}\cdot\frac{1}{2}\,n(n+1)
                  \,=\, \frac{x^2}{2}+\frac{x^2}{2n}\,. \\
\text{b)}\ \ Y_n &\,=\, \sum_{k=1}^{n}(k-1)\,\frac{x}{n}\cdot\frac{x}{n}
                  \,=\, \frac{x^2}{n^2}\sum_{k=1}^{n} (k-1)
                  \,=\, \frac{x^2}{n^2}\cdot\frac{1}{2}\,n(n-1)
                  \,=\, \frac{x^2}{2}-\frac{x^2}{2n}\,. \\
\text{c)}\ \ Y_n &\,=\, \sum_{k=1}^{n}\frac{1}{2}\left[(k-1)\,\frac{x}{n}+k\,\frac{x}{n}\right]
                                                                            \cdot\frac{x}{n}
                  \,=\, \frac{x^2}{n^2}\sum_{k=1}^{n} \left(k-\frac{1}{2}\right)
                  \,=\, \frac{x²}{n^2}\cdot\frac{1}{2}\,n^2
                  \,=\, \frac{x^2}{2}\,.
\end{align*}
Havaitaan, että kaikissa tapauksissa on $\,\lim_nY_n=\tfrac{1}{2}x^2$. Lause
\ref{summakaavalause} soveltuu ($L=1$), joten $\,y(x)=\tfrac{1}{2}x^2,\ x>0$. \loppu

Esimerkissä pätee approksimaatiolle $Y_n \approx y(x)$ Lauseen \ref{summakaavalause} mukaan 
virhearvio $\,\abs{y(x)-Y_n} \le x^2/n\,$ ($L=1,\ b-a=x,\ h_n=x/n$). Havaitut virheet ovat
\ a) $y(x)-Y_n=-\tfrac{x^2}{2n}$, \ b) $y(x)-Y_n=\tfrac{x^2}{2n}$, \ $y(x)-Y_n=0$, joten
nämä ovat arvion kanssa sopusoinnussa.

Jos (P):n ratkaisu $y(x)$ halutaan laskea useammassa pisteessä, voidaan algoritmi A1--A3
käynnistää jokaista laskentapistettä varten erikseen, kuten esimerkissä. Numeerisesti
laskettaessa algoritmi kannattaa kuitenkin ottaa tehokkaampaan käyttöön huomioimalla, että
kaavan (A3) mukaan luku $Y_n$ tullaan laskeneeksi käytännössä palautuvasti muodossa
\[
Y_0=0, \quad Y_k=Y_{k-1}+f(\xi_k)(x_k-x_{k-1}), \quad k=1 \ldots n.
\]
Luvut $Y_k=\sum_{i=1}^k f(\xi_i)(x_i-x_{i-1}),\ k=1 \ldots n-1$ saadaan tällöin summauksen
välituloksina. Nämä ovat algoritmin A1--A3 mukaisia approksimaatioita luvuille $y(x_k)$, kun
jakopisteinä ovat $x_i,\ i=0 \ldots k$ välillä $[a,x_k]$. Samoin oletuksin kuin edellä myös
näiden virhe on $\Ord{h_n}$, tarkemmin $|y(x_k)-Y_k| \le L(x_k-a)h_n,\ k=1 \ldots n-1$.
Luvut $Y_k$ kannattaa siis kaikki huomioida, jolloin saadaan käsitys funktiosta $y(x)$ koko
tarkasteltavalla välillä ilman lisätyötä.
\jatko \begin{Exa} (jatko) Kuviossa on valittu $x=2,\ n=4$ ja yhdistetty esimerkin
valinnoilla lasketut pisteet $(x_k,Y_k)$ murtoviivaksi. Tämä esittää funktion $y(x)$
kuvaajaa likimäärin välillä $[0,2]$. Tulos on tarkin tapauksessa (c), mutta muissakin
tapauksissa tarkkuus kasvaa kiinteällä $x$, kun $n\kohti\infty$.
\end{Exa} 
\begin{figure}[H]
\setlength{\unitlength}{0.5cm}
\begin{center}
\begin{picture}(12,11)(1,2.5)
\put(1,0){\vector(1,0){12}} \put(13.5,-0.2){$x$}
\put(1,0){\vector(0,1){12.5}} \put(0.9,13){$y$}
\multiput(1,0)(2.5,0){5}{\line(0,-1){0.1}}
\put(1,5){\line(-1,0){0.1}} \put(1,10){\line(-1,0){0.1}}
\put(0.2,4.7){$1$} \put(0.2,9.7){$2$}
\put(0.8,-1){$0$} \put(5.8,-1){$1$} \put(10.8,-1){$2$}
\path(1,0)(3.5,0.625)(6,2.5)(8.5,5.625)(11,10) \put(11.5,9.9){(c)}
\path(1,0)(3.5,0.781)(6,3.125)(8.5,7.031)(11,12.5) \put(11.5,12.4){(a)}
\path(1,0)(3.5,0.469)(6,1.875)(8.5,4.219)(11,7.5) \put(11.5,7.4){(b)}
\end{picture}
\end{center}
\end{figure}


\subsection{Määrätty integraali}
\index{mzyzyrzy@määrätty integraali|vahv}

Algoritmin A1--A3 mukaan lasketulla luvulla $y(b)=\lim_nY_n$, missä $y(x)$ on probleeman (P)
ratkaisu, on matematiikassa oma nimensä ja merkintänsä: Lukua kutsutaan funktion $f$
\kor{määrätyksi} \kor{integraaliksi yli välin} $[a,b]$, merkitään
\[
y(b)=\int_a^b f(x)\,dx
\]
ja luetaan 'integraali $a$:sta $b$:hen $f(x)\,dx$'. Sanotaan edelleen, että $a$ on integraalin
\kor{alaraja}, $b$ on \kor{yläraja} ja $[a,b]$ on \kor{integroimisväli}. Muuttuja $x$, eli
\kor{integroimismuuttuja}, on 'dummy' (kuten summausindeksi), ts.\ muuttuja voidaan vaihtaa
integraalin merkityksen muuttumatta.

Määrätty integraali on siis integroitavasta funktiosta $f$ ja integroimisvälistä riippuva
reaaliluku, joka on käytännösä laskettavissa lukujonon raja-arvona algoritmin (A1)--(A3)
mukaisesti. Kun $Y_n$:n laskukaavassa (A3) käytetään erotuksille $x_k-x_{k-1}$
lyhenysmerkintää $\Delta x_k$, niin määrätyn integraalin laskeminen edellä esitetyllä tavalla
voidaan tiivistää \kor{summakaavaksi} \index{summakaava (määr.\ integraalin)}
\begin{equation} \label{int-4: summakaava}
\boxed{\quad \int_a^b f(x)\,dx = \Lim_n \sum_{k=1}^n f(\xi_k)\Delta x_k. \quad}
\end{equation}
\index{raja-arvo!f@määrätyn integraalin}%
Tässä raja-arvomerkintään '$\Lim_n$' on sisällytetty ensinnäkin algoritmissa A1--A3 asetetut
rajoitukset koskien pisteiden $x_k$ ja $\xi_k$ valintaa, toiseksi ehto
\eqref{int-4: tiheysehto} ja kolmanneksi vaatimus, että \pain{kaikki} algoritmin (A1)--(A3)
mukaiset lukujonot $\seq{Y_n}$ suppenevat kohti \pain{samaa} raja-arvoa (= määrätyn integraalin
arvo). Edellä esitetyn perusteella viimeksi mainittu vaatimus toteutuu (ja summakaava on siis
pätevä) ainakin oletuksin, että probleema (P) on ratkeava (perusoletus toistaiseksi!) ja $f$ on
Lipschitz-jatkuva välillä $[a,b]$.

Em.\ oletuksilla probleeman (P) ratkaisulle johdettiin edellä myös yleisempi laskukaava, joka
voidaan nyt kirjoittaa määrättynä integraalina:
\begin{equation} \label{int-4: P:n ratkaisukaava}
y(x)=\int_a^x f(t)\,dt, \quad a < x \le b.
\end{equation}
Koska (P):n ratkaisu on (eräs) $f$:n integraalifunktio välillä $(a,b)$, niin kaava
\eqref{int-4: P:n ratkaisukaava} yhdessä summakaavan \eqref{int-4: summakaava} kanssa antaa
tähänastisesta integroimistekniikasta (myös derivoimissäännöistä!) riippumattoman menetelmän
integraalifunktion määräämiseksi.\footnote[2]{Integraalifunktion merkinnän '$\int f(x)\,dx$'
taustalla on määrätyn integraalin summakaava. Merkinnän otti käyttöön \hist{G.W. Leibniz}
ajatellen summakaavan 'rajankäyntejä' $\sum\hookrightarrow\int$ ja
$\Delta x_k\hookrightarrow dx$, kun $n\kohti\infty$. Leibnizin omien päiväkirjamerkintöjen
mukaan integraalimerkinnän tarkka keksimispäivä oli 29.10.1675. \index{Leibniz, G. W.|av}}

Toisaalta jos välillä $(a,b)$ tunnetaan $f$:n integraalifunktio $F$ 
(esim.\ alkeisfunktiona tai sarjana) ja $F$ on jatkuva välillä $[a,b]$
(välttämäntön lisäoletus!), niin luvun $\int_a^b f(x)\,dx\,$ laskeminen käy päinsä suoraan
$F$:n avulla. Nimittäin näillä oletuksilla probleeman (P) ratkaisu on $y(x)=F(x)-F(a)$,
jolloin
\[
\int_a^b f(x)\, dx \,=\, y(b) \,=\,F(b)-F(a).
\]
Tämä tunnettuun integraalifunktioon $F$ perustuva määrätyn integraalin laskukaava esitetään
yleensä \kor{sijoituskaavana} \index{sijoituskaava (määr.\ integraalin)}
\begin{equation} \label{int-4: sijoituskaava}
\boxed{\kehys\quad \int_a^b f(x)\,dx = \sijoitus{x=a}{x=b} F(t) 
                                     = \sijoitus{a}{b} F(x). \quad}
\end{equation}
Tässä oikea puoli luetaan 'sijoitus $a$:sta $b$:hen $F(x)$'. Kaava \eqref{int-4: sijoituskaava}
saadaan päteväksi myös tapauksessa $a>b$, kun \pain{sovitaan}, että määrätylle integraalille
on voimassa \kor{vaihtosääntö} \index{vaihtoszyzy@vaihtosääntö!a@määrätyn integraalin}
\begin{equation} \label{int-4: vaihtosääntö}
\boxed{\quad \int_{a}^b f(x)\, dx=-\int_{b}^a f(x)\, dx. \quad}
\end{equation}
Mukavuussyistä oletetaan tämä päteväksi myös kun $a=b$, jolloin tulee sovituksi, että
\[
\int_a^a f(x)\, dx=0.
\]

Huomautettakoon määrätyn integraalin käsitteestä, että toistaiseksi kyse ei ole muusta kuin
erikoisesta merkinnästä luvulle $y(b)$ siinä tapauksessa, että probleemalla (P) on ratkaisu
$y(x)$. Samaa ajatusta kertaavat myös sijoituskaava \eqref{int-4: sijoituskaava} ja
vaihtosääntö \eqref{int-4: vaihtosääntö}. Seuraavassa luvussa nähdään, että
määrätystä integraalista tulee (P):n ratkeavuudesta (ja yleisemminkin integraalifunktion
olemassaolosta) riippumaton käsite, kun määritelmäksi otetaan suoraan summakaava
\eqref{int-4: summakaava}, johon edellä päädyttiin numeerisena laskukaavana luvulle $y(b)$.
Myöhemmissä luvuissa määrätyn integraalin käsite saa edelleen lisää 'eloa' erilaisista
sovelluksista. Kuten tullaan näkemään, sovelluksissa määrättyyn integraaliin päädytään
suoraan summakaavan kautta. Tällöin sijoituskaava \eqref{int-4: sijoituskaava} näyttäytyy
oikotienä integraalin arvon 'tarkkaan' laskemiseen silloin, kun funktiolle $f$ on löydettävissä
(edellisten lukujen menetelmin) välillä $[a,b]$ jatkuva integraalifunktio $F$.  

Seuraavassa esitetään määrätyn integraalin kolme keskeistä ominaisuutta. Nämä pysyvät voimassa,
kun määrittelyn perustaksi jatkossa otetaan summakaava \eqref{int-4: summakaava}, mutta
toistaiseksi ominaisuudet perustellaan vedoten oletettuun probleeman (P) ratkeavuuteen. 


\subsection{Additiivisuus. Lineaarisuus. Vertailuperiaate}

Jos probleema (P) on ratkeava, niin jokaisella $c\in(a,b)$ voidaan kirjoittaa
$y(b)=y(c)+[y(b)-y(c)]$ eli määrätyn integraalin avulla ilmaistuna
\index{additiivisuus!a@integraalin}%
\begin{equation} \label{int-4: additiivisuussääntö}
\boxed{\quad \int_a^b f(x)\,dx = \int_a^c f(x)\,dx + \int_c^b f(x)\,dx. \quad}
\end{equation}
Tämän säännön perusteella sanotaan, että määrätty integraali on \kor{additiivinen}
\kor{integroimisvälin suhteen}. Kun huomiodaan myös vaihtosääntö \eqref{int-4: vaihtosääntö},
niin todetaan, että additiivisuussääntö \eqref{int-4: additiivisuussääntö} on pätevä lukujen
$a,b,c$ suuruusjärjestyksestä riippumatta edellyttäen, että probleema (P) on ratkaistavissa,
kun $a$:n tilalla on $\min\{a,b,c\}$ ja $b$:n tilalla $\max\{a,b,c\}$.

Olkoon (P):n ratkaisu $F(x)$ ja lisäksi olkoon (P):llä ratkaisu $G(x)$, kun $f$:n tilalla on
funktio $g$. Tällöin jos $\alpha,\beta\in\R$, niin $y(x)=\alpha F(x)+\beta G(x)\,$ on
(P):n ratkaisu, kun $f$:n tilalla on $\alpha f +\beta g$. Erityisesti on siis
$y(b)=\alpha F(b)+\beta G(b)$, eli määrätylle integraalille pätee \kor{lineaarisuussääntö}
(vrt.\ määräämättömän integraalin vastaava sääntö \ref{integraalifunktio}:\,(I-1))
\index{lineaarisuus!b@integroinnin}%
\begin{equation} \label{int-4: lineaarisuussääntö}
\boxed{\quad \int_a^b [\alpha f(x)+\beta g(x)]\, dx 
        = \alpha\int_a^b f(x)\, dx + \beta\int_a^b g(x)\, dx, \quad \alpha,\beta\in\R. \quad}
\end{equation}
Jos mainittujen oletusten lisäksi oletetetaan, että että $f(x) \le g(x)\ \forall x\in[a,b]$,
niin $F'(x)-G'(x) = f(x)-g(x) \le 0,\ x\in(a,b)$, jolloin $F(x)-G(x)$ on vähenevä välillä
$[a,b]$ (Lause \ref{monotonisuuskriteeri}). Erityisesti on $F(b)-G(b) \le F(0)-G(0) = 0$ eli
$F(b) \le G(b)$. Näin ollen määrätyille integraaleille pätee \kor{vertailuperiaate}
\index{vertailuperiaate!a@integraalien}%
\begin{equation} \label{int-4: vertailuperiaate}
\boxed{\quad f(x)\leq g(x)\quad\forall x\in [a,b] 
             \ \impl \ \int_a^b f(x)\,dx \le \int_a^b g(x)\,dx. \quad}
\end{equation}
Koska $\pm f(x) \le \abs{f(x)}$ ja $\int_a^b [\pm f(x)]\,dx = \pm\int_a^b f(x)\,dx$
(lineaarisuussääntö!), niin vertailuperiaattetta soveltaen seuraa erityisesti
\begin{equation} \label{int-4: kolmioepäyhtälö}
\boxed{\quad \left|\int_a^b f(x)\, dx\right|\leq \int_a^b\abs{f(x)}\,dx. \quad}
\end{equation}
Tätä sanotaan 
\index{kolmioepäyhtälö!f@integraalien}%
\kor{integraalien kolmioepäyhtälöksi}.\footnote[2]{Epäyhtälö
\eqref{int-4: kolmioepäyhtälö} on sukua järjestetyn kunnan kolmioepäyhtälölle, joka pätee
summakaavan \eqref{int-4: summakaava} äärellisille summille.}
\begin{Exa} Sijoituskaavan \eqref{int-4: sijoituskaava} perusteella on
\begin{align*}
&\int_0^2 \sin x\, dx=\sijoitus{0}{2}(-\cos x)=\sijoitus{2}{0}\cos x=1-\cos 2, \\
&\int_x^{x^2} t\, dt =\sijoitus{x}{x^2}\frac{1}{2}t^2=\frac{1}{2}(x^4-x^2), \quad x\in\R.
\end{align*}
\end{Exa}
\begin{Exa} Laske $\int_0^2 f(x)\,dx$, kun $f(x)=\max\{\sqrt{x},x^2\}$. \end{Exa}
\ratk Todetaan ensin, että $f(x)=\sqrt{x}$ välillä $[0,1]$ ja $f(x)=x^2$ välillä $[1,2]$.
Tällöin käyttämällä ensin additiivisuussääntöä \eqref{int-4: additiivisuussääntö} ja sitten
sijoituskaavaa \eqref{int-4: sijoituskaava} saadaan
\[
\int_0^2 f(x)\,dx \,=\, \int_0^1 \sqrt{x}\,dx + \int_1^2 x^2\,dx
                  \,=\, \sijoitus{0}{1}\frac{2}{3}\,x^{3/2}+\sijoitus{1}{2}\frac{1}{3}\,x^3
                  \,=\, 3. \ \loppu
\]
\begin{Exa} Määritä funktion $f(x)=\min\{3x,4-x^2\}$ integraalifunktio $\R$:ssä käyttäen
hyväksi määrättyä integraalia. \end{Exa}
\ratk Funktio $\,y(x)=\int_0^x f(t)\,dt\,$ on kysytty integraalifunktio lisäehdolla $y(0)=0$.
Koska $f(x)=3x$, kun $x\in[-4,1]$, ja $f(x)=4-x^2$, kun $x \ge 1$ tai $x \le -4$, niin pätee
\begin{align*}
x\in[-4,1]       \qimpl y(x) &= \int_0^x 3t\,dt = \sijoitus{0}{x}\frac{3}{2}\,t^2 
                              = \frac{3}{2}\,x^2, \\[1mm]
x\in[1,\infty)   \qimpl y(x) &= y(1)+[y(x)-y(1)] \\[3mm]
                             &= \frac{3}{2}+\int_1^x (4-t^2)\,dt \\
                             &= \frac{3}{2}+\sijoitus{1}{x}\left(4t-\frac{1}{3}\,t^3\right)
                              = -\frac{1}{3}\,x^3 + 4x -\frac{13}{6}\,, \\[2mm]
x\in(-\infty,-4] \qimpl y(x) &= y(-4)+[y(x)-y(-4)] \\[3mm]
                             &= 24+\int_{-4}^x (4-t^2)\,dt \\
                             &= 24+\sijoitus{-4}{x}\left(4t-\frac{1}{3}\,t^3\right)
                              = -\frac{1}{3}\,x^3 + 4x +\frac{56}{3}\,.
\end{align*}
Tässä $y(1)=3/2$ ja $y(-4)=24$ saatiin ensimmäisestä lausekkeesta. Yleinen integraalifunktio
on $F(x)=y(x)+C,\ C\in\R$, joten
\[
\int f(x)\,dx\,= \begin{cases}
                  -\frac{1}{3}\,x^3 + 4x +\frac{56}{3}+C,  &\text{kun}\,\ x < -4, \\
                 \,\frac{3}{2}\,x^2+C,                     &\text{kun}\ -4 \le x \le 1, \\
                  -\frac{1}{3}\,x^3 + 4x -\frac{13}{6}+C,  &\text{kun}\,\ x>1.
                \end{cases} \quad\loppu
\]
\begin{Exa} Sievennä lauseke $\D \,\frac{d}{dx}\int_{\sqrt{x}}^{1/\sqrt{x}}e^{-t^2},\ x>0$.
\end{Exa}
\ratk Funktiolla $e^{-t^2}$ on $\R$:ssä sarjamuotoinen integraalifunktio $F(t)$, mutta
sievennyksessä riittää tieto, että $F(t)$ on olemassa:
\begin{align*}
\frac{d}{dx}\int_{\sqrt{x}}^{1/\sqrt{x}}e^{-t^2}\, dt 
           &\,=\,\frac{d}{dx} [F(\frac{1}{\sqrt{x}})-F(\sqrt{x})] \\
           &\,=\,-\frac{1}{2x\sqrt{x}}f(\frac{1}{\sqrt{x}})-\frac{1}{2\sqrt{x}}f(\sqrt{x}) \\
           &\,=\, -\frac{1}{2\sqrt{x}}\left(\frac{e^{-1/x}}{x}+e^{-x}\right). \loppu
\end{align*}
\begin{Exa} Arvioi virhe approksimaatiossa
$\ \D\int_{10}^{20}\frac{x^4}{x^5+1}\,dx \approx \ln 2$.
\end{Exa}
\ratk Integroimisvälillä on
\[
\frac{x^4}{x^5+1}=\frac{1}{x}-\frac{1}{x(x^5+1)}\,,
\]
joten lineaarisuussäännön \eqref{int-4: lineaarisuussääntö} ja sijoituskaavan
\eqref{int-4: sijoituskaava} nojalla on
\[
\int_{10}^{20}\frac{x^4}{x^5+1}\,dx \,=\, \int_{10}^{20}\frac{1}{x}\,dx-\delta
                                   \,=\, \sijoitus{10}{20}\ln x -\delta
                                   \,=\, \ln 2 - \delta,
\]
missä on edelleen vertailuperiaatteen \eqref{int-4: vertailuperiaate} ja sijoituskaavan nojalla
\[
0 \,\le\, \delta = \int_{10}^{20}\frac{1}{x(x^5+1)}\,dx
                 \,\le\, \int_{10}^{20}\frac{1}{x^6}\,dx
                 \,=\, \sijoitus{10}{20}-\frac{1}{5x^5}
                 \,<\, 2 \cdot 10^{-6}.
\]
Siis approksimaatio on ylälikiarvo, jonka virhe on alle $2 \cdot 10^{-6}$. \loppu

\Harj
\begin{enumerate}

\item
Ratkaise probleema (P), tai osoita ratkeamattomuus, kun $[a,b]=[0,1]$ ja
\begin{align*}
&\text{a)}\,\ f(x)=\begin{cases} 
                   \,\ln x, &\text{kun}\ x>0, \\ \,0, &\text{kun}\ x=0,
                   \end{cases} \qquad
 \text{b)}\,\ f(x)=\begin{cases} 
                   \,\frac{1}{x}, &\text{kun}\ x>0, \\ \,0, &\text{kun}\ x=0,
                   \end{cases} \\
&\text{c)}\,\ f(x)=\begin{cases} 
                   \,\frac{1}{\sqrt{x}}, &\text{kun}\ x>0, \\ \,1, &\text{kun}\ x=0,
                   \end{cases} \qquad\
 \text{d)}\,\ f(x)=\begin{cases} 
                   \,\frac{1}{\sqrt{|2x-1|}}, &\text{kun}\ x \neq \frac{1}{2}, \\ 
                   \,0,                       &\text{kun}\ x=\frac{1}{2}.
                   \end{cases}
\end{align*} 

\item
Ratkaise alkuarvotehtävä $\,y'(x)=f(x),\ x>0,\ y(0)=0$ käyttämällä algoritmia A1--A3 ja
tasavälisiä jakoja välillä $[0,x]$, kun \ a) $f(x)=x^2$, \newline
b) $f(x)=x^3$, \ c) $f(x)=e^{-x}$, \ d) $f(x)=2^x$. \newline
Lisätietoja: $\,\sum_{k=1}^n k^2 = \frac{1}{6}\,n(n+1)(2n+1),\,\
\sum_{k=1}^n k^3 = \frac {1}{4}\,n^2(n+1)^2$.

\item
Laske seuraavat raja-arvot tulkitsemalla ne määrätyiksi integraaleiksi.
\begin{align*}
&\text{a)}\ \ \lim_{n\kohti\infty} \frac{1}{n^4}\left[1^3+2^3+ \ldots +(4n-1)^3\right] \qquad
 \text{b)}\ \ \lim_{n\kohti\infty} \sum_{k=1}^n \frac{1}{n+k} \\
&\text{c)}\ \ \lim_{n\kohti\infty} \frac{\pi}{n} \sum_{k=1}^{n-1} \sin\frac{k\pi}{n} \qquad
 \text{d)}\ \ \lim_{n\kohti\infty} \frac{1}{n^2} \sum_{k=0}^{n-1} \sqrt{n^2-k^2} \\
&\text{e)}\ \ \lim_{n\kohti\infty} \frac{1}{n^2} \sum_{k=0}^{n} \sqrt{n^2+k^2} \qquad
 \text{f)}\ \ \lim_{n\kohti\infty} \sum_{k=0}^n \frac{n}{n^2+k^2}
\end{align*}

\item
Laske sääntöjen \eqref{int-4: sijoituskaava}--\eqref{int-4: lineaarisuussääntö} avulla tarkasti
(jos mahdollista) tai numeerisena likiarvona:
\begin{align*}
&\text{a)}\,\ \int_{-2}^2 (x^2+3)^2\,dx \qquad
 \text{b)}\,\ \int_4^9 \left(\sqrt{x}-\frac{1}{\sqrt{x}}\right)\,dx \qquad
 \text{c)}\,\ \int_{1}^{10} \frac{1}{x^3+x}\,dx \\
&\text{d)}\,\ \int_{-1}^1 2^x\,dx \qquad
 \text{e)}\,\ \int_0^4 \abs{\sin\theta}\,d\theta \qquad
 \text{f)}\,\ \int_0^\pi \max\{\cos x,\,\sin 2x\}\,dx \\
&\text{g)}\,\ \int_1^2|x^3+x^2-3|\,dx \qquad
 \text{h)}\,\ \int_0^\pi|x-\cos x|\,dx \qquad
 \text{i)}\,\ \int_0^4 \min\{4x,\,e^x\}\,dx
\end{align*}

\item
Laske seuraavien funktioiden derivaatat vapaan muuttujan suhteen.
\begin{align*}
&\text{a)}\ \ \int_\pi^x \sin^3 t\,dt \qquad
 \text{b)}\ \ \int_x^{2\pi}(\sin^2u-u^4 e^{-u})\,du \qquad
 \text{c)}\ \ \int_{x}^{5x} \frac{e^s}{s^2+1}\,ds \\
&\text{d)}\ \ \int_0^{3\sinh 2x} \sqrt{9+t^2}\,dt \qquad
 \text{e)}\ \ \int_{-\pi}^t \frac{\cos y}{1+y^2}\,dy \qquad
 \text{f)}\ \ \int_{\sin\theta}^{\cos\theta} \sqrt{1-x^2}\,dx
\end{align*}

\item
Määritä seuraavien funktioiden integraalifunktiot $\R$:ssä käyttäen hyväksi määrättyä
integraalia:
\begin{align*}
&\text{a)}\ \ f(x)=|x|-|x-2| \qquad
 \text{b)}\ \ f(x)=|x^2-7x+10| \\
&\text{c)}\,\ f(x)=\max\,\{\,x^2+2x+3,\,9-2x-x^2\}
\end{align*}

\item
Määritä seuraavien funktioiden pienimmät arvot ja piirrä funktioiden \newline kuvaajat.
\[
\text{a)}\ \ f(x)=\int_0^1 \abs{x-t}\,dt \qquad
\text{b)}\ \ f(x)=\int_0^\pi (x\cos t-t\cos x)^2\,dt
\]

\item
Todista:
\begin{align*}
&\text{a)}\ \ 1 \le \int_0^1 \frac{1+x^{20}}{1+x^{21}} \le \frac{22}{21} \qquad
 \text{b)}\ \ \frac{2}{\sqrt[4]{e}} \le \int_0^2 e^{x^2-x}\,dx \le 2e^2 \\
&\text{c)}\ \ \int_3^5 \frac{x}{e\ln x}\,dx > 2 \qquad
 \text{d)}\ \ 0 < \int_{50}^{100} \frac{x^3}{x^6+8x+9}\,dx < 1.5 \cdot 10^{-4} \\
&\text{e)}\ \ \int_{100}^{300} \frac{x^5}{x^6+x-1}\,dx 
                               = \ln 3-\delta, \quad 0<\delta<2 \cdot 10^{-11}
\end{align*}

\item (*)
Perustele likimääräinen laskukaava
\[
\int_1^2 \frac{e^{-x}}{x}\,dx \,\approx\, \ln 2+\sum_{k=1}^n \frac{(-1)^k}{k \cdot k!}(2^k-1),
                                          \quad n\in\N,\,\ n \gg 1
\]
ja arvioi tämän virhe, kun $n=10$.

\item (*) 
Olkoon $\D F(x)=\int_0^{2x-x^2} \cos\left(\frac{1}{1+t^2}\right)\,dt, \quad
           G(x)=\int_4^{x^2} e^{t^2}\,dt$. \vspace{2mm}\newline
a) Tutki, saavuttaako $F$ jollakin $x$ pienimmän tai suurimman arvonsa. \newline
b) Laske raja-arvo $\ \lim_{x \kohti 2} G(x)/(x^3-8)$.

\item (*)
Probleemassa (P) olkoon $[a,b]=[-1,1]$, $f(0)=0$ ja $f$:n määritelmä muualla kuin origossa
\[
\text{a)}\,\ f(x)=\frac{1}{x}\left(\sin\frac{1}{x}+x\cos\frac{1}{x}\right), \quad
\text{b)}\,\ f(x)=\frac{1}{x}\left(\sin\frac{1}{x^2}+x^2\cos\frac{1}{x^2}\right).
\]
Näytä, että probleema (P) \ a) ei ratkea, \ b) ratkeaa. 

\end{enumerate} %Integraalifunktion numeerinen laskeminen. Määrätty integraali
\section{Riemannin integraali} \label{riemannin integraali}
\alku
\index{mzyzyrzy@määrätty integraali|vahv}

Tarkastellaan suljetulla välillä $[a,b]$ määriteltyä funktiota $f$, joka olkoon 
\index{rajoitettu!c@funktio}%
\kor{rajoitettu}, ts.\ on olemassa $M\in\R_+$ siten, että pätee
\[
\abs{f(x)}\leq M\quad\forall x\in [a,b].
\]
Välin $[a,b]$ yleistä 
\index{jako (osaväleihin)}%
\kor{jakoa} merkitään jatkossa symbolilla $\X$. Kuten edellisessä luvussa,
jako tarkoittaa äärellistä järjestettyä joukkoa $\X=\{x_0,x_1,\ldots,x_n\}$, missä 
$n\in\N$ ja
\[
a=x_0<x_1<\ldots <x_n=b.
\]
Jaon $\X$ keskeisin parametri on
\index{tiheysparametri}%
\kor{tiheysparametri}, joka määritellään
\[
h_{\X}=\max_{k=1\ldots n} (x_k-x_{k-1}).
\]
Jakoon $\X$ liittyen otetaan vielä käyttöön 
\index{vzy@välipiste(istö)}%
\kor{välipisteistö} $\Xi_\X=\{\xi_1,\ldots,\xi_n\}$,
missä $\xi_k\in [x_{k-1},x_k]$, $k=1\ldots n$ (muuten $\Xi_\X$ on vapaasti valittavissa).

Em. merkinnöin liitetään jokaiseen pariin $(\X,\Xi_\X)$ reaaliluku $\sigma(f,\X,\Xi_\X)$, joka
määritellään
\[
\sigma(f,\X,\Xi_\X)=\sum_{k=1}^n f(\xi_k)(x_k-x_{k-1}).
\]
Edellisen luvun tapaan käytetään raja-arvomerkintää $\Lim_{h_\X\kohti 0} \sigma(f,\X,\Xi_\X)=A$
kuvaamaan sellaista tilannetta, jossa summa lähestyy aina samaa raja-arvoa $A$ ($A\in\R$), kun
$h_\X\kohti 0$. Kuten funktion raja-arvo, myös raja-arvo 'Lim' voidaan määritellä
kahdella tavalla, joko lukujonojen avulla tai '$(\eps,\delta)$-määritelmänä', vrt.\
Määritelmä \ref{funktion raja-arvon määritelmä} ja Lause \ref{approksimaatiolause}.
Jälkimmäinen määrittelytapa muotoiltakoon jälleen lauseena. (Todistus sivuutetaan, vrt.\
Lauseen \ref{approksimaatiolause} todistus.)
\begin{Def} \label{raja-arvo Lim} \index{raja-arvo!f@määrätyn integraalin|emph}
$\ \Lim_{h_\X\kohti 0} \sigma(f,\X,\Xi_\X)=A \in\R,\ $
jos jokaiselle jonolle $\seq{\sigma(f,\X_n,\Xi_{\X_n})}$, jolle pätee $h_{\X_n} \kohti 0$,
on voimassa $\,\lim_{n\kohti\infty} \sigma(f,\X_n,\Xi_{\X_n}) = A$.
\end{Def}
\begin{*Lause} \label{Lim-kriteeri} \vahv{(Lim: $(\eps,\delta)$-kriteeri)} 
$\ \Lim_{h_\X\kohti 0} \sigma(f,\X,\Xi_\X)=A\ $
täsmälleen kun jokaisella $\eps>0$ on olemassa $\delta>0$ siten, että jokaiselle parille
$(\X,\Xi_\X)$, jolle $h_\X<\delta$, pätee
\[
\abs{\sigma(f,\X,\Xi_\X)-A}<\eps.
\]
\end{*Lause}
Määritelmän \ref{raja-arvo Lim} mukaisesti raja-arvo 'Lim' tarkoittaa kaikista mahdollisista
summista $\sigma(f,\X,\Xi)$ poimittujen, ehdon $h_{\X_n} \kohti 0$ täyttävien (ei muita ehtoja!)
\pain{luku}j\pain{ono}j\pain{en} $\seq{\sigma(f,\X_n,\Xi_{\X_n})}$ y\pain{hteistä}
\pain{ra}j\pain{a-arvoa}. Sikäli kuin tällainen yhteinen raja-arvo on olemassa, Määritelmä
\ref{raja-arvo Lim} tarjoaa myös algoritmin sen laskemiseksi: Valitaan mikä tahansa jono
tiheneviä (esim.\ tasavälisiä) jakoja $X_n$, liitetään kuhunkin $\X_n$ jokin välipisteistö 
$\Xi_n$ (esim.\ osavälien keskipisteet tai toinen päätepisteistä) ja lasketaan
$A=\lim_{n}\sigma(f,\X_n,\Xi_{\X_n})=\lim_n A_n$.
% Tässä $A_n$ on laskettavissa jokaisella
%$n\in\N$  ($n$ funktioevaluaatiota ja summaus!), joten kyseessä on toimiva algoritmi.

Edellisessä luvussa todettiin, että jos $f$:llä on välillä $(a,b)$ integraalifunktio $F$, ja $F$
on lisäksi jatkuva välillä $[a,b]$, niin jokaisella $\X$ on olemassa välipisteistö $\Xi_\X$
siten, että pätee
\[
\sigma(f,\X,\Xi_\X)=F(b)-F(a)=\sijoitus{a}{b} F(x)=\int_a^b f(x)\, dx.
\]
Edelleen näytettiin, että sikäli kuin määrätty integraali $\int_a^b f(x)\,dx$ määritellään tällä
tavoin (mainituin
oletuksin) ja lisäksi oletetaan, että $f$ on Lipschitz-jatkuva välillä $[a,b]$, niin sekä
jokaiselle jaolle $X$ että jokaiselle välipisteistölle $\Xi_\X$ pätee
(Lause \ref{summakaavalause})
\[
\sigma(f,\X,\Xi_\X)=\int_a^b f(x)\, dx + \ordoO{h_\X}.
\]
Määritelmän \ref{raja-arvo Lim} (tai Lauseen \ref{Lim-kriteeri}) perusteella todetaan, että
mainituilla (melko voimakkailla) oletuksilla pätee
\[
\Lim_{h_\X\kohti 0} \sigma(f,\X,\Xi_\X)=\int_a^b f(x)\,dx.
\]
Yleisemmin, jos $f$ mahdollisesti ei täytä mainittuja ehtoja, niin tehdään tästä laskukaavasta
integraalifunktiosta riippumaton \pain{määrät}y\pain{n} \pain{inte}g\pain{raalin}
\pain{määritelmä}. 
\begin{Def} \label{Riemannin integraali} \index{Riemannin!a@integraali|emph}
\index{Riemann-integroituvuus|emph} \vahv{(Riemannin\footnote[2]{Saksalainen matemaatikko 
\hist{Georg Friedrich Bernhard Riemann} (1826-1866) on 1800-luvun (ja kaikkienkin aikojen)
matematiikan suuria nimiä. Riemann oli 'puhdas matemaatikko' selvemmin kuin edeltäjänsä,
joista huomattavimmatkin (esim.\ Euler, Lagrange, Cauchy, Gauss) tutkivat matematiikan ohella
fysiikkaa tai muita matematiikan sovelluksia. Riemannin työt koskivat integraalien lisäksi mm.\
kompleksifunktioiden teoriaa (\kor{Riemannin pinnat}), alkulukujen jakautumista, ja geometrian
matemaattisia  perusteita. Monet Riemannin tuloksista olivat uraa uurtavia ja näyttivät suuntaa
myöhemmälle matematiikan tutkimukselle, joka 1800-luvulta lähtien erkaantui yhä selvemmin 
fysiikasta. \index{Riemann, G. F. B.|av}} integraali)} Olkoon $f$ määritelty ja rajoitettu
välillä $[a,b]$. Jos
\[
\Lim_{h_\X\kohti 0} \sigma(f,\X,\Xi_\X)=A\quad (A \in \R),
\]
niin sanotaan, että $f$ on \kor{Riemann-integroituva} (tai integroituva Riemannin mielessä)
välillä $[a,b]$. Lukua $A$ sanotaan $f$:n \kor{Riemannin integraaliksi} (Riemann-integraaliksi)
välillä $[a,b]$ ja merkitään
\[
A=\int_a^b f(x)\, dx.
\]
\end{Def}
Määritelmään liittyen sanotaan summia
\index{Riemannin!b@summa}%
$\sigma(f,\X,\Xi_\X)$ \kor{Riemannin summiksi}. Jatkossa
sanotaan yksinkertaisesti, että $f$ on \kor{integroituva} välillä $[a,b]$, jos $f$ on sekä
määritelty, rajoitettu että Riemann-integroituva ko.\ välillä .
\begin{Exa} \label{Riemann-ex1}
Olkoon $c\in\R$ ja määritellään
\[
f(x)=\begin{cases}
0, &\text{ kun } x<1/2 \\
c, &\text{ kun } x=1/2 \\
1, &\text{ kun } x>1/2
\end{cases}
\]
Tutki $f$:n integroituvuutta välillä $[0,1]$.
\end{Exa}
\ratk Jos $\X=\{x_0,\ldots ,x_n\}$, $n \ge 2$, on välin $[0,1]$ jako, niin jollakin $j\in\N$
on $x_{j-1}<1/2<x_{j+1}$. Tällöin on $f(\xi_k)=0$ kun $k\leq j-1$ ja $f(\xi_k)=1$ kun 
$k\geq j+2$, joten
\begin{align*}
\sigma(f,\X,\Xi_\X) &= \sum_{k=j}^{j+1}f(\xi_k)(x_k-x_{k-1})+\sum_{k=j+2}^n (x_k-x_{k-1}) \\
&=f(\xi_j)(x_j-x_{j-1})+f(\xi_{j+1})(x_{j+1}-x_j)+(1-x_{j+1}).
\end{align*}
Kun $h_\X \kohti 0$, niin $1-x_{j+1} \kohti 1/2$, koska $\,0<x_{j+1}-1/2<x_{j+1}-x_{j-1}<2h_\X$,
ja kolmioepäyhtälön nojalla
\begin{align*}
\abs{f(\xi_j)(x_j-x_{j-1})+f(\xi_{j+1})(x_{j+1}-x_j)} 
                     &\le \max \{\abs{c},1\}(x_{j+1}-x_{j-1}) \\
                     &\le 2\max \{\abs{c},1\} h_\X \kohti 0.
\end{align*}
Siis $f$ on integroituva välillä $[0,1]$ ja $\D\int_0^1 f(x)\, dx=\frac{1}{2}\,$. \loppu

\begin{Exa} \label{Riemann-ex2}
Dirichlet'n funktio
\[
f(x)=\begin{cases}
1, &\text{jos}\ x\in\Q \\
0, &\text{muulloin}
\end{cases}
\]
ei ole integroituva välillä $[0,1]$, sillä jos valitaan $\Xi_\X\subset\Q$, niin 
$\sigma(f,\X,\Xi_\X)=1$ ja jos valitaan $\Xi_\X\cap\Q=\emptyset$, niin $\sigma(f,\X,\Xi_\X)=0$,
olipa $\X$ mikä hyvänsä. \loppu
\end{Exa}
Esimerkeistä nähdään, että kaikki integroituvat funktiot eivät ole jatkuvia eivätkä kaikki
rajoitetut funktiot ole integroituvia. Esimerkissä \ref{Riemann-ex1} integraalin arvo ei riipu
funktion arvosta epäjatkuvuuspisteessä. --- Yleisemminkin pätee, että jos $f$ on välillä $[a,b]$
integroituva, niin integraalin arvo ei muutu, jos $f$ määritellään uudelleen äärellisen monessa
pisteessä (tai jopa suppenevassa jonossa pisteitä, ks.\ 
Harj.teht.\,\ref{H-int-5: funktion poikkeutus}). Integraali on siis 'tunnoton' funktion
yksittäisille pistearvoille samaan tapaan kuin funktion raja-arvo.

Näytetään nyt, että edellisessä luvussa todetut määrätyn integraalin keskeiset ominaisuudet,
eli additiivisuus integroimisvälin suhteen, lineaarisuus integroitavan funktion suhteen ja
integraalien vertialuperiaate, pysyvät voimassa myös Määritelmän \ref{Riemannin integraali}
mukaisille integraaleille.
\begin{Lause} \label{integraalin additiivisuus} \index{additiivisuus!a@integraalin}
\vahv{(Integraalin additiivisuus)} Jos $f$ on integroituva väleillä $[a,b]$ ja $[b,c]$
$(a<b<c)$, niin $f$ on integroituva välillä $[a,c]$, ja
\[
\int_a^c f(x)\, dx=\int_a^b f(x)\, dx+\int_b^c f(x)\, dx.
\]
\end{Lause}
\tod Tarkastellaan Riemannin summia $\sigma(f,\X,\Xi_\X)$ välillä $[a,c]$. Oletetaan aluksi,
että $b\in\X$, ja jaetaan summa kahteen osaan:
\begin{align*}
\sigma(f,\X,\Xi_\X) &= \sum_{k:\,x_k\leq b} f(\xi_k)(x_k-x_{k-1}) 
                         + \sum_{k:\,x_k> b} f(\xi_k)(x_k-x_{k-1}).
\end{align*}
Määritelmien \ref{Riemannin integraali} ja \ref{raja-arvo Lim} mukaan
\begin{align*}
\sum_{k:\,x_k\leq b} f(\xi_k)(x_k-x_{k-1}) 
                &\kohti \int_a^b f(x)\, dx, \quad \text{kun}\ h_\X \kohti 0, \\
\sum_{k:\,x_k> b} f(\xi_k)(x_k-x_{k-1})    
                &\kohti \int_b^c f(x)\, dx, \quad \text{kun}\ h_\X \kohti 0,
\end{align*}
koska $f$ oli integroituva väleillä $[a,b]$ ja $[b,c]$. Päätellään siis, että rajoituksen 
$b\in\X$ ollessa voimassa pätee
\[
h_\X\kohti 0 \ \impl \ \sigma(f,\X,\Xi_\X)\kohti\int_a^b f(x)\, dx+\int_b^c f(x)\, dx.
\]
Tarkastellaan seuraavaksi Riemannin summaa $\sigma(f,\X,\Xi_\X)$, missä $\X=\{x_k\}$ ja 
$x_{m-1}<b<x_m$ jollakin $m$ (jolloin $b\not\in\X$). Verrataan tätä summaan 
$\sigma(f,\X',\Xi_{\X'})$, missä $\X'$ on saatu $\X$:stä lisäämällä vain jakopiste $b$ ja
$\Xi_{\X'}$ on valittu siten, että välipisteet $\xi_k\in \Xi_{\X}$ ja $\xi_l'\in\Xi_{\X'}$ ovat
samat jaoille yhteisillä osaväleillä, eli jos $[x_{k-1},x_k]=[x_{l-1}',x_l']$. Tällöin on
$x_{m-1}'=x_{m-1}$, $x_m'=b$ ja $x_{m+1}'=x_m$, joten
\begin{align*}
\sigma(f,\X,\Xi_\X)&-\sigma(f,\X',\Xi_{\X'}) \\[1mm]
                   &=f(\xi_m)(x_m-x_{m-1})-f(\xi_m')(b-x_m)-f(\xi_{m+1}')(x_{m+1}-b).
\end{align*}
Jos $\abs{f(x)}\leq M \ \forall x\in [a,c]$, niin kolmioepäyhtälöä käyttäen päätellään
\begin{align*}
|\sigma(f,\X,\Xi_\X)-\sigma(f,\X',\Xi_{\X'})|
         &\,\le\, 2M(x_m-x_{m-1}) \\[1mm]
         &\,\le\, 2Mh_\X \kohti 0,\quad \text{kun}\ h_X \kohti 0.
\end{align*}
Summalla $\sigma(f,\X,\Xi_\X)$ on siis sama raja-arvo riippumatta siitä, onko $b\in\X$ tosi vai
ei. \loppu

Kun käytetään edellisessä luvussa sovittua integroimisrajojen vaihtosääntöä
\[
\int_b^a f(x)\, dx=-\int_a^b f(x)\, dx,
\]
niin nähdään, että Lauseen \ref{integraalin additiivisuus} väittämä on tosi lukujen $a,b,c$
suuruussuhteista rippumatta, kunhan $f$ on integroituva väleillä $[\min\{a,b\},\max\{a,b\}]$ ja
$[\min\{b,c\},\max\{b,c\}]$.
\begin{Lause} \label{integraalin lineaarisuus} \index{lineaarisuus!b@integroinnin|emph}
\vahv{(Integraalin lineaarisuus)} Jos $f$ ja $g$ ovat integroituvia välillä $[a,b]$, niin
$\alpha f+\beta g$ on integroituva välillä $[a,b]$ jokaisella $\alpha,\beta\in\R$ ja
\[
\int_a^b [\alpha f(x)+\beta g(x)]\,dx = \alpha\int_a^b f(x)\, dx + \beta\int_a^b g(x)\,dx.
\]
\end{Lause}
\tod Jos $\seq{X_n}$ ja $\seq{\Xi_n}$ ovat jako- ja välipisteistöt välillä $[a,b]$, niin
vastaaville Riemannin summille pätee
\[
\sigma(\alpha f+\beta g,X_n,\Xi_n)\,=\,\alpha\,\sigma(f,X_n,\Xi_n)+\beta\,\sigma(g,X_n,\Xi_n),
                                       \quad \alpha,\beta\in\R.
\]
Väitetty lineaarisuussääntö seuraa tästä Määritelmän \ref{Riemannin integraali} ja lukujonojen
raja-arvojen yhdistelysääntöjen (Lause \ref{raja-arvojen yhdistelysäännöt}) nojalla, kun
oletetaan $\lim_nh_{X_n}=0$. \loppu
\begin{Lause} \label{integraalien vertailuperiaate} \index{vertailuperiaate!a@integraalien|emph}
\vahv{(Integraalien vertailuperiaate)} Jos $f$ ja $g$ ovat integroituvia välillä $[a,b]$,
niin pätee
\[
f(x)\leq g(x)\quad\forall x\in [a,b] \ \impl \ \int_a^b f(x)\, dx\leq\int_a^b g(x)\, dx.
\]
\end{Lause}
\tod Myös tämä väittämä palautuu lukujonojen teoriaan: Vertailuperiaate on ilmeisen pätevä
Riemannin summille $\,\sigma(f,X_n,\Xi_n)\,$ ja $\,\sigma(g,X_n,\Xi_n)$, joten se pätee myös
raja-arvoille (Propositio \ref{jonotuloksia}\,[V1]). \loppu 

 
\subsection{*Riemannin ylä- ja alaintegraalit}

Määritelmä \ref{Riemannin integraali} Riemann-integroituvuudelle on sikäli konkreettiinen, että
se antaa suoraan myös integraalin numeeriseen laskemiseen soveltuvan algoritmin. Seuraavassa
lähestytään Riemannin integraalia toisella, hieman abstraktimmalla tavalla, mikä
johtaa vaihtoehtoiseen Riemann-integroituvuuden kriteeriin (Lause \ref{Riemann-integroituvuus}).
Jatkossa tämä kriteeri osoittautuu käteväksi erilaisissa teoreettisissa tarkasteluissa
(ks.\ myös Harj.teht.\,\ref{H-int-5: tasavälinen jako}--\ref{H-int-5: itseisarvon ja tulon
integroituvuus}).

Olkoon $f$ välillä $[a,b]$ määritelty ja rajoitettu funktio ja $\X = \{x_k,\ k = 0 \ldots n\}$
välin $[a,b]$ jako. Koska $f$ on rajoitettu, niin joukot 
$Y_k = \{f(x) \mid x \in [x_{k-1},x_k]\}$ ovat rajoitettuja. Siis on olemassa luvut 
(vrt.\ Luku \ref{reaalilukujen ominaisuuksia})
\[ 
\overline{f}_k = \sup Y_k, \quad \underline{f}_k = \inf Y_k, \quad k = 1 \ldots n.  
\]
Funktion $f$ jakoon $\X$ liittyviksi
\index{Riemannin!c@ylä- ja alasumma}%
\kor{Riemannin ylä- ja alasummiksi} sanotaan summia
\begin{align*}
\overline{\sigma}(f,\X)\ 
          &=\ \sum_{k=1}^n \overline{f}_k (x_k-x_{k-1}) \quad\ \text{(yläsumma)}, \\
\underline{\sigma}(f,\X)\ 
          &=\ \sum_{k=1}^n \underline{f}_k (x_k-x_{k-1}) \quad\ \text{(alasumma)}. 
\end{align*}
Huomattakoon, että jos $f$ on jatkuva välillä $[a,b]$, niin $f$ saavuttaa maksimi- ja 
minimiarvonsa jokaisella osavälillä $[x_{k-1},x_k]\subset[a,b]$ 
(Lause \ref{Weierstrassin peruslause}), jolloin $\overline{f}_k=f(\xi_k)$ ja 
$\underline{f}_k=f(\eta_k)$ joillakin $\xi_k,\eta_k\in[x_{k-1},x_k]$. Tässä tapauksessa siis
Riemannin ylä- ja alasummat ovat Riemannin summien erikoistapauksia.

Olkoon nyt $\mathcal{X}$ välin $[a,b]$ kaikkien mahdollisten jakojen $\X$ joukko. Tällöin
reaalilukujoukot $\{\overline{\sigma}(f,\X) \mid \X \in \mathcal{X}\}$ ja 
$\{\underline{\sigma}(f,\X) \mid \X \in \mathcal{X}\}$ ovat rajoitettuja, sillä jos
$\abs{f(x)} \le M,\ x \in [a,b]$, niin $\abs{\overline{f}_k} \le M$ ja 
$\abs{\underline{f}_k} \le M\ \ \forall k$, jolloin
\[ 
\abs{\overline{\sigma}(f,\X)},\ \abs{\underline{\sigma}(f,\X)}\ 
                       \le\ M \sum_{k=1}^n (x_k-x_{k-1})\ =\ M(b-a). 
\]
Mainituilla lukujoukoilla on siis sekä supremum että infimum.
\begin{Def} \label{ylä- ja alaintegraali} \index{Riemannin!d@ylä- ja alaintegraali|emph}
Jos $f$ on määritelty ja rajoitettu välillä $[a,b]$ ja
$\mathcal{X}$ on välin $[a,b]$ kaikkien mahdollisten jakojen $\X$ joukko, niin funktion $f$ 
\kor{Riemannin ylä- ja alaintegraalit} välillä $[a,b]$ määritellään
\begin{align*} \label{ylä- ja alaintegraalit}
\overline{\int_a^b} f(x)\,dx 
       &= \inf_{\X \in \mathcal{X}} \overline{\sigma}(f,\X) \quad\ \text{(yläintegraali)}, \\
\underline{\int_a^b} f(x)\,dx 
       &= \sup_{\X \in \mathcal{X}} \underline{\sigma}(f,\X) \quad\ \text{(alaintegraali)}.
\end{align*}
\end{Def}
Määritelmän \ref{ylä- ja alaintegraali} mukaisesti siis jokainen välillä $[a,b]$ määritelty ja
rajoitettu funktio on ko.\ välillä sekä 'yläintegroituva' että 'alaintegroituva'.
\begin{Exa} Jos $f(x)=1$, kun $x\in\Q$, ja $f(x)=0$ muulloin, niin
\[ 
\overline{\int_0^1} f(x)\, dx = 1, \quad \underline{\int_0^1} f(x)\,dx = 0. \loppu
\]
\end{Exa}
Esimerkki johdattelee seuraavaan hyvin eleganttiin Riemann-integroituvuuden kriteeriin.
Todistus esitetään luvun lopussa.
\begin{*Lause} \label{Riemann-integroituvuus} \vahv{(Riemann-integroituvuus)} 
\index{Riemann-integroituvuus|emph}
Välillä $[a,b]$ määritelty ja rajoitettu funktio $f$ on ko.\ välillä Riemann-integroituva
täsmälleen kun $f$:n ylä- ja alaintegraalit ko.\ välillä ovat samat, jolloin $f$:n Riemannin
integraali $=$ ylä- ja alaintegraalien yhteinen arvo. 
\end{*Lause}

Seuraavassa esitetään kaksi Riemannin ylä- ja alasummia koskevaa väittämää, joilla on
teoreettisissa tarkasteluissa käyttöä yhdessä Lauseen \ref{Riemann-integroituvuus} kanssa.
Ensinnäkin näytetään, että jaon
\index{tihennys (jaon)}%
\kor{tihennyksessä} yläsumma pienenee tai pysyy samana ja
vastaavasti alasumma suurenee tai pysyy samana.
\begin{Lause} \label{jaon tihennys} Jos $X$ ja $X'$ ovat välin $[a,b]$ jakoja ja $X' \supset X$ 
(eli $X'$ on $X$:n tihennys), niin
\[
\underline{\sigma}(f,X) \le \underline{\sigma}(f,X') \le
\overline{\sigma}(f,X') \le \overline{\sigma}(f,X).
\]
\end{Lause}
\tod Tihennys voidaan ajatella suoritetuksi lisäämällä jakoon $X$ yksi piste kerrallaan,
kunnes päädytään jakoon $X'$. Väittämä on tosi, jos se on tosi jokaiselle tällaiselle
osatihennykselle. Olkoon siis $X = \{x_k,\ k = 0 \ldots n\}$ ja olkoon $X'=X\cup\{x_k'\}$,
missä $x_k' \in (x_{k-1},x_k)$. Tällöin jos $\overline{f}_k$ ja $\underline{f}_k$ määritellään
kuten edellä (jakoon $X$ liittyen) ja merkitään
\[
\alpha_k = \sup_{x\in[x_{k-1},\,x_k']} f(x), \quad \beta_k = \sup_{x\in[x_k',\,x_k]} f(x),
\]
niin $\alpha_k \le \overline{f}_k$ ja $\beta_k \le \overline{f}_k$
(vrt.\ Harj.teht.\,\ref{reaalilukujen ominaisuuksia}:\ref{H-I-11: sup ja inf}c), joten
\[
\alpha_k(x_k'-x_{k-1})+\beta_k(x_k-x_k')\,\le\,\overline{f}_k(x_k-x_{k-1}).
\]
Koska jaot ovat samat välin $[x_{k-1},x_k]$ ulkopuolella, niin päätellään, että on oltava
$\overline{\sigma}(f,X') \le \overline{\sigma}(f,X)$. Vastaavalla tavalla näytetään, että
$\underline{\sigma}(f,X) \le \underline{\sigma}(f,X')$. Ylä- ja alasummien määritelmän 
perusteella on myös $\underline{\sigma}(f,X') \le \overline{\sigma}(f,X')$, joten väite 
seuraa. \loppu

Seuraava väittämä konkretisoi ylä- ja alaintegraalit lukujonojen avulla. Väittämä nojaa
oleellisesti edelliseen.  
\begin{*Lause} \label{ylä- ja alaintegraalit raja-arvoina} Olkoon $f$ määritelty ja rajoitettu
välillä $[a,b]$. Tällöin jokaiselle jonolle $\seq{X_n}$ välin $[a,b]$ jakoja, jolle
$h_{X_n} \kohti 0$, pätee
\[
\lim_n \overline{\sigma}(f,X_n)=\overline{\int_a^b} f(x)\,dx, \qquad
\lim_n \underline{\sigma}(f,X_n)=\underline{\int_a^b} f(x)\,dx.
\]
\end{*Lause}
\tod Olkoon $\seq{\X_n}$ oletusten mukainen jono ja olkoon $\eps>0$. Yläintegraalin määritelmän
nojalla on olemassa välin $[a,b]$ jako $X = \{x_k,\ k = 0 \ldots K\}$ siten, että
\[
0\ \le\ \overline{\sigma}(f,X)-\overline{\int_a^b} f(x)\,dx\ <\ \frac{\eps}{2}\,.
\]
Merktitään $X_n'=X_n \cup X$, jolloin $X_n'$ on jakojen $X_n$ ja $X$ tihennys. 
Tihennyksessä $X_n \ext X_n'$ jakautuvat jaon $X_n$ synnyttämistä osaväleistä täsmälleen ne,
joiden sisäpisteenä on ainakin yksi joukon $X$ piste. Tällaisia osavälejä on enintään $K-1$, 
ja näistä kunkin pituus on enintään $h_{X_n}$. Koska vain nämä osavälit vaikuttavat erotukseen
$\overline{\sigma}(f,X_n)-\overline{\sigma}(f,X_n')$, niin merkitsemällä mainittujen osavälien
osuuksia summissa $\overline{\sigma}(f,X_n)$ ja $\overline{\sigma}(f,X_n')$ symboleilla
$\overline{\sigma}_0(f,X_n)$ ja $\overline{\sigma}_0(f,X_n')$ ja olettaen, että
$|f(x)| \le M,\ x\in[a,b]$, voidaan Lauseen \ref{jaon tihennys} ja kolmioepäyhtälöm perusteella
arvioida
\begin{align*}
0\le\overline{\sigma}(f,X_n)-\overline{\sigma}(f,X_n')
 &\,=\,\left|\overline{\sigma}_0(f,X_n)-\overline{\sigma}_0(f,X_n')\right| \\ 
 &\,\le\,\left|\overline{\sigma}_0(f,X_n)\right|+\left|\overline{\sigma}_0(f,X_n')\right|
  \,\le\, (K-1) \cdot 2M \cdot h_{X_n}\,.
\end{align*}
Toisaalta koska $X_n'$ on myös $X$:n tihennys, niin pätee (Lause \ref{jaon tihennys})
\[
\overline{\sigma}(f,X_n')\le\overline{\sigma}(f,X).
\] 
Yhdistämällä epäyhtälöt seuraa
\begin{align*}
\overline{\int_a^b} f(x)\,dx\,
                \le\,\overline{\sigma}(f,X_n)\,
               &\le\,\overline{\sigma}(f,X_n')+2M(K-1)\,h_{X_n} \\[2mm]
               &\le\,\overline{\sigma}(f,X)+2M(K-1)\,h_{X_n} \\
               &<\,\overline{\int_a^b} f(x)\,dx +\frac{\eps}{2}+2M(K-1)\,h_{X_n}\,.
\end{align*}
Koska $h_{X_n} \kohti 0$ kun $n\kohti\infty$ ja koska $K$ on $n$:stä riippumaton, niin tässä 
on $\,2M(K-1\,h_{X_n}<\eps/2\,$ jostakin indeksistä $N$ eteenpäin, jolloin seuraa
\[
\overline{\int_a^b} f(x)\,dx\,\le\,\overline{\sigma}(f,X_n)\,
                                <\,\overline{\int_a^b} f(x)\,dx + \eps, \quad n>N.
\]
Tässä $\eps>0$ oli mielivaltainen ja $N\in\N$, joten lukujonon suppenemisen määritelmän nojalla
on todistettu väittämän yläsummia koskeva osa. Alasummia koskeva osaväittämä todistetaan 
vastaavalla tavalla. \loppu

Esimerkkinä Lauseiden \ref{Riemann-integroituvuus} ja \ref{ylä- ja alaintegraalit raja-arvoina}
soveltamisesta todistettakoon
\begin{Lause} \label{integroituvuus osavälillä} \vahv{(Integroituvuus osavälillä)}
Jos välillä $[a,b]$ määritelty ja rajoitettu funktio $f$ on ko.\ välillä  Riemann-integroituva,
niin $f$ on Riemann-integroituva myös jokaisella osavälillä $[c,d]\subset[a,b]$.
\end{Lause}
\tod Olkoon $\seq{X_n}$ jono välin jakoja siten, että $h_{X_n} \kohti 0$ ja
$c,d \in X_n\ \forall n$. Tällöin $X_n$ sisältää jokaisella $n$ välin $[c,d]$ jaon $X_n'$,
jolle $h_{X_n'} \le h_{X_n} \kohti 0$. Kun merkitään $X_n=$ $\{x_k,\ k=0 \ldots n\}$ ja
$X_n'=\{x_k,\ k=l-1 \ldots m\}$, niin
\begin{align*}
\overline{\sigma}(f,X_n)-\underline{\sigma}(f,X_n)
&\,=\, \sum_{k=1}^n (\overline{f}_k-\underline{f}_k)(x_k-x_{k-1})
 \,\ge\, \sum_{k=l}^m (\overline{f}_k-\underline{f}_k)(x_k-x_{k-1}) \\
&\,=\, \overline{\sigma}(f,X_n')-\underline{\sigma}(f,X_n') \,\ge\,0.
\end{align*}
Koska tässä pätee oletusten ja Lauseiden \ref{Riemann-integroituvuus} ja 
\ref{ylä- ja alaintegraalit raja-arvoina} perusteella 
\[
\overline{\sigma}(f,X_n)-\underline{\sigma}(f,X_n) 
\kohti \overline{\int_a^b} f(x)\,dx - \underline{\int_a^b} f(x)\,dx = 0,
\]
niin $\,\overline{\sigma}(f,X_n')-\underline{\sigma}(f,X_n') \kohti 0$ (Propositio
\ref{jonotuloksia} [V2]). Koska myös $h_{X_n'} \kohti 0$, niin Lauseen
\ref{ylä- ja alaintegraalit raja-arvoina} perusteella $f$:n ylä- ja alaintegraalit välillä 
$[c,d]$ ovat samat, eli $f$ on Riemann-integroituva ko.\ välillä
(Lause \ref{Riemann-integroituvuus}). \loppu

\vahv{Lauseen \ref{Riemann-integroituvuus} todistus}. \ \fbox{$\impl$} Oletetaan, että $f$ on
Riemann-integroituva välillä $[a,b]$ Määritelmän \ref{Riemannin integraali} mukaisesti. Olkoon
$\seq{X_n}$ jono välin $[a,b]$ jakoja, jolle $h_{X_n} \kohti 0$. Valitaan indeksi $n$ ja
merkitään $X_n = \{x_k,\ k = 0 \ldots K\}$. Nyt voidaan jokaisella $k = 1 \ldots K$ valita
$\xi_k,\xi_k'\in[x_{k-1},x_k]$ siten, että pätee
\begin{align*}
0\,&\le\,\overline{f}_k-f(\xi_k)\,<\,2^{-n}, \quad\ 
                \overline{f}_k=\sup_{x\in[x_{k-1},\,x_k]} f(x), \\
0\,&\le\,f(\xi_k')-\underline{f}_k\,<\,2^{-n}, \quad\ 
                \underline{f}_k=\inf_{x\in[x_{k-1},\,x_k]} f(x).
\end{align*}
Tällöin
\begin{align*}
0\,&\le\,\overline{\sigma}(f,X_n)-\sigma(f,\Xi_n,X_n)\,
                                <\,2^{-n}(b-a), \quad\ \Xi_n=\{\xi_k\}, \\
0\,&\le\,\sigma(f,\Xi_n',X_n)-\underline{\sigma}(f,X_n)\,
                                <\,2^{-n}(b-a), \quad\ \Xi_n'=\{\xi_k'\}.
\end{align*}
Koska tässä $\sigma(f,\Xi_n,X_n) \kohti A\in\R$ ja $\sigma(f,\Xi_n',X_n) \kohti A$ oletuksen
mukaan, niin seuraa, että myös $\overline{\sigma}(f,X_n) \kohti A$ ja 
$\underline{\sigma}(f,X_n) \kohti A$. Lauseen \ref{ylä- ja alaintegraalit raja-arvoina} mukaan
$A$ on tällöin $f$:n ylä- ja alaintegraalien yhteinen arvo.

\fbox{$\Leftarrow$} Oletetetaan, että $f$:n ylä- ja alaintegraaleilla välillä $[a,b]$ on
yhteinen arvo $A$. Tällöin jos $\seq{X_n}$ on jono välin $[a,b]$ jakoja ja $h_{X_n} \kohti 0$,
niin Lauseen \ref{ylä- ja alaintegraalit raja-arvoina} perusteella 
$\overline{\sigma}(f,X_n) \kohti A$ ja $\underline{\sigma}(f,X_n) \kohti A$. Tällöin myös 
$\sigma(f,\Xi_{\X_n},X_n) \kohti A\ \forall\,\Xi_{X_n}$, koska 
$\underline{\sigma}(f,X_n)\le\sigma(f,\Xi_{X_n},X_n)\le\overline{\sigma}(f,X_n)$
(Propositio \ref{jonotuloksia}\,[V2]). Siis $f$ on integroituva Määritelmän
\ref{Riemannin integraali} mukaisesti. \loppu

\Harj
\begin{enumerate}

\item
a) Olkoon $f(x)=x^2$, kun $x\in\Z$, ja $f(x)=0$, kun $x\in\R,\ x\not\in\Z$. Näytä, että $f$ on
integroituva jokaisella välillä $[a,b]\subset\R$ ja että $\int_a^b f(x)\,dx=0$. \newline
b) Olkoon $f(x)=1$, kun $x\in\{\,2^{-k},\ k=0,1,\ldots\,\}$ ja $f(x)=0$ muulloin. Laskemalla
yläsummat $\overline{\sigma}(f,X_n)$, missä $X_n$ on välin $[0,1]$ tasavälinen jako $2^n$
osaväliin, näytä, että $\int_0^1 f(x)\,dx=0$.

\item
a) Todista: Jos $f$ on integroituva välillä $[-a,a]$, niin pätee
\begin{align*}
&f\,\ \text{parillinen}   \qimpl \int_{-a}^a f(x)\,dx = 2\int_0^a f(x)\,dx, \\
&f\,\ \text{pariton} \quad\qimpl \int_{-a}^a f(x)\,dx =0.
\end{align*}
b) Todista: Jos $f$ on $L$-jaksoinen ja interoituva välillä $[0,L]$, niin $f$ on integroituva
välillä $[a,a+L]$ jokaisella $a\in\R$ ja
\[
\int_{a}^{a+L} f(x)\,dx=\int_0^L f(x)\,dx.
\]

\item \label{H-int-5: Riemann-1}
Olkoon
$\ \D f(x)= \begin{cases}
            1/x, &\text{kun}\ x>0 \\ 0, &\text{kun}\ x=0
\end{cases}$ \vspace{1mm}\newline
a) Totea, että $f$ on määritelty välillä $[0,1]$ mutta ei rajoitettu. \newline
b) Näytä, että $f$ ei ole Riemann-integroituva välillä $[0,1]$ konstruoimalla jono Riemannin
summia, jolle pätee $\,h_{\X_n} \kohti 0\,$ ja $\,\sigma(\X_n,\Xi_{X_n}) \kohti \infty$.

\item
Näytä, että jos $f$ on integroituva jokaisella välillä $[a,b]\subset\R$ ja integraali jokaisen
välin yli $=0$, niin $f(x)=0$ jokaisessa pisteessä, jossa $f$ on joko vasemmalta tai oikealta
jatkuva.

\item %\label{H-int-5: väittämiä}
Näytä Proposition \ref{jaon tihennys} avulla, että mikään Riemannin alasumma ei voi olla
suurempi kuin yläsumma, ts.\ kaikille välin $[a,b]$ jaoille $\X,\X'$ pätee
$\underline{\sigma}(f,\X)\le\overline{\sigma}(f,\X')$.
%b) Todista Korollaari \ref{integroituvuuskorollaari}. 

\item (*) \label{H-int-5: tasavälinen jako}
Näytä, että Määritelmä \ref{Riemannin integraali} ei muutu, jos välin $[a,b]$ jaot $X_n$
raja-arvossa 'Lim' (Määritelmä \ref{raja-arvo Lim}) rajoitetaan tasavälisiksi. Voidaanko myös
välipisteistöjen $\Xi_{\X_n}$ valintaa rajoittaa (esim.\ $\Xi_{\X_n}\subset\X_n$)
Riemann-integroituvuuden määritelmän muuttumatta?

\item (*) \label{H-int-5: Riemann-2}
Näytä, että jos $f$ on määritelty välillä $[a,b]$ mutta ei ole rajoitettu, niin on olemassa
jono Riemannin summia, jolle pätee $h_{\X_n} \kohti 0$ ja
$\left|\sigma(\X_n,\Xi_{X_n})\right| \kohti \infty$ (vrt.\ Harj.teht. \ref{H-int-5: Riemann-1}).
--- Mikä Määritelmän \ref{Riemannin integraali} oletus on siis tarpeeton?

\item (*) \label{H-int-5: funktion poikkeutus}
Olkoon $X$ suppenevan reaalilukujonon termeistä muodostettu (äärellinen tai numeroituva) joukko,
olkoon $f$ määritelty ja rajoitettu välillä $[a,b]$ ja olkoon $g$ määritelty välillä $[a,b]$
siten, että $g(x)=f(x)$, kun $x \not\in X$. Joukosta $\{g(x) \mid x\in X\cap[a,b]\}\subset\R$ 
tiedetään ainoastaan, että se on rajoitettu. Näytä, että jos $f$ on välillä $[a,b]$ 
Riemann-integroituva, niin samoin on $g$ ja $\int_a^b g(x)\,dx=\int_a^b f(x)\,dx$.

\item (*) \label{H-int-5: monotonisuus ja integroituvuus}
a) Olkoon $f$ määritelty ja rajoitettu välillä $[a,b]$ ja $\mathcal{X}$ välin $[a,b]$ jakojen
joukko. Liitetään jokaiseen $\X = \{x_k,\ k = 0 \ldots n\} \in \mathcal{X}$ luku
\[
\delta(f,\X)=\sum_{k=1}^n (\overline{f}_k-\underline{f}_k),
\]
missä $\overline{f}_k$ ja $\underline{f}_k$ ovat $f$:n supremum ja infimum osavälillä 
$[x_{k-1},x_k]$. Näytä, että jos joukko $\{\delta(f,\X)\,|\,\X \in \mathcal{X}\}$ on 
rajoitettu, niin $f$ on Riemann-integroituva välillä $[a,b]$. \vspace{1mm}\newline
b) Näytä  että jos $f$ on määritelty ja monotoninen (kasvava tai vähenevä) välillä $[a,b]$, niin
$f$ on Riemann-integroituva välillä $[a,b]$.

\item (*) \label{H-int-5: itseisarvon ja tulon integroituvuus}
a) Näytä, että jos $f$ on määritelty, rajoitettu ja Riemann-integroituva välillä $[a,b]$, niin
myös seuraavat funktiot ovat Riemann-integroituvia välillä $[a,b]$:
\[
f_+(x)=\max\{f(x),0\}, \quad f_-(x)=\min\{f(x),0\}.
\]
b) Näytä, että seuraavista väittämistä ensimmäinen on tosi, toinen epätosi:
\begin{align*}
f\ \ \text{Riemann-integroituva}       &\qimpl \abs{f}\,\ \text{Riemann-integroituva}, \\
\abs{f}\,\ \text{Riemann-integroituva} &\qimpl \,f\,\ \ \text{Riemann-integroituva}.
\end{align*}
c) Näytä, että jos $f$ ja $g$ ovat välillä $[a,b]$ määriteltyjä, rajoitettuja ja
Riemann-integroituvia, niin samoin on tulo $fg$. \kor{Vihje}: Kirjoita $f=f_++f_-$,
$g=g_++g_-$ (ks.\ a)-kohta).

\end{enumerate} %Riemannin integraali
\section{Analyysin peruslause} \label{analyysin peruslause}
\alku
Palataan Luvussa \ref{määrätty integraali} tarkasteltuun probleemaan (P), jossa etsittiin
annetun funktion $f$ integraalifunktiota $y(x)$ välillä $(a,b)$ lisäehdoilla, että $y(x)$ on
jatkuva välillä $[a,b]$ ja $y(a)=0$. Kysymys probleeman ratkeavuudesta jätettiin tuolloin
avoimeksi. Näytetään nyt, että riittävä ehto ratkeavuudelle on, että funktio $f$ on
 j\pain{atkuva} \pain{välillä} $[a,b]$. Kyseessä on eräs matemaattisen analyysin
perustavimmista tuloksista, ja se tunnetaankin nimellä
\kor{Analyysin peruslause}\footnote[2]{Englanninkielinen nimi on 'The Fundamental Theorem of
Calculus'. Lauseen muotoilut kirjallisuudessa ovat vaihtelevia. --- Usein peruslauseena
esitetään vain Lauseen \ref{Analyysin peruslause} toinen osaväittämä, jolloin lause jää
sisällöltään melko kevyeksi, ks.\ Lause \ref{Analyysin peruslause 2} jäljempänä.}.
\begin{*Lause} \label{Analyysin peruslause} \vahv{(Analyysin peruslause)}
\index{Analyysin peruslause|emph} \index{Riemann-integroituvuus!b@jatkuvan funktion|emph}
Jos $f$ on jatkuva välillä $[a,b]$, niin
\begin{enumerate}
\item $f$ on Riemann-integroituva välillä $[a,b]$.
\item Funktio $F(x)=\int_a^x f(t)\, dt$ on jatkuvasti derivoituva välillä $[a,b]\,$ ja \\
      $F'(x)=f(x)$ ko.\ välillä.
\end{enumerate}
\end{*Lause}
Seuraavassa todistetaan Lauseen \ref{Analyysin peruslause} ensimmäinen (vaativampi) osaväittämä
heikennetyssä muodossa, uudistamalla Luvussa \ref{määrätty integraali} tehty Lipschitz-oletus
\[
\abs{f(x_1)-f(x_2)}\leq L\abs{x_1-x_2}\quad\forall x_1,x_2\in [a,b].
\]
\begin{Lause} \label{Analyysin peruslause 1} Välillä $[a,b]$ Lipschitz-jatkuva funktio on
Riemann-integroituva.% ko.\ välillä.
\end{Lause}
\tod Olkoon $X_n=\{x_k,\ k=0 \ldots n\}$ välin $[a,b]$ jako. Koska $f$ on (ainakin) jatkuva
välillä $[a,b]$, niin $f$ saavuttaa väleillä $[x_{k-1},x_k]$ maksimi- ja minimi-\linebreak
arvonsa (Lause\ref{Weierstrassin peruslause}), joten jakoon $X_n$ liittyvät Riemannin ylä- ja
alasummat \linebreak (ks.\ edellinen luku) voidaan kirjoittaa
\[
\overline{\sigma}(f,X_n)=\sum_{k=1}^n f(\xi_k)(x_k-x_{k-1}), \quad\,\
\underline{\sigma}(f,X_n)=\sum_{k=1}^n f(\eta_k)(x_k-x_{k-1}),
\]
missä $\xi_k,\eta_k\in[x_{k-1},x_k]$. Tämän ja oletetun Lipschitz-jatkuvuuden perusteella on
\begin{align*}
0 \,\le\, \overline{\sigma}(f,X_n)-\underline{\sigma}(f,X_n)
 &\,=\, \sum_{k=1}^n[f(\xi_k)-f(\eta_k)](x_k-x_{k-1}) \\
 &\,\le\, \sum_{k=1}^n L|\xi_k-\eta_k|(x_k-x_{k-1}) \\
 &\,\le\, \sum_{k=1}^n Lh_{X_n}(x_k-x_{k-1}) \,=\, L(b-a)h_{X_n}.
\end{align*}
On päätelty, että jos $\seq{X_n}$ on jono jakoja, jolle $h_{X_n} \kohti 0$, niin
\[
\overline{\sigma}(f,X_n)-\underline{\sigma}(f,X_n)\,=\,\mathcal{O}(h_{X_n})\,\kohti\,0.
\]
Tämän ja Lauseiden \ref{ylä- ja alaintegraalit raja-arvoina} ja \ref{Riemann-integroituvuus}
perusteella seuraa väite. \loppu

Analyysin peruslauseen 1.\ väittämän mukaisesti Lause \ref{Analyysin peruslause 1} on tosi myös,
jos $f$ on pelkästään jatkuva välillä $[a,b]$. Todistuksen pääidea on tässäkin tapauksessa sama,
mutta todistuksen on nojattava syvällisempään jatkuvuuden logiikkaan, tarkemmin sanoen
\pain{tasaiseen} j\pain{atkuvuuteen} (ks.\ Luku \ref{jatkuvuuden logiikka}). Todettakoon tässä
ainoastaan, että pelkän jatkuvuusoletuksen perusteella em.\ todistuskonstruktion
loppupäätelmäksi tulee
\[
\overline{\sigma}(f,X_n)-\underline{\sigma}(f,X_n) = o(1)
                  \,\kohti\,0, \quad \text{kun}\ h_{X_n} \kohti 0.
\]
Väitetty integroituvuus (Analyysin peruslauseen 1.\ väittämä) tulee siis näinkin todistetuksi.

Analyysin peruslauseen 2. väittämän todistamiseksi tarkastellaan (sovelluksiakin silmällä
pitäen, ks.\ esimerkki jäljempänä) yleisempää tilannetta, jossa funktio $f$ on integroituva 
välillä $[a,b]$ mutta ei välttämättä jatkuva koko välillä $[a,b]$. 
Tällöinkin funktio $F(x) = \int_a^x f(t)\,dt$ on määritelty koko välillä $[a,b]$ 
(Lause \ref{integroituvuus osavälillä}). Seuraavan lauseen toinen väittämä todistaa Lauseen
\ref{Analyysin peruslause} toisen osaväittämän, joten Analyysin peruslause tulee samalla
kokonaaan todistetuksi.
\begin{Lause} \label{Analyysin peruslause 2} Olkoon $f$ määritelty, rajoitettu ja
Riemann-integroituva välillä $[a,b]$ ja $F(x)=\int_a^x f(t)\,dt$, $\,x\in[a,b]$. Tällöin pätee:
\begin{enumerate}
\item Jos $\abs{f(x)}\leq M \ \forall x\in [a,b]$, niin $F$ on välillä $[a,b]$ Lipschitz-jatkuva
      vakiolla $L=M$.
\item Jos $f$:llä on oikeanpuoleinen raja-arvo $f(x^+)$ pisteessä $x\in[a,b)$, niin $F$ on 
      pisteessä $x$ oikealta derivoituva ja $D_+F(x)=f(x^+)$. Vastaavasti jos $f$:llä on 
      vasemmanpuoleinen raja-arvo $f(x^-)$ pisteessä $x\in(a,b]$, niin $F$ on vasemmalta 
      derivoituva pisteessä $x$ ja $D_-F(x)=f(x^-)$.
\end{enumerate}
\end{Lause}
\tod \ 1. Jos $\,-M \le f(x) \le M,\ x\in[a,b]$, niin integraalien vertailuperiaatteen
(Lause \ref{integraalien vertailuperiaate}) nojalla pätee jokaisella osavälillä 
$[x_1,x_2]\subset[a,b]\ (x_1<x_2)$
\[
-M(x_2-x_1) = \int_{x_1}^{x_2} (-M)\,dt \le \int_{x_1}^{x_2} f(t)\,dt 
              \le \int_{x_1}^{x_2} M\,dt = M(x_2-x_1).
\]
Integraalin additiivisuuden (Lause \ref{integraalin additiivisuus}) perusteella seuraa
\[
\abs{F(x_2)-F(x_1)} = \left|\int_{x_1}^{x_2} f(t)\,dt\right| 
                      \le M \abs{x_1-x_2}\ \ \forall x_1,x_2\in[a,b]\ \impl\ \text{väite 1.}
\]

2. Olkoon oikeanpuoleinen raja-arvo $f(x^+)$ olemassa pisteessä $x\in[a,b)$ ja olkoon $\eps>0$.
Tällöin on olemassa $\delta\in(0,b-x]$ siten, että $\abs{f(t)-f(x^+)}<\eps$, kun 
$t\in[x,x+\delta)$ (Lause \ref{approksimaatiolause}). Tällöin jos $0<\Delta x<\delta$, niin
integraalin lineaarisuuden, additiivisuuden ja vertailuperiaatteen nojalla
\begin{multline*}
\left|\frac{F(x+\Delta x)-F(x)}{\Delta x}-f(x^+)\right| 
               \,=\, \left|\frac{1}{\Delta x}\int_x^{x+\Delta x} [f(t)-f(x^+)]\,dt\right| \\
               \,\le\, \frac{1}{\Delta x}\int_x^{x+\Delta x} |f(t)-f(x^+)|\,dt
               \,\le\, \frac{1}{\Delta x}\int_x^{x+\Delta x} \eps\,dt \,=\,\eps.
\end{multline*}
Koska tässä $\eps>0$ oli mielivaltainen ja epäyhtälö on pätevä jokaisella 
$\Delta x\in(0,\delta)$, missä $\delta>0$, niin $F$ on määritelmän mukaan oikealta derivoituva
pisteessä $x$ ja $D_+F(x)=f(x^+)$. Väittämän toinen osa todistetaan vastaavasti. \loppu
\begin{Exa}: \vahv{Liikelaki}. \index{zza@\sov!Liikelaki} \label{liikelaki} \ Jos kappaleen
(massa = $m$) suoraviivaisessa liikkeessä vaikuttaa liikesuuntaan voima $f(t)$ ($t$=aika),
niin \pain{liikemäärän} \pain{säil}y\pain{mislaki} aikavälillä $[0,\infty)$ on
\[
mv(t)-mv(0)= \int_0^t f(t')dt',
\]
missä $v(t)=$ kappaleen nopeus hetkellä $t$ ja oikealla puolella oleva (Riemannin) integraali
on voiman $f$ \pain{im}p\pain{ulssi} aikavälillä $[0,t]$. Jos $f$ on jatkuva pisteessä $t>0$,
niin Lauseen \ref{Analyysin peruslause 2} perusteella liikemäärän säilymislaista seuraa
(puolittain derivoimalla) \pain{liike}y\pain{htälö}
\[
mv'(t)=f(t).
\]
Jos $f$ on pisteessä $t$ epäjatkuva (fysikaalisesti mahdollista!), ei liikeyhtälö ole (yleensä)
voimassa ko.\ hetkellä. Liikemäärän säilymislaki sen sijaan säilyttää pätevyytensä, edellyttäen
ainoastaan, että $f$ on (Riemann-)integroituva aikaväleillä $[0,t]$. Liikemäärän
säilymislakia voidaan näin ollen pitää 'alkuperäisenä' fysiikan lakina ja liikeyhtälöä 
pikemminkin tämän seuraamuksena mainitun lisäoletuksen ($f$ jatkuva pisteessä $t$) vallitessa. 
Esimerkiksi jos $v(0)=v_0$ ja
\[
f(t)=\begin{cases}
F,   &\text{ kun } t\in [t_1,t_2], \\
\,0, &\text{ muuten},
\end{cases}
\]
missä $F \neq 0$ on vakio ja $0<t_1<t_2\,$, niin liikemäärän säilymislain mukainen
(fysikaalisesti oikea!) ratkaisu on
\begin{align*}
v(t) = \begin{cases}
v_0,                         &\text{kun}\,\ t\in [0,t_1], \\[3mm]
v_0+\dfrac{F}{m}\,(t-t_1),   &\text{kun}\,\ t\in [t_1,t_2], \\[3mm]
v_0+\dfrac{F}{m}\,(t_2-t_1), &\text{kun}\,\ t\in [t_2,\infty).
\end{cases}
\end{align*}
Liikeyhtälö $mv'(t)=f(t)$ on tässä tapauksessa voimassa väleillä $(0,t_1)$, $(t_1,t_2)$ ja
$(t_2,\infty)$. Pisteissä $t_1$ ja $t_2$ nopeus $v(t)$ on Lauseen \ref{Analyysin peruslause 2}
mukaisesti jatkuva (koska $f$ on rajoitettu) ja sekä vasemmalta että oikealta derivoituva.
Koska $f$:llä on näissä
pisteissä hyppyepäjatkuvuus, ovat toispuoliset derivaatat erisuuret, ja näin ollen $v(t)$ ei
ole derivoituva pisteissä $t_1$ ja $t_2$. \loppu
\end{Exa}
\begin{figure}[H]
\setlength{\unitlength}{1cm}
\begin{center}
\begin{picture}(10,3)(-1,-0.4)
\put(-1,0){\vector(1,0){10}} \put(8.8,-0.5){$t$}
\put(0,-1){\vector(0,1){4}} \put(0.2,2.8){$v(t)$}
\path(0,1)(3,1)(6,2)(9,2) 
\put(0,1){\line(-1,0){0.1}} \put(-0.5,0.9){$v_0$}
\put(3,0){\line(0,-1){0.1}} \put(2.9,-0.5){$t_1$}
\put(6,0){\line(0,-1){0.1}} \put(5.9,-0.5){$t_2$}
\end{picture}
\end{center}
\end{figure}

\subsection{Funktion $\,\int_{g_1(x)}^{g_2(x)} f(t)\,dt$ derivaatta}

Integraalin additiivisuuden nojalla otsikon funktio voidaan esittää muodossa \linebreak
$\,F(g_2(x))-F(g_1(x))$, missä $F(x)=\int_a^x f(t)\,dt$. Yhdistetyn funktion derivoimissäännön
ja Lauseen \ref{Analyysin peruslause 2} perusteella päädytään derivoimissääntöön
\[
\boxed{\quad \frac{d}{dx}\int_{g_1(x)}^{g_2(x)} f(t)\, dt 
                      = f(g_2(x))g_2'(x)-f(g_1(x))g_1'(x) \quad}
\]
seuraavin oletuksin:
\begin{itemize}
\item[(i)]   $g_1(t)$ ja $g_2(t)$ ovat derivoituvia pisteessä $t=x$.
\item[(ii)]  $f(t)$ on määritelty, rajoitettu ja Riemann-integroituva jollakin välillä $[a,b]$,
             missä $\,a<\min\{g_1(x),g_2(x)\}\,$ ja $b>\max\{g_1(x),g_2(x)\}$.
\item[(iii)] $f(t)$ on jatkuva pisteissä $g_1(x)$ ja $g_2(x)$.
\end{itemize}
\begin{Exa} Jokaisella $x \neq 0$ (myös kun $x=1$\,(!)) pätee
\[
\frac{d}{dx}\int_x^{1/x} \frac{e^t}{t}\,dt\,
               =\,-\frac{1}{x^2}\left(x\,e^{1/x}\right)-\frac{e^x}{x}\,
               =\,-\frac{1}{x}(e^x+e^{1/x}),
\]
sillä oletukset (i)--(iii) ovat voimassa. Samasta syystä pätee jokaisella $x>0$
\[
\frac{d}{dx}\int_x^{x^2} \frac{e^t}{t}\,dt\,
               =\,2x \cdot \frac{e^{x^2}}{x^2}-\frac{e^x}{x}\,
               =\,\frac{1}{x}(2e^{x^2}-e^x).
\]
Sen sijaan jos $x \le 0$, niin tämä lasku ei ole pätevä, koska tällöin ei oletus (ii) 
(tapauksessa $x=0$ ei myöskään (iii)) ole voimassa. \loppu
\end{Exa}

\subsection{Osittaisintegrointi ja sijoitus määrätyssä integraalissa}
\index{osittaisintegrointi|vahv}
\index{muuttujan vaihto (sijoitus)!b@integraalissa|vahv}

Määrätylle integraalille pätevät riittävin säännöllisyysoletuksin seuraavat laskukaavat, 
vrt.\ vastaavat määräämättömän integraalin kaavat Luvussa \ref{osittaisintegrointi}. 
\begin{enumerate}
\item \pain{Osittaisinte}g\pain{rointikaava}
\[
\int_a^b f'(x)g(x)\, dx=\sijoitus{a}{b} f(x)g(x)-\int_a^b f(x)g'(x)\, dx.
\]
\item \pain{Si}j\pain{oituskaava}
\[
\int_a^b f(x)\, dx = \int_\alpha^\beta f(u(t))u'(t)\, dt,\quad u(\alpha)=a, \ u(\beta)=b.
\]
\end{enumerate}
Osittaisintegrointikaava on pätevä silloin kun $f$ ja $g$ ovat välillä $[a,b]$ jatkuvasti 
derivoituvia. Tällöin $f'g$ ja $fg'$ ovat jatkuvia välillä $[a,b]$, jolloin Analyysin
peruslauseen ja integraalin lineaarisuuden perusteella pätee
\begin{align*}
\int_a^b f'(x)g(x)\,dx+\int_a^b f(x)g'(x)\,dx
            &= \int_a^b [f'(x)g(x)+f(x)g'(x)]\,dx \\     
            &= \int_a^b \frac{d}{dx}\,f(x)g(x)\,dx = \sijoitus{a}{b} f(x)g(x).
\end{align*}

Määrätyn integraalin sijoituskaava vaatii hieman pitemmät perustelut. Olkoon esimerkiksi
$\alpha<\beta$ (kaavan pätevyys ei tätä edellytä). Tällöin on oletettava
\begin{itemize}
\item[(i)] $u$ on jatkuvasti derivoituva välillä $[\alpha,\beta]$,
\item[(ii)] $f$ on jatkuva välillä $[A,B]$, jolle pätee
$t\in [\alpha,\beta] \ \impl \ u(t)\in [A,B]$.
\end{itemize}
Kun merkitään
\[
F(x)=\int_a^x f(s)ds, \quad x\in[A,B],
\]
niin Analyysin peruslauseen mukaan on
\[
F'(x)=f(x),\quad x\in [A,B].
\]
Tällöin on oletuksien (i)-(ii) perusteella
\[
\frac{d}{dt}F(u(t))=f(u(t))u'(t),\quad t\in [\alpha,\beta],
\]
joten
\begin{align*}
\int_\alpha^\beta f(u(t))u'(t)\, dt &= \sijoitus{\alpha}{\beta} F(u(t)) \\
&= F(u(\beta))-F(u(\alpha)) \\
&= F(b)-F(a)=\int_a^b f(x)\, dx.
\end{align*}
Tapauksessa $\alpha >\beta$ on puhuttava välistä $[\beta,\alpha]$, muuten perustelut ovat samat.
Huomattakoon, että koska funktion $u$ injektiviisyttä ei vaadittu, eivät $\alpha$ ja $\beta$
välttämättä ole yksikäsitteiset, vrt.\ kuvio.
\begin{figure}[H]
\setlength{\unitlength}{1cm}
\begin{center}
\begin{picture}(12,6)
\put(0,0){\vector(1,0){12}} \put(11.8,-0.4){$t$}
\put(0,0){\vector(0,1){6}} \put(0.2,5.8){$x$}
% f(x)=0.05(x-2)(x-8)(x-10)+2
\curve(
    1.50,   0.619,
    2.00,   2.000,
    2.50,   3.031,
    3.00,   3.750,
    3.50,   4.194,
    4.00,   4.400,
    4.50,   4.406,
    5.00,   4.250,
    5.50,   3.989,
    6.00,   3.600,
    6.50,   3.181,
    7.00,   2.750,
    7.50,   2.344,
    8.00,   2.000,
    8.50,   1.756,
    9.00,   1.650,
    9.50,   1.719,
   10.00,   2.000,
   10.50,   2.531,
   11.00,   3.350)
%\put(0.9,0.85){$\bullet$}\put(4.9,2.85){$\bullet$}
%\drawline(1,1)(5,3)
%\drawline(3,1.1)(5,2.1)
\dashline{0.2}(0,2)(11,2)
\dashline{0.2}(2,0)(2,2) \dashline{0.2}(8,0)(8,2) \dashline{0.2}(10,0)(10,2)
\dashline{0.2}(0,3.75)(7,3.75)
\dashline{0.2}(3,0)(3,3.75) \dashline{0.2}(5.8,0)(5.8,3.75)
\put(0,1){\line(-1,0){0.1}} \put(0,5){\line(-1,0){0.1}}
\put(-0.6,0.9){$A$} \put(-0.6,4.9){$B$}
\put(-0.45,1.9){$a$} \put(-0.43,3.65){$b$}
\put(1.9,-0.4){$\alpha$} \put(7.9,-0.4){$\alpha$} \put(9.9,-0.4){$\alpha$}
\put(2.9,-0.5){$\beta$} \put(5.71,-0.5){$\beta$}
\put(11,3.6){$x=u(t)$}
\end{picture}
\end{center}
\end{figure}
\begin{Exa}
Kun integraalissa
\[
\int_0^2 \frac{e^{-x}}{1+\sqrt{x}}\, dx
\]
tehdään sijoitus
\[
\sqrt{x}=t \ge 0\ \ \ekv\ \ x=u(t)=t^2,
\]
niin
\[
dx=u'(t)dt=2tdt
\]
ja
\[
u(\alpha)=0\ \impl \alpha=0, \quad u(\beta)=2\ \impl\ \beta=\sqrt{2},
\]
joten integraali saa muodon
\[
\int_0^2 \frac{e^{-x}}{1+\sqrt{x}}\, dx=\int_0^{\sqrt{2}} \frac{e^{-t^2}}{1+t}\,2t\,dt
                                       =2\int_0^{\sqrt{2}} \frac{te^{-t^2}}{1+t}\,dt. \loppu
\]
\end{Exa}
Esimerkin muunnetussa integraalissa (toisin kuin alkuperäisessä) integroitava funktio on 
säännöllinen (sileä) koko integroimisvälillä. Integroitavan funktion säännöllisyys on yleisesti
eduksi silloin, kun integraalin arvo lasketaan numeerisilla menetelmillä, ks.\ Luku 
\ref{numeerinen integrointi} jäljempänä.

\subsection{Integraalilaskun väliarvolause}

Seuraava lause on väliarvolauseiden sarjan kolmas ja viimeinen, vrt.\ Lauseet 
\ref{ensimmäinen väliarvolause} ja \ref{toinen väliarvolause}. Verrattuna aiempiin
väliarvolauseisiin tämä lause ei ole kovin itsenäinen, sillä se seuraa helposti Analyysin
peruslauseesta ja Differentiaalilaskun väliarvolauseesta
(Harj.teht.\,\ref{H-int-6: todistuksia}a).
\begin{Lause} \label{kolmas väliarvolause} \vahv{(Integraalilaskun väliarvolause)}
\index{vzy@väliarvolauseet!c@integraalilaskun|emph} \index{Integraalilaskun väliarvolause|emph}
Jos f on jatkuva välillä $[a,b]$, niin jollakin $\xi\in(a,b)$ on
\[
\int_a^b f(x)\,dx = f(\xi)(b-a).
\]
\end{Lause}
Seuraavaa Lauseen \ref{kolmas väliarvolause} yleisempää (myös itsenäisempää) muotoa sanotaan
integraalilaskun \kor{yleistetyksi} väliarvolauseeksi.
\begin{Lause} \label{kolmas väliarvolause - yleistys}
Jos $f$ on jatkuva välillä $[a,b]$, $g$ on Riemann--integroituva välillä $[a,b]$ ja joko
$g(x) \ge 0$ tai $g(x) \le 0$ koko välillä $[a,b]$, niin jollakin $\xi \in[a,b]$ on
\[
\int_a^b f(x)g(x)\,dx=f(\xi)\int_a^b g(x)\,dx.
\]
\end{Lause}
\tod Jos $g(x)\geq 0$ ja $f_{\text{min}}$ ja $f_{\text{max}}$ ovat $f$:n minimi- ja maksimiarvot
välillä $[a,b]$, niin päätellään
\begin{align*}
&f_{\text{min}}\,g(x) \le f(x)g(x) \le f_{\text{max}}\,g(x),\quad x\in [a,b] \\[1mm]
&\impl\; f_{\text{min}} \int_a^b g(x)\,dx \le \int_a^b f(x)g(x)\,dx 
                                  \le f_{\text{max}} \int_a^b g(x)\,dx \\
&\impl\; \int_a^b f(x)g(x)\,dx = \eta \int_a^b g(x)\,dx, \quad 
                                    \eta\in [f_{\text{min}},f_{\text{max}}]. 
\end{align*}
Ensimmäisen väliarvolauseen (Lause \ref{ensimmäinen väliarvolause}) mukaan on tässä 
$\eta=f(\xi)$ jollakin $\xi\in [a,b]$, joten väite seuraa. Tapauksessa $g(t)\leq 0$ voidaan
käyttää jo todistettua väittämää, kun $g$:n tilalla on $-g$. \loppu

Funktion $f$ 
\index{keskiarvo!a@funktion}%
\kor{keskiarvoksi} (engl.\ average value) välillä $[a,b]$ sanotaan lukua
\[
K(f) = \frac{1}{b-a}\int_a^b f(x)\,dx.
\]
Jos $\,\D g(x) \ge 0\ \forall x\in[a,b]\,\ \text{ja}\,\ \int_a^b g(x)\,dx>0$, niin lukua
\[
K_g(f)= \frac{\int_a^b f(x)g(x)\, dx}{\int_a^b g(x)\,dx}
\]
sanotaan $f$:n \kor{painotetuksi keskiarvoksi} (engl.\ weighted average) välillä $[a,b]$ ja
\index{painotettu keskiarvo}%
funktiota $g$ tällöin \kor{painofunktioksi}. Lauseen \ref{kolmas väliarvolause - yleistys}
mukaan välillä $[a,b]$ jatkuvalle funktiolle pätee $K_g(f)=f(\xi)$ jollakin $\xi\in[a,b]$.

\subsection{Taylorin lauseen integraalimuoto}

Monissa sovelluksissa (esim.\ numeeristen menetelmien virhettä arvioitaessa, ks.\
Harj.teht.\,\ref{H-int-6: virhekaava}) seuraava Taylorin lauseen vaihtoehtoinen
\kor{integraalimuoto} on hyvin kätevä --- vrt.\ Lause \ref{Taylor}, jota kutsutaan Taylorin
lauseen \kor{väliarvomuodoksi}.
\begin{Lause} \label{Taylor-integraali} \index{Taylorin lause!a@integraalimuoto|emph}
\vahv{(Taylorin lause -- integraalimuoto)} Jos $f$ on $n+1$ kertaa jatkuvasti derivoituva 
välillä $[x_0,x]$ $(x>x_0)$ tai $[x,x_0]$ ($x<x_0$) ja $T_n(x,x_0)$ on $f$:n Taylorin polynomi
astetta $n$ pisteessä $x_0$, niin
\[
f(x)-T_n(x,x_0)=\int_{x_0}^x \frac{1}{n!}(x-t)^n f^{(n+1)}(t)\,dt.
\]
\end{Lause}
\tod Lähdetään ilmeisestä identiteetistä
\[
f(x)=f(x_0)+\int_{x_0}^x f'(t)\,dt,
\]
joka on sama kuin lauseen väittämä tapauksessa $n=0$. Jos tässä $f$ on kahdesti jatkuvasti
derivoituva integroimisvälillä (oletus tapauksessa $n=1$), voidaan integroida osittain
seuraavasti:
\begin{align*}
f(x) &= f(x_0)-\sijoitus{x_0}{x} (x-t)f'(t)+\int_{x_0}^x (x-t)f''(t)\,dt \\
&= f(x_0)+f'(x_0)(x-x_0)+\int_0^x (x-t)f''(t)\,dt.
\end{align*}
Näin on todistettu väittämä tapauksessa $n=1$. Jatkamalla osittaisintegrointia nähdään 
vastaavasti, että väittämä on tosi, kun $n=2$, jne.\ (Yleinen todistus:
Harj.teht.\,\ref{H-int-6: todistuksia}b.) \loppu

\Harj
\begin{enumerate}

\item
Jos $f$ jatkuva välillä $[a,b]$, niin mikä on funktion
\[
g(x)=\int_a^b [f(t)-x\,]^2\,dt
\]
pienin arvo ja missä se saavutetaan?

\item
Laske:
\begin{align*}
&\text{a)}\,\ \frac{d}{dx}\int_2^x \frac{\sin t}{t}\,dt \qquad
 \text{b)}\,\ \frac{d}{dt}\int_t^3 \frac{\sin x}{x}\,dx \qquad
 \text{c)}\,\ \frac{d}{dx}\int_{x^2}^x \frac{\sin t}{t}\,dt \\
&\text{d)}\,\ \frac{d}{dx}x^2\int_0^{x^2} \frac{\sin u}{u}\,du \qquad
 \text{e)}\,\ \frac{d}{dt} \int_{-t^{-1}}^t \frac{\cos y}{1+y^2}\,dy \\
&\text{f)}\,\ \frac{d}{d\theta}\int_{\sin\theta}^{\cos\theta} \frac{1}{1-x^2}\,dx \qquad
 \text{g)}\,\ \frac{d}{dx}F(\sqrt{x}),\,\ F(t)=\int_0^t \cos(x^2)\,dx \\
&\text{h)}\,\ H'(2),\,\ H(x)=3x\int_4^{x^2} e^{-\sqrt{t}}\,dt
\end{align*}

\item
Ratkaise $y(x)$:
\[
\text{a)}\ \ y(x)=1-\int_0^x y(t)\,dt \qquad
\text{b)}\ \ y(x)=\pi\left(1+\int_1^x y(t)\,dt\right)
\]

\item
Laske integroimalla osittain kohdissa a)--f) ja sijoituksella kohdissa g)--l):
\begin{align*}
&\text{a)}\ \ \int_0^1 x e^{-x}\,dx \qquad
 \text{b)}\ \ \int_0^\pi x\cos x\,dx \qquad
 \text{c)}\ \ \int_0^\pi x\sin x\,dx \\
&\text{d)}\ \ \int_0^2 \sqrt{x}\ln x\,dx \qquad
 \text{e)}\ \ \int_0^1 \Arctan x\,dx \qquad
 \text{f)}\ \ \int_0^\pi e^{-x}\sin x\,dx \\
&\text{g)}\ \ \int_{-2}^2 \frac{1}{2x^2+4x+3}\,dx \qquad
 \text{h)}\ \ \int_0^{\ln 2} \frac{e^x}{1+e^x}\,dx \qquad
 \text{i)}\ \ \int_1^4 \frac{\sqrt{x}}{\sqrt[4]{x}+1}\,dx \\
&\text{j)}\ \ \int_0^1 \sqrt{\frac{1-x}{1+x}}\,dx \qquad
 \text{k)}\ \ \int_0^{\pi/4} \frac{1}{1+\tan x}\,dx \qquad
 \text{l)}\ \ \int_0^{\pi} \frac{\sin x}{2-\cos x}\,dx
\end{align*}

\item
Olkoon $R>0$ ja $ab \neq 0$. Laske
\[
\text{a)}\ \ \int_0^R x^2\sqrt{R^2-x^2}\,dx, \qquad
\text{b)}\ \ \int_0^{\pi/2} \frac{1}{a^2\cos^2 x+b^2\sin^2 x}\,dx.
\]

\item
Yksinkertaista seuraavat funktiolausekkeet sijoitusta käyttäen ($x>0$).
\[
\text{a)}\ \ f(x)=\int_0^x \sqrt{t}\,\sqrt{x^2-t^2}\,dx \qquad
\text{b)}\ \ f(x)=\int_{-x/2}^{x/2} \frac{1}{(x^2-t^2)^{3/2}}\,dt
\]

\item
Laske funktion $f(x)=\sin^2x$ \ a) keskiarvo, \,b) funktiolla $g(x)=x$ painotettu keskiarvo
välillä $[0,\pi]$.


\item \label{H-int-6: todistuksia}
a) Todista Integraalilaskun väliarvolause soveltamalla Differentiaalilaskun väliarvolausetta
funktioon $F(x)=\int_a^x f(t)\,dt$. \ b) Todista Lause \ref{Taylor-integraali} induktiolla. \
c) Johda Lauseesta \ref{Taylor-integraali} Taylorin lauseen väliarvomuoto (Lause \ref{Taylor})
käyttämällä hyväksi Lausetta \ref{kolmas väliarvolause - yleistys}.

\item (*)
Onko funktiolla $\D\,F(x)=\int_0^{2x-x^2}\cos\left(\frac{1}{1+t^2}\right)dt\,$
suurin tai pienin arvo? Perustele!

\item (*) \index{osittaissummaus}
Näytä oikeaksi \kor{osittaissummauksen} kaava
\[
\sum_{k=1}^n f_k g_k = \sijoitus{k=1}{k=n} F_k g_k - \sum_{k=1}^{n-1} F_k\,(g_{k+1}-g_k), \quad
                       F_k = \sum_{i=1}^k f_i.
\]             
Mikä on vastaava kaava määrätylle integraalille?

\item (*) \label{H-int-6: virhekaava}
Olkoon $f$ kolmesti jatkuvasti derivoituva välillä $[a-h,a+h]$. Johda Lauseen
\ref{Taylor-integraali} avulla keskeisdifferenssiapproksimaation virhekaava
\begin{align*}
&f'(a)-\frac{f(a+h)-f(a-h)}{2h} \,=\, -\frac{1}{h}\int_{a-h}^{a+h} k(t)f'''(t)\,dt, \\[2mm]
&\text{missä}\quad k(t)=\begin{cases}
      \,\frac{1}{2}(a-h-t)^2, &\text{kun}\ \ t\in[\,a-h,a\,], \\
      \,\frac{1}{2}(a+h-t)^2, &\text{kun}\ \ t\in[\,a,a+h\,].
      \end{cases}
\end{align*}
Johda tästä edelleen virhekaavalle väliarvomuoto
(vrt.\ Propositio \ref{keskeisdifferenssin tarkkuus}) 
\[
f'(a)-\frac{f(a+h)-f(a-h)}{2h} \,=\, -\frac{1}{6}\,h^2 f'''(\xi), \quad\ \xi\in[a-h,a+h].
\]

\end{enumerate} %Analyysin peruslause
\section{Riemannin integraalin laajennukset} \label{integraalin laajennuksia}
\alku

Riemannin integraalin määritelmässä on perusoletuksena, että integroitava funktio $f$ on sekä 
\pain{määritelt}y että \pain{ra}j\pain{oitettu} koko integroimisvälillä, jonka on oltava 
(äärellinen) \pain{sul}j\pain{ettu} väli. Ilman näitä rajoituksia ei integraalia voi
yleisesti määritellä Riemannin summien (tai ylä- ja alasummien) raja-arvona, ts.\ rajoitukset
ovat välttämättömiä, jotta $f$ olisi Riemann-integroituva. Monien sovellusten kannalta
kuitenkin mainitut rajoitukset ovat turhan voimakkaita tai jopa keinotekoisia. Esimerkiksi
koska tiedetään, että integraalin arvo ei riipu $f$:n arvoista yksittäisessä (tai äärellisen
monessa) pisteessä, niin tuntuu turhalta  ylipäänsä vaatia, että $f$ on määritelty jokaisessa
integroimisvälin pisteessä. Tällaisten turhien rajoitusten poistamiseksi on tullut tavaksi
konstruoida Riemannin integraalin määritelmälle erilaisia laajennuksia, joita sanotaan
\index{epzyoi@epäoleellinen integraali}%
\kor{epäoleellisiksi} (engl.\ improper, kirjaimellisesti 'sopimaton' tai 'hyvien tapojen
vastainen') Riemannin
integraaleiksi. Laajennukset eivät johda kovin selkeään yleiseen integraalin määritelmään, vaan
niiden tarkoituksena on lähinnä 'paikata' alkuperäistä määritelmää niin, että tavallisimmat
sovelluksissa esiintyvät tapaukset tulevat katetuiksi. Jatkossa esitetään näistä laajennuksista
tavallisimmat.\footnote[2]{Riemannin integraalin ongelmat havaittiin jo 1800-luvun
jälkipuoliskolla, jolloin mittojen ja integraalien teoria kehittyi voimakkaasti. Lopullisen
ratkaisun ongelmaan toi ranskalaisen \hist{Henri Lebesgue}n (1875-1941) vuonna 1906 esittämä
määritelmä, joka on sittemmin tunnettu \kor{Lebesguen integraalina}. Lebesguen integraali
poistaa Riemannin integraalin kauneusvirheet, mutta sen määrittely vaatii syvällisempiä
mittateoreettisia pohdiskeluja. \index{Lebesgue, H.|av} \index{Lebesguen integraali|av}}

Jos $f$ on rajoitettu ja Riemann-integroituva välillä $[a,b]$, niin funktion
\[
F(x)=\int_a^x f(t)\, dt
\]
Lipschitz-jatkuvuuden (Lause \ref{Analyysin peruslause 2}) perusteella
\[
\int_{a+\eps}^b f(t)\, dt = F(b)-F(a+\eps) = \int_a^b f(t)\, dt + \ordoO{\eps},
\]
kun $0<\eps<b-a$, joten
\[
\int_a^b f(x)\, dx = \lim_{\eps \kohti 0^+} \int_{a+\eps}^b f(x)\, dx.
\]
Oikealla oleva raja-arvo ei riipu $f$:n arvoista pisteessä $a$ --- itse asiassa raja-arvo ei
riipu edes siitä, onko $f$ määritelty tässä pisteessä. Tämän perusteella on varsin luontevaa
määritellä integraali ko.\ raja-arvona (mikäli olemassa) silloinkin, kun $f$ ei ole pisteessä 
$a$ määritelty ja/tai $f$ ei ole rajoitettu ko. pistettä lähestyttäessä.
\begin{Exa} \label{epäoleellinen esim1} Jos $f(x)=1/\sqrt{x-a}$, niin $f$ ei ole välillä 
$[a,b]$ rajoitettu (eikä pisteessä $a$ määritelty), mutta em.\ raja-arvo on olemassa: 
\begin{align*}
\int_{a+\eps}^b \frac{1}{\sqrt{x-a}}\,dx\ 
                      =\ \sijoitus{x=a+\eps}{x=b} 2\sqrt{x-a}\
                     &=\ 2\sqrt{b-a}-2\sqrt{\eps} \\ 
                     &\kohti\ 2\sqrt{b-a}, \quad \text{kun}\,\ \eps \kohti 0^+. \loppu
\end{align*}
\end{Exa}
Kun em.\ raja-arvomenettelyä sovelletaan integroimisvälin kummassakin päätepisteessä, tullaan
seuraavaan määritelmään.
\begin{Def} \vahv{(Riemannin integraalin 1. laajennus)} \label{1. R-laajennus}
\index{Riemann-integroituvuus!a@laajennettu|emph} Olkoon $a<c<b$. Jos $f$ on rajoitettu ja
Riemann-integroituva väleillä $[a+\eps,c]$ ja $[c,b-\eps]$ aina kun $\eps>0$
($\eps<\min\{c-a,b-c\}$), niin määritellään
\[
\int_a^b f(x)\, dx = \lim_{\eps \kohti 0^+} \int_{a+\eps}^c f(x)\,dx
                   + \lim_{\eps \kohti 0^+} \int_c^{b-\eps} f(x)\,dx,
\]
sikäli kuin oikealla puolella olevat raja-arvot ovat olemassa. Sanotaan tällöin, että $f$ on
integroituva välillä $[a,b]$.
\end{Def}
Määritelmän voi esittää hieman lyhyemmin muodossa
\[
\int_a^b f(x)\,dx = \lim_{\eps_1,\eps_2 \kohti 0^+} \int_{a+\eps_1}^{b-\eps_2} f(x)\,dx,
\]
missä raja-arvomerkintä oikealla tarkoittaa, että $\eps_1 \kohti 0^+$ ja $\eps_2 \kohti 0^+$
toisistaan riippumatta. Jos oletetaan, että $f$:llä on välillä $(a,b)$ integraalifunktio $F$,
niin noudattaen lyhennettyä merkintätapaa on Määritelmän \ref{1. R-laajennus} mukaisesti
\[
\int_a^b f(x)\,dx = \lim_{\eps_1,\eps_2 \kohti 0^+}\sijoitus{a+\eps_1}{b-\eps_2} F(x)
                  = F(a^+)-F(b^-).
\]
Siis $f$ on tässä tapauksessa integroituva täsmälleen kun raja-arvot $F(a^+)$ ja $F(b^-)$ ovat
olemassa eli kun $F$ on jatkuva välillä $[a,b]$.
\jatko \begin{Exa} (jatko) Funktiolla $\,f(x)=1/\sqrt{x-a}\,$ on välillä $[a,b]$ jatkuva
integraalifunktio $F(x)=2\sqrt{x-a}$. Integraalin arvoa laskettaessa on käytännössä
turvallista (ja hyväksyttyä) ohittaa Määritelmän \ref{1. R-laajennus} raja-arvoprosessi eli
laskea suoraan
\[
\int_a^b \frac{1}{\sqrt{x-a}}\,dx \,=\, \sijoitus{x=a}{x=b} 2\sqrt{x-a} 
                                  \,=\, 2\sqrt{b-a}. \loppu
\]
\end{Exa}

Koska integraalin additiivisuutta on luonnollista pitää myös em.\ tavalla laajennettujen
(epäoleellisten) integraalien ominaisuutena, saadaan välittömästi seuraava laajennus.
\begin{Def}
\vahv{(Riemannin integraalin 2. laajennus)} \label{2. R-laajennus}
\index{Riemann-integroituvuus!a@laajennettu|emph} Jos $a=c_0<c_1<\ldots <c_n=b$ ja $f$ on
integroituva väleillä $[c_{k-1},c_k]$, $k=1\ldots n$, niin $f$ on integroituva välillä $[a,b]$
ja
\[
\int_a^b f(x)\, dx=\sum_{k=1}^m \int_{c_{k-1}}^{c_k} f(x)\, dx.
\]
\end{Def}
Tämän määritelmän mukaisesti on esimerkiksi välillä $[a,b]$ paloittain jatkuva funktio ko.\ 
välillä integroituva. Määritelmän mukaisesti integraalin arvo on tällöin riippumaton siitä, 
miten $f$ on määritelty (tai onko lainkaan määritelty) epäjatkuvuuspisteissä.
\begin{Exa} Jos $f(x)=k$, kun $k-1 < x < k$, $k=1 \ldots n\,$ ($f(k)$ joko määritelty tai ei,
kun $k=0 \ldots n$), niin Määritelmän \ref{2. R-laajennus} mukaisesti
\[
\int_0^n f(x)\, dx \,=\, \sum_{k=1}^n \int_{k-1}^k f(x)\, dx
                   \,=\, \sum_{k=1}^n k \,=\, \frac{1}{2}n(n+1). \loppu
\]
\end{Exa}
Lopuksi laajennetaan integraalin käsite koskemaan äärettömiä välejä $[a,\infty)$, \newline
$(-\infty,b]$ ja $(-\infty,\infty)\,$:
\begin{Def} \vahv{(Riemannin integraalin 3. laajennus)} \label{3. R-laajennus}
\index{Riemann-integroituvuus!a@laajennettu|emph} Jos $f$ on integroituva väleillä $[a,M]$
jokaisella $M>a$ ja on olemassa raja-arvo
\[
A=\lim_{M\kohti\infty} \int_a^M f(x)\,dx,
\]
niin sanotaan, että $f$ on integroituva välillä $[a,\infty)$, lukua $A$ sanotaan $f$:n
integraaliksi ko. välillä, ja merkitään
\[
A=\int_a^\infty f(x)\,dx.
\]
Vastaavasti määritellään
\begin{align*}
\int_{-\infty}^b f(x)\, dx 
                 &= \lim_{M\kohti\infty} \int_{-M}^b f(x)\, dx, \\
\int_{-\infty}^\infty f(x)\, dx 
                 &= \int_{-\infty}^c f(x)\, dx + \int_c^\infty f(x)\,dx, \quad c\in\R.
\end{align*}
\end{Def}
\begin{Exa}
Funktio $f(x)=1/(1+x^2)$ on integroituva välillä $(-\infty,\infty)$, sillä Määritelmän
\ref{3. R-laajennus} mukaisesti
\begin{align*}
\int_{-\infty}^\infty \frac{1}{1+x^2} 
        &\,=\, \int_{-\infty}^0\frac{1}{1+x^2}\,dx + \int_0^\infty \frac{1}{1+x^2}\,dx \\
        &\,=\, \lim_{M \kohti\infty} \int_{-M}^0\frac{1}{1+x^2}\,dx
              +\lim_{M\kohti\infty} \int_0^M \frac{1}{1+x^2}\,dx \\
        &\,=\, \lim_{M\kohti\infty}\sijoitus{-M}{0} \Arctan x
              +\lim_{M\kohti\infty}\sijoitus{0}{M} \Arctan x
         \,=\, \frac{\pi}{2}+\frac{\pi}{2}=\pi. \loppu
\end{align*}
%Symmetria huomioiden ja raja-arvoprosessi ohittaen laskusta tulee lyhyempi:
%\[
%\int_{-\infty}^\infty \frac{1}{1+x^2}\,dx 
%        \,=\, 2\int_0^\infty \frac{1}{1+x^2}\,dx
%        \,=\, 2\sijoitus{0}{\infty} \Arctan x
%        \,=\, 2\left(\frac{\pi}{2}-0\right) \,=\, \pi. \loppu
%\]
\end{Exa}
Jos $f$ on tarkasteltavalla (äärellisellä tai äärettömällä) välillä integroituva joko 
tavanomaisessa tai laajennetussa mielessä, niin sanotaan, että $f$:n integraali ko. välin yli
\index{suppeneminen!c@integraalin} \index{hajaantuminen!c@integraalin}%
\kor{suppenee}. Jos $f$ ei ole integroituva, niin sanotaan, että integraali
\kor{hajaantuu}. Kuten esimerkissä, integraalin suppeneminen selviää helposti
integraalifunktion avulla, sikäli kuin sellainen on integroimsvälillä löydettävissä. Nimittäin
Määritelmän \ref{3. R-laajennus} mukaan myös äärettömillä integroimisväleillä $[a,\infty)$,
$(-\infty,b]$ ja $(-\infty,\infty)$ pätee: jos $f$:n integraalifunktio välillä $(a,b)$ on $F$,
niin integraalin $\int_a^b f(x)\,dx$ arvo (sikäli kuin olemassa) on
\[
\int_a^b f(x)\,dx \,=\, \lim_{x \kohti b^+}F(x)-\lim_{x \kohti a^-}F(x),
\]
eli integraali suppenee täsmälleen kun molemmat raja-arvot oikealla puolella ovat olemassa.
Mikäli koko integroimisvälillä toimivaa integraalifunktiota ei löydy, voidaan väli aina jakaa
äärelliseen määrään osavälejä, käyttää integraalin additiivisuutta ja tutkia integraalin
suppeneminen kullakin osavälillä erikseen.
\begin{Exa}
Raja-arvoissa hieman 'oikaisten' voidaan laskea
\begin{align*}
\int_{-\infty}^\infty \frac{1}{1+x^2}\,dx 
           &= \sijoitus{-\infty}{\infty}\Arctan x 
            = \dfrac{\pi}{2}-\left(-\frac{\pi}{2}\right) = \pi \quad \text{(suppenee)}, \\
\int_0^1 \frac{1}{x}
           &= \sijoitus{0}{1} \ln x = \infty \quad \text{(hajaantuu)}.
%\int_0^\infty e^{-x} \, dx                &= \sijoitus{0}{\infty} (-e^{-x})=0-(-1)=1. \\
\intertext{Varomaton on sen sijaan lasku}
\int_{-\infty}^\infty \frac{1}{x^2}\,dx   &= \sijoitus{-\infty}{\infty} -\frac{1}{x} = 0,
\end{align*}
sillä huomaamatta jäi, että $F(x)=-1/x$ ei ole funktion $f(x)=1/x^2$ integraalifunktio koko
välillä $(-\infty,\infty)$. Jakamalla integraali kahteen osaan väleille $(-\infty,0]$ ja 
$[0,\infty)$ nähdään, että se hajaantuu. \loppu
\end{Exa} 

\subsection{Epäoleellisten integraalien vertailu}

Silloin kun epäoleellisen integraalin suppenevuutta ei voida suoraan tutkia integraalifunktion
avulla, tai se on hankalaa, voidaan suppenevuus yleensä selvittää vertaamalla integraalia 
yksinkertaisempaan, laskettavissa olevaan integraaliin. Vertailussa lähdetään ennestään
tutusta vertailuperiaatteesta (Lause \ref{integraalien vertailuperiaate})
\[
f(x)\leq g(x)\quad\forall x\in [a,b] \ \impl \ \int_a^b f(x)\, dx\leq\int_a^b g(x)\,dx.
\]
Sovellettaessa tätä epäoleellisiin integraaleihin äärellisellä välillä $[a,b]$ riittää
olettaa, että $f(x) \le g(x)$ on voimassa avoimella välillä $(a,b)$
(Määritelmä \ref{1. R-laajennus}) tai avoimilla osaväleillä $\ (c_{k-1},c_k)\subset(a,b)$,
$\,k=1 \ldots n\,$ (Määritelmä \ref{2. R-laajennus}), sillä koska vertailuperiaate on pätevä
tavallisille Riemannin integraaleille, se pysyy voimassa myös määritelmien raja-arvoille.
Vertailtaessa integraaleja $\int_a^\infty f(x)\,dx$ ja $\int_a^\infty g(x)\,dx$
(Määritelmä \ref{3. R-laajennus}) riittää vastaavasti olettaa, että $f(x) \le g(x)$ väleillä
$\,(c_{k-1},c_k)$, $\,k=1,2,\ldots$ missä $\,a=c_0<c_1<\ldots$ ja $\,\lim_kc_k=\infty$.

Tarkastellaan esimerkkinä integraalia $\int_a^\infty f(x)\,dx$.
(Integraali $\int_{-\infty}^b f(x)\,dx$ palautuu tähän muuttujan vaihdolla $t=-x$.)
Funktioille, jotka
eivät vaihda merkkiään integroimisvälillä (lukuunottamatta mahdollisia erillisiä pisteitä,
jotka eivät vaikuta integraalin arvoon), voi integraalien suppenemistarkasteluissa käyttää
seuraavia vertailuperiaatteita, vrt.\ vastaavat periaatteet positiivitermisille sarjoille
(Lause \ref{sarjojen vertailu}). Yksinkertaisuuden vuoksi oletetetaan tässä funktiot
määritellyksi ja epäyhtälöt voimassa oleviksi koko välillä $[a,\infty)$. 
\begin{Lause} \label{integraalien majorointi ja minorointi}
\index{majoranttiperiaate|emph} \index{minoranttiperiaate|emph}% 
(\vahv{Majorantti- ja minoranttiperiaatteet integraaleille})\newline
Jos $f$ ja $g$ ovat määriteltyjä välillä $[a,\infty)$, integroituvia välillä $[a,M]$ jokaisella
$\,M>a\,$ ja $\,0 \le f(x) \le g(x)\ \forall x\in[a,\infty)$, niin pätee
\begin{align*}
&\int_a^\infty g(x)\,dx\ \text{suppenee} \,\ \qimpl \int_a^\infty f(x)\,dx\ \text{suppenee}, \\
&\int_a^\infty f(x)\,dx\ \text{hajaantuu} \qimpl \int_a^\infty g(x)\,dx\ \text{hajaantuu}.
\end{align*}
\end{Lause}
\tod Väittämät ovat loogisesti ekvivalentteja, joten riittää todistaa ensimmäinen.
Oletetaan siis, että $\int_a^\infty g(x)\,dx$ suppenee. Koska $f(x) \ge 0$ välillä
$[a,\infty)$, niin funktio $F(x)=\int_a^x f(t)\,dt$ on ko.\ välillä kasvava:
\[
F(x_2)-F(x_1) \,=\, \int_{x_1}^{x_2} f(t)\,dt \,\ge\, 0, \quad \text{kun}\ a \le x_1 < x_2.
\]
Samasta syystä $G(x)=\int_a^x g(t)\,dt$ on kasvava välillä $[a,\infty)$, ja koska
$f(x) \le g(x),\ x\in[a,\infty)\,$ ja integraali $\int_a^\infty g(t)\,dt\,$ suppenee, niin
voidaan päätellä:
\[
F(x)\,\le\,G(x)\,\le\,\lim_{x\kohti\infty}G(x)\,=\,\int_a^\infty g(t)\,dt, \quad x\in[a,\infty).
\]
Näin ollen $F(x)$ on välillä $[0,\infty)$ sekä kasvava että rajoitettu. Lauseen
\ref{monotonisen funktion raja-arvo} mukaan on tällöin olemassa raja-arvo
$\,\lim_{x\kohti\infty}F(x)=\int_a^\infty f(x)\,dx$. \loppu
\begin{Exa} Lauseen \ref{integraalien majorointi ja minorointi} perusteella integraali
$\,\int_1^\infty e^{-x}/x\,\,dx\,$ suppenee, sillä välillä $[1,\infty)\,$ on
$\,e^{-x}/x \le e^{-x}\,$ ja integraali $\,\int_0^\infty e^{-x}\,dx\,$ suppenee:
\[
\int_1^\infty e^{-x}\,dx\,=\,\sijoitus{1}{\infty} (-e^{-x})\,=\,1. \loppu
\]
\end{Exa}
Lauseen \ref{integraalien majorointi ja minorointi} majorantti- ja minoranttiperiaatteet
yleistyvät sopivin oletuksin koskemaan muitakin epäoleellisia integraaleja. Esimerkiksi
olettaen että $0 \le f(x) \le g(x)\ \forall x\in(a,b)$, Määritelmän \ref{1. R-laajennus}
mukaisille integraaleille pätee
\[
\int_a^b g(x)\,dx\ \text{suppenee} \qimpl \int_a^b f(x)\,dx\ \text{suppenee}.
\]
\begin{Exa} Integraalia $\,\int_0^1 e^{-x}/x\,\,dx$ voi tutkia muuttujan vaihdolla $t=1/x$,
jolloin Lause \ref{integraalien majorointi ja minorointi} soveltuu. Suorempi päättely kuitenkin
riittää: Integraali hajaantuu, koska $\,e^{-x}/x \ge e^{-1}x^{-1} > 0,\ x\in(0,1]\,$ ja
$\,\int_0^1 x^{-1}\,dx$ hajaantuu. \loppu
\end{Exa}

\subsection{Cauchyn kriteeri integraaleille}

Seuraava yleinen suppenemiskriteeri toimii kaikille epäoleellisille integraaleille muotoa
$\,\int_a^\infty f(x)\,dx$, vrt.\ vastaava Cauchyn kriteeri sarjoille
(Lause \ref{Cauchyn sarjakriteeri}).
\begin{Lause} \label{Cauchyn integraalikriteeri} (\vahv{Cauchyn kriteeri integraaleille})
\index{Cauchyn!e@kriteeri integraaleille|emph}
Olkoon $f$ on integroituva välillä $[a,M]$ jokaisella $M>a$. Tällöin integraali
$\int_a^\infty f(x)\, dx$ suppenee täsmälleen kun jokaisella $\eps>0$ on olemassa $M>a$ siten,
että pätee
\[
\bigl|\int_{M_1}^{M_2} f(x)\,dx\,\bigr| \,<\, \eps, \quad \text{kun}\ M_1,M_2>M.
\]
\end{Lause}
\tod $\boxed{\impl}\quad$ Jos integraali suppenee ja $M_1,M_2>a$, niin
\begin{align*}
\int_{M_1}^{M_2} f(x)\, dx &= \int_a^{M_2} f(x)\, dx-\int_a^{M_1} f(x)\, dx \\
                            &\kohti \int_a^\infty f(x)\, dx-\int_a^\infty f(x)\, dx=0, \quad 
                                                      \text{kun}\,\ M_1,M_2\kohti\infty,
\end{align*}
joten Cauchyn kriteeri on täytetty.

$\boxed{\Leftarrow}\quad$ Oletetaan, että Cauchyn kriteeri on täytetty ja merkitään
\[
A_n=\int_a^n f(x)\, dx,\quad n\in\N, \ n>a.
\]
Oletuksen perusteella
\[
A_n-A_m=\int_m^n f(x) \, dx\kohti 0, \quad \text{kun}\,\ n,m\kohti\infty,
\]
joten $\{A_n\}$ on Cauchyn jono, ja siis (Lause \ref{Cauchyn kriteeri})
\[
A_n\kohti A\in\R, \quad \text{kun}\ n\kohti\infty.
\]
Käyttämällä tätä tulosta ja vetoamalla uudelleen oletukseen todetaan, että jos $\eps>0$, niin
valitsemalla $M\in\R$ ja $n\in\N$ riittävän suuriksi on
\begin{align*}
\bigl|\int_a^M f(x)\, dx-A\,\bigr|\ &\le\ \bigl|\int_a^M f(x)\,dx-A_n\,\bigr|+|A_n-A| \\
                                    &= \bigl|\int_n^M f(x)\,dx\,\bigr|+|A_n-A|<\eps.
\end{align*}
Näin ollen $f$ on integroituva välillä $[a,\infty)$:
\[
\lim_{M\kohti\infty}\int_a^M f(x)\,dx = A. \loppu
\]
\begin{Exa}
Suppeneeko vai hajaantuuko integraali $\D\,\int_0^\infty \frac{\cos x}{\sqrt{x}}\,dx$\,?
\end{Exa}
\ratk Jaetaan integraali kahteen osaan:
\[
\int_0^\infty \frac{\cos x}{\sqrt{x}}\,dx = \int_0^{\pi/2} \frac{\cos x}{\sqrt{x}}\,dx
                                          + \int_{\pi/2}^\infty \frac{\cos x}{\sqrt{x}}\,dx.
\]
Integroimalla osittain ensimmäisessä osassa todetaan
\[
\int_0^{\pi/2} \frac{\cos x}{\sqrt{x}}\,dx  
            = \sijoitus{0}{\pi/2} 2\sqrt{x}\cos x + \int_0^{\pi/2} 2\sqrt{x}\sin x\,dx
            = 2 \int_0^{\pi/2} \sqrt{x}\sin x\,dx.
\]
Funktio $f(x)=\sqrt{x}\sin x$ on jatkuva välillä $[0,\pi/2]$, joten tämä osa integraalista
suppenee. Integraalin loppuosassa integroidaan osittain toisin päin:
\[
\int_{\pi/2}^M \frac{1}{\sqrt{x}}\cos x \, dx 
                   = \sijoitus{\pi/2}{M}\frac{1}{\sqrt{x}}\sin x
                            +\frac{1}{2}\int_{\pi/2}^M x^{-3/2}\sin x\,dx.
\]
Tässä
\[
\sijoitus{\pi/2}{M}\frac{1}{\sqrt{x}}\sin x \kohti -\sqrt{\frac{2}{\pi}}, \quad
                                                        \text{kun}\,\ M\kohti\infty,
\]
joten suppenemiskysymys siirtyy koskemaan integraalia
\[
\int_{\pi/2}^\infty x^{-3/2}\sin x \, dx.
\]
Kolmioepäyhtälön (Luku \ref{määrätty integraali}, epäyhtälö \eqref{int-4: kolmioepäyhtälö})
avulla arvioiden todetaan, että jos $\,\pi/2<M_1<M_2$, niin
\begin{align*}
\left|\int_{M_1}^{M_2} x^{-3/2}\sin x\,dx\right| 
           &\le \int_{M_1}^{M_2} x^{-3/2}\abs{\sin x\,}\,dx \\
           &\le \int_{M_1}^{M_2} x^{-3/2} \, dx \\
           &= -2\sijoitus{M_1}{M_2} x^{-1/2}\kohti 0, \quad \text{kun}\,\ M_1,M_2\kohti\infty,
\end{align*}
joten Cauchyn kriteerin perusteella myös tämä osa integraalista suppenee. Siis vastaus on:
Suppenee! \loppu

\subsection{Integraalit ja sarjat}
\index{sarjaoppi (klassinen)|vahv}

Edellä on jo nähty, että integraalien $\int_a^\infty f(x)\,dx$ ja sarjojen suppenemisteoriat
muistuttavat toisiaan. Laskennallisemmaltakin kannalta integraalien ja sarjojen yhteys on
merkittävä, sillä osoittautuu, että integraaalien avulla voidaan sekä yksinkertaistaa
sarjojen suppenemistarkasteluja että tehostaa sarjojen summien numeerista laskemista.

Tarkastellaan esimerkkinä sarjaa $\,\sum_{k=1}^\infty f(k)$, missä oletetaan, että $f$ on
määritelty välillä $\,[1,\infty)\,$ ja lisäksi jollakin $n\in\N$ pätee:
\begin{itemize}
\item[(i)]  $f$ on vähenevä välillä $\,[n,\infty)$ ja $\,\lim_{x\kohti\infty}f(x)=0$.
\end{itemize}
Oletuksesta (i) seuraa, että $f(x) \ge 0$ välillä $[n,\infty)$ ja jokaisella $k \ge n$ on
voimassa
\[
f(k+1)\leq f(x)\leq f(k), \quad \text{kun}\,\ x\in [k,k+1].
\]
Näin ollen
\[
\int_k^{k+1} f(k+1)\, dx\ \le\ \int_k^{k+1} f(x)\, dx\ \leq\ \int_{k}^{k+1} f(k)\, dx,
\]
eli
\[
f(k+1)\ \leq\ \int_k^{k+1} f(x)\, dx\ \leq\ f(k).
\]
\begin{figure}[H]
\setlength{\unitlength}{1cm}
\begin{center}
\begin{picture}(11,5)(-1,-1)
\put(-1,0){\vector(1,0){11}} \put(9.8,-0.5){$x$}
\put(0,-1){\vector(0,1){5}} \put(0.2,3.8){$y$}
\curve(4,4,6,2.5,8,1.8)
\dashline{0.1}(5,0)(5,3.13) \dashline{0.1}(6,0)(6,3.13) \dashline{0.1}(6,2.5)(5,2.5)
\dashline{0.1}(6,3.13)(5,3.13)
\put(4.95,3.08){$\scriptstyle{\bullet}$} \put(5.93,2.43){$\scriptstyle{\bullet}$}
\put(2,0){\line(0,-1){0.1}} \put(5,0){\line(0,-1){0.1}} \put(6,0){\line(0,-1){0.1}}
\put(1.9,-0.5){$n$} \put(4.9,-0.5){$k$} \put(5.9,-0.5){$k+1$}
\put(8.2,1.7){$y=f(x)$}
\end{picture}
\end{center}
\end{figure}
Summaamalla em.\ tulos yli indeksien $\,k=n\ldots N-1$ ($N>n$) ja käyttämällä integraalin
additiivisuutta saadaan
\[
\sum_{k=n+1}^N f(k) \,\le\, \int_n^N f(x)\,dx \ \le\ \sum_{k=n}^{N-1} f(k)
                                             \ =\, \sum_{k=n+1}^{N-1} f(k) + f(n).
\]
Koska tässä $\lim_nf(n)=0$ oletuksen (i) mukaan, niin Cauchyn sarja- ja integraalikriteerien
(Lauseet \ref{Cauchyn sarjakriteeri} ja \ref{Cauchyn integraalikriteeri}) perusteella
päätellään, että sarja $\sum_{k=1}^\infty f(k)$ suppenee täsmälleen kun integraali
$\int_n^\infty f(x)\,dx$ suppenee. Lisäksi rajalla $N\kohti\infty$ saadaan arviot
\begin{align*}
           &\sum_{k=n+1}^\infty f(k) 
              \,\le\, \int_n^\infty f(x)\,dx \,\le\, \sum_{k=n+1}^\infty f(k) + f(n) \\
\ekv \quad &\int_n^\infty f(x)\,dx - f(n) 
              \,\le\, \sum_{k=n+1}^\infty f(k) \,\le\, \int_n^\infty f(x)\,dx.
\end{align*}
Tästä seuraa sarjan summan laskemisen kannalta mielenkiintoinen tulos:
\begin{align*}
\sum_{k=1}^\infty f(k) &\,=\, \sum_{k=1}^{n} f(k) + \int_n^\infty f(x)\,dx\,-\eps_n \\
                      &\,=\, \sum_{k=1}^{n-1} f(k) + \int_n^\infty f(x)\,dx\,+\delta_n,
\end{align*}
missä $\,0 \le \eps_n \le f(n)\,$ ja $\,0 \le \delta_n \le f(n)\,$ ($\delta_n=f(n)-\eps_n$).
Tämä tarkoittaa, että kun sarjan 'häntä' (indeksistä $n$ tai $n+1$ alkaen) lasketaan 
integraalina, niin tällä tavoin saatavista approksimaatioista
\begin{align}
\sum_{k=1}^\infty f(k)\ 
         &\approx\ \sum_{k=1}^{n-1} f(k) + \int_n^\infty f(x)\,dx \label{appr a} \tag{a} \\
         &\approx\ \sum_{k=1}^{n} f(k) + \int_n^\infty f(x)\,dx \label{appr b} \tag{b}
\end{align}
\eqref{appr a} antaa sarjan summalle alalikiarvon ja \eqref{appr b} ylälikiarvon, ja kummankin
virhe on enintään $f(n)=$ sarjan $n$:s termi (= approksimaatioiden erotus). Tulos on siis
pätevä, mikäli em.\ oletus (i) on voimassa.
%Seuraavaa esimerkkiä voi verrata aiempiin suppenemistarkasteluihin Luvussa \ref{potenssisarja}
%(Lause \ref{harmoninen sarja}).
\begin{Exa} \label{sarja vs integraali}
Millä $\alpha$:n arvoilla ($\alpha\in\R$) sarja $\,\sum_{k=1}^\infty 1/k^\alpha\,$ suppenee?
\end{Exa}
\ratk Sarja epäilemättä hajaantuu, jos $\alpha \le 0$. Jos $\alpha>0$, niin oletus (i) on
voimassa, kun $n=1$. Integraali $\,\int_1^\infty x^{-\alpha}\,dx\,$ suppenee täsmälleen kun
$\alpha>1$, joten sarja suppenee täsmälleen samalla ehdolla. \loppu

Esimerkin kysymys ratkaistiin jo aiemmin (Lause \ref{harmoninen sarja}) mutta vaivalloisemmin.
Paitsi yksinkertaistaa suppenemistarkastelun, integraaliin vertaaminen helpottaa usein myös
sarjan summan numeerista laskemista.
\jatko \begin{Exa} (jatko) Esimerkin suppenevan (yliharmonisen) sarjan summalle pätee
approksimaation (b) perusteella
\[
\sum_{k=1}^\infty\frac{1}{k^\alpha}\ 
   \approx\ \sum_{k=1}^n\frac{1}{k^\alpha}+\frac{1}{\alpha-1}\,n^{1-\alpha}, \quad \alpha>1.
\]
Tämä on ylälikiarvo, jota voi verrata pelkkään sarjan katkaisuun eli alalikiarvoon
\[
\sum_{k=1}^\infty\frac{1}{k^\alpha}\ \approx\ \sum_{k=1}^n\frac{1}{k^\alpha}\,.
\]
Edellisen virhe on enintään $\,n^{-\alpha}$, jälkimmäisen 
$n^{1-\alpha}/(\alpha-1)+\mathcal{O}(n^{-\alpha})$. Esim.\ jos $\alpha=5/4$, on tarkkuusero
käytännössä dramaattinen, ks.\ Harj.teht.\,\ref{H-int-7: hidas sarja}. \loppu
\end{Exa}
Huomautettakoon, että esimerkissä vieläkin tehokkaampi algoritmivaihtoehto on
approksimaatioiden (a) ja (b) keskiarvo eli
\[
\sum_{k=1}^\infty \frac{1}{k^\alpha} \,\approx\,
\sum_{k=1}^n\frac{1}{k^\alpha}+\frac{1}{\alpha-1}\,n^{1-\alpha}-\frac{1}{2}\,n^{-\alpha}\,.
\]
Tämän virhe on suuruusluokkaa $\mathcal{O}(n^{-1-\alpha})$
(ks.\ Harj.teht.\,\ref{numeerinen integrointi}:\ref{H-int-9: hidas sarja}).

\Harj
\begin{enumerate}

\item
Laske tai osoita hajaantuvaksi kohdissa a)--o). Kohdissa p)--ö) määritä $k$:n arvot ($k\in\R$),
joilla integraali suppenee.
\begingroup
\allowdisplaybreaks
\begin{align*}
&\text{a)}\ \ \int_0^6 \frac{2x}{x^2-4}\,dx \qquad
 \text{b)}\ \ \int_0^6 \frac{2x}{\abs{x^2-4}^{2/3}}\,dx \qquad
 \text{c)}\ \ \int_0^1 \frac{x^5}{(1-x^2)^{3/2}}\,dx \\
&\text{d)}\ \ \int_0^1 x^{-4/5}\ln x\,dx \qquad
 \text{e)}\ \ \int_0^{\pi/2} \tan x\,dx \qquad
 \text{f)}\ \ \int_0^\pi \frac{x}{\cos^2 x}\,dx \\
&\text{g)}\ \ \int_0^2 \frac{x^2}{\sqrt{2x-x^2}}\,dx \qquad
 \text{h)}\ \ \int_{-\infty}^\infty \frac{1}{x^2+7x+12}\,dx \qquad
 \text{i)}\ \ \int_0^\infty \frac{1}{x^3+1}\,dx \\
&\text{j)}\ \ \int_{-\infty}^\infty \frac{1}{x^2+7x+13}\,dx \qquad
 \text{k)}\ \ \int_0^\infty \frac{1}{x^3-1}\,dx \qquad
 \text{l)}\ \ \int_1^\infty \frac{1+\sqrt{x}}{x^2+x}\,dx \\
&\text{m)}\ \ \int_1^\infty \frac{1+\sqrt{x}}{x\sqrt{x}+2x}\,dx \qquad
 \text{n)}\ \ \int_0^\infty \frac{1}{\sqrt{e^x-1}}\,dx \qquad
 \text{o)}\ \ \int_{-\infty}^\infty e^{-\abs{x}}\cos x\,dx \\
&\text{p)}\ \ \int_0^1 \frac{1+x+x^2}{x^k}\,dx \qquad
 \text{q)}\ \ \int_1^\infty \frac{1+x+x^2}{x^k}\,dx \qquad
 \text{r)}\ \ \int_0^\infty \frac{1}{kx^2+1}\,dx \\
&\text{s)}\ \ \int_0^1 \frac{x^{k-1}+x^{-k}}{1+x}\,dx \qquad
 \text{t)}\ \ \int_0^{\pi/2} \sin^k x\,dx \qquad
 \text{u)}\ \ \int_0^\infty x^k e^{-x}\,dx \\
&\text{v)}\ \ \int_0^\infty x^k\sin x\,dx \qquad
 \text{x)}\ \ \int_0^\infty \frac{\abs{k}^x}{\sqrt{e^x+1}}\,dx \qquad
 \text{y)}\ \ \int_0^\infty x^k\ln x\,dx \\
&\text{z)}\ \ \int_0^1 \frac{e^x}{(1-x^2)^k}\,dx \qquad
 \text{å)}\ \ \int_0^\infty \abs{x^2-1}^k e^{-x}\,dx \qquad
 \text{ä)}\ \ \int_{-\infty}^\infty \frac{\ln\abs{x}}{\abs{x^3+1}^k}\,dx \\
&\text{ö)}\ \ \int_{-\infty}^\infty\left[\sqrt[4]{|\sin x\cos x|}\ e^{|x|}\,(\ln|x|)^8\right]^kdx
\end{align*}%
\endgroup

\item
Johda palautuskaava seuraaville integraaleille, kun $n\in\N\cup\{0\}$.
\[
\text{a)}\ \ \int_0^\infty x^n e^{-x}\,dx \qquad
\text{b)}\ \ \int_0^\infty \frac{1}{(x^2+1)^{n+1}}\,dx \qquad
\text{c)}\ \ \int_0^1 (\ln x)^n\,dx
\]

\item
Todista suppeneminen sijoituksella $t=1/x$\,:
\[
\text{a)}\ \ \int_0^1 e^{-1/x}\,dx \qquad
\text{b)}\ \ \int_0^1 \cos\frac{1}{x}\,dx \qquad
\text{c)}\ \ \int_0^\infty \frac{1}{x}\sin\frac{1}{x}\,dx
\]

\item
Sievennä ja piirrä kuvaaja:
\[
f(x)=\int_0^\pi \frac{x\sin t}{\sqrt{1+2x\cos t+x^2}}\,dx.
\]

\item \label{H-int-7: Gamma} \index{Gammafunktio ($\Gamma$-funktio)}
\kor{Gammafunktio} ($\Gamma$-funktio) määritellään 
\[
\Gamma(x)=\int_0^{\infty} t^{x-1} e^{-t}\,dt.
\]
Näytä tosiksi seuraavat väittämät ja hahmottele näiden perusteella
$\Gamma$-\-funktion kulku. (Kohdassa e) on $\Gamma(1/2)=\sqrt{\pi} \approx 1.77$, ks.\
Propositio \ref{Gamma(1/2)}.) \vspace{1.5mm}\newline
a) \ $\Gamma$ on määritelty välillä $(0,\infty)$. \newline
b) \ $\Gamma(x+1) = x \Gamma(x)\ \forall x>0$. \newline
c) \ $\Gamma(n)=(n-1)!\,\ \forall n\in\N$. \newline
d) \ $\lim_{x \kohti 0^+} \Gamma(x)=\lim_{x\kohti\infty} \Gamma(x)=\infty$. \newline
e) \ $\Gamma(1/2)=2\,\Gamma(3/2)=\int_{-\infty}^\infty e^{-x^2}\,dx$.
\item
a) Halutaan selvittää, millä $\alpha$:n arvoilla ($\alpha\in\R$) sarja
\[
\sum_{k=2}^\infty \frac{1}{k(\ln k)^\alpha}
\]
suppenee. Ratkaise kysymys integraaliin vertaamalla. \vspace{1mm}\newline
b) Olkoon $n\in\N$ suuri luku. Perustele integraalin avulla arviot
\[
\sum_{k=1}^n k^\alpha \,=\,
    \begin{cases}
    \,\frac{1}{\alpha+1}\,n^{\alpha+1}+\mathcal{O}(n^\alpha), &\text{kun}\ \alpha>-1, \\
    \,\ln n+\mathcal{O}(n^{-1}),                               &\text{kun}\ \alpha=-1.
                        \end{cases}
\]
Mikä on jäännöstermi $\mathcal{O}(n^\alpha)$ tarkasti tapauksissa $\alpha=0$ ja $\alpha=1$\,?

\item \label{H-int-7: hidas sarja}
Käytettävissä on tietokone, joka laskee summan $s_n=\sum_{k=1}^n k^{-5/4}$ ajassa $10^{-10}n$
sekuntia. Summien $s_n$ avulla halutaan laskea raja-arvo $s=\lim_ns_n$ kuuden desimaalin
tarkkuudella (virhe < $5 \cdot 10^{-7}$). Arvioi laskenta-aika \newline
a) vuosimiljardeina approksimaatioilla $\,s \approx s_n$, \newline
b) mikrosekunteina approksimaatioilla $\,s \approx s_n+4/\sqrt[4]{n}$.

\item (*) \index{Eulerin!b@vakio}
\kor{Eulerin vakio} $\gamma=0.5772156649..\,$ määritellään
\[
\gamma\,=\,\lim_{n\kohti\infty} \left(\sum_{k=1}^n \frac{1}{k}\,-\,\ln n\right)\,
        =\,\sum_{k=1}^\infty a_k, \quad a_k=\frac{1}{k}-\ln\left(1+\frac{1}{k}\right).
\]
a) Perustele jälkimmäinen laskukaava sekä ko.\ sarjan suppeneminen. \newline
b) Näytä, että suurilla $n\in\N$ pätee
   $\,\sum_{k=1}^na_k + \frac{1}{2n} = \gamma + \mathcal{O}(n^{-2})$.
        
\item (*)
Arvioi luku $n!$ sekä ylhäältä että alhaalta vertaamalla lukua 
$\ln(n!) = \ln 1 + \ln 2 + \dots + \ln n$ integraaliin. Kuinka moninumeroinen luku on $1000!$
(kymmenjärjestelmässä)\,?

\end{enumerate} %Riemannin integraalin laajennukset
\section{Pinta-ala ja kaarenpituus} \label{pinta-ala ja kaarenpituus}
\alku
\index{pinta-ala!c@tasoalueen|vahv}

Tässä luvussa tarkastellaan määrätyn integraalin käyttöä kahdessa tasogeometrisessa tehtävässä:
käyrän viivan rajoittaman \kor{tasoalueen} ($\Rkaksi$:n tai $E^2$:n osajoukon) pinta-alan 
määrittämisessä ja yksinkertaisen käyränkaaren kaarenpituuden laskemisessa.

Olkoon $f(x) \ge 0$ välillä $[a,b]$. Tarkastellaan käyrän $y=f(x)$ ja suorien $y=0$, $x=a$ ja
$x=b$ rajaamaa $\Rkaksi$:n osajoukkoa
\[
A = \{\,(x,y) \in \Rkaksi \mid x \in [a,b]\ \ja\ 0 \le y \le f(x)\,\}.
\]
\begin{figure}[H]
\setlength{\unitlength}{1cm}
\begin{center}
\begin{picture}(10,6)(-1,-1)
\put(-1,0){\vector(1,0){10}} \put(8.8,-0.4){$x$}
\put(0,-1){\vector(0,1){6}} \put(0.2,4.8){$y$}
\put(1.5,0){\line(0,1){3.765}} \put(7,0){\line(0,1){3.3179}}
\put(1.4,-0.5){$a$} \put(6.9,-0.5){$b$}
\put(3,1.5){$A$} \put(8,3.7){$y=f(x)$}
\curve(
    1.0000,    3.0000,
    1.5000,    3.7635,
    2.0000,    4.0000,
    2.5000,    3.8848,
    3.0000,    3.5679,
    3.5000,    3.1733,
    4.0000,    2.8000,
    4.5000,    2.5211,
    5.0000,    2.3843,
    5.5000,    2.4117,
    6.0000,    2.6000,
    6.5000,    2.9202,
    7.0000,    3.3179,
    7.5000,    3.7130,
    8.0000,    4.0000)
\end{picture}
\end{center}
\end{figure}
Halutaan määrittää $A$:n pinta-ala. Tämä on ei-negatiivinen reaaliluku, jota merkitään
jatkossa symbolilla $\mu(A)$. --- Kyseessä on itse asiassa funktio tyyppiä
$\mu:\ \mathcal{M} \kohti [0,\infty)$, missä $\mathcal{M}$ koostuu $\Rkaksi$:n osajoukoista,
tarkemmin sanoen sellaisista 
\index{mitta, mitallisuus}%
nk.\ \kor{mitallisista} joukoista, joiden pinta-ala on
määriteltävissä. Lukua $\mu(A)$ sanotaan tämän mukaisesti myös $A$:n (pinta-ala)\kor{mitaksi}
(engl.\ measure)\footnote[2]{Käsitteet 'mitta' ja 'mitallisuus' viittaavat matematiikan lajiin
nimeltä \kor{mittateoria}. Yleisemmän mittateorian perusideoita ei tässä yhteydessä
tarkastella, vaan asiaan palataan myöhemmin usean muuttujan analyysin yhteydessä.}. Ajatellen
toistaiseksi vain oletettua muotoa olevia tasoalueita asetetaan pinta-alamitalle $\mu$ seuraavat
kaksi aksioomaa:
\index{pinta-alamitta!a@tason|(}%
\begin{itemize}
\item[A1.] \kor{Vertailuperiaate}: Jos $0 \le m \le f(x) \le M\ \forall x\in[a,b]$,
           niin \index{vertailuperiaate!b@mittojen}%
           \[ 
           m(b-a) \le \mu(A) \le M(b-a). 
           \]
\item[A2.] \kor{Additiivisuus}: Jos $A_1=\{\,(x,y) \in A \mid x\in[a,c]\,\}$ ja
           $A_2=\{\,(x,y) \in A \mid x\in[b,c]\,\}$, missä $a<c<b$, niin $A$ on mitallinen
           täsmälleen kun $A_1$ ja $A_2$ ovat molemmat mitallisia, ja pätee
           \index{additiivisuus!b@mitan}%
           \[ 
            \mu(A) = \mu(A_1)+\mu(A_2). 
           \]
\end{itemize} 
\begin{figure}[H]
\setlength{\unitlength}{1cm}
\begin{center}
\begin{picture}(10,6)(-1,-1)
\put(-1,0){\vector(1,0){10}} \put(8.8,-0.4){$x$}
\put(0,-1){\vector(0,1){6}} \put(0.2,4.8){$y$}
\put(1.5,0){\line(0,1){3.765}} \put(7,0){\line(0,1){3.3179}} \put(4,0){\line(0,1){2.8}}
\put(1.4,-0.5){$a$} \put(6.9,-0.5){$b$} \put(3.9,-0.5){$c$}
\put(2.6,1.3){$A_1$} \put(5.3,1.3){$A_2$} \put(8,3.7){$y=f(x)$}
\curve(
    1.0000,    3.0000,
    1.5000,    3.7635,
    2.0000,    4.0000,
    2.5000,    3.8848,
    3.0000,    3.5679,
    3.5000,    3.1733,
    4.0000,    2.8000,
    4.5000,    2.5211,
    5.0000,    2.3843,
    5.5000,    2.4117,
    6.0000,    2.6000,
    6.5000,    2.9202,
    7.0000,    3.3179,
    7.5000,    3.7130,
    8.0000,    4.0000)
\end{picture}
\end{center}
\end{figure}
Oletetaan nyt $A$ sellaiseksi, että aksioomien A1--A2 mukainen pinta-alamitta $\mu(A)$ on 
määriteltävissä, ja tutkitaan, mitä tästä seuraa. Ensinnäkin todetaan additiivisuusaksioomaa
A2 toistuvasti soveltamalla, että jos $\X=\{x_k,\ k=0 \ldots n\}$ on välin $[a,b]$ jako 
(ts.\ $a=x_0<x_1< \ldots <x_n=b$), ja merkitään
\[
A_k = \{\,(x,y) \in A \mid x_{k-1} \le x \le x_k\,\}, \quad k = 1 \ldots n,
\]
niin jokainen $A_k$ on mitallinen ja
\[
\mu(A) = \sum_{k=1}^n \mu(A_k).
\]
Vertailuperiaatteen A1 mukaan pätee tässä
\begin{align*}
m_k \le f(x) &\le M_k\ \forall x\in[x_{k-1},x_k] \\[2mm]
             &\qimpl m_k(x_k-x_{k-1}) \le \mu(A_k) \le M_k(x_k-x_{k-1}) \\
             &\qimpl \mu(A_k) = \eta_k (x_k-x_{k-1}), \quad \eta_k\in[m_k,M_k].
\end{align*}
Jos $f$ on jatkuva välillä$[a,b]$, niin Weierstrassin lauseen 
(Lause \ref{Weierstrassin peruslause}) mukaan voidaan tässä valita
\[
m_k = \min_{x\in[x_{k-1},x_k]}\{f(x)\}, \quad M_k = \max_{x\in[x_{k-1},x_k]}\{f(x)\},
\]
jolloin Ensimmäisen väliarvolauseen (Lause \ref{ensimmäinen väliarvolause}) mukaan on olemassa 
$\xi_k\in[x_{k-1},x_k]$ siten, että $f(\xi_k)=\eta_k$.

Yhdistämällä em.\ päätelmät todetaan, että jos $f$ on jatkuva välillä $[a,b]$, niin jokaiseen
välin $[a,b]$ jakoon $\X$ liittyen $\mu(A)$ on ilmaistavissa eräänä ko.\ jakoon liittyvänä 
Riemannin summana:
\[
\mu(A) = \sum_{k=1}^n f(\xi_k)(x_k-x_{k-1}), \quad \xi_k\in[x_{k-1},x_k], \quad k = 1 \ldots n.
\]
Koska Analyysin peruslauseen ja Riemannin integraalin määritelmän mukaan tämän kaavan oikealla
puolella olevat summat lähestyvät määrättyä integraalia $\int_a^b f(x)\,dx$ jaon tihetessä,
riippumatta pisteiden $\xi_k$ valinnasta, niin seuraa, että $\mu(A)$ on määriteltävä kaavalla
\begin{equation} \label{pinta-alakaava}
\boxed{\quad\kehys \mu(A) = \int_a^b f(x)\,dx. \quad}
\end{equation}
Näin määritellylle mitalle aksioomat A1--A2 myös toteutuvat integraalin vastaavien
ominaisuuksien (Lauseet \ref{integraalien vertailuperiaate} ja \ref{integraalin additiivisuus})
perusteella. On siis päätelty, että jos $f$ on jatkuva välillä $[a,b]$, niin $A$ on aksioomien
A1--A2 mukaisesti mitallinen jos (ja vain jos!) mitta määritellään kaavalla
\eqref{pinta-alakaava}. \index{pinta-alamitta!a@tason|)}
\begin{Exa} \label{Arkhimedeen kaava} \index{paraabelin segmentti} \kor{Paraabelin segmentin}
\[
A = \{\,(x,y) \in \Rkaksi \mid x\in[0,2]\ \ja\ 0 \le y \le 2x-x^2\,\}
\]
pinta-ala on kaavan \eqref{pinta-alakaava} mukaisesti\footnote[2]{Paraabelin segmentin
pinta-alan laski ensimmäisenä antiikin huomattavin matemaatikko \hist{Arkhimedes}
(287-212 eKr). Arkhimedeen menetelmä perustui segmenttiä approksimoiviin monikulmioihin, ks.\
Harj.teht.\,\ref{numeerinen integrointi}:\ref{H-int-9: Arkhimedes}. \index{Arkhimedes|av}}
\[
\mu(A) = \int_0^2 (2x-x^2)\,dx = \sijoitus{0}{2}\left(x^2-\frac{1}{3}\,x^3\right) 
                               = \underline{\underline{\frac{4}{3}}}\,. \loppu
\]
\end{Exa}

Laskukaava \eqref{pinta-alakaava} seuraa myös pelkästään olettamalla, että $f$ on välillä
$[a,b]$ rajoitettu ja Riemann-integroituva. Nimittäin jos $\X$ on välin $[a,b]$ jako, niin 
em.\ päätelmien mukaisesti voidaan $\mu(A)$ (sikäli kuin olemassa) arvioida Riemannin
ylä- ja alasummilla (ks.\ Luku \ref{riemannin integraali}):
\[
\underline{\sigma}(f,\X) \le \mu(A) \le \overline{\sigma}(f,\X).
\]
Tällöin seuraa Lauseesta \ref{Riemann-integroituvuus}, että mitta on määriteltävä
kaavalla \eqref{pinta-alakaava}, jolloin aksioomat A1--A2 myös toteutuvat. On päädytty
seuraavaan tulokseen. 
\begin{Lause} Jos $f$ on määritelty, rajoitettu ja Riemann-integroituva välillä $[a,b]$ ja
$f(x) \ge 0,\ x\in[a,b]$, niin tasoalueen
\[
A = \{\,(x,y) \in \Rkaksi \mid x \in [a,b]\ \ja\ 0 \le y \le f(x)\,\}
\]
pinta-ala $\mu(A)$ määräytyy aksioomista A1--A2 yksikäsitteisesti kaavalla
\eqref{pinta-alakaava}.
\end{Lause}
\begin{Exa} Laske ympyräneljänneksen
\[
A = \{\,(x,y) \in \Rkaksi \mid x,y \ge 0\ \ja\ x^2+y^2 \le R^2\,\}
\]
pinta-ala. \end{Exa}
\ratk Kaavan \eqref{pinta-alakaava} mukaan laskien saadaan
\begin{align*}
\mu(A) &= \int_0^R \sqrt{R^2-x^2}\,dx \\
       &\qquad [\ \text{sijoitus}\,\ x=R\sin t,\,\ dx=R\cos t\,dt\ ] \\
       &= \int_0^{\pi/2} R^2\cos^2 t\, dt \\
       &= \sijoitus{0}{\pi/2} \frac{1}{2}\,R^2(t+\sin t \cos t) 
        = \underline{\underline{\frac{1}{4}\,\pi R^2}}. \loppu
\end{align*}
\begin{Exa} Määritä $\mu(A)$, kun
\[
A = \{\,(x,y) \in \Rkaksi \mid x\in[0,2]\ \ja\ 0 \le y \le \max\{x,2e^{-x}\}\,\}.
\]
\end{Exa}
\begin{multicols}{2} \raggedcolumns
\ratk Tässä on ensin ratkaistava
\[
x = 2e^{-x}\ \impl\ x=c \approx 0.852606,
\]
jolloin on (vrt.\ kuva)
\[
f(x) = \begin{cases} 
       \,2e^{-x}, &\text{kun}\ x\in[0,c], \\ x, &\text{kun}\ x\in[c,2].
\end{cases}
\]
\begin{figure}[H]
\setlength{\unitlength}{1.5cm}
\begin{center}
\begin{picture}(10,3)(-0.5,0)
\put(-0.5,0){\vector(1,0){4}} \put(3.3,-0.35){$x$}
\put(0,-0.5){\vector(0,1){3}}
\put(0.8526,0.8526){\line(1,1){1.1474}}
\put(0.8526,0){\line(0,1){0.8526}} \put(2,0){\line(0,1){2}}
\curve(
0.0000, 2.0000,
0.2000, 1.6375,
0.4000, 1.3406,
0.6000, 1.0976,
0.8526, 0.8526)
\put(-0.3,1.9){$2$}
\put(1.9,-0.4){$2$} \put(0.7526,-0.4){$c$}
\end{picture}
\end{center}
\end{figure}
\end{multicols}
Kaavan \eqref{pinta-alakaava} ja integraalin additiivisuuden perusteella saadaan
\begin{align*}
\mu(A) = \int_0^2 f(x)\,dx 
             &= \int_0^c 2e^{-x}\,dx + \int_c^2 x\,dx \\
             &= \sijoitus{0}{c}(-2e^{-x}) + \sijoitus{c}{2}\tfrac{1}{2}x^2 \\
             &= 4-2e^{-c}-\tfrac{1}{2}c^2 \approx \underline{\underline{2.78393}}. \loppu
\end{align*}

Sopimalla pinta-alamitalle lisää ominaisuuksia voidaan integraalien avulla laskea myös 
yleisempien tasoalueiden pinta-aloja. Tarkasteltakoon tässä ainoastaan esimerkkinä kahden
käyrän väliin jäävän alueen pinta-alaa. Olkoon
$0 \le g(x) \le f(x)\ \forall x\in[a,b]$. Halutaan märätä $\mu(A)$, kun
\[
A = \{(x,y)\in\Rkaksi \mid x\in[a,b]\ \ja\ g(x) \le y \le f(x)\}.
\]
\begin{figure}[H]
\setlength{\unitlength}{1cm}
\begin{center}
\begin{picture}(10,6)(-1,-1)
\put(-1,0){\vector(1,0){10}} \put(8.8,-0.4){$x$}
\put(0,-1){\vector(0,1){6}} \put(0.2,4.8){$y$}
\put(1.5,0.9375){\line(0,1){2.8275}} \put(5.5,1.5375){\line(0,1){0.8742}}
\put(1.5,0){\line(0,1){0.1}} \put(5.5,0){\line(0,1){0.1}}
\put(1.4,-0.5){$a$} \put(5.4,-0.5){$b$}
\put(3,2){$A$} \put(6,1.6){$y=g(x)$} \put(3,3.8){$y=f(x)$}
\curve(
    1.0000,    3.0000,
    1.5000,    3.7635,
    2.0000,    4.0000,
    2.5000,    3.8848,
    3.0000,    3.5679,
    3.5000,    3.1733,
    4.0000,    2.8000,
    4.5000,    2.5211,
    5.0000,    2.3843,
    5.5000,    2.4117,
    6.0000,    2.6000,
    6.5000,    2.9202)
\curve(
    1.0000,    1.2000,
    1.2500,    1.0534,
    1.5000,    0.9375,
    1.7500,    0.8344,
    2.0000,    0.7500,
    2.2500,    0.6844,
    2.5000,    0.6375,
    2.7500,    0.6034,
    3.0000,    0.6000,
    3.2500,    0.6034,
    3.5000,    0.6375,
    3.7500,    0.6844,
    4.0000,    0.7500,
    4.2500,    0.8344,
    4.5000,    0.9375,
    4.7500,    1.0534,
    5.0000,    1.2000,
    5.5000,    1.5375,
    6.0000,    1.9500,
    6.5000,    2.4375)
\end{picture}
\end{center}
\end{figure}
Kun merkitään
\begin{align*}
A_1 &= \{(x,y)\in\Rkaksi \mid x\in[a,b]\ \ja\ 0 \le y \le g(x)\}, \\
A_2 &= \{(x,y)\in\Rkaksi \mid x\in[a,b]\ \ja\ 0 \le y \le f(x)\},
\end{align*}
niin olettamalla additiivisuuslaski $\mu(A_2)=\mu(A_1)+\mu(A)$ (pinta-alamitan uusi aksiooma!)
seuraa laskukaava
\[
\mu(A)=\mu(A_2)-\mu(A_1)=\int_a^b [f(x)-g(x)]\,dx.
\]
Tätä voidaan pitää pätevänä aina kun $f$ ja $g$ ovat integroituvia välillä $[a,b]$ ja
$f(x) \ge g(x)\ \forall x\in[a,b]$. Ellei jälkimmäinen oletus ole voimassa, sovelletaan kaavaa 
asettamalla $f$:n tilalle $\max\{f,g\}$ ja $g$:n tilalle $\min\{f,g\}$, jolloin yleiseksi
laskukaavaksi kahden käyrän väliin jäävän alueen pinta-alalle välillä $[a,b]$ tulee
\begin{equation} \label{pinta-alakaava 2}
\boxed{\kehys\quad \mu(A)=\int_a^b \abs{f(x)-g(x)}\,dx. \quad}
\end{equation}
\begin{Exa} Laske käyrien $y=x$ ja $y=2e^{-x}$ väliin jäävän alueen pinta-ala välillä $[0,2]$.
\end{Exa}
\ratk Kaavan \eqref{pinta-alakaava 2} ja edellisen esimerkin perusteella
\begin{align*}
\mu(A) = \int_0^2 \abs{2e^{-x}-x}\,dx &= \int_0^c (2e^{-x}-x)\,dx+\int_c^2(x-2e^{-x})\,dx \\
                                      &= \sijoitus{0}{c}\left(-2e^{-x}-\frac{1}{2}\,x^2\right)
                                       + \sijoitus{c}{2}\left(\frac{1}{2}\,x^2+2e^{-x}\right) \\
                                      &=4+2e^{-2}-4e^{-c}-c^2 
                                       \approx \underline{\underline{1.83852}}. \loppu
\end{align*}                     

\subsection{Kaarenpituus}
\index{kaarenpituus|vahv}

Luvussa \ref{kaarenpituus} esitetyn määritelmän mukaisesti voidaan kaaren
\[
S = \{\,(x,y) \in \Rkaksi \mid x\in[a,b]\ \ja\ y=f(x)\,\}
\]
\index{kaarenpituusmitta}%
\kor{kaarenpituusmitta} $\mu(S)$ määritellä välin $[a,b]$ jakoihin $\X$ liittyvien summien
\[
\sigma(f,\X) = \sum_{k=1}^n \sqrt{(x_k-x_{k-1})^2+[f(x_k)-f(x_{k-1})]^2}
\]
pienimpänä ylärajana
\[
\mu(S) = \sup_\X \sigma(f,\X),
\]
sikäli kuin summien joukko on rajoitettu. Jos oletetaan, että $f$ on jatkuva välillä $[a,b]$ ja
derivoituva välillä $(a,b)$, niin Differentiaalilaskun väliarvolauseen mukaan on olemassa 
$\,\xi_k\in(x_{k-1},x_k),\ k=1 \ldots n\,$ siten, että
\[
f(x_k)-f(x_{k-1}) = f'(\xi_k)(x_k-x_{k-1}), \quad k=1 \ldots n,
\]
jolloin
\[
\sigma(f,\X) = \sum_{k=1}^n \sqrt{1+[f'(\xi_k)]^2}\,(x_k-x_{k-1}).
\]
Mainituin oletuksin summa $\sigma(f,\X)$ on siis eräs välin $[a,b]$ jakoon $\X$  ja funktioon
$g(x)=\sqrt{1+[f'(x)]^2}\,$ liittyvä Riemannin summa. Päätellään, että sikäli kuin $g$ on
Riemann-integroituva välillä $[a,b]$ (Analyysin peruslauseen mukaan riittää, että $f$ on 
jatkuvasti derivoituva välillä $[a,b]$), on kaarenpituusmitta määriteltävissä ja pätee 
integraalikaava
\begin{equation} \label{kaarenpituuskaava}
\boxed{\kehys\quad \mu(S) = \int_a^b \sqrt{1+[f'(x)]^2}\,dx. \quad}
\end{equation}
\begin{Exa} \label{ympyrän kaari: pituus} Laske kaarenpituus puoliympyrän kaarelle
\[
S = \{\,(x,y) \in \Rkaksi \mid -R \le x \le R\ \ja\ y=\sqrt{R^2-x^2}\,\}.
\]
\end{Exa}
\ratk Tässä on $\,f(x)=\sqrt{R^2-x^2}$, joten kaavan \eqref{kaarenpituuskaava} mukaan on
\[
\mu(S) \,=\, \int_{-R}^R \sqrt{1+\frac{x^2}{R^2-x^2}}\,dx 
       \,=\, \int_{-R}^R \frac{R}{\sqrt{R^2-x^2}}\,dx.
\]
Tämä on suppeneva (epäoleellinen) integraali. Sijoituksella $x=R\cos t,\ t\in[0,\pi]$ saadaan
\[
\mu(S) = \int_\pi^0 (-R)\,dt = \int_0^\pi R\,dt = \underline{\underline{\pi R}}. \loppu
\]
\begin{Exa} Laske käyrän $y=x^2$ kaarenpituus välillä $[0,1]$. \end{Exa}
\ratk Kaavan \eqref{kaarenpituuskaava} ja Luvun \ref{osittaisintegrointi} Esimerkin
\ref{paha integraali} perusteella (sijoitus $2x=t$)
\begin{align*}
\mu(S) = \int_0^1 \sqrt{1+4x^2}\,dx 
      &= \sijoitus{t=0}{t=2}
        \frac{1}{4}\left[t\sqrt{1+t^2}+\ln\left(t+\sqrt{1+t^2}\right)\right] \\
      &= \frac{1}{2}\sqrt{5}+\frac{1}{4}\ln(2+\sqrt{5}) \approx \underline{\underline{1.47894}}. 
                                                                                       \loppu
\end{align*}
\begin{Exa} Käyrän $y=\sin x$ kaarenpituus välillä $[0,\pi]$ on
\[
\mu(S)=\int_0^\pi \sqrt{1+\cos^2 x}\,dx.
\]
Tämä on nk.\ 
\index{elliptinen integraali}%
\kor{elliptinen integraali}, jota ei voi laskea suljetussa muodossa. \loppu
\end{Exa} 
Kuten esimerkeistä nähdään, kaarenpituusintegraali voi olla melko yksinkertaisissakin
tapauksissa hankala tai peräti mahdoton suljetussa muodossa integroimisen kannalta.
Vaihtoehtona on tällöin integraalin laskeminen suoraan numeerisin menetelmin
(ks.\ seuraava luku).

\Harj
\begin{enumerate}

\item
Laske seuraavien tasoalueiden pinta-alat (tarkasti tai likiarvona).
\begin{align*}
&\text{a)}\ \ 0 \le y \le \abs{\cos 3x},\,\ x\in[0,\pi] \qquad
 \text{b)}\ \ 0 \le \min\{x,\cos x\},\,\ x\in[0,\pi/2] \\[1mm]
&\text{c)}\ \ \min\{\cos x,\sin 2x\} \le y \le \max\{\cos x,\sin 2x\},\,\ x\in[0,\pi] \\
&\text{d)}\ \ \abs{x} \le y \le \sqrt{x^2+1}\,,\,\ \abs{x} \le \sqrt{a^2-1}\,,\,\ a>1
\end{align*}

\item
Laske seuraavien käyrien väliin jäävän alueen pinta-ala annetulla välillä 
(tarkasti tai likiarvona). \vspace{1mm}\newline
a) \ $y=\sinh x$, $y=\cosh x$, $[-\ln 2,\ln 2] \qquad$
b) \ $y=\ln x$, $y=e^{x}$, $[1/2,2]$ \newline
c) \ $y=\cos x$, $y=x$, $[0,\pi] \hspace{3.1cm}$
d) \ $y=\sin x$, $y=x^2$, $[0,1]$

\item
Laske likiarvo käyrien $y=e^x$ ja $y=3-x^2$ väliin jäävän rajoitetun taso\-alueen pinta-alalle.

\item \label{H-int-8: ellipsin pinta-ala}
Näytä, että ellipsin
\[
S: \quad \frac{x^2}{a^2}+\frac{y^2}{b^2}=1 \quad (a,b>0)
\]
sisään jäävän alueen pinta-ala on $\mu(A)=\pi ab$.

\item
Laske seuraavien käyrien kaarenpituudet annetulla välillä. \vspace{1mm}\newline
a) \ $y=2\sqrt{x},\,\ [0,4] \qquad\qquad\,$
b) \ $y=\ln x,\,\ [1,e]$ \newline
c) \ $y=\ln\cos x,\,\ [0,\pi/4] \qquad$
d) \ $y=(1-x/3)\sqrt{x}\,,\,\ [0,3]$

\item
Näytä, että ellipsin puolikaaren $\,S:\,(x/a)^2+(y/b)^2=1\ \ja\ y \ge 0\,$ pituus saadaan
(elliptisenä) integraalina
\[
\mu(S)=\int_{-\pi/2}^{\pi/2} \sqrt{a^2\cos^2 t+b^2\sin^2 t}\,dt.
\]

\item (*)
Laske käyrien $y=e^{-|x|}\cos x$ ja $y=e^{-|x|}\sin x$ väliin jäävän tasoalueen pinta-ala.

\item (*)
Laske seuraavien alueiden pinta-alat ($a>0$): \vspace{1mm}\newline
a) \ Sykloidin $\,x=a(t-\sin t),\ y=a(1-\cos t)$ ja $x$-akselin väliin jäävä alue välillä
$x\in[0,2\pi a]$. \newline
b)\, Asteroidin $\,\abs{x}^{2/3}+\abs{y}^{2/3}=a^{2/3}$ sisään jäävä alue.

\item (*)
Laske seuraavien kaarien kaarenpituudet ($a>0$): \vspace{1mm}\newline
a) \ Sykloidin kaari $\,S:\ x=a(t-\sin t),\ y=a(1-\cos t),\ t\in[0,2\pi]$. \newline
b)\, Asteroidin kaari $\,S:\ \abs{x}^{2/3}+\abs{y}^{2/3}=a^{2/3},\ x,y \ge 0$.

\item (*) \label{H-int-8: ketjuviiva} \index{ketjuviiva}
Painovoiman vaikuttaessa suunnassa $-\vec j\,$ noudattaa $xy$-tasolla vapaasti riippuva ketju
\kor{ketjuviivaa}
\[
y=a\cosh\left(\frac{x-b}{a}\right)+c,
\]
missä $a>0$ ja $b,c\in\R$. \vspace{1mm}\newline
a) Laske vakiot $a,b,c$ ja hahmottele kejuviivan kulku, kun ketjun päät ovat pisteissä $(-2,0)$
ja $(2,0)$ ja ketjun pituus $=8$. \vspace{1mm}\newline
b) Jos ketjun päät ovat pisteissä $(-2,4)$ ja $(4,4)$ ja ketju kulkee origon kautta, niin mikä
on ketjun pituus?

\end{enumerate}
 %Pinta-ala ja kaarenpituus
\section{Numeerinen integrointi} \label{numeerinen integrointi}
\alku
\index{numeerinen integrointi|vahv}

\kor{Numeerisella integroinnilla} tarkoitetaan määrätyn integraalin, eli reaaliluvun
\[
I(f,a,b)=\int_a^b f(x)\,dx
\]
laskemista numeerisin keinoin (likimäärin). Jatkossa integroitavalle funktiolle $f$ asetetaan
melko voimakkaita säännöllisyysvaatimuksia. Vähimmäisvaatimus on, että $f$ jatkuva välillä
$[a,b]$, jolloin $f$ on ko.\ välillä myös Riemann-integroituva
(Lause \ref{Analyysin peruslause}).

Numeerisen integroinnin menetelmät ovat yleensä nk.\
\index{kvadratuuri!numeerinen} \index{numeerinen kvadratuuri}%
\kor{numeerisia kvadratuureja}
(tai kvadratuurisääntöjä, engl.\ quadrature rule)\footnote[2]{Kvadratuuri tarkoittaa
kirjaimellisesti 'neliöimistä'. Termi viittaa integraalin ja pinta-alan väliseen yhteyteen.},
joissa integraalia approksimoidaan äärellisenä summana muotoa
\begin{equation} \label{yleinen kvadratuuri}
\int_a^b f(x)\,dx\approx\sum_{i=1}^N w_i f(x_i).
\end{equation}
\index{kvadratuuripiste, -paino}%
Tässä pisteitä $x_i,\ i=1 \ldots N$ (yleensä $x_i\in[a,b]$) sanotaan \kor{kvadratuuripisteiksi}
ja lukuja $w_i$ \kor{kvadratuuripainoiksi}. Kvadratuuripisteitä ja -painoja valittaessa 
integroimisväli jaetaan usein ensin osaväleihin $[x_{k-1},x_k],\ k=1 \ldots n\,$ 
(kuten määrätyn integraalin määrittelyssä alunperin). Tällöin integraalin additiivisuuden
nojalla on
\begin{equation} \label{integraalin hajotelma}
I(f,a,b)=\int_a^b f(x)\,dx = \sum_{k=1}^n \int_{x_{k-1}}^{x_k} f(x)\,dx = \sum_{k=1}^n I_k\,.
\end{equation}
Jos integroitava funktio on säännöllinen ja osavälit $[x_{k-1},x_k]$ riittävän lyhyitä, voidaan
osaintegraalit
\[
I_k=\int_{x_{k-1}}^{x_k} f(x)\,dx
\]
laskea likimäärin käyttämällä suhteellisen yksinkertaista kvadratuuria muotoa 
\eqref{yleinen kvadratuuri}, missä $a=x_{k-1}$ ja $b=x_{k}$. Perustana kvadratuurin
yksinkertaistumiselle lyhyellä välillä on Taylorin lause, jonka mukaan säännöllinen funktio on
likimain (matala-asteinen) polynomi lyhyellä välillä (ks.\ Luku \ref{taylorin lause}). 
Numeerisen integroinnin virheanalyysin lähtöajatuksena onkin juuri vertailu polynomeihin.

Jatkossa ajatellaan väli $[a,b]$ alkuperäisen integroimisvälin lyhyeksi osaväliksi ja $f$ ko.\
välillä (riittävän) säännölliseksi. Tällaisten lähtöoletusten vallitessa yksinkertaisimmat
numeeriset kvadratuurit ovat:

\vspace{3mm}

\begin{tabular}{ll}
\kor{Keskipistesääntö}: \index{keskipistesääntö}
        \quad & $\displaystyle{\int_a^b f(x)\,dx\approx (b-a)f(\frac{a+b}{2})}$. \\ \\
\kor{Puolisuunnikassääntö}: \index{puolisuunnikassääntö}
        \quad & $\displaystyle{\int_a^b f(x)\,dx\approx \frac{1}{2}(b-a)[f(a)+f(b)]}$. \\ \\
\kor{Simpsonin sääntö}: \index{Simpsonin sääntö}
        \quad & $\displaystyle{\int_a^b f(x)\,dx\approx 
                         \frac{1}{6}(b-a)\left[f(a)+4f(\frac{a+b}{2})+f(b)\right]}$.
\end{tabular}

\vspace{4mm}

Näistä etenkin Simpsonin sääntö (osaväleillä käytettynä) on laskinten ja tietokoneiden
yleisesti käyttämä numeerisen integroinnin menetelmä.\footnote[2]{Simpsonin sääntö on
numeerisen integroinnin klassikko, jota on aikanaan käytetty paljon käsinlaskussakin. Säännön
keksi englantilainen matemaatikko \hist{Thomas Simpson} (1710-1761). \index{Simpson, T.|av}}
Puolisuunnikassääntöön viitataan usein myös nimellä
\index{trapetsi}%
\kor{Trapetsi}. Nimensä mukaisesti sääntö
antaa integraalin arvoksi \kor{puolisuunnikkaan} (trapetsin) pinta-alan:
\begin{figure}[H]
\setlength{\unitlength}{1cm}
\begin{center}
\begin{picture}(10,5)(0,0)
\put(0,0){\vector(1,0){9}} \put(8.8,-0.4){$x$}
\put(1.5,-1){\vector(0,1){6}} \put(1.7,4.8){$y$}
\put(3.9,-0.5){$a$} \put(6.9,-0.5){$b$} 
\put(5.3,1.3){$A$} \put(8,3.7){$y=f(x)$}
\thicklines
\put(4,0){\line(0,1){2.8}} \put(7,0){\line(0,1){3.3179}} 
\put(4,0){\line(1,0){3}} \drawline(4,2.8)(7,3.3179)
\thinlines
\curve(
    1.0000,    3.0000,
    1.5000,    3.7635,
    2.0000,    4.0000,
    2.5000,    3.8848,
    3.0000,    3.5679,
    3.5000,    3.1733,
    4.0000,    2.8000,
    4.5000,    2.5211,
    5.0000,    2.3843,
    5.5000,    2.4117,
    6.0000,    2.6000,
    6.5000,    2.9202,
    7.0000,    3.3179,
    7.5000,    3.7130,
    8.0000,    4.0000)
\end{picture}
\end{center}
\end{figure}

Kun ym.\ sääntöjä käytetään osaväleillä hajotelmassa \eqref{integraalin hajotelma}, niin
tuloksena on alkuperäisen integraalin approksimaatio muotoa \eqref{yleinen kvadratuuri}.

Sanotaan tällöin, että kyseessä on \kor{yhdistetty} (engl. composite) kvadratuuri.
\index{yhdistetty!a@keskipistesääntö}%
Esimerkiksi \kor{yhdistetty keskipistesääntö} on kvadratuuri muotoa \eqref{yleinen kvadratuuri},
missä valitaan
\[
x_i=\frac{1}{2}(x_{i-1}+x_i), \quad w_i=x_{i}-x_{i-1}\,, \quad i=1 \ldots n=N.
\]
Yhdistetty keskipistesääntö on siis eräs integraalia approksimoivista Riemannin summista.
Suoraan Riemannin summiin perustuvista approksimaatioista tämä on yleensä tarkin 
(ks.\ virhearviot jäljempänä).

Jos hajotelmassa \eqref{integraalin hajotelma} jako on tasavälinen ja vastaavassa yhdistetyssä
kvadratuurissa \eqref{yleinen kvadratuuri} merkitään $\,h=x_i-x_{i-1}$ (= peräkkäisten
kvadratuuripisteiden väli) ja $f_i=f(x_i)$, niin \kor{yhdistetty trapetsi} saa muodon
\index{yhdistetty!b@trapetsi, Simpson}%
\[
\int_a^b f(x)\,dx\approx h\left[\frac{1}{2}f_0+f_1+\cdots + f_{n-1}+\frac{1}{2}f_n\right]
\]
ja \kor{yhdistetty Simpsonin sääntö} muodon
\[
\int_a^b f(x)\,dx
       \approx \frac{h}{3}\left[\,f_0+4f_1+2f_2+\cdots +2f_{2n-2}+ 4f_{2n-1}+f_{2n}\,\right].
\]
Yhdistetyssä Simpsonin säännössä joka toinen kvadratuuripiste on osavälijaon jakopiste ja joka
toinen (suurimmalla painokertoimella varustettu) on osavälin keskipiste. 

\begin{Exa} \label{Trapetsi vastaan Simpson 1}
Kun integraalia
\[
\int_0^1 xe^{-x^2}/(x+1)\,dx
\]
approksimoidaan käyttäen tasavälistä yhdistettyä trapetsia ja Simpsonia, saadaan seuraavat
tulokset:
\begin{center}
\begin{tabular}{lll}
pisteitä & Trapetsi & Simpson \\ \hline \\
3 & 0.17578506065833 & 0.20372346078015 \\
21 & 0.20230210857337 & 0.20256800959472 \\
201 & 0.20256528212086 & 0.20256794027440 \\
2001 \quad & 0.20256791368608 \quad & 0.20256794026753
\end{tabular}
\end{center}
Tässä Simpson on Trapetsia selvästi nopeampi. Tämä on odotettavissa, koska integroitava funktio
on sileä, ks.\ virhearviot jäljempänä. \loppu
\end{Exa}
\begin{Exa} \label{Trapetsi vastaan Simpson 2}
Sama vertailuasetelma kuin edellisessä esimerkissä. Lasketaan
\[
\int_0^1 e^{-x}/(\sqrt{x}+1)\,dx.
\]
Tulokset ovat tässä tapauksessa:
\begin{center}
\begin{tabular}{lll}
pisteitä & Trapetsi & Simpson \\ \hline \\
3 & 0.47363365 & 0.43418824 \\
21 & 0.40741212 & 0.40604377 \\
201 & 0.40520890 & 0.40516458 \\
2001 \quad & 0.40513820 \quad & 0.40513679
\end{tabular}
\end{center}
Konvergenssi on nyt selvästi hitaampaa kuin edellisessä esimerkissä, eikä eri menetelmillä
ole tässä merkittävää eroa. Syynä on ilmeisesti se, että integroitava funktio on vähemmän
säännöllinen kuin edellisessä esimerkissä (jatkuva, mutta ei jatkuvasti derivoituva
integroimisvälillä). \loppu
\end{Exa}
Em.\ esimerkeissä on merkille pantavaa, että kun Esimerkin 2 integraalissa tehdään sijoitus
$\sqrt{x}=t$, on tuloksena Esimerkin 1 integraali (kertoimella 2). Sijoitus siis kannattaa
tässä tehdä, vaikka se ei muuten (suljetussa muodossa integroimisen kannalta) tee tehtävää 
helpommaksi. Myös osittaisintegroinnilla, sarjakehitelmillä ym. voi usein manipuloida tehtävää
niin, että integroitavasta funktiosta tulee säännöllisempi, jolloin numeeriset kvadratuurit
toimivat paremmin.

\subsection{Virhearvioista}

Osavälijakoon perustuvan yhdistetyn kvadratuurin virhe arvioidaan ensin erikseen kullakin
osavälillä. Tarkastellaan jatkossa edellä esitettyä kolmea esimerkkiä: Keskipistesääntö, 
Trapetsi ja Simpson.

Arvioitaessa virhettä yksittäisellä osavälillä on ensimmäisenä tehtävänä tutkia, \pain{kuinka} 
\pain{korkea-asteisille} p\pain{ol}y\pain{nomeille} \pain{kvadratuuri} \pain{on} 
\pain{tarkka}, ts. on määrättävä indeksi $m$ siten, että kvadratuuri on tarkka polynomille
astetta $m$ mutta ei astetta $m+1$. Mainituissa esimerkkitapauksissa on tulos seuraava:
\begin{center}
\begin{tabular}{ll}
Sääntö & \hspace{2mm} Tarkka polynomeille astetta \\ \hline \\
Keskipistesääntö & $\quad m=1$ \\ Trapetsi & $\quad m=1$ \\ Simpson & $\quad m=3$
\end{tabular}
\end{center}

Aloitetaan keskipistesäännöstä. Oletetetaan, että funktio $f$ on tarkasteltavalla (osa)välillä
$[a,b]$ kahdesti jatkuvasti derivoituva. Merkitään $c=(a+b)/2$ ja käytetään
integroimisvirheelle symbolia $E(f)\,$:
\[ 
E(f) = \int_a^b f(x)\,dx - (b-a)f(c). 
\]
Integroimisvirheen arvioiminen perustuu kahteen perushavaintoon: Ensinnäkin nähdään, että pätee
\begin{equation} \label{havainto 1}
E(f+g) = E(f)+E(g). \tag{a} 
\end{equation}
(Yleisemmin $E(f)$ on lineaarinen $f$:n suhteen, eli 
$E(\alpha f + \beta g) = \alpha E(f) + \beta E(g)$, $\alpha,\beta \in \R$.) Toinen
perushavainto on em.\ taulukkoon perustuva:
\begin{equation} \label{havainto 2}
f(x) = A + Bx \quad (A,B \in \R) \qimpl E(f)=0. \tag{b}
\end{equation}
Olkoon nyt $p(x)=f$:n Taylorin polynomi astetta $n=1$ pisteessä $c$, eli
\[ 
p(x) = f(c) + f'(c)(x-c). 
\]
Tällöin on havaintojen \eqref{havainto 1}--\eqref{havainto 2} perusteella, ja koska
$(f-p)(c)=0$,
\[
E(f) = E(f-p+p) = E(f-p) + E(p) = E(f-p) = \int_a^b (f-p)(x)\,dx.
\]
Taylorin lauseen (Lause \ref{Taylor}) perusteella on
\[ 
f(x)-p(x) = \frac{1}{2}f''(\xi)(x-c)^2 = f''(\xi)\,\omega(x), \quad x \in [a,b], 
\]
missä $\xi = \xi(x) \in (a,b)$ ja $\,\omega(x)=\tfrac{1}{2}(x-c)^2 \ge 0$. Näin ollen jos
\[ 
M_1 = \min_{x\in[a,b]} f''(x), \quad M_2 = \max_{x\in[a,b]} f''(x), 
\]
niin voidaan päätellä
\begin{align*}
&M_1\,\omega(x) \,\le\, (f-p)(x) \,\le\, M_2\,\omega(x), \quad x \in [a,b] \\[1mm]
&\impl\quad M_1\,\int_a^b \omega(x)\,dx 
                \,\le\, \int_a^b (f-p)(x)\,dx \,\le\, M_2\,\int_a^b \omega(x)\,dx \\
&\ekv\quad \frac{1}{24}\,M_1\,(b-a)^3    
                \,\le\, \int_a^b (f-p)(x)\,dx \,\le\, \frac{1}{24}\,M_2\,(b-a)^3.
\end{align*}
Siis jollakin $\eta \in [M_1,M_2]$ pätee
\[ 
E(f) = \frac{\eta}{24}\,(b-a)^3. 
\]
Koska $f''$ on jatkuva välillä $[a,b]$, niin $M_1=f''(x_1)$ ja $M_2=f''(x_2)$ joillakin
$x_1,x_2\in[a,b]$ (Lause \ref{Weierstrassin peruslause}), jolloin $\eta = f''(\xi)$ jollakin 
$\xi \in [a,b]$ (Lause \ref{ensimmäinen väliarvolause}). Keskipistesäännölle on näin johdettu
virhekaava
\[
\boxed{\quad E(f) = \frac{1}{24}\,(b-a)^3 f''(\xi), \quad \xi \in [a,b]
                                     \quad\text{(Keskipistesääntö)}. \quad}
\]

Trapetsille voidaan johtaa virhekaava vastaavaan tapaan. Tässä tapauksessa on
vertailupolynomiksi $p$ kuitenkin syytä valita Taylorin polynomin sijasta ensimmäisen asteen
interpolaatiopolypainnomi, jolle pätee $p(a)=f(a)$ ja $p(b)=f(b)$
(ks.\ Luku \ref{interpolaatiopolynomit}). Virhekaavaksi saadaan
(Harj.teht.\,\ref{H-int-9: Trapetsin virhekaava})
\[
\boxed{\quad E(f) = - \frac{1}{12}\,(b-a)^3 f''(\xi), \quad \xi \in [a,b] 
                                      \quad \text{(Trapetsi)}. \quad}
\]

Keskipistesäännön ja trapetsin virhekaavoista nähdään, että jos $f$ on ylöspäin kaareutuva
($f''(x)>0$ välillä $[a,b$]), niin keskipistesääntö antaa integraalille liian pienen ja 
Trapetsi liian suuren arvon. Em.\ virheanalyysiin liittyen ilmiön voi selittää niin, että 
keskipistesääntö integroi oikein $f$:n Taylorin polynomin $T_1(x,c)$, kun taas Trapetsi
integroi oikein $f$:n interpolaatiopolynomin, vrt. kuvio.
%\begin{figure}[H]
%\begin{center}
%\import{kuvat/}{kuvaipol-6.pstex_t}
%\end{center}
%\end{figure}
\begin{figure}[H]
\setlength{\unitlength}{1cm}
\begin{center}
\begin{picture}(10,3)(0,0)
\put(2,0){\vector(1,0){7}} \put(8.8,-0.4){$x$}
\put(2,0){\vector(0,1){3}} \put(2.2,2.8){$y$}
\dashline{0.1}(4.0,1.8)(7.0,2.3179) \dashline{0.1}(4.0,1.1306)(7.0,1.7)
\put(4,0){\line(0,1){0.1}} \put(5.5,0){\line(0,1){0.1}} \put(7,0){\line(0,1){0.1}}
\put(3.9,-0.5){$a$} \put(6.9,-0.5){$b$} \put(5.4,-0.5){$c$} 
\put(7.2,1.652){$\scriptstyle{\text{Keskipistesääntö}}$}
\put(7.2,2.316){$\scriptstyle{\text{Trapetsi}}$}
\curve(
    4.0000,    1.8000,
    4.5000,    1.5211,
    5.0000,    1.3843,
    5.5000,    1.4117,
    6.0000,    1.6000,
    6.5000,    1.9202,
    7.0000,    2.3179)
\end{picture}
\end{center}
\end{figure}
Myös Simpsonin säännölle voidaan johtaa virhekaava samantyyppisellä päättelyllä kuin edellä.
Olettaen, että $f$ on neljä kertaa jatkuvasti derivoituva välillä $[a,b]$, saadaan
virhekaavaksi (Harj.teht.\,\ref{H-int-9: Simpsonin 2})
\[
\boxed{\quad E(f) = - \frac{1}{90}\,h^5 f^{(4)}(\xi), 
              \quad h=(b-a)/2, \quad \xi \in [a,b] \quad \text{(Simpson)}. \quad}
\]

Em.\ virhekaavoista voidaan edelleen johtaa virhearvioita yhdistetyille kvadratuureille.
Tarkastellaan jälleen esimerkkinä keskipistesääntöä, rajoittuen tasaväliseen jakoon eli
olettaen väli $[a,b]$ jaetuksi osaväleihin $[x_{k-1},x_k],\ k=1\ldots n$, missä
$x_k-x_{k-1}=h=(b-a)/n\ \forall k$. Olkoon funktio $f$ kahdesti jatkuvasti
derivoituva välillä $[a,b]$, ja olkoon $M_1$ ja $M_2$ $f''$:n minimi- ja maksimiarvot välillä
$[a,b]$. Tällöin jos $E_k(f) =$ keskipistesäännön integrointivirhe osavälillä $[x_{k-1},x_k]$,
niin em.\ virhekaavan perusteella
\[ 
\frac{1}{24}\,M_1 h^3 \le E_k(f) \le \frac{1}{24}\,M_2 h^3. 
\]
Summaamalla yli $k$:n ja huomioimalla, että $\,nh=b-a\,$ saadaan kokonaisvirheelle 
$E(f) = \sum_{k=1}^n E_k(f)$ arviot
\[ 
\frac{1}{24}\,(b-a)M_1 h^2 \le E(f) \le \frac{1}{24}\,(b-a)M_2 h^2. 
\]
Tässä on $M_1=f(x_1),\ M_2=f(x_2)\,$ joillakin $x_1,x_2 \in [a,b]$, joten päädytään 
väliarvomuotoiseen virhekaavaan kuten edellä. Tasavälisen trapetsin ja Simpsonin tapauksessa
menetellään vastaavasti, jolloin tuloksena ovat seuraavat virhekaavat:
\begin{center}
\fbox{\begin{tabular}{ll}
Tasavälinen, yhdistetty kp.-sääntö:    & $E(f)=\ \frac{1}{24}(b-a)f''(\xi)\,h^2$ \ykehys\\ \\
Tasavälinen, yhdistetty trapetsi:      & $E(f)=-\frac{1}{12}(b-a)f''(\xi)\,h^2$ \\ \\
Tasavälinen, yhdistetty Simpson:       & $E(f)=-\frac{1}{180}(b-a)f^{(4)}(\xi)\,h^4 \ $ \akehys
\end{tabular}}
\end{center}
Tässä on kaikissa tapauksissa $\xi \in [a,b]$, ja $h$ on lähimpien kvadratuuripisteiden väli,
eli joko $h=$ osavälin pituus (Keskipistesääntö, Trapetsi), tai $h=$ puolet osavälin
pituudesta (Simpson).

Muun kuin tasavälisen jaon tapauksessa voidaan em.\ yhdistettyjen kvadratuurien virhe arvioida
muodossa $\abs{E(f)} \le \ldots\,$, missä oikealla puolella on $\pm f^{(k)}(\xi)$:n sijasta
$\abs{f^{(k)}(x)}$:n maksimiarvo välillä $[a,b]$ ja $h$ on peräkkäisten kvadratuuripisteiden
\pain{suurin} väli. Yleisesti jos osavälijakoon perustuvan yhdistetyn numeerisen kvadratuurin
virhe on suuruusluokkaa $\ordoO{h^r}$ (integroitavan funktion $f$ ollessa riittävän
säännöllinen), mutta ei luokkaa $\ordoO{h^{r+1}}$ (vaikka $f$ olisi kuinka säännöllinen), niin
sanotaan, että ko. menetelmän (tarkkuuden)
\index{kertaluku!b@tarkkuuden}%
\kor{kertaluku} (engl.\ order of accuracy) on $r$.
Kuten em.\ virhetarkastelusta voi päätellä, on yleisesti $r=m+1$, jos numeerinen kvadratuuri
integroi osaväleillä tarkasti polynomin astetta $m$ mutta ei polynomia astetta $m+1$.
Yhdistetty keskipistesääntö ja trapetsi ovat siis toisen kertaluvun menetelmiä, ja yhdistetyn
Simpsonin kertaluku on $r=4$. Esimerkkejä ensimmäisen kertaluvun menetelmistä ovat Riemannin
summakaavat, joissa kvadratuuripisteet (eli välipisteet $\xi_k$, vrt.\ Luku
\ref{määrätty integraali}) eivät ole osavälien keskipisteitä
(ks.\ Harj.teht\,\ref{H-int-9: Riemannin summa}). Riemannin summiin perustuvista
approksimaatioista yhdistetty keskipistesääntö on siis tarkkuutensa puolesta omassa luokassaan.

\subsection{Sovellusesimerkki: Adaptiivinen Simpson}
\index{adaptiivinen Simpson|vahv}
\index{zza@\sov!Adaptiivinen Simpson|vahv}

Laskimissa ja numeeris--symbolisissa tietokoneohjelmistoissa numeerisen integroinnin komentojen
(esim. Mathematica: \verb|NIntegrate|) takana on usein yhdistetty Simpson, mahdollisesti myös
korkeamman kertaluvun yhdistettyjä kvadratuureja. Jakoa osaväleihin ei näissä ohjelmissa
yleensä suoriteta tasavälisesti, vaan integroitavaan funktioon sopeutuen \kor{adaptiivisesti}. 
Adaptiivisen Simpsonin menetelmän ideana on selvittää 'nuuskimalla', kuinka suuri on 
integroitavan funktion neljäs derivaatta $f^{(4)}(x)$ tarkasteltavalla osavälillä 
$[x_{k-1},x_k]$. Perusajatus on yksinkertainen: Jos osaväli on lyhyt (niinkuin laskun kuluessa
ennen pitkää on), niin voidaan olettaa, että $f^{(4)}(x)$ on ko.\ välillä likimain vakio $=M_k$.
Tällöin vakio $M_k$ saadaan selville tulkitsemalla itse algoritmin antamia tuloksia 
\index{a posteriori}%
\kor{a posteriori} (eli laskemisen jälkeen) seuraavasti: Sovelletaan ensin Simpsonin sääntöä
välillä $[x_{k-1},x_k]$ -- tulos $I_1$. Jaetaan sitten väli puoliksi ja sovelletaan Simpsonin
sääntöä kummallakin osavälillä erikseen -- tulos $I_2$. Jos nyt integraalin tarkka arvo $=I$ 
välillä $[x_{k-1},x_k]$, niin Simpsonin virhekaavan mukaan
\[
\begin{cases} 
 I-I_1=-\frac{1}{90}M_kh^5, \\ 
 I-I_2=-\frac{1}{90}M_k\cdot 2\cdot\left(\frac{h}{2}\right)^5
\end{cases}
\impl\ M_k=96h^{-5}(I_1-I_2), \quad h=\tfrac{1}{2}(x_k-x_{k-1}).
\]
Todellisuudessa $f^{(4)}$ ei ole aivan vakio edes lyhyellä välillä (jos olisi, saataisiin myös
integraalin tarkka arvo $I$ selville!), mutta saatu arvio on yleensä adaptiivisiin 
tarkoituksiin riittävä: Sen avulla voidaan ohjata algoritmia tihentämään jakoa siellä, missä 
laskun antama luku $M_k$ on itseisarvoltaan suuri. Algoritmi pyrkii tarkemmin tihentämään jakoa
niin, että \pain{virhetihe}y\pain{s}, eli integroimisvirhe osavälillä jaettuna osavälin 
pituudella, on suunnilleen sama jokaisella osavälillä. (Tällainen jako on osoitettavissa 
laskutyön kannalta optimaaliseksi.) 

Adaptiivinen algoritmi toimii käytännössä hämmästyttävän hyvin myös useimmissa sellaisissa
tilanteissa, joissa funktio ei ole lainkaan niin säännöllinen kuin em.\ laskussa oletetaan
(eli neljä kertaa jatkuvasti derivoituva). Näin käy vaikkapa Esimerkin 
\ref{Trapetsi vastaan Simpson 2} integroimistehtävässä, jossa integroitava funktio ei ole
edes jatkuvasti derivoituva ($f(x) \approx 1-\sqrt{x}$ pisteen $x=0$ lähellä). Laskemalla
lukuja $M_k$ ym.\ tavalla algoritmi päätyy tihentämään jakoa voimakkaasti origon lähellä
ja pystyy tämän 'koneälyn' ansiosta laskemaan integraalin arvon vaaditulla tarkkuudella lähes
yhtä nopeasti kuin Esimerkin \ref{Trapetsi vastaan Simpson 1} tilanteessa 
(ks.\ Harj.teht.\,\ref{H-int-9: sqrt-integraali}c).

\Harj
\begin{enumerate}

\item
Integraali $\int_0^1 f(x)\,dx$ lasketaan tasavälisellä, yhdistetyllä Trapetsilla jakamalla
integroimisväli $n$ osaväliin. Laske näin saadun likiarvon virhe tarkasti, kun \
a) $f(x)=x(1-x)$, \ b) $f(x)=e^x$.

\item
Laske seuraaville integraalille likiarvot käyttämällä $11$ pisteen tasavälistä, yhdistettyä
Trapetsia ($10$ osaväliä) ja Simpsonin sääntöä ($5$ osaväliä) ja vertaa tarkkaan arvoon.
\[
\text{a)}\,\ \int_0^1 \frac{1}{1+x^2}\,dx \qquad
\text{b)}\,\ \int_0^1 \frac{4x^3}{1+x}\,dx \qquad
\text{c)}\,\ \int_0^1 \frac{1}{1+\sqrt[4]{x}}\,dx
\]

\item \label{H-int-9: korjattu trapetsi} \index{korjattu trapetsikaava}
Numeerisen integroinnin nk.\ \kor{korjattu trapetsikaava} on
\[
\int_a^b f(x)\,dx \approx \frac{1}{2}(b-a)[f(a)+f(b)]-\frac{1}{12}(b-a)^2[f'(b)-f'(a)].
\]
Näytä, että kaava on tarkka polynomeille astetta $m=3$. Millainen on vastaava yhdistetty
kvadratuuri, jos jako osaväleihin on tasavälinen?

\item \label{H-int-9: Trapetsin virhekaava}
Olkoon $f$ kahdesti jatkuvasti derivoituva välillä $[a,b]$ ja $g=f-p$, missä $p$ on ensimmäisen
asteen (Lagrangen) interpolaatiopolynomi, jolle pätee $p(a)=f(a)$ ja $p(b)=f(b)$. Näytä
osittain integroimalla, että pätee
\[
-\frac{1}{2}\int_a^b (x-a)(b-x)g''(x)\,dx = \int_a^b g(x)\,dx.
\]
Johda Trapetsin virhekaava tästä tuloksesta.

\item \label{H-int-9: Simpson 1}
a) Olkoon $p$ funktion $f$ toisen asteen interpolaatiopolynomi, joka määritellään ehdoilla
$p(a-h)=f(a-h)$, $p(a)=f(a)$ ja $p(a+h)=f(a+h)$ välillä $[a-h,a+h]$. Näytä, että Simpsonin
sääntö ko.\ välillä on sama kuin approksimaatio
\[
\int_{a-h}^{a+h} f(x)\,dx \approx \int_{a-h}^{a+h} p(x)\,dx.
\]
b) Todista Simpsonin säännön virhekaava siinä tapauksessa, että $f^{(4)}$ on vakio 
välillä $[a-h,a+h]$ (eli $f$ on polynomi astetta $4$).

\item \label{H-int-9: Riemannin summa}
Integraalin approksimoiminen Riemannin summalla, jossa välipisteiksi osaväleillä $[x_{k-1},x]$
valitaan $\xi_k=x_{k-1}$, vastaa osaväleillä tehtyä approksimaatiota
$\int_a^b f(x)\,dx \approx (b-a)f(a)$. Johda tälle virhekaava
\[
E(f)=\frac{1}{2}(b-a)^2 f'(\xi), \quad \xi\in[a,b].
\]
Mikä on virhekaava koko välillä, jos jako osaväleihin on tasavälinen?

\item(*) \label{H-int-9: Arkhimedes} \index{zzb@\nim!Arkhimedeen algoritmi}
(Arkhimedeen algoritmi) Suora leikkaa paraabelin $K:\ y=x^2$ pisteissä $(a,a^2)$ ja 
$(b,b^2)$, jolloin suora erottaa paraabelista segmentin $A$ välillä $[a,b]$ ($a<b$). Olkoon 
$A_n$ segmentin pinta-alan $\mu(A)$ likiarvo, joka saadaan jakamalla väli $[a,b]$ tasavälisesti
$2^n$ osaväliin ($n\in\N$) ja käyttämällä yhdistettyä trapetsia. Näytä pelkin algebran keinoin,
että jono $\seq{A_n}$ on geometrinen sarja. Laske $\mu(A)$ ($=\lim_n A_n$) tällä perusteella.

\item (*) \label{H-int-9: Simpsonin 2}
a) (Vrt.\ Tehtävä \ref{H-int-9: Simpson 1}a.) Olkoon $p$ funktion $f$ yleistetty kolmannen
asteen interpolaatiopolynomi, joka määritellään ehdoilla $\,p(a-h)=f(a-h)$, $\,p(a+h)=f(a+h)$,
$\,p(a)=f(a)\,$ ja $\,p'(a)=f'(a)$ välillä $[a-h,a+h]$. Näytä, että Simpsonin sääntö ko.\
välillä on sama kuin approksimaatio
\[
\int_{a-h}^{a+h} f(x)\,dx \approx \int_{a-h}^{a+h} p(x)\,dx.
\]
b) Johda Simpsonin virhekaava. \kor{Vihje}: Lause \ref{usean pisteen Taylor}.

\item (*) \label{H-int-9: korjatun trapetsin virhe}
Olkoon $f$ neljä kertaa jatkuvasti derivoituva välillä $[a,b]$. Johda korjatulle
trapetsikaavalle (Harj.teht.\,\ref{H-int-9: korjattu trapetsi}) virhekaava
\[
E(f) = \frac{1}{720}\,(b-a)^5 f^{(4)}(\xi), \quad \xi\in[a,b].
\]
\kor{Vihje}: Hermiten interpolointi ja Lause \ref{usean pisteen Taylor}!

\item (*) \label{H-int-9: sqrt-integraali}
Tarkastellaan numeerisen integroinnin virhettä laskettaessa integraali $\int_0^1\sqrt{x}\,dx\,$
yhdistetyillä kvadratuureilla, joissa väli $[0,1]$ jaetaan eri tavoilla $n$ 
osaväliin. \vspace{1mm}\newline
\ a) Näytä, että jos jako on tasavälinen, niin sekä yhdistetyn trapetsin että yhdistetyn
Simpsonin virhe on \ $\sim n^{-3/2}$ suurilla $n$:n arvoilla. \newline 
b) Valitaan välin $[0,1]$ jakopisteiksi $x_k=(k/n)^{4/3},\ k=0 \ldots n$ ja käytetään
yhdistettyä trapetsia. Näytä, että virhe $\,=\mathcal{O}(n^{-2}\ln n)$. \newline
c) Valitaan välin $[0,1]$ jakopisteiksi $x_k=(k/n)^{8/3},\ k=0 \ldots n$ ja \mbox{käytetään}
yhdistettyä Simpsonia. Näytä, että virhe $\,=\mathcal{O}(n^{-4}\ln n)$. (Adaptiivinen Simpsonin
algoritmi päätyy likimain tähän jakoon suurilla $n$:n arvoilla.)
 
\item (*) \label{H-int-9: hidas sarja}
(Hitaan sarjan kiihdytys) Suppeneva integraali $\int_n^\infty f(x)\,dx$, missä $n\in\N$ on suuri
luku, voidaan laskea likimäärin käyttämällä joko a) yhdistettyä trapetsia tai b) yhdistettyä,
korjattua trapetsia (Harj.teht.\,\ref{H-int-9: korjattu trapetsi}) välin $[n,\infty)$ jaossa
osaväleihin $\,[k,k+1]$, $k\in\N$, $k \ge n$. Lähtien näistä ajatuksista ja mainittujen
kvadratuurien virhearvioista 
(ks.\ Harj.teht.\,\ref{H-int-9: korjatun trapetsin virhe}) johda seuraavat tulokset, kun
$f(x)=1/x^\alpha,\ \alpha>1$\,:
\begin{align*}
&\text{a)}\,\ \sum_{k=1}^\infty \frac{1}{k^\alpha} \,=\,
 \sum_{k=1}^n \frac{1}{k^\alpha}+\frac{1}{\alpha-1}\,n^{1-\alpha}-\frac{1}{2}\,n^{-\alpha}
 +\Ord{n^{-\alpha-1}}\,. \\
&\text{b)}\,\ \sum_{k=1}^\infty \frac{1}{k^\alpha} \,=\,
 \sum_{k=1}^n \frac{1}{k^\alpha}+\frac{1}{\alpha-1}\,n^{1-\alpha}-\frac{1}{2}\,n^{-\alpha}
 +\frac{\alpha}{12}\,n^{-\alpha-1}+\Ord{n^{-\alpha-3}}\,.
\end{align*}
\kor{Vihje}: Huomioi, että
\begin{align*}
\text{a)}\,\ \sum_{k=n+1}^\infty f(k)
&\,=\, \sum_{k=n}^\infty \frac{1}{2}\,\bigl[\,f(k)+f(k+1)\,\bigr] - \frac{1}{2}\,f(n), \\
\text{b)}\,\ \sum_{k=n+1}^\infty f(k)
&\,=\, \sum_{k=n}^\infty \left(\frac{1}{2}\,\bigl[\,f(k)+f(k+1)\,\bigr]
                             -\frac{1}{12}\,\bigl[\,f'(k+1)-f'(k)\,\bigr]\right) \\
&\quad -\frac{1}{2}\,f(n) - \frac{1}{12}\,f'(n).
\end{align*}

\end{enumerate}
 
 %Numeerinen integrointi

\chapter{Differentiaaliyhtälöt}
\index{differentiaaliyhtälö|vahv}

''Luonnonlait on kirjoitettu matematiikan kielellä'', lausui fyysikko ja tähtitieteilijä
\index{Galilei, G.}%
\hist{Galileo Galilei} (1564-1642). Galilein yhä ajankohtainen lausuma tuo erityisesti mieleen
\kor{differentiaaliyhtälöt}, sillä melkeinpä kaikki luonnonlait ovat sellaisia. Paitsi
luonnonlaeissa, differentiaaliyhtälöitä tavataan nykyisin mitä erilaisimmissa (ihmisen
luomissa) fysiikan, biologian, talousieteen ym.\ matemaattisissa malleissa. --- Tällaisten
'kantaäitinä' voi pitää aiemmin Luvuissa \ref{exp(x) ja ln(x)} ja 
\ref{eksponenttifunktio fysiikassa} tarkasteltua eksponentiaalisen kasvun tai vaimenemisen
mallia.

Tässä luvussa käydään ensin läpi differentiaaliyhtälöihin liittyvät peruskäsitteet
(Luku \ref{DY-käsitteet}), minkä jälkeen tarkastellaan differentiaaliyhtälöiden klassisia
ratkaisumenetelmiä ja näihin liittyviä sovellusesimerkkejä
(Luvut \ref{separoituva DY}--\ref{2. kertaluvun lineaarinen DY}). Differentiaaliyhtälöiden
perinteinen ratkaisemistekniikka on Luvuissa \ref{integraalifunktio}--\ref{osamurtokehitelmät}
läpikäytyyn integroimistekniikkaan pitkälti perustuva (ja tähän tekniikkaan verrattavissa oleva)
matematiikan taitolaji. Joillekin differentiaaliyhtälöiden erikoistyypeille ratkaisut ovat 
löydettävissä pelkällä 'sivistyneellä arvauksella', muissa tapauksissa ratkaiseminen pyritään
palauttamaan 'integroimisiin' eli \kor{kvadratuureihin}.

Differentiaaliyhtälöiden matemaattisessa paljoudessa perinteisin menetelmin ratkeavia voi pitää
harvinaisuuksina, mutta sovelluksissa tällaiset erikoistapaukset ovat kuitenkin melko yleisiä.
Luvuissa \ref{separoituva DY}--\ref{2. kertaluvun lineaarinen DY} käydään läpi näistä 
erikoistapauksista tavallisimmat. Luvussa \ref{DYn numeeriset menetelmät} tarkastellaan
perinteisiä ratkaisumenetelmiä yleispätevämpiä \kor{numeerisia} menetelmiä, joilla
differentiaaliyhtälöiden ja myös useamman yhtälön muodostamien 
\kor{differentiaaliyhtälösysteemien} ratkaisuja voidaan laskea likimäärin. Likimääräisen
ratkaisemisen ideoihin perustuu myös differentiaaliyhtälöiden teorian päälause,
\kor{Picardin--Lindelöfin lause}, joka esitellään Luvussa \ref{Picard-Lindelöfin lause}.
 %Differentiaaliyhtälöt
\section{Differentiaaliyhtälöiden peruskäsitteet} \label{DY-käsitteet}
\alku

\kor{Differentiaaliyhtälön} (jatkossa lyhennys DY), tai tarkemmin 
\kor{tavallisen differentiaaliyhtälön} (engl.\ ordinary differential equation eli ODE) yleinen
muoto on \index{differentiaaliyhtälö!d@yleinen}
\[
F(x,y,y',\ldots,y^{(n)})=0,
\]
missä $x$ on reaalimuuttuja, $y=y(x)$ on tuntematon reaaliarvoinen funktio, ja $F$ jokin
(tunnettu) $n+2$ reaalimuuttujan lauseke (kyseessä on funktio tyyppiä $F:\R^{n+2}\kohti\R$, 
vrt.\ Luku \ref{kahden ja kolmen muuttujan funktiot}). Kun yhtälö kirjoitetaan tarkemmin
muodossa
\[
F(x,y(x),y'(x),\ldots,y^{(n)}(x))=0,
\]
niin nähdään selvemmin, että yhtälössä on vain yksi vapaa muuttuja ($x$). Nimitys 'tavallinen'
DY tulee juuri tästä ominaisuudesta. ('Epätavallisista' differentiaaliyhtälöistä ei
toistaiseksi puhuta.)

Ym.\ differentiaaliyhtälössä indeksi $n\in\N$ on yhtälön
\index{kertaluku!c@differentiaaliyhtälön}%
\kor{kertaluku} (engl.\ order). Jos 
yhtälö kirjoitetaan muotoon
\[
y^{(n)}=f(x,y,\ldots,y^{(n-1)}),
\]
niin tällaista muotoa sanotaan differentiaaliyhtälön
\index{differentiaaliyhtälön!a@normaalimuoto} \index{normaalimuoto!a@DY:n, DY-systeemin}%
\kor{normaalimuodoksi}. Ensimmäisen
kertaluvun differentiaaliyhtälön normaalimuoto on siis
\[
y'=f(x,y).
\]
Normaalimuoto on matemaattisen teorian kannalta sikäli edullinen, että siihen perustuen voidaan
differentiaaliyhtälön ratkeavuusehdot asettaa paljon helpommin kuin yleisemmässä muodossa.

\subsection{Yksittäisratkaisu.  Yleinen ratkaisu}
\index{differentiaaliyhtälön!b@yksittäisratkaisu, yleinen ratk.|vahv}
\index{yksittäisratkaisu (DY:n)|vahv}
\index{yleinen ratkaisu (DY:n)|vahv}

Jos differentiaaliyhtälön kertaluku on $n$, niin sen ratkaisu, tarkemmin 
\kor{yksittäisratkaisu}, on jokainen funktio $y(x)$, joka on jollakin \pain{avoimella}
\pain{välillä} $(a,b)$ (voi olla $\,a=-\infty\,$ ja/tai $\,b=\infty\,$) \pain{$n$} \pain{kertaa} \pain{derivoituva} ja 
toteuttaa yhtälön ko.\ välillä. Differentiaaliyhtälön \kor{yleisellä ratkaisulla} tarkoitetaan
sellaista \pain{funktio}j\pain{oukkoa}, joka sisältää kaikki (tai 'melkein kaikki', ks.\
huomatukset jäljempänä) ratkaisut. Jos yhtälön kertaluku on $n$, niin yleinen ratkaisu on
pääsääntöisesti muotoa
\[
y=Y(x,C_1,\ldots,C_n),
\]
missä $C_1,\ldots,C_n$ ovat vapaasti (tai esim.\ joiltakin väleiltä vapaasti) valittavia 
\kor{vakioita}, ja $Y$ on jokin $n+1$:n muuttujan funktio. Sikäli kuin vakiot ovat täysin
vapaasti valittavissa, yleinen ratkaisu on siis funktiojoukko
\[
\mathcal{Y}=\{\,y(x)=Y(x,C_1,\ldots,C_n) \ | \ C_1,\ldots,C_n\in\R\,\}.
\]
Differentiaaliyhtälön 'ratkaisemisella' tarkoitetaan yleensä yleisen ratkaisun määrittämistä. 
\begin{Exa}
Differentiaaliyhtälöiden
\begin{itemize}
\item[a)] $y'=e^{2x}$, $\quad\ \text{b)}\,\ y'=2y$
\end{itemize}
yleiset ratkaisut $\R$:ssä (välillä $(-\infty,\infty)$) ovat
\begin{itemize}
\item[a)] $y(x)=\frac{1}{2}\,e^{2x}+C\,\ (C\in\R)$,
          $\quad\ \text{b)}\,\ y(x)=Ce^{2x}\,\ (C\in\R)$. \loppu
\end{itemize}
%Tässä $C\in\R$ on vapaasti valittavissa ja ratkaisut ovat voimassa koko $\R$:ssä.
\end{Exa}

\begin{Exa} Ratkaise: \ a) \ $y'''=0$, \ \ b) \ $y'''+2y''=0$.
\end{Exa}
\ratk a) Integroimalla saadaan
\begin{align*} y'''(x) = 0 &\qimpl y''(x) = C_1 \\[2mm]
                           &\qimpl y' = C_1 x + C_2 \\
                           &\qimpl y(x) = \frac{1}{2}C_1 x^2 + C_2 x + C_3.
\end{align*}
Tässä voidaan $C_1$:n tilalle yhtä hyvin kirjoittaa $2C_1$, jolloin yleiselle ratkaisulle 
saadaan luontevampi muoto
\[ 
y(x) = C_1 x^2 + C_2 x + C_3 \quad (C_1,C_2,C_3 \in \R). 
\]

b) Ratkaisu voidaan tässä keksiä kirjoittamalla ensin $y''(x)=u(x)$, jolloin yhtälö 
yksinkertaistuu muotoon $\,u'+2u=0$. Tämän yleinen ratkaisu on $\,u(x)=C_1e^{-2x}$. Koska 
$u=y''$, niin integroimalla seuraa
\begin{align*}
y''(x) = C_1 e^{-2x} &\qimpl y'(x)=\int C_1 e^{-2x}\,dx=-\frac{1}{2}C_1e^{-2x}+C_2 \\
                     &\qimpl y(x)=\int y'(x)\,dx=\frac{1}{4}C_1e^{-2x}+C_2x+C_3.
\end{align*}
Kirjoittamalla $C_1$:n tilalle $4C_1$ saadaan yleiselle ratkaisulle muoto
\[
y(x)=C_1e^{-2x}+C_2x+C_3 \quad (C_1,C_2,C_3 \in \R). \loppu
\]

\subsection{Erikoisratkaisut}
\index{differentiaaliyhtälön!c@erikoisratkaisu|vahv}
\index{erikoisratkaisu (DY:n)|vahv}

Jos differentiaaliyhtälöllä on muitakin ratkaisuja kuin yleiseen ratkaisulausekkeeseen 
$y=Y(x,C_1,\ldots,C_n)$ sisältyvät, niin tällaisia 'yllätysratkaisuja' sanotaan 
\kor{erikoisratkaisuiksi}. Milloin ratkaisu on 'erikoinen' ja milloin ei, voi riippua
yleisen ratkaisun esitysmuodosta.
\begin{Exa} \label{erikoinen dy}
Differentiaaliyhtälöiden
\begin{itemize}
\item[a)] $y'=y^2,\quad \text{b)}\,\ (y')^2=y$
\end{itemize}
yleiset ratkaisut ovat
\begin{itemize}
\item[a)] $y(x)=\dfrac{1}{C-x}\,,\quad \text{b)}\,\ y(x)=\dfrac{1}{4}(x-C)^2$,
\end{itemize}
kuten saatetaan helposti tarkistaa ($C\in\R$). Yhtälöillä on myös ilmeinen ratkaisu,
jota ei saada yleisen ratkaisun lausekkeesta, nimittäin
\[
y(x)=0.
\]
Tätä on siis pidettävä erikoisratkaisuna. Tapauksessa a) voi kuitenkin yleisen ratkaisun
esittää myös muodossa
\begin{itemize}
\item[a)] $y(x)=\dfrac{C}{1-Cx}$
\end{itemize}
(aiemmassa ratkaisussa kirjoitettu $C$:n tilalle $1/C$), jolloin $y(x)=0$ sisältyy tähän
($C=0$). Tapauksessa b) vieläkin 'erikoisempi' ratkaisu on
\[
y(x)=\begin{cases}
\,0,                  &\text{ kun } x\leq C, \\
\,\frac{1}{4}(x-C)^2, &\text{ kun } x>C.
\end{cases} \quad\loppu
\]
\end{Exa}
%\begin{Exa}
%Differentiaaliyhtälön
%\[
%y'=\sqrt[3]{y^3+1}
%\]
%yleinen ratkaisu koostuu funktioista, jotka eivät ole alkeisfunktiota. Yksi 
%alkeisfunktioratkaisu on kuitenkin keksittävissä: $y(x)=-1$. Tätä on syytä epäillä
%erikoisratkaisuksi. \loppu
%\end{Exa}

\subsection{Käyräparven differentiaaliyhtälö}
\index{differentiaaliyhtälö!e@käyräparven|vahv}
\index{kzyyrzy@käyräparvi|vahv}

Differentiaaliyhtälön yleinen ratkaisu voidaan tulkita geometrisesti \kor{käyräparveksi} 
(eli käyrien joukoksi). Jos tunnetaan käyräparvi, niin sen differentiaaliyhtälö on 
johdettavissa derivoimalla. Nimittäin jos käyräparven funktiot ovat muotoa 
$y=Y(x,C_1,\ldots,C_n)$, niin derivoimalla $n$ kertaa saadaan $n+1$:n yhtälön ryhmä, josta 
vakiot $C_1,\ldots,C_n$ ovat (ainakin periaatteessa) eliminoitavissa. Tällöin saadaan ko.\ 
käyräparvelle differentiaaliyhtälö kertalukua $n$.
\begin{Exa}
Minkä differentiaaliyhtälön yleinen ratkaisu on 
\[ 
\text{a)}\ \ y=(C+x)e^x, \quad\ \text{b)}\ \ y=\frac{C_1}{x+C_2}\ ? 
\]
\end{Exa}
\ratk a) \ Derivoimalla kerran saadaan yhtälöryhmä
\[
\left\{
\begin{aligned}
y &=Ce^x+xe^x \\
y'&=Ce^x+xe^x+e^x
\end{aligned}
\right.
\]
Vähennyslaskulla saadaan differentiaaliyhtälöksi
\[
y'-y=e^x.
\]
b) \ Derivoidaan kahdesti:
\[
\begin{cases}
\,y\  =C_1(x+C_2)^{-1} \\
\,y'\,=-C_1(x+C_2)^{-2} \\
\,y'' =2C_1(x+C_2)^{-3}
\end{cases}
\]
Eliminoimalla $C_1$ ja $C_2$ saadaan differentiaaliyhtälöksi
\[
yy''=2(y')^2. \loppu
\]

\subsection{Kohtisuorat leikkaajat}
\index{differentiaaliyhtälö!f@kohtisuorien leikkaajien|vahv}
\index{kzyyrzy@käyräparvi|vahv}
\index{kohtisuora leikkaus!b@käyräparvien|vahv}

Jos yksiparametrisen käyräparven $y=Y(x,C)$ differentiaaliyhtälö on 
\[ 
y'=f(x,y), 
\] 
niin ratkaisemalla differentiaaliyhtälö
\[ 
y'=-\frac{1}{f(x,y)} 
\]
löydetään käyräparven \kor{kohtisuorat leikkaajat}
(vrt.\ Esimerkki \ref{derivaatta geometriassa}:\ref{kohtisuora leikkaus}).
\begin{Exa}
Ympyräparven $\,x^2+y^2=C^2\,$ differentiaaliyhtälöksi saadaan implisiittisesti derivoimalla
\[
x+yy'=0.
\]
Kohtisuorien leikkaajien differentiaaliyhtälön
\[
x-y/y'=0
\]
yleinen ratkaisu on $y=Cx$, kuten saattoi (geometrisesti) arvata. \loppu
\end{Exa}

\subsection{Alku- ja reunaehdot}
\index{alkuehto (DY:n)|vahv}
\index{reunaehto|vahv}

Koska differentiaaliyhtälön ratkaisu sisältää määräämättömiä vakioita, tarvitaan 
sovellustilanteissa lisäehtoja, jotta ratkaisu olisi yksikäsitteinen. Lisäehdot on pääteltävä 
sovellustilanteesta, eli ne kuuluvat matemaattiseen malliin samoin kuin itse
differentiaaliyhtälökin.

Jos differentiaaliyhtälön kertaluku on $n$, niin ratkaisussa on yleensä $n$ määrämätöntä 
vakiota, jolloin tarvitaan $n$ lisäehtoa. Yksinkertaisin tapa asettaa
lisäehdot on kiinnittää jossakin pisteessä $x_0$ derivaattojen $y^{(k)}(x_0)$ arvot, kun
$k=0\ldots n-1$. Näin saadaan
\index{differentiaaliyhtälön!d@alkuarvotehtävä} \index{alkuarvotehtävä}%
\kor{alkuarvotehtävä} (engl.\ initial value problem)
\[
\left\{
\begin{aligned}
&F(x,y',\ldots,y^{(n)}) = 0,\quad x\in (a,b), \\
&y(x_0)\qquad = A_0, \\
&\quad\vdots \qquad\quad\,\ \vdots \; \; \; \vdots \\
&y^{(n-1)}(x_0) = A_{n-1}.
\end{aligned}
\right.
\]
Tässä voi olla $x_0\in(a,b)$ tai myös $x_0=a$ tai $x_0=b$. Jos $x_0$ on välin päätepiste, niin 
alkuehdot on tulkittava derivaattojen $y^{(k)}(x)$ \pain{tois}p\pain{uolisina} 
\pain{ra}j\pain{a-arvoina} (tai toispuolisina derivaattoina) kun $x\kohti a^+$ ($x_0=a$) tai 
$x\kohti b^-$ ($x_0=b$). Edellytys on tällöin, että differentiaaliyhtälön ratkaisuille nämä ovat
olemassa.
\begin{Exa}
Jos kappale (massa $=m$) on hetkellä $t=0$ levossa pisteessä $x_0$, ja kappaleeseen vaikuttaa 
voima $f(t)$ kun $t>0$, niin kappaleen sijainti $x(t)$ hetkellä $t$ saadaan selville 
ratkaisemalla alkuarvoprobleema
\[
\left\{ \begin{aligned}
&mx''(t) = f(t),\quad t>0, \\
&x(0)\, = x_0, \\
&x'(0)  = 0.
\end{aligned} \right.
\]
Alkuehdot voi asettaa täsmällisemmin muodossa $x(0^+)=x_0,\ D_+ x(0)=0$. \loppu
\end{Exa}
Lisäehtoja voidaan myös asettaa useammassa pisteessä. Alkuarvotehtävän ohella tyypillisin on
\index{differentiaaliyhtälön!e@reuna-arvotehtävä} \index{reuna-arvotehtävä}%
\kor{kahden pisteen reuna-arvotehtävä} (engl. two-point boundary value problem), jossa ehdot
asetetaan tarkasteltavan välin päätepisteissä. Jos $n=2$, niin kahden pisteen reuna-arvotehtävän
normaalimuoto on
\[ \left\{ \begin{aligned}
&y'' =f(x,y,y'),\quad x\in(a,b), \\
&y(a)=A, \ y(b)=B.
\end{aligned} \right. \]
Reunaehtojen asettelussa on tässä oletettava, että ratkaisu on oikealta jatkuva $a$:ssa ja
vasemmalta jatkuva $b$:ssä. Koska ratkaisu on joka tapauksesa (kahdestikin) derivoituvana
jatkuva välillä $(a,b)$, niin lisäoletukset tarkoittavat samaa kuin jatkuvuus välillä $[a,b]$.
\begin{Exa}
Ratkaise kahden pisteen reuna-arvotehtävä
\[
\left\{ \begin{aligned}
&y'''+2y''=0,\quad x\in (0,1), \\
&y(0)=1, \ y'(0)=-1, \ y(1)=0.
\end{aligned} \right.
\]
\end{Exa}
\ratk Yleinen ratkaisu on (ks.\ Esimerkki 2)
\[
y(x)=C_1e^{-2x}+C_2x+C_3,
\]
joten saadaan yhtälöryhmä
\[ \left\{ \begin{array}{rrrrrrrrr}  y(0)&=&        C_1& &   &+&C_3&=& 1 \\
                                    y'(0)&=&    -2\,C_1&+&C_2& &   &=&-1 \\
                                     y(1)&=&e^{-2}\,C_1&+&C_2&+&C_3&=& 0
\end{array} \right. 
   \qimpl \left\{ \begin{aligned} C_1&=\ 0 \\ C_2&=-1 \\ C_3&=\ 1 \end{aligned} \right. \] 
Ratkaisu on siis $\,y(x)=-x+1$. \loppu 

\subsection{Differentiaaliyhtälösysteemit}
\index{differentiaaliyhtälö!g@--systeemi|vahv}

\kor{Differentiaaliyhtälösysteemillä} tarkoitetaan useamman differentiaaliyhtälön muodostamaa
yhtälöryhmää. 
\index{differentiaaliyhtälön!a@normaalimuoto} \index{normaalimuoto!a@DY:n, DY-systeemin}%
\kor{Normaalimuotoinen} tavallinen differentiaaliyhtälösysteemi on jollakin
$n\in\N,\ n \ge 2$ muotoa
\[
 \left\{ \begin{aligned} 
         y'_1 &= f_1(x,y_1, \ldots, y_n), \\
         y'_2 &= f_2(x,y_1, \ldots, y_n), \\
              &\vdots \\
         y'_n &= f_n(x,y_1, \ldots, y_n).
         \end{aligned} \right.
\]
Tässä $x$ on riippumaton muuttuja ja funktiot $y_i(x),\ i=1 \ldots n,$ ovat tuntemattomia.
Ratkaisu (yksittäinen tai yleinen) on ko.\ systeemin jollakin avoimella välillä toteuttavien
funktioiden $y_i$ muodostama (järjestetty) joukko
\[
\my(x)=(y_1(x), \ldots y_n(x)).
\]
Tämä on itse asiassa funktioiden muodostama \pain{vektori} eli vektoriarvoinen funktio.
Tapauksissa $n=2,3$ ratkaisu on haluttaessa tulkittavissa tason tai avaruuden parametriseksi 
käyräksi (parametrina tässä $x$, vrt.\ Luku \ref{parametriset käyrät}).
\index{alkuarvotehtävä}%
\kor{Alkuarvotehtävässä} vaaditaan, että ratkaisu toteuttaa differentiaaliyhtälöiden lisäksi
$n$ lisäehtoa muotoa
\[ 
y_i(x_0) = A_i, \quad i = 1 \ldots n.
\]

Normaalimuotoinen korkeamman kertaluvun differentiaaliyhtälö
\[
y^{(n)}=f(x,y',\ldots,y^{(n-1)})
\]
voidaan aina kirjoittaa normaalimuotoiseksi differentiaaliyhtälösysteemiksi. Nimittäin kun 
kirjoitetaan
\[
y_1=y,\ y_2=y',\ \ldots,\ y_n=y^{(n-1)},
\]
niin nämä yhtälöt yhdessä differentiaaliyhtälön kanssa muodostavat systeemin
\[
 \left\{ \begin{aligned} 
         y'_1 \quad  &= y_2, \\
                     &\vdots \\
         y'_{n-1}\,  &= y_n, \\
         y'_n \quad  &= f(x,y_1, \ldots y_n).
         \end{aligned} \right.
\]
Tämä on em.\ normaalimuotoa.
\begin{Exa} Differentiaaliyhtälön $\,y'''=x^2y^3+x(y')^2+y''\,$ systeemimuoto on
\[
 \left\{ \begin{aligned} 
         y'_1 &= y_2, \\
         y'_2 &= y_3, \\
         y'_3 &= x^2y_1^3+xy_2^2+y_3.
         \end{aligned} \right. \loppu
\]
\end{Exa}

Korkeamman kertaluvun differentiaaliyhtälön kirjoittaminen systeemimuotoon auttaa sekä 
teoreettisissa tarkasteluissa että numeerisissa ratkaisumenetelmissä (ks.\ Luvut 
\ref{vakikertoimiset ja Eulerin DYt}, \ref{DYn numeeriset menetelmät},
\ref{Picard-Lindelöfin lause}). Joskus systeemimuoto on edullinen myös klassisissa
ratkaisumenetelmissä (ks.\ Luku \ref{toisen kertaluvun dy}).

\subsection{Ratkaisujen säännöllisyys}
\index{differentiaaliyhtälön!f@ratkaisun säännöllisyys|vahv}

Alkuarvotehtävää voi pitää implisiittisenä funktion $y(x)$ määritelmänä, jolloin implisiittisen
derivoinnin avulla on mahdollista laskea ratkaisufunktion korkeampia derivaattoja 
alkuarvopisteessä $x_0$. Tällä tavoin voidaan usein myös selvittää, kuinka säännöllinen ratkaisu
on sellaisella välillä, jolla se on ($n$ kertaa derivoituvana) olemassa.
\begin{Exa} \label{Airyn DY} Olkoon $a>0$ ja tarkastellaan alkuarvotehtävää
\[
\begin{cases} \,y'=x+y^2, \quad x\in(-a,a), \\ \,y(0)=0. \end{cases}
\]

Jos oletetaan tehtävä ratkeavaksi (kysymystä tarkastellaan myöhemmin Luvussa 
\ref{Picard-Lindelöfin lause}), niin differentiaaliyhtälöstä voi päätellä, että ratkaisu on 
ko.\ välillä mielivaltaisen monta kertaa derivoituva (sileä). Nimittäin ensinnäkin, koska
ratkaisu $y(x)$ on välillä $(-a,a)$ derivoituva (perusoletus), niin differentiaaliyhtälön
mukaan se on myös kahdesti derivoituva:
\[ 
y''(x) = \frac{d}{dx}[x+(y(x))^2] 
       = 1 + 2y(x)y'(x) = 1+2y(x)[x+(y(x))^2], \quad x \in (-a,a). 
\]
Tämän mukaan $y''$ on edelleen derivoituva välillä $(-a,a)$, eli $y$ on kolmesti derivoituva,
jne. Päätellään, että $y(x)$ on mielivaltaisen monta kertaa derivoituva välillä $(-a,a)$.
Ratkaisufunktion derivaatat alkuarvopisteessä $x=0$ ovat myös suoraan laskettavissa, sillä
koska $y(0)=0$ (alkuehto), niin differentiaaliyhtälön mukaan on oltava $y'(0)=0$, jolloin em.\
lauseke $y''(x)$:lle antaa $y''(0)=1$. Jatkamalla implisiittistä derivointia nähdään, että 
ratkaisufunktion \pain{kaikki} derivaatat pisteessä $x=0$ määräytyvät yksikäsitteisesti (!).
\loppu
\end{Exa}

\subsection{Ratkaiseminen kvadratuureilla}
\index{differentiaaliyhtälön!g@ratkaiseminen kvadratuureilla|vahv}

Jatkossa tarkastellaan lähinnä sellaisia diferentiaaliyhtälöiden erkoistapauksia, joille on
mahdollista laskea 'tarkka' ratkaisu palauttamalla tehtävä tunnettujen funktioiden
integraalifunktioiden etsimiseksi. Sanotaan tällöin, että differentiaaliyhtälö ratkeaa 
\index{kvadratuuri}%
\kor{kvadratuureilla} eli 'integroimisilla'. (Kvadratuuri tarkoittaa sananmukaisesti
'neliöimistä', vrt.\ alaviite Luvussa \ref{numeerinen integrointi}.) Ratkeaminen kvadratuureilla
\pain{ei} edellytä, että integraalifunktiot ovat alkeisfunktioita.
\begin{Exa}
Alkuarvotehtävä
\[
\left\{ \begin{aligned}
&\,y''' = \sin x/x=1-\frac{x^2}{6}+\frac{x^4}{120}-\ldots,\quad x>0, \\
&\,y(0) =1,\ y'(0)=y''(0)=0
\end{aligned} \right.
\]
ratkeaa kolmella peräkkeisellä kvadratuurilla:
\begin{align*}
y''(x) &= y''(x)-y''(0) =\int_0^x y'''(t)dt =\int_0^x (\sin t / t)dt \\
       &= x-\frac{x^3}{18}+\frac{x^5}{600}-\ldots \\[1mm]
y'(x)  &= y'(x)-y'(0)=\int_0^x y''(t)dt \\
       &= \frac{x^2}{2}-\frac{x^4}{72}+\frac{x^6}{3600}-\ldots
\end{align*}
\begin{align*}
y(x)   &= y(0)+[y(x)-y(0)] = y(0) + \int_0^x y'(t)dt \\
       &= 1+\frac{x^3}{6}-\frac{x^5}{360}+\frac{x^7}{25200}-\ldots \quad\loppu
\end{align*}
\end{Exa}
\jatko\jatko \begin{Exa} (jatko) Esimerkin differentiaaliyhtälö $\,y'=x+y^2\,$ ei ratkea
kvadratuureilla.
\index{Riccatin differentiaaliyhtälö} \index{differentiaaliyhtälö!q@Riccatin}%
(Yhtälö on \kor{Riccatin} tyyppiä,
ks.\ Harj.teht.\,\ref{lineaarinen 1. kertaluvun DY}:\ref{H-dy-4: Riccatin DY}.) \loppu
\end{Exa}
Viimeisessä esimerkissä ei tarkoiteta, ettei ratkaisuja ole, vaan ainoastaan, että
ratkaiseminen ei palaudu 'integroimisiin'. Ratkaisu on tällöin määrättävä muilla keinoin.
Esim.\ jos alkuarvo $y(x_0)$ on tunnettu, voidaan käyttää alkuarvotehtävien numeerisia
ratkaisumenetelmiä (ks.\ Luku \ref{DYn numeeriset menetelmät}). Myös Taylorin polynomit 
antavat likimääräistä tietoa ratkaisun kulusta (vrt.\ Harj.teht.\,\ref{H-dy-1: DY ja Taylor}).

\Harj
\begin{enumerate}

\item
Tarkista, että $y=2x+Ce^{x}$ on differentiaaliyhtälön $y'=y+2(1-x)$ yleinen ratkaisu. Piirrä
pisteiden $(0,1)$ ja $(0,-1)$ kautta kulkevat ratkaisukäyrät.

\item
Ratkaise kvadratuureilla (yleinen ratkaisu tai alkuarvotehtävän ratkaisu):
\begin{align*}
&\text{a)}\ \ y''=\sin x \qquad
 \text{b)}\ \ y'''=24x+\cos x \qquad
 \text{c)}\ \ y''=\ln x \\[1mm]
&\text{d)}\ \ y''=\frac{1}{x^2+1}\,,\,\ x\in\R,\,\ y(0)=1,\ y'(0)=0 \\
&\text{e)}\ \ y''=\frac{1}{x^2-2x}\,,\,\ x\in(0,2),\,\ y(1)=y'(1)=0
\end{align*}

\item Määritä seuraavien differentiaaliyhtälöiden yleiset ratkaisut palauttamalla yhtälöt
ensimmäiseen kertalukuun (sijoitus $u(x)=y^{(k)}(x)$).
\[
\text{a)}\ \ y''-y'=0 \qquad
\text{b)}\ \ 2y'''+3y''=0 \qquad
\text{c)}\ \ y^{(5)}-5y^{(4)} = 0
\]

\item
Määritä seuraavien käyräparvien differentiaaliyhtälöt
\begin{align*}
&\text{a)}\ \ y=\sin(x+C) \qquad
 \text{b)}\ \ y=C_1+C_2\ln\abs{x} \qquad
 \text{c)}\ \ y=C_1+\frac{1}{x+C_2} \\
&\text{d)}\ \ y=(1+C_1)\ln\abs{x+C_2}-C_1 x+C_2 \qquad
 \text{e)}\ \ x=C_1e^y+C_2e^{-y}+3 \\[2mm]
&\text{f)}\ \ \text{$x$-kaselia sivuavat ympyrät} \qquad
 \text{g)}\ \ \text{suoraa $\,y=x\,$ sivuavat ympyrät}
\end{align*}

\item 
Käyrän $y=F(x)$ liukuessa pitkin $y$-akselia muodostuu käyräparvi. Määritä ko.\ parven
kohtisuorat leikkaajat, kun a) $F(x)=e^x$, b) $F(x)=\ln\abs{x}$.

\item
Käyrä $y=u(x)$ leikkaa kohtisuorasti differentiaaliyhtälön $y'=x+y^2$ ratkaisukäyrät.
Minkä differentialiyhtälön ratkaisu $u$ on?

\item
Esitä normaalimuotoisena differentiaaliyhtälösysteeminä:
\begin{align*}
&\text{a)}\ \ yy''+xy'=0 \qquad
 \text{b)}\ \ y''=(x+y')^2+y''' \qquad
 \text{c)}\ \ y^{(4)}=\frac{y'y''}{1+x+y'''} \\
&\text{d)}\ \ \begin{cases} u'=(u+v)^2, \\ v''=x+uv' \end{cases} \quad
 \text{e)}\ \ \begin{cases} u''=uv, \\ v''=-xuv \end{cases} \quad
 \text{f)}\ \ \begin{cases} u^{(4)}=u''v''+v, \\ v''=x+u'''+2v' \end{cases}
\end{align*}

\item
Alkuarvotehtävällä
\[
\begin{cases} \,y'=xy+\sin y, \quad x\in\R, \\ \,y(x_0)=y_0 \end{cases}
\]
on yksikäsitteinen ratkaisu jokaisella $(x_0,y_0)\in\Rkaksi$. \ a) Päättele, että ratkaisu
on $\R$:ssä mielivaltaisen monta kertaa derivoituva (sileä). \ b) Näytä, että ratkaisukäyrä
joko sivuaa $x$-akselia tai ei kosketa sitä lainkaan.

\item
Alkuarvotehtävä $xy'=x+y,\ y(1)=1$ määrittelee pisteen $P=(1,1)$ kauttaa kulkevan käyrän $S$.
Määritä $S$:n kaarevuuskeskiö pisteessä $P$.

\item \label{H-dy-1: DY ja Taylor}
Määritä alkuarvotehtävän ratkaisufunktion $y(x)$ Taylorin polynomi $T_n(x,0)$ yleisellä tai 
annetulla $n$:n arvolla: \newline
a) \ $y''=y,\ y(0)=y'(0)=1$ \newline
b) \ $y''=-y,\ y(0)=1,\ y'(0)=0$ \newline
c) \ $y'=x+y^2,\ y(0)=1;\ n=3$ \newline
d) \ $y'=xy+\sin y,\ y(0)=1;\ n=3$ \newline
e) \ $yy''+y'+y=0,\ y(0)=1,\ y'(0)=0;\ n=4$ \newline
f) \ $y''=yy'-x^2,\ y(0)=y'(0)=1;\ n=4$

\item (*) \index{zzb@\nim!Koirakäyrä}
(Koirakäyrä) Rekan perävaunu on kiinnitetty vetoautoon akselitapilla, joka on origossa.
Vetoauton nokka osoittaa postiivisen $y$-akselin suuntaan. Perävaunun keskiviiva on
$x$-akselilla, ja perävaunun akselin (siis sen jolla pyörät ovat) keskipiste on pisteessä
$(a,0),\, a > 0$. Vetoauton liikkuessa pitkin positiivista $y$-akselia piirtää perävaunun
akselikeskiö erään käyrän $y=y(x)$ välillä $0<x\le a$. Johda tälle ''koirakäyrälle''
differentiaaliyhtälö
\[
y'=-\frac{\sqrt{a^2-x^2}}{x},
\]
ja määritä käyrä $y=y(x)$ tämän ratkaisuna. Huomioi myös alkuehto.

\item (*)
Seuraavat käyräparvet on annettu joko parametrimuodossa tai napakoordinaattien avulla.
Johda käyräparvien ja niiden kohtisuorien leikkaajien differentiaaliyhtälöt normaalimuodossa
$y'=f(x,y)$.
\begin{align*}
&\text{a)}\ \ \begin{cases} \,x=t+\cos t+C, \\ \,y=1+\sin t \end{cases} \qquad
 \text{b)}\ \ \begin{cases} \,x=e^t+t+C, \\ \,y=2e^t-t+2C \end{cases} \\[2mm]
&\text{c)}\ \ r=C\cos\varphi \qquad
 \text{d)}\ \ r=C\varphi \qquad
 \text{e)}\ \ r=Ce^\varphi
\end{align*}

\item (*) \index{verhokäyrä}
Käyrää $S:\,y=y_0(x)$ sanotaan yksiparametrisen käyräparven $y=Y(x,C)$ \kor{verhokäyräksi}
(engl.\ envelope),
jos $S$ sivuaa jokaista parven käyrää (eli $S$:llä ja jokaisella parven käyrällä on yhteinen
piste ja siinä yhteinen tangentti). \vspace{1mm}\newline
a) Näytä, että jos käyräparven differentiaaliyhtälö on $F(x,y,y')=0$, niin myös $y=y_0(x)$ on
tämän differentiaaliyhtälön (erikois)ratkaisu. \newline
b) Suoraparvella $y=Cx+2C^2,\ C\in\R$ on verhokäyränä eräs toisen asteen polynomikäyrä
(paraabeli). Määritä tämä ja varmista piirtämällä kuvio! 

\item (*) \index{zzb@\nim!Sotaharjoitus 3}
(Sotaharjoitus 3) Tykinkuulan lentorata on parametrinen käyrä $\vec r=\vec r(t)$ ($t=$ aika).
Lentoradan pisteessä $(x,y,z) \vastaa \vec r\,$ kuulaan vaikuttavat voimat ovat
\begin{align*}
\vec G &= -mg\vec k, \\
\vec T &= T_1(x,y,z,t)\vec i+T_2(x,y,z,t)\vec j, \\
\vec F &= -k\abs{\vec v}\vec v,
\end{align*}
missä $\vec G$ on painovoima ($m=$ massa, $g=$ maan vetovoiman kiihtyvyys), $\vec T$ edustaa
tuulta ja $\vec F$ vauhdin neliöön verrannollista ilmanvastusvoimaa ($\vec v=\dvr,\ k=$ vakio).
Esitä ammuksen liikeyhtälö $\,m\vec r\,''=\vec G+\vec T+\vec F$ normaalimuotoisena 
differentiaaliyhtälösysteeminä kokoa $n=6$ kirjoittamalla
\[
(x,y,z,x',y',z') = (y_1, \ldots , y_6).
\]

\end{enumerate} %Differentiaaliyhtälöiden peruskäsitteet
\section{Separoituva differentiaaliyhtälö} \label{separoituva DY}
\alku
\index{differentiaaliyhtälö!h@separoituva|vahv}
\index{separoituva DY|vahv}

Ensimmäisen kertaluvun normaalimuotoinen differentiaaliyhtälö on \kor{separoituva}, jos se on
muotoa
\[
y'=\frac{f(x)}{g(y)}\,,
\]
eli jos muuttujat $x$ ja $y$ erottuvat (separoituvat) oikealla puolella. Kirjoittamalla $g(y)$:n
tilalle $1/g(y)$ ja/tai $f(x)$:n tilalle $1/f(x)$ saadaan separoituvuudelle muita 
ilmenemismuotoja. Oletettu muoto on kätevä lähinnä ratkeavuusoletuksien muotoilun kannalta.
Jatkossa oletetaan funktioista $f$ ja $g$, että joillakin $x_1<x_2$ ja $y_1<y_2$ pätee
\begin{itemize}
\item[(i)] $f$ on jatkuva välillä $[x_1,x_2]$ ja $g$ on jatkuva välillä $[y_1,y_2]$,
\item[(ii)] $g(y)\neq 0\quad\forall y\in [y_1,y_2]$.
\end{itemize}
Valitaan nyt $x_0\in (x_1,x_2)$, $y_0\in (y_1,y_2)$ (avoimet välit!) ja tarkastellaan 
alkuarvotehtävää
\begin{equation} \label{separ 1}
\begin{cases} 
\,y'=f(x)/g(y), \quad x\in(x_0-\delta,x_0+\delta) \\ \,y(x_0)=y_0
\end{cases} \tag{$\star$}
\end{equation}
Tässä valitaan $\delta$ ensinnäkin niin, että toteutuu
\[
0 < \delta \le \delta_1 = \min \{x_0-x_1,x_2-x_0\},
\]
jolloin $[x_0-\delta,x_0+\delta]\subset [x_1,x_2]$. Jatkossa tehdään myös toinen oletus, joka
koskee alkuarvotehtävän \eqref{separ 1} ratkeavuutta. Tämäkin ehto rajoittaa $\delta$:n
valintaa, mutta toistaiseksi jätetään avoimeksi, miten. Oletus muotoillaan seuraavasti:
\begin{itemize}
\item[(iii)] Alkuarvotehtävällä \eqref{separ 1} on välillä $(x_0-\delta,x_0+\delta)$ 
             derivoituva ratkaisu $y(x)$, ja pätee 
             $y(x)\in[y_1,y_2]\,\ \forall x\in(x_0-\delta,x_0+\delta)$.
\end{itemize}

Tehtyjen oletuksien (i)--(iii) perusteella funktiot $f(x)$, $y(x)$ ja $g(y(x))$ ovat
jatkuvia välillä $(x_0-\delta,x_0+\delta)$ ja $g(y(x)) \neq 0$ ko.\ välillä. Tähän ja Analyysin
peruslauseeseen vedoten voidaan päätellä
\begin{align*}
y'(x)=\frac{f(x)}{g(y(x))} 
           &\qekv g(y(x))y'(x)=f(x) \\
           &\qekv \frac{d}{dx}\int_{y_0}^{y(x)} g(t)\,dt
                     =\frac{d}{dx}\int_{x_0}^x f(t)\,dt,\quad x\in (x_0-\delta,x_0+\delta)
\end{align*}
eli
\begin{align*}
y'(x)= \frac{f(x)}{g(y(x))} \qekv
       &\frac{d}{dx}[G(y(x))-F(x)]=0,\quad x\in (x_0-\delta,x_0+\delta), \\
       &G(y)=\int_{y_0}^y g(t)dt,\quad F(x)=\int_{x_0}^x f(t)dt.
\end{align*}
Tämän mukaan on oltava $G(y(x))-F(x)=C=$ vakio välillä $(x_0-\delta,x_0+\delta)$
(Lause \ref{Integraalilaskun peruslause}). Toisaalta koska $G(y(x_0))=G(y_0)=0$ ja $F(x_0)=0$,
niin $C=0$, joten on päätelty:
\[
G(y(x))=F(x),\quad x\in (x_0-\delta,x_0+\delta).
\]
Varmistetaan nyt, että tämä yhtälö määrittelee implisiittisesti funktion $y(x)$. Ensinnäkin
oletuksista (i)--(ii) funktiolle $g$ ja Lauseesta \ref{Weierstrassin peruslause} seuraa, että
\[
\text{joko:}\ \  g(y) \ge c>0 \ \ \forall y\in [y_1,y_2], \qquad
\text{tai:}\ \   g(y) \le -c<0\ \ \forall y\in [y_1,y_2].
\]
(Tässä $c$:n suurin arvo $=|g(y)|$:n minimiarvo välillä $[y_1,y_2]$.) Tämän perusteella funktio
$G(y)$ on joko aidosti kasvava tai aidosti vähenevä välillä $[y_1,y_2]$, siis $G$ on 1--1 ko.\
välillä.
\begin{figure}[H]
\setlength{\unitlength}{1cm}
\begin{center}
\begin{picture}(14,4)(-1,0)
\multiput(0,0)(7,0){2}{
\put(0,0){\vector(1,0){6}} \put(5.8,0.2){$x$}
\put(0,0){\vector(0,1){4}} \put(0.2,3.8){$y$}
}
\curve(1,1.5,3,2,5,3)
\dashline{0.2}(3,0)(3,2) \dashline{0.2}(3,2)(0,2)
\put(4.55,3.2){$x=G(y)$}
\put(-0.4,1.9){$y_0$} \put(-0.4,1.4){$y_1$} \put(-0.4,2.9){$y_2$}
\put(0,1.5){\line(1,0){0.1}} \put(0,3){\line(1,0){0.1}} 
\put(2.9,-0.5){$x_0$} 
\curve(8,3,10,2,12,1.5)
\dashline{0.2}(10,0)(10,2) \dashline{0.2}(10,2)(7,2)
\put(7.5,3.2){$x=G(y)$}
\put(6.6,1.9){$y_0$} \put(6.6,1.4){$y_1$} \put(6.6,2.9){$y_2$}
\put(7,1.5){\line(1,0){0.1}} \put(7,3){\line(1,0){0.1}}
\put(9.9,-0.5){$x_0$} 
\end{picture}
\end{center}
\end{figure}
Koska siis $G$ on 1--1 välillä $[y_1,y_2]$ ja koska oletuksen (iii) mukaan $y(x)\in[y_1,y_2]$
kun $x\in(x_0-\delta,x_0+\delta)$, niin voidaan kirjoittaa
\[
G(y(x))=F(x) \ \ekv \ y(x)=\inv{G}(F(x)),\quad x\in(x_0-\delta,x_0+\delta).
\]
Oletuksien (i)--(iii) voimassa ollessa on näin muodoin päätelty:
\[
y'(x)=\frac{f(x)}{g(y(x))} \qekv y(x)=\inv{G}(F(x)),\quad x\in(x_0-\delta,x_0+\delta).
\]
Lähtemällä saadusta ratkaisukaavasta nähdään em.\ päättelyketjusta, että kaavan mukainen
$y(x)$ on alkuarvotehtävän \eqref{separ 1} yksikäsitteinen (derivoituva) ratkaistu, sikäli kuin
oletukseen (iii) sisältyvä ehto $y(x)\in[y_1,y_2]\ \forall x\in(x_0-\delta,x_0+\delta)$ on 
voimassa. On siis enää varmistettava, että myös tämä ehto toteutuu saadulle (ainoalle 
mahdolliselle) ratkaisulle $y(x)=G^{-1}(F(x))$.

Todetaan ensinnäkin, että em.\ arvioista funktiolle $g(y)$ sekä integraalien 
vertailuperiaatteesta (Lause \ref{integraalien vertailuperiaate}) seuraa
\[
\Bigl|\int_{y_0}^{y(x)} g(t)dt\Bigr| \,\ge\, c\,\abs{y(x)-y_0} \quad 
                                            \forall x\in (x_0-\delta,x_0+\delta).
\]
Toisaalta oletuksen (i) ja Lauseiden \ref{Weierstrassin peruslause} ja
\ref{integraalien vertailuperiaate} perusteella pätee
\[
\Bigl|\int_{x_0}^x f(t)dt\Bigr|\,\le\,M\abs{x-x_0}, \quad x\in[x_1,x_2],
\]
missä $M=\max_{x\in[x_1,x_2]}|f(x)|$. Yhdistämällä arviot seuraa
\begin{align*}
c\,\abs{y(x)-y_0}        &\,\le\, \abs{G(y(x))} \,=\, \abs{F(x)} \,\le\, M\abs{x-x_0} \\
  \impl \ \abs{y(x)-y_0} &\,\le\, \frac{M}{c}\abs{x-x_0} \,<\, \frac{M\delta}{c}
                                                   \quad \forall x\in (x_0-\delta,x_0+\delta).
\end{align*}
Näin ollen kun valitaan $\delta$ siten, että toteutuu
\[
0<\delta\le\min\{\delta_1,\delta_2\}, \quad \delta_2 = \frac{c}{M}\min \{y_0-y_1,y_2-y_0\},
\]
niin oletuksen (iii) toteutuminen on varmistettu ja näin todistettu
\begin{Lause} \label{separoituvan DY:n ratkaisu} \index{differentiaaliyhtälön!h@ratkeavuus|emph}
Jos $x_0\in (x_1,x_2)$ ja $\,y_0\in (y_1,y_2)$,
niin oletuksien (i),\,(ii) ollessa voimassa on olemassa $\delta>0$ siten, että 
alkuarvotehtävällä \eqref{separ 1} on yksikäsitteinen ratkaisu $y(x)$ välillä 
$(x_0-\delta,x_0+\delta)\subset [x_1,x_2]$. Ratkaisukaava ko.\ välillä on
\[
\int_{y_0}^{y(x)} g(t)dt = \int_{x_0}^x f(t)dt.
\]
\end{Lause}
Helposti muistettava, muodollinen menettely separoituvan differentiaaliyhtälön yleisen ratkaisun
hakemiseksi on kaavio
\begin{align*}
\frac{dy}{dx} = \frac{f(x)}{g(y)} \quad &\impl \quad g(y)\,dy = f(x)\,dx \\
                                        &\impl\,\ \int g(y)\,dy = \int f(x)\,dx.
\end{align*}
Tässä ensimmäinen vaihe on 'separointi' ja toinen 'integrointi'. Menettely on pätevä Lauseen
\ref{separoituvan DY:n ratkaisu} ehdoin, ja alkuehdon $y(x_0)=y_0$ toteuttava ratkaisu saadaan
siis muuttamalla määräämättömät integraalit määrätyiksi.
\begin{Exa} Aiemmin Luvussa \ref{exp(x) ja ln(x)} ratkaistussa alkuarvotehtävässä
\[
\left\{ \begin{aligned}
&y' = y,\quad x\in\R, \\
&y(0) = y_0
\end{aligned} \right.
\]
differentiaaliyhtälö on em.\ separoituvaa tyyppiä ($f(x)=1,\ g(y)=1/y$), joten ratkaisukaava on
Lauseen \ref{separoituvan DY:n ratkaisu} mukaan
\[
\int_{y_0}^{y(x)} \frac{1}{t}\,dt = \int_0^x dt.
\]
Kaava on pätevä, jos joko $y_0,y(x)\in(0,\infty)$ tai $y_0,y(x)\in(-\infty,0)$, jolloin
ratkaisuksi saadaan
\[
\ln (y/y_0)=x \ \ekv \ y(x)=y_0e^x, \quad x\in\R.
\]
Saatu ratkaisu on käypä myös tapauksessa $y_0=0$, johon  separointimenettely ei sovellu. \loppu
\end{Exa}
\begin{Exa}
Määritä differentiaaliyhtälön $\,1+y^2+xyy'=0\,$ yleinen ratkaisu.
\end{Exa}
\ratk Separoimalla ja integroimalla saadaan
\begin{align*}
\frac{ydy}{1+y^2} = -\frac{dx}{x} \ &\impl\ \int \frac{ydy}{1+y^2} = -\int \frac{dx}{x} \\
                                    &\impl\ \frac{1}{2}\ln (1+y^2) = - \ln\abs{x} +C.
\end{align*}
Kun kirjoitetaan $C$:n tilalle $\ln C,\ C>0$, saadaan sieventämällä
\[
x^2(1+y^2)=C^2.
\]
Ratkaisemalla tästä $y=y(x)$ saadaan kaksihaarainen ratkaisu
\[
y(x)=\pm\frac{1}{x}\sqrt{C^2-x^2},\quad C>0.
\]
Tämä on pätevä väleillä $(-C,0)$ ja $(0,C)$. \loppu

Esimerkissä vaihto $C\ext\ln C$ perustui siihen, että $\ln C$ saa kaikki mahdolliset
reaaliarvot, kun $C\in\R_+$. Tällaisilla manipulaatioilla voidaan ratkaisu usein saattaa
yksinkertaisempaan muotoon.
\begin{Exa} \label{muuan separoituva dy}
Ratkaise differentiaaliyhtälö $\,y'=\sin y$.
\end{Exa}
\begin{align*}
\text{\ratk}\quad \frac{dy}{\sin y} = dx 
                       &\qimpl \int\frac{dy}{\sin y} = \int dx \\
                       &\qimpl \ln\Bigl|\tan\frac{1}{2}y\Bigr| = x +\ln C,\quad C>0 \\
                       &\qekv \Bigl|\tan\frac{1}{2}y\Bigr| = Ce^x,\quad C>0 \\
                       &\qekv \tan\frac{1}{2} y = Ce^x,\quad C\neq 0 \quad 
                                                       (\,\pm C \hookrightarrow C) \\[2mm]
                       &\qekv y(x) = 2\arctan(Ce^x), \quad C \neq 0 \\[3mm]
                       &\qekv y(x) = 2\Arctan(Ce^x)+2n\pi, \quad C \neq 0,\ n\in\Z.
\end{align*}

Saatu ratkaisu on käypä myös kun $C=0$, sillä $y(x)=2n\pi$ on alkuperäisen
differentiaaliyhtälön ratkaisu jokaisella $n\in\Z$. Nämä ratkaisut voidaan lukea yleiseen
ratkaisuun kuuluviksi, vaikka separointimenettely ei niihin suoraan ulotu. Yleiseen ratkaisuun
kuulumattomia erikoisratkaisuja ovat lisäksi
\[
y(x)=(2n+1)\pi, \quad n\in\Z.
\]
Nämä ovat em.\ ratkaisufunktioiden raja-arvoja, kun $C\kohti\pm\infty$.
\loppu

\subsection{Autonominen differentiaaliyhtälö}
\index{differentiaaliyhtälö!i@autonominen|vahv}
\index{autonominen DY|vahv}

Separoituvaa differentiaaliyhtälöä muotoa
\[
y'=f(y),
\]
missä siis $f$ ei riipu vapaasta muuttujasta $x$, sanotaan \kor{autonomiseksi}. Kuten 
Esimerkissä 3 edellä, jokaista $f$:n nollakohtaa $y_i$ vastaa vakioratkaisu
\[
y(x)=y_i.
\]
Tämä on erikoisratkaisu tai mahdollisesti yleiseen ratkaisulausekkeeseen sisällytettävissä
oleva, vrt.\ esimerkit edellä. Separointimenettelyllä ei tällaisia ratkaisuja saada suoraan,
koska funktio $g(y)=1/f(y)$ ei ole jatkuva $f$:n nollakohtien ympäristössä 
(vrt.\ Lause \ref{separoituvan DY:n ratkaisu}). 
\begin{Exa} Ratkaise differentiaaliyhtälöt $\,y'=y(1-y)$.
\end{Exa}
\ratk Separoimalla ja integroimalla saadaan
\begin{align*}
dx = \frac{dy}{y(1-y)}\ &\impl\ \int dx = x = \int \frac{1}{y(1-y)}\,dy 
                                            = \int \left(\frac{1}{y}+\frac{1}{1-y}\right)dy \\
                        &\impl\ x = \ln\left|\frac{y}{1-y}\right|+\ln C, \quad C>0.
\end{align*}
Tämän mukaan on
\begin{align*}
\frac{y}{1-y} = \pm C^{-1}e^x &\qekv y(x) = \frac{e^x}{e^x \pm C}\,, \quad C>0 \\
                              &\qekv y(x) = \frac{e^x}{e^x+C}\,,\quad C\neq 0.
\end{align*}
Rajoitus $C \neq 0$ voidaan poistaa saadusta yleisen ratkaisun lausekkeesta, koska myös $y(x)=1$
on ratkaisu. Tämä on kaikkien ratkaisujen yhteinen raja-arvo (asymptoottinen ratkaisu), kun
$x\kohti\infty$. Ratkaisu on edelleen myös $y(x)=0$, joka saadaan yleisen ratkaisulausekkeen
raja-arvona, kun $C\kohti\infty$. \loppu
\begin{Exa}: \vahv{Logistinen kasvumalli}. \index{zza@\sov!Logistinen kasvumalli} \ Jos
fysikaalisen suureen (esim.\ väkiluvun) kasvua ajan $t$ funktiona kuvaa autonominen
differentiaaliyhtälö
\[
y'=ay-by^2,
\]
missä $a$ ja $b$ ovat (dimensiottomina) positiivisia vakioita, niin kasvumallia 
(myös differentiaaliyhtälöä) sanotaan
\index{differentiaaliyhtälö!j@logistinen} \index{logistinen DY}%
\kor{logistiseksi}. Sijoituksella $y(t)=(a/b)\,u(t)$ logistinen DY muuntuu muotoon
\[
u'=a(u-u^2).
\]
Ratkaisu separoimalla (vrt.\ edellinen esimerkki):
\[
a\,dt = \frac{du}{u(1-u)} \qimpl u(t)=\frac{e^{at}}{e^{at} + C}\,, \quad C\in\R. 
\]
Ratkaisu on fysikaalisesti järkevä (kasvava) vain kun $C>0$. Tässä tapauksessa voidaan 
kirjoittaa $C=e^{at_0},\ t_0\in\R$. Kun vielä merkitään $\tau=1/a$ (aikavakio), niin ratkaisut
saadaan muotoon
\[
y(t)=A\,Y\left(\frac{t-t_0}{\tau}\right), \quad A=a/b, \quad \tau=1/a, \quad 
Y(t)=\frac{e^{t}}{e^{t}+1} \quad (t_0\in\R). \loppu
\]
\end{Exa}
\begin{figure}[H]
\setlength{\unitlength}{1cm}
\begin{center}
\begin{picture}(10,5)(0,-0.5)
\put(-1,0){\vector(1,0){12}} \put(10.8,-0.5){$t$}
\put(0,-1){\vector(0,1){5}} \put(0.2,3.8){$y(t)$}
\put(0,3){\line(-1,0){0.1}} \put(-0.6,2.9){$A$}
\put(5,0){\line(0,-1){0.1}} \put(4.9,-0.5){$t_0$}
\put(3,0){\line(0,-1){0.1}} \put(2.6,-0.5){$t_0-\tau$}
\put(7,0){\line(0,-1){0.1}} \put(6.6,-0.5){$t_0+\tau$}
\curve(
-1.0,  0.1423,
 0.0,  0.2276,
 1.0,  0.3576,
 2.0,  0.5473,
 3.0,  0.8068,
 4.0,  1.1326,
 5.0,  1.5000,
 6.0,  1.8674,
 7.0,  2.1932,
 8.0,  2.4527,
 9.0,  2.6424,
10.0,  2.7724,
11.0,  2.8577)
\end{picture}
\end{center}
\end{figure}

\subsection{Differentiaaliyhtälö $\,y'=f(y/x)$}
\index{differentiaaliyhtälö!k@tasa-asteinen|vahv}
\index{tasa-asteinen DY|vahv}

\kor{Tasa-asteinen} differentiaaliyhtälö $y'=f(y/x)$ muuntuu sijoituksella
\[
u(x)=y(x)/x \ \impl \ y(x)=xu(x) \ \impl \ y'=xu'+u
\]
yhtälöksi
\[
xu'=f(u)-u.
\]
Tämä on separoituva, joten tasa-asteinen yhtälö on \pain{se}p\pain{aroituvaksi} 
p\pain{alautuva}. Jos $f(u_0)-u_0=0$, niin erikoisratkaisu on
\[
u(x)=u_0 \qimpl y(x)=u_0x.
\]
Yleinen ratkaisu löydetään separoimalla.
\begin{Exa}
Ratkaise differentiaaliyhtälö
\[
y'=\frac{x+y}{x-y}\,.
\]
\end{Exa}
\ratk Tämä on tasa-asteinen, joten tehdään sijoitus $u=y/x$\,:
\[
\impl\ x\frac{du}{dx} \,=\, \frac{1+u}{1-u}-u \,=\, \frac{1+u^2}{1-u}\,.
\]
Separoimalla ja integroimalla seuraa
\begin{align*}
\int\frac{dx}{x} = \ln\abs{x} &= \int\frac{1-u}{1+u^2}\,du \\
                              &= \Arctan u-\frac{1}{2}\ln (1+u^2)+\ln C \\
              \ekv\ \Arctan u &= \ln\left(\frac{\abs{x}\sqrt{1+u^2}}{C}\right) \\
     \ekv\ \Arctan\frac{y}{x} &= \ln\left(\frac{\sqrt{x^2+y^2}}{C}\right)\quad (C>0).
\end{align*}
Siirtymällä napakoordinaatistoon saadaan ratkaisulle helpompi muoto
\[
\varphi = \ln\frac{r}{C} \ \ekv \ r=Ce^\varphi\quad (C>0).
\]
Ratkaisut ovat nk.\
\index{logaritminen spiraali}%
\kor{logaritmisia spiraaleja}. \loppu
\input{plots/logspiraalit.tex}

\subsection{Differentiaaliyhtälö $\,y'=f(ax+by+c)$}
\index{differentiaaliyhtälö!l@$y'=f(ax+by+c)$|vahv}

Differentiaaliyhtälö $y'=f(ax+by+c)$, missä $a,b,c\in\R$ ja $b \neq 0$, on myös separoituvaksi
palautuva. Tässä luonteva sijoitus on
\[
u=ax+by+c \ \impl \ y=\frac{1}{b}(u-ax-c),
\]
jolloin
\[
\frac{dy}{dx}=\frac{1}{b}\left(\frac{du}{dx}-a\right),
\]
ja näin ollen $u$:lle saadaan separoituva (itse asiassa autonominen) DY
\[
u'=bf(u)+a.
\]
\begin{Exa}
Ratkaise differentiaaliyhtälö
\[
y'=(x-y)^2+1.
\]
\end{Exa}
\ratk Sijoituksella
\[
x-y=u \ \ekv \ y=x-u \ \impl \ y'=1-u'
\]
saadaan differentiaaliyhtälöksi
\[
u'=-u^2.
\]
Separoimalla saadaan yleiseksi ratkaisuksi $u(x)=1/(x+C)$, joten alkuperäisen yhtälön yleinen
ratkaisu on
\[
y(x)=x-\frac{1}{x+C}\,.
\]
Tämän lisäksi on muunnetun yhtälön ratkaisua $u(x)=0$ vastaava erikoisratkaisu
\[
y(x)=x. \loppu
\]

\Harj
\begin{enumerate}

\item 
Määritä seuraavien separoituvien tai sellaisiksi palautuvien differentiaaliyhtälöiden
yleiset ratkaisut tai ratkaise alkuarvotehtävä. Alkuarvotehtävän tapauksessa selvitä myös,
millä välillä ratkaisu on pätevä.
\begin{align*}
&\text{a)}\ \ x^2 y'=y^2 \qquad 
 \text{b)}\ \ y'=-2xy \qquad 
 \text{c)}\ \ y'=(1-y)^2 \qquad
 \text{d)}\ \ y'=y^2-1 \\[2mm]
&\text{e)}\ \ (1+x)y'=1+y \qquad
 \text{f)}\ \ 1+y^2-xy'=0 \qquad
 \text{g)}\ \ (1+x^3)y'=x^2y \\[2mm]
&\text{h)}\ \ (1-x^2)y'=1-y^2 \qquad
 \text{i)}\ \ y'=\tan y \qquad
 \text{j)}\ \ xy'=\cot y \\[2mm]
&\text{k)}\ \ (x-3)y'=-y,\ y(-1)=1 \qquad
 \text{l)}\ \ y'=y^2,\ y(0)=4 \\[2mm]
&\text{m)}\ \ y'+5x^4y^2=0,\ y(0)=1 \qquad
 \text{n)}\ \ y'-5x^4y^2=0,\ y(0)=1 \\[2mm] 
&\text{o)}\ \ y'\Arctan y=1,\ y(1)=1 \qquad
 \text{p)}\ \ y'\sin x=y\ln y,\ y(\pi/2)=1 \\[2mm]
&\text{q)}\ \ (1+e^x)yy'=e^x,\ y(1)=1 \qquad
 \text{r)}\ \ \cos^2 x\cos(\ln y)y'=y,\ y(\pi/4)=2 \\[1mm]
&\text{s)}\ \ y'=\frac{y}{x-y} \qquad
 \text{t)}\ \ (3x+y)y'=y \qquad
 \text{u)}\ \ xyy'=x^2+y^2 \\
&\text{v)}\ \ (2x^2+y^2)y'=2xy \qquad
 \text{x)}\ \ xy'=y+\tan\frac{y-x}{x} \qquad
 \text{y)}\ \ y'=e^{x-2y} \\  
&\text{z)}\ \ xy'=xe^{y/x}+y,\ y(1)=0 \qquad
 \text{å)}\ \ xy'=y\ln\frac{y}{x},\ y(1)=1 \\
&\text{ä)}\ \ (x+y)y'=1-x-y,\ y(0)=1 \qquad 
 \text{ö)}\ \ y'=(2x+y+3)^2,\ y(0)=0
\end{align*}

\item
Ratkaise alkuarvotehtävä $\,e^y y'=x+e^y-1,\ y(0)=y_0$ sijoituksella $u=x+e^y$. Millä $y_0$:n
arvoilla ratkaisu on pätevä koko $\R$:ssä?

\item
Määritä differentiaaliyhtälön $y'=2x\abs{y-1}$ ratkaisukäyrät, jotka sivuavat $x$-akselia.

\item
Määritä seuraavien käyräparvien kohtisuorat leikkaajat.
\[
\text{a)}\,\ y=Cx^2 \quad\
\text{b)}\,\ y=e^x+C \quad\
\text{c)}\,\ x^2+2y^2=C^2 \quad\
\text{d)}\,\ e^{x+y}=(x+C)^2
\]

\item
Määritä käyrät, joilla on ominaisuus: käyrän ja sen normaalin leikkauspisteen etäisyys ko.\
normaalin ja $x$-akselin leikkauspisteestä $=a=$ vakio.

\item \index{integraaliyhtälö}
Määritä välillä $[0,\infty)$ jatkuva funktio $y(x)$, joka toteuttaa \kor{integraaliyhtälön}
\[
1+\int_0^x \frac{y(t)}{t^2+1}\,dt = y(x), \quad x \ge 0.
\]

\item \index{zzb@\nim!Hypetia, Utopia ja Apatia}
(Hypetia, Utopia ja Apatia) H:n, U:n ja A:n valtakunnissa oli elintaso $=1$ vuonna 2000 ja
$=1.01$ vuonna 2001. Ennusta ko.\ valtakuntien elintasot vuonna $2100$, kun tiedetään, että
elintason kasvunopeus on H:ssa suoraan verrannollinen elintason neliöön, U:ssa suoraan
verrannollinen elintasoon ja A:ssa kääntäen verrannollinen elintasoon.

\item 
Vuonna 1960 oli maapallon väkiluku 3.0 mrd ja ja kasvunopeus tuolloin 50 milj./v. Jos
oletetaan, että väkiluku $y(t) \kohti 12$ mrd kun $t\kohti\infty$, niin mikä olisi väkiluvun
pitänyt olla vuonna 2000 logistisen kasvumallin mukaan? (Todellinen väkiluku oli 6.1 mrd.)

\item
Laskuvarjohyppääjän putoamisnopeus noudattaa varjon auettua liikelakia $\,mv'=mg-av-bv^2$,
missä $m=80$ kg, $g=9.8\ \text{m/s}^2$, $a=120$ kg/s ja $b=4.0$ kg/m. Määritä $v(t),\ t \ge 0$
(aikayksikkö = s), kun varjon aukeamishetkellä $t=0$ putoamisnopeus on $20$ m/s. 
Miten rajanopeus $v_\infty=\lim_{t\kohti\infty}v(t)$ saadaan helpoimmin selville? Hahmottele
$v(t)$ aikaväleillä $[0,1]$ ja $[0,60]$.

\item
Avaruudessa suuntaan $-\vec k$ etenevät valonsäteet heijastuvat pyörähdyspinnasta
$S:\,z=u(r),\ r=\sqrt{x^2+y^2}\,$ siten, että jokainen heijastunut säde kulkee origon
kautta. Funktion $u(r)$ on tällöin toteutettava tasa-asteinen differentiaaliyhtälö
$\,ru'=\sqrt{r^2+u^2}+u$
(vrt.\ Harj.teht.\,\ref{derivaatta geometriassa}:\ref{H-dif-2: tutka}). Ratkaise!

\item (*)
Differentiaaliyhtälö
\[
y'=f\left(\frac{2x-y+1}{x-2y+1}\right)
\]
palautuu tasa-asteiseksi sijoituksilla $x=x_0+t$, $y=y_0+u$, kun $x_0$ ja $y_0$ valitaan
sopivasti. Ratkaise yhtälö tällä tavoin, kun $f(x)=x$.

\item (*)
Seuraavat differentiaaliyhtälöt palautuvat tasa-asteisiksi sijoituksilla \newline
$x=t^\alpha$, $y=z^\beta$, kun $\alpha$ ja $\beta$ valitaan sopivasti. Ratkaise! \newline
a)\ \  $42(x^2-xy^2)y'+y^3=0 \quad$ b)\ \ $(x^2y^2-1)y'+2xy^3=0$

\item (*)
Lieriön muotoisessa astiassa on vettä $200$ litraa. Astian pohjaan avataan aukko, jonka
pinta-ala $=20$ cm$^2$. Vesi purkautuu aukosta nopeudella $v=\sqrt{2gh}$, missä
$g=9.8$ m/s$^2$ ja $h=$ veden korkeus astiassa kyseisellä hetkellä. (Purkausnopeus oletetaan
samaksi aukon eri kohdissa.) Laske, missä ajassa astia tyhjenee, kun veden korkeus on aluksi
$\,h=1$ m.

\item (*)
Käyrällä $S: y=y(x),\ x \ge 0$, missä $y(x)$ on jatkuvasti derivoituva välillä $[0,\infty)$,
on ominaisuus: Käyrän kaarenpituus välillä $[0,x]$ on sama kuin käyrän ja $x$-akselin välisen
alueen pinta-ala ko.\ välillä. Määritä kaikki käyrät, jotka toteuttavat tämän ehdon ja lisäksi
alkuehdon $y(0)=2$.

\end{enumerate} %Separoituva differentiaaliyhtälö
\section{Palautuvat toisen kertaluvun DY:t}
\label{toisen kertaluvun dy}
\alku

Yleinen toisen kertaluvun normaalimuotoinen differentiaaliyhtälö $y''=f(x,y,y')$ ratkeaa
kvadratuureilla eräissä erikoistapauksissa. Näistä tarkastellaan tässä vain kahta
(sovelluksissa melko yleistä) tapausta, joissa $f$ ei riipu joko $y$:stä tai $x$:stä, jolloin 
differentiaaliyhtälö on joko muotoa $y''=f(x,y')$ tai $y''=f(y,y')$. Kummassakin tapauksessa on
kyse \pain{ensimmäiseen} \pain{kertalukuun} p\pain{alautuvasta} differentiaaliyhtälöstä. Sikäli 
(ja vain sikäli) kuin ko.\ 1.\ kertaluvun DY on separoituva, sellaiseksi palautuva tai muuten
kvadratuureilla ratkaistavissa, ratkeaa alkuperäinen DY kvadratuureilla.

\index{differentiaaliyhtälö!la@$y''=f(x,y')$}%
Differentiaaliyhtälö $y''=f(x,y')$ palautuu välittömästi 1.\ kertalukuun, sillä sijoituksella
$y'=u$ tämä voidaan kirjoittaa ryhmäksi
\[
\begin{cases} \,u'=f(x,u), \\ \,y'=u. \end{cases}
\]
Jos tässä ensimmäinen yhtälö on ratkaistavissa muotoon $u=U(x,C_1)$, ratkeaa jälkimmäinen
yhtälö yhdellä kvadratuurilla muotoon $y(x)=Y(x,C_1,C_2)$. Kvadratuureilla ratkeava on esim.\
muotoa $y''=f(y')$ oleva DY, ks.\ sovellusesimerkki luvun lopussa. Siirrytään tässä
tarkastelemaan mainittua toista erikoistapausta.

\subsection{Differentiaaliyhtälö $y''=f(y,y')$}
\index{differentiaaliyhtälö!m@$y''=f(y,y')$|vahv}

Tässäkin tapauksessa muunnetaan ratkaistava DY ensin ryhmäksi sijoituksella $y'=u\,$:
\[ 
\begin{cases} \,u'=f(y,u), \\ \,y'=u. \end{cases}
\]
Esitetään tässä $u$ yhdistettynä funktiona muodossa $u=u(x)=U(y(x))=U(y)$. (Jos ratkaisu on
$y=Y(x)$, niin $U(y)=(u \circ Y^{-1})(y)$ väleillä, joilla $Y$ on kääntyvä.)
Mainituin merkinnöin ja em.\ differentiaaliyhtälöiden perusteella seuraa
\begin{align*}
f(y,U(y)) &= \frac{du}{dx} = \frac{d}{dx}U(y(x)) = \frac{dU}{dy}\cdot\frac{dy}{dx} 
                           = \frac{dU}{dy}\cdot u(x) = \frac{dU}{dy}\cdot U(y) \\
          &\qimpl \frac{dU}{dy} =  \frac{f(y,U)}{U}\,.
\end{align*}
Sikäli kuin saatu ensimmäisen kertaluvun DY on ratkaistavissa muotoon $U(y)=V(y,C_1)$, voidaan
edelleen $y$ ratkaista separoituvasta (autonomisesta) yhtälöstä
\[
y' = U(y) = V(y,C_1).
\]
Alkuperäinen toisen kertaluvun DY on näin purettu kahdeksi peräkkäiseksi ensimmäisen
kertaluvun DY:ksi, joista jälkimmäinen on separoituva.

Em.\ ratkaisutavasta saadaan käytännössä helpommin muistettava, kun kirjoitetaan
yksinkertaisesti $u(x)=u(y)$ (eli ajatellaan $y$ muuttujaksi $x$:n sijasta),
ja suoritetaan differentiaaliyhtälöryhmässä muodollinen jakolasku puolittain:
\[ \left. \begin{aligned}
\frac{du}{dx} &= f(y,u)\ \\ \frac{dy}{dx} &= u 
          \end{aligned} \right\} \qimpl \frac{du}{dy} =  \frac{f(y,u)}{u}\,.
\]
\begin{Exa} Ratkaise diferentiaaliyhtälö $\,y''=yy'$. \end{Exa}
\ratk Tässä on $f(y,u)=yu$, joten em.\ ratkaisutavalla saadaan
\begin{align*}
\frac{du}{dy}= y &\qimpl u=\int y\,dy=\frac{1}{2}\,y^2+\frac{C_1}{2} \\
                 &\qekv \frac{dy}{dx}=\frac{1}{2}\,y^2+\frac{C_1}{2}
                  \qimpl x= \int \frac{2}{y^2+C_1}\,dy, \quad C_1\in\R.
\end{align*}
Tästä eteenpäin on eroteltava tapaukset $C_1>0$, $C_1=0$ ja $C_1<0$. Ensinnäkin jos $C_1>0$,
niin kirjoitetaan $C_1$:n tilalle $C_1^2\ (C_1>0)$, jolloin saadaan
\begin{align*}
x &= \frac{2}{C_1^2}\int\frac{1}{(y/C_1)^2+1}\,dy
   = \frac{2}{C_1}\Arctan\frac{y}{C_1}-\frac{2C_2}{C_1} \\
  &\qimpl\quad \underline{\underline{y(x)
                  =C_1\tan\left(\frac{C_1 x}{2}+C_2\right), \quad C_1>0,\ C_2\in\R}}.
\end{align*}
Tapauksessa $C_1<0$ saadaan vastaavasti kirjoittamalla $C_1$:n tilalle $-C_1^2$
\begin{align*}
x &= \frac{2}{C_1^2}\int\frac{1}{(y/C_1)^2-1}\,dy
   = -\frac{2}{C_1}\artanh\frac{y}{C_1}-\frac{2C_2}{C_1} \\
  &\qimpl\quad \underline{\underline{y(x)
                  =-C_1\tanh\left(\frac{C_1 x}{2}+C_2\right), \quad C_1>0,\ C_2\in\R}}.
\end{align*}
Lopulta tapauksessa $C_1=0$ ovat ratkaisut
\[
x = \int \frac{2}{y^2}\,dy = -\frac{2}{y}-C \qimpl \underline{\underline{y(x)
                           = -\frac{2}{x+C}\,, \quad C\in\R}}.
\]
Näiden lisäksi on vielä otettava mukaan autonomisten differentiaaliyhtälöiden
$\,y'=\tfrac{1}{2}(y^2+C_1)\,$ erikoisratkaisut, kun $C_1=-C^2 \le 0\,$:
\[
\underline{\underline{y(x)=C, \quad C\in\R}}. \loppu
\]

\subsection{Differentiaaliyhtälö $y''=f(y)$}
\index{differentiaaliyhtälö!n@$y''=f(y)$|vahv}

Myös edellä tarkastellun DY-tyypin erikoistapaus $y''=f(y)$ on sovelluksissa yleinen. Em.\
ratkaisukaavion mukaisesti tämä palautuu separoituviksi 1. kertaluvun yhtälöiksi, joten yhtälö
ratkeaa aina kvadratuureilla.
\index{harmoninen värähtely} \index{zza@\sov!Harmoninen värähtely}%
\begin{Exa}: \vahv{Harmoninen värähtely}. \ Kappale, jonka massa $=m$, on kiinnitetty jouseen,
jonka jousivakio $=k$. Hetkellä $t$ kappaleen etäisyys tasapaino\-asemasta on $y(t)$, jolloin
liikeyhtälö on
\[
my''=-ky.
\]
Ratkaise $y(t)$ alkuehdoilla $\,y(0)=0$, $y'(0)=v_0$.
\end{Exa}
\ratk Tässä on $f(y)=-(k/m)y$, joten saadaan
\begin{align*}
\frac{du}{dy} &= \frac{f(y)}{u}=-\frac{ky}{mu} \\
              &\impl\ mu\frac{du}{dy}+ky=0 \\
              &\ekv\ \frac{d}{dy}\left(\frac{1}{2}\,mu^2+\frac{1}{2}\,ky^2\right)=0 \\
              &\ekv\ \frac{1}{2}\,mu^2+\frac{1}{2}\,ky^2=C.
\end{align*}
Tässä välituloksessa on $u=y'=$ kappaleen nopeus, $E_k=\tfrac{1}{2}mu^2=$ 
\pain{liike-ener}g\pain{ia}, ja $E_p=\tfrac{1}{2}ky^2=$ jousen p\pain{otentiaaliener}g\pain{ia}.
Tulos tunnetaan \pain{ener}g\pain{ia}p\pain{eriaatteena}: Kokonaisenergia = vakio. Koska
alkuhetkellä on $y=0,\ u=v_0$, niin on oltava
\[
C=\frac{1}{2}\,mv_0^2.
\]
Jatkamalla saadusta energian säilymislaista tällä $C$:n arvolla, kirjoittamalla
\[
T=\sqrt{\frac{m}{k}},\quad L=v_0T,
\]
olettamalla, että $\abs{y}<L$, ja käyttämällä alkuehtoa $y(0)=0$ saadaan
\[
T\frac{dy}{dt}=\sqrt{L^2-y^2} \qimpl \int_0^{y(t)} \frac{1}{\sqrt{L^2-y^2}}\,dy 
              = \int_0^t \frac{1}{T}\,dt=\frac{t}{T}\,.
\]
Välillä $\,t/T\in(-\pi/2,\pi/2)\,$ ratkaisu on kirjoitettavissa muodossa
\[
\Arcsin (y/L) = t/T\,\ \ekv\,\ y(t)=L\sin (t/T),\quad t/T\in (-\pi/2,\pi/2).
\]
Lopputuloksesta nähdään, että rajoitus $t/T\in (-\pi/2,\pi/2)$ voidaan poistaa. Saatiin siis
kaikkilla $t$:n arvoilla pätevä ratkaisu $\,y(t)=v_0 T\sin(t/T)$, missä $T=\sqrt{m/k}$. \loppu

\begin{Exa}: \vahv{Räjähdys}. \index{zza@\sov!Rzyjzy@Räjähdys} \ Nopeaa kemiallista rekatiota
kuvaa DY-systeemi
\[
u'=3y^2,\,\ y'=u,
\]
missä $u=u(t)$ ja $y=y(t)$ ($t=$ aika). Oletetaan alkuehdot $y(0)=1$ ja $u(0)=0$. Laske
reaktioon kuluva aika $T$, ts.\ määritä suurin $T$ siten, että alkuarvotehtävällä on ratkaisu
välillä $(0,T)$.
\end{Exa}
\ratk Derivoimalla jälkimmäinen yhtälö ja eliminoimalla $u$ nähdään, että $y$ toteuttaa
differentiaaliyhtälön $y''=3y^2$. Ratkaisussa lähdetään kuitenkin alkuperäisestä 
systeemimuodosta, sillä se on valmiiksi ratkaisukaavioon sopiva:
\begin{align*}
\frac{du}{dy}=\frac{3y^2}{u}\ &\impl\ u\frac{du}{dy}-3y^2=0 \\
                              &\ekv\ \frac{d}{dy}\left(\frac{1}{2}\,u^2-y^3\right)=0 \\
                              &\ekv\ \frac{1}{2}\,u^2-y^3=C.
\end{align*}
Koska $y=1$ ja $u=0$, kun $t=0$, niin $C=-1$, joten
\[
u=y'=\sqrt{2}\sqrt{y^3-1}\quad (y\geq 1).
\]
Separoimalla ja integroimalla, ja käyttämällä alkuehtoa $y(0)=1$, saadaan
\[
\frac{1}{\sqrt{2}}\int_1^{y(t)}\frac{dx}{\sqrt{x^3-1}}=\int_{0}^t dx 
                                                      =t\ \ekv\ y(t)=\inv{\Phi}(t),
\]
missä
\[
\Phi(s)=\frac{1}{\sqrt{2}}\int_1^s\frac{dx}{\sqrt{x^3-1}},\quad s\geq 1.
\]
Funktio $\Phi$ ei ole alkeisfunktio, mutta koska $x^3-1=(x-1)(x^2+x+1)>3(x-1)$, kun $x>1$,
niin
\[
\frac{1}{\sqrt{x^3-1}} < \frac{1}{\sqrt{3}}\frac{1}{\sqrt{x-1}}, \quad \text{kun}\ x>1,
\]
joten (majoranttiperiaatteen nojalla, ks.\ Luku \ref{integraalin laajennuksia}) $\Phi(s)$ on
määritelty suppenevana integraalina jokaisella $s>1$. Edelleen koska
\[
\frac{1}{\sqrt{x^3-1}} \sim \frac{1}{x^{3/2}}, \quad \text{kun}\ x \gg 1,
\]
niin $\Phi$:llä on raja-arvo, kun $s\kohti\infty$, ja myös tämä on laskettavissa suppenevana
integraalina:
\[
\lim_{s\kohti\infty} \Phi(s) =\int_1^\infty \frac{dx}{\sqrt{x^3-1}}=\Phi_\infty.
\]
Koska $\Phi'(s)=1/\sqrt{s^3-1}>0$ kun $s>1$, niin $\Phi$ on välillä $[1,\infty)$ aidosti
kasvava. Siis $\Phi:[1,\infty) \map [0,\Phi_\infty)$ on bijektio, ja näin ollen 
$y(t)=\inv{\Phi}(t)$ on myös bijektio:
\begin{multicols}{2} \raggedcolumns
\[
y:[0,T)\Kohti [1,\infty),
\]
missä
\begin{align*}
T = \Phi_\infty &= \frac{1}{\sqrt{2}}\int_1^\infty \frac{dx}{\sqrt{x^3-1}} \\[2mm]
                &\approx \underline{\underline{1.71732}}. \loppu
\end{align*}
\begin{figure}[H]
\setlength{\unitlength}{1cm}
\begin{center}
\begin{picture}(4,4)(-1,0)
\put(0,0){\vector(1,0){3}} \put(2.8,-0.4){$t$}
\put(0,0){\vector(0,1){3}} \put(0.2,2.8){$y$}
\curve(0,1,1.2,1.7,1.6,3)
\dashline{0.2}(1.72,0)(1.72,3)
\put(1,0){\line(0,-1){0.1}}
\put(0,1){\line(-1,0){0.1}}
\put(0.9,-0.5){$1$} \put(-0.4,0.9){$1$} \put(1.7,-0.5){$T$}
\end{picture}
%\caption{$y=\inv{\Phi(\sqrt{2}x)}$}
\end{center}
\end{figure}
\end{multicols}

\subsection{Sovellusesimerkki: Ketjuviiva}
\index{differentiaaliyhtälö!o@ketjuviivan|vahv}
\index{ketjuviiva|vahv}
\index{zza@\sov!Ketjuviiva|vahv}

\begin{multicols}{2} \raggedcolumns
\pain{Tehtävän kuvaus}

Ketju on asetettu riippumaan siten, että ketjun päät ovat pisteissä $(b,b)$ ja $(-b,b)$, $b>0$,
ja ketju kulkee origon kautta. Gravitaation suunta $=-\vec j$ ja ketjun paino pituusyksikköä 
kohti $=\rho$.
\input{plots/ketjuviiva.tex}
\end{multicols}
\pain {Tehtävä}\ \, Määritä ketjun muoto, eli \kor{ketjuviiva}
\[
y=y(x),\quad x\in [-b,b].
\]
\pain{Ratkaisu} 

Koska tehtävän kuvaus on fysikaalinen, tarvitaan ensin fysikaalisista periaatteista johdettu 
\pain{matemaattinen} \pain{malli}. Tämän johtamiseksi tarkastellaan pientä ketjun palaa välillä
$[x,x+\Delta x]$. Kuvaan on merkitty ketjunpalaan vaikuttavat vaakasuorat sidosvoimat $(T)$,
pystysuorat sidosvoimat $(F)$ ja gravitaatiovoima $(\Delta G)$. Koska kyseessä on ketju, ei
sidosmomentteja (taivutusmomentteja) ole. 
\begin{figure}[H]
\setlength{\unitlength}{1cm}
\begin{center}
\begin{picture}(10,10)(0,0)
\put(0,0){\line(1,0){10}}
\curve(2.6,3.4,4.6,5.1,6.6,7.4)
\curve(3,3,5,4.7,7,7)
\curvedashes[0.15cm]{0,1,2}
\curve(6.8,7.2,7.3,7.8688,7.8, 8.575)
\curve(2.8,3.2,2.3,2.8687,1.8,2.575)
\drawline(2.6,3.4)(3,3) \drawline(6.6,7.4)(7,7)
\dashline{0.1}(6.8,0)(6.8,7.2) \dashline{0.1}(6.8,7.2)(2.8,7.2) \dashline{0.1}(2.8,7.2)(2.8,0)
%\linethickness{0.5mm}
\put(2.8,3.2){\vector(-1,0){2.8}} \put(2.8,3.2){\vector(0,-1){1.8}}
\put(6.8,7.2){\vector(1,0){2.8}} \put(6.8,7.2){\vector(0,1){1.8}}
\put(4.8,4.9){\vector(0,-1){2}} \put(5,3){$\Delta G$}
\put(2.8,7.3){$\overbrace{\hspace{4cm}}^{\textstyle{\Delta x}}$}
\put(1.7,5.1){$\Delta y \begin{cases} \vspace{3.45cm} \end{cases}$}
\put(3,1.4){$F(x)$} \put(0,3.4){$T(x)$}
\put(7,8.7){$F(x+\Delta x)$} \put(7.8,6.7){$T(x+\Delta x)$}
\put(2.7,-0.5){$x$} \put(6.7,-0.5){$x+\Delta x$}
\end{picture}
%\caption{Pieni ketjun pala välillä $[x,x+\Delta x]$}
\end{center}
\end{figure}
Tasapainoyhtälöt ovat
\begin{alignat*}{2}
T(x+\Delta x)-T(x)        &= 0,                      &\quad &\text{(vaakasuunta)} \\
F(x+\Delta x)-F(x)        &= \Delta G,               &\quad &\text{(pystysuunta)} \\
F(x)\Delta x-T(x)\Delta y &= \Ord{\Delta G\Delta x}. &\quad &\text{(momenttitasapaino)} \\
\end{alignat*}
Ketjun palan pituus on likimain (vrt.\ Luku \ref{pinta-ala ja kaarenpituus})
\begin{align*}
\Delta s \cong \sqrt{(\Delta x)^2+(\Delta y)^2} 
                 &= \sqrt{1+\Bigl(\frac{\Delta y}{\Delta x}\Bigr)^2}\,\Delta x \\
                 &\approx \sqrt{1+[y'(x)]^2}\,\Delta x,
\end{align*}
joten gravitaatiovoima $\Delta G$ on likimain
\[
\Delta G=\rho g \Delta s \approx \rho g \sqrt{1+[y'(x)]^2} \Delta x.
\]
Jakamalla $(\Delta x)$:llä saadaan rajalla $\Delta x\kohti 0$ seuraava matemaattinen malli:
\[
\left\{ \begin{aligned}
T' &= 0, \\
F' &= \rho g \sqrt{1+(y')^2}, \\
F-Ty' &= 0.
\end{aligned} \right.
\]
Ensimmäisen ja kolmannen yhtälön perusteella
\[
T(x)=T_0=\text{vakio},\quad F(x)=T_0 y'(x).
\]
Sijoittamalla keskimmäiseen yhtälöön saadaan ketjuviivan differentiaaliyhtälö:
\[
ay''=\sqrt{1+(y')^2},\quad a=\frac{T_0}{\rho g}.
\]
Tästä tulee separoituva sijoituksella $u(x)=y'(x)\,$:
\begin{align*}
au'=\sqrt{1+u^2}\ &\impl\ \int\frac{du}{\sqrt{1+u^2}}=\int\frac{dx}{a} \\
                  &\impl\ \text{arsinh} \, u = \frac{x}{a}-\frac{c}{a}\quad (C=-\frac{c}{a}) \\
                  &\ekv\ u(x) = \sinh \left(\frac{x-c}{a}\right).
\end{align*}
Tästä ja ehdosta $y(0)=0$ seuraa
\[
y(x)=\int_0^x u(t)\,dt = a\cosh \left(\frac{x-c}{a}\right)-a\cosh\frac{c}{a}.
\]
Koska $y(b)=y(-b)$, niin on oltava $c=0$, eli
\[
y(x)=a\left(\cosh\frac{x}{a}-1\right).
\]
Ehto $y(b)=b$ (joka enää on käyttämättä) johtaa transkendenttiseen yhtälöön
\[
\cosh \alpha =\alpha +1,\quad \alpha=b/a.
\]
Ratkaisu (ketjuviiva) on siis
\[
y(x)=\frac{b}{\alpha}\left(\cosh\frac{\alpha x}{b}-1\right),\quad 
                                          \cosh \alpha=\alpha+1, \quad \alpha>0.
\]
Vakion $\alpha$ numeerinen arvo on
\[
\alpha=1.6161375137743..
\]

Em.\ ratkaisusta nähdään, että y\pain{leinen} \pain{ket}j\pain{uviiva} (gravitaation 
vaikuttaessa suunnassa $-\vec j\,$) on muotoa
\[
y(x) = a\cosh\left(\frac{x-c}{a}\right)+C,
\]
missä $a\in\R_+$ ja $c,C\in\R$ ovat määräämättömiä vakioita. Vakiot määräytyvät esimerkiksi
antamalla (kuten esimerkissä) kolme pistettä, joiden kautta ketju kulkee. Tunnettu voi myös
olla ripustuspisteessä vaikuttava vaakasuora tukivoima $T_0$ ja/tai ketjun pituus, joka välillä
$[x_1,x_2]$ on (vrt.\ Luku \ref{pinta-ala ja kaarenpituus})
\begin{align*}
L= \int_{x_1}^{x_2}\sqrt{1+[y'(x)]^2}\,dx 
                &= \int_{x_1}^{x_2}\sqrt{1+\sinh^2\left(\frac{x-c}{a}\right)}\,dx \\
                &= \int_{x_1}^{x_2}\cosh\left(\frac{x-c}{a}\right)\,dx 
                               = a\sijoitus{x_1}{x_2}\sinh\left(\frac{x-c}{a}\right). \loppu
\end{align*}

\Harj
\begin{enumerate}

\item
Ratkaise kvadratuureilla (yleinen ratkaisu tai alkuarvotehtävän ratkaisu). Alkuarvotehtävissä
määritä myös suurin väli, jolla ratkaisu on pätevä.
\begin{align*}
&\text{a)}\ \ y''=(y')^2 \quad
 \text{b)}\ \ 2yy''=1+(y')^2 \quad
 \text{c)}\ \ y^2y''=y' \quad
 \text{d)}\ \ y''=(y')^3e^y \\
&\text{e)}\ \ y'y'''=2(y'')^2 \quad
 \text{f)}\ \ x^2y''=(y')^2 \quad
 \text{g)}\ \ y''=2x(y')^2 \quad
 \text{h)}\ \ x^3y''=2(y')^2 \\
&\text{i)}\ \ (1+y^2)y''=y'+(y')^3,\ y(0)=1,\ y'(0)=-1 \\
&\text{j)}\ \ y''=1-(y')^2,\ y(0)=0,\ y'(0)=-1 \\
&\text{k)}\ \ y^3y''=-1,\ y(0)=1,\ y'(0)=-1 \\
&\text{l)}\ \ \sqrt{y}y''=1,\ y(0)=1,\ y'(0)=-2 \\
&\text{m)}\ \ 3y^2y'y''=-1,\ y(0)=y'(0)=1 \\
&\text{n)}\ \ 2yy''-3(y')^2=4y^2,\ y(0)=2,\ y'(0)=4 \\
&\text{o)}\ \ 2x^2y''y'''=-1,\ y(1)=0,\ y'(1)=1,\ y''(1)=2 \\
&\text{p)}\ \ 2y^{(4)}+3(y'')^2=0,\ y(0)=y'(0)=0,\ y''(0)=y'''(0)=-1
\end{align*}

\item
Määritä se yhtälön $yy''+2(y')^2=0$ ratkaisukäyrä, joka sivuaa suoraa $y=x$ pisteessä $(1,1)$.

\item
Millaisille käyrille $y=y(x)$ pätee: Käyrän kaarenpituus välillä $[0,x]$ on verrannollinen
lukuun $y'(x)\,$?

\item
Määritä käyrät $S: y=y(x)$, joiden (merkkiselle) kaarevuudelle pisteessä $P=(x,y) \in S$ pätee
laskukaava
\[
\kappa \,=\, \frac{y''}{[1+(y')^2]^{3/2}} \,=\, y^{-1}\sin\alpha\cos\alpha,
\]
missä $\alpha\in[0,\pi)$ on käyrän tangentin suuntakulma pisteessä $P$.

\item
Pitkin $x$-akselia liikkuvaan kappaleeseen, jonka massa $=m$, vaikuttaa $x$-akselin suuntainen 
voima $\vec F=f(x)\vec i$, missä $x=x(t)$ on kappaleen paikka hetkellä $t$. Johda 
liikeyhtälöstä $mx''=f(x)$ energiaperiaate
\[
\frac{1}{2}\,mv_2^2-\frac{1}{2}\,mv_1^2\,=\,\int_{x_1}^{x_2} f(x)\,dx,
\]
missä $v_i=$ kappaleen vauhti pisteessä $x_i$ ($i=1,2$).  

\item (*) %\index{zzb@\nim!Heiluri}
Matemaattisessa heilurissa pistemäinen massa heiluu painottoman sauvan (pituus $R$)
varassa, jolloin liikeyhtälö on $R\theta '' = -g \sin\theta$, missä $\theta=\theta(t)$ on
heilahduskulma mitattuna tasapainoasemasta.\vspace{1mm}\newline 
a) Kertomalla liikeyhtälö $\theta'$:lla johda energiaperiaate muotoa $F(\theta,\theta')$=vakio.
\newline
b) Käyttäen a-kohdan tulosta määrää $\theta(t),\, t\ge 0$ alkuehdoilla 
\[
\theta(0)=0,\, \theta'(0) = \frac{1}{T}\,, \quad T=\frac{1}{2}\sqrt{\frac{R}{g}}\,.
\]
(Tässä erikoistapauksessa $\theta(t)$ on alkeisfunktio!)

\item (*) \index{zzb@\nim!Jzy@Jänis ja huuhkaja}
(Jänis ja huuhkaja) Hetkellä $H$ jänis lähtee origosta juoksemaan pitkin positiivista
$y$-akselia vauhdilla $v$ (vakio). Pisteessä $(4,0)$ oleva huuhkaja huomaa samalla hetkellä
jäniksen ja lähtee lentämään matalalla siten, että lentosuunta on koko ajan jänistä kohti.
Johda huuhkajan lentokäyrälle $y=y(x)$ differentiaaliyhtälö kertalukua $2$ olettaen, että 
huuhkajan lentovauhti on a) $v$, b) $2v$. Huomioiden myös alkuehdot määritä
lentokäyrä. Missä pisteessä (jos missään) huuhkaja tavoittaa jäniksen?

\item (*) \label{H-dy-3: nopein liuku} \index{sykloidi!brakistokroni} \index{brakistokroni}
\index{zzb@\nim!Laskiainen, 2.\ lasku}%
(Laskiainen, 2.\ lasku) Lumilautailija laskee origosta pisteeseen $\,P=(a,-b)\,$ pitkin käyrää
$y=-y(x)$ ($a>0$, $b \ge 0$, gravitaation suunta $-\vec j\,$). Jos laskijan alkuvauhti $=0$
eikä kitkaa ole, niin energiaperiaatteen mukaan laskijan vauhti pisteessä $(x,-y(x))$ määräytyy
yhtälöstä $\frac{1}{2}[v(x)]^2=gy(x)$, missä $g$ on maan vetovoiman kiihtyvyys. \ a) Näytä,
että laskuaika pisteeseen $P$ on
\[
t=\int_0^a \sqrt{\frac{1+[y'(x)]^2}{2gy(x)}}\,dx.
\]
b) Tästä lausekkeesta on osoitettavissa, että lasku on nopein mahdollinen, jos $y(x)$ toteuttaa
differentiaaliyhtälön $\,2yy''+1+(y')^2=0$. Näytä, että tämä nopeimman laskun käyrä
(\kor{brakistokroni}) on sykloidin kaari.

\end{enumerate} %Palautuvat toisen kertaluvun DY:t
\section{Ensimmäisen kertaluvun lineaarinen DY} \label{lineaarinen 1. kertaluvun DY}
\sectionmark{\ 1. kertaluvun lineaarinen DY}
\alku
\index{lineaarinen differentiaaliyhtälö!c@1.\ kertaluvun|vahv}

Separoituvan (tai sellaiseksi palautuvan) differentiaaliyhtälön ohella toinen sovelluksissa
hyvin yleinen differentiaaliyhtälön tyyppi on \kor{lineaarinen} DY. Tässä luvussa tarkastelun
kohteena on 1.\ kertaluvun lineaarinen differentiaaliyhtälö, jonka yleinen muoto on
\begin{equation} \label{lin-1: ty}
y'+P(x)y=R(x).
\end{equation}
Tässä $P$ on nk.\
\index{kerroin (DY:n)} \index{oikea puoli (DY:n)}%
\kor{kerroinfunktio}. Sekä $P$ että yhtälön \kor{oikea puoli} eli $R$ oletetaan
j\pain{atkuviksi} välillä $(a,b)$, jolla ratkaisua haetaan. Kuten yleensä, ratkaisulta $y$
edellytetään derivoituvuus välillä $(a,b)$. Koska derivoituvuudesta seuraa jatkuvuus, niin
differentiaaliyhtälöstä nähdään, että edelleen myös $y'$ on jatkuva välillä $(a,b)$.

Kun otetaan käyttöön \kor{operaattori}merkintä
\[
\dyf y=\frac{dy}{dx}+P(x)y\footnote[2]{Tapauksessa $P=0$ on $\dyf=\dif$, vrt.\
Luku \ref{derivaatta}. Yleisemmin voidaan kirjoitaa $\dyf=\dif+P(x)$, jolloin kyse on
operaattorien $f\map\dif f$ ja $f \map P(x)f$ yhteenlaskusta 'funktion funktioina'.},
\]
niin yhtälö \eqref{lin-1: ty} voidaan kirjoittaa lyhyesti
\[
\dyf y=R(x).
\]
Yhtälöä sanotaan lineaariseksi sen vuoksi, että pätee
\[
\dyf(c_1y_1+c_2y_2)=c_1\dyf y_1+c_2\dyf y_2\,, \quad c_1,c_2\in\R,
\]
eli $\dyf$ on
\index{differentiaalioperaattori!b@lineaarisen DY:n}%
\kor{lineaarinen} operaattori.\footnote[3]{Jos operaattori ei ole lineaarinen, niin se on
\kor{epälineaarinen}. Epälineaarinen on vaikkapa operaattori $\dyf(y)=y'+y^2$.
\index{epzy@epälineaarinen operaattori|av}}.

Jos tunnetaan yksikin lineaarisen yhtälön \eqref{lin-1: ty} yksittäisratkaisu $y=y_0(x)$, niin
yhtälön yleisen ratkaisun etsiminen helpottuu huomattavasti. Nimittäin jos $y=y(x)$ on mikä
tahansa toinen ratkaisu, niin yhtälöistä
\[
\left\{ \begin{aligned}
&\dyf y = R(x), \\
&\dyf y_0 = R(x)
\end{aligned} \right.
\]
seuraa $L$:n lineaarisuuden nojalla
\[
0=\dyf y - \dyf y_0 = \dyf(y-y_0),
\]
eli erotus $u(x)=y(x)-y_0(x)$ toteuttaa yksinkertaisemman yhtälön
\begin{equation} \label{lin-1: hy}
y'+P(x)y=0.
\end{equation}
Tätä sanotaan (lineaariseksi 1. kertaluvun) 
\index{lineaarinen differentiaaliyhtälö!a@homogeeninen}
\index{lineaarinen differentiaaliyhtälö!b@täydellinen}%
\kor{homogeeniseksi} yhtälöksi. Jos $R \neq 0$, niin
\index{epzy@epähomogeeninen DY}%
yhtälöä \eqref{lin-1: ty} sanotaan vastaavasti \kor{epähomogeeniseksi} tai, etenkin 
ratkaisemisen yhteydessä, \kor{täydelliseksi} yhtälöksi. On päätelty:
\[
\boxed{\begin{aligned}
\quad\ykehys \text{Lineaarisen}\ 
            &\text{yhtälön \eqref{lin-1: ty} yleinen ratkaisu} \\[2mm]
            &= \ \text{täydellisen yhtälön \eqref{lin-1: ty} yksittäisratkaisu} \\
            &+ \ \text{homogeenisen yhtälön \eqref{lin-1: hy} yleinen ratkaisu}. \quad\akehys
\end{aligned}}
\]
Sikäli kuin yksittäisratkaisu $y_0(x)$ tunnetaan, on jäljellä siis homogeenisen yhtälön
\eqref{lin-1: hy} yleisen ratkaisun etsiminen. Tämä käy helposti, koska yhtälö on separoituva:
Luvun \ref{separoituva DY} menetelmin saadaan ratkaisuksi
\[
y(x)=Ce^{-\int P(x)dx}.
\]
(Tässä ei integraalifunktioon $\int P(x)\,dx$ tarvitse sisällyttää toista määräämätöntä
vakiota $C_1$, koska tämän vaikutus sisältyy jo vakioon $C$ kertoimena $e^{-C_1}$.) Yhtälö
\eqref{lin-1: ty} on näin ratkaistu yhdellä kvadratuurilla:
\[
y(x)=y_0(x)+Ce^{-\int P(x)dx}.
\]

Jos yksityisratkaisua ei tunnetta, ratkeaa yhtälö \pain{kahdella} kvadratuurilla. Nimittäin 
yksittäisratkaisu saadaan aina selville käyttäen 
\index{lineaarinen differentiaaliyhtälö!g@vakio(ide)n variointi} \index{vakio(ide)n variointi}%
\kor{vakion varioinnin} nimellä tunnettua 
(yhden kvadratuurin sisältävää) menettelyä. Vakion variointi tarkoittaa homogeenisen yhtälön
yleisessä ratkaisussa tehtävää muutosta $C \ext C(x)$ ja näin saatavan funktion
\[
y(x)=C(x)e^{-\int P(x)dx}
\]
kokeilua täydellisen yhtälön mahdollisena ratkaisuna. Sijoitus yhtälöön antaa
\[
\dyf y=C'(x)e^{-\int P(x)dx}+C(x)\dyf[e^{-\int P(x)dx}]=R(x).
\]
Tässä on $\dyf[e^{-\int P(x)dx}]=0$, joten saadaan
\begin{align*}
C'(x)e^{-\int P(x)dx} = R(x) &\qekv C'(x)  = \,e^{\int P(x)dx} R(x) \\
                             &\qekv\,C(x)\,= \int e^{\int P(x)dx}R(x)\,dx+C.
\end{align*}
Analyysin peruslauseen nojalla tämä ratkaisu on pätevä välillä $(a,b)$, jolla $P$ ja $R$ ovat
jatkuvia. Kun vakio $C$ jätetään yksittäisratkaisussa kiinnittämättä, voidaan yhtälön
\eqref{lin-1: ty} yleiseksi ratkaisuksi kirjoittaa suoraan
\[
y(x) \,=\, e^{-\int P(x)dx}C(x) \,=\, e^{-\int P(x)dx}\left(C+\int e^{\int P(x)dx}R(x)\,dx\right).
\]
Alkuehdon $y(x_0)=Y_0$ ($x_0\in(a,b)$) toteuttava ratkaisu on yksikäsitteinen ja
kirjoitettavissa suoraan määrättyjen integraalien avulla:
\begin{equation} \label{lin-1: ty-ratk}
y(x)=e^{-\int_{x_0}^x P(t)dt}\left(Y_0+\int_{x_0}^x e^{\int_{x_0}^s P(t)dt}R(s)\,ds\right).
\end{equation}
Jos tässä kiinnitetään vain $x_0$ ja annetaan $Y_0$:n olla muuttuva $(Y_0 \ext C)$, niin saadaan 
jälleen yleinen ratkaisu.
\begin{Exa} Ratkaise differentiaaliyhtälö
\[
y'+\frac{2y}{1-x^2}=\frac{1+x}{(1-x)^3}\,.
\]
\end{Exa}
\ratk Ratkaistaan ensin homogeeninen yhtälö:
\begin{align*}
\frac{dy}{dx}+\frac{2y}{1-x^2}=0 \ &\impl \ \int\frac{dy}{y}=\int\frac{2dx}{x^2-1} \\
&\impl \ \ln\abs{y}=\ln\Bigl|\frac{x-1}{x+1}\Bigr|+\ln\abs{C} \\
&\impl \ y=C\,\frac{x-1}{x+1}\,.
\end{align*}
Yleinen ratkaisu vakion varioinnilla: 
\begin{align*}
y=C(x)\,\frac{x-1}{x+1} \ &\impl \ C'(x)\,\frac{x-1}{x+1}=\frac{1+x}{(1-x)^3} \\
\impl\ C'(x) &= -\frac{(x+1)^2}{(x-1)^4}=-\frac{[(x-1)+2]^2}{(x-1)^4} \\
             &= -\frac{1}{(x-1)^2}-\frac{4}{(x-1)^3}-\frac{4}{(x-1)^4} \\
\impl\ C(x)  &= \frac{1}{x-1}+\frac{2}{(x-1)^2}+\frac{4}{3(x-1)^3}+C \\
             &= \frac{3x^2+1}{3(x-1)^3}+C \\
\impl\ y(x)=C(x)\,\frac{x-1}{x+1}\,
             &=\, \frac{3x^2+1}{3(x-1)^2(x+1)}+C\,\frac{x-1}{x+1}\,.
\end{align*}
Ratkaisu on pätevä väleillä $(-\infty,-1)$, $(-1,1)$ ja $(1,\infty)$. \loppu

\subsection{Integroivan tekijän menettely}
\index{lineaarinen differentiaaliyhtälö!h@integroivan tekijän menettely|vahv}

Vakion variointia suoraviivaisempi tapa ratkaista yhtälö \eqref{lin-1: ty} on kertoa yhtälö
ensin nk.\ 
 \index{integroiva tekijä (DY:n)}%
\kor{integroivalla tekijällä}, joka määritellään
\[
H(x)= e^{\int P(x)dx}.
\]
Koska $H'=P(x)H$, niin
\[
H(x)[y'+P(x)y\,] = \frac{d}{dx}[H(x)y\,].
\]
Näin ollen, ja koska $H(x)>0\ \forall x$, niin saadaan ratkaisukaavio
\begin{align*}
y'+P(x)y = R(x) &\qekv H(x)[y'+P(x)y] = H(x)R(x) \\[3.5mm]
                &\qekv [H(x)y\,]' = H(x)R(x) \\[1.5mm]
                &\qekv H(x)y(x) = \int H(x)R(x)\,dx \\
                &\qekv y(x) = H(x)^{-1}\int H(x)R(x)\,dx, \quad x\in(a,b).
\end{align*}
Päättely nojaa jälleen Analyysin peruslauseeseen ja tehtyihin jatkuvuusoletuksiin. Lopputulos
on sama kuin edellä, ja alkuarvotehtävän ratkaisu saadaan jälleen vaihtamalla määräämättömät
integraalit määrätyiksi.
\begin{Exa}
Ratkaise differentiaaliyhtälö $\ y'+y\cot x=e^{\cos x}$.
\end{Exa}
\ratk Koska $\int \cot x\,dx=\ln\abs{\sin x}+C$, niin esim.\ välillä $(0,\pi)$ 
(jolla $\cot x$ on jatkuva) voidaan integroivaksi tekijäksi valita
\[
H(x)=e^{\ln\sin x}=\sin x.
\]
Em.\ ratkaisukaaviota seuraten päätellään
\begin{align*}
y'+y\cot x=e^{\cos x} &\qekv y'\sin x+y\cos x = \sin x e^{\cos x} \\[3mm]
                      &\qekv (y\sin x)'=\sin x\,e^{\cos x} \\[1mm]
                      &\qekv y\sin x = \int \sin x\,e^{\cos x}\,dx = -e^{\cos x}+C \\
                      &\qekv y(x)=\frac{C-e^{\cos x}}{\sin x}\,.
\end{align*}
Ratkaisu on pätevä väleillä $(n\pi,(n+1)\pi),\ n\in\Z$. \loppu

\subsection{Vakiokertoiminen yhtälö}
\index{lineaarinen differentiaaliyhtälö!f@vakiokertoiminen|vahv}
\index{vakiokertoiminen DY|vahv}

Sovelluksissa hyvin yleinen yhtälön \eqref{lin-1: ty} erikoistapaus on \kor{vakiokertoiminen}
1. kertaluvun lineaarinen DY, joka on muotoa
\begin{equation} \label{lin-1: ty-vak}
y'+ay=f(x)\quad (a\in\R,\ a \neq 0).
\end{equation}
(Tapaus $a=0$ sivuutetaan entuudestaan tuttuna.) Tämän ratkaisu alkuehdolla $y(x_0)=Y_0$ on 
ratkaisukaavan \eqref{lin-1: ty-ratk} mukaisesti
\begin{align*}
y(x) &= e^{-a(x-x_0)}\left(Y_0+\int_{x_0}^x e^{a(t-x_0)}f(t)\,dt\right) \\
     &= Y_0e^{-a(x-x_0)} + \int_{x_0}^x e^{a(t-x)}f(t)\,dt.
\end{align*}
Ellei haluta käyttää suoraan tätä kaavaa (yleisessä ratkaisussa $Y_0 \ext C$), niin 
yksittäisratkaisu on usein löydettävissä helposti myös 'sivistyneellä arvauksella'. 
Edellytyksenä on, että $f(x)$ on riittävän yksinkertaista muotoa.
\begin{Exa}
Etsi yhtälölle $\,y'+y=f(x)\,$ yksittäisratkaisu $y_0(x)$, kun
\[
\text{a)}\,\ f(x)=x^2, \quad 
\text{b)}\,\ f(x)=e^x, \quad 
\text{c)}\,\ f(x)=e^{-x}, \quad 
\text{d)}\,\ f(x)=\cos 2x.
\]
\end{Exa}
\ratk a) \ Kokeillaan, olisiko yksittäisratkaisu toisen asteen polynomi muotoa
$y_0(x)=Ax^2+Bx+C$. Sijoitus yhtälöön antaa
\[
Ax^2+(2A+B)x+(B+C)=x^2.
\]
Tämä toteutuu jokaisella $x$, kun
\[
\left\{ \begin{alignedat}{4}
&A & & & & & &=1 \\
2&A & \ + \ &B & & & &=0 \\
& & &B & \ + \ &C & &=0
\end{alignedat} \right. \,\ \ekv \,\
\left\{ \begin{aligned}
A &= 1 \\
B &= -2 \\
C &= 2
\end{aligned} \right.
\]
Siis yksittäisratkaisu on $y_0(x)=x^2-2x+2$.

\ratk b) \ Kokeillaan yksittäisratkaisua muotoa $y_0(x)=Ae^x\,$:
\[
2Ae^x=e^x\quad\forall x \ \ekv \ A=\frac{1}{2}.
\]
Siis $y_0(x)=\tfrac{1}{2}e^x$ toimii.

\ratk c) \ Tässä yritys $y_0(x)=Ae^{-x}$ ei onnistu, sillä tämä sattuu olemaan homogeeniyhtälön
ratkaisu. Yritetään
\[
y_0(x)=Axe^{-x}\,:\quad Ae^{-x}-Axe^{-x}+Axe^{-x}=e^{-x}\ \impl\ A=1.
\]
Siis $y_0(x)=xe^{-x}$ toimii.

\ratk d) \ Yritys: $\,y_0(x)=A\cos 2x + B\sin 2x$
\begin{align*}
&\impl \ (A+2B)\cos 2x + (-2A+B)\sin 2x=\cos 2x \\
&\impl \left\{ \begin{alignedat}{3}
&A & \ + \ 2&B & &=1 \\
-2&A & \ + \ &B & &=0
\end{alignedat} \right. \,\ \ekv \,\ \left\{ \begin{aligned}
A &= 1/5 \\
B &= 2/5
\end{aligned} \right.
\end{align*}
Siis $y_0(x)=\tfrac{1}{5}(\cos 2x+2\sin 2x)$ on yksittäisratkaisu. \loppu \newline

Yksittäisratkaisua etsittäessä on syytä huomioida myös \pain{lineaarisuussääntö}:
\begin{align*}
\dyf y_1 &= f_1(x) \ \ja \ \dyf y_2=f_2(x) \\
&\impl \ \dyf(c_1y_1+c_2y_2)=c_1f_1(x)+c_2f_2(x).
\end{align*}
\jatko \begin{Exa}
(jatko) \ \, Etsi yksittäisratkaisu, kun
\[
\text{e)}\,\ f(x)=4\sin^2 x-5x^2-2, \quad
\text{f)}\,\ f(x)=6\sinh x-10\cos 2x.
\]
\end{Exa}
\ratk Koska
\[
\text{e)}\,\ f(x)=-2\cos 2x-5x^2, \quad
\text{f)}\,\ f(x)=3e^x-3e^{-x}-10\cos 2x,
\]
niin lineaarisuussääntöä ja kohtien a)-d) tuloksia käyttäen saadaan
\begin{align*}
\text{e)}\ \ y_0(x) &= -2\cdot\frac{1}{5}(\cos 2x+2\sin 2x)-5(x^2-2x+2) \\
                    &= -\frac{2}{5}\cos 2x-\frac{4}{5}\sin 2x-5x^2+10x-10, \\[2mm]
\text{f)}\ \ y_0(x) &= 3\cdot\frac{1}{2}\,e^x-3xe^{-x}-10\cdot\frac{1}{5}(\cos 2x+2\sin 2x) \\
             &= \frac{3}{2}\,e^x-3xe^{-x}-2\cos 2x-4\sin 2x. \loppu
\end{align*}

Seuraavan taulukon säännöt ovat em.\ esimerkin yleistyksiä. Taulukossa oletetaan
differentiaaliyhtälö \eqref{lin-1: ty-vak}, missä
$a\neq 0$. Lisäksi $n\in\N\cup\{0\}$ ja $\omega\neq 0$.
\vspace{0.5cm} \newline
\begin{tabular}{|l|l|} \hline
$f(x)$ &$y_0(x)$ muotoa \\ \hline & \\
$x^n$ &$A_nx^n+\cdots + A_0$ \\ & \\
$x^ne^{bx}, \quad b\neq-a$ &$(A_nx^n+\cdots + A_0)e^{bx}$ \\ & \\
$x^ne^{-ax}$ &$Ax^{n+1}e^{-ax}$ \\ & \\
$x^n(A\cos \omega x + B\sin \omega x)$ &$(A_nx^n+\cdots + A_0)\cos \omega x 
                                                   + (B_nx^n+\cdots + B_0)\sin \omega x$ \\ 
& \\ \hline 
\end{tabular}

\subsection{Jaksolliset ratkaisut}
\index{lineaarinen differentiaaliyhtälö!i@jaksolliset ratkaisut|vahv}
\index{jaksollinen ratkaisu (DY:n)|vahv}

Sovelluksissa yleinen vakikertoimisen yhtälön \eqref{lin-1: ty-vak} erikoistapaus on sellainen,
jossa yhtälön oikea puoli $f$ on j\pain{aksollinen}. Tällöin yhtälölle on löydettävissä 
yksittäisratkaisu $y_p$, joka on samoin jaksollinen, ja $y_p$:n jakso = $f$:n jakso.
(Tapauksessa $a=0$ ei jaksollista ratkaisua yleisesti ole, ks.\ 
Harj.teht.\,\ref{H-dy-4: jaksolliset ratkaisut}.) Olkoon $f$ koko $\R$:ssä  määritelty, jatkuva
ja $L$-jaksoinen funktio. Etsitään $L$-jaksoinen ratkaisu määräämällä tämä yhden jakson
pituisella välillä, esim.\ välillä $[0,L]$. Tällä välillä riittää asettaa jaksollisuusehto
$y(0)=y(L)$, joten on siis löydettävä funktio $y_p$, joka on välillä $[0,L]$ jatkuva, välillä
$(0,L)$ derivoituva ja toteuttaa
\[
\begin{cases} \,y'+ay=f(x), \quad x\in(0,L), \\ \,y(0)=y(L). \end{cases}
\]
Ratkaisu on muotoa $y(x)=Ce^{-ax}+y_0(x)$, missä $y_0$ on jokin differentiaaliyhtälön
yksittäisratkaisu, esim. (vrt.\ ratkaisukaava edellä)
\[
y_0(x)=\int_0^x e^{a(t-x)}f(t)\,dt.
\]
Jaksollisuusehto määrää vakion $C$, jolloin ratkaisuksi saadaan (olettaen $a \neq 0$)
\[
y_p(x)=\frac{y_0(L)-y_0(0)}{1-e^{-aL}}\,e^{-ax}+y_0(x), \quad x\in[0,L].
\]
Kun jatketaan $y_p$ koko $\R$:ään jaksollisuusehdolla $y_p(x \pm L)=y_p(x)$, niin on 
pääteltävissä (Harj.teht.\,\ref{H-dy-4: jaksolliset ratkaisut}a), että näin määritelty $y_p$ on
differentiaaliyhtälön ratkaisu $\R$:ssä. Koska $y_p$ siis on yksittäisratkaisu, niin yleinen
ratkaisu $\R$:ssä on
\[
y(x)=Ce^{-ax}+y_p(x).
\]
Alkuehto $y(x_0)=Y_0$ määrää $C$:n arvoksi $C=(Y_0-y_p(x_0))e^{ax_0}$. 

Sovelluksissa on yleensä $a>0$ ja muuttujan $x$ tilalla \pain{aika} ($t$), jolloin ym.\ 
ratkaisussa on siis kaksi osaa, jaksollinen osa $y_p(t)$ ja nk.\ \pain{transientti}, joka 
'kuolee pois', kun $t\kohti\infty$. Seuraavassa esimerkki sähkötekniikasta.

\index{lineaarinen differentiaaliyhtälö!i@jaksolliset ratkaisut|vahv}%
\index{jaksollinen ratkaisu (DY:n)|vahv}%
\index{zza@\sov!Szyhkzza@Sähköpiiri: RC|vahv}
\subsection{Sovellusesimerkki: Sähköpiiri RC}
\begin{multicols}{2} %\raggedcolumns
\pain{Tehtävän kuvaus} \vspace{0.2cm} \newline 
Oheisessa sähköpiirissä on sarjaan kytketty vastus ($R$) ja kondensaattori, jonka kapasitanssi
$=C$. Virta piirissä hetkellä $t$ on $i(t)$, kondensaatorin varaus $=y(t)$ ja jännite 
kondensaattorin yli $=u(t)$. 
\begin{figure}[H]
\setlength{\unitlength}{1cm}
\begin{center}
\begin{picture}(7.5,4)(-0.5,-0.5)
\multiput(0,0)(6,0){2}{
\put(-0.5,0){\line(1,0){1}}
\multiput(-0.25,0)(0.25,0){3}{\line(-1,-1){0.25}}
}
\path(0,0)(0,1)
\put(0,1.5){\circle{1}} 
\curve(-0.2,1.45,-0.1,1.55,0,1.5)\curve(0,1.5,0.1,1.45,0.2,1.55)
\path(0,2)(0,3)(1,3)(2,3.5)
\path(2,3)(3,3)(3,3.25)(5,3.25)(5,2.75)(3,2.75)(3,3)
\path(5,3)(6,3)(6,1.6)(6.5,1.6)(5.5,1.6)
\path(6.5,1.4)(5.5,1.4)
\path(6,1.4)(6,0)
\put(0,3.2){$e(t)$} \put(3.8,3.4){$R$} \put(6.1,1.7){$C$} \put(4.7,1.4){$y(t)$} 
\put(5.5,3.2){$i(t)$}
\put(5,3){\vector(1,0){1}}
\curve(6.4,2.7,6.7,1.5,6.4,0.3)
\put(6.4,0.3){\vector(-1,-2){0.01}}
\put(6.9,1.4){$u(t)$}
\end{picture}
\end{center}
\end{figure}
\end{multicols}

\begin{multicols}{2} \raggedcolumns
Piiriä syötetään kokoaaltotasasuunnatulla jännitteellä
\[
e(t)=E\abs{\sin \omega t}.
\]
\begin{figure}[H]
\setlength{\unitlength}{1cm}
\begin{center}
\begin{picture}(5,2)(-0.5,-0.5)
\put(0,0){\vector(1,0){4.5}} \put(4.3,-0.5){$t$}
\put(0,0){\vector(0,1){1.5}} \put(0.2,1.3){$e(t)$}
\multiput(0,0)(1,0){4}{
\setlength{\unitlength}{0.3cm}
\renewcommand{\yscale}{2}
\curve(
  0,       0,
0.2, 0.19867,
0.4, 0.38942,
0.6, 0.56464,
0.8, 0.71736,
  1, 0.84147,
1.2, 0.93204,
1.4, 0.98545,
1.6, 0.99957,
1.8, 0.97385,
  2,  0.9093,
2.2,  0.8085,
2.4, 0.67546,
2.6,  0.5155,
2.8, 0.33499,
  3.14, 0)}
\setlength{\unitlength}{1cm}
\dashline{0.2}(0,0.6)(4,0.6)
\put(-0.4,0.5){$E$}
\end{picture}
\end{center}
\end{figure}
\end{multicols}
\pain{Matemaattinen malli} \ (Ol. kytkin suljettu)
\begin{align*}
Ri(t)+u(t) &= e(t), \\
y(t) &= Cu(t), \\
y'(t) &= i(t).
\end{align*}
\pain{Alkuoletukset} \vspace{0.2cm} \newline
Ajanhetkillä $t\leq t_0$ on $i(t)=q(t)=0$. Kytkin suljetaan hetkellä $t=t_0$, ja pidetään sen
jälkeen suljettuna. \vspace{0.2cm} \newline
\pain{Tehtävä} \vspace{0.2cm} \newline
Määritä kondensaattorin varaus $y(t)$, kun $t \ge t_0\,$. \vspace{0.2cm} \newline
\pain{Ratkaisu} \vspace{0.2cm} \newline
Merkitään $Q=EC$ ja $\tau=RC$ ($\tau=$ aikavakio). Eliminoimalla $i(t)$ ja $u(t)$ saadaan
ratkaistavaksi alkuarvotehtävä
\[
\begin{cases}
\,\tau y'+y = Q\abs{\sin \omega t},\quad t>t_0, \\ \,y(t_0) =0.
\end{cases}
\]
Koska differentiaaliyhtälön oikea puoli on jaksollinen, jaksona
\[
T=\pi/\omega,
\]
on yhtälölle löydettävissä yksittäisratkaisu $y_p$, joka on samoin $T$-jaksoinen. Tämä on
välillä $[0,T]$ jatkuva, välillä $(0,T)$ derivoituva ja toteuttaa
\[
\begin{cases}
\,\tau y_p'+y_p = Q\sin \omega t,\quad t\in (0,T), \\ \,y_p(0) =y_p(T).
\end{cases}
\]
Etsitään ensin differentiaaliyhtälön yksittäisratkaisu $y_0$ välillä $(0,T)\,$:
\begin{align*}
&y_0(t) =A\sin \omega t + B\cos \omega t \\
&\impl \ (A-\tau\omega B)\sin \omega t + (\tau\omega A+B)\cos\omega t 
                                                    = Q\sin \omega t,\quad 0<t<T \\
&\ekv \ \left\{ \begin{alignedat}{3}
&A \ - \ & \tau\omega&B & &=Q \\
\tau\omega&A \ + \ & &B & &=0
\end{alignedat} \right. \\ 
&\ekv \ A=\frac{Q}{k^2+1},\quad B=-\frac{kQ}{k^2+1},\qquad k=\tau\omega.
\end{align*}
(Tässä uusi parametri $k$ on dimensioton). Etsityn $T$-jaksoisen ratkaisun on siis oltava 
välillä $[0,T]$ muotoa
\[
y_p(t)=Ce^{-t/\tau}+\frac{Q}{k^2+1}(\sin \omega t-k\cos \omega t).
\]

Vakio $C$ määräytyy jaksollisuusehdosta:
\begin{align*}
y_p(0)=y_p(T) \ &\ekv \ C-\frac{kQ}{k^2+1} = Ce^{-T/\tau}+\frac{kQ}{k^2+1} \\
&\ekv \ C=\frac{2kQ}{(k^2+1)(1-e^{-T/\tau})}\,.
\end{align*}
Siis välillä $[0,T]$ pätee
\[
y_p(t) = \frac{Q}{k^2+1}
         \left(\frac{2k}{1-e^{-T/\tau}}\,e^{-t/\tau}+\sin \omega t-k\cos \omega t\right),
\]
missä
\[
Q=EC, \quad \tau=RC, \quad T=\frac{\pi}{\omega}\,,\quad k=\tau\omega=\pi\frac{\tau}{T}\,.
\]
Alkuehdon $y(t_0)=0$ toteuttava ratkaisu on
\[
y(t)=-y_p(t_0)e^{-(t-t_0)/\tau}+y_p(t),\quad t\geq t_0.
\]
Jos sattuu olemaan $y_p(t_0)=0$, ei transientti 'herää'. \loppu

\Harj
\begin{enumerate}

\item 
Ratkaise (yleinen ratkaisu tai alkuarvotehtävän ratkaisu):
\begin{align*}
&\text{a)}\ \ y'+2xy=2xe^{-x^2} \qquad
 \text{b)}\ \ (1+x^2)y'-2xy=(1+x^2)^2 \\
&\text{c)}\ \ \cos x y'-\sin x y=xe^x \qquad
 \text{d)}\ \ y'-y=\cosh x \\
&\text{e)}\ \ xy'+2y=x^3,\,\ y(1)=1 \qquad
 \text{f)}\ \ y'+y\cos x=\sin x\cos x,\,\ y(0)=1 \\
&\text{g)}\ \ y'+y\tan x=\sin^3 x,\,\ y(0)=1 \qquad
 \text{h)}\ \ y'+\abs{x-\abs{x}}y=x,\,\ y(0)=0 \\
&\text{i)}\ \ y'+2y=x^3-x,\,\ y(0)=1 \qquad
 \text{j)}\ \ y'-y=e^x-\sin x,\,\ y(0)=0
\end{align*}

\item 
a) Differentiaaliyhtälöllä $y'+P(x)y=(x+1)^2e^x$ on ratkaisu $y=(x^2-1)e^x$. Määritä yleinen
ratkaisu. \newline
b) Olkoon funktiot $f$ ja $g$ derivoituvia ja $f'$ ja $g'$ jatkuvia välillä $(a,b)$,
ja olkoon $f(x) \neq 0\ \forall x\in(a,b)$. Minkä lineaarisen differentiaaliyhtälön yleinen
ratkaisu välillä $(a,b)$ on $y(x)=Cf(x)+g(x),\ C\in\R\,$?

\item
Määritä $\R$:sssä jatkuva funktio $y(x)$, joka toteuttaa yhtälön
\[
2\int_0^x ty(t)\,dt=x^2+y(x), \quad x\in\R.
\]

\item \label{H-dy-4: Bernoullin DY}
\index{differentiaaliyhtälö!q@Bernoullin} \index{Bernoullin differentiaaliyhtälö}
Näytä, että \kor{Bernoullin} DY $\,y'=A(x)y+B(x)y^k\ (k\in\R,\ k \neq 0,\ k \neq 1)$ palautuu 
lineaariseksi sijoituksella $u=y^{1-k}$. Ratkaise tällä periaatteella
\begin{align*}
&\text{a)}\ \ xy'+y=x^3y^2 \qquad\qquad\,\
 \text{b)}\ \ y'+y=y^2(\cos x-\sin x) \\
&\text{c)}\ \ 3y'+y=(1-2x)y^4 \qquad
 \text{d)}\ \ y'+2y/(1-x)=4(x^2-x)\sqrt{y}
\end{align*}

\item
Käyrän $y=y(x)$ pisteeseen $P$ asetetaan tangentti, joka leikkaa $y$-akselin pisteessä $Q$.
Määritä kaikki käyrät, joilla on ominaisuus: kolmion $OPQ$ ($O=$ origo) pinta-ala $=a^2=$ vakio.

\item
Funktio $R(x)$ on jatkuva $\R$:ssä ja $-2x^2 \le R(x) \le x^2\ \forall x\in\R$. Funktio $y(x)$
on alkuarvotehtävän $y'+3x^2y=R(x),\ y(0)=0$ ratkaisu. Mitä arvoja $y(-1)$ voi saada?

\item \label{H-dy-4: jaksolliset ratkaisut}
Olkoon $f$ ja $y_p$ koko $\R$:ssä määriteltyjä, jatkuvia ja $L$-jaksoisia funktioita, ja
lisäksi olkoon $y_p$ differentiaaliyhtälön $\,y'+ay=f(x)\ (a\in\R)$ ratkaisu
välillä $(0,L)$. \vspace{1mm}\newline
a) Näytä, että $y_p'+ay_p=f(x)\ \forall x\in\R$. \newline
b) Olkoon $F$ ja $Y$ $f$:n ja $y_p$:n keskiarvot välillä $[0,L]$. Näytä, että $F=aY$. 
--- Johtopäätös, jos $a=0$\,? 

\item
Olkoon
\[
f(x) = \begin{cases} 
       \,0, &\text{kun}\ x=k\in\Z, \\ \,2x-2k-1, &\text{kun}\ x\in(k,k+1),\ k\in\Z.
       \end{cases}
\]
Määritä funktio $y_p$, joka on (1) jaksollinen, (2) koko $\R$:ssä määritelty ja jatkuva ja 
(3) on differentiaaliyhtälön $\,y'+y=f(x)\,$ ratkaisu väleillä $(k,k+1)$, $k\in\Z$. Hahmottele
$y_p$ graafisesti välillä $[0,2]$. Toteutuuko differentiaaliyhtälö pisteessä $x=1$\,? 

\item (*)
a) Näytä, että alkuarvotehtävän $y'=x+\abs{y},\ y(-2)=1$ ratkaisulla on minimi
$y(\ln 2-1)=\ln 2-1$. \newline
b) Määritä differentiaaliyhtälön $y'=\abs{y-x}$ yleinen ratkaisu. Laske myös $y_1(1)$ ja
$y_2(1)$ yksittäisratkaisuille, jotka toteuttavat ehdot $y_1(-1)=-1/2$ ja $y_2(-1)=1/2$.

\item (*) \label{H-dy-4: Riccatin DY}
\index{differentiaaliyhtälö!q@Riccatin} \index{Riccatin differentiaaliyhtälö}
\kor{Riccatin} DY on muotoa $\,y'=A(x)+B(x)y+C(x)y^2$. Jos tälle tunnetaan yksittäisratkaisu
$y_0$, niin yhtälö palautuu Bernoullin DY:ksi (ks. Tehtävä \ref{H-dy-4: Bernoullin DY}) 
sijoituksella $y=y_0+u$. Ratkaise tällä periaatteella seuraavat differentiaaliyhtälöt annettua
lisätietoa käyttäen. \vspace{1mm}\newline
a) \ $y'=1+x+x^2-(2x+1)y+y^2, \quad y_0(x)=$ polynomi astetta $1$ \newline
b) \ $y'=y^2-x^2y-(x-1)^2, \quad y_0(x)=$ polynomi astetta $2$ \newline
c) \ $x^2y'+(xy-2)^2=0, \quad y_0(x)=a/x\,\ (a\in\R)$

\item (*)
Funktio $y(x)$ on derivoituva välillä $(0,\infty)$, oikealta jatkuva pisteessä $x=0$ ja
$y(0^+)=1$. Lisäksi tiedetään, että $y'(x)+y(x) \ge f(x)$ kun $x>0$, missä $f$ on  välillä
$[0,\infty)$ jatkuva funktio. Näytä, että jokaisella $x>0$ pätee
\[
y(x) \ge e^{-x} + \int_0^x e^{t-x} f(t)\,dt.
\]

\item (*)
Olkoon $\tau>0$. Ratkaise differentiaaliyhtälö
\[
\tau y'+y=\abs{2k-t}, \quad t\in(0,\infty)\cap[2k-1,2k+1], \quad k\in\N\cup\{0\} 
\]
muodossa $y(t)=Ce^{-t/\tau}+y_p(t)$, missä $y_p$ on jaksollinen, jaksona $T=2$. Hahmottele
graafisesti alkuehdon $y(0)=2$ toteuttava ratkaisu $\tau$:n arvoilla $10$, $1$ ja $0.1$.

\end{enumerate} %Ensimmäisen kertaluvun lineaarinen DY
\section{Lineaariset, vakiokertoimiset DY:t.  \\ Eulerin differentiaaliyhtälö} 
\label{vakikertoimiset ja Eulerin DYt}
\sectionmark{Vakiokertoimiset ja Eulerin DY:t}
\alku
\index{lineaarinen differentiaaliyhtälö!d@2.\ kertaluvun|vahv}
\index{lineaarinen differentiaaliyhtälö!f@vakiokertoiminen|vahv}
\index{vakiokertoiminen DY|vahv}

Edellisessä luvussa ratkaistiin jo ensimmäisen kertaluvun lineaarinen ja vakiokertoiminen
differentiaaliyhtälö $y'+ay=f(x)$. Sovelluksissa yleinen on myös vastaava toisen kertaluvun
differentiaaliyhtälö, jonka yleinen muoto on
\begin{equation} \label{linvak-ty}
y''+ay'+by=f(x), \quad a,b\in\R.
\end{equation}
Tämän ratkaisemiseksi tarkastellaan ensin vastaavaa homogeenista yhtälöä
\begin{equation} \label{linvak-hy}
y''+ay'+by=0.
\end{equation}
Kun tämän ratkaisua etsitään yritteellä
\[
y(x)=e^{rx},
\]
niin sijoittamalla yhtälöön saadaan vakion $r$ määrämiseksi
\index{karakteristinen yhtälö (polynomi)!a@differentiaaliyhtälön}%
\kor{karakteristinen yhtälö}
\[
\boxed{\kehys\quad r^2+ar+b=0 \quad\text{(karakteristinen yhtälö).}\quad}
\]
Karakteristisen yhtälön juuret ovat
\[
r=-\frac{a}{2}\pm\sqrt{\frac{a^2}{4}-b}\,,
\]
jolloin on kolme mahdollisuutta:

\underline{1. \ $a^2-4b>0$}. \ Tässä tapauksessa karakteristisella yhtälöllä on kaksi erisuurta 
reaalijuurta $r_1$ ja $r_2$ ja homogeeniyhtälöllä \eqref{linvak-hy} siis ratkaisut
\[
y_1(x)=e^{r_1x},\quad y_2(x)=e^{r_2x}.
\]
Ratkaisu on myös $y(x)=C_1y_1(x)+C_2y_2(x)\ (C_1,C_2\in\R)$, sillä jos yhtälö \eqref{linvak-hy}
kirjoitetaan $\dyf y=0$, niin $\dyf$ on lineaarinen operaattori, jolloin on
$\dyf(C_1y_1+C_2y_2)=C_1\dyf y_1+C_2\dyf y_2=0$. Siis jokainen $y(x)$ muotoa
\[
y(x)=C_1e^{r_1x}+C_2e^{r_2x}, \quad C_1,C_2\in\R
\]
on homogeeniyhtälön \eqref{linvak-hy} ratkaisu. Näytetään nyt, että kyseessä on yhtälön
\eqref{linvak-hy} yleinen ratkaisu eli mikä tahansa ratkaisu on tätä muotoa. Tätä silmällä
pitäen kirjoitetaan ensin karakteristinen yhtälö muotoon
\[
r^2+ar+b \,=\, (r-r_1)(r-r_2) \,=\, r^2-(r_1+r_2)r+r_1r_2 \,=\, 0,
\]
missä siis $a=-r_1-r_2$ ja $b=r_1r_2$. Olkoon $y(x)$ homogeeniyhtälön \eqref{linvak-hy}
ratkaisu. Tällöin derivoimalla funktio $u(x)=y'(x)-r_2y(x)$ todetaan, että
\[
u'-r_1u = y''-(r_1+r_2)y'+r_1r_2y = y''+ay'+by=0.
\]
Siis $u(x)$ ja $y(x)$ ovat ratkaistavissa differentiaaliyhtälöryhmästä
\[
\begin{cases}
\,u'-r_1u=0, \\ \,y'-r_2y=u(x).
\end{cases}
\]
Kun tässä ensimmäisen yhtälön yleinen ratkaisu $u(x)=C_1e^{r_1x}\ (C_1\in\R)$ sijoitetaan
jälkimmäiseen yhtälöön, niin todetaan tämän yksittäisratkaisuksi
$y_0(x)=(r_1-r_2)^{-1}C_1e^{r_1x}$ (vrt.\ edellinen luku) ja yleiseksi ratkaisuksi siis
\[
y(x)=\frac{C_1}{r_1-r_2}\,e^{r_1x}+C_2e^{r_2x}.
\]
Tässä voidaan $C_1$:n tilalle kirjoittaa yhtä hyvin $(r_1-r_2)C_1$, joten todetaan, että
jokainen yhtälön \eqref{linvak-hy} ratkaisu on muotoa $y(x)=C_1e^{r_1x}+C_2e^{r_2x}$,
$C_1,C_2\in\R$. Jokainen tällainen funktio oli myös ratkaisu, joten kyseessä on yleinen
ratkaisu.

\underline{2. \ $a^2-4b=0$}. \ Tässä tapauksessa karakteristisella yhtälöllä on kaksoisjuuri 
$r=-a/2$, joten em.\ yhtälöryhmässä on $r_1=r_2=r$. Kun ensimmäisen yhtälön yleinen ratkaisu
$u(x)=C_1e^{rx}\ (C_1\in\R)$ sijoitetaan jälkimmäiseen, niin tämän yksittäisratkaisu on
$y_0(x)=C_1xe^{rx}$ (vrt.\ edellinen luku) ja yleinen ratkaisu siis
\[
y(x)=(C_1x+C_2)e^{rx},\quad r=-\frac{a}{2}\,.
\]
Tämä on myös yhtälön \eqref{linvak-hy} yleinen ratkaisu.

\underline{3. \ $a^2-4b<0$}. \ Tässä tapauksessa karakteristisen yhtälön juuret muodostavat
konjugaattiparin
\[
r_{1,2}=\alpha\pm i\beta,\quad \alpha=-\frac{a}{2}, \ \beta=\sqrt{b-\frac{a^2}{4}}\,.
\]
Sijoituksella $y(x)=e^{\alpha x}u(x)$ yhtälö \eqref{linvak-hy} muuntuu muotoon
\[
u''+\beta^2 u = 0.
\]
Tämä on ratkaistavissa 1.\ kertalukuun palutuvana DY:nä (ks.\ Luku \ref{toisen kertaluvun dy}),
mutta suoremminkin voi päätellä, että ratkaisuja ovat
$u_1(x)=\cos\beta x$ ja $u_2(x)=\sin\beta x$ ja yleinen ratkaisu siis ilmeisesti
$u(x)=C_1\cos\beta x+C_2\sin\beta x,\ C_1,C_2\in\R$. Näin saadaan yhtälön \eqref{linvak-hy}
yleiseksi ratkaisuksi
\[
\kehys y(x)=e^{\alpha x}(C_1\cos\beta x + C_2\sin\beta x).
\]
\begin{Exa}
Ratkaise reuna-arvotehtävä
\[
\begin{cases}
y''-y'+2y=0,\quad x\in (0,1), \\
y(0)=1, \ y(1)=0.
\end{cases}
\]
\end{Exa}
\ratk Karakteristisen yhtälön $r^2-r-2$ juuret ovat $r_1=2$, $r_2=-1$, joten 
differentiaaliyhtälön yleinen ratkaisu on
\[
y(x)=C_1e^{2x}+C_2e^{-x}.
\]
Reunaehdot toteutuvat, kun
\[
\left\{ \begin{alignedat}{3}
&C_1 \ + \ & &C_2 & &=1 \\
e^2&C_1 \ + \ & e^{-1}&C_2 & &=0
\end{alignedat} \right. \,\ \ekv \,\ 
\begin{cases}
\,C_1=-1/(e^3-1), \\
\,C_2=\,e^3/(e^3-1).
\end{cases}
\]
Siis reuna-arvotehtävän ratkaisu on
\[
y(x)=\frac{1}{e^3-1}(-e^{2x}+e^{3-x}). \loppu
\] 

\subsection{Täydellinen vakiokertoiminen yhtälö}
\index{lineaarinen differentiaaliyhtälö!b@täydellinen|vahv}

Koska differentiaaliyhtälö \eqref{linvak-ty} on lineaarinen, niin sen ratkaisulle pätee sama
yleisperiaate kuin ensimmäisen kertaluvun lineaariselle DY:lle: Täydellisen yhtälön yleinen
ratkaisu = homogeenisen yhtälön yleinen ratkaisu + täydellisen yhtälön yksittäisratkaisu
$y_0(x)$. Yksittäisratkaisun määräämiseksi tarkastellaan tässä yhteydessä vain 'sivistyneen
arvauksen' menetelmiä, jotka toimivat silloin, kun $f(x)$ yhtälössä \eqref{linvak-ty} on
riittävän yksinkertaista muotoa. (Yleisempi menetelmä esitetään seuraavassa luvussa; ks.\ myös
Harj.teht.\,\ref{H-dy-5: linvak-ty}.)
\begin{Exa}
Ratkaise $y''+y=x^2+2e^{2x}$.
\end{Exa}
\ratk Yritetään yksittäisratkaisua muodossa
\[
y(x)=Ax^2+Bx+C+De^{2x}.
\]
Sijoittamalla yhtälöön todetaan tämä ratkaisuksi kun $A=1$, $B=0$, $C=-2$ ja $D=2/5$.
Homogeenisen yhtälön yleinen ratkaisu on
\[
y(x)=C_1\cos x+C_2\sin x,
\]
joten täydellisen yhtälön yleinen ratkaisu on
\[
y(x)=C_1\cos x+C_2\sin x+x^2-2+\frac{2}{5}e^{2x}. \loppu
\]

Jos yleisemmin yhtälön \eqref{linvak-ty} oikea puoli $R(x)$ on muotoa
\[
\text{a)} \ \, R(x)=x^ne^{\alpha x}, \quad \text{tai} \quad 
\text{b)} \ \, R(x)=x^n(A\cos\omega x + B\sin\omega x),
\]
missä $n\in \{0,1,2,\ldots\}$, $\alpha\in\R$, ja $\omega\in\R$, $\omega\neq 0$, niin yhtälön
\eqref{linvak-ty} yksittäisratkaisu on löydettävissä vastaavasti muodossa
\begin{itemize}
\item[a)] $y(x)=p(x)e^{\alpha x}$,
\item[b)] $y(x)=p(x)\cos\omega x+q(x)\sin\omega x$,
\end{itemize}
missä $p$ ja $q$ ovat polynomeja. Pääsääntöisesti $p$ ja $q$ ovat astetta $n$. Poikkeuksen
muodostavat ne tapaukset, joissa pääsäännön mukainen yrite sattuu olemaan homogeenisen yhtälön
ratkaisu. Tapauksessa a) tämä on mahdollista kun $n=0$ tai $n=1$, tapauksessa b) kun $n=0$. 
Tällöin polynomin astetta on nostettava yhdellä, tapauksessa a) mahdollisesti kahdella, jotta
yksittäisratkaisu löytyisi.
\begin{Exa}
Ratkaise alkuarvotehtävä
\[
\begin{cases} \,y''+2y'+y=e^{-x},\,\ x\in\R, \\ \,y(1)=y'(1)=0. \end{cases}
\]
\end{Exa}
\ratk Karaktristisella yhtälöllä on kaksoisjuuri $r=-1$, joten homogeenisen yhtälön yleinen
ratkaisu on $y(x)=(C_1+C_2x)e^{-x}$. Koska sekä $f(x)=e^{-x}$ että $xe^{-x}$ ovat homogeenisen
yhtälön ratkaisuja, niin yksittäisratkaisua on etsittävä muodossa
\[
y(x)=(Ax^2+Bx+C)\,e^{-x}.
\]
Tämä osoittautuu ratkaisuksi, kun valitaan $A=1/2$ ja $B,C\in\R$, eli saatiin suoraan yleinen
ratkaisu
\[
y(x)=\Bigl(\frac{1}{2}x^2+Bx+C\Bigr)e^{-x}, \quad B,C\in\R.
\]
(Samaan tulokseen olisi tultu, jos yritteessä olisi valittu 'viisaammin' $B=C=0$ ja lisätty 
homogeenisen yhtälön yleinen ratkaisu vasta jälkikäteen.) Alkuehdot toteutuvat, kun
$A=-1$ ja $B=1/2$, joten alkuarvotehtävän ratkaisu on
\[
y(x) = \frac{1}{2}(x-1)^2e^{-x}.  \loppu
\]

\subsection{Kompleksiarvoiset ratkaisut}
\index{lineaarinen differentiaaliyhtälö!j@kompleksiarvoiset ratkaisut|vahv}
\index{kompleksiarvoinen ratkaisu (DY:n)|vahv}

Vakiokertoimisia differentiaaliyhtälöitä ratkaistaessa on usein kätevää suorittaa laskut 
kompleksiarvoisia funktioita käyttäen silloinkin, kun pyritään reaaliseen lopputulokseen.
Menetelmä on kätevä erityisesti silloin, kun karakteristisen yhtälön juuret ovat 
kompleksilukuja, tai kun täydellisen yhtälön \eqref{linvak-ty} oikealla puolella esiintyy
trigonometrisia funktioita. Kompleksifunktioilla laskettaessa hyväksytään homogeenisen yhtälön
ratkaisuyritteessä
\[
y(x)=e^{rx}
\]
suoraan myös kompleksiset $r$:n arvot. Kyseessä on tällöin reaalimuuttujan kompleksiarvoinen
funktio, jonka derivaatta määritellään normaaliin tapaan eli erotusosamäärän raja-arvona. Kun
derivaatan määritelmässä merkitään po.\ funktion tapauksessa
\[
z=rx,\quad \Delta z=r\Delta x,
\]
niin nähdään, että
\[
\frac{y(x+\Delta x)-y(x)}{\Delta x}=r\,\frac{e^{z+\Delta z}-e^z}{\Delta z}\,.
\]
Tässä $\Delta x\kohti 0 \ \impl \ \Delta z\kohti 0$, joten kompleksifunktion $e^z$ 
derivoimissäännön perusteella (ks.\ Luku \ref{kompleksinen eksponenttifunktio}) voidaan todeta,
että pätee odotetusti
\[
\boxed{\quad \frac{d}{dx}\,e^{rx}=re^{rx},\quad r\in\C. \quad}
\]
\begin{Exa} Ratkaise differentiaaliyhtälö \eqref{linvak-hy} kompleksifunktioiden avulla, kun
$a^2-4b<0$.
\end{Exa}
\ratk Karakteristisen yhtälön juuret ovat $r_{1,2}=\alpha \pm i\beta$, joten yleinen ratkaisu
saadaan aiempaan tapaan ratkaisemalla yhtälöryhmä
\[
\begin{cases}
\,u'-r_1u=0, \\ \,y'-r_2y=u(x).
\end{cases}
\]
Kuten aiemmin (vrt.\ tapaus $r_1,r_2\in\R$ edellä) saadaan yleiseksi ratkaisuksi
\begin{align*}
y(x) &= C_1e^{r_1 x}+C_2e^{r_2 x} \\
     &= e^{\alpha x}(C_1e^{i\beta x}+C_2e^{-i\beta x}),\quad C_1,C_2\in\C.
\end{align*}
--- Huomattakoon, että tässä ei kertoimia $C_1,C_2$ ole syytä rajoittaa reaalisiksi. Kun
huomioidaan Eulerin kaava
\[
e^{\pm i\beta x}=\cos\beta x\pm i\sin \beta x,
\]
niin nähdään, että ratkaisu on esitettävissä yhtäpitävästi muodossa
\begin{align*}
&y(x)=e^{\alpha x}(D_1\cos\beta x+D_2\sin\beta x), \\[1mm]
&\text{missä} \quad
\begin{cases} \,D_1 = C_1+C_2, \\ \,D_2 =i(C_1-C_2). \end{cases}
\end{align*}
Tästä nähdään, että valitsemalla $C_2=\overline{C}_1$ ($= C_1$:n konjugaatti) saadaan yleinen
reaalinen ratkaisu aiemmin esitetyssä muodossa. \loppu

Esimerkin menettelyllä voidaan hakea homogeenisen yhtälön \eqref{linvak-hy} yleinen ratkaisu
myös kompleksikertoimisessa tapauksessa ($a,b\in\C$), mikäli sellainen tilanne eteen tulisi.
Tutkitaan sen sijaan sovelluksissa hyvin yleistä laskentatapaa, jossa reaalikertoimiselle
täydelliselle yhtälölle
\[
\dyf y = y''+ay'+by = \begin{cases} \,\sin\omega x \\ \,\cos\omega x \end{cases}
\]
etsitään yksittäisratkaisu käyttäen hyväksi kompleksifunktioita. Menetelmä perustuu seuraavaan
yksinkertaiseen havaintoon: Jos $f(x)=f_1(x)+if_2(x)$, missä $f_1$ ja $f_2$ ovat reaaliarvoisia,
niin $y$ on yhtälön $\dyf y=f$ yksittäisratkaisu täsmälleen kun $y=y_1+iy_2$, missä
$\dyf y_1=f_1$ ja $\dyf y_2=f_2$. (Tämä on helposti todettavissa $\dyf$:n lineaarisuuden ja
kertoimien $a,b$ reaalisuuden perusteella --- kompleksikertoimisessa tapauksessa sääntö ei 
päde.)
\begin{Exa} Etsi yksittäisratkaisu differentiaaliyhtälölle $y''-y'+y=\sin\omega x$ 
($\omega\neq 0$) käyttäen kompleksifunktioita.
\end{Exa}
\ratk Koska $\sin \omega x = \text{Im} \, (e^{i\omega x})$, niin probleema voidaan ratkaista
etsimällä kompleksiarvoinen yksittäisratkaisu differentiaaliyhtälölle
\[
y''-y'+y=e^{i\omega x},
\]
ja ottamalla ratkaisusta imaginaariosa. Kompleksinen ratkaisu löytyy helposti sijottamalla
yrite $y(x)=Ae^{i\omega x}\ (A\in\C)$ yhtälöön:
\begin{align*}
     &(-\omega^2-i\omega+1)Ae^{i\omega x}=e^{i\omega x}\quad\forall x \\
     &\qimpl A=\frac{1}{1-\omega^2-i\omega}=\frac{1-\omega^2+i\omega}{\omega^4-\omega^2+1} \\
     &\qimpl y(x) \,=\, \frac{1-\omega^2+i\omega}{\omega^4-\omega^2+1}\,e^{i\omega x}
          \,=\, \frac{1-\omega^2+i\omega}{\omega^4-\omega^2+1}\,(\cos\omega x + i\sin\omega x).
\end{align*}
Kysytty ratkaisu on tämän imaginaariosa, eli
\[
y(x)=\frac{1}{\omega^4-\omega^2+1}\,[\,\omega\cos\omega x + (1-\omega^2)\sin\omega x\,]. \loppu
\]

Esimerkissä olisi luonnollisesti tultu toimeen myös reaalisella yritteellä 
$y(x)=A\cos\omega x+B\sin\omega x\ (A,B\in\R)$, mutta laskusta olisi tullut ikävämpi.

\subsection{Eulerin differentiaaliyhtälö}
\index{lineaarinen differentiaaliyhtälö!fa@Eulerin|vahv}
\index{Eulerin!c@differentiaaliyhtälö|vahv}

Differentiaaliyhtälöä muotoa
\[
x^2y''+axy'+by=f(x)
\]
sanotaan (toisen kertaluvun) \kor{Eulerin} DY-tyypiksi. Perusmuodossa
\[
y''+ax^{-1}y'+bx^{-2}y=x^{-2}f(x)
\]
on $x=0$ kertoimien epäjatkuvuuspiste, joten Eulerin differentiaaliyhtälöä on tarkasteltava
erikseen väleillä $(-\infty,0)$ ja $(0,\infty)$ --- sovelluksissa yleensä välillä $(0,\infty)$.
Kummallakin välillä yhtälö palautuu vakiokertoimiseksi sijoituksella
\[
\abs{x}=e^t \ \ekv \ t=\ln\abs{x},
\]
sillä kun esim. välillä $(0,\infty)$ kirjoitetaan
\[
y(x)=y(e^t)=u(t)=u(\ln x),
\]
niin saadaan
\begin{align*}
y'(x)  &\,=\, \frac{1}{x}u'(\ln x)=\frac{1}{x}u'(t), \\
y''(x) &\,=\, \frac{1}{x^2} u''(\ln x)-\frac{1}{x^2}u'(\ln x)
        \,=\, \frac{1}{x^2}[u''(t)-u'(t)],
\end{align*}
joten yhtälö saadaan vakiokertoimiseen muotoon
\[
u''(t)+(a-1)u'(t)+bu(t)=f(e^t).
\]
Välillä $(-\infty,0)$ tulee oikealle puolelle $f(-e^t)$, muuten tulos on sama.

Homogeenista Eulerin yhtälöä ei käytännössä tarvitse muuntaa vakiokertoimiseksi, sillä
ratkaisua voi etsiä suoraan yritteellä
\[
y(x)=x^r.
\]
Tällöin $r$:n määräämiseksi saadaan toisen asteen karakteristinen yhtälö. Riippuen siitä, 
millaisia juuret ovat, yleiseksi ratkaisuksi tulee jokin seuraavista:
\begin{itemize}
\item[a)] $y(x)=C_1x^{r_1}+C_2x^{r_2}$,
\item[b)] $y(x)=x^r(C_1+C_2\ln\abs{x})$,
\item[c)] $y(x)=x^\alpha[\,C_1\cos (\beta\ln\abs{x})+C_2\sin (\beta\ln\abs{x})\,]$.
\end{itemize}
Tapauksessa a) juuret ovat reaaliset ja erisuuruiset, tapauksessa b) on reaalinen kaksoisjuuri,
ja tapauksessa c) juurina on konjugaattipari $\alpha\pm i\beta$. Väitetyt yleisen ratkaisun
muodot voi päätellä em.\ muunnoksen avulla.

Täydellinen Eulerin yhtälö ratkaistaan suotuisissa tapauksissa 'sivistyneellä arvauksella'.
Arvauksen muodon voi johtaa ajatellen muunnosta vakiokertoimiseen tilanteeseen.
\begin{Exa}
Ratkaise alkuarvotehtävä
\[
\begin{cases} \,x^2y''+3xy'+y=1/x,\,\ x>0, \\ \,y(1)=1,\ y'(1)=0. \end{cases}
\]
\end{Exa}
\ratk Homogeenisessa yhtälössä sijoitus $y(x)=x^r$ johtaa karakteristiseen yhtälöön
\[
r(r-1)+3r+1=r^2+2r+1=0.
\]
Tällä on kaksoisjuuri $r=-1$, joten homogeenisen yhtälön yleinen ratkaisu on
\[
y(x)=\frac{1}{x}(C_1+C_2\ln x).
\]
Koska oikea puoli $f(x)=x^{-1}$ on homogeenisen yhtälön ratkaisu, samoin kuin 
$x^{-1}\ln\abs{x}$, niin täydellisen yhtälön yksittäisratkaisu on etsittävä muodossa
\begin{align*}
y_0(x)   &= Ax^{-1}(\ln x)^2.
\intertext{Kun sijoitetaan yhtälöön tämä sekä derivaatat}
y_0'(x)  &= Ax^{-2}[\,-(\ln x)^2+2\ln x\,], \\
y_0''(x) &= Ax^{-3}[\,2(\ln x)^2-6\ln x+2\,],
\end{align*}
niin saadaan ratkaisu, kun $A=1/2$. Siis yleinen ratkaisu on
\[
y(x)=\frac{1}{x}\left(C_1+C_2\ln x+\frac{1}{2}(\ln x)^2\right).
\]
Alkuehdot toteutuvat, kun valitaan $C_1=C_2=1$. \loppu

\subsection{Korkeamman kertaluvun vakiokertoimiset DY:t}
\index{lineaarinen differentiaaliyhtälö!e@korkeamman kertaluvun|vahv}

Yleinen lineaarinen, vakiokertoiminen, homogeeninen, $n$:nnen kertaluvun differentiaaliyhtälö
\[
y^{(n)}+a_{n-1}y^{(n-1)}+\cdots +a_0y=0
\]
voidaan ratkaista yritteellä $y(x)=e^{rx}$, kuten tapauksissa $n=1,2$. Saadaan
\index{karakteristinen yhtälö (polynomi)!a@differentiaaliyhtälön}%
karakteristinen yhtälö
\[
r^n+a_{n-1}r^{n-1}+\cdots+a_0=0,
\]
jonka kutakin juurta vastaa juuren kertaluvun mukainen määrä lineaarisesti riippumattomia 
ratkaisuja. Jos kyseessä on $m$-kertainen reaalijuuri, niin nämä ratkaisut ovat
\[
x^ke^{rx},\quad k=0\ldots m-1.
\]
Jos kyseessä on $m$-kertainen konjugaattipari $\alpha\pm i\beta$, niin vastaavat ratkaisut ovat
\[
x^ke^{\alpha x}\cos\beta x,\quad x^ke^{\alpha x}\sin\beta x,\quad k=0\ldots m-1.
\]
Homogeenisen yhtälön yleinen ratkaisu saadaan kaikkien näiden ratkaisujen (yhteensä $n$ kpl)
lineaarisena yhdistelynä.
\begin{Exa}
Ratkaise $\,y'''-y=x^2,\,\ x\in\R$.
\end{Exa}
\ratk Yksittäisratkaisu on $y(x)=-x^2$. Homogeenista yhtälöä ratkaistaessa tulee
karakteristiseksi yhtälöksi $\,r^3-1=0$. Juuret ovat
\[
r_1=1, \quad r_{2,3}= -\frac{1}{2} \pm i\frac{\sqrt{3}}{2}\,,
\]
joten yleinen ratkaisu on
\[
y(x)=C_1e^x
    +e^{-x/2}\left(C_2\cos\frac{\sqrt{3}x}{2}+C_3\sin\frac{\sqrt{3}x}{2}\right)-x^2. \loppu
\]

Kuten esimerkissä, voidaan täydellisen yhtälön ratkaisu usein löytää kokeilemalla. Yleisempiin
menetelmiin ei juuri käytännön tarvetta olekaan. 

Todettakoon lopuksi, että yleinen $n$:nnen kertaluvun homogeeninen Eulerin DY
\[
x^{n}y^{(n)}+a_{n-1}x^{n-1}y^{(n-1)}+ \ldots + a_0y = 0
\]
ratkaistaan yritteellä $y(x)=x^r$ samaan tapaan kuin edellä.

\Harj
\begin{enumerate}

\item
Ratkaise (yleinen ratkaisu kohdissa a)--n), yleinen ratkaisu kaikilla lukupareilla
$(a,b)\in\Rkaksi$ kohdissa o)--t), alku- tai reuna-arvotehtävän ratkaisu 
kohdissa u)--ö)): \vspace{1mm}\newline
a) \ $y''+y'-30y=0 \qquad\qquad$
b) \ $y''+4y'+y= 0$ \newline
c) \ $y''+6y'+10y=0 \qquad\quad\ \ $
d) \ $y''+4y'+6y=0$ \newline
e) \ $y''+4y'+5y=3x-2 \qquad$
f) \ $y''-y'-2y=x^2$ \newline
g) \ $y''-7y'+6y=\sin x \qquad\,\ \ $
h) \ $y''+4y=\sin 3x$ \newline
i) \ $y''+2y'+5y=e^{-x} \qquad\quad\ \ $
j) \ $y''+y'-2y=x+e^x$ \newline
k) \ $y''-4y=xe^{2x} \qquad\qquad\quad\,\ $
l) \ $y''+4y'+4y=(x-1)^2e^{-2x}$ \newline
m) \ $y''-6y'+9y=(xe^x)^3 \quad\ \ $
n) \ $y''+4y'+4y=e^{-2x}+\sin x$ \newline
o) \ $y''+ay'=e^{bx} \qquad\qquad\quad\ \ $
p) \ $y''-a^2y=e^{bx}$ \newline
q) \ $y''+a^2y=e^{bx} \qquad\qquad\quad\,\ $
r) \ $y''+a^2y=\sin bx$ \newline
s) \ $y''+a^2y=x\sin bx \qquad\quad\ \ $
t) \ $y''+2y'+(1+a^2)y=e^{bx}$ \newline
u) \ $y''+2y'+y=0,\,\ y(0)=0,\ y'(0)=1$ \newline
v) \ $y''+2y'+2y=0,\,\ y(0)=1,\ y'(0)=0$ \newline
x) \ $y''-4y'-5y=0,\,\ y(0)=1,\ \lim_{x\kohti\infty}y(x)=0$ \newline
y) \ $y''+3y'+2y=0,\,\ y(0)=y(1)=1$ \newline
x) \ $y''-4y'+5y=\sin x,\,\ y(0)=0,\ y'(0)=-1$ \newline
å) \ $y''-3y'+x^2-1=0,\,\ y(0)=1,\ y'(0)=0$ \newline
ä) \ $y''-5y'+6y=e^x,\,\ y(0)=y(1)=0$ \newline
ö) \ $4y''+8y'+5y=x^2,\,\ y(0)=y(\pi)=1$

\item
Etsi seuraaville differentiaaliyhtälöille ensin kompleksiarvoinen yksittäisratkaisu, kun
yhtälön oikea puoli on $f(x)=e^{i\omega x}$ ($\omega\in\R,\ \omega \neq 0$), ja määritä sen
avulla reaalinen yksittäisratkaisu. \vspace{1mm}\newline
a) \ $y''-2y'+2y=\sin\omega x \qquad$ 
b) \ $y''-3y'+2y=\sin\omega x$ \newline
c) \ $y''+4y'+4y=\cos\omega x \qquad$
d) \ $y''+y'+y=\sin\omega x-2\cos\omega x$

\item \index{zzb@\nim!Resonanssi}
(Resonanssi) Kappaleeseen, jonka massa $=m$ ja paikka $=y(t)$ hetkellä $t$, kohdistuu
jousivoima ja lisäksi ulkoinen kuorma $f(t)$, jolloin $y(t)$ toteuttaa liikeyhälön
$\,my''=-ky+f(t)$. Määritä $y(t),\ t \ge 0$ alkuehdoilla $y(0)=y'(0)=0$, kun
$f(t)=F\sin\omega t$ ($F=$ vakio). Tarkastele erikseen tapaukset \,a) $\,\omega\neq\omega_0\,$
ja \,b) $\omega=\omega_0$, missä $\,\omega_0=\sqrt{k/m}\,$ (nk.\ resonanssitaajuus).

\item \index{zzb@\nim!Tzz@Töyssy}
(Töyssy) Auton etupyörä osuu töyssyyn, jolloin pyörän liikeyhtälö pystysuunnassa on
$\,my''=F_1+F_2$, missä $m=$ pyörän massa, $F_1=-ky$ on jousivoima ja $F_2=-cy'$ on
iskunvaimentimen vastusvoima. Hahmottele ratkaisu $y(t)$ (aikayksikkö s) alkuehdoilla
$y(0)=0$ m, $y'(0)=60$ m/s, kun $m=15$ kg, $k=6 \cdot 10^5$ kg/s$^2$ ja iskunvaimennin on
a) uusi: $c=7500$ kg/s, b) pian vaihdettava: $c=6000$ kg/s,\, c) heti vaihdettava:
$c=4500$ kg/s.

\item
Ratkaise (kohdassa i) kaikilla $(a,b)\in\Rkaksi$): \vspace{1mm}\newline
a) \ $x^2y''-xy'+y=0 \qquad\qquad\ \ $
b) \ $x^2y''-6xy'+7y=0$ \newline
c) \ $x^2y''-2xy'+2y=x \qquad\quad\ \ $
d) \ $x^2y''+xy'-y=(x+1)^2/x^2$ \newline
e) \ $x^2y''+xy'+y=x^3-2x \qquad$
f) \ $x^2y''-3xy'+4y=x^2\ln\abs{x}$ \newline
g) \ $(x^2+2x+1)y''+(x+1)y'+y=x^2$ \newline
h) \ $(3x+2)^2y''+(9x+6)y'-36y=81x+18$ \newline
i) \,\ $x^2y''+(2a+1)xy'+by=0$

\item
a) Määritä kaikki välillä $(0,\infty)$ derivoituvat funktiot, jotka toteuttavat yhtälön
$\,x^2y'(x)=2\int_0^x y(t)\,dt,\ x>0$. \newline
b) Ratkaise differentiaaliyhtälö $\,x^2y'''+2(x^2-x)y''+(x^2-2x+2)y'=x^3$ tekemällä sijoitus
$u=y'/x$.

\item
Ratkaise (yleinen ratkaisu tai alkuarvotehtävän ratkaisu): \vspace{1mm}\newline
a) \ $y^{(5)}+2y'''+y'=0 \qquad\quad\,$
b) \ $y^{(7)}+3y^{(6)}+3y^{(5)}+y^{(4)}=0$ \newline
c) \ $y'''+3y''-2y=\sin x \qquad$
d) \ $y^{(4)}+3y'''+3y''+y'=2e^{-2x}-2x$ \newline
e) \ $y^{(4)}+a^4y=x^2,\,\ a \ge 0$ \newline
f) \ $y'''-(a+2)y''+(2a+1)y'-ay=x+1,\,\ a\in\R$ \newline
g) \ $x^3y'''+2ax^2y''-4a(xy'-y)=0,\,\ a\in\R$ \newline
h) \ $y'''-5y''+17y'-13y=0,\,\ y(0)=y'(0)=0,\ y''(0)=24$ \newline
i) \ $y'''-y''-y'+y=e^x,\,\ y(0)=y'(0)=y''(0)=0$ \newline
j) \ $y'''-ay'=0,\,\ y(0)=0,\ y'(0)=1,\ y''(0)=2,\,\ a\in\R$ \newline
k) \ $x^3y'''+xy'-y=\sqrt{x},\,\ y(1)=1,\ y'(1)=y''(1)=0$

\item (*) \index{zzb@\nim!Pystyammunta}
(Pystyammunta) Luoti ammutaan suoraan ylöspäin lähtönopeudella $800$ m/s. Lennon aikana
luotiin vaikuttaa nopeuteen verrannollinen vastusvoima, jolloin nousukorkeus $y(t)$ toteuttaa
differentiaaliyhtälön
\[
y''+ay'+g=0, 
\]
missä $g=9.8$ m/s$^2$, $a=0.10$ s$^{-1}$ luodin nousuvaiheessa ja $a=0.20$ s$^{-1}$
putoamisvaiheessa. Määritä luodin lennon lakikorkeus, lentoon kuluva aika ja luodin paluunopeus
sen pudotessa maahan. 

\item (*) \label{H-dy-5: linvak-ty}
Vakiokertoimisessa differentiaaliyhtälössä $\,y''+ay'+by=f(x)\,$ olkoon karakteristisen
yhtälön juuret $r_1,r_2\in\C$. Kirjoittamalla differentiaaliyhtälö systeemiksi funktioille $y$
ja $u=y'-r_2y$ näytä, että yhtälön yksittäisratkaisu alkuehdoilla $y(x_0)=y'(x_0)=0$ on
\[
y_0(x)=\int_{x_0}^x e^{r_1(x-t)}\left[\int_{x_0}^t e^{r_2(t-s)}f(s)\,ds\right]dt.
\]
Tarkista kaavan toimivuus, kun $a=-2$, $b=1$, $x_0=0$ ja $f(x)=x^2-4x+2$.

\end{enumerate} %Lineaariset, vakiokertoimiset DY:t. Eulerin differentiaaliyhtälö
\section{*Yleinen 2.\ kertaluvun lineaarinen DY} \label{2. kertaluvun lineaarinen DY}
\sectionmark{\ Yleinen 2. kertaluvun lineaarinen DY}
\alku
\index{lineaarinen differentiaaliyhtälö!d@2.\ kertaluvun|vahv}

Yleinen toisen kertaluvun lineaarinen differentiaaliyhtälö on muotoa
\begin{equation} \label{lin-2: ty}
y''+P(x)y'+Q(x)y=R(x).
\end{equation}
Perusoletus on, että sekä
\index{kerroin (DY:n)} \index{oikea puoli (DY:n)}%
\kor{kerroinfunktiot} $P$ ja $Q$ että yhtälön \kor{oikea puoli} $R$ 
ovat jatkuvia tarkasteltavalla välillä $(a,b)$. Ratkaisuna pidetään tällöin jokaista ko.\
välillä kahdesti derivoituvaa funktiota $y(x)$, joka toteuttaa yhtälön. --- Yhtälöstä on
tällöin luettavissa, että myös $y''$ on jatkuva välillä $(a,b)$. Jatkossa tarkastellaan aluksi
yhtälön \eqref{lin-2: ty} homogeenista erikoistapausta, jossa $R(x)=0$.

\subsection{Homogeeninen yhtälö: Ratkaisuavaruus}
\index{lineaarinen differentiaaliyhtälö!a@homogeeninen|vahv}
\index{ratkaisuavaruus (DY:n)|vahv}

Kun homogeeninen differentiaaliyhtälö kirjoitetaan
\begin{equation} \label{lin-2: hy}
\dyf y=y''+P(x)y'+Q(x)y=0,
\end{equation}
niin operaattori $\dyf$ on jälleen lineaarinen:
\index{differentiaalioperaattori!b@lineaarisen DY:n}%
\[
\dyf(c_1y_1+c_2y_2)=c_1\dyf y_1+c_2\dyf y_2, \quad c_1,c_2\in\R.
\]
Siis jos $y_1$ ja $y_2$ ovat yhtälön \eqref{lin-2: hy} ratkaisuja, niin ratkaisu on myös mikä
tahansa näiden lineaarinen yhdistely
\[
y(x)=C_1y_1(x)+C_2y_2(x),\quad C_1,C_2\in\R.
\]
Ratkeavuusteorian keskeisin tulos on, että tämä itse asiassa on yhtälön \eqref{lin-2: hy}
yleinen ratkaisu edellyttäen ainoastaan, että funktiot $y_1$ ja $y_2$ ovat
\index{lineaarinen riippumattomuus}%
\kor{lineaarisesti riippumattomat} välillä $(a,b)$. Tällä tarkoitetaan, että pätee (vrt.\ Luku 
\ref{funktioavaruus})
\[
c_1y_1(x)+c_2y_2(x)=0 \quad \forall x\in (a,b) \ \impl \ c_1=c_2=0.
\]
\index{lineaarinen riippuvuus}%
Jos tämä ei päde, niin sanotaan, että $y_1$ ja $y_2$ ovat \kor{lineaarisesti riippuvat} välillä
$(a,b)$. Tässä tapauksessa on joko $y_2=cy_1$ tai $y_1=cy_2$ ko.\ välillä jollakin $c\in\R$. 
\begin{*Lause} \label{lin-2: lause 1} Jos differentiaaliyhtälössä \eqref{lin-2: hy} 
kerroinfunktiot $P$ ja $Q$ ovat jatkuvia välillä $(a,b)$, niin yhtälöllä on ko. välillä kaksi
lineaarisesti riippumatonta ratkaisua. Edelleen jos $y_1$ ja $y_2$ ovat mitkä tahansa kaksi
yhtälön \eqref{lin-2: hy} lineaarisesti riippumatonta ratkaisua välillä $(a,b)$, niin yleinen
ratkaisu ko.\ välillä on
\[
y(x)=C_1y_1(x)+C_2y_2(x),\quad C_1,C_2\in\R.
\]
\end{*Lause}
\begin{Exa} \label{lin-2: lause 1 - esim}
Eulerin differentiaaliyhtälön
\[
x^2y''-3xy'-3y=0
\]
perusmuodossa \eqref{lin-2: hy} kerroinfunktiot ovat $P(x)=-3/x$ ja $Q(x)=-3/x^2$. Nämä ovat
jatkuvia väleillä $(-\infty,0)$ ja $(0,\infty)$, ja näillä väleillä yleinen ratkaisu on
(ks.\ edellinen luku)
\[
y(x)=C_1x+C_2x^3, \quad C_1,C_2\in\R.
\]
Tulos on sopusoinnussa Lauseen \ref{lin-2: lause 1} kanssa, sillä $y_1(x)=x$ ja $y_2(x)=x^3$
ovat lineaarisesti riippumattomat jokaisella välillä $(a,b)$:
\[
c_1x+c_2x^3 = 0 \ \ \forall x \in (a,b) \qekv c_1=c_2 = 0. \loppu
\]
\end{Exa}
\jatko \begin{Exa} (jatko) Jos esimerkin differentiaaliyhtälöä $\,x^2y''-3xy'-3y=0$
tarkastellaan välillä $(-\infty,\infty)$, niin ratkaisuksi havaitaan myös
\[
y_3(x) = |x|^3 = \begin{cases} 
                 \,x^3, &\text{kun}\ x \ge 0, \\ \,-x^3, &\text{kun}\ x<0,
                 \end{cases}
\]
sillä tämä on kahdesti derivoituva ja toteuttaa yhtälön koko $\R$:ssä. Tätä funktiota ei
kuitenkaan voi ilmaista muodossa $C_1y_1(x)+C_2y_2(x)$, sillä $y_1(x)=x$ ja $y_2(x)=x^3$
(samoin niiden lineaariset yhdistelyt) ovat parittomia, kun taas $y_3$ on parillinen. --- Tulos
ei ole ristiriidassa Lauseen \ref{lin-2: lause 1} kanssa, sillä lauseen oletus
kerroinfunktioiden $P$ ja $Q$ jatkuvuudesta ei toteudu välillä $(-\infty,\infty)$. Huomattakoon
myös, että väleillä $(0,\infty)$ ja $(-\infty,0)$ on $y_3(x)=\pm y_2(x)$, joten näillä
väleillä $y_3$ ei tuo mitään uutta yleiseen ratkaisuun. \loppu
\end{Exa}
Lauseen \ref{lin-2: lause 1} tulos on tulkittavissa niin, että yhtälön \eqref{lin-2: hy}
ratkaisujen joukko $\mathcal{Y}$ on tehdyin oletuksin \pain{2--ulotteinen} \pain{vektoriavaruus}
(funktioavaruus, vrt.\ Luku \ref{funktioavaruus}). Tämän avaruuden \pain{kanta} on mikä tahansa
lineaarisesti riippumaton ratkaisupari $\{y_1,y_2\}$, jolloin koko ratkaisuavaruus voidaan
esittää muodossa
\[
\mathcal{Y}=\{y=c_1y_1+c_2y_2 \ | \ c_1,c_2\in\R\}.
\]
Lauseen \ref{lin-2: lause 1} (osittainen) todistus esitetään jäljempänä. Tätä ennen 
tarkastellaan lauseen seuraamuksia ajatellen yhtälön \eqref{lin-2: hy} ratkaisemista
kvadratuureilla.

\subsection{Homogeeninen yhtälö: Ratkaiseminen kvadratuureilla}
\index{lineaarinen differentiaaliyhtälö!a@homogeeninen|vahv}

Ensinnäkin joudutaan toteamaan, että yhtälö \eqref{lin-2: hy} \pain{ei} aina ratkea
kvadratuureilla. Kvadratuureihin palautumaton on esimerkksi niinkin yksinkertainen yhtälö kuin
\[
y''=xy.
\]
Tämän huonon uutisen jälkeen todettakoon, että jos yhtälölle \eqref{lin-2: hy} on 
keksittävissä edes y\pain{ksi} ei-triviaali (eli nollasta poikkeava) ratkaisu $y_1$, niin toinen, 
$y_1$:stä lineaarisesti riippumaton ratkaisu --- ja niin muodoin yleinen ratkaisu --- on
konstruoitavissa kvadratuureilla. Jatkossa rajoitutaan tähän tapaukseen, eli oletetaan yksi
ratkaisu $y_1 \neq 0$ tunnetuksi.

Jos tunnetaan yhtälön \eqref{lin-2: hy} ratkaisu $y_1\neq 0$, niin toista ratkaisua voidaan 
etsiä tuttuun tapaan \pain{vakion} \pain{varioinnilla}, eli muodossa
\[
y(x)=C(x)y_1(x).
\]
Tällöin on 
\[
y' = y_1C'+y_1'C, \quad y'' = y_1C''+2y_1'C'+y_1''C.
\]
Koska $\dyf y_1=0$, seuraa $C$:lle differentiaaliyhtälö
\[
y_1(x)C''+[2y_1'(x)+P(x)y_1(x)]C'=0.
\]
Sijoituksella $u=C'$ tästä tulee separoituva:
\[
u'+K(x)u=0,\quad K(x)=2\,\frac{y_1'(x)}{y_1(x)}+P(x)=2\frac{d}{dx}\ln\abs{y_1(x)}+P(x).
\]
Separoimalla ja integroimalla saadaan
\begin{align*}
\int\frac{du}{u}  &= -\int K(x)\,dx \\
\ekv \ \ln\abs{u} &= -2\ln\abs{y_1(x)}-\int P(x)\,dx + A \quad (A\in\R).
\end{align*}
Valitsemalla $A=0$ (yksi ratkaisu $u\neq 0$ riittää!) saadaan
\[
u(x)=\frac{e^{-\int P(x)dx}}{[y_1(x)]^2}=C'(x).
\]
Näin ollen eräs ratkaisu on
\begin{align*}
y_2(x) &= y_1(x)\int u(x)\,dx \\
       &= y_1(x) \int \frac{e^{-\int P(x)dx}}{[y_1(x)]^2}\,dx.
\end{align*}
Ratkaisu on pätevä ainakin $y_1$:n nollakohtien välisillä avoimilla väleillä. Tällaisilla
väleillä suhde $u=y_2/y_1$ ei selvästikään ole vakio (koska on positiivisen funktion
integraalifunktio), joten löydetty ratkaisu on $y_1$:stä lineaarisesti riippumaton.
Homogeenisen yhtälön \eqref{lin-2: hy} yleinen ratkaisu on tällöin (ainakin mainituilla
väleillä) kirjoitettavissa
\[
y(x)=C_1y_1(x)+C_2y_2(x).
\]
\begin{Exa} \label{lin-2: Ex4}
Differentiaaliyhtälöllä
\[
y''+xy'-y=0
\]
on ilmeinen ratkaisu $\,y_1(x)=x$, joten etsitään toista muodossa
\[
y(x)=xC(x) \qimpl y'=xC'+C, \quad y''=xC''+2C'.
\]
Sijoitus yhtälöön antaa
\[
xu'+(x^2+2)u=0, \quad u=C'.
\]
Ratkaisu separoimalla (ol.\ $x>0$):
\begin{align*}
&\int\frac{du}{u} \,=\, \ln u
                  \,=\, -\int\left(x+\frac{2}{x}\right)dx 
                  \,=\, \ln\left(x^{-2}e^{-x^2/2}\right)+\ln A \\
&\qimpl u=C'=Ax^{-2}e^{-x^2/2}.
\end{align*}
Valitsemalla $A=1$, asettamalle ehto $\lim_{x\kohti\infty}C(x)=0$ ja integroimalla osittain
saadaan
\[
C(x) = -\int_x^\infty u(t)\,dt = -x^{-1}e^{-x^2/2}+\int_x^\infty e^{-t^2/2}\,dt.
\]
Yleinen ratkaisu on näin ollen
\[
y(x) \,=\, C_1x + C_2\,xC(x)
     \,=\, C_1x + C_2\left(-e^{-x^2/2}+x\int_x^\infty e^{-t^2/2}\,dt\right), \quad C_1,C_2\in\R. 
\]
Vaikka laskun välivaiheissa oletettiin $x>0$, on ratkaisu pätevä koko $\R$:ssä. \loppu
\end{Exa}

\subsection{Homogeeninen yhtälö: ratkeavuusteoria}
\index{lineaarinen differentiaaliyhtälö!a@homogeeninen|vahv}

Palataan Lauseeseen \ref{lin-2: lause 1}, jota ei ole todistettu. Lauseen väittämistä
syvällisin koskee --- kuten tavallista --- ratkaisujen $y_1$ ja $y_2$ olemassaoloa. Jatkossa
johdetaan tämä tulos, samoin kuin Lauseen  \ref{lin-2: lause 1} muut väittämät, seuraavasta
peruslauseesta, joka puolestaan on erikoistapaus paljon yleisemmästä alkuarvotehtävän
ratkeavuutta koskevasta väittämästä. Väittämää ei tässä vaiheessa todisteta, vaan asiaan
palataan myöhemmin luvussa \ref{Picard-Lindelöfin lause}. Puheena olevaan erikoistapaukseen
sovellettuna ratkeavuusväittämä on seuraava: 
\begin{*Lause} \label{lin-2: lause 2} \index{differentiaaliyhtälön!h@ratkeavuus|emph} 
Jos $P$ ja $Q$ ovat jatkuvia välillä $(a,b)$ ja $x_0\in (a,b)$, niin alkuarvotehtävällä
\[
\left\{ \begin{aligned}
&y''+P(x)y'+Q(x)y=0,\quad x\in (a,b), \\
&y(x_0)\,=A, \\
&y'(x_0) =B
\end{aligned} \right.
\]
on yksikäsitteinen ratkaisu jokaisella $A,B\in\R$.
\end{*Lause}

Jatkossa siis todistetaan väittämä: \ 
$\text{Lause \ref{lin-2: lause 2}}\ \impl\ \text{Lause \ref{lin-2: lause 1}}$. Todistuksessa
näyttelee keskeistä roolia seuraava käsite.
\begin{Def} \index{Wronskin determinantti|emph} Välillä $(a,b)$ derivoituvien funktioiden
$y_1$ ja $y_2$ \kor{Wronskin determinantti} on
\[
W_{12}(x)=\begin{vmatrix}
y_1(x) & y_2(x) \\
y_1'(x) & y_2'(x)
\end{vmatrix} = (y_1y_2'-y_2y_1')(x), \quad x\in(a,b).
\]
\end{Def}
\begin{Prop} \label{lin-2: prop} Jos $y_1$ ja $y_2$ ovat derivoituvia välillä $(a,b)$ ja 
$W_{12}(x_0)=(y_1y_2'-y_2y_1')(x_0)\neq 0$ jollakin $x_0\in (a,b)$, niin $y_1$ ja $y_2$ ovat
lineaarisesti riippumattomat välillä $(a,b)$.
\end{Prop}
\tod Jos joillakin $c_1,c_2\in\R$ pätee
\[
c_1y_1(x)+c_2y_2(x)=0,\quad x\in (a,b),
\]
niin derivoimalla seuraa, että myös
\[
c_1y_1'(x)+c_2y_2'(x)=0,\quad x\in (a,b).
\]
Kun nämä yhtälöt kirjoitetaan pisteessä $x_0$, niin saadaan yhtälöryhmä
\[ \left\{ \begin{aligned}
\,y_1(x_0)c_1+y_2(x_0)c_2 &= 0, \\ y_1'(x_0)c_1+y_2'(x_0)c_2 &= 0.
           \end{aligned} \right. \]
Koska $W_{12}(x_0)\neq 0$, niin yhtälöryhmän ainoa ratkaisu on 
$c_1=c_2=0$.\footnote[2]{Jos $\,a_{ij},x_i,b_i\in\R,\ i,j=1,2$ ja 
$\,W=a_{11}a_{22}-a_{21}a_{12} \neq 0$, niin pätee
\[
\left\{ \begin{aligned} 
        a_{11}x_1+a_{12}x_2 &= b_1 \\ a_{21}x_1+a_{22}x_2 &= b_2 
        \end{aligned} \right.
\qekv \left\{ \begin{aligned} 
        x_1 &= W^{-1}(a_{22}b_1-a_{12}b_2), \\ x_2 &= W^{-1}(a_{11}b_2-a_{21}b_1). 
        \end{aligned}\right.
\]}
Siis oletuksesta $\,c_1y_1(x)+c_2y_2(x)=0,\ x\in(a,b)\,$ seuraa, että $c_1=c_2=0$, joten $y_1$
ja $y_2$ ovat lineaarisesti riippumattomat välillä $(a,b)$. \loppu
\begin{Exa} Funktiot $y_1(x)=x^3$ ja $y_2(x)=|x|^3$ ovat (kehdestikin) derivoituvia ja
lineaarisesti riippumattomia $\R$:ssä (vrt.\ Esimerkki \ref{lin-2: lause 1 - esim} edellä).
Laskemalla näiden funktioiden Wronskin determinantti todetaan, että
$W_{12}(x)=0\ \forall x\in\R$. Tämän (vasta)esimerkin perusteella päätellään:
\begin{align*}
&\text{$y_1$ ja $y_2$ lineaarisesti riippumattomat välillä $(a,b)$} \\[1mm]
&\qquad\qquad\ \ \not\impl\quad W_{12}(x_0) \neq 0 \quad \text{jollakin}\ x_0\in(a,b). \loppu
\end{align*}
\end{Exa}
Esimerkin perusteella Proposition \ref{lin-2: prop} (implikaatio)väittämä ei päde kääntäen. Jos
sen sijaan oletuksia vahvistetaan niin, että $y_1$ ja $y_2$ oletetaan differentiaaliyhtälön
\eqref{lin-2: hy} ratkaisuiksi välillä $(a,b)$, niin käänteinenkin väittämä on tosi. Tällaisella
oletuksella saadaan seuraava, Propositiota \ref{lin-2: prop} huomattavasti vahvempi tulos.
\begin{Lause} \label{lin-2: lause 3} Jos $y_1$ ja $y_2$ ovat differentiaaliyhtälön 
\eqref{lin-2: hy} ratkaisuja välillä $(a,b)$ \ \ ($P$ ja $Q$ jatkuvia välillä $(a,b)$) ja 
$W_{12}(x)$ on funktioiden $y_1,y_2$ Wronskin determinantti välillä $(a,b)$, niin pätee
\begin{align*}
      &W_{12}(x_0)=0 \quad \text{jollakin}\ x_0\in (a,b) \\[1mm]
\qekv &W_{12}(x)\ =0 \quad \text{jokaisella}\ x\in(a,b) \\[1mm]
\qekv &\text{$y_1$ ja $y_2$ lineaarisesti riippuvat välillä $(a,b)$}.
\end{align*}
\end{Lause}
\tod Oletuksien mukaan
\[
\begin{cases}
y_1''+P(x)y_1'+Q(x)y_1=0,\quad &x\in (a,b), \\
y_2''+P(x)y_2'+Q(x)y_2=0,\quad &x\in (a,b).
\end{cases}
\]
Kertomalla ensimmäinen yhtälö $y_2$:lla ja toinen $y_1$:llä ja vähentämällä seuraa
\[
y_1y_2''-y_2y_1''+P(x)(y_1y_2'-y_2y_1')=0,\quad x\in (a,b).
\]
Mutta
\[
y_1y_2''-y_2y_1''=\frac{d}{dx}(y_1y_2'-y_2y_1').
\]
Siis on päätelty, että $y(x)=W_{12}(x)$ on jokaisella $x_0\in(a,b)$ ratkaisu alkuarvotehtävälle
\begin{equation} \label{Wronskin yhtälö}
\begin{cases}
\,y'+P(x)y=0,\quad x\in (a,b) \\ \,y(x_0)=W_{12}(x_0)
\end{cases}
\end{equation}
Koska ratkaisu on (vrt.\ Luku \ref{lineaarinen 1. kertaluvun DY})
\[
W_{12}(x)=W_{12}(x_0)\,e^{-\int_{x_0}^x P(t)dt},\quad x\in (a,b)
\]
ja koska $e^t \neq 0\ \forall t\in\R$, niin seuraa ensimmäinen osaväittämä: \ 
$W_{12}(x_0)=0\ \ekv\ W_{12}(x)=0\,\ \forall x\in(a,b)$.

Tapauksessa $y_1(x)=0\ \forall x\in(a,b)$ on toinenkin osaväittämä tosi, joten oletetaan, että
$y_1(x_0) \neq 0$ jollakin $x_0\in(a,b)$. Tällöin on $y_1$:n jatkuvuuden nojalla 
$y_1(x) \neq 0$ jollakin välillä $(x_0-\delta,x_0+\delta)\subset(a,b)$, $\delta>0$. Tällä 
välillä $y_2/y_1$ on derivoituva ja
\[
\frac{d}{dx}\left[\frac{y_2(x)}{y_1(x)}\right]=\frac{W_{12}(x)}{[y_1(x)]^2}\,.
\]
Näin ollen jos $\,W_{12}(x)=0\,$ välillä $(a,b)$, niin jollakin $C\in\R$ pätee
\[
\frac{y_2(x)}{y_1(x)}=C\ \ekv\ -Cy_1(x)+y_2(x)= 0 \quad \text{välillä}\ (x_0-\delta,x_0+\delta).
\]
Kun merkitään $\,y(x)=-Cy_1(x)+y_2(x),\ x\in(a,b)$, niin on siis $y(x)=0$ mainitulla osavälillä,
jolloin on erityisesti $y(x_0)=y'(x_0)=0$. Koska $y$ on myös differentiaaliyhtälön 
\eqref{lin-2: hy} ratkaisu välillä $(a,b)$, niin $y$ ratkaisee siis Lauseen \ref{lin-2: lause 2}
alkuarvotehtävän, kun $A=B=0$. Koska ilmeinen ratkaisu on myös $y=0$ ja koska ratkaisu on 
mainitun lauseen mukaan yksikäsitteinen, niin on siis oltava 
$\,y(x)=-Cy_1(x)+y_2(x)=0,\ x\in(a,b)$. Tällöin $y_1$ ja $y_2$ ovat lineaarisesti riippuvat
välillä $(a,b)$. Päättely perustui oletukseen, että $\,W_{12}(x)=0\,$ välillä $(a,b)$, joten
toisen osaväittämän osa \fbox{$\impl$} tuli todistetuksi. Jäljelle jäävä osa \fbox{$\Leftarrow$}
on jo todistettu, sillä tämä on Proposition \ref{lin-2: prop} väittämän loogisesti
ekvivalenttinen muoto. \loppu

Nyt ollaan valmiita esittämään (Lauseeseen \ref{lin-2: lause 2} nojaava)
 
\underline{Lauseen \ref{lin-2: lause 1} todistus} \ Olkoot $y_1$ ja $y_2$ Lauseen 
\ref{lin-2: lause 2} alkuarvotehtävän ratkaisut, kun\, a) $A=1,\ B=0$, \ b) $A=0,\ B=1$.
Tällöin funktioiden $y_1$ ja $y_2$ Wronskin determinantti $x_0$:ssa on
\[
W_{12}(x_0)=\begin{vmatrix}
1 & 0 \\
0 & 1
\end{vmatrix}=1.
\]
Proposition \ref{lin-2: prop} mukaan $y_1$ ja $y_2$ ovat lineaarisesti riippumattomat välillä
$(a,b)$, joten ensimmäinen osaväittämä on todistettu.

Toisen osaväittämän todistamiseksi oletetaan, että $y_1$ ja $y_2$ ovat mitkä tahansa kaksi 
differentiaaliyhtälön \eqref{lin-2: hy} lineaarisesti riippumatonta ratkaisua välillä $(a,b)$
ja että $u$ on kolmas saman DY:n ratkaisu. Tällöin jos voidaan määritellä kertoimet $c_1$ ja 
$c_2$ siten, että funktiolle $\,y(x)=c_1y_1(x)+c_2y_2(x)\,$ pätee 
$\,y(x_0)=u(x_0),\ y'(x_0)=u'(x_0)$, niin $u$ ja $y$ ovat saman alkuarvotehtävän ratkaisuja,
jolloin Lauseen \ref{lin-2: lause 2} mukaan on oltava $u(x)=y(x),\ x\in(a,b)$. Kertoimet
määräytyvät yhtälöryhmästä
\[ \left\{ \begin{aligned}
y_1(x_0)c_1+y_2(x_0)c_2 &= u(x_0), \\ y'_1(x_0)c_1+y'_2(x_0)c_2 &= u'(x_0).
           \end{aligned} \right. \]
Tämä ratkeaa yksikäsitteisesti ehdolla $\,y_1(x_0)y'_2(x_0)-y'_1(x_0)y_2(x_0) \neq 0$, eli
ehdolla $W_{12}(x_0) \neq 0$, missä $\,W_{12}\,$ on funktioiden $y_1$ ja $y_2$ Wronskin
determinantti. Mutta oletuksen ja Lauseen \ref{lin-2: lause 3} perusteella
$W_{12}(x_0) \neq 0$ jokaisella $x_0\in(a,b)$. Siis $u$ on esitettävissä funktioiden $y_1$ ja
$y_2$ lineaarisena yhdistelynä. \loppu

\subsection{Täydellinen yhtälö: vakioiden variointi}
\index{lineaarinen differentiaaliyhtälö!b@täydellinen}
\index{lineaarinen differentiaaliyhtälö!g@vakio(ide)n variointi|vahv}
\index{vakio(ide)n variointi|vahv}

Jos homogeenisen yhtälön \eqref{lin-2: hy} yleinen ratkaisu tunnetaan, niin täydellisen
yhtälön \eqref{lin-2: ty} ratkaisemiseksi riittää jälleen löytää tälle yksittäisratkaisu.
Yleinen menetelmä perustuu vakioiden variointiin: Jos homogeenisen yhtälön yleinen ratkaisu
on $y(x)=C_1y_1(x)+C_2y_2(x)$, niin täydelliselle yhtälölle löydetään yksittäisratkaisu
muodossa
\[
y(x)=C_1(x)y_1(x)+C_2(x)y_2(x).
\]
Tällöin
\[
y'=C_1(x)y_1'(x)+C_2(x)y_2'(x)+C_1'(x)y_1(x)+C_2'(x)y_2(x).
\]
Tämän lausekkeen yksinkertaistamiseksi asetetaan lisäehto
\[
C_1'(x)y_1(x)+C_2'(x)y_2(x)=0
\]
(osoittautuu mahdolliseksi!), jolloin
\begin{align*}
y'  &= C_1(x)y_1'(x)+C_2(x)y_2'(x), \\
y'' &= C_1(x)y_1''(x)+C_2(x)y_2''(x) +C_1'(x)y_1'(x)+C_2'(x)y_2'(x)
\end{align*}
ja näin ollen
\begin{align*}
\dyf y &= y''+P(x)y'+Q(x)y \\
       &= C_1(x)\dyf y_1+C_2(x)\dyf y_2+C_1'(x)y_1'(x)+C_2'(x)y_2'(x).
\end{align*}
Tässä on $\dyf y_1=\dyf y_2=0$, joten oletettu lisäehto huomioiden on saatu yhtälöryhmä
\[
\begin{cases}
\,y_1(x)C_1'+y_2(x)C_2'=0, \\
\,y_1'(x)C_1'+y_2'(x)C_2'=R(x).
\end{cases}
\]
Koska tässä on (Lause \ref{lin-2: lause 3})
\[
(y_1y_2'-y_2y_1')(x)=W_{12}(x)\neq 0,
\]
niin yhtälöryhmä ratkeaa:
\[
C_1'=-\frac{y_2(x)R(x)}{W_{12}(x)}\,,\quad C_2'=\frac{y_1(x)R(x)}{W_{12}(x)}\,.
\]
Integroimalla $C_1(x)$ ja $C_2(x)$ näistä yhtälöistä on löydetty täydellisen yhtälön
\eqref{lin-2: ty} yksittäisratkaisu. Yhtälön \eqref{lin-2: ty} yleiseksi ratkaisuksi tulee näin
muodoin
\[
y(x)=C_1y_1(x)+C_2y_2(x)
    -y_1(x)\int\frac{y_2(x)R(x)}{W_{12}(x)}\,dx+y_2(x)\int\frac{y_1(x)R(x)}{W_{12}(x)}\,dx.
\]

\begin{Exa}
Ratkaise Eulerin differentiaaliyhtälö
\[
x^2y''+4xy'+2y=f(x), \quad x>0.
\]
\end{Exa}
\ratk Homogeenisen yhtälön yleinen ratkaisu on (ks.\ edellinen luku)
\[
y(x) \,=\, \frac{C_1}{x}+\frac{C_2}{x^2} \,=\, C_1y_1(x)+C_2y_2(x).
\]
Em.\ laskukaavoissa on
\[
R(x)=\frac{f(x)}{x^2}\,, \quad W_{12}(x)=(y_1y_2'-y_1'y_2)(x)=-\frac{1}{x^4}\,,
\] 
joten vakioiden variointi antaa
\begin{align*}
&C_1'= f(x) \qimpl C_1(x)=\int f(x)\,dx, \\
&C_2'=-xf(x) \qimpl C_2(x)=-\int xf(x)\,dx.
\end{align*}
Yksittäisratkaisu on siis
\[
y(x) = \frac{1}{x}\int f(x)\,dx - \frac{1}{x^2}\int xf(x)\,dx.
\]
Tämä on myös yleinen ratkaisu, kun oikealla puolella integraaleihin sisällytetään
määräämättömät integroimisvakiot. \loppu


\subsection{Tunnettuja differentiaaliyhtälöitä}

Toisen kertaluvun lineaarinen, homogeeninen differentiaaliyhtälö on sovelluksissa yleinen
differentiaaliyhtälön tyyppi. Seuraavassa muutamia tunnettuja differentiaaliyhtälöitä, jotka on
nimetty niitä tutkineiden matemaatikkojen mukaan.
\index{lineaarinen differentiaaliyhtälö!fb@Besselin, Hermiten, Laguerren, $\quad$
       Legendren, T\v{s}eby\v{s}evin}             
%\index{lineaarinen differentiaaliyhtälö!l@Besselin DY}
%\index{lineaarinen differentiaaliyhtälö!l@Hermiten DY}
%\index{lineaarinen differentiaaliyhtälö!l@Laguerren DY}
%\index{lineaarinen differentiaaliyhtälö!l@Legendren DY}
%\index{lineaarinen differentiaaliyhtälö!l@T\v{s}eby\v{s}evin DY}
\index{Besselin differentiaaliyhtälö}
\index{Hermiten!c@differentiaaliyhtälö}
\index{Laguerren differentiaaliyhtälö}
\index{Legendren differentiaaliyhtälö}
\index{Tsebysev@T\v{s}eby\v{s}evin differentiaaliyhtälö}%
\vspace{3mm}\newline
\kor{Legendre}: $\qquad(x^2-1)y''+2xy'-n(n+1)y=0$ \vspace{3mm}\newline
\kor{T\v{s}eby\v{s}ev}: $\qquad (1-x^2)y''-xy'+n^2y=0$ \vspace{3mm}\newline
\kor{Hermite}: $\qquad\ y''-2xy'+2ny=0$ \vspace{3mm}\newline
\kor{Laguerre}: $\qquad xy''+(1-x)y'+ny=0$ \vspace{3mm}\newline
\kor{Bessel}: $\quad\qquad x^2y''+xy'+(x^2-p^2)y=0$ \vspace{3mm}\newline
Näistä muilla paitsi Besselin differentiaaliyhtälöllä on ratkaisuna polynomi astetta $n$, kun
$n\in\N\cup\{0\}$ (sovelluksien kannalta kiinnostavin tapaus). Ko.\ polynomit on nimetty
samoin kuin differentiaaliyhtälöt. Besselin differentiaaliyhtälössä $p \ge 0$ on reaalinen
paramatetri

\subsection{Sarjaratkaisut}
\index{lineaarinen differentiaaliyhtälö!k@sarjaratkaisut|vahv}
\index{sarjaratkaisu (lineaarisen DY:n)|vahv}

Jos toisen kertaluvun lineaarisella, homogeenisella differentiaaliyhtälöllä ei ole yhtään 
'tunnistettavaa' ratkaisua, niin ratkaisemista voidaan yrittää \kor{sarjamenetelmällä}.
Tällöin ratkaisua etsitään joko potenssisarjana tai potenssisarjoja sisältävänä lausekkeena.
Menetelmän perusidea on sama kuin integroinnin sarjamenetelmässä 
(ks.\ Luku \ref{osamurtokehitelmät}).

Esimerkkinä sarjamenetelmän soveltamisesta olkoon Besselin differentiaaliyhtälö, jonka
perusmuoto on
\[
y''+\frac{1}{x}y'+(1-\frac{p^2}{x^2})y=0.
\]
Yleinen ratkaisu (sovelluksissa yleensä välillä $(0,\infty)$) kirjoitetaan muodossa
\[
y(x)=C_1 J_p(x)+C_2 Y_p(x),
\]
missä $J_p$ ja $Y_p$ ovat
\index{Besselin funktio}%
\kor{Besselin funktioita}, tarkemmin \kor{ensimmäisen lajin} ($J_p$)
ja \kor{toisen lajin} ($Y_p$) Besselin funktioita. (Tapauksessa $p=1/2$ nämä ovat 
poikkeuksellisesti alkeisfunktioita, ks.\ Harj.teht.\,\ref{H-dy-6: Besselin erikoistapaus}.)
Funktio $J_p$ on muotoa
\[
J_p(x)=\sum_{k=0}^\infty a_kx^{2k+p}=x^p F(x),
\]
missä potenssisarja $F(x)=\sum_{k=0}^\infty a_kx^{2k}$ suppenee kaikkialla. Muilla kuin 
kokonaislukuarvoilla on vastaavasti
\[
Y_p(x)=\sum_{k=0}^\infty a_kx^{2k-p}=x^{-p} G(x),\quad p\notin \{0,1,2,\ldots\}
\] 
missä $G$:n sarja jälleen suppenee kaikkialla. Jos $p$ on kokonaisluku, niin $Y_p$ sisältää
myös logaritmisen termin muotoa $x^{-p}H(x)\ln\abs{x}$, missä $H(x)$ on ilmaistavissa 
potenssisarjana.

Olkoon $p=0$, ja yritetään sarjaratkaisua $y(x)=\sum_{k=0}^\infty a_kx^k$.
Sijoitus yhtälöön antaa
\[
\sum_{k=2}^\infty k(k-1)a_kx^{k-2}+\sum_{k=1}^\infty ka_kx^{k-2}+\sum_{k=0}^\infty a_kx^k=0.
\]
Kun kahdessa ensimmäisessä summassa vaihdetaan summausindeksiksi $k$:n tilalle $k+2$, saadaan
\[
\sum_{k=0}^\infty (k+2)(k+1)a_{k+2}x^k+\bigl[\,a_1x^{-1}+\sum_{k=0}^\infty (k+2)a_{k+2}x^k\,\bigr] 
                                                 + \sum_{k=0}^\infty a_kx^k=0 \quad \forall x
\]
eli yhdistämällä summat
\[
a_1x^{-1}+\sum_{k=0}^\infty [(k+2)^2a_{k+2}+a_k]x^k = 0\quad\forall x.
\]
Nähdään, että tämä toteutuu, kun valitaan
\begin{align*}
a_1 &= a_3=a_5=\ldots =0 \\
a_0 &= 1,\quad a_{k+2}=-\frac{1}{(k+2)^2}\,a_k,\quad k=0,2,4 \ldots
\end{align*}
Kertoimille $a_{2k}$ saadaan (pienen mietiskelyn jälkeen) lauseke
\[
a_{2k}=(-1)^k\frac{1}{(k!)^2\cdot 4^k},\quad k=0,1,\ldots
\]
Ratkaisu, nimeltään $J_0$, on siis
\[
J_0(x)=\sum_{k=0}^\infty (-1)^k\frac{1}{(k!)^2}\left(\frac{x}{2}\right)^{2k}.
\]
Tämä sarja suppenee (ja on termeittäin derivoitavissa) kaikkialla, joten on löydetty
koko $\R$:ssä pätevä ratkaisu differentiaaliyhtölölle $xy''+y'+xy=0$. Toisen lajin ratkaisu
$Y_0(x)$ on pätevä vain erikseen väleillä $(-\infty,0)$ ja $(0,\infty)$. Tämäkin on 
ilmaistavissa potenssisarjojen avulla (ks. Harj.teht.\,\ref{H-dy-6: Besselin toinen laji}).

\Harj
\begin{enumerate}

\item
Minkä toisen kertaluvun lineaarisen ja homogeenisen differentiaaliyhtälön ratkaisuja
(jollakin avoimella välillä) ovat
a) $y_1(x)=x+1$ ja $y_2(x)=x^2$, b) $y_1(x)=x$ ja $y_2(x)=e^x$, c) $y_1(x)=x^2$ ja 
$y_2(x)=\sin x\,$?

\item
Ratkaise käyttäen annettua, yksittäisratkaisua koskevaa lisätietoa: \vspace{1mm}\newline
a) \ $xy''-(x+3)y'+y=0; \quad$ polynomi \newline
b) \ $x^2(\ln x-1)y''-xy'+y=0; \quad$ polynomi \newline
c) \ $(x^2-2x-1)y''-(2x-1)y'+2y=0; \quad$ polynomi \newline
d) \ $ y''+(\tan x-2\cot x)y'+2(\cot^2x)y=0; \quad \sin x$ \newline
e) \ $y''+2(1-\tan^2x)y=0; \quad \cos^2x$ \newline
f) \ $(x-2)y''-(4x-7)y'+(4x-6)y=0; \quad e^{ax}$

\item
Differentiaaliyhtälöllä $\,y''+P(x)y'+Q(x)y=0\,$ on helposti arvattava ei-triviaali ratkaisu,
jos $1+P(x)+Q(x)=0$. Ratkaise tämän (ja tarvittaessa toisenkin) arvauksen perusteella:
\vspace{1mm}\newline
a) \ $(x-1)y''-xy'+y=0$ \newline
b) \ $(2x-x^2)y''+(x^2-2)y'+(2-2x)y=0$ \newline
c) \ $y''-(2+2x+x^{-1})y'+(1+2x+x^{-1})y=0$ \newline
d) \ $(x+1)y''-xy'-y=(x+1)^2$ 

\item
Oletetaan, että differentiaaliyhtälön $\,y''+P(x)y'+Q(x)y=0\,$ kerroinfunktioille pätee:
$P$ on derivoituva ja $P'$ ja $Q$ ovat jatkuvia tarkasteltavalla välillä $(a,b)$. Tällöin
voidaan sijoituksella $y(x)=K(x)u(x)$ muuntaa DY muotoon
\[
u''+I(x)u=0.
\]
Miten $K(x)$ on valittava ja mikä on $I(x)$:n lauseke?

\item
Ratkaise vakioiden varioinnilla:
\vspace{1mm}\newline
a) \ $y''+y=\tan x \qquad\qquad\qquad\,$
b) \ $y''-y=1/(e^x+1)$ \newline
c) \ $y''+2y'+y=1/(x^2+1) \quad\,\ $
d) \ $x^2y''+2xy'-2y=x^2e^x$ \newline
e) \ $y''+5y'+4y=f(x) \qquad\quad\ \ $
f) \ $y''+4y'+4y=f(x)$ \newline
g) \ $y''+4y'+5y=f(x) \qquad\quad\,\ $
h) \ $x^2y''+5xy'+3y=f(x)$ \newline
i) \ $x^2y''+5xy'+4y=f(x) \qquad\ $
j) \ $x^2y''-2y=f(x)$

\item \index{Legendren polynomi}
a) \kor{Legendren polynomi} astetta $n$ määritellään
\[
P_n(x)=\frac{1}{2^n n!}\frac{d^n}{dx^n}\,(x^2-1)^n.
\]
Totea, että $P_n$ on polynomi astetta $n$ ja näytä, että tämä on Legendren 
differentiaaliyhtälön ratkaisu $n$:n arvoilla $0,1,2,3$. \vspace{1mm}\newline
b) Määritä Hermiten ja Laguerren polynomit astetta $n \le 2$, ts.\ kyseisten
differentiaaliyhtälöiden polynomiratkaisut, kun $n=0,1,2$. \vspace{1mm}\newline
Määritä\, c) Legendren,\, d) T\v{s}eby\v{s}evin,\, e) Hermiten,\, f) Laguerren
DY:n yleinen ratkaisu, kun $n=1$.

\item \label{H-dy-6: Besselin erikoistapaus}
Tapauksessa $p=1/2\,$ Besselin differentiaaliyhtälö on ratkaistavissa alkeisfunktioilla.
Määritä ratkaisu sijoituksella $\,y(x)=u(x)/\sqrt{x}$.

\item 
Seuraavien differentiaaliyhtälöiden yleinen ratkaisu on esitettävissä pisteen $x=0$ lähellä 
muodossa $y(x)=C_1\sum_{k=0}^\infty a_k x^k+C_2\sum_{k=0}^\infty b_k x^k$, missä $a_0=b_1=1$ ja
$a_1=b_0=0$. Määritä molempien potenssisarjojen neljä ensimmäistä nollasta poikkeavaa
termiä. \vspace{1.5mm}\newline
a) \ $y''+2xy'+5y=0 \quad$
b) \ $y''+2xy+6y=0 \quad$
c) \ $y''+xy'-x^2y=0 \quad$ \newline
d) \ $y''+(x+1)y'+(x-1)y=0 \quad$
e) \ $(1-x^2)y''-xy'+ay=0,\ a\in\R$

\item (*) \index{Tsebysev@T\v{s}eby\v{s}evin polynomi}
\kor{T\v{s}eby\v{s}evin polynomi} astetta $n$ määritellään välillä $[-1,1]$ kaavalla
\[
T_n(x)=\cos\,(n\Arccos x), \quad x\in[-1,1].
\]
a) Näytä, että $T_n$ on polynomi astetta $n$ (sijoita $x=\cos\theta$). \ b) Näytä, että $T_n$
on T\v{s}eby\v{s}evin differentiaaliyhtälön ratkaisu välillä $(-1,1)$. \ c) Laske polynomin
$T_5$ kertoimet ja hahmottele $T_5(x)$ graafisesti välillä $[-2,2]$. Mitkä ovat $T_5$:n
nollakohdat ja missä pisteissä on $T_5(x)=\pm 1\,$?

\item (*) \label{H-dy-6: Besselin toinen laji}
Näytä, että jos $p=0$, niin Besselin differentiaaliyhtälöllä on välillä $(0,\infty)$
(toisen lajin) ratkaisu muotoa $Y_0(x)=J_0(x)\ln x+F(x)$, missä $F$ on ilmaistavissa kaikkialla
suppenevana potenssisarjana.

\item (*)
Näytä, että differentiaaliyhtälön $y''=xy$ yleinen ratkaisu on esitettävissä potenssisarjana
muodossa
\[
y(x)=C_1 \sum_{k=0}^\infty a_k x^{3k} + C_2 \sum_{k=0}^\infty b_k x^{3k+1}, \quad a_0=b_0=1.
\]
Määrittele kerroinjonot $\seq{a_k}$ ja $\seq{b_k}$ palautuvina lukujonoina ja päättele, että
sarja suppenee kaikkialla. Millainen on yleisemmin differentiaaliyhtälön $y''=x^my\,$ yleinen
sarjaratkaisu, kun $m\in\N\,$?

\end{enumerate} %*Yleinen toisen kertaluvun lineaarinen DY
\section[Differentiaaliyhtälöiden numeeriset ratkaisumenetelmät]{Differentiaaliyhtälöiden
numeeriset \\ ratkaisumenetelmät} 
\label{DYn numeeriset menetelmät}
\sectionmark{DY:n numeeriset ratkaisumenetelmät}
\alku
\index{differentiaaliyhtälön!i@numeerinen ratkaiseminen|vahv}

Edellisissä luvuissa on kehitelty differentiaaliyhtälöiden ratkaisemistaitoa lähinnä nk.\
'tarkan' ratkaisemisen näkökulmasta. Tämä ratkaisuoppi on hyvin perinteistä, ja sillä on oma
pysyvä arvonsa. --- Etenkin silloin kun ratkaisun voi esittää alkeisfunktioden avulla 
suhteellisen yksinkertaisena lausekkeena, on perinteistä ratkaisutapaa vaikea sivuuttaa. 
Sovelluksissa törmätään kuitenkin hyvin usein differentiaaliyhtälöihin (ja etenkin 
-systeemeihin), joihin mikään tähän asti tarjotuista resepteistä ei tehoa. Silloin on tyydyttävä
etsimään ratkaisu likimäärin \kor{numeerisesti}, käytännössä tietokoneen tai tehokkaan 
laskimen avustuksella.\footnote[2]{Differentiaaliyhtälöitä ratkottiin numeerisesti jo ennen 
tietokoneiden aikaa, mutta ratkaiseminen rajoittui vain hyvin tärkeinä pidettyihin kohteisiin,
kuten tykinammusten lentoratojen määrittämiseen. Tietokoneiden aikakaudella 
differentiaaliyhtälöiden numeerisesta ratkaisemisesta on tullut arkipäivää mitä erilaisimmilla
tieteen ja tekniikan aloilla. Samalla on ratkaisumenetelmistä kehittynyt laaja menetelmätiede,
josta on kirjoitettu paksuja kirjoja.} Numeerinen ratkaiseminen siis poikkeaa melkoisesti
perinteisistä menetelmistä, joilla ratkaisu yleensä saadaan (jos saadaan) 'käsipelillä'. 
Mitään periaatteellista eroa ei numeerisen ja 'tarkan' ratkaisemisen välillä silti ole, sillä
myös numeerisin keinoin saadaan tarkka ratkaisu, kun ajatellaan algoritmin tuottama 
likimääräisratkaisujen jono 'loppuun asti' lasketuksi. --- Itse asiassa juuri näin meneteltiin
Luvussa \ref{määrätty integraali}, kun etsittiin yleispätevää ratkaisutapaa 
differentiaaliyhtälölle $y'=f(x)$.

Jatkossa tarkastellaan lyhyesti \pain{alkuarvotehtävien} numeerisessa ratkaisussa käytettävien
\index{askelmenetelmä}%
nk.\ \kor{askelmenetelmien} perusiedoita. Otetaan tarkastelun kohteeksi jälleen ensimmäisen
kertaluvun normaalimuotoinen alkuarvotehtävä
\begin{equation} \label{dy-7: ivp}
\begin{cases}
\,y'=f(x,y),\quad x>x_0, \\
\,y(x_0)=y_0.
\end{cases}
\end{equation}
Kaikissa askelmenetelmissä on ideana lähteä alkuarvopisteestä $x_0$ ja edetä lyhyin 
\pain{askelin}, ensin pisteeseen $x_1$, sitten pisteeseen $x_2$, jne. Jokaisella askeleella
$x_k \kohti x_{k+1}$ lasketaan luku $y_{k+1}$, joka pyrkii olemaan likiarvo tarkan ratkaisun
$y(x)$ arvolle pisteessä $x_{k+1}$:
\[
y_{k+1}\approx y(x_{k+1}),\quad k=0,1,\ldots
\]
Lukua $y_{k+1}$ laskettaessa tunnetaan jo luvut $y_0,y_1, \ldots y_{k}$ (alkuehdosta ja 
aikaisemmilta askelilta), joten $y_{k+1}$ voidaan laskea p\pain{alautuvasti}. Jos laskukaavassa
esiintyy vain edeltävä luku $y_k$, on kyseessä 
\index{yksiaskelmenetelmä} \index{moniaskelmenetelmä}%
\kor{yksiaskelmenetetelmä}, muussa tapauksessa \kor{moniaskelmenetelmä}.
Erilaiset askelmenetelmät erottaa siis lopulta toisistaan vain se,
millaista palautuskaavaa tai yleisempää algoritmia luvun $y_{k+1}$ laskemisessa käytetään.
Tarkkuutta voidaan säädellä sekä menetelmän valinnalla että
\index{askelpituus}%
\kor{askelpituuksilla} $x_{k+1}-x_k$ (vrt.\ numeerisen integroinnin menetelmät Luvussa
\ref{numeerinen integrointi}). Seuraavassa tarkastellaan ensin askelmenetelmistä kaikkein
yksinkertaisinta --- myös vanhinta.

\subsection{Eulerin menetelmä}
\index{Eulerin!d@menetelmä|vahv}

\kor{Eulerin menetelmä}\footnote[2]{Sveitsiläinen matemaatikko \hist{Leonhard Euler} 
(1707-1783) on entuudestaan tuttu jo kompleksisesta eksponenttifunktiosta (Eulerin kaava) ja
hänen mukaansa nimetystä differentiaaliyhtälöstä. \kor{Eulerin yhtälöt} (vallan toisessa 
merkityksessä) ovat keskeisiä matematiikan lajissa nimeltä \kor{variaatiolaskenta} ja 
(jälleen toisessa merkityksessä) virtausmekaniikassa. Euler tutki aikansa matematiikkaa ja 
fysiikkaa hyvin laajasti, myös tähtitiedettä, musiikkia, merenkulkua, ym. Euler oli 1700-luvun
merkittävimpiä ja kaikkien aikojen monipuolisimpia matemaatikkoja. \index{Euler, L.|av}} on
yksiaskelmenetelmä, jossa luku $y_{k+1}$ lasketaan palautuskaavasta 
\[
\boxed{\kehys\quad y_{k+1}=y_k+hf(x_k,y_k),\quad h=x_{k+1}-x_k, \quad k=0,1,\ldots \quad}
\]
Kaavan voi johtaa esim.\ seuraavasti: Integroidaan differentiaaliyhtälö $y'=f(x,y)$ puolittain
välin $[x_k,\,x_{k+1}]$ yli, jolloin seuraa
\[
y(x_{k+1})=y(x_k)+\int_{x_k}^{x_{k+1}} f(x,y(x))dx.
\]
Oletetaan, että $f(x,y(x))=y'(x)$ on likimain vakio välillä $[x_k,x_{k+1}]$ ja tehdään tähän
nojaten approksimaatio
\[
\int_{x_k}^{x_{k+1}} f(x,y(x))dx \approx hf(x_k,y(x_k)), \quad h=x_{k+1}-x_k.
\]
(Kyse on yksinkertaisesta numeerisesta integroinnista, vrt.\ Luku 
\ref{numeerinen integrointi}.) Eulerin menetelmään päädytään, kun tulos
\begin{align*}
&y_{k+1} \approx y_k+hf(x_k,y_k), \quad y_k=y(x_k),\ y_{k+1}=y(x_{k+1})
\intertext{'murjotaan' muotoon}
&y_{k+1} = y_k+hf(x_k,y_k), \quad y_k \approx y(x_k),\ y_{k+1} \approx y(x_{k+1}).
\end{align*}
Jos askelpituus $h$ oletetaan tunnetuksi (esim.\ vakio), niin Eulerin menetelmän askel koostuu
yhdestä funktioevaluaatiosta, yhdestä kertolaskusta ja kahdesta yhteenlaskusta:
\[
x_k=x_{k-1}+h, \quad t_k=f(x_k,y_k), \quad y_{k+1}=y_k+ht_k.
\]
\begin{Exa} Laske Neperin luvulle $e$ likiarvo $y_n$ ratkaisemalla alkuarvotehtävä 
$y'=y,\ y(0)=1$ Eulerin menetelmällä ja vakioaskelpituudella $h=1/n,\ n\in\N$. Arvioi virhe.
\end{Exa}
\ratk Eulerin menetelmällä laskien saadaan
\begin{align*}
y_0=1, \quad &y_k=y_{k-1}+hy_{k-1}=(1+h)y_{k-1}, \quad k=1,2 \ldots \\
      \qimpl &y_n = (1+h)^n = \underline{\underline{\left(1+\frac{1}{n}\right)^n}}.
\intertext{Virheen arvioimiseksi tarkastellaan lauseketta}
             &\ln\frac{y_n}{e} = n\,\ln\left(1+\frac{1}{n}\right)-1.
\end{align*}
Taylorin lauseen (ks.\ Luku \ref{taylorin lause}) mukaan on 
$\,\ln(1+x)=x-\frac{1}{2}x^2+\ordoO{\abs{x}^3}$, joten seuraa
\begin{align*}
\ln\frac{y_n}{e}\,=\,-\frac{1}{2n}+\ordoO{n^{-2}} 
                  &\qimpl \frac{y_n}{e}\,=\,e^{-\frac{1}{2n}+\ordoO{n^{-2}}}
                                       \,=\,1-\frac{1}{2n}+\ordoO{n^{-2}} \\
                  &\qimpl e-y_n = \underline{\underline{\frac{e}{2n}+\ordoO{n^{-2}}}}. \loppu
\end{align*}

\subsection{Eulerin menetelmän virhe}

Edellä esimerkissä saatu virhearvio $\,y_n-y(x_n)=\ordoO{h}\,$ pätee Eulerin menetelmälle melko
yleisin edellytyksin. Yleisemmän virhearvion johtamiseksi olkoon askelpituus $h=$ vakio ja 
kirjoitetaan $Y_k=y(x_k)=y(x_0+kh)$, missä $y(x)$ on alkuarvotehtävän \eqref{dy-7: ivp} 
\index{konsistenssivirhe}%
ratkaisu. Arvioidaan ensin Eulerin menetelmän nk.\ \kor{konsistenssivirhe}, eli luku $\delta_k$
yhtälössä
\[
Y_{k+1}=Y_k+hf(x_k,Y_k)+\delta_k.
\]
Kun merkitään $\,u(x)=f(x,y(x))=y'(x)$, niin osittain integroimalla seuraa (ks.\ myös Eulerin
menetelmän johto edellä)
\begin{align*}
\delta_k &= \int_{x_k}^{x_{k+1}} u(x)\,dx - hu(x_k) \\
         &=\sijoitus{x_k}{x_{k+1}} (x-x_{k+1})u(x) - \int_{x_k}^{x_{k+1}} (x-x_{k+1})u'(x)\,dx
                                                   - hu(x_k) \\
         &= \int_{x_k}^{x_{k+1}} (x_{k+1}-x)y''(x)\,dx.
\end{align*}
Oltetetaan nyt (Oletus \#1), että $y$ on kahdesti jatkuvasti derivoituva ja pätee arvio
$\abs{y''(x)} \le M$ (kyseeseen tulevilla $x$:n arvoilla). Tällöin konsistenssivirhe on enintään
\[
\abs{\delta_k} \le \int_{x_k}^{x_{k+1}} (x_{k+1}-x)\,M\,dx = \frac{1}{2}\,Mh^2.
\]
Käyttäen tätä arviota ja kolmioepäyhtälöä voidaan nyt päätellä:
\begin{align*}
&\begin{cases} 
\, Y_{k+1}=Y_k+hf(x_k,Y_k)+\delta_k, \\ \,y_{k+1}=y_k+hf(x_k,y_k)
 \end{cases} \\
&\qimpl \abs{Y_{k+1}-y_{k+1}} \le \abs{Y_k-y_k}+h\abs{f(x_k,Y_k)-f(x_k,y_k)}+\frac{1}{2}\,Mh^2.
\end{align*}
Oletetaan (Oletus \#2), että oikealla puolella voidaan käyttää arviota
\[
\abs{f(x,y_1)-f(x,y_2)} \le L\abs{y_1-y_2}
\]
($x=x_k,\,y_1=Y_k,\,y_2=y_k$), missä $L$ on vakio. Tällöin seuraa
\[
\abs{Y_{k+1}-y_{k+1}} \le (1+hL)\,\abs{Y_k-y_k} + \frac{1}{2}\,Mh^2.
\]
Soveltamalla tätä epäyhtälöä palautuvasti, kun $k=0,1 \ldots\,$ seuraa (induktio!)
\[
\abs{Y_n-y_n}\,\le\,\frac{1}{2}\,Mh^2 \sum_{k=0}^{n-1} (1+hL)^k\,
                 =\,\frac{M}{2L}\bigl[(1+hL)^n-1\bigr]\,h, \quad n=1,2 \ldots
\]
Käyttämällä tässä vielä arviota 
$\,(1+hL)^n \le \bigl(e^{hL}\bigr)^n = e^{Lnh} = e^{L(x_n-x_0)}\,$ saadaan Eulerin menetelmän 
virhearvioksi tehdyin oletuksin
\begin{equation} \label{Eulerin virhe}
\boxed{\quad
\abs{y(x_n)-y_n} \le \frac{M}{2L}\bigl(e^{L(x_n-x_0)}-1\bigr)\,h, \quad x_n=x_0+nh. \quad}
\end{equation}
\begin{Exa} Alkuarvotehtävä
\[
\begin{cases} \,y'=e^{-x}+x\sin y, \quad x>0, \\ \,y(0)=1 \end{cases}
\]
ei ratkea perinteisin menetelmin, joten on tyytyminen numeeriseen ratkaisuun.
Differentiaaliyhtälöstä nähdään, että ratkaisulle pätee $\abs{y'(x)} \le 1+x$, kun $x \ge 0$.
Derivoimalla implisiittisesti ja käyttämällä tätä arviota seuraa
\[
y''(x)=-e^{-x}+\sin y+xy'\cos y \qimpl \abs{y''(x)} \le 2+x+x^2,\,\ x \ge 0.
\]
Koska edelleen on
\[
\abs{f(x,y_1)-f(x,y_2)}\,=\,x\abs{\sin y_1-\sin y_2}\,\le\,x\abs{y_1-y_2}, \quad x \ge 0,
\]
niin nähdään, että Eulerin menetelmän virhearvio \eqref{Eulerin virhe} on pätevä, kun arviossa
asetetaan
\[
L=x_n=nh, \quad M=2+x_n+x_n^2 \quad (n \ge 1).
\]
Esimerkiksi jos halutaan ratkaisu välillä $[0,2]$, niin voidaan valita $L=2,\ M=8$, jolloin
virhe on enintään
\[
\abs{y(x_n)-y_n} < 2e^4 h < 110 h \quad (x_n \le 2\,).
\]
Todennäköisesti virhe on huomattavasti tätä arviota pienempi. \loppu
\end{Exa}

\subsection{Differentiaaliyhtälösysteemit}
\index{differentiaaliyhtälö!g@--systeemi|vahv}

Eulerin menetelmän, samoin kuin muidenkin askelmenetetelmien (ks.\ esimerkit jäljempänä) vahva
puoli sovellusten kannalta on, että menetelmät soveltuvat sellaisenaan myös normaalimuotoisen 
differentiaaliyhtälösysteemin ratkaisuun, sikäli kuin kyseessä on alkuarvotehtävä. Myös 
korkeamman kertaluvun differentiaaliyhtälöiden alkuarvotehtävät ratkeavat tällä tavoin 
kirjoittamalla tehtävä ensin systeemimuotoon, vrt.\ Luku \ref{DY-käsitteet}. Esimerkki
valaiskoon asiaa.
\begin{Exa} Esitä Eulerin menetelmään perustuva algoritmi, jolla voidaan ratkaista numeerisesti
alkuarvotehtävä
\[
\begin{cases} \,y'''=x+yy'y'', \quad x>0, \\ \,y(0)=1,\ y'(0)=y''(0)=0. \end{cases}
\]
\end{Exa}
\ratk Alkuehdoista ja ratkaistavan differentiaaliyhtälön systeemimuodosta
\[
\begin{cases} \,y'=u, \\ \,u'=v, \\ \,v'=x+yuv \end{cases}
\]
saadaan algoritmiksi
\[
y_0=1,\ u_0=v_0=0, \quad
\begin{cases}
\,y_{k+1}=y_k+hu_k, \\ 
\,u_{k+1}=u_k+hv_k, \\ 
\,v_{k+1}=v_k+h(x_k+y_ku_kv_k), \quad k=0,1 \ldots \loppu
\end{cases}
\]

\subsection{*Muita askelmenetelmiä}
\index{askelmenetelmä|vahv}

Tavallisten differentiaaliyhtälöiden alkuarvotehtävissä käytettävien 'mahdollisten menetelmien
luettelo' on nykyisin tavattoman pitkä. Seuraavassa vain muutamia esimerkkejä usein käytetyistä
tai muuten hyvin tunnetuista askelmenetelmistä.
\index{implisiittinen Euler} \index{trapetsi} \index{keskipistesääntö}
\index{klassinen Runge-Kutta} \index{Runge-Kutta (klassinen)}%
\vspace{5mm}\newline
\underline{Implisiittinen Euler} \vspace{2mm}\newline
$y_{k+1}=y_k+hf(x_{k+1},y_{k+1})$ \vspace{5mm}\newline
\underline{Trapetsi} \vspace{2mm}\newline
$y_{k+1}=y_k+\frac{h}{2}\bigl[f(x_k,y_k)+f(x_{k+1},y_{k+1})\bigr]$ \vspace{5mm}\newline
\underline{Keskipistesääntö} \vspace{2mm}\newline
$y_{k+1}=y_k+hf\left(x_k+\frac{1}{2}h,\,\frac{1}{2}y_k+\frac{1}{2}y_{k+1}\right)$
\vspace{5mm}\newline
\underline{BDF-2} \vspace{2mm}\newline
$y_{k+1}=\frac{4}{3}y_k-\frac{1}{3}y_{k-1}+\frac{2}{3}hf(x_{k+1},y_{k+1})$ \vspace{5mm}\newline
\underline{BDF-3} \vspace{2mm}\newline
$y_{k+1}=\frac{18}{11}y_k-\frac{9}{11}y_{k-1}+\frac{2}{11}y_{k-2}
                         +\frac{6}{11}hf(x_{k+1},y_{k+1})$ \vspace{5mm}\newline
\underline{BDF-4} \vspace{2mm}\newline
$y_{k+1}=\frac{48}{25}y_k-\frac{36}{25}y_{k-1}+\frac{16}{25}y_{k-2}-\frac{3}{25}y_{k-3}
                         +\frac{12}{25}hf(x_{k+1},y_{k+1})$ \vspace{5mm}\newline
\underline{Klassinen Runge-Kutta} \vspace{2mm}\newline
$y_{k+1}=y_k+\frac{h}{6}\bigl(V_1+2V_2+2V_3+V_4\bigr),$ \vspace{2mm}\newline
$\phantom{y_{k+1}=\,} 
V_1=f(x_k,y_k),\ \ 
V_2=f\left(x_k+\frac{1}{2}h,y_k+\frac{1}{2}hV_1\right)$ \vspace{2mm}\newline
$\phantom{y_{k+1}=\,}
V_3=f\left(x_k+\frac{1}{2}h,y_k+\frac{1}{2}hV_2\right),\ \ V_4=f\left(x_k+h,y_k+hV_3\right)$
\vspace{5mm}\newline
\index{kertaluku!b@tarkkuuden}
Näistä implisiittinen (samoin kuin tavallinen) Eulerin menetelmä on \kor{ensimmäisen kertaluvun}
menetelmä, eli virhe on suotuisissa oloissa luokkaa $\ordoO{h}$. Trapetsi, keskipistesääntö ja
BDF-2 ovat \kor{toisen kertaluvun} menetelmiä (virhe $\ordoO{h^2}$ riittävän säännöllisille 
ratkaisuille), BDF-3 on kertalukua $3$ ja BDF-4 ja klassinen Runge-Kutta molemmat kertalukua 
$4$. Esimerkeistä BDF-2, BDF-3 ja BDF-4 edustavat \pain{moniaskelmenetelmiä} (askelpituus 
oletettu vakioksi). Näiden käynnistämiseksi on laskettava ensin erillisellä algoritmilla 
$y_1 \ldots y_{m-1}$ (BDF-$m$). --- Nimi BDF on lyhenne sanoista Backward Differentiation
Formula (ks. Harj.Teht.\,\ref{H-dy-7: BDF}). 

\index{implisiittinen (askel)menetelmä}%
Kaikissa esimerkeissä, viimeistä lukuun ottamatta, on kyse \kor{implisiittisestä} menetelmästä.
Tällä tarkoitetaan, että laskettava luku $y_{k+1}$ esiintyy laskukaavassa myös funktion $f$ 
'alla'. Tällöin lukua ei saada selville pelkästään funktioevaluaatioilla, vaan on ratkaistava
yhtälö (systeemin tapauksessa yhtälöryhmä) muotoa $y_{k+1}=F(y_{k+1})$. Käytännössä tämä 
ratkaistaan likimäärin, esimerkiksi muutamalla (1--3) Newtonin iteraatiolla tai 
yksinkertaisemmin kiintopisteiteraatiolla: 
\[
t_{j+1}=F(t_j), \quad j=0 \ldots m\,; \quad y_{k+1}=t_{m+1}.
\]
(Vrt.\ Luku \ref{kiintopisteiteraatio} --- yhtälöryhmien iteratiivisista ratkaisumenetelmistä
tulee puhe myöhemmin.) Iteraation alkuarvoksi voidaan ottaa esim.\ Eulerin menetelmän antama
$\,t_0=y_k+hf(x_k,y_k)$. 

Implisiittiset menetelmät kuten implisiittinen Euler ja BDF-$m$ ovat differentiaaliyhtälöiden
'black box'-ratkaisijoina hyvin suosittuja, syystä että ne ovat vaihtelevissa tilanteissa 
suhteellisen varmoja. --- Todettakoon tässä ainoastaan, että varmuus tulee esiin erityisesti
sellaisissa nk.\
\index{kankeus (DY:n)}%
\kor{kankeissa} (engl.\ stiff) tehtävissä, joissa ratkaisu sisältää nopeasti
(eksponentiaalisesti) vaimenevia transientteja. Tällaisessa tehtävässä esimerkiksi 
implisiittinen Euler on tavallista Eulerin menetelmää huomattavasti varmempi vaihtoehto, vaikka
menetelmien muodollisessa tarkkuudessa ei ole eroa (ks.\ Harj.teht.\,\ref{H-dy-7: kankea DY}).
Sen sijaan tehtävissä, joissa mainitun tyyppisiä erityisongelmia ei esiinny, ovat 
yksinkertaisemmat
\index{eksplisiittinen (askel)menetelmä}%
\kor{eksplisiittiset} menetelmät (kuten Euler ja klassinen Runge-Kutta)
edelleen varteenotettavia vaihtoehtoja ja myös käytössä.

\Harj
\begin{enumerate}

\item \label{H-dy-7: Euler-kokeita} 
Laske seuraavissa alkuarvotehtävissä Eulerin menetelmällä likiarvot luvuille 
$\,y(x_k),\ k=1 \ldots 5$, missä $x_k=k/5$ (askelpituus $h=0.2$). Vertaa tarkkaan ratkaisuun!
\[
\text{a)}\ \ \begin{cases} \,y'=-y^2, \\ \,y(0)=1 \end{cases}
\text{b)}\ \ \begin{cases} \,y'=y^2, \\ \,y(0)=1 \end{cases}
\text{c)}\ \ \begin{cases} \,y'=-2xy^2, \\ \,y(0)=1 \end{cases}
\text{d)}\ \ \begin{cases} \,y'=2xy^2, \\ \,y(0)=1 \end{cases}
\]

\item
Alkuarvotehtävä
\[
y(0)=1, \quad y'=\frac{x}{\sqrt{y^2+1}}+\frac{2x^2}{x^2+1}
\]
ratkaistaan numeerisesti välillä $[0,10]$ käyttäen Eulerin menetelmää ja vakioaskelpituutta $h$.
Arvioi vakiot $L$ ja $M$ Eulerin menetelmän virhearviossa ko.\ välillä. Millä askelpituuksilla
voidaan taata, että numeerisen ratkaisun suhteellinen virhe ei ylitä yhdessäkään
laskentapisteessä arvoa $10^{-4}\,$?

\item
Alkuarvotehtävä $y'=-1/y,\ y(0)=3$ ratkaistaan numeerisesti välillä $[0,1]$ käyttäen
askelpituutta $h=0.2$. Laske seuraavilla menetelmillä approksimaatio $y_5$ luvulle
$y(1)\,$: Euler, Trapetsi, BDF-2, BDF-3, Klassinen Runge-Kutta. Vertaa tarkkaan arvoon!

\item
Alkuarvotehtävässä $y'=f(x),\ y(0)=0$ klassinen Runge-Kutta-menetelmä pelkistyy erääksi
tunnetuksi numeerisen integroinnin säännöksi. Mistä säännöstä on kyse?

\item \label{H-dy-7: BDF}
Differentiaaliyhtälön $y'=f(x,y)$ ratkaisumenetelmä BDF-$m$ johdetaan kirjoittamalla yhtälö
pisteessä $x=x_{k+1}$, tekemällä approksimaatio
\[
y'(x_{k+1}) \approx \frac{1}{h} \sum_{j=0}^m \alpha_j y_{k+1-j}, \quad y_i=y(x_i) 
\]
ja vaatimalla, että tämä on tarkka polynomeille astetta $\le m$. Tarkista laskemalla kertoimet
$\alpha_j$, että tuloksena on todella menetelmä BDF-$2$, kun $m=2$. Millainen menetelmä on 
BDF-$1\,$?

\item \label{H-dy-7: kankea DY}
Olkoon $\tau = 0.02$. Halutaan ratkaista numeerisesti alkuarvotehtävä
\[
\begin{cases} \,\tau y'+y=\sin x, \quad x>0, \\ \,y(0)=0.1. \end{cases}
\]
Vertaa tavallisen ja implisiittisen Eulerin menetelmän antamia ratkaisuja laskemalla molemmilla
menetelmillä $\,y_k \approx y(kh),\ k=1 \ldots 5\,$ askelpituuksilla $h=0.01$, $h=0.04$ ja 
$h=0.08$. Hahmottele kaikki ratkaisut graafisesti ja vertaa myös tarkkaan ratkaisuun! 
(Ks.\ sovellusesimerkki Luvussa \ref{lineaarinen 1. kertaluvun DY}.)

\item
Tasossa liikkuvaan avaruusalukseen vaikuttaa origossa olevan maapallon vetovoima
$\vec F=-(k/r^2)\,\vec e_r$, missä $k$ on vakio (napakoordinaatisto). Hetkellä $t$ alus on 
paikassa $(x(t),y(t))$ ja sen nopeus on $u(t)\vec i+v(t)\vec j$. Kirjoita aluksen liikeyhtälö
$m\vec r\,''=\vec F$ karteesisessa koordinaatistossa normaalimuotoiseksi
differentiaaliyhtälösysteemiksi ja esitä Eulerin menetelmään perustuva algoritmi yhtälöiden
numeeriseksi ratkaisemiseksi. Oletetaan tunnetuksi alkuarvot $x_0,y_0,u_0,v_0$ hetkellä $t=0$.

\item (*)
Ratkaistaessa Eulerin menetelmällä alkuarvotehtävää
\[
\begin{cases} \,y'=\dfrac{y^2}{x+y+1}+4x+2, \quad x>0, \\[3mm] \,y(0)=A \end{cases}
\]
askelpituudella $h=0.1$ havaitaan yllättäen, että saadaan tarkka ratkaisu, ts.\
$y_k=y(x_k)\ \forall k\ (x_k=kh)$. Mikä on $A$:n arvo?

\end{enumerate} %Diferentiaaliyhtälöiden numeeriset ratkaisumenetelmät
\section{*Picardin--Lindelöfin lause} \label{Picard-Lindelöfin lause}
\alku

Palataan tutkimaan ensimmäisen kertaluvun alkuarvotehtävää
\begin{equation} \label{Picard-1}
\begin{cases} \,y'=f(x,y), \\ \,y(x_0)=y_0 \end{cases}
\end{equation}
tällä kertaa teoreettisemmalta kannalta. Asetetaan kysymys: Millä, funktiolle $f$ ja luvuille
$x_0,y_0$, asetettavilla ehdoilla voidaan taata, että alkuarvotehtävällä on yksikäsitteinen
ratkaisu pisteen $x_0$ ympäristössä? Vastauksen antaa seuraava kuuluisa lause.
\begin{*Lause} \label{picard-lindelöf} \index{Picardin--Lindelöfin lause|emph}
\index{differentiaaliyhtälön!h@ratkeavuus|emph}%
\vahv{(Picard--Lindelöf\,\footnote[2]{Picardin--Lindelöfin lauseen alkuperäisversion esitti
ja todisti tässä esitettyä yleisemmässä muodossa ranskalainen matemaatikko \hist{Emile Picard}
(1853-1941) vuonna 1893. Suomalainen matemaatikko \hist{Ernst Lindelöf} (1870-1946) tarkensi
tulosta hiukan myöhemmin. Lauseen väittämän jälkimmäinen osa (koskien ratkeavuutta koko
tarkasteltavalla välillä) on Lindelöfin käsialaa.
\index{Picard, E.|av} \index{Lindelöf, E.|av}})} Olkoon $T=[a,b]\times [c,d]\subset\Rkaksi$ ja
oletetaan, että $f:T\kohti\R$ toteuttaa ehdot
\begin{itemize}
\item[(i)]  Funktio $x \map f(x,y)$ on jatkuva välillä $[a,b]$ jokaisella $y\in[c,d]$.
\item[(ii)] Funktio $y \map f(x,y)$ on jokaisella $x\in[a,b]$ Lipschitz-jatkuva välillä $[c,d]$
            vakiolla, joka ei riipu $x$:stä, ts.\ $\exists L\in\R_+$ siten, että pätee
            \[
            |f(x,y_1)-f(x,y_2)| \le L\abs{y_1-y_2} \quad\forall (x,y_1)\in T, \ (x,y_2)\in T.
            \]
\end{itemize}
Tällöin, jos $(x_0,y_0)\in(a,b)\times(c,d)$, niin on olemassa $\delta>0$ siten, että
alkuarvotehtävällä
\[
\begin{cases} \,y'=f(x,y),\quad x\in (x_0-\delta,x_0+\delta), \\ \,y(x_0)=y_0 \end{cases}
\]
on yksikäsitteinen ratkaisu. Edelleen jos oletukset ovat voimassa, kun välin $[c,d]$ tilalla
on $\R$ ($T=[a,b]\times\R$), niin ratkaisu on olemassa ja yksikäsitteinen koko avoimella
välillä $(a,b)$, kun $(x_0,y_0)\in(a,b)\times\R$.
\end{*Lause}

Seuraavassa käydään lyhyesti läpi Lauseen \ref{picard-lindelöf} todistukset ideat ja
päävaiheet. (Todistuksen tarkkaan läpiviemiseen ovat käsitteelliset eväämme hieman vajavaiset.)
Lähtökohtana on ajatus, että alkuarvotehtävän ratkaisu konstruoidaan nk.\ 
\index{kiintopisteiteraatio!Picardin iteraatio|(} \index{Picardin iteraatio|(}%
\kor{Picardin iteraatiolla}. Tässä puolestaan idea on seuraava: Kun ollaan lähellä pistettä
$x_0$, niin voidaan olettaa, että $y(x)$ on lähellä $y_0$:aa. Ratkaisun ensimmäinen
approksimaatio on tämän mukaan vakio:
\[
y(x)\approx y_0(x)=y_0.
\]

Parannetaan tätä nyt asteittain ratkaisemalla palautuvasti alkuarvotehtävät
\[
\begin{cases}
\,y_k'(x)=f(x,y_{k-1}(x)), \\
\,y_k(x_0)=y_0,\quad k=1,2,\ldots
\end{cases}
\]
Koska tässä $y_{k-1}$ tunnetaan aikaisemmilta iteraatiokierroksilta (tai alkuehdosta), saadaan
jokainen $y_k$ määrätyksi palautuvasti tunnetun funktion integraalina:
\begin{equation} \label{Picard-2}
y_0(x) = y_0, \quad y_k(x) = y_0+\int_{x_0}^x f(t,y_{k-1}(t))dt,\quad k=1,2,\ldots
\end{equation}
Näin on määritelty Picardin iteraatio. --- Huomattakoon, että tässä on itse asiassa kyse
\pain{kiinto}p\pain{isteiteraatiosta}, missä 'pisteen' sijasta etsitään funktiota (vrt.\ Luku
\ref{kiintopisteiteraatio}). Iteraatiota sovelletaan alkuarvotehtävän integroituun muotoon
\[ 
y(x) = y_0+\int_{x_0}^x f(t,y(t))\,dt,
\]
\index{integraaliyhtälö}%
eli Picardin iteraatiossa on kyse tämän \kor{integraaliyhtälön} iteratiivisesta 
ratkaisemisesta. --- Katkaisemalla iteraatio saadaan likimääräinen ratkaisu, kuten
kiintopisteiteraatioissa yleensä.

Picardin iteraatiota voi siis pitää alkuarvotehtävän likimääräisen ratkaisemisen menetelmänä.
Toisin kuin edellisessä luvussa tarkastellut numeeriset menetelmät, Picardin iteraatio on
kuitenkin lähinnä 'ajattelumenetelmä', sillä käytännössä iteraatioon sisältyvät integraalit
tulevat yleensä nopeasti hankaliksi.
\begin{Exa} Laske kaksi ensimmäistä Picardin iteraattia $\,(y_k(x), \ k=1,2\,)$
alkuarvotehtävälle
\[
\begin{cases}
\,y'=x+y^2, \\ \,y(0)=1.
\end{cases}
\]
\end{Exa}
\ratk Tässä on $f(x,y)=x+y^2$, $x_0=0$, $y_0=1$, joten $y_0(x)=1$ ja
\begin{align*}
y_1(x) &= 1+\int_0^x (t+1)\,dt=\underline{\underline{1+x+\frac{1}{2}x^2}}, \\
y_2(x) &= 1+\int_0^x \left[t+\left(1+t+\frac{1}{2}t^2\right)^2\right]dt \\
&= \underline{\underline{1+x+\frac{3}{2}x^2+\frac{2}{3}x^3+\frac{1}{4}x^4+\frac{1}{20}x^5}}.
\end{align*}
Seuraavaa iteraatti $y_3(x)$ olisi jo polynomi astetta $11$. \loppu

Palataan Lauseen \ref{picard-lindelöf} todistukseen, jossa siis ideana on konstruoida
alkuarvotehtävän ratkaisu Picardin iteraatiolla \eqref{Picard-2}. Tarkoituksena on tällöin
osoittaa, että iteraatio suppenee kohti välillä $[x_0-\delta,x_0+\delta]$ jatkuvaa funktiota
$y$, joka toteuttaa
\begin{equation} \label{Picard-3}
y(x)=y_0+\int_{x_0}^x f(t,y(t))dt,\quad x\in [x_0-\delta,x_0+\delta].
\end{equation}
Parametrin $\delta$ valinnalla ($\delta$ riittävän pieni) taataan, että 
$[x_0-\delta,x_0+\delta]\subset[a,b]$ ja että $y(x)$ ei 'karkaa' väliltä $[c,d]$, ts.\ 
$\,(x,y(x)) \in T\ \ \forall x \in [x_0-\delta,x_0+\delta]$ (vrt.\ vastaava ajatus Lauseen 
\ref{separoituvan DY:n ratkaisu} todistuksessa). Tällöin oletuksen (i) mukaan yhdistetty 
funktio $x \map f(x,y(x))$ on jatkuva välillä $[x_0-\delta,x_0+\delta]$, jolloin yhtälöstä 
\eqref{Picard-3} ja Analyysin peruslauseesta seuraa, että $y$ on jatkuvasti derivoituva välillä 
$[x_0-\delta,x_0+\delta]$. Tällöin $y$ on myös alkuarvotehtävän \eqref{Picard-1} ratkaisu 
välillä $(x_0-\delta,x_0+\delta)$.

Iteraation \eqref{Picard-2} suppenemisen toteamiseksi otetaan käyttöön välillä 
$[x_0-\delta,x_0+\delta]$ määriteltyä jatkuvaa funktiota mittaava \kor{maksiminormi}
\index{maksiminormib@maksiminormi (funktion)} \index{normi!xb@maksiminormi (funktion)}%
\[ 
\norm{y} = \max\,\{\,\abs{y(x)} \mid  x_0-\delta \le x \le x_0+\delta\,\}. 
\]
Tälle (niinkuin normeille yleensä, vrt.\ Luku \ref{abstrakti skalaaritulo}) pätee 
\pain{kolmioe}p\pain{ä}y\pain{htälö}
\[
\norm{y_1+y_2} \le \norm{y_1}+\norm{y_2}.
\]

Lähdetään nyt Lauseen \ref{picard-lindelöf} oletuksesta (ii) ja arvioidaan
\begin{align*}
\abs{y_{k+1}(x)-y_k(x)}\ 
     &=\ \left| \int_{x_0}^x [f(t,y_k(t))-f(t,y_{k-1}(t))]\,dt \right| \\ 
     &\le\ \left| \int_{x_0}^x L \abs{y_k(t)-y_{k-1}(t)}\,dt \right| \\[2mm]
     &\le\ L \abs{x-x_0} \norm{y_k-y_{k-1}} \\[4mm]
     &\le\ L\delta\,\norm{y_k-y_{k-1}}, \quad x \in [x_0-\delta,x_0+\delta], \quad k=1,2, \ldots
\end{align*}  
Tulos on kirjoitettavissa
\[
\norm{y_{k+1}-y_k}\ \le\ L\delta\,\norm{y_k-y_{k-1}}, \quad k=1,2,\ldots
\]
Merkitään nyt $q=L\delta$, ja oletetaan jatkossa, että $q < 1$ (järjestettävissä valitsemalla
$\delta<1/L$). Käyttämällä saatua arviota palautuvasti seuraa
\[
\norm{y_{k+1}-y_k}\ \le\ q^k\norm{y_1-y_0}, \quad k=1,2,\ldots
\]
Oletuksen (i) mukaan funtkio $g(x)=f(x,y_0)$ on jatkuva välillä $[a,b]$, joten iteraatiokaavan
\eqref{Picard-2} perusteella
\[
\norm{y_1-y_0} = \max_{x\in[x_0-\delta,x_0+\delta]}\,\Bigl|\int_{x_0}^x f(t,y_0)\,dt\Bigr| 
                 \le C\delta,
\]
missä $C$ on $\abs{g}$:n maksimiarvo välillä $[a,b]$. Yhdistämällä saadut arviot ja käyttämällä
kolmioepäyhtälöä seuraa
\begin{align*}
\norm{y_n-y_k}\ &=\ \norm{(y_n-y_{n-1})+(y_{n-1}-y_{n-2})+ \ldots + (y_{k+1}-y_k)} \\[3mm]
                &\le\ \norm{y_n-y_{n-1}}+ \norm{y_{n-1}-y_{n-2}} + \ldots 
                                                                 + \norm{y_{k+1}-y_k} \\[3mm]
                &\le\ (q^{n-1}+q^{n-2} + \ldots + q^k)\,C\delta \\
                &<\ C\delta q^k\sum_{i=0}^\infty q^i\,
                 =\,\frac{C\delta}{1-q}\,q^k, \quad 0 \le k < n.
\end{align*}
Tästä voidaan päätellä ensinnäkin, että jos $\delta$ on riittävän pieni, niin
$\norm{y_n-y_0} \le C\delta/(1-\delta) \le \min\{y_0-c,\,d-y_0\}\ \forall n$, jolloin
$y_n(x)$:n pako väliltä $[c,d]$ on estetty. Toiseksi päätellään, että jono $\seq{y_k(x)}$ on
Cauchy jokaisella $x \in [x_0-\delta,x_0+\delta]$ ja siis suppenee:
\[ 
y_k(x) \kohti y(x)\in[c,d], \quad x \in [x_0-\delta,x_0+\delta]. 
\]
\index{tasainen suppeneminen}%
On edelleen pääteltävissä, että tämä suppeneminen on itse asiassa \kor{tasaista} välillä
$[x_0-\delta,x_0+\delta]$, ts.\ pätee
\[
\norm{y_k-y}\,=\,\max_{x\in[x_0-\delta,\,x_0+\delta]} \abs{y_k(x)-y(x)} 
                 \kohti\ 0, \quad \text{kun}\ k \kohti \infty,
\]
ja että raja-arvona konstruoitu $y(x)$ on jatkuva välillä $[x_0-\delta,x_0+\delta]$. (Tässä on
todistuksen käsitteellisesti vaativin kohta!) Ottamalla lopuksi iteraatiokaavassa 
\eqref{Picard-2} puolittain raja-arvo, kun $k\kohti\infty$, päätellään, että $y(x)$ ratkaisee 
probleeman \eqref{Picard-3}, jolloin Lauseen \ref{picard-lindelöf} ensimmäisestä väittämästä
on todistettu ratkaisun olemassaoloa koskeva osa. Ratkaisun yksikäsitteisyys on todettavissa
suoraviivaisemmin, ks.\ Harj.teht.\,\ref{H-dy-8: ratkaisun yksikäsitteisyys}. 

Lauseen \ref{picard-lindelöf} toista väittämää todistettaessa ei jonon $\seq{y_k(x)}$ 
pako-ongelmaa ole, joten väli $[x_0-\delta,x_0+\delta]$ voidaan korvata välillä $[a,b]$.
Tarkentamalla em.\ laskua osoittautuu, että voidaan myös arvioida
\[
\abs{y_{k+1}(x)-y_k(x)}\ \le\ C\,\frac{L^k\abs{x-x_0}^k}{k!}\,, \quad x\in[a,b].
\]
Tutkittaessa jonon $\seq{y_k(x)}$ suppenemista tulee siis vertailukohdaksi nyt geometrisen
sarjan sijasta eksponenttifunktion sarja
\[ 
e^{L\abs{x-x_0}} = \sum_{k=0}^\infty \frac{L^k\abs{x-x_0}^k}{k!}\,. 
\]
Tämä suppenee kaikkialla, joten voidaan päätellä samalla tavoin kuin edellä, että funktio
$y(x)=\lim_k y_k(x)$ on integraaliyhtälön \eqref{Picard-3} ratkaisu, tällä kertaa koko välillä
 $[a,b]$. Alkuarvotehtävä \eqref{Picard-1} on näin ratkaistu yksikäsitteisesti välillä $(a,b)$.
\index{kiintopisteiteraatio!Picardin iteraatio|)} \index{Picardin iteraatio|)}

\begin{Exa}
Alkuarvotehtävä
\[
\begin{cases} \,y'=\sin y, \\ \,y(x_0)=y_0 \end{cases}
\]
toteuttaa Lauseen \ref{picard-lindelöf} ehdot, kun $T=\R^2$, sillä $f(y)=\sin y$ on kaikilla 
väleillä Lipschitz-jatkuva vakiolla $L=1$. Alkuarvotehtävällä on siis yksikäsitteinen ratkaisu
välillä $(-\infty,\infty)$ jokaisella $(x_0,y_0)\in\R^2$. Jos valitaan esim.\ $x_0=0$, niin
nähdään, että $y_0$:n eri arvoja vastaavat ratkaisut löytyvät kaikki Esimerkissä 
\ref{separoituva DY}:\ref{muuan separoituva dy} laskettujen ratkaisujen joukosta. Näin on tullut
varmistetuksi, että mainitussa esimerkissä löydettiin kaikki differentiaaliyhtälön $y'=\sin y$
ratkaisut. \loppu
\end{Exa}
\begin{Exa}
Alkuarvotehtävälle
\[
\begin{cases} \,y'=\sqrt{\abs{y}}, \\ \,y(0)=0 \end{cases}
\]
Lauseen \ref{picard-lindelöf} ehdot eivät toteudu millään joukon $T=(a,b)\times(c,d)$
valinnalla, sillä on oltava $0\in(a,b)$, jolloin $f(y)=\sqrt{\abs{y}}$ ei ole Lipschitz-jatkuva
välillä $[a,b]$. Ratkaisu ei olekaan yksikäsitteinen 
(ks.\ Esimerkki \ref{DY-käsitteet}:\,\ref{erikoinen dy}b). \loppu
\end{Exa}
\begin{Exa}
Jos lineaarisessa alkuarvotehtävässä
\[
\begin{cases}
\,y'+P(x)y=R(x), \\ \,y(x_0)=y_0
\end{cases}
\]
$P$ ja $Q$ ovat jatkuvia välillä $[a,b]$, niin Lauseen \ref{picard-lindelöf} ehdot ovat 
voimassa, kun valitaan $T=[a,b]\times\R$. Nimittäin
\begin{itemize}
\item[(i)]  Funktio $x \map f(x,y)=-P(x)y+R(x)$ \ on jatkuva välillä $[a,b]$ jokaisella
            $y\in\R$.
\item[(ii)] $\abs{f(x,y_1)-f(x,y_2)}=\abs{P(x)}\abs{y_1-y_2}\leq L\abs{y_1-y_2},\quad 
             (x,y_1), \ (x,y_2) \in T$, missä $L=\D \max_{x\in [a,b]} \abs{P(x)}$.
\end{itemize}
Alkuarvotehtävä siis ratkeaa yksikäsitteisesti välillä $(a,b)$, kun $x_0\in (a,b)$.
(Sama pääteltiin Luvussa \ref{lineaarinen 1. kertaluvun DY} toisin keinoin.) \loppu
\end{Exa}

\pagebreak
\subsection{Differentiaaliyhtälösysteemin ratkeavuus}
\index{differentiaaliyhtälön!h@ratkeavuus|vahv}
\index{differentiaaliyhtälö!g@--systeemi|vahv}

Picardin-Lindelöfin lauseen todellinen hienous piilee siinä, että lauseen väittämä voidaan
yleistää hyvin suoraviivaisesti koskemaan yleistä normaalimuotoista 
differentiaaliyhtälösysteemiä (vrt.\ Luku \ref{DY-käsitteet})
\[ \left\{ \begin{aligned} 
           y'_1(x) &= f_1(x,y_1, \ldots, y_n), \\
           y'_2(x) &= f_2(x,y_1, \ldots, y_n), \\
                  &\vdots \\
           y'_n(x) &= f_n(x,y_1 \ldots, y_n).
           \end{aligned} \right. \]
Alkuehdoiksi oletetaan
\[ 
y_i(x_0) = A_i, \quad i = 1 \ldots n, 
\]
missä $x_0\in(a,b)$. Tarkastellaan esimerkkinä alkuarvotehtävää kokoa $n=2\,$:
\begin{equation} \label{Picard-4}
\left\{ \begin{aligned}
&y' = f_1(x,y,u), \\
&u' = f_2(x,y,u), \quad x\in(a,b), \\
&y(x_0)=y_0,\ u(x_0)=u_0.
        \end{aligned} \right.
\end{equation}
Picardin-Lindelöfin lauseen muotoilussa tälle ongelmalle asetetaan 
\[
T=[a,b]\times[c_1,d_1]\times[c_2,d_2]\subset\R^3 \quad 
        \text{tai} \quad T=[a,b]\times\R^2\subset\R^3
\]
ja oletetaan
\begin{itemize}
\item[(i)]  Funktiot $\,x \map f_1(x,y,u)\,$ ja $\,x \map f_2(x,y,u)\,$ ovat jatkuvia välillä
            $[a,b]$ jokaisella $\,(y,u)\in[c_1,d_1]\times[c_2,d_2]\,$ tai jokaisella
            $\,(y,u)\in\Rkaksi$.
\item[(ii)] $f_1$ ja $f_2$ toteuttavat muuttujien $y,u$ suhteen Lipschitz-ehdon: \newline
            Jollakin $L\in\R_+$ pätee
            \begin{align*}
            \abs{f_i(x,y_1,u_1)-f_i(x,y_2,u_2)} 
                     &\le L\max\{\abs{y_1-y_2},\,\abs{u_1-u_2}\}, \quad i=1,2 \\
                     &\qquad\forall (x,y_1,u_1)\in T, \ (x,y_2,u_2)\in T.
            \end{align*}
\end{itemize}
Näillä oletuksilla päädytään vastaavaan väittämään kuin edellä, ts.\ alkuarvotehtävä 
\eqref{Picard-4} ratkeaa yksikäsitteisesti, joko $x_0$:n lähellä välillä 
$(x_0-\delta,x_0+\delta)$, tai vahvemmin oletuksin koko välillä $(a,b)$. Myös todistus on 
perusidealtaan sama kuin edellä: Lähtökohtana on ratkaisun konstruoiminen Picardin iteraatiolla
\[ \left\{ \begin{aligned} 
y_k(x) &= y_0+\int_{x_0}^x f_1(t,y_{k-1}(t),u_{k-1}(t))\,dt, \\
u_k(x) &= u_0+\int_{x_0}^x f_2(t,y_{k-1}(t),u_{k-1}(t))\,dt, \quad k=1,2, \ldots 
           \end{aligned} \right. \]

Palataan lopuksi Luvussa \ref{2. kertaluvun lineaarinen DY} tarkasteltuun toisen kertaluvun
lineaariseen ja homogeeniseen alkuarvotehtävään
\begin{equation} \label{Picard-5}
\begin{cases}
\,y''+P(x)y'+Q(x)y=0, \\ \,y(0)=A, \ y'(0)=B.
\end{cases}
\end{equation}
Kun tässä kirjoitetaan $u=y'$, niin päästään systeemimuotoon \eqref{Picard-4}, missä
\[
f_1(x,y,u)=u, \quad f_2(x,y,u) = -Q(x)y-P(x)u. 
\]
Jos $P$ ja $Q$ ovat jatkuvia välillä $(a,b)$ (kuten oletettiin Luvussa 
\ref{2. kertaluvun lineaarinen DY}), niin $x \map f_1(x,y,u)$ ja $x \map f_2(x,y,u)$ ovat
jatkuvia välillä $[a_1,b_1]$ jokaisella $(y,u)\in\Rkaksi$, kun $[a_1,b_1]\subset(a,b)$. Koska
\begin{align*}
f_1(x,y_1,u_1)-f_1(x,y_2,u_2)\ &=\ u_1-u_2, \\
f_2(x,y_1,u_1)-f_2(x,y_2,u_2)\ &=\ -Q(x)(y_1-y_2)-P(x)(u_1-u_2), 
\end{align*}
niin nähdään, että myös ehto (ii) toteutuu, kun asetetaan
\[ 
L=\max\{1,\,L_1\}, \quad L_1 = \max_{x\in[a_1,b_1]}\bigl\{\abs{P(x)}+\abs{Q(x)}\bigr\}. 
\]
Näin ollen Picardin--Lindelöfin lauseen (vahvemman väittämän) mukaan alkuarvotehtävä 
\eqref{Picard-5} ratkeaa yksikäsitteisesti välillä $(a_1,b_1)$. Koska tämä on totta aina kun
$[a_1,b_1] \subset (a,b)$, niin tehtävällä on yksikäsitteinen ratkaisu koko avoimella välillä
$(a,b)$. Lause \ref{lin-2: lause 2} on näin tullut todistetuksi Picardin--Lindelöfin lauseen
seuraamuksena.

\Harj
\begin{enumerate}

\item
Mitä voidaan Picardin--Lindelöfin lauseen valossa sanoa seuraavien alkuarvotehtävien 
ratkeavuudesta ja ratkaisujen yksikäsitteisyydestä $\R$:ssä tai $\R$:n osaväleillä?
\begin{align*}
&\text{a)}\ \ y'=\cos y+\frac{x}{y^2+1}\,,\,\ y(0)=A\in\R \\
&\text{b)}\ \ y'=\frac{y}{\cosh x}+e^{x-y^2},\,\ y(0)=A\in\R \\ 
&\text{c)}\ \ y'=\frac{y}{\cos x}+e^{x-y^2},\,\ y(0)=A\in\R \\
&\text{d)}\ \ y'=8xy+\frac{\sin(xy)}{x^2+2x-35}\,,\,\ y(0)=A\in\R \\
&\text{e)}\ \ y''=\cos(xy),\,\ y(0)=A\in\R,\ y'(0)=B\in\R
\end{align*}

\item
Määritä seuraaville alkuarvotehtäville kaikki alkupisteen lähiympäristössä pätevät ratkaisut.
Vertaa Picardin--Lindelöfin lauseen väittämään. \vspace{1mm}\newline
a) \ $y'=y^2,\,\ y(0)=1 \qquad\qquad\quad\,\ $
b) \ $y'=y^2,\,\ y(0)=1000$ \newline
c) \ $y'=y\sqrt{\abs{y}},\,\ y(0)=1 \qquad\quad\,\ \ $
d) \ $y'=\sqrt[4]{\abs{y}},\,\ y(0)=1$ \newline 
e) \ $y'=\sqrt[4]{\abs{y}},\,\ y(0)=0.0001 \qquad\,$
f) \ $y'=\sqrt[4]{\abs{y}},\,\ y(0)=0$

\item 
Halutaan laskea Neperin luku $e$ lähtien tiedosta, että $e=y(1)$, missä $y(x)$ ratkaisee
alkuarvotehtävän $\,y'=y,\ y(0)=1$. Millainen laskukaava saadaan luvulle $e$, kun tehtävä 
ratkaistaan Picardin iteraatiolla?

\item 
Laske seuraavissa alkuarvotehtävissä Picardin iteraatit $y_1(x)$ ja $y_2(x)$ ja vertaa 
tarkkaan ratkaisuun. 
(Vrt.\ Harj.teht.\ \ref{DYn numeeriset menetelmät}:\ref{H-dy-7: Euler-kokeita}.)
\[
\text{a)}\ \ \begin{cases} \,y'=-y^2, \\ \,y(0)=1 \end{cases}
\text{b)}\ \ \begin{cases} \,y'=y^2, \\ \,y(0)=1 \end{cases}
\text{c)}\ \ \begin{cases} \,y'=-2xy^2, \\ \,y(0)=1 \end{cases}
\text{d)}\ \ \begin{cases} \,y'=2xy^2, \\ \,y(0)=1 \end{cases}
\]

\item
Laske alkuarvotehtävälle
\[
\text{a)}\ \ \begin{cases} \,y''=-y, \\ \,y(0)=1,\ y'(0)=0 \end{cases} \quad
\text{b)}\ \ \begin{cases} \,y''=-y, \\ \,y(0)=0,\ y'(0)=1 \end{cases}
\]
likimääräinen ratkaisu iteroimalla $2n$ kertaa ($n\in\N$) Picardin iteraatiolla tehtävän
systeemimuodosta.

\item (*) \label{H-dy-8: ratkaisun yksikäsitteisyys}
Näytä, että Lauseen \ref{picard-lindelöf} oletuksin alkuarvotehtävän \eqref{Picard-1} ratkaisu
(jos olemassa) on yksikäsitteinen välillä $(x_0-\delta,x_0+\delta)$, kun $\delta$ on riittävän
pieni. \kor{Vihje}: Oleta $y_1(x)$ ja $y_2(x)$ ratkaisuiksi, lähde tehtävän integraalimuodosta
\eqref{Picard-3}, käytä oletusta (ii) ja valitse $\delta<1/L$.


\end{enumerate} %*Picardin-Lindelöfin lause

\chapter{Matriisit} \label{matriisit}

\begin{quote}
''Jos yhtälöryhmässä on yli $50$ tuntematonta, on ratkaisemisessa syytä käyttää
tietokonetta''.\footnote[2]{Lentokoneen siiven lujuuslaskentaan perehtyneen insinöörin toteamus
alan julkaisussa 1950-luvulla.}
\end{quote}

Monilla insinöörialoilla on ollut pitkä \kor{lineaaristen yhtälöryhmien} ratkaisemisen perinne
jo ennen tietokoneiden aikaa. Myös lineaaristen yhtälöryhmien matematiikassa,
\index{lineaarialgebra}%
\kor{lineaarialgebrassa}, perinteet ovat pitkät. Ne ulottuvat 1700-luvulle, jolloin kehitettiin
\kor{determinantin} käsitteeseen perustuva yhtälöryhmien ratkeavuusteoria ja ratkaisukaavat.
Determinanteilla on vielä nykyäänkin käyttöä teoreettisissa tarkasteluissa, erinäisissä
laskukaavoissa (vrt.\ Luku \ref{ristitulo}) ja yleisemminkin käsinlaskussa silloin, kun
yhtälöryhmän koko on pieni. Muuten determinanttioppia on pidettävä vanhentuneena, syystä että
tämän opin mukaiset laskukaavat eivät sovellu laajamittaiseen numeeriseen käyttöön.
Tietokoneiden aikakaudella lineaarialgebra onkin pitkälti irtaantunut determinanttiperinteestä
ja vanhoista käsinlaskun menetelmistä. 

Tässä luvussa tutkimuksen kohteena on lineaarialgebran avainkäsite, \kor{matriisi}.
Matriiseilla laskemista eli \kor{matriisialgebraa} tarkastellaan ensin Luvussa 
\ref{matriisialgebra}. Tämän jälkeen Luvuissa \ref{inverssi}--\ref{tuettu Gauss} tutkitaan 
lineaarisia yhtälöryhmiä, niiden ratkaisemista \kor{Gaussin algoritmilla} ja ratkaisemiseen 
keskeisesti liittyvää \kor{käänteismatriisin} käsitettä. Luvussa \ref{determinantti} käydään
lyhyesti läpi vanha determinanttioppi laskukaavoineen. Luvuissa 
\ref{lineaarikuvaukset}---\ref{affiinikuvaukset} ovat tutkimuskohteena matriisialgebraan 
perustuvat \kor{lineaarikuvaukset} ja \kor{affiinikuvaukset} sekä näiden geometriset 
sovellutukset, \kor{geometriset kuvaukset}. Viimeisessä osaluvussa esitellään joitakin 
lineaaristen yhtälöryhmien sovellusesimerkkejä perinteisestä insinöörimatematiikasta.
 %Matriisit
\section{Matriisialgebra} \label{matriisialgebra}
\alku
\index{matriisialgebra|vahv}
\index{laskuoperaatiot!g@$n$-vektoreiden, matriisien|vahv}

\index{lineaarinen yhtälöryhmä} \index{yhtzy@yhtälöryhmä!a@lineaarinen}%
Yleinen $m$ yhtälön ja $n$ tuntemattoman \kor{lineaarinen yhtälöryhmä} (-systeemi) on muotoa
\begin{equation} \label{m.1.1}
\begin{cases}
\begin{aligned}
a_{11} x_1 + \,a_{12} x_2 + \ldots + a_{1n} x_n \quad             &= \quad b_1, \\
a_{21} x_1 + \,a_{22} x_2 + \ldots + a_{2n} x_n \quad             &= \quad b_2, \\
\vdots \qquad\quad \vdots \ \ \qquad\quad\quad \vdots \qquad\quad &\ \,\vdots \quad\ \ \vdots \\
a_{m1} x_1 + a_{m2} x_2 + \ldots + a_{mn} x_n \ \               &= \quad b_m.
\end{aligned}
\end{cases}
\end{equation}
\index{kerroin (yhtälöryhmän)}%
Tässä $m,n \in \N$, $\ a_{ij}$:t ovat yhtälöryhmän \kor{kertoimet}, ja myös luvut 
$b_i,\ i = 1 \ldots m$, jotka muodostavat yhtälöryhmän nk. 'oikean puolen' (tai 'vakiotermin'),
oletetaan tunnetuiksi. Tuntemattomia ovat siis $x_i,\ i = 1 \ldots n$, ja ongelmana yleensä
näiden löytäminen, eli yhtälöryhmän \kor{ratkaiseminen}. Jatkossa oletetaan pääsääntöisesti, 
että $\ a_{ij}, x_i, b_i \in \R$. Yleisemmin voitaisiin olettaa, että 
$\ a_{ij}, x_i, b_i \in \K$, missä $\K$ on mikä tahansa kunta, sillä yhtälöryhmää 
kirjoitettaessa ja ratkaistaessa tarvitaan vain kunnan laskuoperaatioita. Esimerkiksi voisi olla 
$\K = \C$ tai (kuten yksinkertaisissa esimerkeissä usein) $\K = \Q$.

Yhtälöryhmän \eqref{m.1.1} sanotaan olevan \kor{kokoa} $m \times n$ ('$m$ kertaa $n$'). 
Yhtälöryhmällä ei välttämättä ole ratkaisua lainkaan, tai ratkaisu voi olla monikäsitteinen. Jos
yhtälöryhmällä on y\pain{ksikäsitteinen} ratkaisu, valittiinpa oikealla puolella olevat luvut
$b_i$ miten tahansa, niin sanotaan, että yhtälöryhmä on
\index{szyzy@säännöllinen yhtälöryhmä}%
\kor{säännöllinen}. Muulloin, eli jos
ratkaisua ei ole jollakin $(b_i)$ tai ratkaisu on monikäsitteinen, yhtälöryhmä on 
\index{singulaarinen yhtälöryhmä}%
epäsäännöllinen eli \kor{singulaarinen}. Yhtälöryhmän säännöllisyys tai singulaarisuus ei siis
riipu luvuista $b_i$ vaan on ainoastaan kerrointaulukon $(a_{ij})$ ominaisuus. Tullaan mäkemään,
että ehto $m=n$ on välttämätön ehto säännöllisyydelle, ts.\ säännöllisessä ryhmässä on 
yhtälöiden lukumäärän ($m$) ja tuntemattomien lukumäärän ($n$) täsmättävä. Tapauksessa $m>n$ 
\index{ylimääräytyvä (yhtälöryhmä)} \index{alimääräytyvä (yhtälöryhmä)}%
sanotaan yhtälöryhmää \eqref{m.1.1} \kor{ylimääräytyväksi} (engl.\ overdetermined), tapauksessa
$m<n$ \kor{alimääräytyväksi} (engl.\ underdetermined). Nämä ovat siis aina singulaarisia 
systeemejä. 
\begin{Exa} Singulaarisia systeemejä tyyppiä $m=n=2$ ovat esimerkiksi
\[
\begin{cases} 
\begin{aligned} x_1 +  x_2 &= b_1, \\ x_1  +  x_2 &= b_2 \end{aligned} 
\end{cases} \quad \text{ja} \quad
\begin{cases}
\begin{aligned} x_1 + 2x_2 &= b_1, \\ 2x_1 + 4x_2 &= b_2. \end{aligned} 
\end{cases} \loppu 
\] 
\end{Exa}

Jatkossa suoritettavien algebrallisten tarkastelujen lähtökohtana on yleinen lineaarinen 
yhtälöryhmä \eqref{m.1.1}. Kirjoitetaan yhtälöryhmä ensinnäkin taulukkomuotoon 
(1. abstraktiovaihe)
\[
\begin{bmatrix} a_{11} x_1\ + & \ldots & +\ a_{1n} x_n \\ \vdots & & \vdots \\ 
                a_{m1} x_1\ + & \ldots & +\ a_{mn} x_n \end{bmatrix} 
   \quad = \quad \begin{bmatrix} b_1 \\ \vdots \\ b_m \end{bmatrix},
\]
eli
\begin{equation} \label{m.1.2}
\mv{y}\ =\ \mv{b},
\end{equation}
missä
\[ 
\mv{b}\ =\ \begin{bmatrix} b_1 \\ \vdots \\ b_m \end{bmatrix}, \qquad
\mv{y}\ =\ \begin{bmatrix} y_1 \\ \vdots \\ y_m \end{bmatrix}, \qquad 
    y_i = \sum_{j=1}^n a_{ij} x_j. 
\]
Tässä $\mv{b},\,\mv{y}$ ovat nk.\
\index{pystyvektori} \index{vektorib@vektori (algebrallinen)!c@$\R^n$:n}%
\kor{pystyvektoreita} (engl.\ column vector). Jatkossa
käytetään pystyvektoreiden symboleina lihavoituja pieniä kirjaimia 
$\mv{a},\mv{b},\mv{x},\mv{y}$, jne.

Ym.\ pystyvektoreissa $\mv{y},\,\mv{b}$ on $m$ vektorin \kor{koko} (tai tyyppi), mikä voidaan 
ilmaista myös termillä $m$\kor{-vektori}. Luvut $b_i,\,y_i$ ovat ko.\ vektoreiden
\index{alkio (vektorin, matriisin)}%
\kor{alkiot} (tai komponentit). Voidaan myös käyttää merkintöjä 
\[
[\mb]_i = b_i, \quad \mb = (b_i), \quad \mb = (b_i)_{i=1}^m.
\]

Termi 'vektori' viittaa siihen, että pystyvektoreille voidaan määritellä vektoreiden 
peruslaskuoperaatiot, eli vektorien yhteenlasku ja skalaarilla kertominen. Määritelmät ovat
\[ 
\mv{x} + \mv{y} = \begin{bmatrix} x_1 + y_1 \\ \vdots \\ x_n + y_n \end{bmatrix}, \qquad
 \lambda \mv{x} = \begin{bmatrix} \lambda x_1 \\ \vdots \\ \lambda x_n \end{bmatrix} \qquad 
                                                                (\lambda \in \R).
\]
Yhteenlaskussa vektoreiden $\mv{x},\,\mv{y}$ on oltava \pain{samaa} \pain{kokoa}, muuten 
yhteenlaskua ei voi määritellä. Laskuoperaatioissa siis lasketaan yhteen ja kerrotaan 
alkioittain, kuten aiemmin tehtiin Luvuissa \ref{tasonvektorit} ja \ref{ristitulo} 
käsiteltäessä $\R^2$:n lukupareja ja $\R^3$:n lukukolmikkoja vektoreina. Pystyvektoreita kokoa
$n$ voidaankin pitää näiden entuudestaan tuttujen 'algebravektoreiden' yleistyksenä. (Lukuparien
tai \mbox{-kolmikoiden} käsittely pystyvektorina ei mitenkään vaikuta vektorioperaatioihin.)
Yhdenmukaisesti aiempien merkintöjen kanssa sovitaan, että pystyvektorit kokoa $n$ muodostavat
vektoriavaruuden nimeltä $\R^n$. Avaruuden $\R^n$
\index{nollavektori}%
\kor{nollavektori} on aiempaan tapaan vektori, jonka kaikki alkiot ovat nollia. Tätä merkitään
symbolilla $\mv{0}$. Avaruuden $\R^n$ luonnollinen
\index{kanta}%
\kor{kanta} saadaan, kun kirjoitetaan
\[ 
\mx = x_1 \me_1 + \ldots + x_n \me_n = \sum_{i=1}^n x_i \me_i, 
\]
jolloin
\[ 
\me_1 = \begin{bmatrix} 1 \\ 0 \\ \vdots \\ 0 \end{bmatrix}, \quad
\me_2 = \begin{bmatrix} 0 \\ 1 \\ \vdots \\ 0 \end{bmatrix}, \quad \ldots \quad
\me_n = \begin{bmatrix} 0 \\ 0 \\ \vdots \\ 1 \end{bmatrix}, 
\]
eli
\[ 
[\me_i]_j\ =\ \delta_{ij} 
           = \begin{cases} 1, \quad \text{jos}\ i=j, \\ 0, \quad \text{jos}\ i \neq j. 
             \end{cases} 
\]
Tässä symboli $\delta_{ij}$ on nk.\
\index{Kroneckerin delta}%
\kor{Kroneckerin delta} (-symboli). Vektoreita $\me_i$ sanotaan $\R^n$:n
\index{yksikkövektori}%
\kor{yksikkövektoreiksi}. Jokainen $\mx \in \R^n$ voidaan siis kirjoittaa 
(ilmeisen yksikäsitteisesti) vektorien $\me_i$
\index{lineaariyhdistely}%
\kor{lineaariyhdistelynä}, eli $\{\me_1, \ldots \me_n\}$ on $\R^n$:n kanta. Koska kannassa on
$n$ vektoria, sanotaan, että $R^n$:n
\index{dimensio}%
\kor{dimensio} on $n$:
\[ 
\text{dim}\,\R^n = n. 
\]

Toistaiseksi ei siis ole väliä, ovatko $\R^n$:n vektorit 'pystyssä' vai 'kumossa', kunhan 
yhtälössä \eqref{m.1.2} vain tulkitaan yhtäsuuruusmerkki normaaliin tapaan, eli
\[
\mv{y} = \mv{b} \quad \ekv \quad y_i = b_i,\ \ i = 1 \ldots n.
\]
Seuraavssa (toisessa) abstraktion vaiheessa kuitenkin tulee ero, kun yhtälössä \eqref{m.1.2} 
kertoimet $a_{ij}$ erotetaan erilliseksi olioksi kirjoittamalla
\[ 
\mv{y}\ =\ \begin{bmatrix} y_1 \\ \vdots \\ y_m \end{bmatrix}\ =\ 
           \begin{bmatrix} a_{11} & \quad & \ldots & \quad & a_{1n} \\ \vdots & & & & \vdots \\ 
                           a_{m1} & \quad & \ldots & \quad & a_{mn} \end{bmatrix}
           \begin{bmatrix} x_1 \\ \\ \vdots \\ \\ x_n \end{bmatrix}\ =\ \mv{A} \mv{x}, 
\]
jolloin alkuperäinen yhtälöryhmä \eqref{m.1.1} tulee kirjoitetuksi muotoon
\begin{equation} \label{m.1.3}
\mv{A} \mv{x}\ =\ \mv{b}.
\end{equation}
\index{matriisi} \index{rivi (matriisin)} \index{sarake (matriisin)}
\index{alkio (vektorin, matriisin)}%
Olio $\mv{A}$, joka siis näyttää kaksiulotteiselta lukutaulukolta, on nimeltään \kor{matriisi}
(engl.\ matrix). Indeksit $m,n$ määräävät, että matriisi on \kor{tyyppiä} tai \kor{kokoa} 
$m \times n$ ('$m$ kertaa $n$'). Tässä $m$ on matriisin (vaaka)\kor{rivien} (engl.\ row) ja $n$ 
\kor{sarakkeiden} (tai pysyrivien, engl.\ column) lukumäärä. Luvut $a_{ij}$ ovat nimeltään 
matriisin \kor{alkiot} tai \kor{elementit}. Näille voidaan myös käyttää merkintää 
$[\mv{A}]_{ij}$, ja voidaan myös kirjoittaa
\[
\mv{A}\ =\ (a_{ij}) \qquad \text{tai} \quad\quad \mv{A}\ 
        =\ (a_{ij},\ i = 1 \ldots m,\ j = 1 \ldots n).
\]
Määritelmän mukaisesti pystyvektori kokoa $m$ ($m$-vektori) on matriisi tyyppiä $m \times 1$.
Jatkossa erotetaan matriisit kuitenkin yleisemmin vektoreista käyttämällä matriisien symboleina
lihavoituja isoja kirjaimia $\mv{A},\,\mv{B}$, jne.

Kuten vektori, matriisikin 'elää' vasta sille määriteltyjen laskuoperaatioiden kautta. Yhtälössä
\eqref{m.1.3} esiityy jo eräs kaikkein keskeisimmistä operaatioista:
\index{matriisin ($\nel$neliömatriisin)!a@ja vektorin kertolasku}%
\pain{matriisin} j\pain{a} \pain{vektorin} \pain{kertolasku}. Vertaamalla yhtälöitä
\eqref{m.1.1}--\eqref{m.1.3} nähdään heti, mikä on tämän kertolaskun määritelmä: $\mv{A}\mv{x}$
on pystyvektori kokoa $m$, jonka alkiot $[\mv{A}\mv{x}]_i$ määritellään
\[ 
\boxed{ \begin{aligned} \quad \ykehys [\mv{A}\mv{x}]_i\ 
    &=\ \sum_{j=1}^n a_{ij} x_j, \quad i = 1 \ldots m, \\
    &\qquad\quad \,\mv{x}\,= (\,x_i,\ i = 1 \ldots n\,), \\
    &\qquad\quad \mv{A}  = (\,a_{ij},\ i = 1 \ldots m,\ j = 1 \ldots n\,). \quad\akehys \\
        \end{aligned} } 
\]
Määritelmän mukaisesti on $\mv{A}$:n sarakkeiden lukumäärän täsmättävä vektorin $\mv{x}$ kokoon,
jotta $\mv{A}\mv{x}$ olisi määritelty. Tämäntyyppiset y\pain{hteenso}p\pain{ivuusehdot}
rajoittavat kaikkia matriisien välisiä laskuoperaatioita.

Toinen matriisin ja vektorin tulon esitysmuoto saadaan, kun merkitään $\mv{a}_j = \mA$:n $j$:s
sarake, ts.
\[ 
\mv{a}_j = \begin{bmatrix} a_{1j} \\ \vdots \\ a_{mj} \end{bmatrix}. 
\]
Kun \mA\ esitetään sarakkeittensa avulla, niin em.\ määritelmästä nähdään, että pätee
\[ 
\boxed{ \quad\kehys\mA = [\ma_1 \ldots \ma_n] \qimpl \mA\mx = \sum_{j=1}^n x_j \mv{a}_j. \quad}
\]
Siis matriisin \mA\ ja vektorin $\mx = (x_i)$ tulo on \mA:n sarakkeiden $\mv{a}_j$ 
lineaarinen yhdistely, kertoimina $x_j$. Erityisesti siis pätee 
\[
\mA = [\mv{a}_1 \ldots \mv{a}_n ] \qimpl \mA\me_k=\ma_k, \quad k=1 \ldots n.
\]
Tästä on puolestaan helppo päätellä seuraava tulos, jolla on käyttöä jatkossa.
\begin{Prop} \label{matriisien yhtäsuuruus} Jos $\mA$ ja $\mB$ ovat matriiseja kokoa 
$m \times n$, niin pätee
\[
\mA\mx=\mB\mx\ \ \forall \mx\in\R^n \qimpl \mA=\mB.
\]
\end{Prop}

\begin{Exa} Laske $\mA\mx$ ja $\mB\mx$, kun 
\[ 
\mv{A} = \begin{rmatrix} 1 & 2 \\ 0 & 1 \end{rmatrix}, \quad 
\mv{B} = \begin{rmatrix} 1 & 3 \\ 2 &-1 \\ 1 & 1 \end{rmatrix}, \quad
\mv{x} = \begin{rmatrix} 1 \\-1 \end{rmatrix}. 
\]
\ratk
\[ 
\begin{aligned} 
\mv{Ax}\ &=\ \begin{rmatrix} 1 \cdot 1 + 2 \cdot (-1) \\ 0 \cdot 1 + 1 \cdot (-1) \end{rmatrix}\ 
 =\ (1) \begin{rmatrix} 1 \\ 0 \end{rmatrix} + (-1) \begin{rmatrix} 2 \\ 1 \end{rmatrix}\ 
 =\ \begin{rmatrix} -1 \\ -1 \end{rmatrix}, \\[3mm]
\mv{Bx}\ &=\ \begin{rmatrix} 1 \cdot 1 + 3 \cdot (-1) \\ 2 \cdot 1 + (-1) \cdot (-1) \\ 
                             1 \cdot 1 + 1 \cdot (-1) \end{rmatrix}\ 
          =\ (1) \begin{rmatrix} 1 \\ 2 \\ 1 \end{rmatrix}
                  + (-1) \begin{rmatrix} 3 \\ -1 \\ 1 \end{rmatrix}\ 
          =\ \begin{rmatrix} -2 \\ 3 \\ 0 \end{rmatrix}. \loppu \end{aligned}
\] 
\end{Exa}

Ym.\ määritelmästä nähdään, että matriisilla kertominen on kerrottavan vektorin suhteen 
lineaarinen laskutoimitus:
\[ 
\mv{A}(\lambda \mv{x} + \mu \mv{y})\ =\ \lambda \mv{Ax} + \mu \mv{Ay} \quad 
                                               \forall\ \lambda,\mu \in \R.
\]
(Tässä luonnollisesti edellytetään, että $\mv{x}$ ja $\mv{y}$ ovat samaa kokoa ja $\mv{A}$:n 
kanssa yhteensopivia).

Matriiseille määritellään skalaarilla kertominen ja samankokoisten matriisien yhteenlasku 
samalla tavoin kuin vektoreille, eli suoritetaan laskuoperaatiot alkioittain. Siis jos
$\mv{A} = (a_{ij})$ ja $\mv{B} = (b_{ij})$ ovat samaa kokoa, niin
\[ 
[\lambda \mv{A}]_{ij}\ =\ \lambda a_{ij}, \quad\quad [\mv{A}+\mv{B}]_{ij}\ =\ a_{ij} + b_{ij}. 
\]
Vektorioperaatiot $\R^n$:ssä voidaan tulkita näiden operaatioiden erikoistapauksiksi 
($\mv{A}$ ja $\mv{B}$ kokoa $n \times 1$). Kuten samankokoiset vektorit, myös samankokoiset 
matriisit muodostavat näillä laskuoperaatioilla varustettuna vektoriavaruuden. Tämän avaruuden
nolla-alkio on nk.\
\index{nollamatriisi}%
\kor{nollamatriisi}, jonka kaikki alkiot ovat nollia (symboli $\mv{0}$).

\subsection{Matriisitulo}
\index{matriisitulo|vahv}

Kahden matriisin $\mv{A},\,\mv{B}$ tulo eli \kor{matriisitulo} voidaan määritellä, edellyttäen
että matriisit ovat kooltaan yhteensopivat. Olkoon $\mv{x}$ pystyvektori kokoa $n$ ja olkoon 
$\mv{A} = (a_{ij})$ kokoa $m \times p$ ja $\mv{B} = (b_{ij})$ kokoa $p \times n$ 
($m,n,p \in \N$). Tällöin yhdistetty matriisi-vektoritulo $\mv{A}(\mv{Bx})$ on määritelty, kun
$\mv{x} \in \R^n$. Halutaan kirjoittaa tämä muodossa
\begin{equation} \label{m.1.4}
\mA(\mB\mx)\ =\ (\mv{AB})\,\mv{x}, \quad \mx\in\R^n
\end{equation}
ja ottaa tämä matriisitulon $\mv{AB}$ määritelmäksi. Tällöin on oltava
\begin{align*}
[\mv{(AB)x}]_i\ =\ [\mA(\mv{Bx})]_i\ &=\ \sum_{k=1}^p a_{ik}[\mv{Bx}]_k\ 
                                      =\ \sum_{k=1}^p a_{ik}\,\sum_{j=1}^n b_{kj} x_j \\
                                     &=\ \sum_{j=1}^n\,( \sum_{k=1}^p a_{ik} b_{kj}) x_j
                                      =\ \sum_{j=1}^n [\mv{AB}]_{ij}x_j\,.
\end{align*}
Siis nähdään, että \eqref{m.1.4} toteutuu, kun määritellään
\[ \boxed{ \begin{aligned}              
    \quad \ykehys [\mv{AB}]_{ij}\ 
         =\ \sum_{k=1}^p &a_{ik} b_{kj}, \quad i = 1 \ldots m,\ \ j = 1 \ldots n, \quad\quad \\
                  \mv{A} &=  (\,a_{ij},\   i = 1 \ldots m,\    j = 1 \ldots p\,), \\
          \akehys \mv{B} &=  (\,b_{ij},\,\ i = 1 \ldots\,p,\,\ j = 1 \ldots n\,). \\
           \end{aligned} } 
\]
\index{matriisin ($\nel$neliömatriisin)!ab@ja toisen matriisin tulo}%
Matriisitulo on näin määritelty yksikäsitteisesti, sillä Proposition 
\ref{matriisien yhtäsuuruus} mukaan \eqref{m.1.4} voi toteutua vain yhdelle matriisille 
$\mA\mB$. Määritelmästä nähdään, että $(\mv{AB})$:n $j$:s sarake $\ = \mv{Ab}_j$, missä 
$\mv{b}_j=(b_{kj},\ k=1 \ldots p)$ on $\mv{B}$:n $j$:s sarake. Näin ollen jos $\mv{B}$ esitetään
sarakkeittensa avulla, niin matriisitulolle $\mA\mB$ saadaan myös esitysmuoto
\[ 
\boxed{ \quad\kehys \mv{B} 
                 = \begin{bmatrix} \mb_1 & \ldots & \mb_n \end{bmatrix} \quad \impl \quad
          \mA\mB = \begin{bmatrix} \mA\mb_1 & \ldots & \mA\mb_n \end{bmatrix}. \quad } 
\]
Matriisitulo palautuu näin matriisin ja vektorin tuloiksi: Kerrotaan $\mv{B}$:n kukin 
sarakevektori $\mv{A}$:lla ja sijoitetaan tulokset $(\mv{AB})$:n sarakkeiksi. 

Matriisitulon taustalla olevasta sopimuksesta \eqref{m.1.4} voidaan päätellä, että matriisitulo
on liitännäinen. Nimittäin jos $\mv{A},\,\mv{B},\,\mv{C},\,\mv{x}$ ovat kooltaan yhteensopivia,
niin sopimuksen \eqref{m.1.4} mukaan pätee jokaisella $\mx\in\R^n$
\[ 
[\mv{A}(\mv{BC})]\mv{x}\ =\ \mv{A}[(\mv{BC})\mv{x}]\ =\ \mv{A}[\mv{B}(\mv{Cx})] 
                         =\ (\mv{AB})(\mv{Cx})\ =\ [(\mv{AB})\mv{C}]\mv{x}, 
\]
jolloin on oltava (Propositio \ref{matriisien yhtäsuuruus})
\[ 
\boxed{ \quad \rule[-2mm]{0mm}{6mm}\mv{A}(\mv{BC})\ =\ (\mv{AB})\mv{C}. \quad } 
\]
Vaihdannainen matriisitulo ei sen sijaan ole. Nimittäin ensinnäkin tuloon liittyvät 
yhteensopivuussäännöt ovat varsin rajoittavia: Tulot $\mv{AB}$ ja $\mv{BA}$ eivät välttämättä
ole molemmat määriteltyjä, ja vaikka olisivat, eivät välttämättä samaa tyyppiä. Jos $\mv{A}$ ja
$\mv{B}$ ovat molemmat kokoa $n \times n$, eli samankokoisia
\index{neliömatriisi}%
\kor{neliömatriiseja} (engl.\ square matrix), niin asia on näiltä osin kunnossa, mutta
tällöinkin on yleisesti $\mv{AB} \neq \mv{BA}$.

Neliömatriisit ovat lineaaristen yhtälöryhmien algebran kannalta huomattavan tärkeä matriisien
erikoisluokka, josta puhutaan enemmän seuraavassa luvussa. Todettakoon tässä yhteydessä 
kuitenkin, että jos samankokoisten neliömatriisien tapauksessa sattuu olemaan voimassa 
$\mv{AB} = \mv{BA}$, niin sanotaan, että matriisit $\mv{A}$ ja $\mv{B}$ \kor{kommutoivat}\,:
\index{kommutoivat matriisit}%
\[ 
\mv{AB}\ = \mv{BA} \quad \ekv \quad \text{$\mv{A}$ ja $\mv{B}$\, kommutoivat}. \]
Jokainen neliömatriisi kommutoi triviaalisti ainakin itsensä kanssa.
\begin{Exa} Matriisien
\[ 
\mv{A} = \begin{rmatrix} 1 & -1 \\ -1 & 2 \end{rmatrix}, \quad 
\mv{B} = \begin{bmatrix} 0 & 1 \\ 0 & 0 \end{bmatrix}, \quad
\mv{C} = \begin{bmatrix} 1 & 0 \\ 1 & 1 \\ 0 & 1 \end{bmatrix} 
\]
yhdeksästä mahdollisesta keskinäisestä tulosta $\mv{AC}$, $\mv{BC}$ ja $\mv{CC}$ eivät ole 
määriteltyjä, muut ovat:
\begin{gather*}
\mv{AB} = \left[ \mv{A} \begin{rmatrix} 0 \\ 0 \end{rmatrix}\ 
                 \mv{A} \begin{rmatrix} 1 \\ 0 \end{rmatrix} \right] 
        = \begin{rmatrix} 0 & 1 \\ 0 & -1 \end{rmatrix}, \quad 
\mv{BA} = \begin{rmatrix} -1 & 2 \\ 0 & 0 \end{rmatrix}, \\
\mv{CA} = \left[ \mv{C} \begin{rmatrix} 1 \\-1 \end{rmatrix}\ 
                 \mv{C} \begin{rmatrix} -1\\ 2 \end{rmatrix} \right] 
        = \begin{rmatrix} 1 & -1 \\ 0 & 1 \\ -1 & 2 \end{rmatrix}, \quad
\mv{CB} = \begin{rmatrix} 0 & 1 \\ 0 & 1 \\ 0 & 0 \end{rmatrix}, \\
\mv{AA} = \begin{rmatrix} 2 & -3 \\ -3 & 5 \end{rmatrix}, \quad 
\mv{BB} = \begin{rmatrix} 0 & 0 \\ 0 & 0 \end{rmatrix}. \quad \loppu
\end{gather*}                            
\end{Exa}

Esimerkissä neliömatriisit $\mv{A}$ ja $\mv{B}$ siis eivät kommutoi. Neliömatriisille määritelty
tulo $\mv{AA}$ voidaan kirjoittaa  $\mv{A}^2$, ja määritellä yleisemminkin neliömatriisin 
potenssiin korotus:
\[ 
\mA^k = \underbrace{\mv{AA} \cdots \mv{A}}_{k\ \text{kpl}}\,, \quad 
                         k \in \N \quad (\text{$\mv{A}$ neliömatriisi}). 
\]
Esimerkistä nähdään, että neliömatriisin tapauksessa voi olla $\mv{B}^2 = \mv{0}$ 
(= nollamatriisi), vaikka $\mv{B} \neq \mv{0}$.

\subsection{Matriisin transpoosi}
\index{transpoosi (matriisin)|vahv}

Vielä on määrittelemättä yksi keskeinen matriisialgebran operaatio,
\index{matriisin ($\nel$neliömatriisin)!b@transponointi, transpoosi}%
\kor{transponointi}. Tällä tarkoitetaan yksinkertaisesti matriisin $\mv{A}$ rivien ja
sarakkeiden vaihtoa keskenään. Tulosta merkitään $\mv{A}^T$ ja kutsutaan $\mv{A}$:n
\kor{transpoosiksi}\,:
\[ 
\boxed{ \quad \kehys [\mv{A}^T]_{ij} = [\mv{A}]_{ji}. \quad } 
\]
\jatko \begin{Exa} (jatko) \ Esimerkin matriiseille
\[ 
\mv{A}^T = \mv{A}, \quad \mv{B}^T = \begin{bmatrix} 0 & 0 \\ 1 & 0 \end{bmatrix}, \quad
\mv{C}^T = \begin{bmatrix} 1 & 1 & 0 \\ 0 & 1 & 1 \end{bmatrix}. \quad \loppu
\] 
\end{Exa}
Kuten esimerkissä, neliömatriisin tapauksessa (ei muussa) on mahdollista, että 
$\mv{A}^T = \mv{A}$, jolloin sanotaan, että $\mv{A}$ on \kor{symmetrinen}:
\index{symmetrisyys!c@neliömatriisin} \index{neliömatriisi!a@symmetrinen}%
\[ 
\mv{A}^T = \mA \quad \ekv \quad \mv{A}\,\ \text{symmetrinen}. 
\]

Transponoinnin laskusäännöistä tärkeimmät ovat
\[ 
\boxed{ \quad \kehys (\mv{A}^T)^T = \mv{A}, \quad (\mv{AB})^T =  \mv{B}^T \mv{A}^T. \quad } 
\]
Näistä ensimmäinen on ilmeinen, ja myös jälkimmäisen (tulon transponointisäännön) voi johtaa 
suoraviivaisesti matriisitulon määritelmästä (Harj.teht.\,\ref{H-m-1: todistuksia}a). Kun tulon
transponontisääntöä sovelletaan useampikertaiseen matriisituloon, saadaan yleisempi sääntö
\[ 
(\mA_1 \mA_2 \ldots \mA_n)^T\ =\ \mA_n^T \mA_{n-1}^T \ldots \mA_1^T. 
\]
Siis tulon transpoosi $=$ transpoosien tulo käänteisessä järjestyksessä. 

Muista matriisialgebran laskusäännöistä mainittakoon vielä määritelmien perusteella ilmeiset 
(oletetaan matriisien yhteensopivuus)
\[ 
\begin{aligned}
&\mA + \mB = \mB + \mA, \quad\quad (\mA + \mB) + \mC = \mA + (\mB + \mC), \\ 
&\mA(\lambda \mB) = (\lambda \mA) \mB = \lambda (\mA\mB), \quad \lambda \in \R, \\
&\mA(\mB + \mC) = \mA\mB + \mA\mC, \quad\quad (\mA + \mB) \mC = \mA\mC + \mB\mC, \\
&(\mA + \mB)^T = \mA^T + \mB^T, \quad\quad (\lambda \mA)^T = \lambda \mA^T, \quad \lambda \in \R.
\end{aligned} 
\]

\subsection{Vaakavektorit --- euklidinen skalaaritulo ja normi}

Kun pystyvektori $\mv{x} = (x_i)_{i=1}^n$ tulkitaan matriisiksi kokoa $n \times 1$, niin tällä
matriisilla on transpoosi, jota merkitään $\mv{x}^T$ ja sanotaan
\index{vaakavektori}%
\kor{vaakavektoriksi} 
(engl.\ row vector). Vektorialgebran kannalta pysty- ja vaakavektoreilla ei ole eroa, joten 
vektoriavaruus $\R^n$ voidaan yhtä hyvin ajatella vaakavektoreista koostuvaksi. Matriisialgebran
kannalta sen sijaan pysty- ja vaakavektorit ovat erilaisia olioita. Erityisen mielenkiintoinen
on matriisitulo $\mv{x}^T \mv{y}$, missä $\mv{y} = (y_i)_{i=1}^n$ on samaa kokoa oleva 
pystyvektori kuin $\mv{x}$. Matriisitulon määritelmän mukaan  $\mv{x}^T \mv{y}$ on matriisi
kokoa $1 \times 1$ eli skalaari:
\[ 
\mv{x}^T \mv{y} = \sum_{i=1}^n x_i y_i\ \in \R. 
\]
Tässä on itse asiassa määritelty matriisialgebran keinoin $\R^n$:n 
\index{euklidinen!c@skalaaritulo} \index{skalaaritulo!c@$\R^n$:n euklidinen}%
\kor{euklidinen skalaaritulo}
\[ 
\boxed{ \quad \scp{\mv{x}}{\mv{y}} = \mv{x}^T \mv{y} = \mv{y}^T \mv{x} 
                                   =  \sum_{i=1}^n x_i y_i, \quad \mv{x},\mv{y} \in \R^n. \quad}
\]
Tämä on aiemmin Luvuissa \ref{tasonvektorit} ja \ref{ristitulo} määriteltyjen $\R^2$:n
ja $\R^3$:n skalaaritulojen yleistys.
\index{Cauchyn!f@--Schwarzin epäyhtälö}%
Cauchyn-Schwarzin epäyhtälö $\R^n$:n euklidiselle skalaaritulolle on (ks.\ Lause \ref{schwarzR})
\[ 
\abs{\scp{\mv{x}}{\mv{y}}} \le \abs{\mv{x}} \abs{\mv{y}}, \quad \mv{x},\mv{y} \in \R^n, 
\]
missä
\[ 
\abs{\mv{x}} = \scp{\mv{x}}{\mv{x}}^{1/2} = (\mv{x}^T \mv{x})^{1/2}  
                                          = \bigl(\,\sum_{i=1}^n x_i^2 \,\bigr)^{1/2}
\]
\index{euklidinen!b@normi} \index{normi!euklidinen}%
on $\R^n$:n \kor{euklidinen normi}. Viitaten näin määriteltyyn skalaarituloon ja normiin 
käytetään avaruudesta $\R^n$ yleisesti nimitystä
\index{euklidinen!d@avaruus $\R^n$}%
\kor{euklidinen avaruus} $\R^n$. Edellä 
määritelty $\R^n$:n luonnollinen kanta $\{\me_1, \ldots, \me_n\}$ on euklidisen avaruuden 
\index{kanta!a@ortonormeerattu}%
kantana \kor{ortonormeerattu}:
\[ 
\scp{\me_i}{\me_j} = \delta_{ij} \quad \text{(Kroneckerin $\delta$)}.
\]

Jos $\mx\in\R^n$, $\my\in\R^m$ ja \mA\ on matriisi kokoa $m \times n$, niin skalaaritulo 
$\scp{\mA\mx}{\my} = (\mA\mx)^T \my$ on määritelty. Tässä on tulon transponointisäännön mukaan
$(\mA\mx)^T = \mx^T \mA^T$, joten pätee
\[ 
\boxed{\quad \kehys \scp{\mA\mx}{\my} = \scp{\mx}{\mA^T \my}. \quad} 
\] 
Huomautettakoon vielä, että jos $\mA$ on kokoa $m \times p$, $\mB$ on kokoa $p \times n$, ja
merkitään $\mA^T=[\ma_1 \ldots \ma_m]$ (eli $\mA$:n $i$:s rivi $=\ma_i^T$) ja 
$\mB=[\mb_1 \ldots \mb_n]$, niin matriisitulon määritelmän perusteella
\[
[\mA\mB]_{ij} = \ma_i^T\mb_j = \scp{\ma_i}{\mb_j}.
\]
Tulo $\mA\mB$ on siis taulukko, joka muodostuu $\mA$:n rivien $\mB$:n sarakkeiden välisistä
($\R^p$:n) skalaarituloista.

\subsection{Kompleksiset vektorit ja matriisit}
\index{kompleksinen vektori ja matriisi|vahv}

Koska matriisialgebrassa on perimmältään kyse vain matriisialkioiden välisistä 
peruslaskutoimituksista ja niiden yhdistelystä, voidaan matriisialkioiden ajatella kuuluvan 
$\R$:n sijasta mihin tahansa kuntaan $\K$. Yksinkertaisissa laskuesimerkeissä 
(kuten esimerkit edellä) on usein $\K = \Q$. Yleisemmisssä matemaattisissa tarkasteluissa,
samoin monissa sovelluksissa (esim.\ sähkötekniikassa) on sallittava matriisialkioiden 
kompleksiarvoisuus, jolloin $\K = \C$. Tässä tapauksessa pystyvektorit kokoa $n$ muodostavat
euklidisen avaruuden $\C^n$, jonka skalaaritulo on
\[ 
\scp{\mv{x}}{\mv{y}} = \sum_{i=1}^n x_i \bar{y}_i = \mv{x}^T \bar{\mv{y}}. 
\]
Tässä $\bar{\mv{y}}$, yleisemmin $\bar{\mv{A}}$, tarkoittaa kompleksista konjugointia 
alkioittain. 
\begin{Exa} Jos
\[ 
\mx = [\,i,\,1+i,\,2-i\,]^T \in \C^3, \quad \my = [\,1-2i,\,2i,\,-1-i\,]^T \in \C^3, 
\]
niin
\begin{gather*}
\scp{\mx}{\mx} = \abs{\mx}^2 = 1 + (1+1) + (4+1) = 8, \quad 
\scp{\my}{\my} = (1+4) + 4 + (1+1) = 11, \\
\scp{\mx}{\my} = i(1+2i) + (1+i)(-2i) + (2-i)(-1+i) = -1+3i. \quad \loppu
\end{gather*} 
\end{Exa}
Matriisia $\mv{A}^* = \bar{\mv{A}}^T$ sanotaan $\mv{A}$:n
\index{liittomatriisi} \index{matriisin ($\nel$neliömatriisin)!c@liittomatriisi}%
\kor{liittomatriisiksi} (engl.\ adjoint). Jos $\mx \in \C^n$, $\my \in \C^m$ ja \mA\ on
kompleksinen matriisi kokoa $m \times n$, niin liittomatriisin ja $\C^m$:n skalaaritulon
määritelmistä seuraa helposti (vrt. reaalinen tapaus edellä)
\[ 
\boxed{\quad \kehys \scp{\mA\mx}{\my} = \scp{\mx}{\mA^*\my}. \quad} 
\]
Jos $\mv{A} = \{a_{ij}\}_{i,j = 1}^n$ on neliömatriisi ja $\mv{A}^* = \mv{A}$, ts.\ 
$a_{ji} = \bar{a}_{ij}\ \forall i,j$, niin sanotaan, että $\mv{A}$ on
\index{hermiittinen matriisi} \index{neliömatriisi!b@hermiittinen}%
\kor{hermiittinen} (engl.\ Hermitean):
\[ 
\mv{A}^* = \mv{A} \quad \ekv \quad \mv{A}\,\ \text{hermiittinen}. 
\]
Jokainen reaalinen ja symmetrinen matriisi on määritelmän mukaan myös hermiittinen.
\begin{Exa} Yleinen hermiittinen matriisi kokoa $2 \times 2$ on muotoa
\[ 
\mv{A} = \begin{bmatrix} a & c \\ \bar{c} & b \end{bmatrix}, 
\]
missä $a,b \in \R$ ja $c \in \C$. \loppu 
\end{Exa}

\Harj
\begin{enumerate}

\item
Laske $2\mA+3\mB$, $\mA-\mB^T$, $\mA\mB$ ja $\mB\mA$, kun
\[
\mA=\begin{rmatrix} 1&-1&1 \\ -3&2&-1 \\ -2&1&0 \end{rmatrix}, \qquad
\mB=\begin{rmatrix} 1&2&3 \\ 2&4&6 \\ 1&2&3 \end{rmatrix}.
\]

\item
Olkoon
\[
\mA=\begin{bmatrix} 1&1&1\\1&1&1 \end{bmatrix}, \quad
\mB=\begin{rmatrix} 1&1&0\\1&1&0\\1&0&-1 \end{rmatrix}, \quad
\mC=\begin{rmatrix} 0&2\\1&1\\-1&-1 \end{rmatrix}.
\]
Laske $\mA\mB$, $\mA\mC$, $\mB\mC$, $\mC\mA$ ja $\mB\mA^T$. 

\item 
Millä matriisien tyyppiä koskevilla oletuksilla tulot $\mA\mB$ ja $\mB\mA$ ovat \newline
a) molemmat määriteltyjä, b) samaa tyyppiä?

\item
Olkoon 
\[
\ma=[1,3,5,2], \quad \mb=[-1,3,2,4]^T, \quad \mA=((-1)^{i+j},\, i=1,2,\, j=1,2,3,4).
\]
Laske vektorien/matriisien $\ma,\,\mb,\,\mA$ keskinäisistä (kaksittaisista) tuloista kaikki,
jotka ovat määriteltyjä.

\item
Matriisit $\mA$ ja $\mB$ ovat kokoa $10 \times 10$ ja niiden alkiot ovat $a_{ij}=i+j$ ja
$b_{ij}=i-j$. Laske tulomatriisin $\mC=\mA\mB$ alkio $c_{ij}$.

\item
Olkoon
\[
\mA=a\begin{bmatrix} 1&1&1\\1&1&1\\1&1&1 \end{bmatrix}, \quad
\mB=\begin{rmatrix} b&-c&-c\\-c&d&d\\-c&d&d \end{rmatrix}.
\]
Määritä luvut $a,b,c,d$ siten, että $\mA\neq\mv{0}$, $\mB\neq\mv{0}$, $\mA^2=\mA$, $\mB^2=\mB$
ja $\mA\mB=\mv{0}$. 

\item
Määritä matriisi $\mA$, kun tiedetään, että
\[
\mA\begin{bmatrix} 1\\1\\1 \end{bmatrix} = \begin{bmatrix} 1\\3\\4\\2 \end{bmatrix}, \quad
\mA\begin{bmatrix} 1\\1\\0 \end{bmatrix} = \begin{bmatrix} 1\\2\\3\\4 \end{bmatrix}, \quad
\mA\begin{bmatrix} 0\\1\\1 \end{bmatrix} = \begin{bmatrix} 4\\3\\2\\1 \end{bmatrix}.
\]

\item
Hae kaikki matriisit $\mB$, jotka kommutoivat matriisin
\[
\text{a)}\ \ \mA=\begin{rmatrix} 0&1\\0&-1 \end{rmatrix}, \qquad
\text{b)}\ \ \mA=\begin{rmatrix} 1&0\\0&-1 \end{rmatrix}
\]
kanssa. 

\item 
Olkoon $\mA$ ja $\mB$ samaa kokoa olevia neliömatriiseja. Todista: \vspace{1mm}\newline
a) \ $(\mA + \mB)^2=\mA^2+2\mA\mB+\mB^2$ $\ \ekv\ $ $\mA$ ja $\mB$ kommutoivat \newline
b) \ $\mA$ ja $\mB$ symmetriset $\ \not\impl\ $ $\mA\mB$ symmetrinen \newline
c) \ $\mA$ ja $\mB$ symmetriset ja kommutoivat $\ \impl\ $ $\mA\mB$ symmetrinen

\item
Laske \ a) $\mA^k\ \forall k\in\N$, \ b) $\mB=(\mI+\mA)^{100}$, kun
\[
\mA=\begin{bmatrix} 0&1&0&0\\0&0&1&0\\0&0&0&1\\0&0&0&0 \end{bmatrix}, \quad
\mI=\begin{bmatrix} 1&0&0&0\\0&1&0&0\\0&0&1&0\\0&0&0&1 \end{bmatrix}.
\]

\item \label{H-m-1: todistuksia} \index{neliömatriisi!c@vinosymmetrinen}
\index{vinosymmetrinen matriisi}
a) Todista matriisitulon transponointisääntö $(\mA\mB)^T=\mB^T\mA^T$. \vspace{1mm}\newline
b) Näytä, että jokainen neliömatriisi on esitettävissä yksikäsitteisesti muodossa 
$\mA=\mB+\mC$, missä $\mB$ on symmetrinen ja $\mC$ on \kor{vinosymmetrinen}: \newline
$\mC+\mC^T=\mv{0}$.

\item
Olkoon
\[
\mA=\begin{rmatrix} 5&-2&0\\-2&6&2\\0&2&7 \end{rmatrix}, \quad
\mb=\begin{rmatrix} 2\\-1\\3 \end{rmatrix}, \quad
\mx=\begin{bmatrix} x\\y\\z \end{bmatrix}.
\]
Sievennä lauseke $\,f(x,y,z)=\mx^T\mA\mx+\mb^T\mx\,$ toisen asteen polynomiksi.

\item
Laske $\mA\mx$, $\mB\mx$, $\mA\mB$, $(\mA\mB)^*$ ja $\mB^*\mA^*$, kun
\[
\mA=\begin{bmatrix} i&1-i&2+i\\1-i&-2i&3+2i \end{bmatrix}, \quad
\mB=\begin{bmatrix} 2-i&i&-i\\-i&2&1+i\\3i&2+2i&1-i \end{bmatrix}, \quad
\mx=\begin{bmatrix} 3\\3-i\\2+i \end{bmatrix}.
\]

\item (*) \label{H-m-1: matriisin normi} \index{matriisinormi} \index{normi!y@matriisinormi}
Jos reaalinen matriisi $\mA=(a_{ij})$ kokoa $m \times n$ tulkitaan avaruuden $\R^{mn}$
alkiona, niin ko.\ avaruuden euklidista normia vastaa \kor{matriisinormi}
\[
\norm{\mA} = \left(\sum_{i=1}^m\sum_{j=1}^n a_{ij}^2\right)^{1/2}.
\]
Näytä, että pätee $\,\abs{\mA\mx} \le \norm{\mA}\abs{\mx}\ \forall \mx\in\R^n$.

\item (*) 
Olkoon $x\in\R$ ja määritellään
\[
\mA       = \begin{bmatrix} x&1&0&0\\0&x&1&0\\0&0&x&1\\0&0&0&x\end{bmatrix}, \quad
\mI=\begin{bmatrix} 1&0&0&0\\0&1&0&0\\0&0&1&0\\0&0&0&1 \end{bmatrix}, \quad
\exp(\mA) = \mI+\sum_{k=1}^\infty \frac{1}{k!}\,\mA^k.
\]
Laske matriisin $\exp(\mA)$ alkiot $x$:n funktiona.

\item (*) \index{zzb@\nim!Naudat laitumella}
(Naudat laitumella) Nautalaumassa on on $83$ täysin ruskeata eläintä, $77$ sarvipäätä,
$36$ sukupuoleltaan sonnia, $22$ ruskeata sarvipäätä, $15$ ruskeata sonnia, $25$ sarvipäistä
sonnia ja $7$ ruskeata sarvipäistä sonnia. Kaikki lauman eläimet kuuluvat johonkin mainituista
ryhmistä. Montako eläintä laumassa on? \kor{Vihje}: Jaa eläimet pistevieraisiin ryhmiin,
esim.\ ruskeita sarvipäisiä sonneja $x_1$ kpl $\ldots$ ei-ruskeita sarvettomia lehmiä
$x_8$ kpl. Kirjoita lineaarinen yhtälöryhmä ja ratkaise!

\end{enumerate}
 %Matriisialgebra
\section{Neliömatriisit. Käänteismatriisi} \label{inverssi}
\alku
\index{neliömatriisi|vahv}

Tässä ja seuraavassa luvussa tutkimuskohteena on lineaarinen yhtälöryhmä
\[
\mA \mx = \mb,
\]
missä yhtälöitä ja tuntemattomia on yhtä monta, eli \mA\ on neliömatriisi kokoa $n \times n$ ja
$\mb \in \R^n\ (n \in \N)$. (Mukavuussyistä ajatellaan \mA\ ja \mb\ reaalisiksi --- tulokset 
jatkossa pätevät sellaisinaan myös kompleksialueella.)

Sekä matriisitulon että matriisi-vektoritulon kannalta tärkeä neliömatriisien erikoistapaus on
\index{yksikkömatriisi} \index{identiteettikuvaus, --matriisi}%
\kor{yksikkömatriisi} (engl.\ unit matrix) eli \kor{identiteettimatriisi}, jonka symboli on
$\mI$ ja määritelmä
\[
[\mI]_{ij} = (\delta_{ij}) 
           = \begin{bmatrix} 
             1&0&\ldots&0 \\ 0&1&\ldots&0\\ \vdots&\vdots&\ddots&\vdots \\ 0&0&\ldots&1 
             \end{bmatrix} 
           = [\me_1 \ldots \me_n].
\]
Jos $\mA$ ja $\mI$ ovat kokoa $n \times n$ ja $\mx$ on pystyvektori kokoa $n$, niin pätee
\[
\text{(a)}\ \ \mA\mI=\mI\mA=\mA, \qquad \text{(b)}\ \ \mI\mx=\mx.
\]
Ominaisuuden (a) mukaan $\mI$ on matriisikertolaskun ykkösalkio samankokoisten neliömatriisien 
välisissä operaatioissa --- tästä nimi 'yksikkömatriisi'. Ominaisuuden (b) mukaan kuvaus 
$\mx \map \mI \mx$ on $\R^n$:n \kor{identiteettikuvaus} --- tästä nimi 'identiteettimatriisi'.

Jos matriisien kertolaskuun liittyy 'ykkösmatriisi', niin liittyykö myös käänteismatriisi? 
--- Tässä tullaankin neliömatriisien teorian keskeisimpään kysymykseen.
\begin{Def} \label{käänteismatriisin määritelmä} \index{kzyzy@käänteismatriisi|emph}
\index{neliömatriisi!d@säännöllinen/singulaarinen|emph}
\index{szyzy@säännöllinen matriisi|emph} \index{singulaarinen matriisi|emph}
\index{ei-singulaarinen matriisi|emph} \index{kzyzy@kääntyvä matriisi}
\index{matriisin ($\nel$neliömatriisin)!d@$\nel$käänteismatriisi}
Neliömatriisi $\mA$ on \kor{säännöllinen} eli \kor{ei-singulaarinen} eli \kor{kääntyvä}, jos on
olemassa matriisi $\mB$ siten, että pätee
\[
\mA \mB = \mB \mA = \mI.
\]
Sanotaan, että $\mB$ on $\mA$:n \kor{käänteismatriisi} (engl.\ inverse matrix) ja merkitään
\[
\mB = \mA^{-1}.
\]
Jos $\mA$ ei ole säännöllinen, se on epäsäännöllinen eli \kor{singulaarinen}.
\end{Def}

Käänteismatriisin symboli $\mA^{-1}$ luetaan yleensä 'A miinus yksi'. Jos $\mA^{-1}$ on
olemassa, niin se on yksikäsitteinen. Nimittäin jos $\mB$ ja $\mC$ ovat molemmat $\mA$:n
käänteismatriiseja, niin seuraaan väittämän mukaan on oltava $\mB=\mC$.
\begin{Prop} \label{m-prop 1} Jos on olemassa matriisit $\mB$ ja $\mC$ kokoa $n \times n$
siten, että $\mA\mB=\mI$ ja $\mC\mA=\mI$, niin $\mB=\mC$.
\end{Prop}
\tod Oletuksien ja matriisitulon liitännäisyyden perusteella
\[
\mB = \mI \mB = (\mC \mA) \mB = \mC (\mA \mB) = \mC \mI = \mC. \loppu
\]

Jos $\mA$ on säännöllinen, niin Määritelmästä \ref{käänteismatriisin määritelmä} nähdään 
välittömästi, että myös $\mB=\mA^{-1}$ on säännöllinen ja $\mB^{-1}=\mA$. Suorittamalla 
määritelmässä transponointi nähdään myös, että
\[
\mA \mB = \mB \mA = \mI \ \ekv \ \mB^T \mA^T = \mA^T \mB^T = \mI^T = \mI.
\]
Siis: Jos $\mA$ on säännöllinen, niin sekä $\mA^{-1}$ että $\mA^T$ ovat säännöllisiä, ja pätee
\[
\boxed{\quad\kehys (\mA^{-1})^{-1}=\mA, \qquad (\mA^T)^{-1} = (\mA^{-1})^T. \quad}
\]
Jos $\mA$ ja $\mB$ ovat säännöllisiä ja samaa kokoa, niin myös tulo $\mA \mB$ on säännöllinen.
Nimittäin matriisitulon säännöin ja käänteismatriisin määritelmän perusteella
\begin{align*}
(\mA \mB)(\inv{\mB}\inv{\mA}) &= \mA(\mB \inv{\mB})\inv{\mA} 
                               = \mA \mI \inv{\mA} = \mA \inv{\mA} = \mI, \\
(\inv{\mB}\inv{\mA})(\mA \mB) &= \inv{\mB}(\inv{\mA} \mA)\mB 
                               = \inv{\mB} \mI \mB = \inv{\mB} \mB = \mI,
\end{align*}
joten
\[
\boxed{\quad\kehys \inv{(\mA \mB)} = \inv{\mB} \inv{\mA}. \quad}
\]
Tulos on helposti yleistettävissä seuraavasti (vrt.\ vastaava transponointisääntö edellisessä
luvussa)\,:
\begin{Prop} \label{matriisitulon säännöllisyys} Jos $\mA_k,\ k=1 \ldots m$ ovat säännöllisiä
ja samaa kokoa olevia neliömatriiseja, niin tulo $\mA=\mA_1\mA_2 \cdots \mA_m$ on myös 
säännöllinen ja $\mA^{-1}=\mA_m^{-1}\mA_{m-1}^{-1} \cdots \mA^{-1}$.
\end{Prop} 

Lineaarista yhtälöryhmää ratkaistaessa kerroinmatriisin säännöllisyys 'ratkaisee' ongelman 
periaatteelliselta kannalta seuraavasti:
\begin{Prop} \label{kerroinmatriisi} Jos yhtälöryhmässä $\mA\mx=\mb$ kerroinmatriisi $\mA$ on
säännöllinen neliömatriisi kokoa $n \times n$, niin yhtälöryhmällä on jokaisella $\mb \in \R^n$
yksikäsitteinen ratkaisu
\[
\mx = \inv{\mA} \mb.
\]
\end{Prop}
\tod Merkitään $\mB=\mA^{-1}$ ja päätellään:
\begin{align*}
\text{a)}\,\ \mA\mB=\mI &\qimpl \mA(\mB\mb) = (\mA\mB)\mb = \mI\mb = \mb. \\
\text{b)}\,\ \mB\mA=\mI &\qimpl \mx = \mI\mx = (\mB\mA)\mx = \mB(\mA\mx). 
\end{align*}
Tämän perusteella todetaan:\, a) $\mx=\mB\mb$ on yhtälöryhmän $\mA\mx=\mb$ ratkaisu.\newline
b) Jos $\mA\mx=\mb$, niin $\mx=\mB\mb$, eli tämä on ainoa mahdollinen ratkaisu. \loppu

Proposition \ref{kerroinmatriisi} mukaan yhtälöryhmän $\mA\mx=\mb$ kerroinmatriisin mahdollinen
singulaarisuus paljastuu yhälöryhmää ratkaistaessa:
\begin{Kor} \label{singulaarisuuskriteeri} Jos yhtälöryhmä $\mA\mx=\mb$ 
($\mA$ kokoa $n \times n$) joko ei ratkea jollakin $\mb\in\R^n$ tai ratkaisu ei ole
yksikäsitteinen, niin $\mA$ on singulaarinen.
\end{Kor}

\subsection{Neliömatriisin säännöllisyyskriteerit}

Proposition \ref{kerroinmatriisi} mukaan kerroinmatriisin $\mA$ säännöllisyys takaa
yhtälöryhmän $\mA\mx=\mb$ säännöllisyyden eli yksikäsitteisen ratkeavuuden jokaisella
$\mb\in\R^n$. Tämä väittämä pätee myös kääntäen, ja itse asiassa pätee paljon vahvempi tulos:
Jos yhtälöryhmä $\mA\mx=\mb$ joko ratkeaa yksikäsitteisesti kun $\mb=\mv{0}$ (eli ainoa
ratkaisu on $\mx=\mv{0}$) tai ratkeaa jokaisella $\mb\in\R^n$, niin kummassakin tapauksessa
$\mA$ on säännöllinen matriisi. Muotoillaan tulos seuraavasti:
\begin{*Lause} (\vahv{Neliömatriisin säännöllisyys}) \label{säännöllisyyskriteerit}
\index{matriisin ($\nel$neliömatriisin)!e@$\nel$säännöllisyyskriteerit|emph} Jos $\mA$ on
neliömatriisi kokoa $n \times n$, niin seuraavat väittämät ovat keskenään yhtäpitävät.
\begin{itemize}
\item[E0.] $\mA$ on säännöllinen matriisi.
\item[E1.] Yhtälöryhmän $\mA\mx=\mv{0}$ ainoa ratkaisu on $\mx=\mv{0}$.
\item[E2.] Yhtälöryhmällä $\mA\mx=\mb$ on ratkaisu jokaisella $\mb\in\R^n$.
\end{itemize}
\end{*Lause}
Lause \ref{säännöllisyyskriteerit} sisältää kuusi implikaatioväittämää, joista riippumattomia
ovat esim.: E0\,$\impl$\,E1, E0\,$\impl$\,E2, E1\,$\impl$\,E0 ja E2\,$\impl$\,E0. Näistä 
kaksi ensimmäistä ovat Propositioon \ref{kerroinmatriisi} sisältyviä ja siis jo selviä.
Kaksi muuta sen sijaan ovat syvällisempiä. Nämä sisältyvät erikoistapauksina yleisempään
lineaarisia yhtälöryhmiä koskevaan väittämään, joka tunnetaan nimellä 
\kor{Lineaarialgebran peruslause}. Yleisempää lausetta (joka koskee myös systeemejä kokoa 
$m \times n,\ m \neq n$) ei tässä muotoilla. Todetaan sen sijaan Lauseen
\ref{säännöllisyyskriteerit} mielenkiintoinen seuraamus:
\begin{Kor} \label{ihme} Samaa kokoa oleville neliömatriiseille pätee
\[
\mA\mB=\mI \qimpl \mB\mA=\mI.
\]
\end{Kor}
\tod Jos $\mA\mB=\mI$, niin säännöllisyyskriteeri E2 on täytetty (ks.\ Proposition
\ref{kerroinmatriisi} todistus, osa a)), joten Lauseen \ref{säännöllisyyskriteerit} mukaan
$\mA$ on säännöllinen. Tällöin jos $\mC=\mA^{-1}$, niin $\mA\mB=\mC\mA=\mI$, jolloin on oltava
$\mC=\mB$ (Propositio \ref{m-prop 1}). Siis $\mB\mA=\mI$. \loppu

Korollaarin \ref{ihme} mukaan siis jo ehto $\mA\mB=\mI$ riittää takaamaan, että $\mB=\mA^{-1}$
ja $\mA=\mB^{-1}$, ts.\ toinen Määritelmän \ref{käänteismatriisin määritelmä} ehdoista on
turha (!).

Jatkossa ei Lausetta \ref{säännöllisyyskriteerit} pyritä heti todistamaan, vaan seuraavissa
kahdessa luvussa tarkastellaan ensin lineaarisen yhtälöryhmän ratkaisemista algoritmisin
keinoin. Osoittautuu, että ratkaisualgoritmin sivutuotteena sadaan myös Lause
\ref{säännöllisyyskriteerit} todistetuksi.\footnote[2]{Lineaarialgebraa käsittelevässä
kirjallisuudessa todistetaan Lineaarialgebran peruslause (ja siihen sisältyen Lause
\ref{säännöllisyyskriteerit}) yleensä ei-algoritmisesti, abstraktin lineaarialgebran keinoin.}

Tämän luvun loppuosassa kohdisteaan huomio kahteen neliömatriisien erikoisluokkaan,
\kor{ortogonaali-} ja \kor{kolmio}matriiseihin ja edellisten alaluokkaan
\kor{permutaatio}matriiseihin. Permutaatio- ja kolmiomatriiseilla on jatkossa keskeinen rooli
lineaarisen yhtälöryhmän ratkaisualgoritmissa. Kuten nähdään, näiden matriisien osalta
säännöllisyyskysymys voidaan ratkaista tukeutumatta Lauseeseen \ref{säännöllisyyskriteerit}.
Kolmiomatriisin tapauksessa perustana on seuraavista kahdesta väittämästä jälkimmäinen (joka
nojaa edelliseen). Tämä muistuttaa Lauseen \ref{säännöllisyyskriteerit} väittämää E2\,\impl\,E0
mutta perustuu tätä vahvempiin oletuksiin. Muotoillaan ensin alkuperäinen oletusväittämä E2
toisella tavalla:
\begin{Prop} \label{m-prop 2} Pätee: \ E2 $\ \ekv\ \mA\mB=\mI\ $ jollakin $\mB$.
\end{Prop}
\tod \fbox{$\impl$} Jos E2 on tosi, niin on olemassa vektorit $\mb_k\in\R^n$ siten, että 
$\mA\mb_k=\me_k,\ k=1 \ldots n$. Tällöin jos $\mB=[\mb_1 \ldots \mb_n]$, niin 
$\mA\mB = [\mA\mb_1\,\ldots\,\mA\mb_n] = [\me_1 \ldots \me_n] = \mI$. \
\fbox{$\Leftarrow$} \ Ks. Proposition \ref{kerroinmatriisi} todistus, osa a). \loppu
\begin{Prop} \label{m-prop 3} Jos $\mA$ on kokoa $n \times n$ ja yhtälöryhmät $\mA\mx=\mb$
ja $\mA^T\mx=\mb$ ovat molemmat ratkeavia jokaisella $\mb\in\R^n$, niin $\mA$ on säännöllinen
matriisi.
\end{Prop}
\tod Oletuksien ja Propostition \ref{m-prop 2} perusteella on olemassa matriisit $\mB$ ja $\mC$
siten, että $\mA\mB=\mI$ ja $\mA^T\mC=\mI$. Tällöin on myös $(\mA^T\mC)^T=\mC^T\mA=\mI^T=\mI$.
Siis $\mA\mB=\mC^T\mA=\mI$, mistä seuraa (Propositio \ref{m-prop 1}), että
$\mB=\mC^T (=\mA^{-1})$, eli $\mA$ on säännöllinen. \loppu 
 
\subsection{Ortogonaalimatriisit}
\index{ortogonaalimatriisi|vahv}
\index{neliömatriisi!e@ortogonaalinen|vahv}
\index{ortogonaalisuus!c@matriisin|vahv}

Erikoisen säännöllisten matriisien luokan muodostavat \kor{ortogonaaliset} matriisit. Jos 
neliömatriisi $\mA$ esitetään sarakkeidensa avulla muodossa
\[
\mA = [\ma_1 \ldots \ma_n],
\]
niin sanotaan, että $\mA$ on ortogonaalinen, jos $\{\ma_1,\ldots,\ma_n\}$ on ortonormeerattu 
systeemi $\R^n$:ssä, ts.\
\[
\scp{\ma_i}{\ma_j} = \ma_i^T \cdot \ma_j = \delta_{ij}, \quad i,j=1 \ldots n.
\]  
Matriisitulon määritelmän perusteella (ks.\ edellinen luku) tämä on sama kuin ehto
\[
\mA^T \mA = \mI.
\]
Tällöin Korollaarin \ref{ihme} mukaan on myös $\mA\mA^T = \mI$, eli ortogonaalisessa
matriisissa myös rivit muodostavat ortonormeeratun systeemin (!) ja $\inv{\mA} = \mA^T$. 
Toisaalta, jos $\mA^T\mA = \mI$, niin
\[
\delta_{ij} = [\mA^T \mA]_{ij} = \ma_i^T \ma_j,
\]
eli $\mA$ on ortogonaalinen. On siis päädytty (Korollaariin \ref{ihme} ja siis Lauseeseen
\ref{säännöllisyyskriteerit} osaksi vedoten) tulokseen
\[
\boxed{\quad\kehys \mA \text{ ortogonaalinen} \ \ekv \ \inv{\mA} = \mA^T. \quad}
\]
Jos $\mA$ on sekä ortogonaalinen että \pain{s}y\pain{mmetrinen}, niin $\mA\mA=\mI$. Voidaan
siis todeta (tällä kertaa Lauseeseen \ref{säännöllisyyskriteerit} vetoamatta), että pätee:
\begin{Prop} \label{ortog ja sym} Jos $\mA$ on ortogonaalinen ja symmetrinen matriisi, niin 
$\mA$ on säännöllinen ja $\mA^{-1}=\mA$.
\end{Prop}

\subsection{Permutaatiomatriisi $\mI_p$}
\index{permutaatiomatriisi|vahv}

Jatkon kannalta erityisen kiinnostava ortogonaalisten matriisien luokka koostuu
\kor{permutaatiomatriiseista}, joissa on samat sarakkeet kuin yksikkömatriisissa $\mI$ mutta
vaihdetussa (permutoidussa) järjestyksessä. Permutaatiomatriisi merkitään
\[
\mI_p = [\me_{i_1}\ \me_{i_2}\ \ldots \me_{i_n}],
\]
missä $p=(i_1,i_2,\ldots,i_n)$ on järjestetyn lukujoukon $(1,2,\ldots,n)$
\index{permutaatio}%
\kor{permutaatio} (eli samatluvut eri järjestyksessä).
Permutaatio-opin tunnettu (helppo) tulos on, että lukujoukon
$(1,2,\ldots,n)$ erilaisia permutaatioita --- ja siis myös erilaisia permutaatiomatriiseja
$\mI_p$ kokoa $n \times n$ --- on $n!$ kpl. Jokainen permutaatio $p$ on saavutettavissa
suorittamalla ($p$:stä riippuva) määrä $m$ peräkkäisiä
\index{parivaihto}%
\kor{parivaihtoja}, joissa kaksi lukua
vaihdetaan keskenään. Luku $m$ ei ole yksikäsitteinen, koska saman parivaihdon toisto palauttaa
alkuperäisen järjestyksen. Sen sijaan on osoitettavissa (vaikka ei aivan helposti), että
luvun $m$ parillisuus/parittomuus on permutaatiolle ominainen, ts.\ jokainen permutaatio on
\index{parillinen, pariton!c@permutaatio}%
joko \kor{parillinen} tai \kor{pariton}.  

Jokainen permutaatiomatriisi $\mI_p$ on luonnollisesti ortogonaalinen, ts.\ $\mI_p^T\mI_p=\mI$.
Helposti on nähtävissä, että myös $\mI_p$:n rivit saadaan permutoimalla $\mI$:n rivit, eli
$\mI_p^T=\mI_q$, missä $q$ on toinen lukujen $(1,2,\ldots,n)$ permutaatio ($q=p$, jos $\mI_p$ on
symmetrinen). Koska $\mI=\mI_q^T\mI_q=\mI_p\mI_p^T$, niin $\mI_p^T\mI_p=\mI_p\mI_p^T=\mI$. Siis
on päätelty (Lauseeseen \ref{säännöllisyyskriteerit} vetoamatta), että $\mI_p^{-1}=\mI_p^T$.
\begin{Exa} Tapauksessa $n=3$ erilaisia permutaatiomatriiseja on $3!=6$ kpl, vastaten
permutaatioita $p=(1,2,3),(2,1,3),(1,3,2),(3,2,1),(2,3,1),(3,1,2)$. Näistä ensimmäinen ja
kaksi viimeistä ovat parillisia ($0$ tai $2$ parivaihtoa), muut parittomia ($1$ parivaihto).
Kuviosta (merkitty $\bullet=1$ ja $\cdot=0$) nähdään, että jos $p=(2,3,1)$, niin
$\mI_p^T=\mI_q$, missä $q=(3,1,2)$.
\[
\begin{bmatrix}
\bullet & \cdot & \cdot \\ 
\cdot & \bullet & \cdot \\
\cdot & \cdot & \bullet
\end{bmatrix}\,\ 
\begin{bmatrix}
\cdot & \bullet & \cdot \\ 
\bullet & \cdot & \cdot  \\
\cdot & \cdot & \bullet
\end{bmatrix}\,\
\begin{bmatrix}
\bullet & \cdot & \cdot \\ 
\cdot & \cdot & \bullet \\
\cdot & \bullet & \cdot
\end{bmatrix}\,\
\begin{bmatrix}
\cdot & \cdot & \bullet \\ 
\cdot & \bullet & \cdot \\
\bullet & \cdot & \cdot
\end{bmatrix}\,\
\begin{bmatrix}
\cdot & \cdot & \bullet \\ 
\bullet & \cdot & \cdot  \\
\cdot & \bullet & \cdot
\end{bmatrix}\,\
\begin{bmatrix}
\cdot & \bullet & \cdot \\ 
\cdot & \cdot & \bullet \\
\bullet & \cdot & \cdot
\end{bmatrix} \loppu
\]
\end{Exa}

Jos $\mA=[\ma_1 \ldots \ma_n]$ on matriisi kokoa $n \times n$ ja
$\mI_p=[\me_{i_1} \ldots \me_{i_n}]$, niin
\[
\mA\mI_p = [\mA\me_{i_1} \ldots \mA\me_{i_n}] = [\ma_{i_1} \ldots \ma_{i_n}].
\]
Operaatio $\mA \map \mA\mI_p$ siis permutoi $\mA$:n sarakkeet $\mI_p$:n sarakkeiden mukaiseen
järjestykseen $p$. Vastaavasti koska $(\mI_p\mA)^T = \mA^T\mI_p^T$, niin päätellään, että
operaatio $\mA \map \mI_p\mA$ permutoi $\mA^T$:n sarakkeet (eli $\mA$:n rivit) $\mI_p^T$:n
sarakkeiden (eli $\mI_p$:n rivien) mukaiseen järjestykseen. Yhteeneveto:
\[
\boxed{\ykehys \begin{aligned}
\quad&\mA\map\mA\mI_p\,: \quad \text{$\mA$:n sarakkeiden permutointi $\mI_p$:n 
                                             sarakkeiden mukaisesti}. \quad \\
     &\mA\map\mI_p\mA\,: \quad \text{$\mA$:n rivien permutointi $\mI_p$:n rivien mukaisesti}.
\end{aligned} \akehys}
\]
\jatko \begin{Exa} (jatko)
\[
\begin{bmatrix} 1&2&3\\4&5&6\\7&8&9 \end{bmatrix}
\begin{bmatrix} 0&1&0\\0&0&1\\1&0&0 \end{bmatrix} =
\begin{bmatrix} 3&1&2\\6&4&5\\9&7&8 \end{bmatrix}, \quad
\begin{bmatrix} 0&1&0\\0&0&1\\1&0&0 \end{bmatrix}
\begin{bmatrix} 1&2&3\\4&5&6\\7&8&9 \end{bmatrix} =
\begin{bmatrix} 4&5&6\\7&8&9\\1&2&3 \end{bmatrix}. \loppu
\]
\end{Exa}
Edellisen luvun matriisialgebran säännöistä voidaan päätellä, että matriisin sarakkeiden
permutointi operaatiolla $\mA\map\mA\mI_p$ toimii yhtä hyvin matriiseille kokoa $m \times n$,
$m \neq n$, ja vastaavasti rivien permutointi matriiseille kokoa $n \times m$. Neliömatriisin
tapauksessa nähdään myös, että sarakkeiden tai rivien permutoinnissa säännöllinen matriisi
säilyy säännöllisenä (koska säännöllisten matriisien tulo on säännöllinen). Samoin
singulaarinen matriisi säilyy singulaarisena (Harj.teht.\,\ref{H-m-2: pikku väittämiä}d).

Permutaatiomatriisin erikoistapaus on \kor{vaihtomatriisi}, jossa vain kaksi $\mI$:n saraketta
on vaihdettu keskenään. Jos vaihdetut sarakkeet ovat $\me_k$ ja $\me_m$ ($k<m$), niin
vaihtomatriisi on $\mV = [\,\me_1\,\ldots\,\me_{k-1}\ \me_m\ \me_{k+1}\,\ldots\,\me_{m-1}\,
\me_k\ \me_{m+1}\,\ldots\,\me_n\,]$. Tämä on paitsi ortogonaalinen myös symmetrinen:
\[
\mV= \begin{bmatrix}
1 \\
& \ddots \\
& & 0 & \hdotsfor{3} & 1 \\
& & & 1 \\
& & & & \ddots \\
& & & & & 1 \\
& & 1 & \hdotsfor{3} & 0 \\
& & & & & & & 1 \\
& & & & & & & & 1 \\
& & & & & & & & &1 \\
\end{bmatrix}
\]
Koska permutaatioon $p$ päästään peräkkäisillä parivaihdoilla, niin vastaavasti $\mI_p$
saadaan kertomalla $\mI$ oikealta vastaavilla vaihtomatriiseilla $\mV_i,\ i=1 \ldots m$, eli
\[
\mI_p = \mI\mV_1 \ldots \mV_m = \mV_1 \ldots \mV_m.
\]
Koska jokainen $\mV_i$ on symmetrinen, niin tulon transponointisäännön perusteella
\[
\mI_p^{-1} = \mI_p^T = \mV_m \ldots \mV_1.
\]

\subsection{Kolmiomatriisit}
\index{kolmiomatriisi|vahv}

\begin{Def} \index{ylzy@yläkolmiomatriisi|emph} \index{alakolmiomatriisi|emph}
\index{diagonaalimatriisi|emph}
\index{neliömatriisi!ea@diagonaalinen|emph}
Neliömatriisi $\mA = (a_{ij})$ on
\begin{itemize}
\item[-] \kor{yläkolmiomatriisi}, jos $\,a_{ij}=0$, kun $i>j$,
\item[-] \kor{alakolmiomatriisi}, jos $\,a_{ij}=0$, kun $i<j$,
\item[-] \kor{diagonaalinen} l. \kor{diagonaalimatriisi}, jos $\,a_{ij} = 0$, kun $i \neq j$.
\end{itemize}
\end{Def}
\[
\begin{array}{ccc}
\begin{bmatrix} 
\# & \# & \# & \# \\ 0 & \# & \# & \# \\ 0 & 0 & \# & \# \\ 0 & 0 & 0 & \# 
\end{bmatrix} \quad & 
\begin{bmatrix} 
\# & 0 & 0 & 0 \\ \# & \# & 0 & 0 \\ \# & \# & \# & 0 \\ \# & \# & \# & \# 
\end{bmatrix} \quad &
\begin{bmatrix} 
\# & 0 & 0 & 0 \\ 0 & \# & 0 & 0 \\ 0 & 0 & \# & 0 \\ 0 & 0 & 0 & \# 
\end{bmatrix} \\ \\
\text{yläkolmio}\ \ \ & \text{alakolmio}\ \ & \text{diagonaalinen}
\end{array}
\] 
Määritelmän mukaisesti diagonaalimatriisi on molempien kolmiomatriisien
erikoistapaus. Määritelmän matriisityypeille käytetään usein erikoissymboleja $\mU$ 
(yläkolmio, engl.\ Upper triangular), $\mL$ (alakolmio, engl.\ Lower triangular) ja $\mD$ 
(diagonaalimatriisi). Neliömatriisin $\mA$ alkioita $a_{ii}$ sanotaan yleisesti $\mA$:n 
\index{lzy@lävistäjä, -alkio} \index{diagonaali (matriisin)}%
\kor{lävistäjäalkioiksi} ja (järjestettyä) joukkoa  $(a_{ii},\ i = 1 \ldots n)$ $\mA$:n
\kor{lävistäjäksi} eli \kor{diagonaaliksi}. Joukko $(a_{i,i+k},\ i = 1 \ldots n-k)$ on
vastaavasti
\index{ylzy@ylädiagonaali} \index{aladiagonaali}%
$k$:s \kor{ylädiagonaali} ja $(a_{i-k,i},\ i = k+1 \ldots n)$ $k$:s \kor{aladiagonaali}.
Diagonaalimatriisi esitetään usein lävistäjäalkioidensa avulla käyttäen merkintää 
\[
\mD = \text{diag} \, (d_i), \quad d_i=[\mD]_{ii}\,.
\]

Kolmiomatriisin säännöllisyyskysymyksen ratkaisee
\begin{Lause} \label{kolmiomatriisi} \vahv{(Kolmiomatriisin säännöllisyys)}\, 
Jos $\mA = (a_{ij})$ on kolmiomatriisi kokoa $n \times n$, niin $\mA$ on säännöllinen 
täsmälleen kun lävistäjäalkiot $a_{ii}$  ovat kaikki nollasta poikkeavat, ja tällöin pätee
\begin{itemize}
\item[(i)]  $\mA$ diagonaalinen/yläkolmio/alakolmio \\ $\impl$ $\inv{\mA}$ 
            diagonaalinen/yläkolmio/alakolmio
\item[(ii)] $[\inv{\mA}]_{ii} = 1/a_{ii}, \quad i=1 \ldots n.$
\end{itemize}
\end{Lause}
\tod Diagonaalimatriisin tapauksessa nähdään, että jos $a_{kk}=0$ jollakin 
$k \in \{1,\ldots,n\}$, niin yhtälöryhmällä $\mA \mx = \mo$ on monikäsitteinen ratkaisu 
$\mx = x_k \me_k,\ x_k\in\R$, joten Korollaarin \ref{singulaarisuuskriteeri} mukaan $\mA$ on 
singulaarinen. Jos $a_{ii} \neq 0$, $i = 1\ldots n$, niin $\mA \inv{\mA} = \inv{\mA} \mA = \mI$
toteutuu väitteen mukaisella valinnalla, eli
\[
\inv{\mA} = \text{diag} \, (1/a_{ii}).
\]

Oletetaan seuraavaksi, että $\mA$ on alakolmiomatriisi, jolloin yhtälöryhmä $\mA\mx=\mb$ on
auki kirjoitettuna
\[
\begin{cases}
\,a_{11} x_1 &= b_1, \\
\,a_{21} x_1 + a_{22} x_2 &= b_2, \\
\ \vdots & \ \vdots \\
\,a_{n1} x_1 + a_{n2} x_2 + \ldots + a_{nn} x_n &= b_n.
\end{cases}
\]
Jos $a_{ii}\neq 0\ \forall i$, niin yhtälöryhmä ratkeaa purkamalla se palautuvasti alusta:
\begin{align*}
x_1\ &=\ \inv{a_{11}}b_1\ =\ b_{11}b_1, \\[2mm]
x_2\ &=\ -\inv{a_{22}}a_{21}x_1 + \inv{a_{22}}b_2\ 
      =\ -\inv{a_{22}}a_{21}b_{11}b_1 + \inv{a_{22}}b_2
      =\ b_{21}b_1 + b_{22}b_2,
\end{align*}
ja yleisesti (induktio!)
\[
x_i = \sum_{j=1}^i b_{ij}b_j, \quad i = 1 \ldots n,
\]
missä $b_{ii} = \inv{a_{ii}}$. Kun asetetaan $b_{ij}=0,\ j>i$, ja $\mB = \{b_{ij}\}$, niin
$\mB$ on siis alakolmiomatriisi, lävistäjäalkioin $b_{ii}=\inv{a_{ii}}$, ja yhtälöryhmän
ratkaisu on
\[
\mx = \mB\mb.
\]
Tähän siis päädyttiin olettaen, että $a_{ii}\neq 0\ \forall i$. Jos tämä oletus ei toteudu, 
niin jollakin $k \in \{1, \ldots, n\}$ pätee: $\,a_{kk} = 0$ ja 
$a_{ii} \neq 0, \ i=1 \ldots k-1$. Silloin nähdään em.\ algoritmista, että jos valitaan 
$b_i=0, \ i=1 \ldots k-1$, niin on oltava $x_i = 0,\ i=1 \ldots k-1$, jolloin $k$:s yhtälö saa
muodon $0=b_k$. Näin ollen yhtälöryhmä $\mA\mx=\mb$ ei yleisesti ratkea, joten Korollaarin
\ref{singulaarisuuskriteeri} perusteella $\mA$ on singulaarinen.

Jos $\mA$ on yläkolmiomatriisi, niin yhtälöryhmä $\mA\mx=\mb$ purkautuu lopusta lukien:
Ratkaistaan ensin $n$:s yhtälö $a_{nn}x_n=b_n$, sitten $(n-1)$:s yhtälö, jne. Tässäkin
tapauksessa päätellään, että $\mA$ on singulaarinen, jos $a_{kk} = 0$ jollakin $k$, muuten
löytyy matriisi $\mB$, jolle pätee $\mA\mB=\mI$. Matriisi $\mB$ on jälleen samaa tyyppiä kuin
$\mA$ (yläkolmio), ja lävistäjäalkiot ovat $b_{ii}=\inv{a_{ii}}$.

Olkoon nyt $\mA$ yleisemmin kolmiomatriisi (ylä- tai alakolmio) ja $\mA$:n lävistäjäalkiot
nollasta poikkeavat. Tällöin $\mA^T$ on myös kolmiomatriisi, jonka lävistäjäalkiot ovat samat 
kuin $\mA$:n. Em.\ päättelyn mukaan yhtälöryhmät $\mA\mx=\mb$ ja $\mA^T\mx=\mb$ ovat
molemmat ratkeavia jokaisella $\mb\in\R^n$, joten $\mA$ on säännöllinen matriisi
(Propositio \ref{m-prop 3}). Kolmiomatriisin säännöllisyyskysymys on näin ratkaistu Lauseesta
\ref{säännöllisyyskriteerit} riippumatta. \loppu

Em.\ todistuksen sivutuotteena saatiin myös algoritmi kolmiomatriisin $\mA$ käänteismatriisin
laskemiseksi: Ratkaistaan (todistuksessa esitetyllä tavalla) lineaarinen yhtälöryhmä
$\mA\mx=\mb$ y\pain{leisellä} $\mb\in\R^n$. Kun ratkaisu esitetään muodossa
\[
x_i=\sum_{j=1}^n b_{ij}b_j, \quad i=1 \ldots n \qekv \mx=\mB\mb,
\]
niin $\mB=\mA^{-1}$, eli kertoimet $b_{ij}$ ovat käänteismatriisin alkiot. Nämä tulevat
algoritmin kuluesssa lasketuksi palautuvasti riveittäin. Vain diagonaalisen (tai muulla tavoin
erikoisen, ks.\ Harj.teht.\,\ref{H-m-2: matriiseja alkioittain}b) matriisin tapauksessa on
käänteismatriisin $\mA^{-1}$ alkioille mahdollista laskea yksinkertaiset lausekkeet $\mA$:n
alkioiden avulla.

\Harj
\begin{enumerate}

\item \label{H-m-2: pikku väittämiä}
Olkoon $\mA$, $\mB$ ja $\mC$ samaa kokoa olevia neliömatriiseja. Todista: \vspace{1mm}\newline
a) \ $\mA$ säännöllinen ja symmetrinen $\ \impl\ $ $\mA^{-1}$ symmetrinen. \newline
b) \ $\mC\mA=\mC\mB$ ja $\mC$ säännöllinen $\ \impl\ $ $\mA=\mB$. \newline
c) \ $\mA\mB$ singulaarinen $\ \impl\ $ $\mA$ tai $\mB$ singulaarinen. \newline
d) \ $\mA$ singulaarinen ja $\mB$ säännöllinen $\ \impl\ $ 
                                               $\mA\mB$ ja $\mB\mA$ singulaariset. \newline
e) \ $\mA$ ja $\mB$ ortogonaaliset $\ \impl\ $ $\mA\mB$ ortogonaalinen.

\item
Tarkista kokoa $2 \times 2$ olevan matriisin käänteismatriisin laskusääntö
\[
\begin{bmatrix} \,a&b\,\\\,c&d\, \end{bmatrix}^{-1}
=\ \frac{1}{D}\begin{rmatrix} d&-b\\-c&a \end{rmatrix}, \quad D=ad-bc \neq 0. 
\]

\item
a) Näytä, että jokainen ortogonaalinen $2 \times 2$-matriisi voidaan kirjoittaa jollakin
$\theta\in\R$ jompaan kumpaan seuraavista muodoista:
\[
\mA=\begin{rmatrix} \cos\theta&\sin\theta\\-\sin\theta&\cos\theta \end{rmatrix}
\quad \text{tai} \quad
\mA=\begin{rmatrix} \cos\theta&\sin\theta\\\sin\theta&-\cos\theta \end{rmatrix}.
\]
b) Olkoon $\mA$ kokoa $3 \times 3$ oleva ortogonaalimatriisi, jonka alkioista tiedetään:
$a_{11}=\frac{3}{7},\ a_{12}=-\frac{2}{7},\ a_{22}=\frac{6}{7},\ a_{21}<0,\ a_{31}>0,\ a_{13}<0$.
Laske $\mA$ ja $\mA^{-1}$. \vspace{1mm}\newline
c) Totea, että yhtälöryhmässä
\[
\begin{cases}
\,\sqrt{\frac 23}\,x_1+\frac 12\,x_2-\frac{1}{2\sqrt{3}}\,x_3          & = 1\\
\,\frac{1}{\sqrt 3}\,x_1-\frac{1}{\sqrt 2}\,x_2+\frac{1}{\sqrt 6}\,x_3 & = 2\\
\,-\frac{1}{2}\,x_2-\frac{\sqrt 3}{2}\,x_3                             & =  3
\end{cases}
\]
kerroinmatriisi on ortogonaalinen. Ratkaise tätä tietoa käyttäen! \vspace{1mm}\newline
d) Matriisilla
\[
\mA=\begin{rmatrix} 1&1&1&1\\1&1&-1&-1&\\1&-1&1&-1\\-1&1&1&-1 \end{rmatrix}
\]
on ominaisuus: $\lambda\mA$ on ortogonaalinen eräällä $\lambda\in\R$. Määritä tätä tietoa
hyväksi käyttäen vaakavektori $\mx^T$ siten, että $\mx^T\mA=[7,13,-3,-9]$.

\item
a) Olkoon $\mH=\mI-\mx\mx^T$, missä $\mx\in\R^n$, $\mx\neq\mv{0}$ ja $\abs{\mx} \neq 1$. Näytä,
että eräällä $\lambda\in\R$ pätee $\mH^{-1}=\mI+\lambda\mx\mx^T$. \newline
b) Olkoon $\mx\in\R^n,\ \mx\neq\mv{0}$ ja $r=2\abs{\mx}^{-2}$. Näytä, että $\mH=\mI-r\,\mx\mx^T$
on symmetrinen ja ortogonaalinen matriisi. \newline
c) Näytä, että jos matriisille $\mA$ pätee $\mA+\mA^T=\mv{0}$ ja $\mI+\mA$ on säännöllinen
matriisi, niin $\mB=(\mI+\mA)^{-1}(\mI-\mA)$ on ortogonaalinen.

\item
Onko joukon $(1,2,\ldots,10)$ permutaatio $(4,6,2,7,9,1,3,10,5,8)$ parillinen vai pariton?

\item 
Matriiseista $\mV_1$ ja $\mV_2$ kokoa $3\times3$ tiedetään, että kertoessaan vasemmalta
matriisin $\mA$ matriisi $\mV_1$ vaihtaa $\mA:$n rivien 1 ja 2 järjestyksen ja $\mV_2$ rivien
2 ja 3 järjestyksen. Millaisen rivien permutoinnin silloin tuottavat $\mV_1\mV_2$ ja
$\mV_2\mV_1$? Määritä myös $\mV_1$ ja $\mV_2$ sekä mainitut tulot.

\item \label{H-m-2: matriiseja alkioittain} 
a) Olkoon $\mA=(a_{ij})$ kokoa $n\times n$ ja $\mD=\text{diag}\{d_i,\, i=1 \ldots n\}$.
Määritä matriisien $\mA\mD$ ja $\mD\mA$ alkiot. \newline
b) Olkoon $n,k\in\N,\ n \ge 2$, $k \le n$, $\mA=(a_{ij})$ kokoa $n \times n$,
$a_{ii}=a \neq 0$ kun $i \neq k$, $a_{kk}=b \neq 0$, $a_{ik}=c \neq 0$ kun $i>k$ ja $a_{ij}=0$
muulloin. Määritä käänteismatriisin $\mA^{-1}$ alkiot.

\item
Määritä seuraavien kolmiomatriisien käänteismatriisit ratkaisemalla yleinen lineaarinen
yhtälöryhmä, jonka kerroinmatriisina on ko.\ matriisi.
\begin{align*}
&\text{a)}\ \ \begin{rmatrix} 3&0\\-5&2 \end{rmatrix} \qquad
 \text{b)}\ \ \begin{rmatrix} 4&0&0\\-2&2&0\\3&1&-1 \end{rmatrix} \qquad
 \text{c)}\ \ \begin{rmatrix} 3&0&-1&1\\0&2&1&1\\0&0&-1&2\\0&0&0&2 \end{rmatrix} \\[1mm]
&\text{d)}\ \ \begin{bmatrix} 
              1&0&0&0&0\\1&1&0&0&0\\0&1&1&0&0\\0&0&1&1&0\\0&0&0&1&1 
              \end{bmatrix} \qquad
 \text{e)}\ \ \begin{bmatrix}
              1&0&0&0&1\\0&1&0&0&1\\0&0&1&0&1\\0&0&0&1&1\\0&0&0&0&2
              \end{bmatrix} \qquad
 \text{f)}\ \ \begin{bmatrix}
              1&0&0&0&0\\1&1&0&0&0\\1&1&1&0&0\\1&1&1&1&0\\1&1&1&1&1
              \end{bmatrix}   
\end{align*}

\item (*) Näytä Lauseeseen \ref{säännöllisyyskriteerit} vedoten, että neliömatriiseille
pätee: \vspace{1mm}\newline
a)\,\ $\mA\mB\,\ \text{säännöllinen}\ \qimpl \mA\ \text{ja}\ \mB\ \text{säännölliset}$.\newline
b)\,\ $\mA^2+3\mA+2\mI=\mv{0} \qimpl \mA\,\ \text{säännöllinen}$.

\item (*)
Näytä, että on olemassa permutaatiomatriisit $\mU$ ja $\mV$ siten, että pätee
\[
\mA=\begin{bmatrix} 0&2&1&0\\0&1&0&0\\2&2&2&1\\1&2&2&0 \end{bmatrix}
   =\mU \begin{bmatrix} 1&2&2&2\\0&1&2&2\\0&0&1&2\\0&0&0&1 \end{bmatrix} \mV.
\]
Laske käänteismatriisi $\mA^{-1}$ tämän tiedon avulla. Tarkista, että $\mA\mA^{-1}=\mI$.

\item (*)
Olkoon $a,b\in\R$ ja $b \neq 0$. Näytä matriisialgebran avulla, että pätee: 
Differentiaaliyhtälöllä $\,y''+ay'+by=x^{100}\,$ on yksittäisratkaisuna polynomi astetta
$100$.
 
\end{enumerate}

 %Neliömatriisit. Käänteismatriisi
\section{Gaussin algoritmi} \label{Gaussin algoritmi}
\alku
\index{Gaussin algoritmi (eliminaatio)|vahv}

Lineaarisia yhtälöryhmiä, jotka eivät ole kooltaan aivan hirmuisia (esim. $n<10^4$) ratkotaan
käytännössä usein \kor{Gaussin algoritmilla} eli \kor{Gaussin eliminaatiolla}. Algoritmilla on
monia variaatioita, ja niillä monia nimiä, mutta pääidea on kaikissa sama.

Lähdetään ratkaisemaan yhtälöryhmää
\[
\left\{
\begin{alignedat}{3}
a_{11} &x_1 + a_{12} &x_2 + \ldots + a_{1n}         &x_n  \ &= \ &b_1,    \\
a_{21} &x_1 + a_{22} &x_2 + \ldots + a_{2n}         &x_n  \ &= \ &b_2,    \\
       &\vdots       & \vdots \qquad \qquad \quad \ &\vdots &    &\vdots \\
a_{i1} &x_1 + a_{i2} &x_2 + \ldots + a_{in}         &x_n  \ &= \ &b_i,    \\
       &\vdots       & \vdots \qquad \qquad \quad \ &\vdots &    &\vdots \\
a_{n1} &x_1 + a_{n2} &x_2 + \ldots + a_{nn}         &x_n  \ &= \ &b_n. \
\end{alignedat}
\right.
\]
Gaussin algoritmissa oletetaan, että yhtälöryhmällä on ratkaisu $(x_i)$ ja toimitaan ikäänkuin
ratkaisu olisi sijoitettu tuntemattomien paikalle. Ratkaisun ei tarvitse olla yksikäsitteinen
--- tämä selviää algoritmin kuluessa. Samoin selviää, onko ratkaisua ylipäänsä olemassa.

Gaussin algoritmissa yhtälöryhmä muunnetaan askelittain yksinkertaisempaan muotoon. Muunnetut
yhtälöryhmät ovat alkuperäisen kanssa ekvivalenttisia (eli muunnokset voidaan tehdä molempiin 
suuntiin), joten mitään informaatiota ei laskentaprosessin kuluessa menetetä.

Olettaen, että $a_{11} \neq 0$ (tapaukseen $a_{11} = 0$ palataan myöhemmin), Gaussin algoritmin
ensimmäinen nk.\ 
\index{eliminaatioaskel (Gaussin alg.)}%
\kor{eliminaatioaskel} on: Ratkaistaan 1. yhtälöstä
\[
x_1=-\sum_{j=2}^n \frac{a_{1j}}{a_{11}}\,x_j + \frac{b_1}{a_{11}}
\]
ja sijoitetaan tämä lauseke muihin yhtälöihin $x_1$:n paikalle. Sievennysten jälkeen on 
tuloksena alkuperäisen kanssa ekvivalenttinen yhtälöryhmä 
\[
\left\{
\begin{alignedat}{2}
a_{11} x_1 + a_{12} &x_2 + \ldots +       a_{1n}  &x_n   &= b_1, \\
       a_{22}^{(1)} &x_2 + \ldots + a_{2n}^{(1)}  &x_n   &= b_2^{(1)}, \\
                    & \vdots \qquad \qquad  \quad \ \vdots & &  \quad \vdots \\
       a_{i2}^{(1)} &x_2 + \ldots + a_{in}^{(1)}  &x_n  &=  b_i^{(1)}, \\
                    & \vdots \qquad \qquad  \quad \ \vdots & &  \quad \vdots \\
       a_{n2}^{(1)} &x_2 + \ldots + a_{nn}^{(1)}  &x_n   &=  b_n^{(1)}. \
\end{alignedat}
\right.
\]
Tässä ensimmäinen yhtälö on jätetty alkuperäiseen muotoonsa ja on merkitty
\begin{align*}
a_{ij}^{(1)} &= a_{ij} - a_{i1}a_{1j}/a_{11}, \quad i,j = 2 \ldots n, \\
b_i^{(1)} &= b_i - a_{i1}b_1/a_{11}, \quad i = 2 \ldots n.
\end{align*}
Taulukkomuodossa laskien tämä muunnos toteutuu, kun kerrotaan 1. yhtälö puolittain luvulla 
$-a_{i1}/a_{11}$, lasketaan tulos puolittain yhteen $i$:nnen yhtälön kanssa, ja korvataan 
tuloksella aiempi $i$:s yhtälö, $i=2 \ldots n$. 

Ensimmäisen eliminaatioaskeleen jälkeinen yhtälöryhmä on matriisimuodossa
\[
\begin{bmatrix}
a_{11} & a_{12}       & \ldots & a_{1n} \\
     0 & a_{22}^{(1)} & \ldots & a_{2n}^{(1)} \\
\vdots & \vdots       &        & \vdots \\
     0 & a_{n2}^{(1)} & \ldots & a_{nn}^{(1)}
\end{bmatrix}
\begin{bmatrix}
x_1 \\ x_2 \\ \vdots \\ x_n
\end{bmatrix} =
\begin{bmatrix}
b_1 \\ \ b_2^{(1)} \\ \vdots \\ \ b_n^{(1)}
\end{bmatrix}.
\]
Koska tässä yhtälöryhmässä $x_1$ esiintyy vain ensimmäisessä yhtälössä, voidaan tämä yhtälö 
jättää jatkossa omaan rauhaansa ja tarkastella supistettua yhtälöryhmää
\[
\begin{bmatrix}
a_{22}^{(1)} & a_{23}^{(1)} & \ldots & a_{2n}^{(1)} \\
a_{33}^{(1)} & a_{33}^{(1)} & \ldots & a_{3n}^{(1)} \\
\vdots & \vdots      &      & \vdots \\
a_{n2}^{(1)} & a_{n3}^{(1)} & \ldots & a_{nn}^{(1)}
\end{bmatrix}
\begin{bmatrix}
x_2 \\ x_3 \\ \vdots \\ x_n
\end{bmatrix} =
\begin{bmatrix}
b_2^{(1)} \\ b_3^{(1)} \\ \vdots \\ b_n^{(1)}
\end{bmatrix}.
\]
Jos $a_{22}^{(1)} \neq 0$, suoritetaan tähän yhtälöryhmään samanlainen eliminaatiomuunnos kuin
edellä, jolloin se muuntuu muotoon
\[
\begin{bmatrix}
a_{22}^{(1)} & a_{23}^{(1)} & \ldots & a_{2n}^{(1)} \\
0 & a_{33}^{(2)} & \ldots & a_{3n}^{(2)} \\
\vdots & \vdots      &      & \vdots \\
0 & a_{n3}^{(2)} & \ldots & a_{nn}^{(2)}
\end{bmatrix}
\begin{bmatrix}
x_2 \\ x_3 \\ \vdots \\ x_n
\end{bmatrix} =
\begin{bmatrix}
b_2^{(1)} \\ b_3^{(2)} \\ \vdots \\ b_n^{(2)}
\end{bmatrix}.
\]
Tässä $x_2$ esiintyy vain ensimmäisessä yhtälössä, joten yhtälöryhmä supistuu jälleen. 
Yleisesti on $k-1$ eliminaatioaskeleen jälkeen koko yhtälöryhmä saatu muotoon
\[
\left[\begin{array}{llllll}
a_{11}  &  a_{12} & \ldots & a_{1k} & \ldots & a_{1n} \\
0       &  a_{22}^{(1)} & \ldots & a_{2k}^{(1)} & \ldots & a_{2n}^{(1)} \\
\vdots  & \vdots        & \ddots \\
0       & 0             & \ldots & a_{kk}^{(k-1)} & \ldots & a_{kn}^{(k-1)} \\
\vdots  & \vdots        &        & \vdots \\
0       & 0             & \ldots & a_{nk}^{(k-1)} & \ldots & a_{nn}^{(k-1)}
\end{array}\right]
\begin{bmatrix}
x_1 \\ x_2 \\ \vdots \\ x_k \\ \vdots \\ x_n
\end{bmatrix} =
\left[\begin{array}{l}
b_1 \\ b_2^{(1)} \\ \vdots \\ b_k^{(k-1)} \\ \vdots \\ b_n^{(k-1)}
\end{array}\right].
\]
Edellytyksenä tähän muotoon pääsemiseksi on siis, että
\[
a_{ll}^{(l-1)} \neq 0, \quad l=1 \ldots k-1,
\]
missä $a_{11}^{(0)}=a_{11}$. Alkiota $a_{kk}^{(k-1)}$ sanotaan seuraavan eli $k$:nnen 
eliminaatioaskeleen
\index{tukialkio (Gaussin alg.)}%
\kor{tukialkioksi} (engl.\ pivot element). Mikäli kaikki tukialkiot ovat
nollasta poikkeavia ja myös $a_{nn}^{(n-1)} \neq 0$, päädytään $n-1$ askeleen jälkeen
yhtälöryhmään, jonka kerroinmatriisi on säännöllinen kolmiomatriisi:
\[
\left[\begin{array}{llllll}
a_{11} & a_{22} & \ldots & a_{1n} \\
0      & a_{22}^{(1)} & \ldots & a_{2n}^{(1)} \\
\vdots & \vdots & \ddots & \vdots \\
0 & 0 & \ldots  & a_{nn}^{(n-1)}
\end{array}\right]
 \begin{bmatrix}
x_1 \\ x_2 \\ \vdots \\ x_n
\end{bmatrix} =
\left[\begin{array}{l}
b_1 \\ b_2^{(1)} \\ \vdots \\ b_n^{(k-1)}
\end{array}\right].
\]
Kuten pian nähdään, algoritmin läpimeno tällä tavoin osoittaa, että alkuperäisen yhtälöryhmän
kerroinmatriisi $\mA$ on säännöllinen. Joka tapauksessa muunnettu yhtälöryhmä ratkeaa helposti
purkamalla se lopusta lukien, vrt.\ Lauseen \ref{kolmiomatriisi} todistus edellä. Tätä 
algoritmin vaihetta kutsutaan
\index{takaisinsijoitus (Gaussin alg.)}%
\kor{takaisinsijoitukseksi}.
\begin{Exa} \label{ristikkoesimerkki} (Ks.\ Luku \ref{yhtälöryhmät}, Esimerkki \ref{ristikko}) 
Ratkaise Gaussin eliminaatiolla yhtälöryhmä
\[
\begin{rmatrix} 3&-1&0&0&-1&1\\1&-3&0&2&-1&1\\0&0&3&1&-1&-1\\0&-2&1&3&-1&-1\\1&-1&1&1&-4&0\\
                -1&1&\ \ 1&\ \ 1&0&-4\end{rmatrix}
\begin{rmatrix} x_1\\x_2\\x_3\\x_4\\x_5\\x_6 \end{rmatrix} 
\ =\ F \begin{rmatrix} 0\\1\\0\\0\\0\\0 \end{rmatrix} 
   + G \begin{rmatrix} 0\\0\\0\\1\\0\\0 \end{rmatrix},
\]
missä $F,G\in\R$. 
\end{Exa}
\ratk Ratkaisu on $\mx=F\mx_1+G\mx_2$, missä $\mA\mx_1=\mb_1=[0\ 1\ 0\ 0\ 0\ 0]^T$ ja 
$\mA\mx_2=\mb_2=[0\ 0\ 0\ 1\ 0\ 0]^T$, joten on ratkaistava kaksi lineaarista yhtälöryhmää. 
Gaussin algoritmilla voidaan molemmat yhtälöryhmät käsitellä samanaikaisesti: Kirjoitetaan 
yhtälöryhmät taulukkomuotoon
\[ 
       \left[\begin{array}{rrrrrr} 
             3&-1&0&0&-1&1\\1&-3&0&2&-1&1\\0&0&3&1&-1&-1\\
             0&-2&1&3&-1&-1\\ 1&-1&1&1&-4&0\\-1&1&\ \ 1&\ \ 1&0&-4 
             \end{array} \right.
\left. \left|\begin{array}{rr} 
             \ \ 0&0\\1&0\\0&0\\0&1\\0&0\\0&\ \ 0 
             \end{array} \right. \right]
\]
ja ulotetaan eliminaatio koko taulukkoon. Seuraavassa muunnettu taulukko kunkin 
eliminaatioaskeleen jälkeen.
\begin{align*}
&(k=1) \qquad \left[\begin{array}{rrrrrr}
                    3&-1&0&0&-1&1\\0&-\frac{8}{3}&0&2&-\frac{2}{3}&\frac{2}{3}\\
                    0&0&3&1&-1&-1\\0&-2&1&3&-1&-11 \\
                    0&-\frac{2}{3}&1&1&-\frac{11}{3}&-\frac{1}{3} \\[0.5mm]
                    0&\frac{2}{3}&1&1&-\frac{1}{3}&-\frac{11}{3} 
                    \end{array} \right.
       \left. \left|\begin{array}{rr} 
                    \ \ 0&0\\1&0\\0&0\\0&1\\0&0\\0&\ \ 0 
                    \end{array} \right. \right] \\[5mm]
&(k=2) \qquad \left[\begin{array}{rrrrrr} 
                    3&-1&0&0&-1&1\\0&-\frac{8}{3}&0&2&-\frac{2}{3}&\frac{2}{3}\\0&0&3&1&-1&-1\\
                    0&0&1&\frac{3}{2}&-\frac{1}{2}&\ \ -\frac{3}{2}\\[0.5mm]
                    0&0&1&\frac{1}{2}&-\frac{7}{2}&-\frac{1}{2}\\[0.5mm]
                    0&0&1&\frac{3}{2}&-\frac{1}{2}&-\frac{7}{2} 
                    \end{array} \right.
       \left. \left|\begin{array}{rr}
                    0&0\\1&0\\0&0\\-\frac{3}{4}&1\\-\frac{1}{4}&0\\ \frac{1}{4} & \ \ 0
                    \end{array} \right. \right] \\[5mm]
&(k=3) \qquad \left[\begin{array}{rrrrrr} 
                    3&-1&0&0&-1&1\\0&-\frac{8}{3}&0&2&-\frac{2}{3}&\frac{2}{3}\\0&0&3&1&-1&-1\\
                    0&0&0&\frac{7}{6}&-\frac{1}{6}&-\frac{7}{6}\\[0.5mm]
                    0&0&0&\frac{1}{6}&-\frac{19}{6}&-\frac{1}{6}\\[0.5mm]
                    0&0&0&\frac{7}{6}&-\frac{1}{6}&-\frac{19}{6} 
                    \end{array} \right.
       \left. \left|\begin{array}{rr}
                    0&0\\1&0\\0&0\\-\frac{3}{4}&1\\-\frac{1}{4}&0\\ \frac{1}{4} & \ \ 0
                    \end{array} \right. \right] \\[5mm]
&(k=4) \qquad \left[\begin{array}{rrrrrr} 
                   3&-1&0&0&-1&1\\0&-\frac{8}{3}&0&2&-\frac{2}{3}&\frac{2}{3}\\0&0&3&1&-1&-1\\
                   0&0&0&\frac{7}{6}&-\frac{1}{6}&-\frac{7}{6}\\[0.5mm]
                   0&0&0&0&-\frac{22}{7}&0\\0&0&0&0&0&\ -2
                   \end{array} \right.
       \left. \left|\begin{array}{rr}
                    0&0\\1&0\\0&0\\-\frac{3}{4}&1\\-\frac{1}{7}&-\frac{1}{7}\\1&-1
                    \end{array} \right. \right]
\end{align*}
Askelta $k=5$ ei tässä tapauksessa tarvita, koska on $a_{65}^{(4)}=0$.

Ratkaisu saadaan tästä takaisinsijoituksella. --- Käytetään tässä kuitenkin toista, laskutyön
kannalta samanarvoista mahdollisuutta, jossa eliminaatioalgoritmia käytetään uudelleen 
'takaperin'. Koska kerroinmatriisi on yläkolmio, niin takaisin päin eliminoitaessa muuttuu
matriisissa kullakin eliminaatioaskeleella vain tukialkion kohdalla oleva sarake, tai tarkemmin
tämän lävistäjän yläpuolinen osa. Eliminaatiovaiheet takaisin päin ovat seuraavat.
\begin{align*}
&(k=5) \qquad \left[\begin{array}{rrrrrr}
                    3&-1&0&0&-1&0\\[0.5mm] 0&-\frac{8}{3}&0&2&-\frac{2}{3}&0\\[0.5mm]
                    0&0&3&1&-1&0\\[0.5mm] 0&0&0&\frac{7}{6}&-\frac{1}{6}&0\\[0.5mm]
                    0&0&0&0&-\frac{22}{7}&0\\0&0&\ \ 0&\ \ 0&0&-2 
                    \end{array} \right.
       \left. \left|\begin{array}{rr}
                    \frac{1}{2}&-\frac{1}{2}\\[0.5mm]\frac{4}{3}&-\frac{1}{3}\\[0.5mm]
                    -\frac{1}{2}&\frac{1}{2}\\[0.5mm]-\frac{4}{3}&\frac{19}{12}\\[0.5mm]
                    \ \ \ -\frac{1}{7}&\ -\frac{1}{7}\\ 1&-1
                    \end{array} \right. \right] \\[5mm]
&(k=4) \qquad \left[\begin{array}{rrrrrr}
                  3&-1&0&0&0&0\\[0.5mm] 0&-\frac{8}{3}&0&2&0&0\\[0.5mm] 0&0&3&1&0&0\\[0.5mm]
                  0&0&0&\frac{7}{6}&0&0\\[0.5mm] 0&0&0&0&-\frac{22}{7}&0\\0&0&\ \ 0&\ \ 0&0&-2 
                  \end{array} \right.
       \left. \left|\begin{array}{rr}
                    \frac{6}{11}&-\frac{5}{11}\\[0.5mm]\frac{15}{11}&-\frac{10}{33}\\[0.5mm]
                    -\frac{5}{11}&\frac{6}{11}\\[0.5mm]\ -\frac{175}{132}&\frac{35}{22}\\[0.5mm]
                    -\frac{1}{7}&-\frac{1}{7}\\ 1&-1
                    \end{array} \right. \right] \\[5mm]
&(k=3) \qquad \left[\begin{array}{rrrrrr} 
                  3&-1&0&0&0&0\\[0.5mm] 0&-\frac{8}{3}&0&0&0&0\\[0.5mm] 0&0&3&0&0&0\\[0.5mm]
                  0&0&0&\frac{7}{6}&0&0\\[0.5mm] 0&0&0&0&-\frac{22}{7}&0\\0&0&\ \ 0&\ \ 0&0&-2 
                  \end{array} \right.
       \left. \left|\begin{array}{rr}
                    \frac{6}{11}&-\frac{5}{11}\\[0.5mm]\frac{40}{11}&-\frac{100}{33}\\[0.5mm]
                    \frac{15}{22}&-\frac{9}{11}\\[0.5mm]-\frac{175}{132}&\frac{35}{22}\\[0.5mm]
                    -\frac{1}{7}&-\frac{1}{7}\\ 1&-1
                    \end{array} \right. \right] \\[5mm]
&(k=2) \qquad \left[\begin{array}{rrrrrr} 
                     3&0&0&0&0&0\\[0.5mm] 0&-\frac{8}{3}&0&0&0&0\\[0.5mm] 0&0&3&0&0&0\\[0.5mm]
                     0&0&0&\frac{7}{6}&0&0\\[0.5mm] 0&0&0&0&-\frac{22}{7}&0\\
                     0&0&\ \ 0&\ \ 0&0&-2 
                    \end{array} \right.
       \left. \left|\begin{array}{rr}
                    -\frac{9}{11}&\frac{15}{22}\\[0.5mm]\frac{40}{11}&-\frac{100}{33}\\[0.5mm]
                    \frac{15}{22}&-\frac{9}{11}\\[0.5mm]-\frac{175}{132}&\frac{35}{22}\\[0.5mm]
                    -\frac{1}{7}&-\frac{1}{7}\\1&-1
                    \end{array} \right. \right]
\intertext{Lopuksi diagonaalisen systeemin ratkaisu:}
&\phantom{(k=2)} \qquad 
              \left[\begin{array}{rrrrrr} 
                    1&\,\ \ 0&\,\ \ 0&\,\ \ 0&\,\ \ 0&\,\ \ 0\\[0.5mm] 0&1&0&0&0&0\\[0.5mm] 
                    0&0&1&0&0&0\\[0.5mm]
                    0&0&0&1&0&0\\[0.5mm] 0&0&0&0&1&0\\[0.5mm] 0&0&0&0&0&1 
                    \end{array} \right.
       \left. \left|\begin{array}{rr}
                    -\frac{3}{11}&\frac{5}{22}\\[0.5mm]-\frac{15}{11}&\frac{25}{22}\\[0.5mm]
                    \frac{5}{22}&\ -\frac{3}{11}\\[0.5mm]\ -\frac{25}{22}&\frac{30}{22}\\[0.5mm]
                    \frac{1}{22}&\frac{1}{22}\\[0.5mm]-\frac{1}{2}&\frac{1}{2}
                    \end{array} \right. \right].
\end{align*}

Kysytty yhtälöryhmän ratkaisu on näin muodoin
\[ 
\begin{bmatrix} x_1\\x_2\\x_3\\x_4\\x_5\\x_6 \end{bmatrix}\ =\
               \frac{F}{22} \begin{rmatrix} -6\\-30\\5\\-25\\1\\-11 \end{rmatrix}
              +\frac{G}{22} \begin{rmatrix} 5\\25\\-6\\30\\1\\11 \end{rmatrix}. \quad \loppu
\]

\subsection{Gaussin algoritmin työmäärä}
\index{Gaussin algoritmi (eliminaatio)!c@työmäärä|vahv}

Gaussin algoritmi koostuu siis kahdesta vaiheesta, eliminaatiovaiheesta ja 
takaisinsijoituksesta. Englannin kielessä käytetään myös termejä 'forward sweep' ja
'backward sweep'. Näistä ensimmäinen 'pyyhkäisy' on työmäärää (laskenta-aikaa) ajatellen 
huomattavasti raskaampi. Tämä nähdään eliminaatiovaiheen yleisestä algoritmimuodosta, joka on
\begin{align*}
a_{ij}^{(k)} &= a_{ij}^{(k-1)} - a_{ik}^{(k-1)}a_{kj}^{(k-1)}/a_{kk}^{(k-1)}, \quad 
                                                                         i,j = k+1 \ldots n, \\
b_i^{(k)} &= b_i^{(k-1)} - a_{ik}^{(k-1)}b_k^{(k-1)}/a_{kk}^{(k-1)}, \quad i = k+1 \ldots n.
\end{align*}
Tässä on jakolaskut $\,a_{ik}^{(k-1)}/a_{kk}^{(k-1)}\,,\ i=k+1 \ldots n\,$ syytä laskea ensin,
jolloin algoritmi on ohjelmointikielellä
\begin{align*}
l_{ik}\ &\leftarrow\ a_{ik}/a_{kk}\,, \qquad\ \ i=k+1 \ldots n, \\
a_{ij}\ &\leftarrow\ a_{ij} - l_{ik}a_{kj}\,, \quad\, i,j=k+1 \ldots n, \\
b_i\    &\leftarrow\ b_i - l_{ik}b_k\,, \qquad i=k+1 \ldots n.
\end{align*}
Työmäärää arvioitaessa pidettäköön työyksikkönä laskuoperaatiota, joka koostuu yhdestä 
kertolaskusta (tai jakolaskusta) ja yhdestä yhteen- tai vähennyslaskusta. Algoritmista nähdään,
että algoritmin työläin osa koostuu muunnoksista $\,a_{ij}^{(k)} \leftarrow a_{ij}^{(k-1)}$.
Näiden vaatima työmäärä koko eliminaatiovaiheessa on mainittuina operaatioina
\[
W = (n-1)^2+(n-2)^2 + \cdots + 1 = \frac{1}{6}(n-1)n(2n-1).
\]
Muut algoritmin osat, kuten muunnokset $b_i^{(k)} \leftarrow b_i^{(k-1)}$ ja takaisinsijoitukset
ovat työmäärältään suuruusluokkaa $\ordoO{n^2}$, joten Gaussin algoritmin työmäärä on
\[
\boxed{\quad W\ =\ \frac{1}{3}\,n^3 + \ordoO{n^2}. \quad}
\]
Tästä nähdään, että esimerkiksi tuhannen yhtälön yhtälöryhmän ($W \approx 0.3 \cdot 10^9$) 
ratkaiseminen tietokoneella ei ole ongelma laskenta-ajan kannalta. Sen sijaan yhtälöryhmä
kokoa $n=10^6$ ($W \approx 0.3 \cdot 10^{18}$) saattaa jo olla ylivoimainen tehtävä. ---
Näinkin suuria yhtälöryhmiä ratkotaan nykyisin, mutta silloin käytetään yleensä hyväksi
kerroinmatriisin erikoisominaisuuksia. Toinen, perinteisesti suosittu vaihtoehto on käyttää
tehtävään sopivia
\kor{iteratiivisia} (likimääräisiä) ratkaisutapoja. Suhteessa iteratiivisiin menetelmiin
sanotaan Gaussin algoritmiin perustuvaa ratkaisua \kor{suoraksi} ratkaisutavaksi.
Kohtuullisilla $n$:n arvoilla yhtälöryhmän suora ratkaisu Gaussin algoritmilla on edelleen
kilpailukykyinen menetelmä, ja tietokoneiden suorituskyvyn kasvaessa se on myös voittanut alaa
iteratiivisilta menetelmiltä. Etenkin silloin kun algoritmi menee läpi em. suoraviivaisella
tavalla se on myös hyvin helppo ohjelmoida.

\subsection{Neliömatriisin $LU$-hajotelma}
\index{LU@$LU$-hajotelma (neliömatriisin)|vahv}
\index{matriisin ($\nel$neliömatriisin)!f@$\nel$$LU$-hajotelma|vahv}

Gaussin eliminaatiolla, sikäli kuin se onnistuu edellä kuvatulla tavalla, saadaan yhtälöryhmälle
$\mA\mx=\mb$ ratkaisu jokaisella $\mb\in\R^n$, joten Lauseen \ref{säännöllisyyskriteerit}
perusteella kerroinmatriisin on oltava säännöllinen. Ko.\ lauseeseen (jota ei vielä ole 
todistettu) ei kuitenkaan tarvitse vedota, sillä kerroinmatriisin säännöllisyys tulee Gaussin
algoritmilla suoraan todistetuksi. Päättely on seuraava: Ensinnäkin todetaan, että Gaussin
algoritmin eliminaatiovaihe muuntaa alkuperäisen yhtälöryhmän
\[
\mA \mx = \mb
\]
muotoon
\[
\mU \mx = \mc = \mR \mb,
\]
missä $\mU$ on muunnettu kerroinmatriisi, eli yläkolmiomatriisi, jonka lävistäjäalkiot ovat 
$a_{kk}^{(k-1)} \neq 0, \ k=1 \ldots n$. Matriisi $\mR$ ei ole eliminaatiovaiheen jälkeen 
suoraan nähtävissä, mutta se saadaan selville tarkastelemalla yhtälöryhmää
\[
\mL \my = \mb,
\]
missä $\mL$ on alakolmiomatriisi, joka määritellään
\[
[\mL]_{ik} = \begin{cases}
\,0, &\text{jos}\ i<k, \\
\,1, &\text{jos}\ i=k, \\
\,l_{ik}=a_{ik}^{(k-1)}/a_{kk}^{(k-1)}\,, &\text{jos}\ i>k.
\end{cases}
\]
Tässä siis $a_{ik}^{(k-1)}, \ i=k \ldots n$, ovat muunnetun kerroinmatriisin alkioita $k-1$ 
eliminaatioaskeleen jälkeen:
\[
\begin{array}{ccc}
a_{k-1,k-1}^{(k-2)} \\
0 & a_{kk}^{(k-1)} \\
\vdots & \vdots \\
0 & a_{ik}^{(k-1)} & \rightarrow\ l_{ik} = a_{ik}^{(k-1)}/a_{kk}^{(k-1)} \\
\vdots & \vdots \\
0 & a_{nk}^{(k-1)}
\end{array}
\]
Matriisin $\mL$ muodosta nähdään, että jos em.\ yhtälöryhmään sovelletaan Gaussin algoritmia, 
niin jo eliminaatiovaihe antaa ratkaisun $\my=\mL^{-1}\mb$. Toisaalta nähdään vertaamalla 
Gaussin algoritmeja sovellettuna yhtälöryhmiin $\mL\my=\mb$ ja $\mA\mx=\mb$, että 
eliminaatiovaiheen muunnokset $b_i^{(k)} \leftarrow b_i^{(k-1)}$ ovat kummassakin tapauksessa 
täsmälleen samat. Siis on päätelty:
\[
\my = \inv{\mL} \mb = \mR \mb.
\]
Koska tämä pätee $\forall \mb \in \R^n$, on $\mR=\inv{\mL}$ 
(Propositio \ref{matriisien yhtäsuuruus}), ja näin ollen
\[
\mA \mx = \mb \ \ekv \ \mL \mU \mx = \mb
\]
Siis $\mA \mx = \mL \mU \mx \ \forall \mx \in \R^n$, joten
\[
\boxed{\kehys\quad \mA=\mL\mU. \quad}
\]
Tätä sanotaan matriisin \kor{$LU$-hajotelmaksi} (engl. $LU$-decomposition). Sen laskemiseksi ei
siis tarvita muuta kuin Gaussin algoritmin eliminaatiovaihe, kunhan muistetaan tallettaa 
matriisi $\mL$, eli luvut
\[
[\mL]_{ik} = l_{ik} = a_{ik}^{(k-1)}/a_{kk}^{(k-1)}, \quad i=k+1 \ldots n, \ k=1  \ldots n-1.
\]
Nämä tulevat eliminaation kuluessa joka tapauksessa lasketuksi, mutta ilman talletusta $\mL$ ei
jää näkyviin (koska yhtälöryhmän ratkaisualgoritmi laskee valmiiksi vektorin 
$\inv{\mL} \mb = \mR \mb$).
\jatko \begin{Exa} (jatko) Seuraamalla eliminaatiovaiheita ($k=1\,\ldots\,4$) nähdään, että
esimerkin yhtälöryhmässä kerroinmatriisin $LU$-hajotelma on
\[
\mA =\begin{rmatrix} 
     1&0&0&0&0&0 \\ \frac{1}{3}&1&0&0&0&0 \\ 0&0&1&0&0&0 \\ 
     0&\frac{3}{4}&\frac{1}{3}&1&0&0 \\[0.5mm]
     \frac{1}{3}&\frac{1}{4}&\frac{1}{3}&\frac{1}{7}&1&0 \\[0.5mm] 
     -\frac{1}{3}&-\frac{1}{4}&\frac{1}{3}&1&0&1
     \end{rmatrix}
     \begin{rmatrix} 
     3&-1&0&0&-1&1\\0&-\frac{8}{3}&0&2&-\frac{2}{3}&\frac{2}{3} \\[0.5mm]
     0&0&3&1&-1&-1\\0&0&0&\frac{7}{6}&-\frac{1}{6}&-\frac{7}{6} \\[0.5mm] 
     0&0&0&0&-\frac{22}{7}&0\\0&0&0&0&0&-2 
     \end{rmatrix} = \mL\mU. \loppu
\]
\end{Exa} 

Matriisin $LU$-hajotelma siis onnistuu Gaussin eliminaatiolla, jos 
$\,a_{kk}^{(k-1)} \neq 0, \ k=1 \ldots n$. Tällöin $\mL$ ja $\mU$ ovat säännöllisiä 
(Lause \ref{kolmiomatriisi}), joten $\mA$ on säännöllinen ja
\[
\inv{\mA} = \inv{\mU} \inv{\mL}.
\]
Mainittu tukialkioita koskeva ehto on siis riittävä ehto matriisin säännöllisyydelle. 
Seuraavassa luvussa nähdään, että kun Gaussin algoritmia hieman muunnellaan, se ratkaisee 
neliömatriisin säännöllisyyskysymyksen yleisessäkin tapauksessa. 

\subsection{Käänteismatriisin laskeminen}

Sikäli kuin Gaussin eliminaatio yhtälöryhmälle $\mA \mx = \mb$ onnistuu edellä kuvatulla tavalla
(tukialkiot $\neq 0$), niin kerroinmatriisi on siis säännöllinen, jolloin yhtälöryhmän ratkaisu
voidaan kirjoittaa $\mx=\mA^{-1}\mb$. Jos halutaan laskea myös käänteismatriisi $\mA^{-1}$
eikä vain ratkaisua $\mx$, niin eräs (etenkin käsinlaskussa luonteva) menettely on soveltaa
Gaussin algoritmia \pain{s}y\pain{mbolisesti} siten, että $\mb$:n alkioille $\,b_i$ ei anneta
numeroarvoja. Algoritmia käyttäen ratkaisu saadaan tällöin muotoon $\mx = \mB \mb$, jolloin 
on $\mA^{-1}=\mB$.
\begin{Exa} Laske matriisin
\[ 
\mA\ =\ \begin{rmatrix} 1&1&0 \\ 1&0&1 \\ 1&-1&-1 \end{rmatrix} 
\]
käänteismatriisi. 
\end{Exa}
\ratk Eliminaatiolla ja takaisinsijoituksella saadaan
\begin{align*}
&\begin{rmatrix} 1&1&0 \\ 1&0&1 \\ 1&-1&-1 \end{rmatrix} 
 \begin{rmatrix} x_1 \\ x_2 \\ x_3 \end{rmatrix}\ =\ 
 \begin{rmatrix} b_1 \\ b_2 \\ b_3 \end{rmatrix} \\[5mm] \map \quad 
&\begin{rmatrix} 1&1&0 \\ 0&-1&1 \\ 0&0&-3 \end{rmatrix} 
 \begin{rmatrix} x_1 \\ x_2 \\ x_3 \end{rmatrix}\ =\ 
 \begin{cmatrix}  b_1 \\ -b_1 + b_2 \\ 3b_1 - 2b_2 + b_3 \end{cmatrix} \\[5mm] \map \quad
&\begin{rmatrix} 1&0&0 \\ 0&1&0 \\ 0&0&1 \end{rmatrix}   
 \begin{rmatrix} x_1 \\ x_2 \\ x_3 \end{rmatrix}\ =\ 
 \dfrac{1}{3} \begin{cmatrix}  
              b_1 + b_2 + b_3 \\ 2b_1 - b_2 - b_3 \\ -b_1 + 2b_2 - b_3 
              \end{cmatrix}.
\end{align*}
Tuloksesta voidaan lukea käänteismatriisi:
\[ 
\begin{rmatrix} x_1 \\ x_2 \\ x_3 \end{rmatrix}\
    =\ \dfrac{1}{3} \begin{rmatrix} 1&1&1 \\ 2&-1&-1 \\ -1&2&-1 \end{rmatrix}
                    \begin{rmatrix} b_1 \\ b_2 \\ b_3 \end{rmatrix}
    =\ \mA^{-1}\mb. \loppu
\]
Edellistä suoraviivaisempi tapa laskea käänteismatriisi on käyttää hyväksi matriisialgebran
identieettiä (vrt.\ Luku \ref{matriisialgebra})
\[
\mA^{-1}\ =\ \mA^{-1} \mI\ =\ \mA^{-1}\,[\me_1 \ldots \me_n]\ 
                           =\  [\mA^{-1}\me_1 \ldots \mA^{-1}\me_n].
\]
Tämän mukaisesti $\mA^{-1}$:n sarakkeet voidaan laskea ratkaisemalla lineaarinen yhtälöryhmä
$\mA \mx = \mb$ kun $\mb = \me_i,\ i = 1 \ldots n$. Koska Gaussin algoritmissa voidaan käsitellä
yhtä aikaa useita eri vektoreita $\mb$ (vrt.\ Esimerkki \ref{ristikkoesimerkki}), voidaan myös 
$\mA^{-1}$:n sarakkeet määrätä kaikki yhdellä kertaa. Käänteismatriisin laskeminen voidaan 
tällöin kuvata kaksivaiheisena operaationa (eliminaatio ja takaisinsijoitus)
\[
 [\,\mA \mid \mI\ ]\ \map\ [\,\mU \mid \mL^{-1}\,]\ \map\ [\,\mI \mid \mA^{-1}\ ].
\]
Vastaavalla tavalla menetellään itse asiassa em.\ symbolisessa laskussa, sillä kun kirjoitetaan
$\mb = \sum_{i=1}^n b_i \me_i$, niin pätee (vrt.\ Luku \ref{matriisialgebra})
\[
\mA^{-1} \mb\ =\ \sum_{i=1}^n b_i \mA^{-1} \me_i\ =\ [\mA^{-1} \me_1 \ldots \mA^{-1} \me_n]\,\mb.
\]

\jatko \begin{Exa} (jatko) Esimerkin lasku toisin organisoituna:
\begin{align*}
[\,\mA \mid \mI\ ]\ 
    &=\  \left[ \begin{array}{rrr} 1&1&0 \\ 1&0&1 \\ 1&-1&-1 \end{array} \left\vert 
                 \begin{array}{rrr} 1&0&0 \\ 0&1&0 \\ 0&0&1 \end{array} \right. \right] \quad 
     \map \quad 
         \left[ \begin{array}{rrr} 1&1&0 \\ 0&-1&1 \\ 0&0&-3 \end{array} \left\vert
                \begin{array}{rrr} 1&0&0 \\ -1&1&0 \\ 1&-2&1 \end{array} \right. \right] \\[5mm]
    &\,\map \quad 
         \left[ \begin{array}{rrr} 1&0&0 \\[0.5mm] 0&1&0 \\[0.5mm] 0&0&1 \end{array} \left\vert
                \begin{array}{rrr} \frac{1}{3} & \frac{1}{3} & \frac{1}{3} \\[0.5mm] 
                                   \frac{2}{3} & -\frac{1}{3} & -\frac{1}{3} \\[0.5mm]
                                  -\frac{1}{3} & \frac{2}{3} & -\frac{1}{3} \end{array} \right.
         \right]\ =\ [\,\mI \mid \mA^{-1}\ ]. \quad \loppu
\end{align*}
\end{Exa}
Käänteismatriisi voidaan myös laskea käyttäen $LU$-hajotelmaan perustuvaa kaavaa
$\mA^{-1}=\mU^{-1}\mL^{-1}$, eli laskemalla ensin ym.\ tavalla erikseen käänteismatriisit 
$\mL^{-1}$ ja $\mU^{-1}$ ja sitten näiden tulo. Sekä tällä tavoin että ym.\ algoritmin 
mukaisesti laskien tulee työmääräksi $W=n^3+\ordoO{n^2}$ 
(Harj.teht.\,\ref{H-m-3: työmääriä 2}) --- siis noin kolminkertainen työmäärä verrattuna
$LU$-hajotelman laskemiseen tai yksittäisen yhtälöryhmän ratkaisuun. Osoittautuukin, että 
yhtälöryhmiä numeerisesti ratkaistaessa ei käänteismatriisin laskeminen yksinkertaisesti 
kannata. Nimittäin vaikka ratkaistaisiin useita yhtälöryhmiä kerroinmatriisin 
pysyessä samana (esim.\ vaihteleva kuormitus ristikkorakenteessa, vrt.\ Luku 
\ref{yhtälöryhmät}), niin Gaussin algoritmin eliminaatiovaiheen toistaminen pystytään
välttämään pelkän $LU$-hajotelman avulla: Kun hajotelman matriisit $\mL$ ja $\mU$ tallennetaan, 
niin yhtälöryhmä $\mA \mx = \mb$ voidaan purkaa kahdeksi peräkkäiseksi yhtälöryhmäksi:
\[ 
\mL\mU\mx=\mb \qekv \mL \my = \mb, \quad \mU \mx = \my. 
\]
Nämä ratkeavat eteen- ja takaisinsijoituksilla, jolloin yhtälöryhmän ratkaisemisen työmääräksi
tulee $W \sim n^2$ (ks.\ Harj.teht.\,\ref{H-m-3: työmääriä 1}b). Jos $\mA^{-1}$ olisi
tallennettu, olisi työmäärä sama, joten käänteismatriisin laskemisella ei voiteta mitään (!). 

\Harj
\begin{enumerate}


\item
Kirjoita seuraavat yhtälöryhmät taulukkomuotoon ja ratkaise Gaussin algoritmilla. Laske myös
kerroinmatriisien $LU$-hajotelmat.
\begin{align*}
&\text{a)}\ \ \begin{cases} \,x+y=33 \\ \,2x+4y=100 \end{cases} \qquad\qquad
 \text{b)}\ \ \begin{cases} \,x+ay=1 \\ \,ax+y=2 \end{cases} \quad (a \neq \pm 1) \\[2mm]
&\text{c)}\ \ \begin{cases} 
              \,x_1+3x_2-x_3=4 \\ \,-x_1+x_2+2x_3=7 \\ \,2x_1+6x_2+2x_3=20
              \end{cases} \quad
 \text{d)}\ \ \begin{cases}
              \,x_1+2x_2+3x_3=1 \\ \,x_1+x_2-x_3=2 \\ \,x_1-6x_3=4
              \end{cases} \\[2mm]
&\text{e)}\ \ \begin{cases} 
              \,x_1+x_2+2x_3=-1 \\ \,-x_1+x_2=1 \\ \,3x_1-x_2+3x_3=-3
              \end{cases} \quad\,
 \text{f)}\ \ \begin{cases}
              \,x_1-x_2+4x_3=8 \\ \,x_1-2x_2+3x_3=0 \\ \,-2x_1+4x_2-5x_3=8
              \end{cases} \\[2mm]
&\text{g)}\ \ \begin{cases}
              \,2x_1-x_2=1 \\ \,-x_1+2x_2-x_3=2 \\ \,-x_2+2x_3-x_4=-1 \\ \,-x_3+2x_4=1
              \end{cases} \quad\ \
 \text{h)}\ \ \begin{cases}
              \,2x_1-x_2-x_4=-4 \\ \,-x_1+2x_2-x_3=1 \\ \,-x_2+2x_3-x_4=4 \\ \,-x_1-x_3+3x_4=10
              \end{cases}
\end{align*}

\item
Olkoon
\[
\mA   = \begin{bmatrix} 1&2&3&1\\2&1&1&1\\1&2&1&0\\0&1&1&2 \end{bmatrix}, \quad
\mb_1 = \begin{bmatrix} 1\\1\\1\\1 \end{bmatrix}, \quad
\mb_2 = \begin{bmatrix} 5\\3\\4\\0 \end{bmatrix}
\]
Ratkaise Gaussin algoritmilla yhtälöryhmät $\mA\mx=\mb_1$ ja $\mA\mx=\mb_2$ ja laske matriisin
$\mA$ $LU$-hajotelma.

\item
Olkoon
\[
\mA=\begin{bmatrix} 1&0&0&1\\1&1&0&2\\1&0&1&0\\1&1&2&1 \end{bmatrix}.
\]
Laske käänteismatriisi $\mA^{-1}$ Gaussin algoritmilla \vspace{1mm}\newline 
a) ratkaisemalla yhtälöryhmä $\mA\mx=\mb$ yleisellä $\mb\in\R^4$ (symbolinen lasku), \newline
b) ratkaisemalla taulukkomuodossa yhtälöryhmät $\mA\mx=\me_i,\ i=1 \ldots 4$

\item
Näytä, että jos lineaarisen yhtälöryhmän kerroimatriisi $\mA$ on symmetrinen, niin Gaussin 
algoritmi säilyttää symmetrisenä sen matriisin osan, johon eliminaatio ei ole vielä edennyt,
ts.\ $(k-1)$:n eliminaatioaskeleen jälkeen pätee
\[
a_{ji}^{(k-1)}=a_{ij}^{(k-1)} \quad \forall\,i,j \ge k.
\]
Päättele, että Gaussin algoritmin vaatima laskutyö on symmetrisen kerroinmatriisin tapauksessa
$W=\frac{1}{6}\,n^3+\ordoO{n^2}$.

\item \label{H-m-3: työmääriä 1}
Olkoon $\mA,\mB$ matriiseja kokoa $n \times n$, $\mL,\mU$ ala- ja yläkolmiomatriiseja samaa
kokoa ja $\mb\in\R^n$. Näytä oikeaksi seuraavat laskuoperaatioiden työmääriä koskevat arviot
(työyksikkö = kertolasku + yhteenlasku).
\begin{align*}
\text{a)} \qquad\qquad\qquad &\mA,\mb\,\map\ \mA\mb \qquad\qquad\qquad\, n^2+\ordoO{n} \\
\text{b)} \qquad\qquad\qquad &\mL,\mb\ \map\ 
                              \mL^{-1}\mb \qquad\qquad\quad\ \tfrac{1}{2}\,n^2+\ordoO{n} \\
\text{c)} \qquad\qquad\qquad &\mA,\mB \map\ \mA\mB \qquad\qquad\qquad   n^3+\ordoO{n^2} \\
\text{d)} \qquad\qquad\qquad &\mL,\mU\,\map\ 
                              \mL\mU \qquad\qquad\qquad  \tfrac{1}{3}\,n^3+\ordoO{n^2}
\end{align*}

\item (*) \label{H-m-3: työmääriä 2}
Näytä, että laskuoperaation $\mA\map\mA^{-1}$ työmäärä Gaussin algoritmin eri vaihtoehdoissa on
($\mA$ kokoa $n \times n$, työyksikkö = kertolasku + yhteenlasku)
\begin{align*}
\text{a)} \qquad &[\,\mA \mid \mI\ ]\ \map\ [\,\mI \mid \mA^{-1}\ ] \qquad n^3+\ordoO{n^2} \\
\text{b)} \qquad &\mA\map\mA^{-1}\ = \mU^{-1}\mL^{-1} \qquad\, n^3+\ordoO{n^2}
\end{align*}

\end{enumerate}

 %Gaussin algoritmi
\section[Tuettu Gaussin algoritmi. Singulaariset yhtälöryhmät]
{Tuettu Gaussin algoritmi. Singulaariset \\ yhtälöryhmät} 
\label{tuettu Gauss}
\sectionmark{Tuettu Gaussin algoritmi}
\alku
\index{Gaussin algoritmi (eliminaatio)!a@tuettu|vahv}
\index{tuettu Gaussin algoritmi|vahv}
\index{singulaarinen yhtälöryhmä|vahv}

Jos Gaussin algoritmissa tullaan $k-1$ eliminaatioaskeleen jälkeen tilanteeseen, jossa
tukialkio $a_{kk}^{(k-1)} = 0$, niin algoritmissa päästään eteenpäin käyttämällä nk.\
\kor{tuentaa} (engl. pivoting). Tuennassa yksinkertaisesti vaihdetaan matriisin rivien
ja/tai sarakkeiden järjestystä (vastaten yhtälöiden tai tuntemattomien järjestyksen vaihtoa
yhtälöryhmässä).
Pyrkimyksenä on löytää jokin sellainen 
järjestys, jossa tukialkiosta tulee nollasta poikkeava, jolloin Gaussin algoritmia voidaan 
jatkaa. Tarkastellaan aluksi nk.\ \kor{osittaista} eli
\index{rivituenta (Gaussin alg.)}%
\kor{rivituentaa}, jossa menettely on
seuraava: Käydään läpi $k$:nnessa sarakkeessa lävistäjän alapuolella olevat alkiot
\[
a_{ik}^{(k-1)}, \quad i=k+1 \ldots n.
\]
Valitaan (jos mahdollista) indeksi $i=l$ siten, että $a_{lk}^{(k-1)} \neq 0$, ja suoritetaan
yhtälöryhmässä (eli matriisissa $\mA^{(k-1)}$ ja vektorissa $\mb^{(k-1)}$) rivinvaihto
\[
k\,\text{:s rivi} \ \leftrightarrows\ l\,\text{:s rivi}.
\]
Tämä vastaa yhtälöryhmän yhtälöiden järjestyksen vaihtoa. Gaussin algoritmista nähdään, että 
sikäli kuin rivien uudelleen järjestely suoritetaan riveillä $i = k \ldots n$, ei järjestyksen
vaihto häiritse itse algoritmia. Toisin sanoen, lopputulos on sama, jos rivit vaihdetaan jo 
alkuperäisessä yhtälöryhmässä ja sen jälkeen suoritetaan $k-1$ eliminaatioaskelta 
(Harj.teht.\,\ref{H-m-4: tuentakysymys}a). Tukioperaation jälkeen uusi tukialkio on nollasta
poikkeava, jolloin Gaussin algoritmia voidaan jatkaa.\footnote[2]{\kor{Automaattisessa}
rivituennassa suoritetaan rivien vaihto ennen jokaista eliminaatioaskelta niin, että
tukialkioksi tulee alkioista $a_{ik}^{(k-1)},\ i=k \ldots n$ itseisarvoltaan suurin. Tämä
vähentää pyöristysvirheiden kasautumista ratkaisulagoritmin kuluessa (liukulukulaskennassa).
Monissa tietokoneohjelmistojen 'black box'--ratkaisijoissa käytetään automaattista rivituentaa
(esim.\ Mathematica: {\tt LinearSolve}).}

Em. tukioperaatio epäonnistuu vain siinä tapauksessa, että on
\[
a_{ik}^{(k-1)} = 0, \quad i = k \ldots n.
\]
Gaussin algoritmi on tällöin osoittanut alkuperäisen matriisin $\mA$ singulaariseksi. Nimittäin
koska tällöin tuntematon $x_k$ ei esiinny lainkaan muunnetun yhtälöryhmän riveillä 
$i=k \ldots n$, niin yhtälöryhmälle $\mA \mx = \mo$ saadaan monikäsitteinen ratkaisu antamalla
$x_k$:lle mielivaltainen arvo, asettamalla $x_{k+1}=\ldots=x_n=0$, ja ratkaisemalla 
$x_i, \ i=1 \ldots k-1\,$ muunnetusta yhtälöryhmästä takaisinsijoituksella ($x_k$:n avulla). 
Siis $\mA$ on singulaarinen (Korollaari \ref{singulaarisuuskriteeri}).

Osittaisen tukioperaation epäonnistuttua (ja siis matriisin osoittauduttua singulaariseksi) 
voidaan Gaussin algoritmia vielä jatkaa käyttämällä nk.\ 
\index{tzy@täydellinen tuenta (Gaussin alg.)}%
\kor{täydellistä tuentaa}, jossa käydään läpi kaikki matriisialkiot rivistä $k$ ja sarakkeesta
$k$ alkaen, eli alkiot
\[
a_{ij}^{(k-1)}, \quad i,j = k \ldots n.
\]
Oletetaan, että jollakin $(l,m)$ on $a_{lm}^{(k-1)} \neq 0$. (Mahdollisesti valitaan alkioista
itseisarvoltaan suurin kuten automaattisessa rivituennassa, vrt.\ alaviite edellä.)
\[
\left.
\begin{matrix} 0 & & \ldots & & a_{km}^{(k-1)} & \ldots & a_{kn}^{(k-1)} \\ 
               \vdots & & & & \vdots & & \\ 
               a_{lk}^{(k-1)} & & \ldots & & a_{lm}^{(k-1)} & & 
               \\ \vdots & & & & & & 
               \\ a_{nk}^{(k-1)} & & & & & & \end{matrix}
\right\rfloor
\]
Suoritetaan tällöin rivien ja sarakkeiden vaihto
\[
k\,\text{:s rivi} \ \leftrightarrows\ l\,\text{:s rivi}, \quad k\,\text{:s sarake} \ 
                    \leftrightarrows\ m\,\text{:s sarake}.
\]
Sarakkeiden vaihto vastaa yhtälöryhmässä tuntemattomien uudelleen indeksointia. Indekseihin 
$j \ge k$ sovellettuna tämäkään tukioperaatio ei häiritse itse algoritmia, ts.\ operaatio 
voidaan ajatella suoritetuksi jo yhtälöryhmän alkuperäisessä muodossa 
(Harj.teht.\,\ref{H-m-4: tuentakysymys}a). Sikäli kuin tukioperaatio onnistuu oletetulla
tavalla, on operaation jälkeen uusi tukialkio $a_{lm}^{(k-1)} \neq 0$, jolloin Gaussin
algoritmia voidaan jatkaa. Muussa tapauksessa on tultu tilanteeseen, jossa $a_{ij}^{(k-1)} = 0$
kun $i = k \ldots n\,$ ja $j = 1 \ldots n$. Gaussin algoritmi on tällöin päättynyt 
\index{perusmuoto!c@singulaarisen yhtälöryhmän}
\index{singulaarinen yhtälöryhmä!perusmuotoinen}%
\kor{perusmuotoiseen singulaariseen yhtälöryhmään}, joka on siis jollakin 
$k \in \{0, \ldots, n-1\}$ muotoa
\[
\begin{bmatrix}
u_{11} & \ldots &  & & u_{1n} \\
0 & \ddots \\
\vdots & &  u_{kk} & \ldots & u_{kn} \\
0 & &  0 & \ldots & 0 \\
\vdots &  & \vdots & & \vdots \\
0 & &  0  & \ldots & 0
\end{bmatrix}
\begin{bmatrix}
y_1 \\
\vdots \\
y_k \\
\vdots \\
\vdots \\
y_n
\end{bmatrix} =
\begin{bmatrix}
c_1 \\
\vdots \\
c_k \\
\vdots \\
\vdots \\
c_n \\
\end{bmatrix}.
\]
Tässä $\mU=\{u_{ij}\}$ on yläkolmiomatriisi, jonka $n-k$ viimeistä riviä ovat nollarivejä ja 
$u_{ii} \neq 0$, kun $i=1 \ldots k$ (jos $k=0$, niin $\mU=\mA=\mo$). Kertoimet $u_{ij}$ ja $c_i$
on saatu alkuperäisestä  yhtälöryhmästä $\mA \mx = \mb$ tuetulla Gaussin algoritmilla, eli 
eliminaatioita ja tukioperaatioita (rivien ja sarakkeiden vaihtoja) yhdistelemällä. Vektori 
$\my$ sisältää alkuperäiset tuntemattomat $x_i$ uudessa, sarakkeiden vaihtojen määrämässä 
järjestyksessä. Jos algoritmin kuluessa on käytetty vain rivituentaa, on $\my = \mx$.

Yllämainitusta perusmuotoisesta yhtälöryhmästä nähdään, että sillä on ratkaisu täsmälleen kun
\[
c_{k+1}= \ldots = c_n = 0.
\]
Tällöin ratkaisu on monikäsitteinen: $y_i, \ i= k+1 \ldots n$, voidaan valita mielivaltaisesti,
minkä jälkeen $y_1, \ldots, y_k$ määräytyvät takaisinsijoituksella.
\begin{Exa}
Seuraavassa sovelletaan täysin tuettua Gaussin algoritmia singulaariseen yhtälöryhmään.
\begin{align*}
\begin{bmatrix}
0&0&3 \\
1&2&0 \\
2&4&0
\end{bmatrix}
\begin{bmatrix} x_1 \\ x_2 \\ x_3 \end{bmatrix} &= 
\begin{bmatrix} b_1 \\ b_2 \\ b_3 \end{bmatrix} \\[5mm]
\begin{bmatrix}
4&2&0 \\
2&1&0 \\
0&0&3
\end{bmatrix}
\begin{bmatrix} x_2 \\ x_1 \\ x_3 \end{bmatrix} &= 
\begin{bmatrix} b_3 \\ b_2 \\ b_1 \end{bmatrix} 
\quad (\text{rivien vaihto}\ 1 \leftrightarrows 3,\ \ 
\text{sarakkeiden vaihto}\ 1 \leftrightarrows 2\,) \\[5mm]
\begin{bmatrix}
4&2&0 \\
0&0&0 \\
0&0&3
\end{bmatrix}
\begin{bmatrix} x_2 \\ x_1 \\ x_3 \end{bmatrix} &= 
\begin{bmatrix} b_3 \\ b_2 - b_3/2 \\ b_1 \end{bmatrix} 
\quad \text{(eliminaatio)} \\[5mm]
\begin{bmatrix}
4&0&2 \\
0&3&0 \\
0&0&0
\end{bmatrix}
\begin{bmatrix} x_2 \\ x_3 \\ x_1 \end{bmatrix} &= 
\begin{bmatrix} b_3 \\ b_1 \\ b_2 - b_3/2 \end{bmatrix} 
\quad (\text{rivien ja sarakkeiden vaihto}\ 2 \leftrightarrows 3\,)
\end{align*}
Nähdään, että saadulla perusmuotoisella yhtälöryhmällä on ratkaisu täsmälleen, kun
\[
b_2-b_3/2=0\,\ \ekv\,\ b_3 = 2b_2,
\]
ja yleinen ratkaisu on tällöin
\[
\left\{ \begin{aligned} 
x_1\ &=\ t, \\ x_2\ &=\ -t/2 + b_3/4\ =\ (b_2 - t)/2, \\ x_3\ &=\ b_1/3 
\end{aligned} \right. \qquad\quad (t \in \R). \quad\loppu
\]
\end{Exa}
Esimerkin tuloksen olisi voinut nähdä suoremminkin. Yleisen algoritmin etuna on kuitenkin, että
se toimii aina, myös silloin kun laskija ei 'näe'.

\subsection{Neliöatriisin yleinen tulohajotelma}
\index{matriisin ($\nel$neliömatriisin)!g@$\nel$yleinen tulohajotelma|vahv}
\index{tulohajotelma (neliömatriisin)|vahv}

Tuetunkin Gaussin algoritmin avulla voidaan määrätä matriisin $\mA$ käänteismatriisi, sikäli 
kuin $\mA$ on säännöllinen. Kun tukioperaatiot ajatellaan tehdyksi ennen Gaussin algoritmin 
soveltamista, niin ensinnäkin rivituennat vastaavat yhtälöryhmän muunnosta
\[
\mA\mx=\mb \qekv \mV\mA\mx = \mV\mb,
\]
missä $\mV=\mI_p$ on rivien lopullista järjestystä $p$ vastaava permutaatiomatriisi.
--- Kerrattakoon Luvusta \ref{inverssi}, että jos rivinvaihdot vastaavat vaihtomatriiseja
$\mV_i,\ i=1 \ldots m$, niin
\[
\mV = \mV_m \cdots \mV_2\mV_1
\]
(käänteinen järjestys!) ja $\mV^{-1}=\mV^T=\mV_1\mV_2 \cdots \mV_m$. Sarakkeiden vaihtojen
lopputulos taas vastaa muunnosta $\my=\mW^T\mx\ \ekv\ \mx=\mW\my$, missä permutaatiomatriisi
$\mW=\mI_q$ vastaa sarakkeiden lopullista järjestystä $q$. Jos sarakkeiden vaihtoja vastaavat
vaihtomatriisit ovat $\mW_i,\ i=1 \ldots l$, niin
\[
\mW=\mW_1\mW_2 \ldots \mW_l
\]
(sama järjestys kuin vaihdoissa!) ja $\mW^{-1}=\mW^T=\mW_l\mW_{l-1} \ldots \mW_1$. Kun kaikki
tuennat huomioidaan, niin yhtälöryhmä muuntuu muotoon
\[
\mA\mx=\mb \qekv \mV\mA\mx=\mV\mb \qekv (\mV\mA\mW)(\mW^T\mx) = \mV\mb \qekv \mB\my=\mc,
\]
missä $\mc=\mV\mb$ (yhtälöt vaihdetussa järjestyksessä $p$), $\my=\mW^T\mx$ (tuntemattomat
vaihdetussa järjestyksessä $q$) ja
\[
\mB=\mV\mA\mW\ \ekv\ \mA = \mV^T\mB\mW^T.
\]
Kun muunnettuun yhtälöryhmään $\mB\my=\mc$ sovelletaan Gaussin algoritmia, ei tuentaa enää 
tarvita. Eliminaation lopputuloksena $\mB$ muuntuu yläkolmiomatriisiksi $\mU$, jolloin on kaksi
vaihtoehtoa:
\begin{itemize}
\item[a)] $[\mU]_{kk} \neq 0,\ k=1\,\ldots\,n$. Tällöin eliminaatiovaihe menee läpi 
          kokonaisuudessaan, $\mU$ on säännöllinen, $\mB$:lle saadaan $LU$-hajotelma
          \[
          \mB = \mL\mU,
          \]
          ja $\mA$:lle vastaavasti tulohajotelma
          \begin{equation} \label{G-tulohajotelma}
          \boxed{\quad\gehys \mA = \mV^T\mL\mU\mW^T.\quad} \tag{$\star$}
          \end{equation}
          Tällöin $\mA$ on säännöllinen ja
          \[
          \inv{\mA} = \mW\inv{\mU}\inv{\mL}\mV = \mW\inv{\mB}\mV.
          \]
\item[b)] $[\mU]_{ii} \neq 0,\ i=1\,\ldots\,k<n$, $[\mU]_{i,i,}=0,\ i=k \ldots n$
          (ol.\ $\mA\neq\mo$). Tällöin $\mU$ on em.\ singulaarista perusmuotoa, eli
          $[\mU]_{ij}=0$, kun $i>k, \ j=1 \ldots n$, ja eliminaatioalgoritmi pysähtyy
          riville $k$. Pysähtyminen vastaa sitä, että matriisissa $\mL$ astetetaan
          \[
          [\mL]_{ij} = \delta_{ij}, \quad 
                      \text{kun} \quad i \in \{k+1, \ldots, n\}, \ j \in \{1, \ldots, n\}.
          \]
          Kun $\mL$ määrätään muilta osin samalla tavoin kuin säännöllisen matriisin 
          tapauksessa, nähdään, että tulohajotelma \eqref{G-tulohajotelma} on edelleen pätevä.
\end{itemize}
Kysymys neliömatriisin säännöllisyydestä tai singulaarisuudesta on näin muodoin ratkaistu 
Gaussin algoritmilla seuraavasti: Jokainen neliömatriisi voidaan esittää tulomuodossa 
\eqref{G-tulohajotelma}, missä $\mU$ on yläkolmiomatriisi ja $\mL,\mV,\mW$ ovat säännöllisiä
 matriiseja. Matriisi $\mU$ on joko säännöllinen tai singulaarinen, mikä ratkaisee myös $\mA$:n
laadun.
\jatko
\begin{Exa} (jatko) Tässä on
\begin{align*}
\mA &= \begin{bmatrix} 0&0&3 \\1&2&0 \\2&4&0 \end{bmatrix}, \quad
\mV  = \begin{bmatrix} 1&0&0\\0&0&1\\0&1&0 \end{bmatrix} 
       \begin{bmatrix} 0&0&1\\0&1&0\\1&0&0 \end{bmatrix}
     = \begin{bmatrix} 0&0&1\\1&0&0\\0&1&0 \end{bmatrix}, \\[5mm]
\mW &= \begin{bmatrix} 0&1&0\\1&0&0\\0&0&1 \end{bmatrix} 
       \begin{bmatrix} 1&0&0\\0&0&1\\0&1&0 \end{bmatrix}
     = \begin{bmatrix} 0&0&1\\1&0&0\\0&1&0 \end{bmatrix}, \quad \mB = \mV\mA\mW 
     = \begin{bmatrix} 4&0&2\\0&3&0\\2&0&1 \end{bmatrix},
\end{align*}
\[
\my=\mW^T\mx = \begin{bmatrix} 0&1&0\\0&0&1\\1&0&0 \end{bmatrix} 
               \begin{bmatrix} x_1 \\ x_2 \\ x_3 \end{bmatrix}
             = \begin{bmatrix} x_2 \\ x_3 \\ x_1 \end{bmatrix}, \quad
\mc = \mV\mb = \begin{bmatrix} b_3 \\ b_1 \\ b_2 \end{bmatrix}.
\]
Muunnetusta yhtälöryhmästä
\[
\begin{bmatrix} 4&0&2\\0&3&0\\2&0&1 \end{bmatrix} 
\begin{bmatrix} x_2 \\ x_3 \\ x_1 \end{bmatrix} =
\begin{bmatrix} b_3 \\ b_1 \\ b_2 \end{bmatrix}
\]
päästään yläkolmiomuotoon yhdellä eliminaatioaskeleella. Koska tukioperaatiot on suoritettu jo
etukäteen, ei niitä enää eliminaation yhteydessä tarvita. \loppu
\end{Exa}

\subsection{Säännöllisyysaste}

Jos $\mA$ on singulaarinen, niin em. tuetun Gaussin algoritmin lopetusindeksiä $k$ sanotaan 
\index{matriisin ($\nel$neliömatriisin)!ca@säännöllisyysaste (rangi)}
\index{szyzy@säännöllisyysaste (rangi)} \index{rangi (säännöllisyysaste)}%
$\mA$:n \kor{säännöllisyysasteeksi} eli \kor{rangiksi} (engl. rank), merkitään $k=r(\mA)$. 
Luku $n-r(\mA)$ siis kertoo, kuinka monta nollariviä on $\mA$:n tulohajotelman perusmuotoisessa
singulaarisessa yläkolmiomatriisissa $\mU$. Jos $\mA$ on säännöllinen, niin $r(\mA)=n$, jolloin
myös $r(\mA^T)=n$. Yleisemminkin pätee
\[
\boxed{\quad r(\mA^T)=r(\mA). \quad}
\]
Myös tämän voi perustella Gaussin algoritmin antaman tulohajotelman avulla (sivuutetaan
perustelut).

\subsection{Yhtälöryhmä kokoa $m \times n$}

Gaussin algoritmi soveltuu myös sellaiseen lineaariseen yhtälöryhmään $\mA \mx = \mb$, jossa 
yhtälöiden ja tuntemattomien lukumäärät eivät ole samat, ts.\ $\mA$ on kokoa 
$m \times n,\ m \neq n\ (\mb\in\R^m)$.  Kun tuettua Gaussin algoritmia sovelletaan tällaiseen 
yhtälöryhmään, on lopputulos jälleen muunnettu perusmuotoinen yhtälöryhmä $\mU \my = \mc$, missä 
$\mU = \{u_{ij},\ i = 1 \ldots m,\ j = 1 \ldots n\}$ on yläkolmiomatriisi (ts.\ $u_{ij} = 0$ kun $i>j$), ja jollakin $k$, $\ 0 \le k \le \min\{m,n\}$, pätee
\[ 
u_{ii} \neq 0,\ \text{kun}\ i = 1 \ldots k, \quad 
u_{ij} = 0,\ \text{kun}\ i = k+1 \ldots m,\ j = 1 \ldots n. 
\]  
\index{matriisin ($\nel$neliömatriisin)!ca@säännöllisyysaste (rangi)}
\index{szyzy@säännöllisyysaste (rangi)} \index{rangi (säännöllisyysaste)}%
Indeksiä $k$ sanotaan jälleen $\mA$:n \kor{säännöllisyysasteeksi} eli \kor{rangiksi}. Jos $k$ on
suurin mahdollinen, eli jos $k = \min\{m,n\}$, niin sanotaan, että $\mA$:lla on \kor{täysi} 
säännöllisyysaste (engl.\ full rank). Tässä tapauksessa (täydellisesti tuettua) Gaussin 
eliminaatiota voidaan jatkaa, kunnes algoritmi törmää matriisin $\mA$ 'laitaan' ($m>n$) tai 
'pohjaan' ($m<n$). Muunnettu yhtälöryhmä $\mU \my = \mc$ voidaan esittää 
\index{lohkomatriisi}%
\kor{lohkomatriisi}muodossa
\[
\begin{rmatrix} \mU_{11} & \mU_{12} \\ \mU_{21} & \mU_{22} \end{rmatrix} 
\begin{rmatrix} \my_1 \\ \my_2 \end{rmatrix}\ =\ \begin{rmatrix} \mc_1 \\ \mc_2 \end{rmatrix},
\]
missä $\mU_{11} = \{u_{ij},\ i,j = 1 \ldots k\}$ on säännöllinen yläkolmiomatriisi, $\mU_{12}$
on kokoa $k \times (n-k)$, $\mU_{21}$ on nollamatriisi kokoa $(m-k) \times k$, $\mU_{22}$ on 
nollamatriisi kokoa $(m-k) \times (n-k)$, ja $\mc_1$ on $k$-vektori. Koska $\mU_{21}$ ja 
$\mU_{22}$ ovat nollamatriiseja, niin yhtälöryhmän avautuu muotoon 
\[
\left\{ \begin{aligned} 
\mU_{11}\,\my_1 + \mU_{12}\,\my_2\ &=\ \mc_1, \\ \mathbf{0}\ &=\ \mc_2. 
\end{aligned} \right.
\]
Tästä nähdään, että sikäli kuin $k<m$ (näin on aina kun $m>n$), on yhtälöryhmällä sekä 
välttämätön että riittävä
\[ 
\text{\kor{ratkeavuusehto}:} \quad \mathbf{0}\ =\ \mc_2. 
\]
Jos $k<n$ (näin on aina kun $m<n$), ei ratkaisu ole tälläkään ehdolla yksikäsitteinen, sillä 
ratkaisussa voidaan asettaa $\my_1 = \mU_{11}^{-1}(\mc_1 - \mU_{12}\,\my_2)$, olipa 
$\my_2 \in \R^{n-k}$ mikä tahansa. Päätellään siis erityisesti, että yhtälöryhmä $\mA \mx = \mb$
kokoa $m \times n$, $m \neq n$, on aina singulaarinen: Joko ratkaisua ei ole jokaisella 
$\mb \in \R^m$, tai ratkaisu ei ole yksikäsitteinen. Edellinen tilanne vallitsee aina kun $m>n$,
jälkimmäinen aina kun $m<n$. 
\begin{Exa} Määritä singulaarisen yhtälöryhmän
\[
\left\{ \begin{array}{rrrrl} 
x_1 & + x_2 & +  x_3 & = & b_1 \\ 2x_1 & + x_2 &        & = & b_2 \\ -x_1 & & + x_3 & = & b_3 \\
    & + x_2 & + 2x_3 & = & b_4 \\   x_1 & - x_2 & - 3x_3 & = & b_5 
\end{array} \right.
\]
ratkeavuusehdot, yleinen ratkaisu ja kerroinmatriisin säännöllisyysaste. \end{Exa}
\ratk Tässä on $m=5$ ja $n=3$. Gaussin algoritmissa selvitään ilman tuentaa:
\begin{align*}
\begin{rmatrix} 1&1&1\\2&1&0\\-1&0&1\\0&1&2\\1&-1&-3 \end{rmatrix} 
      \begin{rmatrix} b_1\\b_2\\b_3\\b_4\\b_5 \end{rmatrix} \quad &\longmapsto \quad
\begin{rmatrix} 1&1&1\\0&-1&-2\\0&1&2\\0&1&2\\0&-2&-4 \end{rmatrix}
      \begin{rmatrix} b_1\\-2b_1+b_2\\b_1+b_3\\b_4\\-b_1+b_5 \end{rmatrix} \\ 
&\longmapsto \quad 
\begin{rmatrix} 1&1&1\\0&-1&-2\\0&0&0\\0&0&0\\0&0&0 \end{rmatrix} 
     \begin{rmatrix} b_1\\-2b_1+b_2\\-b_1+b_2+b_3\\-2b_1+b_2+b_4\\3b_1-2b_2+b_5 \end{rmatrix}
\end{align*}
Saadun singulaarisen perusmuodon mukaan kerroinmatriisin säännöllisyysaste on $k=2$.
Ratkeavuusehdot voidaan lukea yhtälöryhmän kolmelta viimeiseltä riviltä:
\[ \left\{
\begin{array}{rrrrrrl} 
-b_1 & +b_2 & +b_3 & & & = & 0 \\ -2b_1 & +b_2 & & +b_4 & & = & 0 \\ 
3b_1 & -2b_2 & & & +b_5 & = & 0 
\end{array} \right. \]
eli
\[
\begin{rmatrix} 1&0&0&-2&3\\0&1&0&1&-2\\0&0&1&1&-1 \end{rmatrix} 
\begin{rmatrix} b_5\\b_4\\b_3\\b_2\\b_1 \end{rmatrix} = \begin{rmatrix} 0\\0\\0 \end{rmatrix}.
\]
Koska tässä vasemmalla oleva matriisi on perusmuotoinen yläkolmio (ellei olisi, sovellettaisiin
ensin Gaussin algoritmia), niin nähdään, että $b_1$ ja $b_2$ voidaan valita vapaasti, minkä 
jälkeen $b_3$, $b_4$ ja $b_5$ määräytyvät:
\[
\begin{cases}
\,b_3 = b_1-b_2, \\ \,b_4 = 2b_1-b_2, \\ \,b_5 = -3b_1+2b_2.
\end{cases}
\]
Näiden ehtojen voimassa ollessa saadaan yleinen ratkaisu takaisinsijoituksella:
\[
\begin{cases}
\,x_1 = -b_1+b_2+t, \\ x_2 = 2b_1-b_2-2t, \\ x_3 = t 
\end{cases} \quad (t \in \R).
\]
Ratkaisu ei ole yksikäsitteinen, koska kerroinmatriisin säännöllisyysaste ei ole täysi:
$r(\mA)=2<\min\{m,n\}=3$. \loppu

\subsection{Lauseen \ref{säännöllisyyskriteerit} todistus}

Tulohajotelman \eqref{G-tulohajotelma} perusteella pätee
\[
\mA\mx=\mb \qekv \mU\my=\mC\mb, \qquad \my=\mW\mx,\ \ \mC=\mL^{-1}\mV.
\]
Jos $\mA$ on singulaarinen, niin $\mU$:n riveistä ainakin $n$:s on nollarivi, jolloin 
yhtälöryhmän $\,\mU\my=\mC\mb\,$ $n$:s yhtälö saa muodon $\,0=[\mC\mb]_n\,$. Jos on
$\mb=\mv{0}$, niin yhtälöryhmällä on monikäsitteinen ratkaisu, missä $y_n$ on vapaasti
valittavissa. Jokaista ratkaisua $\my\neq\mo$ ($y_n \neq 0$) vastaa yhtälöryhmän $\mA\mx=\mo$
ratkaisu $\mx=\mW^T\my\neq\mo$. Jos taas $\mb\in\R^n$ valitaan siten, että $[\mC\mb]_n \neq 0$,
niin mainittu yhtälö ei toteudu, jolloin on $\mU\my\neq\mC\mb\ \forall\my\in\R^n$. Tällöin on
myös $\,\mA\mx\neq\mb$, $\mx=\mW^T\my$, $\my\in\R^n$ (koska $\mU\my=\mC\mb\ \ekv\ \mA\mx=\mb$).
Siis $\,\mA\mx\neq\mb\ \forall\mx\in\R^n$. On päätelty:
\[
\mA\ \text{singulaarinen} 
       \qimpl \begin{cases} 
               \text{\ jollakin $\mx\neq\mv{0}$\,\ \ on} &\mA\mx=\mv{0}, \\
               \text{\ jollakin $\mb\in\R^n$ on}\        &\mA\mx\neq\mb\,\ \forall \mx\in\R^n.
               \end{cases}
\]
Nämä väittämät ovat samanarvoisia kuin Lauseen \ref{säännöllisyyskriteerit} osaväittämät 
E1$\,\impl\,$E0 ja E2$\,\impl\,$E0, jotka siis ovat tosia. Koska E0$\,\impl\,$E1 ja
E0$\,\impl\,$E2 olivat tosia jo Proposition \ref{kerroinmatriisi} mukaan, niin lause on
todistettu. \loppu

\pagebreak

\Harj
\begin{enumerate}

\item \label{H-m-4: tuentakysymys}
a) Näytä, että tuetun Gaussin algoritmin kuluessa suoritetut tukioperaatiot (rivien ja
sarkkeiden vaihdot) voidaan suorittaa ennen eliminaatioita lopputuloksen
muuttumatta. \vspace{1mm}\newline
b) Olkoon $\mA$ matriisi kokoa $n \times n$ ja olkoon $\mA_k$ matriisi kokoa $k \times k$, joka
koostuu $\mA$:n alkioista $a_{ij},\ i,j=1 \ldots k$. Näytä, että yhtälöryhmä $\mA\mx=\mb$
ratkeaa Gaussin algoritmilla ilman tukioperaatioita täsmälleen kun $\mA_k$ on säännöllinen
matriisi jokaisella $k=1 \ldots n$.

\item 
Muunna singulaariseen perusmuotoon tuetulla Gaussin algoritmilla, määritä ratkaisut tai totea
ratkeamattomuus:
\begin{align*}
&\text{a)}\ \ \begin{rmatrix} 1&2&3\\3&2&1\\1&1&1 \end{rmatrix}
              \begin{rmatrix} x\\y\\z \end{rmatrix} =
              \begin{rmatrix} 8\\4\\3 \end{rmatrix} \qquad
 \text{b)}\ \ \begin{rmatrix} 0&1&1\\1&2&1\\2&4&2 \end{rmatrix}
              \begin{rmatrix} x_1\\x_2\\x_3 \end{rmatrix} =
              \begin{rmatrix} 5\\6\\12 \end{rmatrix} \\[5mm] 
&\text{c)}\ \ \begin{rmatrix} 0&1&-1&1\\1&1&0&0\\1&0&1&-1\\1&2&-1&1 \end{rmatrix}
              \begin{rmatrix} x_1\\x_2\\x_3\\x_4 \end{rmatrix} =
              \begin{rmatrix} -1\\1\\1\\0 \end{rmatrix} \\[5mm]
&\text{d)}\ \ \begin{rmatrix}
              1&0&1&1&0&0\\0&1&1&0&1&2\\1&1&2&1&1&2\\1&0&1&0&1&3\\0&0&0&0&1&3\\1&1&2&0&2&5
              \end{rmatrix}
              \begin{rmatrix} x_1\\x_2\\x_3\\x_4\\x_5\\x_6 \end{rmatrix} =
              \begin{rmatrix} 1\\1\\2\\0\\0\\1 \end{rmatrix}
 \end{align*}

\item
Määritä $\mA$:n säännöllisyysaste, kaikki yhtälöryhmän $\mA\mx=\mo$ ratkaisut sekä
tulohajotelma $\mA=\mV^T\mL\mU\mW^T$, kun $\mA=$
\begin{align*}
&\text{a)}\ \ \begin{rmatrix} 1&2&-1\\-2&-4&2\\-1&-2&1 \end{rmatrix} \qquad
 \text{b)}\ \ \begin{rmatrix} 1&1&1\\2&2&1\\3&3&2 \end{rmatrix} \qquad
 \text{c)}\ \ \begin{rmatrix} 2&2&1\\3&3&2\\4&4&3 \end{rmatrix} \\[1mm]
&\text{d)}\ \ \begin{rmatrix} 2&1&1&1\\4&2&2&3\\0&0&0&1\\1&2&2&0 \end{rmatrix} \qquad
 \text{e)}\ \ \begin{rmatrix} 1&1&1&1\\2&1&2&1\\0&1&0&1\\1&0&1&1 \end{rmatrix} \qquad
 \text{f)}\ \ \begin{rmatrix} 1&-2&1&0\\-1&2&-1&1\\2&-4&2&4\\1&1&-1&-1 \end{rmatrix}
\end{align*}

\item
Saata seuraavat yhtälöryhmät tuetulla Gaussin algoritmilla singulaariseen perusmuotoon.
Määritä ratkeavuusehdot ja yleinen ratkaisu sekä edelleen kerroinmatriisin tulohajotelma
$\mA = \mV^T\mL\mU\mW^T$.
\begin{align*}
&\text{a)}\ \ \begin{rmatrix} 1&-2&2\\-1&2&1\\5&-10&4 \end{rmatrix}
              \begin{bmatrix} x_1\\x_2\\x_3 \end{bmatrix}
            = \begin{bmatrix} b_1\\b_2\\b_3 \end{bmatrix} \quad\
 \text{b)}\ \ \begin{rmatrix} 0&1&-4&-3\\1&0&2&1\\3&2&-2&-3\\2&1&0&-1 \end{rmatrix}
              \begin{bmatrix} x_1\\x_2\\x_3\\x_4 \end{bmatrix}
            = \begin{bmatrix} b_1\\b_2\\b_3\\b_4 \end{bmatrix} \\[2mm]
&\text{c)}\ \ \begin{rmatrix}
              1&2&-1&2&2\\2&4&-2&0&4\\-1&-2&1&-2&-2\\-1&-2&0&-3&-3\\-2&-4&4&-2&-2
              \end{rmatrix} 
              \begin{bmatrix} x_1\\x_2\\x_3\\x_4\\x_5 \end{bmatrix}
            = \begin{bmatrix} b_1\\b_2\\b_3\\b_4\\b_5 \end{bmatrix}
\end{align*}

\item
Muunna singulaariseen perusmuotoon ja ratkaise, mikäli mahdollista:
\begin{align*}
&\text{a)}\ \ \begin{rmatrix} 1&0&4\\-1&3&4 \end{rmatrix}
              \begin{rmatrix} x_1\\x_2\\x_3 \end{rmatrix} =
              \begin{rmatrix} 1\\3 \end{rmatrix} \qquad
 \text{b)}\ \ \begin{rmatrix} 2&1\\4&6\\3&5 \end{rmatrix} \begin{rmatrix} x\\y \end{rmatrix} =
              \begin{rmatrix} 1\\4\\-2 \end{rmatrix} \\[3mm]
&\text{c)}\ \ \begin{bmatrix} 1&2&2\\2&4&6\\3&6&8\\1&1&1 \end{bmatrix}
              \begin{bmatrix} x_1\\x_2\\x_3 \end{bmatrix} =
              \begin{bmatrix} 2\\2\\4\\1 \end{bmatrix} \qquad
 \text{d)}\ \ \begin{bmatrix} 2&1&4\\6&3&8\\4&2&7\\2&1&3 \end{bmatrix}
              \begin{bmatrix} x\\y\\z \end{bmatrix} =
              \begin{bmatrix} 0\\4\\1\\1 \end{bmatrix} \\[3mm] 
&\text{e)}\ \ \begin{rmatrix} 1&4&7&-3\\-2&3&-6&1\\0&11&8&-5 \end{rmatrix}
              \begin{rmatrix} x_1\\x_2\\x_3\\x_4 \end{rmatrix} =
              \begin{rmatrix} 1\\3\\5 \end{rmatrix}
\end{align*}

\item 
Olkoon \ a) \ $\D \mA = \begin{rmatrix} 1&2&3&4\\2&3&4&1\\3&4&1&2 \end{rmatrix}$,  \ \ 
b) \ $\D \mA = \begin{rmatrix} 1&3&5&-2\\3&-2&7&5\\2&-5&2&7 \end{rmatrix}$. \vspace{1mm}\newline
Määritä Gaussin algoritmilla ratkeavuusehdot ja yleinen ratkaisu yhtälöryhmille $\mA\mx=\mb$
($\mb\in\R^3$) ja $\mA^T\mx=\mb$ ($\mb\in\R^4$).

\end{enumerate} %Tuettu Gaussin algoritmi. Singulaariset yhtälöryhmät
\section{Determinantti} \label{determinantti} 
\alku
\index{determinantti|vahv}

Olkoon $\ma_1,\ldots,\ma_n\in\R^n$ $n$ kappaletta pystyvektoreita kokoa $n$. Asetetaan
\begin{Def} \label{determinantin määritelmä} \index{determinanttifunktio|emph}
Reaaliarvoinen funktio $V(\ma_1,\ldots,\ma_n), \ \ma_i\in\R^n$, on (normeerattu)
\kor{determinanttifunktio}, jos pätee:
\begin{itemize}
\item[(i)]   $V$ on lineaarinen jokaisen vektorin $\ma_i$ suhteen.
\item[(ii)]  Jos järjestetty indeksijoukko $(i_1,\ldots,i_n)$ on saatu joukosta $(1,2,\ldots,n)$
             vaihtamalla kahden alkion paikkaa, niin
             \[
             V(\ma_{i_1},\ldots,\ma_{i_n})=-V(\ma_1,\ldots,\ma_n).
             \]
\item[(iii)] Jos $[\me_i]_j=\delta_{ij}$, niin $V(\me_1,\ldots,\me_n)=1$.
\end{itemize}
\end{Def}
\pain{Selityksiä}:
\begin{itemize}
\item[(i)]   Lineaarisuus tarkoittaa, että jokaisella 
             $k\in\{1,\ldots,n\}$ ja $\forall\lambda,\mu\in\R$ pätee
             \begin{align*}
             V(\ma_1,\ldots,\lambda\ma_k &+ \mu\mb_k,\ldots,\ma_n)\\
             &=\lambda V(\ma_1,\ldots,\ma_k,\ldots,\ma_n)
              +\mu V(\ma_1,\ldots,\mb_k,\ldots,\ma_n).
             \end{align*}
\item[(ii)]  \index{vaihtoszyzy@vaihtosääntö!b@determinantin}%
             Tämä \kor{vaihtosääntö} tarkoittaa, että determinantin arvo vaihtuu vastaluvukseen,
             kun järjestetyssä joukossa $(\ma_1,\ldots,\ma_n)$ kahden vektorin paikka 
             vaihdetaan (Luvusta \ref{inverssi} tuttu parivaihto). Jos vaihdettavat vektorit
             ovat samat, ei determinantin arvo luonnollisesti muutu, joten vaihtosäännöstä
             \index{nollasääntö!b@determinantin}%
             seuraa \kor{nollasääntö}
             \[
             \ma_i=\ma_j\,,\,\ i \neq j \qimpl V(\ma_1,\ldots,\ma_n)=0.
             \]
             Jos yleisemmin $p=(i_1,\ldots,i_n)$ on indeksijoukon $(1,\ldots,n)$ permutaatio,
             niin $p$ on joko parillinen tai pariton, eli $p$ saadaan joko parilllisella tai
             parittomalla määrällä parivaihtoja (vrt.\ Luku \ref{inverssi}). Vaihtosääntö
             yleistyy siis säännöksi: Parillisessa permutaatiossa determinantin arvo säilyy,
             parittomassa vaihtuu vastaluvukseen.
\item[(iii)] Tämä on normeerausehto. Ilman tätä ehtoa determinanttifunktio voitaisiin kertoa
             mielivaltaisella vakiolla, jolloin seuraava väittämä ei olisi tosi.
\end{itemize}
\begin{Prop}
Determinanttifunktio on funktio, ts.\ $V(\ma_1,\ldots\ma_n)\in\R$ on yksikäsitteisesti määrätty,
kun $\ma_1,\ldots,\ma_n\in\R^n$.
\end{Prop}
\tod Lasketaan determinanttifunktion arvo annetuilla säännöillä (i)--(iii). Merkitään 
$[\ma_j]_i=a_{ij}$, jolloin
\[
\ma_j=\sum_{i=1}^n a_{ij}\me_i
\]
ja siis
\[
V(\ma_1,\ldots,\ma_n)=V\Bigl(\,\sum_{i=1}^n a_{i1}\me_i,\ldots,\sum_{i=1}^n a_{in}\me_i\Bigr).
\]
Käyttämällä toistuvasti lineaarisuusominaisuutta (i) nähdään, että tämä purkautuu summaksi
\[
V(\ma_1,\ldots,\ma_n)
=\sum_{i_1=1}^n\cdots\sum_{i_n=1}^n a_{i_11} \cdots a_{i_nn} V(\me_{i_1},\ldots,\me_{i_n}).
\]
Tässä on $n^n$ termiä, mutta nollasäännön perusteella summasta voidaan jättää pois kaikki ne 
termit, joissa sama indeksi $i_k$ esiintyy kahdesti. Näin muodoin
\[
V(\ma_1,\ldots,\ma_n)=\sum_p a_{i_11} \cdots a_{i_nn}V(\me_{i_1},\ldots,\me_{i_n}),
\]
missä summaus käy läpi kaikki joukon $(1,\ldots,n)$ eri permutaatiot $p=(i_1,\ldots,i_n)$
($n!$ kpl). Säännöistä (ii) ja (iii) seuraa edelleen, että 
$V(\me_{i_1},\ldots,\me_{i_n})=\sigma_p=\pm 1$ riippuen siitä, onko permutaatio $p$ parillinen
vai pariton. Siis
\begin{equation} \label{determinantin laskukaava}
\boxed{\kehys\Ykehys \quad V(\ma_1,\ldots,\ma_n)
                           =\sum_p\sigma_p\,a_{i_11}a_{i_22} \cdots a_{i_nn} \quad}
\end{equation}
ja näin muodoin $V(\ma_1,\ldots,\ma_n)$ on yksikäsitteisesti määrätty. \loppu

Determinanttifunktion ominaisuudet (i)--(iii) voi päätellä käänteisesti laskukaavasta
\eqref{determinantin laskukaava}, joten tämä kaava on käypä myös determinanttifunktion 
määritelmänä. Kaavalle saadaan hieman selkeämpi muoto, kun merkitään (vrt.\ Luku \ref{inverssi})
\[
\mI_p = [\me_{i_1} \ldots \me_{i_n}], \quad \Lambda_p = \{(i,j) \mid [\mI_p]_{ij}=1\}, \quad
                                                        p=(i_1, \ldots,i_n).
\]
Tällöin nähdään, että kaava \eqref{determinantin laskukaava} on sama kuin
\begin{equation} \label{determinantin laskukaava 2}
V(\ma_1,\ldots,\ma_n) = \sum_p \sigma_p \prod_{(i,j)\in\Lambda_p} a_{ij}.
\end{equation}
Determinantin arvoa käytännössä laskettaessa kaavat 
\eqref{determinantin laskukaava}--\eqref{determinantin laskukaava 2} ovat petollisia, sillä
pelkkiä kertolaskuja tarvitaan niitä käyttäen peräti $(n-1)n!$ kpl. --- Kaavoja käytetäänkin
algoritmeina yleensä vain tapauksissa $n=2$ ja $n=3$. Teoreettista käyttöä kaavoilla 
\eqref{determinantin laskukaava}--\eqref{determinantin laskukaava 2} on sen sijaan 
yleisemminkin, kuten nähdään jatkossa.

\subsection{Matriisin determinantti}
\index{matriisin ($\nel$neliömatriisin)!ha@$\nel$determinantti|vahv}

\kor{Matriisin determinantti} määritellään yksinkertaisesti ajattelemalla, että 
determinanttifunktiossa $V(\ma_1,\ldots,\ma_n)$ vektorit $\ma_j$ ovat (neliö)matriisin $\mA$ 
sarakkeet. Determinanttia merkitään $\det\mA$ (toisinaan $\abs{\mA}$), ja määritelmä on siis
\[
\det\mA=V(\ma_1,\ldots,\ma_n), \quad \mA=[\ma_1 \ldots \ma_n].
\]
Tämän mukaisesti on kaavassa \eqref{determinantin laskukaava 2} $\,\sigma_p = \det\mI_p$.

Determinantti on siis määritelty vain neliömatriiseille. Jos $\mA$ on annettu taulukkona, niin
determinanttia merkitään taulukkoa rajoittavin pystyviivoin:
\[
\det\mA=\begin{vmatrix}
a_{11} & \ldots & a_{1n} \\
\vdots &  & \vdots \\
a_{n1} & \ldots & a_{nn}
\end{vmatrix}, \quad \mA=(a_{ij}).
\]
Normeeraussäännön (iii) mukaan yksikkömatriisin determinantti on
\[
\det\mI=1.
\]
Kaavasta \eqref{determinantin laskukaava} seuraa myös helposti sääntö
\[
\det(\lambda\mA)=\lambda^n\det\mA, \quad \lambda \in \R.
\]
Myös seuraavat kaksi determinantin ominaisuutta ovat määritelmästä johdettavissa, mutta nämä
ovat vähemmän ilmeisiä. --- Kyseessä ovat determinanttiopin keskeisimmät väittämät.
\begin{Lause} Matriisin determinantille pätee
\[
\boxed{\kehys \quad \det\mA\mB = \det\mA\,\det\mB, \quad\ \det\mA^T = \det\mA. \quad}
\]
\end{Lause}
\tod Olkoon $\mA=[\ma_1 \ldots \ma_n]$ ja $\mB=[\mb_1 \ldots \mb_n]=(b_{ij})$. Käyttämällä Luvun
\ref{matriisialgebra} purkukaavoja voidaan kirjoittaa
\[
\mA\mB 
= [\mA\mb_1 \ldots \mA\mb_n] 
= \Bigl[\,\sum_{i_1=1}^n b_{i_11}\ma_{i_1}\,\ldots\,\sum_{i_n=1}^n b_{i_nn}\ma_{i_n}\Bigr],
\]
jolloin determinantin lineaarisuussäännön (i) perusteella seuraa
\[
\det\mA\mB 
= \sum_{i_1=1}^n\cdots\sum_{i_n=1}^n b_{i_11}\ldots b_{i_nn}\det\,[\ma_{i_1} \ldots \ma_{i_n}].
\]
Tästä summasta voidaan nollasäännön perusteella jälleen jättää pois kaikki termit, joissa sama
indeksi $i_k$ esiintyy kahdesti. Jäljelle jäävissä termeissä on $p=$ $(i_1,\ldots,i_n)$ joukon 
$(1,\ldots,n)$ permutaatio, jolloin vaihtosäännön (ii) mukaan
\[
\det\,[\ma_{i_1} \ldots \ma_{i_n}] = \sigma_p\det\,[\ma_1 \ldots \ma_n] 
                                  = \sigma_p\det\mA, \quad p=(i_1,\ldots,i_n).
\]
Tässä $\sigma_p$ on määritelty kuten kaavassa \eqref{determinantin laskukaava}, joten
\[
\det\mA\mB = \det\mA\,\sum_p \sigma_p\,b_{i_11}\ldots b_{i_nn} = \det\mA\det\mB.
\]

Determinantin transponointisäännön todistamiseksi toetaan ensinnäkin, että tämä pätee kaikille
permutaatiomatriiseille $\mI_p$. Nimittäin koska $\mI_p^T\mI_p=\mI$, niin jo todistetun tulon
determinanttisäännön perusteella on
\[
\det\mI_p^T\det\mI_p = 1.
\]
Koska tässä on $\det\mI_p=\pm 1$ ja $\det\mI_p^T=\det\mI_q=\pm 1$, niin päätellään, että on
oltava $\det\mI_p^T=\det\mI_p$. Kun huomioidaan tämä ja kirjoitetaan $\mI_p^T=\mI_q$, niin
laskukaavan \eqref{determinantin laskukaava 2} perusteella on
\[
\det\mA^T = \sum_p \det\mI_p \prod_{(i,j)\in\Lambda_p} a_{ji}
          = \sum_q \det\mI_q \prod_{(i,j)\in\Lambda_q} a_{ij} = \det\mA. \loppu
\]

Säännöstä $\det\mA^T=\det\mA$ voidaan päätellä, että determinantin vaihtösääntö ja nollasääntö
pätevät myös muodossa: $\det\mA$ vaihtuu vastaluvukseen, jos $\mA$:n kaksi riviä vaihdetaan 
keskenään, ja $\det\mA=0$, jos $\mA$:n kaksi riviä ovat samat.
  
Jos $\mA$ on säännöllinen matriisi, niin soveltamalla tulon determinanttisääntöä tuloon 
$\inv{\mA}\mA=\mI$ nähdään, että
\[
\boxed{\kehys\quad \det\inv{\mA}=\inv{(\det\mA)}. \quad}
\]
Nähdään myös, että pätee
\[
\mA \text{ säännöllinen } \impl \ \det\mA \neq 0,
\]
mikä on loogisesti sama kuin
\[
\det\mA=0 \ \impl \ \mA \text{ singulaarinen.}
\]
Tämä implikaatio pätee käänteiseenkin suuntaan, ja tässä onkin determinanttiteorian tärkein 
tulos sovellutusten kannalta. Itse tulos on suora seuraus edellisen luvun tulohajotelmista, 
tulon determinanttisäännöstä ja kolmiomatriisin determinanttisäännöstä, joka esitetään 
jäljempänä (Propositio \ref{kolmiomatriisin determinantti}).
\begin{Lause} \label{determinanttilause} \index{determinanttikriteeri|emph}
\vahv{(Determinanttikriteeri)} Neliömatriisille pätee
\[
\mA \text{ singulaarinen } \ekv \ \det\mA=0.
\]
\end{Lause}

\subsection{Determinantin laskeminen}

Laskukaavaa \eqref{determinantin laskukaava} käytetään yleensä vain tapauksissa $n=2,3\,$:
\begin{align*}
&n=2: \qquad \begin{vmatrix}
a_{11} & a_{12} \\
a_{21} & a_{22} \\
\end{vmatrix}\ =\ a_{11}a_{22}-a_{21}a_{12}\,. \\[2mm]
&n=3: \qquad \begin{vmatrix}
a_{11} & a_{12} & a_{13} \\
a_{21} & a_{22} & a_{23} \\
a_{31} & a_{32} & a_{33}
\end{vmatrix}\
=\ \begin{array}{l} \\ a_{11}a_{22}a_{33}+a_{31}a_{12}a_{23}+a_{21}a_{32}a_{13} \\
                       -a_{31}a_{22}a_{13}-a_{11}a_{32}a_{23}-a_{21}a_{12}a_{33}\,.
\end{array}
\end{align*}

Nämä voi helposti johtaa myös suoraan  Määritelmästä \ref{determinantin määritelmä}. Esimerkiksi:
\begin{align*}
\begin{vmatrix} 
a_{11} & a_{12} \\ a_{21} & a_{22} 
\end{vmatrix}\ 
       &=\ V\left(\begin{bmatrix} a_{11}\\a_{21} \end{bmatrix}\,,\,
                  \begin{bmatrix} a_{12}\\a_{22} \end{bmatrix}\right) \\[2mm]
       &=\ V\left(a_{11}\begin{bmatrix} 1\\0 \end{bmatrix}
                 +a_{21}\begin{bmatrix} 0\\1 \end{bmatrix}\,,\
                  a_{12}\begin{bmatrix} 1\\0 \end{bmatrix}
                 +a_{22}\begin{bmatrix} 0\\1 \end{bmatrix}\right) \\[2mm]
       &=\ a_{11}a_{12}\begin{vmatrix} 1&1\\0&0 \end{vmatrix}
          +a_{11}a_{22}\begin{vmatrix} 1&0\\0&1 \end{vmatrix}
          +a_{21}a_{12}\begin{vmatrix} 0&1\\1&0 \end{vmatrix}
          +a_{21}a_{22}\begin{vmatrix} 0&0\\1&1 \end{vmatrix} \\[4mm]
       &=\ a_{11}a_{12}\cdot 0 + a_{11}a_{22}\cdot 1 + a_{21}a_{12}\cdot (-1) 
         + a_{21}a_{22}\cdot 0 
        =\ a_{11}a_{22}-a_{21}a_{12}\,.   
\end{align*} 
Tapauksessa $\,n=3\,$ voi laskukaavan muistaa ryhmittelemällä yhteenlaskettavat termit
kaavan \eqref{determinantin laskukaava 2} mukaisesti seuraavasti
\index{Sarrus'n sääntö}%
(nk. \kor{Sarrus'n sääntö})\,:
\[
\begin{aligned}
\text{etumerkki } + \qquad
\begin{vmatrix}
\bullet & \cdot & \cdot \\ 
\cdot & \bullet & \cdot \\
\cdot & \cdot & \bullet
\end{vmatrix} \quad
\begin{vmatrix}
\cdot & \bullet & \cdot \\ 
\cdot & \cdot & \bullet \\
\bullet & \cdot & \cdot
\end{vmatrix} \quad
\begin{vmatrix}
\cdot & \cdot & \bullet \\ 
\bullet & \cdot & \cdot  \\
\cdot & \bullet & \cdot
\end{vmatrix} \\[5mm]
\text{etumerkki } - \qquad
\begin{vmatrix}
\cdot & \cdot & \bullet \\ 
\cdot & \bullet & \cdot \\
\bullet & \cdot & \cdot
\end{vmatrix} \quad
\begin{vmatrix}
\bullet & \cdot & \cdot \\ 
\cdot & \cdot & \bullet \\
\cdot & \bullet & \cdot
\end{vmatrix} \quad
\begin{vmatrix}
\cdot & \bullet & \cdot \\ 
\bullet & \cdot & \cdot  \\
\cdot & \cdot & \bullet
\end{vmatrix} \\
\end{aligned}
\]
\begin{Exa}
\[
\begin{detmatrix} 1&1&3 \\ 2&0&2 \\ 3&-1&1 \end{detmatrix} \quad = 
         \quad \begin{array}{l} 
               \\ 1\cdot 0\cdot 1+1\cdot 2\cdot 3+2\cdot (-1)\cdot 3 \\
                  -3\cdot 0\cdot 3-1\cdot (-1)\cdot 2-2\cdot  1\cdot 1 \ =\ 0. 
               \end{array}
\]
Matriisina tämä on siis singulaarinen. \loppu
\end{Exa}
Sarrus'n sääntö rajoittuu determinantteihin kokoa $3 \times 3$. Hieman isompia determinantteja
käsivoimin purettaessa on seuraava nk. \kor{alideterminanttisääntö} laskukaavoja 
\eqref{determinantin laskukaava}--\eqref{determinantin laskukaava 2} huomattavasti
helppokäyttöisempi. Sääntö on yleispätevä, ja se on myös käypä determinanttifunktion
määritelmänä (ks.\ Harj.teht.\,\ref{H-m-5: alidet-tod}).
\begin{Lause} \label{alideterminanttisääntö} \index{alideterminanttisääntö|emph}
\vahv{(Alideterminanttisääntö)} Jos $\mA$ on neliömatriisi kokoa $n \times n$, niin jokaisella 
$k \in \{1, \ldots, n\}$ pätee
\begin{align*}
\det\mA &= \sum_{i=1}^n (-1)^{k+i} a_{ik} \det\mA^{(i,k)} \\
        &= \sum_{j=1}^n (-1)^{k+j} a_{kj} \det\mA^{(k,j)},
\end{align*}
missä $\mA^{(i,j)}$ on matriisi kokoa $(n-1) \times (n-1)$, joka saadaan poistamalla $\mA$:sta
$i$:s rivi ja $j$:s sarake.
\end{Lause}
\tod Tarkastellaan väitetyistä purkusäännöistä ensimmäistä, kun $k=1$. Determinantin
lineaarisuussäännön (i) perusteella voidaan kirjoittaa
\[
\det\mA = \sum_{i=1}^n a_{i1} \det\mA_i,
\]
missä $\mA_i$ on saatu matriisista $\mA$ korvaamalla ensimmäinen sarake yksikkövektorilla
$\me_i$. Vaihdetaan matriisissa $\mA_i$ ensin ensimmäinen ja $i$:s rivi keskenään ja 
permutoidaan tämän jälkeen rivit n:o $2 \ldots i$ siten, että $i$:s rivi siirtyy toiseksi
riviksi muiden rivien järjestyksen muuttumatta. Jälkimmäisessä operaatiossa tarvitaan $i-2$
parittaista rivin vaihtoa, joten vaihtoja kertyy kaikkiaan $i-1$ kpl. Jos vaihtojen jälkeen
saatavaa matriisia merkitään $\tilde{\mA}_i$, niin on siis 
$\,\det\mA_i=(-1)^{i-1}\det\tilde{\mA}_i$. Vaihtojen seurauksena on $\tilde{\mA}_i\,$:n 
ensimmäinen sarake $=\me_1$ ja $\tilde{\mA}_i^{(1,1)}=\mA^{(i,1)}$. Determinantin laskukaavasta
\eqref{determinantin laskukaava} nähdään tällöin, että 
$\det\tilde{\mA}_i=\det\tilde{\mA}_i^{(1,1)}=\det\mA^{(i,1)}$. Yhdistämällä päätelmät seuraa
\[ 
\det\mA = \sum_{i=1}^n (-1)^{i-1} a_{i1}\det\mA^{(i,1)} 
        = \sum_{i=1}^n (-1)^{i+1} a_{i1}\det\mA^{(i,1)}.
\]
Väitetyistä purkusäännöistä ensimmäinen on näin todistettu tapauksessa $k=1$. Muut väitetyt
säännöt palautuvat tähän tapaukseen vaihto- ja transponointisääntöjen avulla. \loppu

Determinantteja $\det\mA^{(i,j)}$ sanotaan determinantin $\det\mA$ \kor{alideterminanteiksi} 
--- siitä Lauseen \ref{alideterminanttisääntö} purkusääntöjen nimi.
\begin{Exa}
Kun valitaan $k=1$ ja kehitetään determinantti ensimmäisen sarakkeen mukaan, niin saadaan
\[
\begin{detmatrix}
1&2&-2&1 \\
2&-1&1&1 \\
0&1&0&1 \\
1&0&2&2
\end{detmatrix}=
1 \cdot \begin{detmatrix}
-1&1&1 \\
1&0&1 \\
0&2&2
\end{detmatrix}
-2 \cdot \begin{detmatrix}
2&-2&1 \\
1&0&1 \\
0&2&2
\end{detmatrix}
-1 \cdot \begin{detmatrix}
2&-2&1 \\
-1&1&1 \\
1&0&1
\end{detmatrix}.
\]
Alideterminantit voidaan kehittää Sarrus'n säännöllä tai alideterminanttisäännöllä. Sovelletaan 
jälkimmäistä siten, että kaksi ensimmäistä alideterminanttia kehitetään ensimmäisen sarakkeen 
mukaan $(k=1)$ ja viimeinen kolmannen rivin mukaan $(k=3)$:
\begin{align*}
&\begin{detmatrix} -1&1&1\\1&0&1\\0&2&2\end{detmatrix} 
   \,=\,(-1)\,\cdot \begin{detmatrix} 0&1\\2&2 \end{detmatrix}
          -1  \cdot \begin{detmatrix} 1&1\\2&2 \end{detmatrix}
   \,=\, (-1) \cdot (-2) -1 \cdot 0 \,\,= 2, \\[3mm]
&\begin{detmatrix} 2&-2&1\\1&0&1\\0&2&2 \end{detmatrix} 
   \,=\,   2 \cdot \begin{detmatrix} 0&1\\2&2 \end{detmatrix}
          -1 \cdot \begin{detmatrix} -2&1\\2&2 \end{detmatrix}
   \,=\, 2 \cdot (-2) - 1 \cdot (-6) \,=\, 2, \\[3mm]
&\begin{detmatrix} 2&-2&1\\-1&1&1\\1&0&1 \end{detmatrix} 
   \,=\,   1 \cdot \begin{detmatrix} -2&1\\1&1\end{detmatrix}
          +1 \cdot \begin{detmatrix} 2&-2\\-1&1\end{detmatrix}
   \,=\, 1 \cdot (-3) + 1 \cdot 0 \,=\, -3.
\end{align*}
Determinantin arvoksi tulee $1\cdot 2-2\cdot 2-1\cdot (-3)=1$. \loppu
\end{Exa}
Alideterminanttisäännöllä voidaan yleinen, kokoa $n\times n$ oleva determinantti purkaa lopulta
alideterminanteiksi kokoa $2\times 2$. Tarvittava laskuoperaatioiden määrä on kuitenkin
edelleen suuruusluokkaa $W \sim n!$ (ks.\ Harj.teht.\,\ref{H-m-5: alidet-työ}), joten suurilla
$n$:n arvoilla on tämäkin menetelmä kelvoton --- Miten siis ylipäänsä on mahdollista laskea
esimerkiksi determinantti kokoa $n=100$\,? Vastaukseen johtaa seuraava tulos, joka on helposti
todistettavissa alideterminanttisäännöllä.
\begin{Prop} \label{kolmiomatriisin determinantti}
Kolmiomatriisin $\mA=\{a_{ij}\}$ determinantti on matriisin lävistäjäalkioiden tulo:
\[
\det\mA=\prod_{i=1}^n a_{ii}.
\]
\end{Prop}
\tod Kun alideterminanttisäännössä valitaan $k=1$ ja kehitetään determinantti joko ensimmäisen
rivin (alakolmio) tai ensimmäisen sarakkeen (yläkolmio) mukaan, niin $a_{11}$ on ko.\ 
rivin/sarakkeen ainoa nollasta poikkeava alkio, joten Lauseen \ref{alideterminanttisääntö}
mukaan
\[
\det\mA=a_{11}\det\mA^{(1,1)}.
\]
Tässä $\mA^{(1,1)}$ on jälleen kolmiomatriisi, joten sama sääntö soveltuu yhä uudelleen ja 
johtaa väitettyyn lopputulokseen. \loppu

Propositioon \ref{kolmiomatriisin determinantti} perustuen voidaan determinantin arvo laskea
--- kuinkas muuten --- Gaussin algoritmilla. Nimittäin jos algoritmi menee läpi ilman 
tukioperaatioita (rivien/sarakkeiden vaihtoja), niin se johtaa $LU$-hajotelmaan
\[
\mA=\mL\mU,
\]
missä $[\mL]_{kk}=1$ ja $[\mU]_{kk}=a_{kk}^{(k-1)},\ k=1 \ldots n$ (vrt. edellinen luku). 
Tällöin on tulon determinanttisäännön ja Proposition \ref{kolmiomatriisin determinantti} mukaan
\[
\det\mA = \det\mL\,\det\mU = \prod_{k=1}^n a_{kk}^{(k-1)}.
\]
Determinantin arvo on siis yksinkertaisesti tukialkioiden tulo Gaussin algoritmin 
eliminaatiovaiheessa (!). Jos eliminaatiovaiheessa tehdään tukioperaatioita, tulee determinantin
arvoksi
\[
\det\mA=(-1)^m\prod_{k=1}^n[\mU]_{kk},
\]
missä m on tehtyjen rivien/sarakkeiden vaihtojen kokonaismäärä.

Gaussin algoritmia käyttäen tulee siis determinantin laskemisen työmääräksi 
$W \sim \frac{1}{3}n^3$ (kerto- ja yhteenlaskua) --- eli tämä on suurilla $n$:n arvoilla
huomattavasti tehokkaampi menetelmä kuin mikään edellä esitetyistä vaihtoehdoista. Puhtaasti
numeerisissa laskuissa, ja suurilla $n$, determinantilla ei olekaan juuri käytännön merkitystä
--- selvittäähän Gaussin algoritmi matriisin säännöllisyyskysymyksen muutenkin. Determinantin
käsite on kuitenkin käyttökelpoinen erinäisissä lineaarialgebran teoreettisissa tarkasteluissa,
ja myös symbolisessa laskennassa determinantilla on käyttöä. Symbolisen laskennan ongelma voi
olla esimerkiksi sellainen, että tarkasteltava matriisi riippuu jostakin parametrista
(muuttujasta) $s$, ts.\ $\mA=(a_{ij}(s))=\mA(s)$, ja halutaan vaikkapa selvittää, millä $s$:n
arvoilla $\mA(s)$ on singulaarinen. Jos $\mA$ ei ole kooltaan kovin suuri, niin 
determinanttiehto
\[
\det\mA(s) = 0
\]
on ratkaisun lähtökohtana luonnollinen ja (etenkin käsinlaskussa) usein käytetty.

\subsection{Cramerin sääntö}

Determinantin muista käyttömuodoista on syytä vielä mainita (ilman todistusta, ks.\ 
Harj.teht.\,\ref{H-m-5: Cramer}) seuraavat kaksi laskusääntöä.

\begin{Lause} (\vahv{Cramerin\footnote[2]{Sveitsiläinen matemaatikko \hist{Gabriel Cramer}
(1704--1752) oli determinanttiteorian uran\-uurtajia teoksellaan ``Introduction
$\grave{\text{a}}$ l'analyse des lignes courbes alg$\acute{\text{e}}$brigues'' (1750).
\index{Cramer, G.|av}} sääntö}) \label{Cramerin sääntö} \index{Cramerin sääntö|emph}
Jos $\mA$ on säännöllinen neliömatriisi kokoa $n \times n$, niin yhtälöryhmän $\mA \mx = \mb$
ratkaisu on
\[
\mx=(x_i), \quad x_i=\frac{\det\mA^{(i)}}{\det\mA}\,, \quad i=1\,\ldots\,n,
\]
missä $\mA^{(i)}$ saadaan $\mA$:sta korvaamalla $i$:s sarake $\mb$:llä.
\end{Lause}
\begin{Lause} \label{käänteismatriisin determinanttikaava} Säännöllisen matriisin $\mA$ 
käänteismatriisi on 
\[
\mA^{-1}=\mB^T, \quad [\mB]_{ij}= (-1)^{i+j}\,\frac{\det\mA^{(ij)}}{\det\mA}\,.
\]
\end{Lause}
Cramerin säännöllä on pienikokoisia yhtälöryhmiä ratkaistaessa edelleen jonkin verran käyttöä 
käsinlaskussa. Etenkin jos kerroinmatriisin alkiot ovat kokonaislukuja, voidaan säännöllä 
minimoida (käsinlaskussa vaivalloiset) jakolaskut. Myös symbolisessa (käsin)laskennassa
\mbox{Cramerin} säännöllä on käyttöä samaan tapaan kuin determinantilla yleensä. 
--- Numeerisessa matriisilaskennassa sen sijaan \linebreak Cramerin säännöllä on kyseenalainen
kunnia esiintyä 'maailman huonoimpana' lineaarisen yhtälöryhmän ratkaisualgoritmina.

\Harj
\begin{enumerate}

\item \label{H-m-5: ortogonaalisen matriisin determinantti}
Näytä, että ortogonaalisen matriisin determinantilla on vain kaksi mahdollista arvoa:
Joko $\det\mA=1$ tai $\det\mA=-1$.

\item
Laske Sarrus'n säännöllä:
\[
\text{a)}\ \ \begin{detmatrix} 2&3&-1\\-1&2&0\\1&4&3 \end{detmatrix} \qquad
\text{b)}\ \ \begin{detmatrix} 0&-2&-3\\1&0&-2\\3&4&0 \end{detmatrix} \qquad
\text{c)}\ \ \begin{detmatrix} -3&7&2\\-5&4&0\\9&-1&-6 \end{detmatrix}
\]

\item
Olkoon $\alpha,\beta,\gamma\in\R$. Laske
\[
\text{a)}\ \ \begin{detmatrix}
             1&\alpha&\alpha^2\\1&\beta&\beta^2\\1&\gamma&\gamma^2
             \end{detmatrix} \qquad
\text{b)}\ \ \left| \begin{array}{lll}
             \alpha+\beta&\alpha+2\beta&\alpha+3\beta \\
             \alpha+3\beta&\alpha+\beta&\alpha+2\beta \\
             \alpha+2\beta&\alpha+3\beta&\alpha+\beta
             \end{array} \right| \qquad
\text{c)}\ \ \left| \begin{array}{lll}
             2&\alpha&\alpha^2 \\ 1&\alpha^2&\alpha \\ \alpha&\alpha^3&1
             \end{array} \right|
\]

\item
Laske a) alideterminanttikehitelmällä, b) ja c) Gaussin algortimilla, d) molemmilla:
\begin{align*}
&\text{a)}\ \ \begin{detmatrix} 1&-1&0&2\\2&1&0&0\\1&1&2&2\\0&0&1&1 \end{detmatrix} \qquad\,\
 \text{b)}\ \ \begin{detmatrix} 5&4&2&1\\2&3&1&-2\\-5&-7&-3&9\\1&-2&-1&4 \end{detmatrix} \\[3mm]
&\text{c)}\ \ \begin{detmatrix}
              2&1&0&0&4\\0&1&1&0&0\\0&0&1&1&0\\0&0&0&1&1\\3&0&0&0&5 
              \end{detmatrix} \qquad
 \text{d)}\ \ \begin{detmatrix} 1&1&4&5\\1&1&5&4\\2&4&1&1\\4&2&1&1 \end{detmatrix}
\end{align*}

\item
Laske $\det\mB$, kun
$\displaystyle{\
\mB=(-5\mA\mA^T)^7 \quad\text{ja} \quad  
\mA=\begin{rmatrix} 13&8&6\\-13&-8&-4\\8&5&5 \end{rmatrix}.
}$

\item
Määritä kaikki reaaliset tai kompleksiset $\lambda$:n arvot, joilla seuraavat matriisit ovat
singulaarisia.
\begin{align*}
&\text{a)}\ \ \begin{bmatrix} 2-\lambda&-4 \\ 2&6-\lambda \end{bmatrix} \qquad
 \text{b)}\ \ \begin{bmatrix} 
              \lambda-2&3&1 \\ \lambda-4&3&2 \\ \lambda-6&\lambda&3 
              \end{bmatrix} \\[3mm]
&\text{c)}\ \ \begin{bmatrix}
              1&3-\lambda&4 \\ 4-\lambda&2&-1 \\ 1&\lambda-6&2
              \end{bmatrix} \qquad
 \text{d)}\ \ \begin{bmatrix}
              1-\lambda&-1&2 \\ 1&2-\lambda&-13 \\ -2&1&1+\lambda
              \end{bmatrix}
\end{align*}

\item
Olkoon $\ma_1, \ldots, \ma_n\in\R^n$, ja määritellään
\[
\mb_k=\ma_k, \quad \mb_j=\ma_j+\beta_j\ma_k, \quad j=1 \ldots n,\ j \neq k,
\]
missä $1 \le k \le n$ ja $\beta_j\in\R$. Näytä determinanttifunktion $V$ ominaisuuksien 
perusteella, että pätee $\,V(\mb_1, \ldots, \mb_n)=V(\ma_1, \ldots, \ma_n)$. Miten tämä tulos 
liittyy Gaussin algoritmiin sovellettuna matriisiin $\mA=[\ma_1 \ldots \ma_n]^T\,$?

\item \index{neliömatriisi!f@tridiagonaalinen} \index{tridiagonaalinen matriisi}
Neliömatriisi $\mA=(a_{ij})$ olkoon kokoa $n \times n$ ja \kor{tridiagonaalinen}, ts.\
$a_{ij}=0$, kun $\abs{i-j} \ge 2$. Edelleen olkoon $a_{ii}=1,\ i=1 \ldots n$, ja 
$a_{ij}=\lambda$, kun $\abs{i-j}=1$. \ a) Näytä, että determinantille $D_n=\det\mA$ pätee
palautuskaava $D_n=D_{n-1}-\lambda^2 D_{n-2}$. \ b) Laske $D_n,\ n=2 \ldots 10$, kun
$\lambda=2$. c) Jos $\lambda=1$, niin millä $n$:n arvoilla $\mA$ on singulaarinen?

\item (*) \label{H-m-5: alidet-työ}
Näytä, että jos determinantti kokoa $n \ge 3$ lasketaan alideterminanttisäännöllä, niin
tarvittava kertolaskujen lukumäärä on
\[
W_n = n!\sum_{k=1}^{n-1} \frac{1}{k!}\,.
\]

\item (*) \label{H-m-5: alidet-tod}
Määritellään determinanttifunktio $V(\ma_1,\ldots,\ma_n)=\det[\ma_1 \ldots \ma_n]$ palautuvasti
käyttäen Lauseen \ref{alideterminanttisääntö} ensimmäistä purkusääntöä ($k=1$) sekä sääntöä
$\,\det a=a\,$ determinantille kokoa $n=1$. \vspace{1mm}\newline
a) Näytä induktiolla, että mainitulla tavalla määritellyllä funktiolla on determinanttifunktion
ominaisuudet (i)--(iii). \vspace{1mm}\newline
b) Näytä, että jos $p=(i_1,\ldots,i_n)$ on joukon $(1,\ldots,n)$ permutaatio ja
$\mI_p=[\me_{i_1} \ldots \me_{i_n}]$, niin a-kohdan määritelmän mukaisesti on joko
$\det\mI_p=1$ tai $\det\mI_p=-1$. Päättele, että edellisessä tapauksessa permutaatio $p$ on
parillinen (eli saavutettavissa parillisella määrällä parivaihtoja) ja jälkimmäisessä pariton
--- siis jokainen permutaatio on jompaa kumpaa tyyppiä.

\item (*) \label{H-m-5: Cramer}
Valitsemalla $\mb=b_j\me_j,\ j=1 \ldots n\,$ johda Cramerin sääntö alideterminanttisäännöstä. 
Todista edelleen Lause \ref{käänteismatriisin determinanttikaava} lähtien Cramerin säännöstä.

\end{enumerate}  %Determinantti
\section{Lineaarikuvaukset} \label{lineaarikuvaukset}
\alku
\index{lineaarikuvaus|vahv}
\index{funktio A!g@lineaarikuvaus|vahv}

Avaruuksien $\R^n$ ja $\R^m$ väliseksi \kor{lineaarikuvaukseksi} sanotaan funktiota tyyppiä
\[ 
\mv{f}:\ \R^n \kohti \R^m, \quad m,n \in \N, 
\]
($\DF_{\mv{f}}=\R^n,\ \RF_{\mv{f}}\subset\R^m$), jolle on voimassa ehto
\begin{equation} \label{lineaarikuvauksen ehto} 
\mv{f}(\alpha \mx + \beta \my)\ =\ \alpha \mv{f}(\mx) + \beta \mv{f}(\my), \quad 
                                         \mx,\my \in \R^n,\ \alpha,\beta \in \R.
\end{equation}
Lineaarikuvauksia tarkasteltaessa on tapana poiketa hieman tavallisista funktiomerkinnöistä: 
Lineaarikuvausta merkitään
\[ 
\mv{f}(\mx) = A \mx \quad \text{(lineaarikuvaus)}. 
\]
Oikealla siis jätetään sulkeet pois muuttujan ympäriltä, ja tämä ilmenee myös lukutavassa: 
Luetaan '$A\,x$' mieluummin kuin '$A$ $x$:ssä'. Lukutavan mukaisesti ajatellaan kuvaamisen 
(eli funktion arvon määräämisen) tapahtuvan siten, että $A$ \kor{operoi} $\mx$:ään. 
Operoinnin voi ymmärtää abstraktina kertolaskuna (= matriisin ja vektorin tulo), joka on
vektorien yhteenlaskun että skalaarilla kertomisen suhteen \kor{lineaarinen} eli toteutaa
tavanomaiset osittelulait näiden laskuoperaatioiden suhteen:
\index{lineaarisuus!c@lineaarikuvauksen}
\[ 
\boxed{ \quad\kehys A(\alpha \mx + \beta \my) 
            = \alpha A \mx + \beta A \my \quad \text{(lineaarikuvaus)}. \quad } 
\]
Lineaarikuvausten symboleina käytetään yleensä isoja kirjaimia $A,B$ jne. 

Jos \mA\ on matriisi kokoa $m \times n$, niin matriisialgebran sääntöjen perusteella on 
ilmeistä, että funktio $\mv{f}(\mx) = \mA \mx$ on lineaarikuvaus tyyppiä 
$\mv{f}:\ \R^n \kohti \R^m$. Osoittautuu, ettei muunlaisia, tätä tyyppiä olevia
lineaarikuvauksia olekaan:
\begin{Prop} \label{lineaarikuvaukset ja matriisit} Jos $\mv{f}:\ \R^n \kohti \R^m$ on 
lineaarikuvaus, niin on olemassa yksikäsitteinen matriisi \mA\ kokoa $m \times n$, siten että
$\mv{f}(\mx) = \mA\mx,\ \mx \in \R^n$. \end{Prop}
\tod Kun $\mx \in \R^n$ esitetään kannan $\{\me_1, \ldots, \me_n\}$ avulla muodossa
\[ 
\mx\ =\ \sum_{i=1}^n x_i \me_i, 
\]
niin $\mv{f}$:n lineaarisuuden ja matriisialgebran sääntöjen (Luku \ref{matriisialgebra}) 
nojalla
\[ 
\mv{f}(\mx)\ =\ \sum_{i=1}^n x_i \mv{f}(\me_i)\ =\ \mA\mx, \qquad 
                             \mA=[\mv{f}(\me_1) \ldots \mv{f}(\me_n)]. \loppu
\]

Proposition \ref{lineaarikuvaukset ja matriisit} perusteella lineaarikuvausten ja matriisien 
välillä on kääntäen yksikäsitteinen vastaavuus:
\[ 
\text{lineaarikuvaus}\ A:\ \R^n \kohti \R^m \ \ \longleftrightarrow\ \ 
            \text{matriisi}\ \mA\ \text{kokoa}\ m \times n\ \ (A\mx = \mA\mx).
\]
Jos matriisi \mA\ vastaa lineaarikuvausta $A$, niin vastaavuuteen viitataan sanomalla, että 
\index{lineaarikuvaus!lineaarikuvauksen matriisi}%
$\mA$ on ko.\ \kor{lineaarikuvauksen matriisi} (tai esitysmatriisi). Esitysmatriisin kautta 
voidaan siis lineaarikuvauksilla laskeminen aina palauttaa matriisialgebraan. Toisaalta voidaan
matriisin ja vektorin kertolasku aina haluttaessa nähdä lineaarisena kuvauksena, ts.\ erään 
lineaarikuvauksen laskusäännön soveltamisena. Erityisesti kahden matriisin tulo vastaa kahden
lineaarikuvauksen yhdistelyä:
\[ 
AB\ \vast\ \mA\mB. 
\]
Yhdistelyä koskevat rajoitukset ovat samat kuin matriisitulossa: On oletettava, että 
$B:\ \R^n \kohti \R^p$ ja  $A:\ \R^p \kohti \R^m$ joillakin $m,n,p\in\N$, jolloin myös tulo 
$\mA\mB$ on määritelty. 

\index{kzyzy@käänteiskuvaus}%
Lineaarikuvauksen $A:\ \R^n \kohti \R^m$ \kor{käänteiskuvauksen} $A^{-1}$ muodollinen
määritelmä on
\[ 
A\mx = \my \quad \ekv \quad \mx = A^{-1}\my. 
\]
Käänteiskuvaus on tällä tavoin määriteltävissä, mikäli $A$ on 1-1 ('yksi yhteen'), eli toteuttaa
ehdon
\[ 
A\mx_1 = A\mx_2 \quad \impl \quad \mx_1 = \mx_2. 
\]
Koska $A$ on lineaarinen, tämä on sama kuin ehto
\[ 
A\mx = \mv{0} \quad \impl \quad \mx = \mv{0}. 
\] 
Aiemmista tuloksista (Luku \ref{tuettu Gauss}) voidaan päätellä, että tämä ehto voi toteutua 
vain jos $m \ge n$ (välttämätön ehto). Jos ehto toteutuu, on $A$ \kor{kääntyvä}, ja 
käänteiskuvaus on tällöin määritelty kuvauksena tyyppiä $A^{-1}:\ \R^m \kohti \R^n$. 
Määrittelyjoukko ei kuitenkaan ole välttämättä koko $\R^m$, sillä käänteiskuvauksen määritelmän
mukaan $A^{-1}\my$ on määritelty vain sellaisille vektoreille $\my \in \R^m$, joille 
$\my = \mA\mx$ jollakin $\mx \in \R^n$, ts.\ yhtälöryhmällä $\mA\mx = \my$ on oltava ratkaisu. 
Aiempien tulosten perustella yhtälöryhmän $\mA\mx = \my$ ratkeavuus jokaisella $\my \in \R^m$ on
mahdollinen vain jos $m \le n$ (välttämätön ehto). Päätellään siis, että käänteiskuvaus $A^{-1}$
on on olemassa ja määritelty koko $\R^m$:ssä täsmälleen kun $m=n$ ja $A$:n matriisi $\mA$ 
(neliömatriisi) on säännöllinen. Käänteiskuvaus on tällöin lineaarikuvaus tyyppiä 
$A^{-1}:\,\R^n \kohti \R^n$ (samaa tyyppiä kuin $A$), ja käänteiskuvauksen matriisi on 
luonnollisesti $\mA^{-1}$.

Lineaarikuvauksen $A:\ \R^n \kohti \R^n$ käänteiskuvauksen lineaarisuutta käytetään usein
hyödyksi lineaarisia yhtälöryhmiä ratkaistaessa. Tyypillinen sovellutustilanne on sellainen,
että yhtälöryhmä $\mA\mx = \mb$ halutaan ratkaista useilla eri $\mb$:n arvoilla \mA:n pysyessä 
samana. Jos ratkaisua merkitään kuvauksena $\mb \map \mx$ (kuvauksen $\mx\map\mb=\mA\mx$ 
käänteiskuvaus) ja oletetaan, että tunnetaan ratkaisut $\mb_1 \map \mx_1$ ja 
$\mb_2 \map \mx_2$, niin lineaarisuuden perusteella tiedetään 
(vrt.\ Luku \ref{Gaussin algoritmi}, Esimerkki \ref{ristikkoesimerkki})
\[ 
\alpha \mb_1 + \beta \mb_2\ \longmapsto\ \alpha \mx_1 + \beta \mx_2 \quad (\alpha,\beta \in \R). 
\]

\subsection{Kannan vaihto $\R^n$:ssä}
\index{kanta!b@kannan vaihto|vahv}

$\R^n$:n vektorisysteemiä (järjestettyä joukkoa) $\{\ma_1, \ldots \ma_m\}$ sanotaan 
\index{lineaarinen riippumattomuus}%
\kor{lineaarisesti riippumattomaksi}, jos pätee
\[ 
x_1 \ma_1 + \ldots + x_m \ma_m\ =\ \mv{0} \quad \impl \quad x_i = 0, \ \ i = 1 \ldots m. 
\]
Jos vektorit $\ma_i$ kootaan matriisin $\mA$ sarakkeiksi ($\mA$ kokoa $n \times m$), niin 
voidaan kirjoittaa $x_1 \ma_1 + \ldots + x_m \ma_m\ = \mA\mx$ 
(vrt.\ Luku \ref{matriisialgebra}), joten lineaarisen riippumattomuuden ehto saa muodon
\[ 
\mx \in \R^m\ \ja\ \mA\mx = \mv{0} \quad \impl \quad \mx = \mv{0}, \quad\quad \mA 
                          = [\ma_1, \ldots, \ma_m]. 
\]
Luvun \ref{tuettu Gauss} tulosten mukaan tämä ehto voi olla voimassa vain kun $m \le n$. Siis
$\R^n$:ssä on jokainen $n+1$ vektorin (tai useamman vektorin) muodostama systeemi 
\index{lineaarinen riippuvuus}%
\kor{lineaarisesti riippuva}. Toisaalta jos $m=n$, niin lineaarisesti riippumattoman systeemin
muodostaa ainakin $\R^n$:n luonnollinen kanta $\{\me_1, \ldots \me_n\}$, jota ym.\ ehdossa 
vastaa matriisi $\mA=\mI$. Luku $n$ on siis lineaarisen riippumattomuuden kannalta kriittinen
luku, ja tämä onkin jo aiemmin nimetty avaruuden $\R^n$
\index{dimensio}%
\kor{dimensioksi} (=ulotteisuus). Jos
$m=n$, niin ym.\ ehto toteutuu täsmälleen kun $\mA$ on säännöllinen matriisi 
(Lause \ref{säännöllisyyskriteerit}). Näin ollen, ja koska $\mA$ on säännöllinen täsmälleen kun
$\mA^T$ on säännöllinen (ks.\ Luku \ref{inverssi}), saadaan neliömatriisin säännöllisyydelle
seuraavat uudet (geometriset) tulkinnat:
\[ \boxed{ \begin{aligned} 
\ykehys\quad \mA\ \text{säännöllinen} \quad 
             &\ekv \quad \text{\mA:n sarakkeet lineaarisesti riippumattomat} \quad \\
\akehys      &\ekv \quad \text{\mA:n rivit lineaarisesti riippumattomat}. \quad
           \end{aligned} } \] 
Jos $\{\ma_1, \ldots, \ma_n\}$ on $\R^n$:n lineaarisesti riippumaton vektorisysteemi, eli 
matriisi $\mA = [\ma_1, \ldots, \ma_n]$ on säännöllinen, niin mielivaltainen vektori 
$\mb \in \R^n$ voidaan lausua yksikäsitteisesti vektorien $\ma_i$ lineaarisena yhdistelynä. 
Nimittäin kertoimet $x_i$ ratkeavat yksikäsitteisesti yhtälöryhmästä
\[ 
\sum_{i=1}^n x_i \ma_i = \mb \quad \ekv \quad \mA\mx = \mb.
\]
Tällä perusteella voidaan sanoa, että jokainen lineaarisesti riippumaton $n$ vektorin systeemi
$\{\ma_1, \ldots, \ma_n\}$ on $\R^n$:n
\index{kanta}%
\kor{kanta}. Jos siis \mA\ on mikä tahansa säännöllinen
matriisi kokoa $n \times n$, niin $\mA$:n sarakkeet (myös rivit) muodostavat $\R^n$:n kannan. 
--- Huomattakoon, että tällainen kanta on
\index{kanta!a@ortonormeerattu}%
\kor{ortonormeerattu}, eli 
$\scp{\ma_i}{\ma_j} = \delta_{ij}$ (Kroneckerin $\delta$), täsmälleen kun \mA\ on ortogonaalinen
(vrt.\ Luku \ref{inverssi}). Päätellään siis, että jokainen säännöllinen matriisi kokoa 
$n \times n$ edustaa sarakkeittensa (tai riviensä) kautta $\R^n$:n kantaa, ja jokainen 
ortogonaalinen matriisi edustaa ortonormeerattua kantaa.

Jos halutaan suorittaa \kor{kannan vaihto}, lähtökohtana $\R^n$:n luonnollinen kanta 
$\{\me_1, \ldots, \me_n\}$, niin vektorin $\mx$ koordinaatit $x'_i$ kannassa 
$\{\ma_1, \ldots \ma_m\}$ ovat siis ratkaistavissa yhtälöryhmästä
\[ 
\mA\mx' = \mx \quad \impl \quad \mx' = \mA^{-1}\mx, \quad\quad \mA = [\ma_1, \ldots, \ma_n]. 
\]
Kannan vaihdossa, eli koordinaattimuunnoksessa $\mx \map \mx'$, on siis kyse 
lineaarikuvauksesta. Jos uusikin kanta on ortonormeerattu, niin muunnoksen matriisi on helppo
laskea: $\mA^{-1}=\mA^T$.
\begin{Exa} Määritä vektorin $x \vec{i} + y \vec{j} + z \vec{k}$ koordinaatit $x',y',z'$ 
kannassa, jonka muodostavat vektorit $\vec{a}_1 = \vec{i} + \vec{j}$, 
$\ \vec{a}_2 = \vec{j} + \vec{k}$ ja $\vec{a}_3 = \vec{i} + \vec{k}$.
\end{Exa}
\ratk 
\[ 
\begin{rmatrix} 1&0&1\\1&1&0\\0&1&1 \end {rmatrix} \begin{rmatrix} x'\\y'\\z' \end{rmatrix} =
\begin{rmatrix} x\\y\\x \end{rmatrix} \quad \impl \quad 
\begin{rmatrix} x'\\y'\\z' \end{rmatrix} =
     \frac{1}{2} \begin{rmatrix} 1&1&-1\\-1&1&1\\1&-1&1 \end{rmatrix} 
                 \begin{rmatrix} x\\y\\z \end{rmatrix}. \loppu
\]
\begin{Exa} \label{pallokoordinaatisto ja matriisit} 
Pallokoordinaatiston pisteessä $P=(r,\theta,\varphi)$ käytetään avaruusvektoreille
kantaa $\{\vec e_r,\vec e_\theta,\vec e_\varphi\}$, missä (vrt.\ Luku \ref{koordinaatistot})
\[
\left\{\begin{array}{ll}
\vec e_r       &= \ \sin{\theta} \cos{\varphi} \vec i + \sin{\theta}\sin{\varphi} \vec j 
                                                      + \cos{\theta} \vec k, \quad \\[4pt]
\vec e_\theta  &= \ \cos{\theta} \cos{\varphi} \vec i + \cos{\theta}\sin{\varphi} \vec j 
                                                      - \sin{\theta} \vec k, \\[4pt]
\vec e_\varphi &= \ -\sin{\varphi} \vec i + \cos{\varphi} \vec j.
\end{array} \right.
\]
Kun kirjoitetaan
\[
\vec F \,=\, F_1\vec i+F_2\vec j+F_3\vec k
       \,=\, F_r\vec e_r+F_\theta\vec e_\theta+F_\varphi\vec e_\varphi\,,
\]
niin muunnoskaava $(F_r,F_\theta,F_\varphi) \map (F_1,F_2,F_3)$ on matriisimuodossa
\[
\begin{rmatrix}
\sin\theta\cos\varphi&\cos\theta\cos\varphi&-\sin\varphi\\
\sin\theta\sin\varphi&\cos\theta\sin\varphi&\cos\varphi \\
\cos\theta&-\sin\theta&0
\end{rmatrix}
\begin{bmatrix} F_r\\F_\theta\\F_\varphi \end{bmatrix} =
\begin{bmatrix} F_1\\F_2\\F_3 \end{bmatrix}.
\]
Koska $\{\vec e_r,\vec e_\theta,\vec e_\varphi\}$ on ortonormeerattu systeemi, niin
käänteismuunnos saadaan transponoimalla:
\[
\begin{bmatrix} F_r\\F_\theta\\F_\varphi \end{bmatrix} =
\begin{rmatrix}
\sin\theta\cos\varphi&\sin\theta\sin\varphi&\cos\theta\\
\cos\theta\cos\varphi&\cos\theta\sin\varphi&-\sin\theta \\
-\sin\varphi&\cos\varphi&0
\end{rmatrix}
\begin{bmatrix} F_1\\F_2\\F_3 \end{bmatrix}. \loppu
\]
\end{Exa}

\subsection{Koordinaatiston kierto}
\index{koordinaatisto!d@koordinaatiston kierto|vahv}
\index{kierto!b@koordinaatiston|vahv}

Edellä todettiin, että jos $\mA$ on ortogonaalinen matriisi kokoa $n \times n$, niin $\mA$:n
sarakkeet (myös rivit) määrittelevät $\R^n$:n ortonormeeratun kannan. Sanotaan, että ko.\
\index{suunnistus (kannan)} \index{positiivinen suunnistus!a@kannan} 
\index{negatiivisesti!a@suunnistettu (kanta)}%
kanta on \kor{positiivisesti suunnistettu} (tai 'oikeakätinen'), jos $\det\mA=1$ ja
\kor{negatiivisesti suunnistettu} ('vasenkätinen'), jos $\det\mA=-1$. (Muita mahdollisia
arvoja ei $\det\mA$:lla ole, ks. Harj.teht. 
\ref{determinantti}:\,\ref{H-m-5: ortogonaalisen matriisin determinantti}). Jos yhden 
kantavektorin suunta vaihdetaan, niin determinanttiopin mukaisesti $\det\mA$:n merkki vaihtuu,
eli kannan suunnistus vaihtuu.
\begin{Exa} Jos $\{\vec a,\vec b,\vec c\,\}$ on avaruusvektoreiden ortonormeerattu kanta, niin
ko.\ vektoreiden koordinaattivektorit kannassa $\{\vec i,\vec j,\vec k\,\}$ muodostavat
$\R^3$:n ortonormeeratun kannan $\{\ma,\mb,\mc\}$, eli matriisi $\mA=[\ma\,\mb\,\mc]$ on
ortogonaalinen (kuten edellisessä esimerkissä). Geometrisin vektorioperaatioin laskettuna
$\mA$:n determinantti on (vrt.\ Luku \ref{ristitulo})
\[
\det\mA=\det\mA^T=(\vec a\times\vec b)\cdot\vec c=[\,\vec a\,\vec b\,\vec c\,].
\]
Koska $\{\vec a,\vec b,\vec c\,\}$ on ortonormeerattu systeemi, niin on joko 
$\vec c=\vec a\times\vec b$, jolloin $\det\mA=\abs{\vec c\,}^2=1$, tai 
$\vec c=-(\vec a\times\vec b)$, jolloin $\det\mA=-1$. Edellisessä tapauksessa kanta on siis 
positiivisesti, jälkimmäisessä negatiivisesti suunnistettu. \loppu
\end{Exa}
\begin{Exa} \label{napakoordinaatisto ja matriisit} Napakoordinaatiston kannat 
$\{\vec e_r,\vec e_\varphi\}$ (vrt.\ Luku \ref{koordinaatistot}) kattavat eri $\varphi$:n 
arvoilla kaikki mahdolliset tason vektoreiden ortonormeeratut, positiivisesti suunnistetut
systeemit. Näin ollen kaikki ortogonaaliset matriisit $\mA$ kokoa $2 \times 2$, jotka täyttävät
ehdon $\det\mA=1$, ovat muotoa
\[
\mA = \begin{rmatrix} 
      \cos\varphi&\sin\varphi\\-\sin\varphi&\cos\varphi
      \end{rmatrix}, \quad \varphi\in[0,2\pi). \loppu
\]
\end{Exa}

Jos $\mA=[\ma_1 \ldots \ma_n]$ on ortogonaalinen matriisi ja $\det\mA=1$, niin kannan vaihtoa
$\{\me_1,\ldots,\me_n\} \ext \{\ma_1,\ldots,\ma_n\}$ sanotaan \kor{koordinaatiston kierroksi}.
Nimittäin on osoitettavissa, että kannan vaihto on tällöin toteutettavissa j\pain{atkuvana} 
muunnoksena muotoa
\[
t\in[0,1]\ \map\ \{\ma_1(t),\ldots\,\ma_n(t)\}
\]
siten, että seuraavat ehdot toteutuvat:
\begin{itemize}
\item[(i)]   $\ma_i(0)=\me_i$ ja $\ma_i(1)=\ma_i$, $\ i=1 \ldots n$.
\item[(ii)]  Vektorisysteemi $\{\ma_1(t),\ldots,\ma_n(t)\}$ on ortonormeerattu ja positiivisesti
             suunnistettu jokaisella $t\in[0,1]$.
\item[(iii)] Jatkuvuusehto: Funktiot $t\in[0,1]\map\ma_j(t)\in\R^n$ ovat jatkuvia, ts.\
             reaalifunktiot $t \map a_{ij}(t)=[\ma_j(t)]_i,\ i,j=1 \ldots n\,$ ovat jatkuvia 
             välillä $[0,1]$. 
\end{itemize}
Ehdot täyttävä muunnos voidaan kuvitella toteutettavan niin, että vektorisysteemiä
$\{\me_1,\ldots,\me_n\}$ kierretään 'jäykkänä kappaleena', jolloin ortogonaalisuus ja suunnistus
säilyvät. Perussysteemistä $\{\me_1,\ldots,\me_n\}$ lähtien on siis tällaisella jatkuvalla 
kierrolla mahdollista päätyä mihin tahansa annettuun, ortonormeerattuun ja positiivisesti 
suunnistettuun systeemiin $\{\ma_1,\ldots,\ma_n\}$. Jos ko.\ systeemi on negatiivisesti 
suunnistettu, niin jossakin on tapahduttava 'hyppy' (yhden vektorin suunnan vaihto), jolloin 
muunnos ei ole jatkuva. Tämän väittämän mukaisesti siis kaikki $\R^n$:n positiivisesti 
suunnistetut ortonormeeratut systeemit ovat keskenään (samoin negatiivisesti suunnistetut 
\index{kiertoekvivalenssi}%
keskenään) \kor{kiertoekvivalentteja}. Tapauksessa $n=2$ väittämän voi todistaa
Esimerkin \ref{napakoordinaatisto ja matriisit} tuloksen perusteella 
(Harj.teht.\,\ref{H-m-6: jatkuva kierto tasossa}). Yleinen tapaus on haastavampi --- tarkemmat
perustelut sivuutetaan.

\subsection{$\R^n$:n aliavaruudet}
\index{aliavaruus|vahv}

Olkoon annettu $m$ vektoria $\ma_j\in\R^n,\ j=1 \ldots m$ (voi olla $m \le n$ tai $m>n$).
Tällöin vektorijoukko, joka määritellään
\[
W = \{\mpv\in\R^n \mid \mpv=\sum_{j=1}^m y_j\ma_j\,\ 
                                   \text{jollakin}\,\ [y_1,\ldots,y_m]^T\in\R^m\}
\]
on $\R^n$:n \kor{aliavaruus}, sillä määritelyn perusteella pätee $W\subset\R^n$ ja
\[
\mpv_1,\mpv_2 \in W \qimpl \alpha_1\mpv_1+\alpha_2\mpv_2 \in W \quad 
                                          \forall\ \alpha_1,\alpha_2\in\R.
\]
\index{virittää (aliavaruus)}%
Sanotaan, että $W$ on vektoreiden $\ma_1,\ldots,\ma_m$ \kor{virittämä} avaruus. Jos vektorit
$\ma_i$ ovat lineaarisesti riippumattomat (jolloin on oltava $m \le n$), niin dim W $=m$ ja 
$\{\ma_1,\ldots\ma_m\}$ on $W$:n kanta. Ovatko vektorit $\ma_i$ lineaarisesti riippumattomat vai
eivät, selviää Gaussin algoritmilla: Muodostetaan vektorit $\ma_j$ sarakkeina matriisi 
$\mA=[\ma_1\,\ldots\,\ma_m]$ ja etsitään yhtälöryhmän $\mA\mx=\mo$ kaikki ratkaisut $\mx\in\R^m$
(vrt.\ Luku \ref{tuettu Gauss}).
\begin{Exa} Ovatko $\R^4$:n vektorit $\ma_1 = [1,2,1,-1]^T$, $\ma_2 = [2,-1,3,-1]^T$ ja 
$\ma_3 = [0,5,-1,-1]^T$ lineaarisesti riippumattomat? 
\end{Exa}
\ratk Muunnetaan yhtälöryhmä $\mA\mx = \mv{0}$ Gaussin algoritmilla singulaariseen perusmuotoon
$\mU\mx = \mv{0}$:
\[ 
\begin{rmatrix} 1&2&0\\2&-1&5\\1&3&-1\\-1&-1&-1 \end{rmatrix} 
\begin{rmatrix} x_1\\x_2\\x_3 \end{rmatrix}\ =\ 
\begin{rmatrix} 0\\0\\0\\0 \end{rmatrix} \quad \ekv \quad
\begin{rmatrix} 1&2&0\\0&-5&-5\\0&0&0\\0&0&0 \end{rmatrix} 
\begin{rmatrix} x_1\\x_2\\x_3 \end{rmatrix}\ =\ \begin{rmatrix} 0\\0\\0\\0 \end{rmatrix}. 
\]
Tämän perusteella
\[ 
\mA\mx = \mv{0} \quad \ekv \quad \mx = t\,[-2,1,1]^T,\ t \in \R. 
\] 
Vastaus: Eivät, sillä $-2\ma_1+\ma_2+\ma_3 = \mv{0}$. \loppu 

Gaussin algoritmilla selviää myös yleisemmin, mikä on annettujen vektorien
$\ma_j\in\R^n,\ j=1 \ldots m$ virittämän aliavaruuden $W$ dimensio ja mikä ko.\
vektorisysteemin lineaarisesti riippumaton \pain{osas}y\pain{steemi} kelpaa $W$:n kannaksi 
silloin, kun koko systeemi on lineaarisesti riippuva.
\jatko \begin{Exa} (jatko) Esimerkin matriisin $\mA$ sarakkeet virittävät $\R^4$:n aliavaruuden
$W$ ja rivit $\R^3$:n aliavaruuden $U$. Määritä näiden aliavaruuksien dimensiot ja kummallekin 
jokin kanta.
\end{Exa}
\ratk
Esimerkin algoritmissa ei tehty rivien eikä sarakkeiden vaihtoja, joten tuloksesta on suoraan 
luettavissa: dim $W=$ dim $U=2$, $W$:n eräs kanta $=\mA$:n sarakkeet $1$ ja $2$ ja $U$:n eräs 
kanta $=\mA$:n rivit $1$ ja $2$, ts.\
\begin{align*}
W &= \{\mpv=\alpha_1[1,2,1,-1]^T+\alpha_2[2,-1,3,-1]^T,\,\ \alpha_1,\alpha_2\in\R\}, \\
U &= \{\mpv=\alpha_1[1,2,0]+\alpha_2[2,-1,5],\,\ \alpha_1,\alpha_2\in\R\}. \loppu
\end{align*}
\begin{Exa} Matriisin
\[
\mA=\begin{rmatrix}
    3&3&0&3&4&1\\1&-3&2&-1&-1&2\\-3&3&-3&0&2&-1\\6&-2&4&2&1&3
    \end{rmatrix}
\]
sarakkeet virittävät $\R^4$:n aliavaruuden $W$ ja rivit $\R^6$:n aliavaruuden $U$. Määritä
näiden aliavaruuksien dimensiot ja kummallekin jokin kanta.
\end{Exa}
\ratk Yhtälöryhmän
\[
\begin{rmatrix}
3&3&0&3&4&1\\1&-3&2&-1&-1&2\\-3&3&-3&0&2&-1\\6&-2&4&2&1&3
\end{rmatrix}
\begin{bmatrix} x_1\\x_2\\x_3\\x_4\\x_5\\x_6 \end{bmatrix} =
\begin{bmatrix} 0\\0\\0\\0\\0\\0 \end{bmatrix}
\]
singulaariseksi perusmuodoksi saadaan tuetulla Gaussin algoritmilla
\[
\begin{rmatrix}
3&3&4&1&0&3 \\[1mm] 0&-4&-\frac{7}{3}&\frac{5}{3}&2&-2 \\[1mm]
0&0&\frac{5}{2}&\frac{5}{2}&0&0 \\[1mm] 0&0&0&0&0&0
\end{rmatrix} 
\begin{bmatrix} x_1\\x_2\\x_5\\x_6\\x_3\\x_4 \end{bmatrix} =
\begin{bmatrix} 0\\0\\0\\0\\0\\0 \end{bmatrix}.
\]
Tähän muotoon pääsemiseksi ei ole tehty rivinvaihtoja. Sarakkeita sen sijaan on vaihdettu niin,
että uusi järjestys on $125634$, kuten näkyy lopputuloksesta. Päätellään siis: 
dim $W=$ dim $U=3$, $U$:n erään kannan muodostavat $\mA$:n kolme ensimmäistä riviä ja $W$:n 
erään kannan $\mA$:n sarakkeet $1$, $2$ ja $5$. \loppu  

Em.\ esimerkeissä matriisin sarakkeiden ja rivien virittämien avaruuksien dimensiot ovat samat.
Kyse ei ole sattumasta, vaan dimensiot ovat \pain{aina} samat: dimensioiden yhteinen arvo = 
matriisin säännöllisyysaste (ks.\ Luku \ref{tuettu Gauss}).
 
\subsection{*Ortogonaaliprojektio $\R^n$:ssä}
\index{ortogonaaliprojektio|vahv}

Olkoon $\{\ma_1, \ldots, \ma_m\},\ 1 \le m < n$, lineaarisesti riippumaton vektorisysteemi 
$\R^n$:ssä, ja olkoon $W$ ko.\ vektorien virittämä $R^n$:n (aito) aliavaruus:
\[ 
W = \{\mv{v} \in \R^n \mid \mv{v} = \sum_{i=1}^m x_i \ma_i,\ \ x_i \in \R \}. 
\]
Tällöin jos $\mv{u} \in \R^n$, niin $\mv{u}$:n \kor{ortogonaaliprojektio} aliavaruuteen $W$ 
määritellään vektorina $\mw$, joka toteuttaa
\[ 
\mw \in W\ \ \ja\ \ \scp{\mpu-\mw}{\mpv} = 0\ \ \forall\ \mpv \in W. 
\]
Kirjoittamalla tässä $\mpv$ vektoreiden $\ma_1, \ldots, \ma_m$ lineaarikombinaationa 
(mahdollista jokaiselle $\mpv \in W$) ja käyttämällä skalaaritulon bilineaarisuutta nähdään,
että ortogonaalisuusehto riittää asettaa vektoreille $\mpv=\ma_i,\ i= 1\,\ldots\,m\,$:
\[ 
\scp{\mpu-\mw}{\mpv} = 0\ \ \forall\ \mpv \in W  
               \qekv \scp{\mpu-\mw}{\ma_i} = 0,\ \ i = 1 \ldots m. 
\]
Ortogonaaliprojektion haku on näin pelkistetty yhtälöryhmäksi. Kun tässä kirjoitetaan vielä
$\mw = \sum_{j=1}^m x_j \ma_j$ (mahdollista, koska $\mw \in W$), ja käytetään edelleen 
skalaaritulon bilineaarisuutta ja symmetrisyyttä, niin nähdään, että kyseessä on seuraava 
lineaarinen yhtälöryhmä kerroinvektorille $\mx = [x_1, \ldots, x_m]^T\in\R^m$\,:
\[ 
\boxed{\quad\kehys \mA\mx = \mb, \qquad 
              [\mA]_{ij} = \scp{\ma_i}{\ma_j}, \quad [\mb]_i = \scp{\mpu}{\ma_i}, 
                                               \quad i,j = 1 \ldots m. \quad } 
\]
\index{normaalimuoto!b@projektio-ongelman}%
Tätä muotoilua sanotaan projektio-ongelman \kor{normaalimuodoksi}. 

Projektio-ongelmaan päädytään lineaarialgebran sovelluksissa usein minimointiongelman kautta:
Halutaan etsiä annettua vektoria $\mpu\in\R^n$ jonkin \pain{normin} $\norm{\cdot}$ mielessä 
\pain{lähin} vektori aliavaruudesta $W$, ts.\ halutaan ratkaista minimointiongelma
\[ 
\quad \mpv \in W: \quad \norm{\mpu-\mpv} = \text{min!} 
\]
Jos normi liittyy skalaarituloon $\,\scp{\cdot}{\cdot}\,$ siten, että 
$\norm{\mx}^2=\scp{\mx}{\mx}$ (ks.\ Luku \ref{abstrakti skalaaritulo}), niin ongelman ratkaisu
saadaan $\mpu$:n ortogonaaliprojektiona $\mw \in W$ ko.\ skalaaritulon suhteen. Nimittäin jos
$\mpv \in W$ on mielivaltainen ja merkitään $\mpv_0 = \mw-\mpv$, niin
$\scp{\mpu-\mw}{\mpv_0}=0$ (koska $\mpv_0 \in W$), joten seuraa
\begin{align*}
\norm{\mpu-\mpv}^2 &=\ \norm{(\mpu-\mw)+\mpv_0)}^2\ 
                    =\ \scp{\,(\mpu-\mw)+\mpv_0\,}{\,(\mpu-\mw)+\mpv_0\,} \\
                   &=\ \abs{\mpu-\mw}^2 + 2\,\scp{\mpu-\mw}{\mpv_0} + \abs{\mpv_0}^2 \\
                   &=\ \abs{\mpu-\mw}^2 + \abs{\mpv_0}^2.
\end{align*}
Siis $\norm{\mpu-\mpv}$:n minimiarvo saavutetaan, kun $\mpv_0=\mv{0}\ \ekv\ \mpv = \mw$.

Projektio-ongelman yksikäsitteisen ratkeavuuden takaa
\begin{Lause} \label{projektiolause} (\vahv{$\R^n$:n projektiolause}) Jos $W \subset \R^n$ on
lineaarisesti riippumattomien vektorien $\ma_1, \ldots, \ma_m$ virittämä $\R^n$:n aliavaruus 
($m<n$) ja $\mpu \in \R^n$, niin on olemassa yksikäsitteinen $\mw \in W$ siten, että 
$\ \scp{\mpu-\mw}{\mpv} = 0\ \ \forall\ \mpv \in W$. 
\end{Lause}
\tod Lause väittää, että matriisi $\mA\ = (\scp{\ma_i}{\ma_j})$ 
(neliömatriisi kokoa $m \times m$) on säännöllinen. Tämän näyttämiseksi olkoon
$\mw=\sum_{i=1}^m x_i\ma_i$ ja lasketaan ensin
\[ 
\abs{\mw}^2\  =\ \scp{\mw}{\mw}\ =\ \scp{\,\sum_{i=1}^m x_i \ma_i}{\sum_{j=1}^m x_j \ma_j\,}. 
\]
Skalaaritulon bilineaarisuuden ja matriisialgebran nojalla tämä purkautuu muotoon
\[ 
\abs{\mw}^2\ =\ \sum_{i=1}^m \sum_{j=1}^m \scp{\ma_i}{\ma_j} x_i x_j\ =\ \mx^T \mA\mx. 
\]
Tämän perusteella päätellään seuraavasti:
\[ 
\mA\mx = \mv{0} \qimpl \mx^T \mA\mx = 0 \qimpl \abs{\mw}^2 = 0 \qimpl \mw = \mv{0}. 
\]
Koska $\{\ma_1, \ldots, \ma_m\}$ oli lineaarisesti riippumaton systeemi, niin päätellään 
edelleen: $\ \mw=\mv{0}\,\ \impl\,\ \mx=\mv{0}$. \ On siis päätelty:\,
$\mA\mx=\mv{0}\,\ \impl\,\ \mx=\mv{0}$, joten $\mA$ on säännöllinen. \loppu

Projektio-ongelman normaalimuodon ratkaisu $\mx = \mA^{-1}\mb$ antaa siis ortogonaaliprojektion
$\mw$ koordinaatit $W$:n kannassa $\{\ma_1, \ldots, \ma_m\}$. Ratkaisua varten on ensin 
laskettava matriisi \mA\ ja vektori \mb. Nämä riippuvat vektoreista $\mpu$ ja $\{\ma_i\}$ sekä
valitusta $\R^n$:n skalaaritulosta, joka  siis voi olla muukin kuin euklidinen. Vektorin $\mb$
riippuvuus vektorista $\mpu$ voidaan purkaa matriisialgebraksi: 
\[ 
\mpu = \sum_{j=1}^n u_j \me_j \qimpl [\mb]_i\ =\ \scp{\mv{u}}{\ma_i}\ 
               =\ \scp{\ma_i}{\mv{u}}\ =\ \sum_{j=1}^n \scp{\ma_i}{\me_j} u_j. 
\]
Siis $\mb = \mB\mpu$, missä 
\[ 
[\mB]_{ij}\ =\ \scp{\ma_i}{\me_j}, \quad i = 1 \ldots m,\ j = 1 \ldots n. 
\] 
(Euklidisen skalaaritulon tapauksessa on $[\mB]_{ij} = [\ma_i]_j$). Tämän tulkinnan mukaisesti
ortogonaaliprojektio $\mpu\map\mx$ on lineaarikuvaus tyyppiä $P:\ \R^n \kohti \R^m$ ja $P$:n
matriisi on $\mA^{-1}\mB$.
\begin{Exa} \label{projektioesimerkki} Laske vektorin $\mpu=[0,1,0,0]^T$ ortogonaaliprojektio
euklidisen skalaaritulon suhteen aliavaruuteen $W\subset\R^4$, jonka virittävät vektorit
$\ma_1 = [1,1,1,1]^T$, $\ma_2 = [1,1,1,0]^T$ ja $\ma_3 = [0,1,1,1]^T$. Määritä myös 
yksikkövektori $\mv{n}\in\R^4$, joka on ortogonaalinen aliavaruutta $W$ vastaan. 
\end{Exa}
\ratk  Projektioperiaatteen mukaisesti vektorin $\mw = x_1 \ma_1 + x_2 \ma_2 + x_3 \ma_3$ 
koordinaatit $x_i$ määräytyvät ratkaisemalla yhtälöryhmä $\mA\mx = \mb$, missä
\[ 
\mA=(\,\scp{\ma_i}{\ma_j}\,)=\begin{rmatrix} 4&3&3\\3&3&2\\3&2&3 \end{rmatrix}, \quad
\mb=\mB\mv{u}=\begin{rmatrix} 1&1&1&1\\1&1&1&0\\0&1&1&1 \end{rmatrix} 
              \begin{rmatrix} 0\\1\\0\\0 \end{rmatrix}\ =\ 
              \begin{rmatrix} 1\\1\\1 \end{rmatrix}. 
\]
Ratkaisu on $\,\mx = \frac{1}{2}\,[-1,1,1]^T$, joten kysytty projektio on
\[
\mw\ = \frac{1}{2}(-\ma_1 + \ma_2 + \ma_3)^T\ =\ \frac{1}{2}[0,1,1,0]^T.
\]
Vektori 
\[ 
\mv{n}\ =\ \lambda\,(\mpu-\mw)\ =\ \frac{1}{2}\lambda\,[0,1,-1,0]^T
\]
on ortogonaalinen $W$:tä vastaan, joten kysyttyjä yksikkövektoreita ovat
\[ 
\mv{n} = \pm \frac{1}{\sqrt{2}}\,[0,1,-1,0]^T. \qquad \loppu 
\]

\Harj
\begin{enumerate}

\item
a) Lineaarikuvaukselle $A:\,\R^3\map\R^3$ pätee: $A\ma=(0,2,1)$, $A\mb=(1,2,3)$ ja
$A\mc=(1,-1,0)$. Laske $A(3\ma-2\mb+\mc)$. \newline
b) Lineaarikuvaukselle $A:\,\R^2\map\R^2$ pätee: $A(1,1)=(3,-1)$ ja $A(2,-1)=(1,2)$. Laske
$A$:n matriisi. \newline
c) Funktiolle $\mf:\,\R^3\map\R^3$ pätee: $\mf(0,1,1)=(1,0,0)$, $\mf(1,0,1)=(1,1,0)$ ja
$\mf(1,-1,0)=(1,1,1)$. Voiko $\mf$ olla lineaarikuvaus?

\item
Olkoon $A:\,\R^n\map\R^n$ lineaarikuvaus ja $\ma_1,\ldots,\ma_m\in\R^n$. \newline
a) Näytä, että jos $\{\ma_1,\ldots,\ma_m\}$ on lineaarisesti riippuva niin samoin on 
$\{A\ma_1,\ldots,A\ma_m\}$. \newline
b) Näytä, että jos $\{A\ma_1,\ldots,A\ma_m\}$ on lineaarisesti riippumaton, siin samoin on
$\{\ma_1,\ldots,\ma_m\}$. \newline
 c) Oletetaan lisäksi, että $A$ on kääntyvä. Näytä, että tällöin $\{\ma_1,\ldots,\ma_m\}$ on 
lineaarisesti riippuva/riippumaton täsmälleen kun $\{A\ma_1,\ldots,A\ma_m\}$ on lineaarisesti 
riippuva/riippumaton.

\item
Laske matriisimuotoiset koordinaattien muunnoskaavat $\mx=\mA\mx'$ ja $\mx'=\mB\mx$ seuraaville
$\R^n$:n tai vastaavan geometrisen vektoriavaruuden kannan vaihdoille. \vspace{1mm}\newline
a) \ $\{\me_1,\me_2\} \ext \{\me_1+9\me_2,6\me_1+8\me_2\}$ \vspace{1mm}\newline
b) \ $\{\vec i,\vec j\,\} \ext \{\vec i-2\vec j,3\vec i+\vec j\,\}$ \vspace{0.3mm}\newline
c) \ $\{\me_1,\me_2\} \ext \{\ma,\mb\},\ \ma=[5,-1]^T,\ \mb=[1,5]^T$ \vspace{1mm}\newline
d) \ $\{\ma,\mb\} \ext \{\mc,\md\},\,\ \ma=[1,1]^T,\ \mb=[1,2]^T,\ \mc=[2,1]^T,\ \md=[1,-1]^T$
\vspace{0.5mm}\newline
e) \ $\{\vec i,\vec j,\vec k\,\} \ext \{\vec i+2\vec j,3\vec j+4\vec k,6\vec i+5\vec k\,\}$
\vspace{1mm}\newline
f) \ $\,\{\me_1,\me_2,\me_3\} \ext \{\me_1+\me_2+\me_3,\me_1-\me_2+\me_3,-\me_1+\me_2+\me_3\}$
\vspace{0.8mm}\newline
g) \ $\{\me_1,\ldots,\me_4\} \ext \{\ma_1,\ldots,\,\ma_4\},\,\ \ma_i=\me_1+\ldots+\me_i,\ 
i=1 \ldots 4$ \vspace{1.5mm}\newline
h) \ $\{\me_1,\ldots,\me_n\}\ext\{\me_1+\me_2,\me_2+\me_3,\ldots,\me_{n-1}+\me_n,\me_1+\me_n\},\
     n=7$

\item
Tutki, ovatko seuraavat vektorit lineaarisesti riippumattomat.
\vspace{1mm}\newline
a) \ $[0,0,4,1],\ [2,1,-1,1],\ [1,-1,2,1],\ [1,2,1,1]$ \newline
b) \ $[1,-1,1,-1,1],\ [1,1,1,1,1],\ [1,1,-1,1,1],\ [1,1,1,-1,1]$ \newline
c) \ $[1,1,1,1,1,1,1,1,1],\ [0,1,1,1,1,1,1,1,1,1],\ \ldots,\ [0,0,0,0,0,0,0,0,1]$

\item \label{H-m-6: jatkuva kierto tasossa}
Konstruoi jatkuva, kannan ortonormeerauksen ja suunnistuksen säilyttävä muunnos
$t\in[0,1]\map\{\ma(t),\mb(t)\}$ $\R^2$:n kannasta $\{\me_1,\me_2\}$ annettuun,
ortonormeerattuun ja positiivisesti suunnistettuun kantaan $\{\ma,\mb\}$. \newline
\kor{Vihje}: Esimerkki \ref{napakoordinaatisto ja matriisit}.

\item \label{H-m-6: kiertoja}
Näytä, että seuraavissa $\R^3$:n koordinaattien muunnoskaavoissa on kyse koordinaatiston
kierrosta. Mikä on käänteinen muunnoskaava? Minkä pisteiden koordinaatit pysyvät kierrossa
ennallaan?
\begin{align*}
&\text{a)}\ \ \begin{bmatrix} x'\\y'\\z' \end{bmatrix} =
              \frac{1}{7} \begin{rmatrix} 3&-2&6\\2&-6&-3\\6&3&-2 \end{rmatrix}
              \begin{bmatrix} x\\y\\z \end{bmatrix} \\[2mm]
&\text{b)}\ \ \begin{bmatrix} x'\\y'\\z' \end{bmatrix} =
              \frac{1}{15} \begin{rmatrix} 10&-5&10\\-11&-2&10\\-2&-14&-5 \end{rmatrix}
              \begin{bmatrix} x\\y\\z \end{bmatrix}
\end{align*}

\item
Seuraavien matriisien sarakket ja rivit virittävät erään $\R^n$:n aliavaruuden
($n=3,4,5$ tai $7$). Määritä tuetulla Gaussin algoritmilla ko.\ avaruuden dimensio ja jokin 
kanta.
\begin{align*}
&\text{a)}\ \ \begin{rmatrix} 1&5&0\\1&2&3\\1&3&2 \end{rmatrix} \qquad
 \text{b)}\ \ \begin{rmatrix} 3&4&-2&1\\2&1&3&-2\\1&-2&8&-5 \end{rmatrix} \\[3mm]
&\text{c)}\ \ \begin{rmatrix} 
              1&2&-3&1&5\\3&0&3&-3&3\\2&-1&4&-3&1\\0&2&-4&2&4 
              \end{rmatrix} \qquad
 \text{d)}\ \ \begin{rmatrix}
              1&0&-1&-1&3&3&5\\1&1&1&2&2&-1&3\\1&2&3&5&1&-5&1\\3&2&1&3&7&2&11
              \end{rmatrix}
\end{align*}

\item (*)
$\R^3$:n yleinen kierretty (ortonormeerattu ja positiivisesti suunnistettu) kanta
$\{\ma,\mb,\mc\}$ syntyy kiertämällä kantaa $\{\me_1,\me_2,\me_3\}$ $x_1$-akselin ympäri
kulma $\alpha$, kierron tulosta $x_2$-akselin ympäri kulma $\beta$ ja lopuksi näiden kiertojen
tulosta $x_3$-akselin ympäri kulma $\gamma$. Määritä vektorit $\ma,\mb,\mc$ kulmien
$\,\alpha,\beta,\gamma\,$ avulla. Millä kulmien arvoilla syntyy a) pallokoordinaattikanta
$\{\me_r,\me_\theta,\me_\varphi\}$ vastaten pallonpintakoordinaatteja $\,\theta,\varphi\,$,\,
b) tehtävän \ref{H-m-6: kiertoja}a kierretty kanta?

\item (*)
Olkoon $W\subset\R^4$ aliavaruus, jonka virittävät vektorit $\ma=[1,1,1,1]$ ja 
$\mb=[1,1,-1,-1]$. Määritä vektorin $\mpu=[1,2,-1,3]$ ortogonaaliprojektio aliavaruuteen $W$ \
a) euklidisen skalaaritulon, \ b) skalaaritulon $\scp{\mx}{\my}=x_1y_1+x_2y_2+2x_3y_3+3x_4y_4\,$
mielessä. Millaisen minimointiongelman muotoa
\[
\mx \in W:\ f(\mx)\ =\ \text{min!}
\]
nämä ortogonaaliprojektiot ratkaisevat?

\end{enumerate}
 %Lineaarikuvaukset
\section{Affiinikuvaukset. Geometriset kuvaukset} \label{affiinikuvaukset} 
\sectionmark{Affiinikuvaukset}
\alku
\index{affiinikuvaus|vahv}
\index{funktio A!h@affiinikuvaus|vahv}

\kor{Affiinikuvaukset} ovat lineaarikuvauksia hieman yleisempiä kuvauksia tyyppiä
\[ 
\mf:\ \R^n \kohti \R^m, \quad \mf(\mx) = \mA\mx + \mb, 
\]
missä $\mA$ on matriisi kokoa $m \times n$ ja $\mb \in \R^m$. Affiinikuvaukset ovat oma tärkeä
luokkansa lineaaristen ($\mb = \mv{0}$) ja yleisempien
\index{epzy@epälineaarinen kuvaus}%
\kor{epälineaaristen} (= ei-lineaaristen)
kuvausten välimaastossa. Ominaisuuksiltaan affiinikuvaukset ovat lähellä lineaarikuvauksia. 
Esim.\ jos $\mf_1(\mx) = \mA_1 \mx + \mb_1$ on affiinikuvaus tyyppiä 
$\mf_1:\,\R^n \kohti \R^p$ ja $\mf_2(\mx) = \mA_2 \mx + \mb_2$ on affiinikuvaus typpiä 
$\mf_2:\,\R^p \kohti \R^m$, niin yhdistetty kuvaus $\mf_2\circ\mv{f}_1$ on määritelty ja myös 
affiininen:
\[ 
(\mf_2\circ\mf_1)(\mx) = \mA_2(\mA_1\mx + \mb_1) + \mb_2 
                       = \mA_2\mA_1\mx + (\mA_2\mb_1 + \mb_2), \quad \mx \in \R^n. 
\]
Samoin nähdään, että jos $\mf(\mx) =  \mA\mx + \mb$ on tyyppiä $\mf:\ \R^n \kohti \R^n$, niin 
$\mf$ on kääntyvä täsmälleen kun $\mA$ on säännöllinen matriisi, jolloin myös
\index{kzyzy@käänteiskuvaus}%
käänteiskuvaus $\mf^{-1}$ on affiininen.

Edellisessä luvussa todettiin, että $\R^n$:n kannan vaihtoa vastaava koordinaattimuunnos on
lineaarikuvaus. Jos kannan vaihtoon yhdistetään myös
\index{origon siirto}%
\kor{origon siirto}, niin koordinaattimuunnoksesta tulee affiinikuvaus.
\begin{Exa} Esitä affiinikuvauksina koordinaattien väliset muunnoskaavat $E^3$:n 
koordinaatistojen $(O,\vec i,\vec j,\vec k\,)$ ja $(O',\vec a,\vec b,\vec c\,)$ välillä, kun
ensin mainitussa koordinaatistossa on $O'=(1,-2,2)$ ja
\[
\vec a=\vec i+\vec j-\vec k,\ \ \vec b=\vec j+2\vec k,\ \ \vec c=\vec i+\vec k.
\]
\end{Exa}
\ratk Jos pisteen $P$ koordinaatit ovat $(x,y,z)$ ja $(x',y',z')$, niin on oltava voimassa
\[
x\vec i+y\vec j+z\vec k=\Vect{OO'}+x'\vec a+y'\vec b+z'\vec c.
\]
Tämä on tulkittavissa vektorimuotoiseksi muunnoskaavaksi $(x',y',z') \map (x,y,z)$, jonka
matriisimuoto on affiinikuvaus:
\[
\begin{bmatrix} x\\y\\z \end{bmatrix} = \begin{rmatrix} 1\\-2\\2 \end{rmatrix} +
\begin{rmatrix} 1&0&1\\1&1&0\\-1&2&1 \end{rmatrix} \begin{bmatrix} x'\\y'\\z' \end{bmatrix}.
\]
Ratkaisemalla tästä $[x',y',z']^T$ saadaan käänteismuunnokseksi
\[
\begin{bmatrix} x'\\y'\\z' \end{bmatrix} = 
\frac{1}{4}\begin{rmatrix} 1&2&-1\\-1&2&1\\3&-2&1 \end{rmatrix} 
           \begin{bmatrix} x\\y\\z \end{bmatrix} +
\frac{1}{4}\begin{rmatrix} 5\\3\\-9 \end{rmatrix}. \loppu
\]

\subsection{Geometriset kuvaukset}
\index{geometrinen kuvaus|vahv}

\kor{Geometrisilla kuvauksilla} ymmärretään yleensä euklidisen pisteavaruuden $E^2$ tai $E^3$
kuvauksia, joilla on jokin (verrattain yksinkertainen) geometrinen merkitys. 
Koordinaattiavaruuteen palauttamalla voidaan geometriset kuvaukset aina tulkita myös $\R^2$:n 
tai $\R^3$:n kuvauksina. Jatkossa rajoitutaan yksinkertaisimpiin geometrisiin kuvauksiin,
jotka ovat affiinisia. Tällaisia ovat mm.\ seuraavat kuvaukset $P \map Q$ tyyppiä 
$f:\ \Ekaksi\kohti\Ekaksi$ tai $f:\ \Ekolme\kohti\Ekolme$ tai näiden yhdistelyt.
\index{suuntaisprojektio} \index{yhdensuuntaisprojektio} \index{peilaus}
\index{skaalaus (homotetia)} \index{homotetia (skaalaus)} \index{translaatio (siirto)}
\index{siirto (translaatio)} \index{kierto!a@geom.\ kuvaus}%
\begin{itemize}
\item[---] \kor{siirto} eli \kor{translaatio}: $\ \Vect{OQ} = \Vect{OP} + \vec{b}\ $ 
           ($\vec{b}$ annettu)
\item[---] \kor{kierto} tasossa origon ympäri: $\ \Vect{OQ} = \Vect{OP}$ kierrettynä kulman 
            $\varphi$ verran vastapäivään ($\varphi \in \R$ annettu)
\item[---] \kor{yhdensuuntaisprojektio} avaruustasolle $T$ (tason suoralle $S$) suoran $L$,
           tai vektorin $\vec v=L$:n suuntavektori, suunnassa: $\Q \in T$ tai $Q \in S$ ja
           $\Vect{PQ}=\lambda\vec v,\ \lambda\in\R\ $ ($\,\vec v\,$ ei $T$:n/$S$:n suuntainen)
\item[---] \kor{suuntaisprojektio avaruussuoralle} $S$ tason $T$, tai vektorien
           $\vec v_1,\,\vec v_2=T$:n suuntavektorit, suunnassa : $Q \in S$ ja 
           $\Vect{PQ}=\lambda\vec{a}_1 + \mu\vec{a}_2,\ (\lambda,\mu)\in\R^2$
           ($\,\{\vec v,\vec{a}_1,\vec{a}_2\}$ lineaarisesti riippumaton, $\vec v=S$:n
           suuntavektori)
\item[---] \kor{peilaus} annetun tason (avaruudessa) tai suoran (tasossa) suhteen
\item[---] \kor{skaalaus} eli \kor{homotetia} origon suhteen: 
           $\ \Vect{OQ} = \lambda\,\Vect{OP}\ $ ($\lambda \in \R$ annettu)
\end{itemize}
Geometrirista kuvauksista yksinkertaisimmat ovat siirto ja homotetia, joiden affiinimuodot ovat 
$\mv{f}(\mx) = \mI\mx + \mb$ ja $\mv{f}(\mx)=\lambda\mI\mx$ ($\mI$ yksikkömatriisi). Yleinen
affinikuvaus $\mv{f}(\mx) = \mA\mx + \mb$ on lineaarikuvauksen ja siirron yhdistelmä:
\[ 
\mx \Map \mA\mx \Map \mA\mx + \mb. 
\]
Jos yhdensuuntaisprojektiossa taso $T$/suora $S$ tai suuntaisprojektiossa suora $S$ kulkee
origon kautta, on $\mf(\mo)=\mo$, jolloin projektio on lineaarikuvuas. Näiden projektioiden
erikoistapauksia ovat ortogonaaliprojektiot, joissa projisoidaan tasoa/suoraa vastaan
kohtisuorasti.
\begin{Exa} \label{3d-projektio matriiseilla} Taso $T$ kulkee origon kautta ja sen 
suuntavektorit ovat $\ \vec{v}_1 = \vec{i} + \vec{j}\ $ ja $\ \vec{v}_2 = \vec{j} + \vec{k}$. 
Piste $\ P\ \vastaa\ x_1 \vec{i} + x_2 \vec{j} + x_3 \vec{k}\ \vastaa\ [x_1,x_2,x_3]^T = \mx\ $
projisoidaan kohtisuorasti tason $T$ pisteeksi 
\[ 
Q\ \vastaa\ y_1 \vec{i} + y_2 \vec{j} + y_3 \vec{k}\ =\ z_1 \vec{v}_1 + z_2 \vec{v}_2. 
\]
Määritä vastaavien lineaarikuvausten
\[ 
A:\ \mx \Map \my = [y_1,y_2,y_3]^T, \quad\quad  B:\ \mx \Map \mz = [z_1,z_2]^T 
\]
matriisit $\mA,\mB$. 
\end{Exa}
\ratk Projektioehto on
\begin{align*}
(x_1 \vec{i} &+ x_2 \vec{j} + x_3 \vec{k} - z_1 \vec{v}_1 - z_2 \vec{v}_2) \cdot \vec{v}_i = 0,
                                                         \quad i = 1,2 \\[2mm]
             &\ekv \quad \left\{ \begin{aligned} 
                                  x_1 + x_2 - 2z_1 - z_2 &= 0 \\ 
                                  x_2 + x_3 -z_1 -2z_2 &= 0 
                                  \end{aligned} \right. \\
             &\ekv \quad \begin{rmatrix} 2&1\\1&2 \end{rmatrix} 
                         \begin{rmatrix} z_1\\z_2 \end{rmatrix}\ =\ 
                         \begin{rmatrix} 1&1&0\\0&1&1 \end{rmatrix} 
                         \begin{rmatrix} x_1\\x_2\\x_3 \end{rmatrix} \\[2mm]
             &\ekv \quad \mC\mz = \mv{D}\mx.
\end{align*}
Siis $\mz = \mB\mx$, missä
\[ 
\mB\ =\ \mC^{-1}\mv{D}\ =\ \frac{1}{3} \begin{rmatrix} 2&-1\\-1&2 \end{rmatrix} 
                                       \begin{rmatrix} 1&1&0\\0&1&1 \end{rmatrix}\ 
                        =\ \frac{1}{3} \begin{rmatrix} 2&1&-1\\-1&1&2 \end{rmatrix}. 
\]
Koska
\[ y_1 \vec{i} + y_2 \vec{j} + y_3 \vec{k}\ =\ z_1 \vec{v}_1 + z_2 \vec{v}_2\ 
                                            =\ z_1 \vec{i} + (z_1 + z_2) \vec{j} + z_2 \vec{k}, 
\]
niin
\[ \begin{rmatrix} y_1\\y_2\\y_3 \end{rmatrix}\ =\ 
   \begin{rmatrix} 1&0\\1&1\\0&1 \end{rmatrix} \begin{rmatrix} z_1\\z_2 \end{rmatrix}\ =\ 
   \begin{rmatrix} 1&0\\1&1\\0&1 \end{rmatrix} \mB \begin{rmatrix} x_1\\x_2\\x_3 \end{rmatrix}. 
\]
Siis $\my = \mA\mx$, missä
\[ 
\mA\ =\ \frac{1}{3} \begin{rmatrix} 1&0\\1&1\\0&1 \end{rmatrix} 
                    \begin{rmatrix} 2&1&-1\\-1&1&2 \end{rmatrix}\ =\ 
        \frac{1}{3} \begin{rmatrix} 2&1&-1\\1&2&1\\-1&1&2 \end{rmatrix}. 
\]
\pain{Tarkistus}: Soveltamalla yhtälöryhmään $\mA\mx = \mv{0}$ Gaussin algoritmia saadaan 
ratkaisuksi $\mx = t[1,-1,1]^T,\ t\in\R$. Tämä sopii yhteen sen tiedon kanssa, että tason $T$
normaalivektori on $\ \vec n = \vec i-\vec j+\vec k$. \quad \loppu

Ortogonaaliprojektion avulla voidaan myös peilauskuvaus määritellä helposti: Jos $P_0$ on
$P$:n projektiopiste, niin $\Vect{OQ} = \Vect{OP} + 2\,\Vect{PP}_0$ määrittää peilikuvan $Q$.
\jatko \begin{Exa} (jatko) Taso $T$ kulkee pisteen $(1,-1,2)$ kautta ja tason suuntavektoreita
ovat $\ \vec{v}_1 = \vec{i} + \vec{j}\ $ ja $\ \vec{v}_2 = \vec{j} + \vec{k}$. Määritä 
peilauskuvaus $(x_1,x_2,x_3) \triangleq P \Map Q \triangleq (y_1,y_2,y_3)$ 
affiinikuvauksena $\my = \mA\mx + \mb$. 
\end{Exa}
\ratk Merkitään $\ma = [1,-1,2]^T$ ja olkoon projektiopiste $P_0 \triangleq (z_1,z_2,z_3)$. 
Tällöin $\mz-\ma = A(\mx-\ma)$, missä $A$ on edellisessä esimerkissä laskettu lineaarikuvaus. 
Kun tämän matriisia merkitään nyt symbolilla $\mA_0$, niin on siis
\[ 
\mz - \ma = \mA_0(\mx - \ma), \quad  \mA_0\ =\ \frac{1}{3} \begin{rmatrix} 
                                                           2&1&-1\\1&2&1\\-1&1&2 
                                                           \end{rmatrix}. 
\]
Näin ollen
\begin{align*}
\my\ &=\ \mx + 2(\mz - \mx)\ =\ (2\mA_0-\mI)\mx + 2(\mI-\mA_0)\ma \\[2mm]
     &=\ \frac{1}{3} \begin{rmatrix} 1&2&-2\\2&1&2\\-2&2&1 \end{rmatrix} 
                     \begin{rmatrix} x_1\\x_2\\x_3 \end{rmatrix}
                              + \frac{8}{3} \begin{rmatrix} 1\\-1\\1 \end{rmatrix} \\[2mm]
     &=\ \mA\mx + \mb. \quad\quad \loppu
\end{align*}
\begin{Exa} Tulkitse kierto tasossa lineaarikuvauksina $\mx \map \my = \mA\mx$. \end{Exa}
\ratk  Vektorin kierto tasossa annetun kulman $\varphi$ verran on tulkittavissa joko
lineaarikuvauksena tyyppiä $A: \R^2\kohti\R^2$ tai kompleksitason kuvauksena
$z = x_1 + i x_2 \Map w = y_1 + iy_2$. Jälkimmäisen tulkinnan perusteella on
\begin{align*}
 y_1 + i y_2 &= (\cos\varphi + i \sin\varphi)(x_1 + i x_2) \\ 
             &= (x_1\cos\varphi - x_2\sin\varphi) + i(x_1\sin\varphi + x_2\cos\varphi),
\end{align*}
joten $A$:n matriisi on
\[ 
\mA\ =\ \begin{rmatrix} 
        \cos\varphi & -\sin\varphi \\ \sin\varphi & \cos\varphi 
        \end{rmatrix}. \qquad \loppu 
\]
\index{kierto!a@geom.\ kuvaus}%
Myös pisteen \kor{kierto avaruussuoran ympäri} on tulkittavissa affiinikuvauksena: Jos $P_0$ on
$P$:n kohtisuora projektio suoralle $S$ (erikoistapaus yhdensuuntaisprojektiosta 
avaruussuoralle), niin $\Vect{OQ} = \Vect{OP}_0 + \Vect{P_0 Q}$, missä $\Vect{P_0 Q} =$ vektori
$\Vect{P_0 P}$ kierrettynä pisteen $P_0$ kautta kulkevassa tasossa, jonka normaali on $S$. 
\jatko \begin{Exa} (jatko) Vektorin $\vec{r} = x_1 \vec{i} + x_2 \vec{j} + x_3 \vec{k}$ 
kiertyessä $x_1$-akselin ympäri pysyy komponentti  $x_1$ muuttumattomana ja $[x_2,x_3]^T$
muuntuu kuten tasokierrossa. Näin ollen kierto vastaa lineaarikuvausta 
$\mv{f}(\mx) = \mA\mx$, missä
\[ 
\mA\ =\ \begin{rmatrix} 
         1&0&0\\0&\cos\varphi&-\sin\varphi\\0&\sin\varphi&\cos\varphi 
         \end{rmatrix}. \qquad \loppu 
\] 
\end{Exa}

Siirron, taso- tai avaruuskierron, peilauksen ja skaalauksen yhdistelmiä $E^n$:ssä 
($n=2$ tai $n=3$) kutsutaan
\index{euklidinen kuvaus}%
\kor{euklidisiksi kuvauksiksi}\footnote[2]{Kuvausta $K \Map \mv{f}(K)$, missä $\mv{f}$ on
pelkästään siirtojen ja kiertojen yhdistelmä, sanotaan \kor{euklidiseksi liikkeeksi},
mekaniikassa (kun $K \subset \Ekolme$ edustaa kiinteää kappaletta) jäykän kappaleen
liikkeeksi. \index{euklidinen liike|av}}. Euklidisen tason kahta pistejoukkoa
$K_1 \subset E^2$ ja $K_2 \subset E^2$ (nämä voivat olla geometrisia kuvioita kuten kolmioita)
\index{yhdenmuotoisuus}%
sanotaan \kor{yhdenmuotoisiksi}, jos on olemassa euklidinen kuvaus $\mv{f}:\ E^2 \kohti E^2$
siten, että $K_2 = \mv{f}(K_1)$. Jos $\mv{f}$ koostuu vain siirroista, kierroista ja
peilauksista, niin sanotaan, että $K_1$ ja $\mv{f}(K_2)$ ovat
\index{yhtenevyys}%
\kor{yhtenevät}.

\subsection{*$\R^n$:n geometriaa}

Geometrinen ajattelu voidaan ulottaa yleisesti avaruuteen $\R^n$, jolloin $\R^n$:n alkiot ovat
(kuvitellun) euklidisen pisteavaruuden $E^n$ pisteiden vastineita. Sanonta 'pisteessä $\mx$' on
tällöin tulkittava niin, että piste $P \in E^n$ ja sen vastine $\mx \in \R^n$ 
($=P$:n koordinaatit $\R^n$:n valitussa ortonormeeratussa kannassa) samastetaan, tai
jälkimmäinen tulkitaan edellisen 'nimeksi'. Kun omaksutaan tämä ajattelutapa, niin geometrian
pisteet, janat, kolmiot ym.\ voidaan tuoda luontevasti suoraan avaruuteen $\R^n$.

Seuraavassa luetellaan eräitä tavallisimpia geometrian käsitteitä avaruuteen $\R^n$ tuotuina ja
yleistettyinä:
\begin{itemize}
\item[---] \kor{Piste} on alkio $\ma \in \R^n$. \index{piste ($\R^n$:n)}
\item[---] \kor{Suora} on pistejoukko \index{suora}
           $S = \{\mx \in \R^n \mid \mx = \mx_0 + t\mv{v},\ t \in \R\}$, missä \\ 
           $\mx_0,\mv{v} \in \R^n$ ja $\mv{v} \neq \mv{0}$.
\item[---] \kor{Hypertaso} (\kor{taso} tapauksessa $n=3$) on pistejoukko \\
           $T = \{\mx \in \R \mid \mx = \mx_0 + \sum_{i=1}^{n-1} t_i \mv{v}_i,\ \mv{t} 
              = (t_i) \in \R^{n-1}\}$, missä \\
           $\mx_0,\mv{v}_i \in \R^n$ ja $\{\mv{v}_1, \ldots \mv{v}_{n-1}\}$ on lineaarisesti 
           riippumaton. \index{taso!a@hypertaso, $m$-taso} \index{hypertaso}
\item[---] \kor{$m$-taso} ($m$-ulotteinen taso) on pistejoukko \\
           $T = \{\mx \in \R \mid \mx = \mx_0 + \sum_{i=1}^{m} t_i \mv{v}_i,\ \mv{t} 
              = (t_i) \in \R^m\}$, missä \\
           $2 \le m \le n-1,\ \mx_0,\mv{v}_i \in \R^n$ ja $\{\mv{v}_1, \ldots \mv{v}_m\}$ on 
           lineaarisesti riippumaton.
\item[---] \kor{Jana}  on pistejoukko 
           $S = \{\mx \in \R^n \mid \mx 
              = \lambda\ma_1 + (1-\lambda)\ma_2,\ \lambda \in [0,1]\}$, \\
           missä $\ma_1,\ma_2 \in \R^n$ ja $\ma_1 \neq \ma_2$. \index{jana ($\R^n$:n)}
\item[---] \kor{Kolmio} on pistejoukko \\
           $K = \{\mx \in \R^n \mid \mx = \sum_{i=1}^3 \lambda_i \ma_i,\ 
                             \lambda_i \in [0,1],\ \ \sum_{i=1}^3 \lambda_i = 1\}$, missä \\
           $\ma_1,\ma_2,\ma_3 \in \R^n$ ja $\{\ma_2-\ma_1,\ma_3-\ma_1\}$ on lineaarisesti 
           riippumaton. \index{kolmio ($\R^n$:n)}
\item[---] \kor{$m$-simpleksi} ($2 \le m \le n+1$) on pistejoukko \\
           $K = \{\mx \in \R^n \mid \mx 
              = \sum_{i=1}^m \lambda_i \ma_i,\ \lambda_i \in [0,1],\ \ 
                \sum_{i=1}^m \lambda_i = 1\}$, missä \\
           $\ma_i \in \R^n$ ja $\{\ma_2-\ma_1, \ldots, \ma_m-\ma_1\}$ on lineaarisesti 
           riippumaton. \index{simpleksi}
\item[---] \kor{$m$-suuntaissärmiö} ($2 \le m \le n$) on pistejoukko \\
           $K = \{\mx \in \R^n \mid \mx 
              = \mx_0+\sum_{i=1}^m \lambda_i \ma_i,\ \lambda_i \in [0,1]\}$, missä 
           $\mx_0,\ma_i\in\R^n$ ja $\{\ma_1,\ldots,\ma_m\}$ on lineaarisesti riippumaton
           \index{suuntaissärmiö ($\R^n$:n)}
\item[---] \kor{Pallo (kuula)} on pistejoukko 
           $B(\mx_0,R) = \{\mx \in \R^n \mid \abs{\mx-\mx_0} \le R\}$, missä \\
           $\mx_0 \in \R^n$ ja $R>0$. \index{kuula}
\item[---] \kor{Pallopinta} on pistejoukko $K = \{\mx \in \R^n \mid \abs{\mx-\mx_0} = R\}$, 
           missä \\
           $\mx_0 \in \R^n$ ja $R>0$. \index{pallo(pinta)}
\item[---] \kor{$n$-suorakulmio} (koordinaatiakselien mukaan suunnattu) on pistejoukko \\
           $K = \{\mx = (x_i) \in \R^n \mid x_i \in [a_i,b_i],\ i = 1 \ldots n\}$, missä \\
           $a_i,b_i \in \R,\ a_i < b_i,\ i = 1 \ldots n$. \index{n@$n$-suorakulmio}
\end{itemize}
Määritelmien  mukaisesti hypertaso = $(n-1)$--ulotteinen taso, $2$-simpleksi = jana, 
$3$-simpleksi = kolmio ja $\R^3$:n $4$-simpleksi = tetraedri. Hypertason ja $m$-tason 
määritelmissä sanotaan vektoreita $\mv{v}_i$ ko.\ tason
\index{suuntavektori}%
\kor{suuntavektoreiksi}. Simpleksin 
\index{kzy@kärkipiste (simpleksin)}%
määritelmässä pisteet $\ma_i$ ovat simpleksin \kor{kärkipisteet} (engl.\ vertex). Jos
$n$-suorakulmiossa on $b_i-a_i = a,\ i = 1 \ldots n$, niin kyseessä on 
\index{n@$n$-kuutio}%
\kor{$n$-kuutio} (sivun pituus $\,a$).

Ym.\ määritelmässä on hypertaso ja $m$-taso esitetty
\index{parametri(sointi)!e@hypertason, $m$-tason}%
\kor{parametrisessa} muodossa suuntavektoreittensa $\mv{v}_i$ avulla. Hypertasolle voidaan myös
kirjoittaa \kor{hypertason yhtälö} muodossa
\[ 
T:\ \ \scp{\mx-\mx_0}{\mv{n}} = \mv{n}^T(\mx-\mx_0) = 0, 
\]
missä vektori $\mv{n}$ on $T$:n
\index{normaali(vektori)!b@tason, hypertason}%
\kor{normaali}, eli kohtisuorassa vektoreita $\mv{v}_i$ vastaan.
Tällainen $\mv{n}$ on löydettävissä esim.\ ortogonaaliprojektion avulla, vrt.\ Esimerkki 
\ref{projektioesimerkki} edellisessä luvussa. Yleisemmin voidaan $m$-taso määritellä 
yhtälöryhmällä
\[ 
T:\ \ \scp{\mx-\mx_0}{\mv{n}_i} = 0, \quad i = 1 \ldots n-m, 
\]
missä vektorit $\mv{n}_i$ ovat keskenään lineaarisesti riippumattomia $T$:n normaalivektoreita.
Hypertasoihin ja $m$-tasoihin liittyviä geometrisia ongelmia voidaan tyypillisesti ratkaista 
matriisialgebran avulla.
\begin{Exa} Millä ehdolla hypertasot
\[ 
T_i:\ \ \scp{\mx-\ma_i}{\mv{n}_i} = 0, \quad i = 1, \ldots, n 
\]
leikkaavat täsmälleen yhdessä pisteessä? 
\end{Exa}
\ratk Matriisialgebran mukaan tasojen $T_i$ yhtälöt muodostavat lineaarisen yhtälöryhmän 
$\mA\mx = \mb$, missä \mA:n $i$:s rivi $= \mv{n}_i^T$ ja $[\mb]_i = \scp{\ma_i}{\mv{n}_i}$. 
Siis vastaus on: Täsmälleen sillä ehdolla, että $\mA$ on säännöllinen eli että normaalivektorit
$\mv{n}_i$ muodostavat lineaarisesti riippumattoman systeemin ($\R^n$:n kannan). \loppu  

\subsection{*Lineaarinen optimointi $\R^n$:ssä}
\index{lineaarinen optimointi|vahv} \index{optimointi!a@lineaarinen optimointi|vahv}

Reaaliarvoisen $\R^n$:n lineaarikuvauksen $L:\ \R^n \kohti \R$ yleinen muoto on
\[ 
L\mx = \ma^T \mx,
\]
missä $\ma \in \R^n$ (oletetaan jatkossa: $\ma\neq\mv{0}$). Tällaisen funktion maksimi- tai 
minimiarvon etsimistä annetussa joukossa $K \subset \R^n$ sanotaan \kor{lineaarisen optimoinnin}
tehtäväksi. Jos $T$ on hypertaso, jonka yhtälö on
\[ 
T:\ \ \scp{\ma}{\mx-\mx_0}=\ma^T(\mx-\mx_0)= 0, 
\]
niin $L\mx = \ma^T \mx_0=$ vakio, kun $\mx \in T$, eli tällaiset hypertasot ovat $L$:n 
\index{tasa-arvopinta}%
\kor{tasa-arvopintoja}. Toisaalta jos tarkastellaan $L$:n arvoja pisteen $\mv{c}$ kautta 
kulkevalla, tasa-arvopintoja vastaan kohtisuoralla suoralla
\[ 
S:\ \ \mx = \mv{c} + t \ma,\ \ t \in \R, 
\]
niin nähdään, että tällä suoralla on
\[
L\mx = L(\mv{c}+t\ma) = \alpha t+\beta, \quad \alpha = \abs{\ma}^2 > 0,\ \beta = \ma^T\mv{c}.
\]
Jos nyt $L$:n tasa-arvopintojen yhtälössä valitaan $\mx_0 = \mv{c}+t\ma \in S$, ja merkitään 
tätä tasa-arvopintaa (hypertasoa) $T_t$:llä, niin parametria $t$ vaihtelemalla tulee koko
avaruus $\R^n$ 'pyyhityksi' hypertasoilla $T_t$. Tällöin nähdään heti, että jos $L$:llä on 
pistejoukossa $K$ maksimi- tai minimiarvo, niin tämän on oltava vastaavasti pienin\,/\,suurin 
luku reaalilukujoukossa
\[
A = \{t\in\R \mid T_t \cap K \neq \emptyset\}.
\]
Optimointiongelma on näin 'ratkaistu' geometrisesti. --- Idean soveltaminen käytäntöön kylläkin
edellyttää, että joukko $A$ on määrättävissä. Seuraavassa esimerkissä $K$ on geometrisesti 
yksinkertainen ja $n$ kohtuullinen, jolloin ongelma ratkeaa 'käsipelillä'. 
\begin{Exa} Määritä funktion $f(x_1,x_2,x_3,x_4) = 2x_1-4x_2+5x_3-6x_4$ pienin ja suurin arvo
$\R^4$:n pallossa
\[
K= \{\mx\in\R^4 \mid \abs{\mx-\mx_0} \le R\}, \quad 
               \text{missä}\,\ \mx_0=(1,-1,3,2)\,\ \text{ja}\,\ R=3. 
\]
\end{Exa}
\ratk Tässä on $f(\mx)=L\mx=\ma^T\mx,\ \ma=[2,-4,5,-6]^T,\ \mx=[x_1,x_2,x_3,x_4]^T$. Jos $T_t$
on $\R^4$:n hypertaso, joka kulkee pisteen $P_t \vastaa \mx_0+t\ma$ kautta, niin pätee
\[
T_t \cap K \neq \emptyset \qekv \abs{t\ma}=\abs{t}\abs{\ma} 
                                          \le 3 \qekv \abs{t}\ 
                                          \le\ \frac{3}{\sqrt{2^2+4^2+5^2+6^2}}\ 
                                          =\ \frac{1}{3}\,.
\]
Siis $\{t\in\R \mid T_t \cap K \neq \emptyset\} = [-1/3,\,1/3]$. Koska hypertasolla $T_t$ on
\[
f(\mx)\ =\ \ma^T(\mx_0+t\ma)\ =\ \ma^T\mx_0+\abs{\ma}^2 t\ =\ 9+81 t, \quad \mx \in T_t\,,
\]
niin
\[
f_{min}=9+81\cdot\left(-\frac{1}{3}\right)=\underline{\underline{-18}}, \quad 
f_{max}=9+81\cdot\frac{1}{3}=\underline{\underline{36}}. \loppu
\]

Sovelluksissa hyvin yleinen on lineaarisen optimoinnin ongelma, jossa $K$ on määritelty 
\index{lineaarinen epäyhtälö}%
\kor{lineaaristen epäyhtälöiden} avulla muodossa
\[
K = \{\mx\in\R^n \mid \ma_i^T\mx \le c_i\,,\ i=1\,\ldots\,m\},
\]
missä $\ma_i\in\R^n,\ \ma_i\neq\mv{0}$, ja $c_i\in\R$. Joukkoa $K$ voi luonnehtia geometrisesti
\index{monitahokas ($\R^n$:n)}%
\kor{monitahokkaaksi}, jota hypertasot $T_i\,:\ \ma_i^T\mx=c_i$ rajoittavat (yleensä on $m>n$).
Tällaista ongelmaa on perinteisesti kutsuttu
\index{lineaarinen optimointi!lineaarinen ohjelmointi}%
\kor{lineaarisen ohjelmoinnin} (engl.\ linear programming) ongelmaksi. Ongelman sekä suoraan
(periaatteessa tarkkaan) että likimääräiseen iteratiiviseen ratkaisuun on kehitetty algoritmeja.

\Harj
\begin{enumerate}

\item
Millaiseksi muuntuu \, a) suorakulmio, jonka kärjet ovat pisteissä $(1,1)$, $(3,1)$, $(3,2)$ ja
$(1,2)$, \ b) yksikkympyrä $S:\,x^2+y^2=1$ affiinikuvauksessa
\[
\mf(x,y)=\begin{bmatrix} 1&2\\3&4 \end{bmatrix} \begin{bmatrix} x\\y \end{bmatrix} +
         \begin{bmatrix} 5\\6 \end{bmatrix} \ ?
\]
Piirrä kuva! Määritä myös käyrän $S'=\mf(S)$ yhtälö.

\item
Esitä affiinikuvuaksena: $\mf:\,\R^2\kohti\R^2$ tai $\mf:\,\R^3\kohti\R^3$\,: 
\vspace{1mm}\newline
a) \ Kohtisuora projektio tasossa suoralle $x+2y+1=0$ \newline
b) \ Projektio suoralle $x-3y-2=0$ suoran $y=2x$ suunnassa \newline
c) \ Kierto tasossa pisteen $(2,3)$ ympäri $60\aste$ vastapäivään \newline
d) \ Projektio tasolle $T:\,x+2z-3y+3=0$ vektorin $\vec i+\vec j-\vec k$ suunnassa \newline
e) \ Peilaus tason $T:\,3x-2y+z-5=0$ suhteen \newline
f) \ Peilaus avaruussuoran $S:\,\vec r=(1+t)\vec i+(t-2)\vec j+(2t-3)\vec k$ suhteen

\item
Taso $T$ kulkee origon kautta ja sen suuntavektoreita ovat $\vec a_1=\vec i+\vec j+\vec k$ ja
$\vec a_2=\vec j-2\vec k$. Suora $S$ kulkee pisteen $(0,1,2)$ kautta ja sen suuntavektori on
$\vec a_3=-\vec i+\vec j+\vec k$. \vspace{1mm}\newline
a) Olkoon $\my=[y_1,y_2,y_3]^T$ vektorin $x_i\vec i+x_2\vec j+x_3\vec k$ koordinaatit kannassa
$\{\vec a_1,\vec a_2,\vec a_3\}$. Määritä lineaarikuvaus $\mx\map\my$, ts.\ matriisi $\mC$
siten, että $\my=\mC\mx$. \newline
b) Käyttäen a-kohdan tulosta ja matriisialgebraa määritä affiinikuvaus $\mf(\mx)=\mA\mx+\mb$,
joka suorittaa pisteen $P=(x_1,x_2,x_3)$ suuntaisprojisoinnin suoralle $S$ tason $T$
suunnassa ($\mf:\,\R^3\kohti\R^3$). \newline
c) Määrittele affiinikuvaus tyyppiä $\mf:\,\R^3\kohti\R^2$, joka suorittaa pisteen
$P=(x_1,x_2,x_3)$ projisoinnin suoran $S$ sunnassa tason $T$ pisteeksi $Q$, joka on ilmoitettu
koordinaatteina kannassa $\{\ma_1,\ma_2\}$. 

\item
Vektori $\vec b$ saadaan vektorista $\vec a=2\vec i+\vec j+2\vec k$ kiertämällä $\vec a$:ta
ensin $x$-akselin ympäri kulma $30\aste$ ja näin saatua vektoria edelleen $y$-akselin ympäri
$60\aste$ (kierrot positiivisten koordinaattiakselien suunnasta katsoen vastapäivään). Määritä
kummankin kierto-operaation (lineaarikuvauksia) matriisit, näiden avulla koko operaation
matriisi $\mA$ ja $\mA$:n avulla vektori $\vec b$. Mitä vektoria mainituilla tavoilla
kierrettäessä saadaan lopputulokseksi $\vec b=\vec k$\,?

\item
Osoita lineaarikuvaus
\[
\mf(x,y)=\frac{1}{25}\begin{rmatrix} 7&-24\\-24&-7 \end{rmatrix} 
                     \begin{bmatrix} x\\y \end{bmatrix}
\]
peilaukseksi erään suoran suhteen ja määritä suoran yhtälö.

\item
Kolmion $A$ kärjet ovat pisteissä $(0,0)$, $(1,0)$ ja $(0,1)$ ja kolmion $B$ kärjet ovat 
pisteissä $(-1,1)$, $(-2,2)$ ja $(-3,0)$. Määritä kaikki kolmiot $C$, joille pätee $C=\mv{f}(B)$
ja $B=\mv{f}(A)$, missä $\mv{f}$ on tason affiinikuvaus.

\item
Seuraavat lineaarikuvaukset määrittelevät yhdensuuntaisprojektion origon kautta kulkevalle
tasolle $T$ suoran $S$ sunnassa. Määritä $T$:n yhtälö ja $S$:n suuntavektori.
\begin{align*}
&\text{a)}\ \ \begin{bmatrix} x'\\y'\\z' \end{bmatrix} =
              \frac{1}{5}\begin{rmatrix} 4&1&-3\\1&4&3\\-1&1&2 \end{rmatrix}
                         \begin{bmatrix} x\\y\\z \end{bmatrix} \\[3mm]
&\text{b)}\ \ \begin{bmatrix} x'\\y'\\z' \end{bmatrix} =
              \frac{1}{10}\begin{rmatrix} 7&-6&-9\\-2&6&-6\\-1&-2&7 \end{rmatrix}
                          \begin{bmatrix} x\\y\\z \end{bmatrix} 
\end{align*}

\item 
Näytä, että jos affiinimuunnos $\mf:\,\R^n\kohti\R^n$, missä $n=2$ tai $n=3$, on kääntyvä, niin
$\mf$  \vspace{1mm}\newline
a) kuvaa suoran suoraksi ($n=2,3$) ja tason tasoksi ($n=3$) \newline
b) kuvaa janan janaksi ja kolmion kolmioksi \newline
c) kuvaa tetraedrin tetraedriksi ($n=3$), \newline
d) säilyttää suorien ja tasojen yhdensuuntaisuuden, \newline
e) säilyttää kahden yhdensuuntaisen janan pituuksien suhteen. \newline
f) Miten ominaisuus c) muuttuu, jos $\mf$ ei ole kääntyvä?

\item
Missä joukon $A\subset\R^4$ pisteessä funktio
\[
f(\mx)=x_1-2x_2-x_3+2x_4
\]
saavuttaa suurimman arvonsa, kun $A$ on \vspace{1mm}\newline
a) yksikkökkuutio: $A=\{\mx\in\R^4 \mid \abs{x_i} \le 1,\ i=1 \ldots 4\}$, \newline
b) yksikköpallo: $A=\{\mx\in\R^4 \mid x_1^2+\ldots+x_4^2 \le 1\}$\,?

\item (*)
Määritä affiinikuvaus $\mf:\,\R^3\kohti\R^3$, joka suorittaa kierron suoran
$S:\ x=1+6t,\ y=1-3t,\ z=-2t\,$ ympäri, suunnasta $\vec v=6\vec i-3\vec j-2\vec j$ katsoen
kulman $\varphi$ verran vastapäivään. \kor{Vihje}: Määritä $\mv{f}$ ensin koordinaatistossa,
jossa $S$ on koordinaattiakseli --- ks.\
Harj.teht.\,\ref{lineaarikuvaukset}:\,\ref{H-m-6: kiertoja}a!

\item (*)
Ratkaise lineaarisen optimoinnin tehtävä $f(x,y,z)=x+2y+3x=$max! ehdoilla
\[
\begin{cases}
x \ge 0,\ y \ge 0,\ z \ge 0, \\ 2x+y-2z \le 6, \\ -x+5y+z \le 8, \\ 4x+2y+7z \le 23.
\end{cases}
\]
\kor{Vihje}: Riittää tutkia monitahokkaan kärkipisteet!

\item (*) \index{zzb@\nim!Illaksi kotiin}
(Illaksi kotiin) Liikemies näkee lentokoneesta kotitalonsa suunnassa
$-2\vec i-3\vec j-\vec k$. Talon perusosa on suorakulmainen särmiö, jonka korkeus $=1$ ja
jonka maan pinnalla olevat nurkat ovat $xy$-tason pisteissä $(0,0)$, $(3,0)$, $(3,2)$ ja
$(0,2)$. Talossa on symmetrinen harjakatto, harjan korkeus $=2$ ja suunta $=\vec i$. Räystäät
talon sivuilla ja päädyissä ulottuvat eäisyydelle $0.2$ seinistä. Määrittele affiinikuvaus
$\mf:\,\R^3\map\R^2$, joka muuntaa kolmiulotteisen todellisuuden liikemiehen näkemäksi
projektiokuvaksi (oleta luonteva pään asento). Piirrä kuva talosta sellaisena kuin liikemies
sen näkee.

\end{enumerate} %Affiinikuvaukset. Geometriset kuvaukset
\section[Lineaaristen yhtälöryhmien sovellusesimerkkejä]{*Lineaaristen yhtälöryhmien 
perinteisiä \\ sovellusesimerkkejä} 
\label{yhtälöryhmät} 
\sectionmark{Perinteiset lineaariset yhtälöryhmät}
\alku

Tietokoneiden avulla suoritettavalle laajamittaiselle tieteellis-tekniselle laskennalle 
--- esimerkiksi numeeriselle sään ennustamiselle --- on tyypillistä, että huomattava osa 
tietokoneen käyttämästä laskenta-ajasta kuluu suurten lineaaristen yhtälöryhmien ratkaisuun. 
Yhtälöryhmiin päädytään, kun tarkastelun kohteena olevat luonnonlait (säätä ennustettaessa 
virtausmekaniikan perusyhtälöt) \kor{diskretoidaan}, eli muunnetaan numeerisen laskennan 
kannalta soveliaaseen likimääräiseen muotoon. Laskemisen palauttaminen nimenomaan
\pain{lineaarisiksi} yhtälöryhmiksi on malleissa tavallista senkin vuoksi, että lähinnä
vain tämän tyyppisten perustehtävien numeerisen ratkaisemisen tietokone viime kädessä 'osaa'.

Modernien tieteellis-teknisten laskentatehtävien ohella suuren lineaarisen yhtälöryhmän ongelma
on keskeinen myös monissa perinteisemmissä --- usein tietokonetta paljon vanhemmissa --- 
matemaattisissa malleissa, joita esiintyy erityisesti insinööritieteissä. Seuraavassa esitetään
kaksi tällaista klassista sovellusesimerkkiä, toinen sähkötekniikasta ja toinen rakenteiden 
mekaniikasta.

\subsection{Sähköpiiri: Vastusverkko}
\index{zza@\sov!Szyhkzzc@Sähköpiiri: Vastusverkko|vahv}

\begin{multicols}{2}
\parbox{2in}{Tarkastellaan klassista sähköpiiriä, joka on pelkkien \pain{vastusten} muodostama
verkko. Olkoot verkon solmupisteet $P_i,\ i = 1 \ldots n$, ja olkoot pisteet $P_i,\,P_j$ 
yhdistetty vastuksella $R_{ij}$ (jollei suoraa yhteyttä ole, asetetaan $1/R_{ij}\,=\,0$
yhtälöryhmässä \eqref{m-8.3} jäljempänä). Oheisessa kuvassa on yleinen neljän solmun verkko.}
\begin{figure}[H]
\begin{center}
\epsfig{file=kuvat/kuvaM-1.ps}
\end{center}
\end{figure}
\end{multicols}
Jos merkitään $I_{ij} =$ \pain{virta} vastuksen $R_{ij}$ läpi, positiivisena $P_i$:stä poispäin
(vrt.\ kuvio), niin 
\index{Kirchhoffin laki}%
\pain{Kirchhoffin} \pain{lain} (\hist{G.R. Kirchhoff}, 1845) mukaan
\begin{equation} \label{m-8.1}
\sum_{j=1}^n I_{ij}\ =\ I_i, \quad i = 1 \ldots n,
\end{equation}
missä $I_i =$ solmuun $P_i$ ulkoapäin syötetty virta. Toisaalta
\index{Ohmin laki}%
\pain{Ohmin} \pain{lain} (\hist{G.S. Ohm}, 1826) mukaan
\begin{equation} \label{m-8.2}
R_{ij} I_{ij}\ =\ V_i - V_j, \quad i,j = 1 \ldots n,
\end{equation}
missä $V_i =$ j\pain{ännite} solmussa $P_i$. Ohmin lakia \eqref{m-8.2} käyttäen voidaan virrat
$I_{ij}$ yhtälöryhmässä \eqref{m-8.1} lausua jännitteiden avulla, jolloin tuloksena on systeemi
\begin{equation} \label{m-8.3}
\sum_{j=1}^n a_{ij} V_j\ =\ I_i, \quad i = 1 \ldots n,
\end{equation}
missä
\[
a_{ii}\ =\ \sum_{j=1}^n (1/R_{ij}), \quad a_{ij}\ =\ -1/R_{ij},\,\ i \neq j.
\]
Tämä on lineaarinen yhtälöryhmä kokoa $n \times n$, missä $x_i = V_i$, $b_i = I_i$. 
\begin{Exa} Jos nelisolmuisessa verkossa (vrt.\ kuvio edellä) on 
$R_{ij} = R,\ i,j = 1 \ldots 4,\ i \neq j$, niin jännitteille
$V_i$ saadaan yhtälöryhmä
\[ \left\{ \begin{array}{rrrrrrrrl} 3V_1&-& V_2&-& V_3&-& V_4&\ \ =\ \ &RI_1\\
                                    -V_1&+&3V_2&-& V_3&-& V_4&\ \ =\ \ &RI_2\\
                                    -V_1&-& V_2&+&3V_3&-& V_4&\ \ =\ \ &RI_3\\
                                    -V_1&-& V_2&-& V_3&+&3V_4&\ \ =\ \ &RI_4
\end{array} \right. \]
Tämä on singulaarinen systeemi: Jos esim.\ $I_1 = I \neq 0$ ja $I_2 = I_3 = I_4 = 0$, niin 
ratkaisua ei ole. Jos myös $I_1 = 0$, niin $V_i = V,\ i = 1 \ldots 4$ on ratkaisu millä tahansa
$V \in \R$, ts.\ ratkaisuja on äärettömän monta. \loppu 
\end{Exa}
 
Esimerkin singulaarisuusongelma on yhtälöryhmän \eqref{m-8.3} ominaisuus yleisemminkin.
Nimittän kertoimien $a_{ij}$ ym.\ lausekkeista nähdään, että
\[
\sum_{j=1}^n a_{kj}\ =\ \sum_{i=1}^n a_{ik}\ =\ 0, \quad k = 1 \ldots n.
\]
Tästä on kaksi seuraamusta: Ensinnäkin jos yhtälöryhmän \eqref{m-8.3} yhtälöt lasketaan 
puolittain yhteen saadaan
\[
(\sum_{i=1}^n a_{i1})\,V_1 + \ldots + (\sum_{i=1}^n a_{in})\,V_n\ =\ \sum_{i=1}^n I_i,
\]
eli
\begin{equation} \label{m-8.4}
0\ =\ \sum_{i=1}^n I_i.
\end{equation}
Toiseksi nähdään, että jos $\seq{V_i}$ on systeemin \eqref{m-8.4} ratkaisu, niin myös 
$\seq{V_i + V}$ on ratkaisu millä tahansa $V \in \R$, sillä
\[
\sum_{j=1}^n a_{ij} (V_j + V)\ =\ \sum_{j=1}^n a_{ij} V_j\ +\ (\sum_{j=1}^n a_{ij})\,V\ 
                               =\ I_i + 0\ =\ I_i, \quad i = 1 \ldots n.
\]
On siis päätelty: (a) Jos yhtälöryhmällä \eqref{m-8.3} on ratkaisu, niin ehto \eqref{m-8.4} 
toteutuu, ts.\ tämä on \pain{välttämätön} \pain{ehto} ratkeavuudelle. (b) Jos yhtälöryhmä 
\eqref{m-8.3} on ratkeava, niin ratkaisuja on äärettömän monta. Yhtälöryhmä \eqref{m-8.3} on 
siis yleisesti singulaarinen.

Singulaarisuusongelma on merkki siitä, että matemaattista mallia ei ole ajateltu 
(fysikaaliselta kannalta) aivan loppuun asti. --- Ongelmaan onkin yksinkertainen 
'sähkömiehen ratkaisu': \pain{Maadoitetaan} verkon yksi solmu $P_k$. Matemaattisessa mallissa
tämä merkitsee solmussa $P_k$ asetettavia ehtoja
\begin{equation} \label{m-8.5}
V_k\ =\ V_{ref}, \quad I_k\ =\ - \sum_{i \neq k} I_i.
\end{equation}
Ensimmäinen ehto asettaa jännitteelle $V_k$ jonkin valinnaisen referenssiarvon $V_{ref}$, esim.\
$V_{ref} = 0$. Tämä on luvallista, koska fysikaalisesti merkitseviä ovat vain jännite-erot. 
Toisessa ehdossa voidaan $I_k$ tulkita \pain{maadoitusvirraksi}, joka määräytyy fysikaalisesti
siten, että syöttövirtojen tasapainoehto \eqref{m-8.4} (myös fysikaalinen ehto!) toteutuu. Ehto
$V_k = V_{ref}$ poistaa $V_k$:n tuntemattomien joukosta, jolloin yhtälöryhmä \eqref{m-8.3} 
voidaan kirjoittaa muotoon
\[
\sum_{j \neq k} a_{ij} V_j\ =\ I_i - a_{ik} V_{ref}\ = J_i, \quad i = 1 \ldots n.
\]
Tästä voidaan edelleen $k$:s yhtälö poistaa tarpeettomana. Nimittäin koska
\[
\sum_{i=1}^n a_{ij} = 0 \qimpl a_{kj} = - \sum_{i \neq k} a_{ij}, \qquad j = 1 \ldots n,
\]
niin olettamalla yhtälöryhmän \eqref{m-8.3} muut yhtälöt ($i \neq k$) sekä ehdon \eqref{m-8.5}
jälkimmäinen osa toteutuviksi päätellään
\begin{align*}
\sum_{j=1}^n a_{kj} V_j\ &=\ - \sum_{j=1}^n (\sum_{i \neq k} a_{ij})\,V_j \\
                         &=\ - \sum_{i \neq k} (\sum_{j=1}^n a_{ij} V_j)\ 
                          =\ - \sum_{i \neq k} I_i\ =\ I_k.
\end{align*}
(Tässä tehty summeerausjärjestyksen vaihto perustuu yhteenlaskun vaihdantalakiin.) Tuloksen 
perusteella $k$:s yhtälö on muiden yhtälöiden ja lisäehdon \eqref{m-8.5} seuraus, siis 
tarpeeton.

Em. toimenpiteiden jälkeen on sähköpiirin matemaattinen malli supistunut lineaariseksi 
yhtälöryhmäksi kokoa $(n-1) \times (n-1)$. Tapauksessa $V_{ref} = 0$ redusoitu malli saadaan
yksinkertaisesti poistamalla yhtälöryhmän taulukkomuodosta taulukon $k$:s sarake
(vastaten ehtoa $V_k = 0$) ja $k$:s rivi ($k$:nnen yhtälön poisto). Malli saa tällöin muodon 
\begin{equation} \label{m-8.6}
\sum_{j \neq k} a_{ij} V_j\ =\ I_i, \quad i = 1 \ldots n,\ \ i \neq k.
\end{equation}
Osoittautuu (tarkemmat perustelut sivuutetaan), että yhtälöryhmä \eqref{m-8.6} on aina 
säännöllinen.
\jatko \begin{Exa} (jatko) \ Jos asetetaan solmussa $P_4$ maadoitusehto $V_4 = 0$, niin 
tapauksessa $I_1 = I,\ I_2 = I_3 = 0$ on redusoidulla systeemillä
\[ \left\{ \begin{array}{rrrrrrl}  3V_1&-& V_2&-& V_3&\ \ =\ \ &RI\\
                                  -V_1&+&3V_2&-& V_3&\ \ =\ \ & 0\\
                                  -V_1&-& V_2&+&3V_3&\ \ =\ \ &0
\end{array} \right. \] 
(yksikäsitteinen) ratkaisu
\[
V_1 = \frac{1}{2}\,RI, \quad V_2 = \frac{1}{4}\,RI, \quad V_3 = \frac{1}{4}\,RI.
\]
Ratkaisun perusteella solmujen $P_1$ ja $P_4$ välinen efektiivinen vastus (kokonaisvastus) on
\[
R_{eff}\ =\ \frac{V_1 - V_4}{I}\ =\ \frac{1}{2}\,R. \quad \loppu
\] \end{Exa}

\subsection{Ristikkorakenne}
\index{zza@\sov!Ristikkorakenne|vahv}

Tarkastellaan kimmoisten \pain{sauvo}j\pain{en} muodostamaa ristikkorakennetta, jossa sauvat on
liitetty toisiinsa solmupisteissä $P_i,\ i = 1 \ldots p$. Oletetaan, että liitokset välittävät
sauvasta toiseen vain vetoa/puristusta, eivät vääntöä (esim.\ löysä pulttiliitos). Sauvassa, 
jonka päätepisteet ovat $P_i,\,P_j$, olkoon sauvaa venyttävä \pain{voima}
\[
\vec{F}_{ij}\ =\ F_{ij} \vec{t}_{ij},
\]
missä $\vec{t}_{ij}$ on vektorin $\Vect{P_i P}_j$ suuntainen yksikkövektori 
(kyseessä on puristus, jos $F_{ij} < 0$). Tällöin \pain{voimatasa}p\pain{aino} pisteessä $P_i$
edellyttää, että
\begin{equation} \label{m-8.7}
\sum_{j \in \Lambda_i} \vec{F}_{ij}\,+\,\vec{G}_i\,=\,\vec{0}, \quad i = 1 \ldots p,
\end{equation}
missä $\{\,P_j \mid j \in \Lambda_i\,\}$ on niiden solmupisteiden joukko, joihin $P_i$ on 
yhdistetty sauvalla, ja $\vec{G}_i =$ ulkoinen kuorma solmussa $P_i$.
\begin{figure}[H]
\begin{center}
\epsfig{file=kuvat/kuvaM-2.ps}
\end{center}
\end{figure}
Kuvassa kuormitustilaa on purettu kahdeksi nk.\ \pain{va}p\pain{aaka}pp\pain{alekuvioksi}, 
joissa kummasakin vallitsee voimatasapaino. --- Huomattakoon, että liitoksia koskeva 
tasapainoehto \eqref{m-8.7} on hyvin samankaltainen kuin sähköpiirejä koskeva Kirchhoffin laki
\eqref{m-8.1}, joka myös on tasapainoehto (virtoja koskeva).

Kun yhtälöryhmässä \eqref{m-8.7} vektoriyhtälöt puretaan koordinaattimuotoon karteesisessa 
koordinaatistossa, niin skalaaristen yhtälöiden kokonaismääräksi tulee $2p$ tai $3p$ riippuen
siitä, onko kyseessä taso- vai avaruusristikkorakenne. Tuntemattomien $F_{ij}$ lukumäärä on 
luonnollisesti sama kuin sauvojen lukumäärä rakenteessa. Yhtälöryhmä \eqref{m-8.7} voi olla 
suoraan yksikäsitteisesti ratkeava, jolloin sanotaan, että rakenne on 
\pain{staattisesti} \pain{määrä}y\pain{t}y\pain{vä}. Tavallisempi tilanne on kuitenkin, että 
systeemi \eqref{m-8.7} on alimääräytyvä, eli \pain{staattisesti} \pain{määräämätön}. Tällöin 
tarvitaan tasapainolakien lisäksi toinen fysiikan laki,
\index{Hooken laki}%
\pain{Hooken} \pain{laki} 
(\hist{R. Hooke}, 1678). Hooken lain mukaan jokainen sauva käyttäytyy ristikkoa kuormitettaessa
kuten jousi, ts.\ voima $F_{ij}$ on suoraan verrannollinen sauvan \pain{ven}y\pain{mään}.

\mbox{\parbox{2in}{Oletetaan, että ristikkoa kuormitettaessa solmu $P_i$ 
(paikkavektori $\vec{r}_i$) siirtyy paikkaan $Q_i$, ja merkitään \pain{siirt}y\pain{mää} 
$\vec{u}_i = \Vect{P_i Q}_i$. Olettaen, että siirtymät ovat pieniä verrattuna sauvan pituuteen, 
voidaan kunkin sauvan venymä (negatiivinen venymä tarkoittaa puristumaa) laskea likimäärin 
kaavasta (vrt.\ kuvio)}}
\mbox{\parbox{4in}{\epsfig{file=kuvat/kuvaM-3.ps}}}
\[
\abs{(\vec{r}_i + \vec{u}_i) - (\vec{r}_j + \vec{u}_j)}\ - \ \abs{\vec{r}_i - \vec{r}_j}\ 
                                     \approx\ (\vec{u}_j - \vec{u}_i) \cdot \vec{t}_{ij}.
\]

(Approksimaatiossa oletetaan, että siirtymien $\vec{u}_i$ ja $\vec{u}_j$ sauvaa vastaan 
kohtisuorat komponentit eivät vaikuta sauvan pituuteen.) Hooken lain mukaan venymä on suoraan
verrannollinen voimaan, eli
\begin{equation} \label{m-8.8}
F_{ij}\ =\ K_{ij} (\vec{u}_j - \vec{u}_i) \cdot \vec{t}_{ij},
\end{equation}
missä $K_{ij}$ on sauvalle ominainen kerroin (jousivakio). --- Huomattakon jälleen analogia 
sähköpiireihin: Siirtymät $\vec{u}_i$ vastaavat jännitteitä, voimat $F_{ij}$ virtoja, ja 
Hooken laki \eqref{m-8.8} vertautuu vastuksia koskevaan Ohmin lakiin. Jos sauvat ovat 
tasapaksuisia ja tehty homogeenisesta materiaalista, on jousivakio $K_{ij}$ suoraan 
verrannollinen sauvan poikkipinta-alaan ja kääntäen verrannollinen sauvan pituuteen. Jousivakio
riippuu myös materiaalin laadusta nk.\ \pain{kimmokertoimen} (materiaalivakio) kautta. 

Hooken lakia \eqref{m-8.8} käyttäen tulee yhtälöryhmästä \eqref{m-8.7} lineaarinen, 
tuntemattomina siirtymävektorien $\vec{u}_i$ koordinaatit. Tuntemattomien määrä on siis joko 
$2p$ (tasoristikko) tai $3p$ (avaruusristikko). Jotta yhtälöryhmästä saataisiin säännöllinen, on
ristikko vielä 'maadoitettava', eli tuettava riittävästi. Tuentaehdot voivat olla esim.\ muotoa 
$\vec{u}_k = \vec{0}$, jolloin solmun $P_k$ siirtyminen rakennetta kuormitettaessa on estetty.
Tällaisessa solmussa ei voimatasapainoehtoja tarvitse kirjoittaa, sillä tasapainosta huolehtivat
(tuntemattomat) tukivoimat, jotka vastaavat sähköpiirin maadoitusvirtoja. Tuentaehdoilla on 
yleisesti estettävä sellaiset siirtymätilat, jotka eivät aiheuta rakenteessa mitään kuormituksia
($F_{ij} = 0\ \forall\ i,j$). Erityisesti koko rakenteen liikkuminen jäykkänä kappaleena on 
matemaattisessa mallissa estettävä. Kun tällaiset riittävät (fysikaalisesti usein ilmeiset)
lisäehdot on asetettu, tulee yhtälöryhmästä \eqref{m-8.7}--\eqref{m-8.8} säännöllinen.
Siirtymät $\vec{u}_i$ voidaan tällöin ratkaista ensin, ja näiden avulla edelleen 
(fysikaalisesti ehkä kiinnostavammat) kuormitukset $F_{ij}$ Hooken laeista \eqref{m-8.8}.

\begin{Exa} \label{ristikko} \ \ 

\parbox{2in}{Oheisessa (taso)ristikossa on sauvoja kahta tyyppiä, jousivakiot $K_1$ ja $K_2$. 
Sauvoja on $7$ kpl ja yhtälöryhmässä \eqref{m-8.7} on $3 \times 2 = 6$ yhtälöä 
(tuetuissa solmuissa $A,\,B$ voimatasapainosta huolehtivat tukivoimat), joten rakenne on 
staattisesti määräämätön. Kun siirtymiä solmuissa $P_i$ merkitään}
\mbox{\parbox{4in}{\epsfig{file=kuvat/kuvaM-4.ps}}}
\[
\vec{u}_i\ =\ u_i \vec{i} + v_i \vec{j}, \quad i = 1,2,3,
\]
niin lyhennysmerkinnöin
\[
a   = 1 + 2 K_2 / K_1, \quad b = 2 K_2 / K_1, \quad g_1 = G_1 / K_2, \quad g_2 = - G_2 / K_2
\]
yhtälöryhmä \eqref{m-8.7}--\eqref{m-8.8} saa muodon
\[ \left\{ \begin{array}{rrrrrrrrrrrrl} 
au_1&-&v_1& &    & &    &-& u_3&+& v_3&\quad 
                   = \quad&0\\ u_1&-&av_1& &    &+&bv_2&-& u_3&+& v_3&\quad = \quad&g_1\\
    & &   & &au_2&+& v_2&-& u_3&-& v_3&\quad 
                   = \quad&0\\    &-&bv_1&+& u_2&+&av_2&-& u_3&-& v_3&\quad = \quad&g_2\\
 u_1&-&v_1&+& u_2&+& v_2&-&4u_3& &    &\quad 
                   = \quad&0\\-u_1&+& v_1&+& u_2&+& v_2& &    &-&4v_3&\quad = \quad&0
\end{array} \right. \]

Tämä on säännöllinen ryhmä, josta siirtymät $u_i,\,v_i$ ovat ratkaistavissa. Sauvoja venyttävät
tai puristavat voimat ovat tämän jälkeen laskettavissa Hooken \mbox{laeista \eqref{m-8.8}.} 
Esimerkiksi sauvaa $P_1 P_3$ venyttää (negatiivisena puristaa) voima
\[
F_{13}\ =\ K_2\,[(u_3 - u_1)\vec{i} + (v_3 - v_1)\vec{j}\,] 
                          \cdot \frac{1}{\sqrt{2}}\,(-\vec{i}+\vec{j}) 
        = \frac{K_2}{\sqrt{2}}\,(u_1 - v_1 - u_3 + v_3). \loppu
\] 
\end{Exa}

\Harj
\begin{enumerate}

\item
Vastuspiirin solmut ovat tasasivuisen kolmion $ABC$ kärjet ja kolmion keskipiste $D$. Sivuilla
$AB$, $BC$ ja $CA$ on vastukset $R_1=1$, $R_2=2$ ja $R_3=3$. Kolmion kärjet $A$, $B$ ja $C$
on yhdistetty keskipisteeseen vastuksilla $R_4=R_5=2$ ja $R_6=3$. Olettaen virtasyöttö $I$
solmuun $A$ ja maadoitus solmuun $C$, muodosta lineaarinen yhtälöryhmä solmujen $A,B,D$ 
jännitteille ja laske solmujen $A$ ja $C$ välinen vastus.

\item
Kuution kärjet $P_i,\ i=0 \ldots 7$, on numeroitu siten, että $P_0$ ja $P_7$ ovat vastakkaiset
kärjet kuution yhdellä sivulla. Kärjet ovat solmuja virtapiirissä, jonka johdot kulkevat
pitkin kuution särmiä ja jokaisella särmällä vastus $=R$. Solmujen jännitteistä ($x_i$)
tiedetään, että $x_0=E$ (jännitelähde) ja $x_7=0$. Muodosta lineaarinen yhtälöryhmä
jännitteille $x_i,\ i=1 \ldots 6$. Ratkaise ja laske solmujen $P_0$ ja $P_7$ välinen vastus.

\item (*) \index{zzb@\nim!Silta}
(Silta) Tason ristikkorakenteen solmut $P_1,\ldots,P_7$ ovat pisteissä $(0,0)$, $(1,1)$,
$(2,0)$, $(3,1)$, $(4,0)$, $(5,1)$ ja $(6,0)$. Solmut on yhdistetty sauvoilla siten, että
muodostuu viisi vierekkäistä tasakylkistä ja suorakulmaista kolmiota. Solmu $P_1$ on jäykästi
tuettu ja solmun $P_7$ siirtyminen suuntaan $\vec j$ on estetty, muita tukia ei ole. Solmuihin
$P_3$ ja $P_5$ vaikuttavat kuormat ovat $\vec F_3=-a G\vec j$ ja $\vec F_5=-bG\vec j$, missä
$a,b \ge 0$ ovat dimensiottomia vakioita. Muita kuormia ei ole. Merkitse sauvoja
venyttäviä (negatiivisena puristavia) voimia symboleilla $T_i=x_iG$ ($11$ sauvaa!) ja kirjoita
näiden avulla solmujen tasapainoyhtälöt. Ratkaise systeemi (mahdollista, koska rakenne on
staattisesti määräytyvä), ja vastaa ratkaisun perusteella kysymykseen: Jos jokainen sauva
kestää puristusta $G$:n verran ja vetoa rajattomasti, niin millaiset kuormat ovat mahdollisia
rakenteen romahtamatta? Piirrä kuva $ab$-tasoon!

\end{enumerate} %*Lineaaristen yhtälöryhmien perinteisiä sovellusesimerkkejä

\chapter[Usean muuttujan differentiaalilaskenta]{Usean muuttujan \\ 
differentiaalilaskenta}

Tässä luvussa tarkastelun kohteena ovat \kor{vektorimuuttujan}, eli useamman reaalimuuttujan
funktiot tyyppiä $f:\,\R^n\kohti\R$ ja vektorimuuttujan \kor{vektoriarvoiset} funktiot
tyyppiä $\mf:\,\R^n\kohti\R^m$. Edellisten erikoistapaukset, kahden ja kolmen
\mbox{reaalimuuttujan} funktiot, ovat ennestään tuttuja Luvusta
\ref{kahden ja kolmen muuttujan funktiot}. Myös vektoriarvoisia funktioita on tavattu jo
aiemmin, sillä sekä Luvussa \ref{parametriset käyrät} esitellyt parametriset käyrät ja
parametriset pinnat että Luvuissa \ref{lineaarikuvaukset}--\ref{affiinikuvaukset} käsitellyt
lineaari- ja affiinikuvaukset ovat tällaisten erikoistapauksia. Vektoriarvoisia funktioita ovat
myös tason ja avaruuden \kor{vektorikentät}, joissa vektori ymmärretään
geometris--fysikaalisena 'nuolivektorina'.

Luvussa \ref{usean muuttujan jatkuvuus} yleistetään Luvuista
\ref{jatkuvuuden käsite}--\ref{funktion raja-arvo} tutut jatkuvuuden ja raja-arvon käsitteet
useamman muuttujan funktioille. Tämän jälkeen luvun keskeisen sisällön muodostavat erilaiset
\pain{derivaatan} käsitteen laajennukset useamman muuttujan tilanteisiin ja näihin
laajennuksiin liittyvä laskutekniikka (differentiaalilaskenta) ja sovellukset. Yleisempinä
derivaatan käsitteinä esitellään tässä luvussa \kor{osittaisderivaatat} ja niistä muodostettu
\kor{gradientti} (Luvut \ref{osittaisderivaatat}--\ref{gradientti}), vektorikenttiin liittyvät
\kor{divergenssi} ja \kor{roottori}
(Luvut \ref{divergenssi ja roottori}--\ref{div ja rot käyräviivaisissa}) ja lopulta yleisempiin
vektoriarvoisiin funktioihin eli \kor{epälineaarisiin kuvauksiin} liittyen 
\kor{Jacobin matriisi} (Luku \ref{jacobiaani}). Sovelluksina tarkastellaan mm.\ geometrisia
tehtäviä kuten \kor{pinnan tangenttitason} määrittämistä (Luku \ref{gradientti}), fysiikan 
\kor{osittaisdifferentiaaliyhtälöitä} 
(Luvut \ref{divergenssi ja roottori}--\ref{div ja rot käyräviivaisissa}), 
\kor{epälineaarisia yhtälöryhmiä} ja niiden numeerista ratkaisemista (Luku \ref{jacobiaani})
ja \kor{optimoinnin} ongelmia (Luku \ref{usean muuttujan ääriarvotehtävät}). Luvussa
\ref{käänteisfunktiolause} esitellään kolme epälineaarisiin yhtälöryhmiin liittyvää
\kor{epälineaarisen analyysin} vahvaa lausetta. Viimeisessä osaluvussa
(Luku \ref{usean muuttujan taylorin polynomit}) määritellään \kor{Taylorin polynomi} useamman
muuttujan funktioille.  %Usean muuttujan differentiaalilaskenta
\section{Usean muuttujan funktiot: Jatkuvuus ja raja-arvot} 
\label{usean muuttujan jatkuvuus}
\sectionmark{Usean muuttujan jatkuvuus}
\alku
\index{jatkuvuusb@jatkuvuus (usean muuttujan)|vahv}
\index{funktion raja-arvo|vahv}

Kahden ja kolmen reaalimuuttujan funktioiden algebraa on tarkasteltu aiemmin Luvussa 
\ref{kahden ja kolmen muuttujan funktiot}. Tässä ja seuraavissa luvuissa kohteena ovat kahden
ja kolmen muuttujan funktioiden lisäksi yleisemmät $n$ reaalimuuttujan reaaliarvoiset funktiot 
muotoa
\[
y=f(x_1,\ldots,x_n).
\]
Tässä voi käyttää myös matriisilaskun merkintää
\[
y=f(\mx), \quad \mx\in\R^n
\] 
\index{funktio A!d@$n$ muuttujan (vektorim.)} \index{vektorimuuttujan funktio}%
($\mx$ pysty- tai vaakavektori) ja puhua \kor{vektorimuuttujan} funktiosta. Mahdollista
(ja matemaattisissa teksteissä  tavallistakin) on matriisilaskun merkinnän sijasta käyttää myös
vektorimuuttujalle yksinkertaista symbolia $x$, eli merkitä
\[
y=f(x),\quad x=(x_1,\ldots,x_n)\in\R^n.
\]
Jatkossa käytetään eri merkintätapoja rinnakkain, jolloin matriisialgebran merkinnöissä $\mx$ 
tarkoittaa pystyvektoria. Kahden ja kolmen muuttujan tapauksissa merkitään vektorimuuttuja
vanhaan tapaan $(x,y)$ tai $(x,y,z)$. 

Funktion j\pain{atkuvuuden} ja \pain{ra}j\pain{a}-\pain{arvon} käsitteet, jotka toistaiseksi
on liitetty vain yhden muuttujan funktioihin 
(ks.\ Luvut \ref{jatkuvuuden käsite}--\ref{funktion raja-arvo}), ovat yleistettävissä melko
suoraviivaisesti useamman muuttujan funktioita koskeviksi. Aloitetaan kahden muuttujan
funktioista.

\subsection{Funktio $f(x,y)$}

Kahden muuttujan funktion jatkuvuuden ja raja-arvon määritelmien alustukseksi tarvitaan
\begin{Def} \label{lukuparijonon suppeneminen} \index{raja-arvo!c@lukuparien jonon|emph}
\index{suppeneminen!ab@lukuparien jonon|emph}
Jono $\seq{(x_n,y_n)}$, missä 
$(x_n,y_n) \in \R^2,\ n\in\N$, \kor{suppenee kohti} (lähestyy) \kor{lukuparia} (pistettä) 
$(x,y)$ täsmälleen kun $x_n \kohti x$ ja $y_n \kohti y$. 
\end{Def}
Kun lukuparien jonon suppenemiselle käytetään aiempaan tapaan merkintää '\kohti', niin on siis 
sovittu:
\[ 
(x_n,y_n) \kohti (x,y) \qekv x_n \kohti x\ \ \ja\ \ y_n \kohti y. 
\]
Tämän kanssa yhtäpitävä sopimus on 
(ks.\ Harj.teht.\,\ref{jonon raja-arvo}:\,\ref{H-I-6: lukujonopari}d)
\[
(x_n,y_n)\kohti(x,y) \qekv (x_n-x)^2+(y_n-y)^2 \kohti 0.
\]
Tämän mukaan siis $(x_n,y_n)\kohti(x,y)$ tarkoittaa yksinkertaisesti, että pisteen
$(x_n,y_n)$ (geometrinen) etäisyys pisteestä $(x,y)$ lähestyy (lukujonona) $0$:aa, kun
$n\kohti\infty$. Jos erityisesti $(x,y)=(0,0)$, niin polaarimuunnoksen $x_n=r_n\cos\varphi_n$,
$y_n=r_n\sin\varphi_n$ perusteella on
\[
(x_n,y_n)\kohti(0,0) \qekv r_n \kohti 0 \qquad \text{(polaarikoordinaatisto)}.
\]
Koska tämä ei aseta mitään rajoituksia jonolle $\seq{\varphi_n}$, niin lähestyminen voi
tapahtua esim.\ pitkin puolisuoraa, jolla $\varphi_n=\varphi=$ vakio, tai se voi olla
suunnaltaan 'hyppelehtivää', esim.\ spiraalimaista. Joka tapauksessa mahdollisia
lähestymis\-\pain{suuntia} (vastaten puolisuoria) on äärettömän monta. --- Tämä on olennainen
ero verrattuna yhden muuttujan tilanteeseen, jossa pistettä voi lähestyä vain kahdesta eri
suunnasta.

Kun kaksiulotteinen 'lähestyminen' $(x_n,y_n)\kohti(x,y)$ on näin määritelty, on
Määritelmien \ref{funktion jatkuvuus} (jatkuvuus) ja \ref{funktion raja-arvon määritelmä}
(funktion raja-arvo) yleistäminen kahden muuttujan tilanteeseen suoraviivaista:
\begin{Def} \label{kahden muuttujan jatkuvuus} 
\index{jatkuvuusb@jatkuvuus (usean muuttujan)!a@funktion $f(x,y)$|emph}
Funktio $f:\ \DF_f\subset\R^2\kohti\R$ on
\kor{jatkuva} pisteessä $(x,y)\in\DF_f$ täsmälleen kun kaikille reaalilukuparien jonoille
$\seq{(x_n,y_n)}$ pätee
\[
(x_n,y_n)\in\DF_f\,\ \forall n\,\ \ja\,\ (x_n,y_n)\kohti(x,y)\,\ 
                                  \impl\,\ f(x_n,y_n) \kohti f(x,y).
\]
\end{Def}
\begin{Def} \label{kahden muuttujan raja-arvo}
\index{raja-arvo!d@kahden muuttujan funktion|emph}
Funktiolla $f(x,y)$ on pisteessä $(a,b)$
\kor{raja-arvo} $A\in\R$, jos jokaiselle reaalilukuparien jonolle $\seq{(x_n,y_n)}$ pätee
\[
(a,b)\neq(x_n,y_n)\in\DF_f\,\ \forall n\,\ \ja\,\ (x_n,y_n) \kohti (a,b)\,\ 
                                             \impl\,\ f(x_n,y_n) \kohti A, 
\]
ja oletus on voimassa jollekin jonolle $\seq{(x_n,y_n)}$. Raja-arvo merkitään
\[
\lim_{(x,y)\kohti(a,b)}f(x,y)=A.
\]
\end{Def}
Kuten yhden muuttujan tapauksessa, Määritelmä \ref{kahden muuttujan raja-arvo} estää raja-arvon
määrittelyn sellaisessa (eristetyssä) pistessä $(a,b)$, jota kohti lähestyminen
$(x_n,y_n)\kohti(a,b)$ joukosta $\DF_f$ käsin ei ole mahdollista lisäehdolla
$(x_n,y_n)\neq(a,b)\ \forall n$. Määritelmässä \ref{kahden muuttujan jatkuvuus} tätä lisäehtoa
ei ole, joten jos $(a,b)\in\DF_f$ on $\DF_f$:n eristetty piste, niin $f$ on jatkuva pisteessä
$(a,b)$ (vrt.\ Harj.teht.\,\ref{jatkuvuuden käsite}:\,\ref{H-V-1: eristetty piste}).
\begin{Exa} \label{udif-1: esim 1} Määritellään funktiot
\[
f_1(x,y)=\frac{x^2y}{x^2+y^2}\,, \quad
f_2(x,y)=\frac{xy}{x^2+y^2}\,, \quad (x,y)\in\R^2,\ (x,y)\neq(0,0).
\]
Tutki, ovatko funktiot jatkuvia origossa, kun asetetaan $f_1(0,0)=f_2(0,0)=0$.
\end{Exa}
\ratk Polaarimuunnoksilla
\begin{align*}
&g_1(r,\varphi) \,=\, f_1(r\cos\varphi,r\sin\varphi) \,=\, r\cos^2\varphi\sin\varphi, \\
&g_2(r,\varphi) \,=\, f_2(r\cos\varphi,r\sin\varphi) \,=\, \cos\varphi\sin\varphi
\end{align*}
päätellään: Jos $(x_n,y_n) \vastaa (r_n,\varphi_n)$, niin
\[
r_n \kohti 0 \qimpl |f_1(x_n,y_n)| \,=\, |g_1(r_n,\varphi_n)|
                                   \,\le\, r_n \,\kohti\, 0 \,=\, f(0,0),
\]
joten $f_1$ on jatkuva pisteessä $(0,0)$. Sen sijaan $f_2$ on epäjatkuva origossa, sillä
$r_n \kohti 0\ \not\impl\ g_2(r_n,\varphi_n) \kohti 0$.  \loppu

Raja-arvoille ja jatkuvuudelle pätevät samanlaiset yhdistelysäännöt kuin yhden muuttujan 
tapauksessa, sillä näiden tulosten taustalla oleva logiikka ja algebra ei olennaisesti riipu 
muuttujien lukumäärästä. Jatkuvuuden yhdistelytulokset ovat seuraavat:
\index{jatkuvuusb@jatkuvuus (usean muuttujan)!b@yhdistelysäännöt|emph}%
\begin{Lause} \label{yhdistelylause 1} Jos $f:\ D_f \kohti \R,\ D_f \subset \R^2$, ja
$g:\ D_g \kohti \R,\ D_g \subset \R^2$, ovat jatkuvia pisteessä $(x,y) \in D_f \cap D_g$, niin 
myös $\lambda f\ (\lambda \in \R)$, $f+g$ ja $fg$ ovat jatkuvia pisteessä $(x,y)$. Jos lisäksi
$g(x,y) \neq 0$, niin myös $f/g$ on jatkuva pisteessä $(x,y)$. 
\end{Lause}
\begin{Lause} \label{yhdistelylause 2} Jos $f: D_f \kohti \R,\ D_f \subset \R^2$, on jatkuva 
pisteessä $(x,y) \in D_f$ ja $g: D_g \kohti \R,\ D_g \subset \R$, on jatkuva pisteessä
$f(x,y)\in\DF_g$, niin yhdistetty funktio $g \circ f$ on jatkuva pisteessä $(x,y)$. 
\end{Lause}

Määritelmän \ref{kahden muuttujan jatkuvuus} mukainen jatkuvuus yksittäisessä pisteessä
laajenee luonnollisella tavalla jatkuvuudeksi joukossa: $f$ on jatkuva joukossa
$A\subset\DF_f$, jos $f$ on jatkuva $A$:n jokaisessa pisteessä. Kuten yhden muuttujan
tapauksessa, tavanomaiset 'yhden lausekkeen funktiot' ovat jatkuvia koko määrittelyjoukossaan.
\jatko \begin{Exa} (jatko) Jos esimerkissä jätetään $f_1(0,0)$ ja $f_2(0,0)$ erikseen
määrittelemättä, niin esimerkin funktiot ovat Lauseen \ref{yhdistelylause 1} perusteella
jatkuvia koko yhteisessä määritelyjoukossaan
\[
\DF_{f_1} = \DF_{f_2} = \{(x,y)\in\R^2 \mid (x,y)\neq(0,0)\}.
\]
Funktiolla $f_1$ on origossa raja-arvo
\[
\lim_{(x,y)\kohti(0,0)}f_1(x,y)=0
\]
(Määritelmä \ref{kahden muuttujan raja-arvo}), joten $f_1$ saadaan origossa (ja siis koko
$\R^2$:ssa) jatkuvaksi asettamalla $f_1(0,0)=0$ (funktion jatkaminen!). Funktiolla $f_2$ ei
tätä raja-arvoa ole, joten $f_2$:n jatkaminen origoon ei ole mahdollista. \loppu
\end{Exa}
\begin{Exa} Rationaalifunktion $f(x,y)=p(x,y)/q(x,y)$ ($p$ ja $q$ polynomeja) määrittelyjoukko
on $\DF_f=\{(x,y)\in\R^2 \mid q(x,y) \ne 0\}$. Lauseen \ref{yhdistelylause 1} mukaan $p$ ja $q$
ovat jatkuvia koko määrittelyjoukossaan ($=\R^2$), samoin $f$. \loppu
\end{Exa}
\begin{Exa} Missä pisteissä funktio $f(x,y) = \sqrt{y-x^2}/(x-y^2)$ on a) määritelty,
b) jatkuva? 
\end{Exa}
\ratk a) Määrittelyjoukko on
$\,D_f = \{\,(x,y) \in \R^2 \mid y \ge x^2\ \ja\ x \neq y^2\,\}$. \newline 
b) Kyseessä on yhdistetty ja yhdistelty funktio $f = (g \circ f_1)/f_2$, missä 
\[ \begin{aligned} &f_1(x,y)=y-x^2,\ D_{f_1} = \R^2, \quad f_2(x,y)=x-y^2,\ D_{f_2} = \R^2, \\
                   &g(t) = \sqrt{t},\ D_g = [0,\infty) \subset \R. 
   \end{aligned} \]
Lauseiden \ref{yhdistelylause 1} ja \ref{yhdistelylause 2} perusteella $f$ on jatkuva koko 
määrittelyjoukossaan. \loppu 

Kuten yhden muuttujan tapauksessa, myös kahden muuttujan funktion jatkuvuus ja raja-arvo
voidaan määritellä vaihtoehtoisesti vetoamatta luku(pari)jonoi\-hin. Vaihtoehtoisessa
'$(\eps,\delta)$-määritelmässä' tarvitaan avointa väliä $(x-\delta,x+\delta)$ vastaava
\index{ympzy@($\delta$-)ympäristö}%
\kor{ympäristö}, tarkemmin \kor{pisteen avoin $\delta$-ympäristö}, jota merkitään
$U_\delta(x,y)$ ($\delta>0$). Kuten yhdessä ulottuvuudessa, tällä tarkoitetaan joukkoa, joka
ympäröi pistettä $\delta$:n verran, tai $\delta$:aan verrannollisen matkan, j\pain{oka}
\pain{suuntaan}. Tyypillisesti $U_\delta(x,y)$ ajatellaan joko neliön tai kiekon muotoiseksi,
eli
\begin{align*}
\text{joko}&: \quad U_\delta(x,y)=(x-\delta,x+\delta)\times(y-\delta,y+\delta), \\
 \text{tai}&: \quad U_\delta(x,y)=\{(x',y')\in\R^2 \mid (x'-x)^2+(y'-y)^2<\delta^2\}.
\end{align*}
Jatkuvuuden vaihtoehtoinen määritelmä on (vrt.\ Määritelmä \ref{vaihtoehtoinen jatkuvuus})
\begin{Def} \label{kahden muuttujan vaihtoehtoinen jatkuvuus}
\index{jatkuvuusb@jatkuvuus (usean muuttujan)!a@funktion $f(x,y)$|emph}
Funktio $f:\ \DF_f\subset\R^2\kohti\R$ on jatkuva pistessä $(x,y)\in\DF_f$ täsmälleen kun
jokaisella $\eps>0$ on olemassa $\delta>0$ siten, että
\[
|f(x',y')-f(x,y)| < \eps\ \ \forall (x',y') \in U_\delta(x,y)\cap\DF_f.
\]
\end{Def}
Määritelmät \ref{kahden muuttujan jatkuvuus} ja \ref{kahden muuttujan vaihtoehtoinen jatkuvuus}
voidaan osoittaa yhtäpitäviksi samaan tapaan kuin yhden muuttujan tapauksessa (vrt.\ Lause
\ref{jatkuvuuskriteerien yhtäpitävyys}).

Ympäristö $U_\delta(x,y)$ on nimensä mukaisesti esimerkki \kor{avoimesta} joukosta, jonka
yleisempi määritelmä on (vrt.\ Määritelmä \ref{analyyttinen funktio} joukoille $A\in\C$)
\begin{Def} \label{avoin joukko} \index{avoin joukko|emph } 
Joukko $A\subset\R^2$ on \kor{avoin}, jos jokaisella$(x,y) \in A$ on olemassa $\delta>0$ siten,
että $U_\delta(x,y) \subset A$.
\end{Def}
 
\subsection{Funktio $f(x,y,z)$}

Määritelmillä \ref{kahden muuttujan jatkuvuus} ja \ref{kahden muuttujan raja-arvo} on ilmeiset
vastineensa kolmen muuttujan funktioille $f(x,y,z)$. Esimerkiksi jatkuvuusehto pisteessä
$(x,y,z)\in\DF_f$ on \index{jatkuvuusb@jatkuvuus (usean muuttujan)!c@funktion $f(x,y,z)$}%
\[
(x_n,y_n,z_n)\in\DF_f\,\ \forall n\,\ \ja\,\ (x_n,y_n,z_n)\kohti(x,y,z)\,\
                                      \impl\,\ f(x_n,y_n,z_n) \kohti f(x,y,z),
\]
missä (vrt.\ Harj.teht.\,\ref{jonon raja-arvo}:\,\ref{H-I-6: lukujonopari}e)
\begin{align*}
(x_n,y_n,z_n)\kohti(x,y,z) &\qekv x_n \kohti x\ \ja\ y_n \kohti y\ \ja\ z_n \kohti z \\
                           &\qekv (x_n-x)^2+(y_n-y)^2+(z_n-z)^2 \kohti 0.
\end{align*}
Erityisesti jos $(x,y,z)=(0,0,0)$, niin pallokoordinaattimuunnoksen $(x_n,y_n,z_n)$
$\vastaa(r_n,\theta_n\varphi_n)$ perusteella on
\[
(x_n,y_n,z_n)\kohti(0,0,0) \qekv r_n \kohti 0 \qquad \text{(pallokoordinaatisto)}.
\]
\begin{Exa} Täsmälleen millä ehdoilla luvuille $\alpha,\beta,\gamma\in\R$ on
\[
\lim_{(x,y,z)\kohti(0,0,0)}f(x,y,z)=0, \quad \text{kun} \quad
f(x,y,z)=\frac{|x|^\alpha|y|^\beta|z|^\gamma}{x^2+y^2+z^2}\,?
\]
\end{Exa}
\ratk Tehdään pallokoordinaatimuunnos:
\[
f(x,y,z) = g(r,\theta,\varphi) = r^{\alpha+\beta+\gamma-2}|
                                 \sin\theta|^{\alpha+\beta}|\cos\theta|^\gamma
                                 |\cos\varphi|^\alpha|\sin\varphi|^\beta.
\]
Päätellään, että $\,r_n \kohti 0\ \impl\ g(r_n,\theta_n,\varphi_n) \kohti 0\,$ on tosi
täsmälleen ehdoilla
\[
\alpha+\beta+\gamma > 2,\,\ \alpha \ge 0,\,\ \beta \ge 0,\,\ \gamma \ge 0. \loppu
\]

\subsection{Funktio $f(\mx),\  \mx\in\R^n$}

Yleiselle $n$ muuttujan reaaliarvoiselle funktiolle $f$ jatkuvuuseehto pisteessä
$\mx\in\DF_f$ on
\index{jatkuvuusb@jatkuvuus (usean muuttujan)!d@funktion $f(\mx),\ \mx\in\R^n$}%
\[
\mx_k\in\DF_f\,\ \forall k\,\ \mx_k \kohti \mx\,\ \impl\,\ f(\mx_k) \kohti f(\mx),
\]
\index{suppeneminen!ac@$\R^n$:n jonon} \index{raja-arvo!e@$\R^n$:n jonon}%
missä $\seq{\mx_k}$ on $\R^n$:n vektorijono ja \kor{suppeneminen} $\mx_k \kohti \mx$ kohti
\kor{raja-arvoa} $\mx\in\R^n$ tarkoittaa:
\[
\mx_k \kohti \mx \qekv (\mx_k)_i \kohti (\mx)_i=x_i\,,\,\ i=1 \ldots n
                 \qekv |\mx_k-\mx| \kohti 0,
\]
missä $|\cdot|$ on $\R^n$:n euklidinen normi. Määritelmien
\ref{kahden muuttujan vaihtoehtoinen jatkuvuus} ja \ref{avoin joukko} $n$-ulotteisissa
vastineissa ympäristö $U_\delta(\mx)$ tulkitaan joko $n$-ulotteiseksi kuutioksi, jonka särmän
pituus $=2\delta$, tai $n$-ulotteiseksi kuulaksi, jonka säde $=\delta$.

\subsection{*Jatkuvuus kompaktissa joukossa}

Asetetaan (vrt.\ Määritelmät \ref{avoimet ym. joukot} ja \ref{jatkuvuus kompaktissa joukossa})
\begin{Def} \label{kompakti joukko - Rn} Joukko $K\subset\R^n$ on
\begin{itemize} \index{suljettu joukko|emph} \index{rajoitettu!b@joukko|emph}
\index{kompakti joukko|emph}
\item[-] \kor{suljettu}, jos kaikille $\R^n$:n vektorijonoille $\seq{\mx_k}$ pätee: \newline
         $\,\mx_k \in K\ \forall k\ \ja\ \mx_k \kohti \mx\in\R^n\ \impl\ \mx \in K$,
\item[-] \kor{rajoitettu}, jos $\exists C\in\R_+$ siten, että 
         $\,|\mx| \le C\ \forall \mx \in K$,
\item[-] \kor{kompakti}, jos $K$ on suljettu ja rajoitettu.
\end{itemize}
\end{Def}
Suljetun joukon vaihtoehtoinen määritelmä on (vrt.\ Lause \ref{avoin vs suljettu})

\begin{Lause} \label{avoin vs suljettu - Rn} \index{komplementti (joukon)|emph}
Joukko $A\subset\R^n$ on suljettu täsmälleen kun $A$:n \kor{komplementti}
$\complement(A)=\{\mx\in\R^n \mid \mx \not\in A\}$ on avoin.
\end{Lause}
\tod Harj.teht.\,\ref{H-udif-1: avoin vs suljettu}.
\begin{Exa} Kompakteja joukkoja $K\subset\R^n$ ovat esimerkiksi (suljettu) $n$-ulot\-teinen
suorakulmainen särmiö
\[
K=[a_1,b_1]\times[a_2,b_2]\times\ldots\times[a_n,b_n]
\]
ja (suljettu) $n$-ulotteinen kuula
\[
K = \{\mx\in\R^n \mid |\mx-\mx_0| \le R\}.
\]
Myös jokainen äärellinen $\R^n$:n osajoukko on kompakti
(Harj.teht.\,\ref{H-udif-1: äärellinen joukko}). \loppu
\end{Exa}
Kompaktin joukon käsite on keskeinen seuraavassa Weierstrassin lauseen
\ref{Weierstrassin peruslause} yleistyksessä, joka tapuksessa $n=1$ on todistettu aiemmin
Lauseena \ref{weierstrass}. Yleisempää todistusta ei esitetä; todettakoon ainoastaan, että
todistuksen logiikka on hyvin samanlainen kuin yhden muuttujan tapauksessa, ks.\ Luku
\ref{jatkuvuuden logiikka}. Asetetaan ensin (vrt.\ Määritelmä
\ref{jatkuvuus kompaktissa joukossa})
\begin{Def} \label{jatkuvuus kompaktissa joukossa - Rn} 
\index{jatkuvuusb@jatkuvuus (usean muuttujan)!e@kompaktissa joukossa|emph}
Funktio $f:\ \DF_f\subset\R^n\kohti\R$ on \kor{jatkuva kompaktissa joukossa} $K\subset\DF_f$,
jos jokaiselle $\R^n$:n vektorijonolle $\seq{\mx_k}$ pätee
\[
\mx_k \in K\ \forall k\,\ \ja\,\ \mx_k \kohti \mx\,\ \impl\,\ f(\mx_k) \kohti f(\mx).
\]
\end{Def}
\begin{*Lause} \label{weierstrass - Rn} \index{Weierstrassin lause|emph}
Jos funktio $f:\ \DF_f\subset\R^n\kohti\R$ on jatkuva
kompaktissa joukossa $K\subset\DF_f$, niin $f$ saavuttaa $K$:ssa pienimmän ja suurimman arvonsa.
\end{*Lause}

\Harj
\begin{enumerate}

\item
Kohdissa a)--h) määritä raja-arvo tai päättele, ettei raja-arvoa ole. Kohdissa i)--j) määritä
kaikki kertoimien $a,b,c$ arvot, joilla raja-arvo on olemassa. Siirtyminen napa- tai 
pallokoordinaatistoon voi auttaa.
\begin{align*}
&\text{a)}\ \lim_{(x,y)\kohti(2,-1)} (xy+y^2) \qquad 
 \text{b)}\ \lim_{(x,y)\kohti(0,0)} \sqrt{x^2+2y^2} \qquad
 \text{c)}\ \lim_{(x,y)\kohti(0,0)} \frac{x^2+y^2}{y} \\
&\text{d)}\ \lim_{(x,y)\kohti(0,0)}\,\frac{x}{x^2+y^2} \qquad\ 
 \text{e)}\ \lim_{(x,y)\kohti(0,0)}\,\frac{y^3}{x^2+y^2} \qquad\quad\ \
 \text{f)}\ \lim_{(x,y)\kohti(0,0)}\,\frac{x^2y^2}{x^2+y^4} \\
&\text{g)}\ \lim_{(x,y)\kohti(0,0)}\,\frac{\sin(xy)}{x^2+y^2} \qquad\ 
 \text{h)}\ \lim_{(x,y,z)\kohti(0,0,0)}\,\frac{\sin(xyz)}{x^2+y^2+z^2} \\
&\text{i)}\ \lim_{(x,y)\kohti(0,0)}\,\frac{xy}{ax^2+bxy+cy^2} \qquad
 \text{j)}\ \lim_{(x,y,z)\kohti(0,0,0)}\ \frac{x^2+y^2-z^2}{ax^2+by^2+cz^2}
\end{align*}

\item
Kohdissa a)--i) jatka funktion $f$ määritelmä niin, että funktiosta tulee jatkuva
mahdollisimman suuressa $\R^2$:n tai $\R^3$:n osajoukossa. Kohdissa j)--m) määritä kaikki
pisteet, joissa $f$ on epäjatkuva. 
\begin{align*}
&\text{a)}\ \ f(x,y)=\frac{x\sqrt[4]{\abs{y}}}{|x|+|y|} \qquad
 \text{b)}\ \ f(x,y)=\frac{\sqrt{(1+x^2)(1+y^2)}-1}{x^2+y^2} \\
&\text{c)}\ \ f(x,y)=\frac{x^3-8y^3}{x-2y} \qquad
 \text{d)}\ \ f(x,y)=\frac{x+y}{x^3+y^3} \qquad
 \text{e)}\ \ f(x,y)=\frac{x^2+y}{x^4-y^2} \\
&\text{f)}\ \ f(x,y)=\frac{\sin(x+y)}{x^2-y^2} \qquad
 \text{g)}\ \ f(x,y)=\frac{\sin(x^3+y^3)}{x+y} \\
&\text{h)}\ \ f(x,y,z)=\frac{1-\cos(xyz)}{x^2y^2z^2} \qquad
 \text{i)}\ \ f(x,y,z)=\frac{x^2-y^2+z^2+2xz}{x+y+z} \\
&\text{j)}\ \ f(x,y)=\begin{cases}
              4x^2+y^2-47, &\text{kun}\ x^2+y^2<25 \\ x^2-2y^2+28, &\text{kun}\ x^2+y^2 \ge 25
              \end{cases} \\
&\text{k)}\ \ f(x,y)=\begin{cases}
                     x^2+xy-2y^2, &\text{kun}\ 0<\abs{y}<x \\ 0, &\text{muulloin}
                     \end{cases} \\
&\text{l)}\ \ f(x,y)=
 \begin{cases}
 y\sin\left(\dfrac{1}{x}+\dfrac{1}{y}\right), &\text{kun}\ x \neq 0\,\ \text{ja}\,\ y \neq 0 \\
 0, &\text{kun}\ x=0\,\ \text{tai}\,\ y=0
 \end{cases} \\
&\text{m)}\ \ f(x,y,z)=\begin{cases}
                       \dfrac{\sin(xyz)}{\sin(xy)}\,, &\text{kun}\ xy \neq 0 \\
                       z, &\text{kun}\ xy=0
                       \end{cases}
\end{align*}

\item (*) \label{H-udif-1: kiero raja-arvo}
Määritellään $f:\,\Rkaksi\map\R$ seuraavasti:
\[
f(x,y)=\begin{cases} 
       \,1, &\text{kun}\ x^2<y<2 x^2\\ \,0, &\text{muulloin}
       \end{cases}
\]
Osoita, että funktiolla on sama raja-arvo origossa lähestyttäessä origoa mitä
tahansa suoraa $S:\ ax+by=0\ (a,b\in\R)$ pitkin, mutta siitä huolimatta ei ole olemassa
raja-arvoa $\,\lim_{(x,y)\to (0,0)} f(x,y)$.

\item (*) \label{H-udif-1: avoin vs suljettu}
Todista Lause \ref{avoin vs suljettu - Rn}.

\item (*)
Olkoon $f(x,y)=p(x,y)/q(x,y)$, missä $p$ ja $q$ ovat määriteltyjä ja jatkuvia koko $\R^2$:ssa
(esim.\ polynomeja). \vspace{1mm}\newline
a) Näytä suoraan Lauseen \ref{kahden muuttujan vaihtoehtoinen jatkuvuus} avulla, että
$\DF_f\subset\R^2$ on avoin. \vspace{1mm}\newline
b) Päättele $\DF_f$ avoimeksi osoittamalla ensin Määritelmään \ref{kompakti joukko - Rn}
nojaten, että $\{(x,y)\in\R^2 \mid q(x,y)=0\}$ on suljettu joukko ja vetoamalla Lauseeseen
\ref{avoin vs suljettu - Rn}.

\item (*) \label{H-udif-1: äärellinen joukko}
Olkoon $\ma_i\in\R^n,\ i=1 \ldots m\,\ (m\in\N)$, $\ma_i\neq\ma_j$ kun $i \neq j$, ja
$\,K=\{\ma_1,\ldots,\ma_m\}$. \vspace{1mm}\newline
a) Näytä, että $K\subset\R^n$ on kompakti joukko. \vspace{1mm}\newline
b) Näytä, että jos $f:\ \DF_f\subset\R^n\kohti\R$ on mikä tahansa funktio, jolle pätee
$\ma_i\in\DF_f,\ i=1 \ldots m$, niin $f$ on jatkuva $K$:ssa Määritelmän
\ref{jatkuvuus kompaktissa joukossa - Rn} mukaisesti. \vspace{1mm}\newline
c) Tarkista, että Lauseen \ref{weierstrass - Rn} väittämä on sopusoinnussa b)-kohdan tuloksen
kanssa.

\end{enumerate} %Usean muuttujan jatkuvuus ja raja-arvot
\section{Osittaisderivaatat} \label{osittaisderivaatat}
\alku
\index{osittaisderivaatta|vahv}

Kahden tai useamman muuttujan funktion \kor{osittaisderivaatalla} (engl. partial derivative) 
tarkoitetaan yksinkertaisesti funktion derivaattaa jonkin muuttujan suhteen muiden muuttujien
pysyessä vakioina raja-arvoprosessissa. Osittaisderivaatan symboli on $\partial$, joka luetaan
samoin kuin tavallinen derivaatta eli 'dee' (engl.\ 'partial'). Esimerkiksi kahden muuttujan
funktion $f(x,y)$ osittaisderivaattoja merkitään
\begin{align*}
\frac{\partial f}{\partial x} (x,y) 
         &= \lim_{\Delta x\kohti 0} \frac{f(x+\Delta x,y)-f(x,y)}{\Delta x}\,, \\
\frac{\partial f}{\partial y} (x,y) 
         &= \lim_{\Delta y\kohti 0} \frac{f(x,y+\Delta y)-f(x,y)}{\Delta y}\,.
\end{align*}
Vähemmän tilaa vieviä (ja laskentaakin nopeuttavia) merkintätapoja ovat
\[
\partial_x f, \ \partial_y f,\quad \text{tai} \quad f_x, \ f_y.
\]
Näistä ensimmäinen merkintätapa viittaa siihen tosiasiaan, että kuten tavallinen derivaatta,
myös osittaisderivaatta on 'funktion funktio' eli operaattori. \mbox{Alaindeksoitu} merkintä on 
tavallinen etenkin silloin, kun muuttujat ovat fysikaalisia paikka- ja aikamuuttujia.

Osittaisderivaatta käyttäytyy monessa suhteessa kuten tavallinen derivaatta. Se on esimerkiksi
derivoitavan funktion suhteen \kor{lineaarinen}:
\index{lineaarisuus!a@derivoinnin}%
\[
\partial_x\bigl[\lambda f(x,y)+\mu g(x,y)\bigr]
              =\lambda\partial_x f(x,y)+\mu\partial_x g(x,y),\quad \lambda,\mu\in\R.
\]
Osittaisderivoinnille pätee myös mm.\ tulon derivoimissääntö
\[
\frac{\partial}{\partial x} (fg)
              =\frac{\partial f}{\partial x} g + f\frac{\partial g}{\partial x}\,,
\]
sillä tämänkin säännön kannalta kyse on tavallisesta derivoinnista yhden valitun muuttujan 
suhteen.
\begin{Exa} \label{osder-esim 1} Funktion
\[
f(x,y) = \begin{cases} 
          \,0, &\text{kun}\ (x,y)=(0,0), \\ \,xy/(x^2+y^2)\,, &\text{muulloin}
         \end{cases}
\]
osittaisderivaatat muualla kuin origossa ovat
\[
f_x(x,y) = \frac{y^3-x^2y}{(x^2+y^2)^2}\,, \quad
f_y(x,y) = \frac{x^3-xy^2}{(x^2+y^2)^2}\,, \quad (x,y)\neq(0,0).
\]
Tässä on käytetty tavallisia (yhden muuttujan) rationaalifunktion derivoimissääntöjä. Koska
$f(x,0)=f(0,y)=0,\ x,y\in\R$, niin $f$:llä on osittaisderivaatat myös origossa:
\[
f_x(0,0)=f_y(0,0)=0. \loppu
\]
\end{Exa}
Esimerkin funktio on origossa epäjatkuva (ks.\ edellinen luku, Esimerkki \ref{udif-1: esim 1}).
--- Siis funktio voi olla epäjatkuva yksittäisessä pisteessä, vaikka
osittaisderivaatat ovat olemassa kaikkialla. Tämä ero yhden muuttujan funktioihin (joille
derivaatan olemassaolo pisteessä takaa jatkuvuuden ko.\ pisteessä) selittyy
sillä, että osittaisderivaattojen raja-arvoissa pistettä lähestytään vain koordinaattiakselien
suunnissa, kun usean muuttujan jatkuvuuden määritelmässä (ks.\ edellinen luku) mahdollisia
lähestymissuuntia on äärettömän monta. Usean muuttujan funktion derivoituvuus eli nk.\
\kor{differentioituvuus} onkin määriteltävä koordinaatistosta riippumattomana
'joka suuntaan derivoituvuutena', jotta se vastaa yhden muuttujan derivoituvuuden käsitettä.
Määritelmä asetetaan seuraavassa luvussa.

Funktion osittaisderivaattoja voidaan (funktioina) derivoida edelleen eri muuttujien suhteen,
jolloin saadaan
\index{kertaluku!d@osittaisderivaatan}%
\kor{toisen kertaluvun} osittaisderivaattoja, näitä derivoimalla
\kor{kolmannen kertaluvun} osittaisderivaattoja, jne. Korkeamman kertaluvun
osittaisderivaattoja merkitään
\begin{align*}
\frac{\partial}{\partial x}\left(\frac{\partial f}{\partial x}\right) 
       &= \frac{\partial^2 f}{\partial x^2}
        =\partial_x(\partial_x f)=(f_x)_x =f_{xx}, \\[1mm]
\frac{\partial}{\partial x}\left(\frac{\partial f}{\partial y}\right) 
       &= \frac{\partial^2 f}{\partial x\partial y}
        =\partial_x(\partial_y f)=(f_y)_x =f_{yx}, \\[1mm]
\frac{\partial}{\partial x}\left(\frac{\partial^2 f}{\partial x\partial y}\right) 
       &= \frac{\partial^3 f}{\partial x^2\partial y}
        =\partial_x^2(\partial_y f)=(f_y)_{xx}=f_{yxx}, \\[1mm]
\frac{\partial^2}{\partial y^2}\left(\frac{\partial^2 f}{\partial x^2}\right) 
       &= \frac{\partial^4 f}{\partial y^2\partial x^2}
        =\partial_y^2(\partial_x^2 f)=(f_{xx})_{yy}=f_{xxyy}\,, \quad \text{jne.}
\end{align*}
\begin{Exa} \label{osder-esim 2} Funktion $f(x,y)=\ln (x^2+y^2)$ osittaiderivaatat toiseen
kertalukuun asti pisteissä $(x,y)\neq(0,0)$ (eli $f$:n määrittelyjoukossa) ovat
\begin{align*}
f_x    &= \frac{2x}{x^2+y^2}\,,\quad f_y=\frac{2y}{x^2+y^2}\, \\
f_{xx} &= \frac{2(y^2-x^2)}{(x^2+y^2)^2}\,,\quad f_{yy} = -f_{xx}, \quad
          f_{xy} = f_{yx}=-\frac{4xy}{(x^2+y^2)^2}\,. \loppu
\end{align*}
\end{Exa}
Esimerkin tuloksessa $\,f_{xy}=f_{yx}\,$ on kyse yleisemmästä osittaisderivoinnin
\kor{vaihtosäännöstä} \index{vaihtoszyzy@vaihtosääntö!c@osittaisderivoinnin}
\index{osittaisderivaatta!a@vaihtosääntö}%
\begin{equation} \label{vaihtosääntö}
\boxed{\quad\kehys f_{xy}=f_{yx} \quad \text{(vaihtosääntö)}.\quad}
\end{equation}
Sääntö on pätevä lievin säännöllisyysehdoin, jotka yleensä voidaan olettaa. Ko.\ ehdot sekä
säännön perustelu esitetään luvun lopussa (Lause \ref{osittaisderivoinnin vaihtosääntö}).
Vaihtosäännön mukaan derivointijärjestys on vapaa myös korkeamman kertaluvun
osittaisderivaatoissa; esim.\ $f_{xxy}=f_{xyx}=f_{yxx}$.

\subsection{Monen muuttujan osittaisderivaatat}
\index{osittaisderivaatta!b@monen muuttujan|vahv}

Vaihtosääntö \eqref{vaihtosääntö} pätee myös useamman kuin kahden muuttujan tapauksessa, koska
kahden muuttujan suhteen derivoitaessa funktio voidaan ajatella vain ko.\ muuttujista
riippuvaksi. Siis jos funktion $f(x_1,\ldots,x_n)$ osittaiderivaatat
$\partial f/\partial x_k$ ja $\partial^2 f/(\partial x_k\partial x_l)$ ovat olemassa ja
jatkuvia, kun $k,l=1 \ldots n,\ k \neq l$, niin
\[
\frac{\partial^2 f}{\partial x_k\partial x_l}
    =\frac{\partial^2 f}{\partial x_l\partial x_k},\quad k,l\in\{1,\ldots,n\},\,\ k \neq l.
\]
Koska derivointijärjestys näin muodoin on vapaa (lievin ehdoin, jotka yleensä oletetaan),
riittää korkeammissa osittaisderivaatoissa vain tietää, kuinka monta kertaa kunkin muuttujan
suhteen derivoidaan. Yleistä osittaisderivaattaa merkitään tällöin käyttäen nk.\
\index{indeksi!c@moni-indeksi} \index{moni-indeksi}%
\kor{moni-indeksiä} (engl.\ multi-index), eli järjestettyä indeksijoukkoa (indeksivektoria)
muotoa
\[
\alpha=(\alpha_1,\ldots,\alpha_n),\quad \alpha_i\in\N\cup\{0\}.
\]
Tällöin $\partial^\alpha f$ tarkoittaa osittaisderivaattaa
\[
\partial^\alpha f
=\frac{\partial^{\abs{\alpha}} f}{\partial x_1^{\alpha_1}\cdots\partial x_n^{\alpha_n}}
=\partial_1^{\alpha_1}\cdots\partial_n^{\alpha_n} f,
\]
\index{kertaluku!d@osittaisderivaatan}%
missä $\partial_i=\partial/\partial x_i$, ja osittaisderivaatan \kor{kertalukua} on merkitty
\[
\abs{\alpha}=\sum_{i=1}^n \alpha_i.
\]
Jos moni-indeksissä $\alpha$ on $\alpha_k=0$, ei ko.\ muuttujan suhteen derivoida, ts.\
$\partial_k^0 f=f$.
\begin{Exa} Jos $x_1=x$ ja $x_2=y$, niin vaihtosääntö huomioiden
\begin{align*}
\partial^{(1,1)}f(x,y)    &= \partial_x\partial_y f(x,y) = f_{xy}(x,y), \\
\partial^{(0,2)}f(x,y)    &= (\partial_y)^2 f(x,y) = f_{yy}(x,y), \\
\partial^{(2,1)}f(x,y)    &= (\partial_x)^2\partial_y f(x,y) = f_{xxy}(x,y). \loppu
\end{align*}
\end{Exa}
\begin{Exa} Tulon derivoimissääntöä ja lineaarisuussääntöä soveltaen
\begin{align*}
\partial^{(1,1)}(fg)(x,y) &= \partial_x\partial_y (fg)(x,y) \\
                         &= \partial_x\left(f_yg+fg_y\right)(x,y) \\
                         &= \left(f_{xy}g+f_xg_y+f_yg_x+fg_{xy}\right)(x,y). \loppu
\end{align*}
\end{Exa}
\begin{Exa} \vahv{Neliömuoto}. \label{neliömuoto} \index{neliömuoto}
Yleinen $\R^n$:n nk. \kor{neliömuoto} (engl. quadratic form) on funktio muotoa
\[
f(x_1,\ldots,x_n)=\frac{1}{2}\sum_{i,j=1}^n a_{ij}x_ix_j,\quad a_{ij}\in\R,
\]
tai matriisialgebran merkinnöin
\[
f(\mx)=\frac{1}{2}\mx^T\mA\mx,\quad \mx=(x_i), \ \mA=(a_{ij}).
\]
Tässä voidaan olettaa, että $a_{ij}=a_{ji}$ ($\mA$ symmetrinen), koska $f$ riippuu vain 
summista $a_{ij}+a_{ji}$ (\,$a_{ij}x_ix_j+a_{ji}x_jx_i=(a_{ij}+a_{ji})x_ix_j$\,). Kun tehdään tämä
oletus, niin tietystä muuttujasta $x_k$ riippuvat termit erottuvat muodossa
\[
f(x_1,\ldots,x_n)=\frac{1}{2}a_{kk}x_k^2+\sum_{j\neq k} a_{kj} x_kx_j+ [\ldots],
\]
missä $[\ldots]$ ei sisällä muuttujaa $x_k$. Näin ollen
\[
\frac{\partial f}{\partial x_k}=\sum_{j=1}^n a_{kj} x_j = [\mA\mx]_k,\,\ k=1\ldots n
              \qimpl \left(\frac{\partial f}{\partial x_i}\right) = \mA\mx.
\]
Toisen kertaluvun osittaisderivaatat ovat
\[
\left(\frac{\partial^2 f}{\partial x_i\partial x_j}\right) = \left(a_{ij}\right) = \mA.
\]
Korkeamman kertaluvun osittaisderivaatat häviävät. \loppu
\end{Exa}

\subsection{Osittaisdifferentiaalioperaattorit}
\index{differentiaalioperaattori!c@osittaisderivoinnin|vahv}
\index{osittaisdifferentiaalioperaattori|vahv}

Kun osittaisderivaatoilla lasketaan, on usein kätevää ajatella symbolit $\partial_x$,
$\partial_y$, jne.\ derivoinnin yksittäisistä kohteista irrotetuiksi operaattoreiksi.
Esimerkiksi jos $\partial^\alpha$ ja $\partial^\beta$ ovat $n$ muuttujan
differentaalioperaattoreita, niin riittävän säännöllisille funktioille $f$ pätee vaihtosäännön
perusteella
\[
\partial^\alpha\left[\partial^\beta f(x)\right] = \partial^\beta\left[\partial^\alpha f(x)\right]
                                                = \partial^{\alpha+\beta} f(x).
\]
(Tässä $\alpha+\beta$ on moni-indeksien summa vektoreina.) Tämän voi ilmaista lyhyemmin
pelkästään operaattoreita koskevana laskusääntönä
\[
\partial^\alpha\partial^\beta=\partial^\beta\partial^\alpha=\partial^{\alpha+\beta}.
\]
\index{operaattoritulo} \index{kommutoivat operaattorit}%
Tällöin $\partial^\alpha\partial^\beta$ tarkoitaa \kor{operaattorituloa} eli kaksivaiheista
(yhdistettyä) operaatiota $f \map \partial^\beta f \map \partial^\alpha\partial^\beta f$.
Vaihtosäännön perusteella differentiaalioperaattorit $\partial^\alpha$ ja $\partial^\beta$ siis
\kor{kommutoivat}, eli pätee vaihdantalaki
$\partial^\alpha\partial^\beta=\partial^\beta\partial^\alpha$. --- Huomattakoon, että operaattori
$\partial^\alpha$ on määritelmänsä mukaisesti itsekin $(\abs{\alpha}-1)$-kertainen
operaattoritulo, jonka tekijöitä ovat yksittäiset differentiaalioperaattorit 
$\partial_k=\partial/\partial x_k$.

Operaattoritulon (eli yhdistetyn operaattorin) ohella toinen yleinen
differentiaalioperaattorien yhdistelytapa on
\index{lineaariyhdistely}%
\kor{lineaariyhdistely}. Tässä periaate on sama kuin funktioilla yleensä:
\[
(\lambda\partial^\alpha+\mu\partial^\beta)f=\lambda\partial^\alpha f+\mu\partial^\beta f,\quad 
\lambda,\mu\in\R.
\]
\begin{Exa}
Kun määritellään kahden muuttujan funktioille differentiaalioperaattori
$A=\partial_x+\partial_y$, niin
\begin{align*}
A^2 &= (\partial_x+\partial_y)(\partial_x+\partial_y) \\
    &= \partial_x(\partial_x+\partial_y)+\partial_y(\partial_x+\partial_y) \\
    &= (\partial_x)^2+\partial_x\partial_y+\partial_y\partial_x + (\partial_y)^2 \\
    &= (\partial_x)^2+2\partial_x\partial_y + (\partial_y)^2.
\end{align*}
Tämän mukaisesti on $\,A^2 f= A(Af)=f_{xx}+2f_{xy}+f_{yy}$. Yleisemminkin $A^k,\ k\in\N$
purkautuu binomikaavalla, esim.\
\[
A^3 f=f_{xxx}+3f_{xxy}+3f_{xyy}+f_{yyy}. \loppu
\]
\end{Exa}
Lineaarisen yhdistelyn kaavassa voivat kertoimet $\lambda,\mu$ olla myös muuttuvia, ts.\
\index{muuttuvakertoiminen diff.-oper.}%
funktioita. Muodostettaessa tällaisten \kor{muuttuvakertoimisten} operaattorien tuloja
derivointi kohdistuu myös kertoimiin, jolloin tulos muuttuu vakiokertoimisesta tapauksesta.
Operaattorituloja purettaessa tarvitaan tällöin myös tulon derivoimissääntöä.
\begin{Exa} Määritellään kahden muttujan funktioille differentiaalioperaattorit
$\ A=x\partial_x+y\partial_y\,$ ja $\,B=y^2\partial_x-x^2\partial_y$. Tällöin
\begin{align*}
AB &= (x\partial_x+y\partial_y)(y^2\partial_x-x^2\partial_y) \\
   &= x\partial_x(y^2\partial_x)-x\partial_x(x^2\partial_y)
                                +y\partial_y(y^2\partial_x)-y\partial_y(x^2\partial_y) \\
   &= xy^2(\partial_x)^2-2x^2\partial_y-x^3\partial_x\partial_y+2y^2\partial_x
                        +y^3\partial_y\partial_x-x^2y(\partial_y)^2 \\
   &= xy^2(\partial_x)^2-x^2y(\partial_y)^2+(y^3-x^3)\partial_x\partial_y
                        +2y^2\partial_x-2x^2\partial_y, \\[2mm]
BA &= AB-y^2\partial_x+x^2\partial_y. \loppu
\end{align*}
\end{Exa}
Esimerkissä differentiaalioperaattorit eivät kommutoi, ts. $AB \neq BA$. Muuttuvakertoimisessa
tapauksessa tämä on pääsääntö. Vakiokertoimiset operaattorit sen sijaan kommutoivat aina 
(vaihtosäännön ehdoin).

\subsection{Ketjusäännöt}
\index{osittaisderivaatta!c@ketjusäännöt|vahv}
\index{ketjusääntö (os.-derivoinnin)|vahv}

Jos funktiossa $f(x_1,\ldots,x_n)$ muuttujista $x_i$ tehdään muuttujan $t$ funktioita, ts.\
$x_i=x_i(t)$, niin saadaan yhdistetty funktio $F(t)=f(x_1(t),\ldots,x_n(t))$. Tälle pätee
seuraava yhdistetyn funktion derivoimissäännön yleistys, josta käytetään nimeä
\kor{ketjusääntö} (engl.\ chain rule).
\begin{equation} \label{ketjusääntö a}
\boxed{
\begin{aligned}
\ykehys &F(t) = f\bigl(x_1(t),\ldots,x_n(t)\bigr) \quad \\
  \quad &\impl \ F'(t) = \sum_{k=1}^n \frac{\partial f}{\partial x_i}\,x_i'(t) 
               \quad \text{(ketjusääntö)}. \quad
\end{aligned}} \tag{2a}
\end{equation}
Jos $t$:n tilalla on vektorimuuttuja $(u_1,\ldots,u_m)$, niin ketjusäännön yleisempi muoto on
\begin{equation} \label{ketjusääntö b}
\boxed{
\begin{aligned} \ykehys\quad 
F(u_1,\ldots,u_m) &= f\bigl(x_1(u_1,\ldots,u_m),\ldots,x_n(u_1,\ldots,u_m)\bigr) \quad \\
\impl \ \frac{\partial F}{\partial u_k} &= \sum_{i=1}^n \frac{\partial f}{\partial x_i}
\frac{\partial x_i}{\partial u_k} \quad \text{(yleinen ketjusääntö)}.
\end{aligned}} \tag{2b}
\end{equation}
Lauseessa \ref{ketjusääntö} jäljempänä esitetään riittävät sännöllisyysehdot ketjusäännön
\eqref{ketjusääntö a} pätevyydelle tapauksessa $n=2$. Vastaavat ehdot ja todistus yleiselle
säännölle \eqref{ketjusääntö b} ($n,m\in\N$) ovat tästä tapauksesta helposti yleistettävissä.
\begin{Exa} Funktio $y(x)$ on alkuarvotehtävän 
\[ \begin{cases} y' = f(x,y), \\ y(0)=1 \end{cases} \]
ratkaisu. Määritä $y$:n Taylorin polynomi $T_2(x,0)$, kun tiedetään, että $f(0,1)=2$,
$f_x(0,1)=-2$ ja $f_y(0,1)=4$.
\end{Exa}
\ratk Differentiaaliyhtälön ja alkuehdon perusteella on $y'(0)=f(0,y(0))=f(0,1)=2$, jolloin
differentiaaliyhtälön ja ketjusäännön \eqref{ketjusääntö a} perusteella on edelleen
\[ 
y''(x) = \frac{d}{dx} f(x,y(x)) = f_x(x,y(x)) + f_y(x,y(x))y'(x), 
\]
joten $y''(0) = -2 + 4 \cdot 2 = 6$. Siis kysytty Taylorin polynomi on
\[ 
T_2(x,0) = y(0) + y'(0)\,x + \tfrac{1}{2}\,y''(0)\,x^2 = 1+2x+3x^2. \loppu 
\]

\subsection{Implisiittinen (osittais)derivointi}
\index{osittaisderivaatta!d@implisiittinen osittaisdervointi|vahv}
\index{implisiittinen osittaisderivointi|vahv}

Eräs ketjusääntöjen \eqref{ketjusääntö a}--\eqref{ketjusääntö b} sovellus on yleinen 
implisiittisen derivoinnin tai osittaisderivoinnin periaate. Olkoon esimerkiksi $F$ kahden
muuttujan funktio ja oletetaan, että yhtälö
\[
F(x,y)=0
\]
määrittelee derivoituvan funktion $y=y(x)$. Silloin on $F(x,y(x))=0$, joten säännön 
\eqref{ketjusääntö a} perusteella (oletetaan $F$:n riittävä säännöllisyys)
\[
0 = \frac{d}{dx} F(x,y(x)) = F_x(x,y(x))+F_y(x,y(x))y'(x).
\]
Sikäli kuin $F_y(x,y(x))\neq 0$, voidaan tästä ratkaista $y'(x)\,$:
\[
y'(x)=-\frac{F_x(x,y)}{F_y(x,y)}\,, \quad y=y(x).
\]
\begin{Exa} Yhtälö
\[
F(x,y)=xy^3-e^{-xy^2}+y+\sin(y\cos x)=0
\]
määrittelee origon ympäristössä funktion $y(x)$. Jos halutaan laskea $y'(0)$, on ensin
laskettava $y(0)=y_0$ ratkaisemalla (numeerisin keinoin)
\[
y+\sin y=1 \ \impl \ y=y_0.
\]
Tämän jälkeen seuraa ketjusäännöstä \eqref{ketjusääntö a}
\[
F_x(0,y_0)+F_y(0,y_0)y'(0)=0,
\]
missä
\begin{align*}
F_x &= y^3+y^2e^{-xy^2}-y\sin x\cos (y\cos x) \\
F_y &= 3xy^2+2xye^{-xy^2}+1+\cos x\cos (y\cos x) \\
&\impl \ y'(0)=-\frac{y_0^3+y_0^2}{1+\cos y_0}. \loppu
\end{align*}
\end{Exa}

Kun implisiittifunktiossa on useampia muuttujia, voidaan ketjusääntöä \eqref{ketjusääntö b} 
käyttää \kor{implisiittiseen osittaisderivointiin}. Esimerkiksi jos yhtälö
\[
F(x,y,z)=0
\]
määrittelee funktion $z=z(x,y)$, niin säännön \eqref{ketjusääntö b} mukaan pätee
\begin{align*}
0 &= \partial_x F(x,y,z(z,y))=\frac{\partial F}{\partial x} + \frac{\partial F}{\partial z}
\frac{\partial z}{\partial x}, \\
0 &= \partial_y F(x,y,z(z,y))=\frac{\partial F}{\partial y} + \frac{\partial F}{\partial z}
\frac{\partial z}{\partial y}.
\end{align*}
Sikäli kuin on $F_z(x,y,z) \neq 0$, voidaan tästä ratkaista $\partial z/\partial x$ ja
$\partial z/\partial y$.
\begin{Exa} Pisteen $(x,y,z)=(1,2,0)$ ympäristössä määritellään
\[
z=f(x,y) \ \ekv \ 4x^2-y^2+z^2+e^z=1.
\]
Implisiittisesti derivoimalla saadaan
\[
8x+(2z+e^z)\frac{\partial z}{\partial x} = 0,\quad 
-2y+(2z+e^z)\frac{\partial z}{\partial y} = 0.
\]
Tässä on $\partial z / \partial x = f_x(x,y)$ ja $\partial z / \partial y = f_y(x,y)$, joten 
sijoittamalla $x=1$, $y=2$, $z=0$ saadaan
\[
f_x(1,2)=-8,\quad f_y(1,2)=4.
\]
Derivoimalla edelleen implisiittisesti voidaan laskea $f$:n korkeamman kertaluvun 
osittaisderivaattoja. Esimerkiksi
\begin{align*}
0\,&=\,\partial_y\left(8x+(2z+e^z)\pder{z}{x}\right)\,
    =\,(2+e^z)\pder{z}{x}\pder{z}{y}+(2z+e^z)\frac{\partial^2 z}{\partial x \partial y} \\
\impl\ 
    0\,&=\,3\cdot(-8)\cdot 4 + 1\cdot f_{xy}(1,2)\ \impl\ f_{xy}(1,2)= 96. \loppu
\end{align*}
\end{Exa}

\subsection{Derivointi integraalin alla}
\index{osittaisderivaatta!e@derivointi integraalin alla|vahv}
\index{derivointi integraalin alla|vahv}

Halutaan johtaa derivoimissääntö funktiolle $F$, joka on määritelty integraalin avulla muodossa
\[
F(x) = \int_a^b f(x,t)\,dt.
\]
Koska integraalin lineaarisuuden nojalla pätee
\[
\frac{F(x+\Delta x)-F(x)}{\Delta x} = \int_a^b \frac{f(x+\Delta x,t)-f(x,t)}{\Delta x}\,dt,
\]
niin näyttäisi, että derivointi on vietävissä integraalin alle osittaisderivoinniksi 
muutettuna:
\begin{equation} \label{d-int a}
\boxed{\quad\kehys F(x) =\int_a^b f(x,t)\,dt\,\ \impl\,\ 
                   F'(x)=\int_a^b \partial_x f(x,t)\,dt. \quad} \tag{3a}
\end{equation}
Useamman muuttujan tapauksessa tämä vastaa sääntöä
\begin{equation} \label{d-int b}
\boxed{\quad\kehys \frac{\partial}{\partial x_k}\int_a^b f(x_1,\ldots,x_n,t)\,dt 
              = \int_a^b \frac{\partial}{\partial x_k}f(x_1,\ldots,x_n,t)\,dt. \quad} \tag{3b}
\end{equation}
Säännöt \eqref{d-int a}--\eqref{d-int b} ovat todella voimassa melko yleispätevästi, joten
niitä voidaan pitää (kuten vaihtosääntöä \eqref{vaihtosääntö}) käytännössä sovellettavina
pääsääntöinä, joista on vain harvinaisia poikkeuksia 
(ks.\ Harj.teht.\,\ref{H-udif-1: poikkeus}a). --- Epäselvissä tapauksissa on sääntöjen
\eqref{d-int a}--\eqref{d-int b} perustelussa viime kädessä nojattava suoraan derivaatan
määritelmään (ks.\ Harj.teht.\,\ref{H-udif-1: Gamman derivaatta}). Siinä tapauksessa, että
kyseessä on tavanomainen (Riemannin) integraali välillä $[a,b]$, rittävät takeet säännön
\eqref{d-int a} pätevyydelle antaa Lause \ref{derivointi integraalin alla} alla.
\begin{Exa} $\displaystyle{F(x)=\int_1^2 \frac{e^{-xt}}{t}\,dt, \quad F'(0)=\,?}$ 
\end{Exa}
\ratk Sääntö \eqref{d-int a} on pätevä (ks.\ Lause \ref{derivointi integraalin alla}), joten
\[
F'(0)= \left[\int_1^2 \partial_x\left(\frac{e^{-xt}}{t}\right)\,dt\right]_{x=0} 
     = \left[\int_1^2(-e^{-xt})\,dt\right]_{x=0} = \int_1^2 (-1)\,dt = -1. \loppu
\]

\subsection{Osittaisderivoinnin sääntöjen perustelut}

Seuraavassa perustellaan osittaisderivoinnin vaihtosääntö \eqref{vaihtosääntö}, ketjusääntö
\eqref{ketjusääntö a} ja derivoinnin ja integroinnin vaihtosääntö \eqref{d-int a}, rajoittuen
viimeksi mainitussa tapauksessa tavalliseen Riemannin integraaliin välillä $[a,b]$.
Jatkuvuusoletuksissa viitataan edellisen luvun määritelmiin.
\begin{Lause} \label{osittaisderivoinnin vaihtosääntö}
\index{osittaisderivaatta!a@vaihtosääntö|emph}
\index{vaihtoszyzy@vaihtosääntö!c@osittaisderivoinnin|emph}
\vahv{(Osittaisderivoinnin vaihtosääntö)} \ Jos kahden muuttujan funktio $f$ on jatkuva ja
osittaisderivaatat $f_x$ ja $f_y$ olemassa ja jatkuvia pisteen $(x,y)$ ympäristössä
$U_\delta(x,y)\subset\DF_f$ ja lisäksi osittaisderivaatat $f_{xy}$ ja $f_{yx}$ ovat olemassa
ko.\ ympäristössä ja jatkuvia pisteessä $(x,y)$, niin $f_{xy}(x,y)=f_{yx}(x,y)$. 
\end{Lause}
\tod Tarkastellaan jonoa $\seq{(\Delta x_n,\Delta y_n)}$, jolle pätee
$(\Delta x_n,\Delta y_n)\kohti(0,0)$ ja
$\abs{\Delta x_n}^2+\abs{\Delta y_n}^2 < \delta^2\ \forall n$, jolloin
$(x+\Delta x_n,y+\Delta y_n) \in U_\delta(x,y)\ \forall n$. Tutkitaan lauseketta
\[
\delta_n[f]=f(x+\Delta x_n,y+\Delta y_n)-f(x,y+\Delta y_n)-f(x+\Delta x_n,y)+f(x,y).
\] 
Ryhmittelemällä tämän lausekkeen termejä eri tavoin, merkitsemällä
\[
F_n(t)=f(t,y+\Delta y_n)-f(t,y), \quad G_n(t)=f(x+\Delta x_n,t)-f(x,t)
\]
ja käyttämällä Differentiaalilaskun väliarvolausetta
(ks.\ Harj.teht.\,\ref{H-udif-2: perusteluja}) seuraa
\begin{align*}
\delta_n[f]\ &=\ F_n(x+\Delta x_n)-F_n(x) \\
             &=\ F_n'(\xi_n)\Delta x_n \\
             &=\ \partial_x [f(t,y+\Delta y_n)-f(t,y)]_{t=\xi_n}\Delta x_n \\
             &=\ [f_x(\xi_n,y+\Delta y_n)-f_x(\xi_n,y)]\Delta x_n \\
             &=\ \bigl[\partial_y [f_x(\xi_n,y)]_{y=\eta_n}\Delta y_n\bigr]\Delta x_n \\
             &=\ f_{xy}(\xi_n,\eta_n)\Delta x_n\Delta y_n, \\[2mm]
\delta_n[f]\ &=\ G_n(y+\Delta y_n)-G_n(y) \\
             &=\ G_n'(\eta'_n)\Delta y_n \\
             &=\ \partial_y [f(x+\Delta x_n,t)-f(x,t)]_{t=\eta'_n}\Delta y_n \\
             &=\ [f_y(x+\Delta x_n,\eta'_n)-f_y(x,\eta'_n)]\Delta y_n \\
             &=\ \bigl[\partial_x [f_y(x,\eta'_n)]_{x=\xi'_n}\Delta x_n\bigr]\Delta y_n \\
             &=\ f_{yx}(\xi'_n,\eta'_n)\Delta x_n\Delta y_n.
\end{align*}
Tässä on $|\xi_n-x|,|\xi'_n-x|\le|\Delta x_n|$ ja $|\eta_n-y|,|\eta'_n-y|\le|\Delta y_n|$
(Harj.teht.\,\ref{H-udif-2: perusteluja}), joten $(\xi_n,\eta_n)\kohti(x,y)$ ja
$(\xi'_n,\eta'_n)\kohti(x,y)$, koska $(\Delta x_n,\Delta y_n)\kohti(0,0)$. Näin ollen ja koska
$f_{xy}$ ja $f_{yx}$ ovat oletuksen mukaan jatkuvia pisteessä $(x,y)$, niin seuraa
\[
f_{xy}(\xi_n,\eta_n) \kohti f_{xy}(x,y) \quad \text{ja} \quad
f_{yx}(\xi'_n,\eta'_n) \kohti f_{yx}(x,y), \quad \text{kun}\ n\kohti\infty.
\]
Mutta em.\ tulosten mukaan on tässä $f_{xy}(\xi_n,\eta_n)=f_{yx}(\xi'_n,\eta'_n)\ \forall n$,
joten on oltava myös $f_{xy}(x,y)=f_{yx}(x,y)$. \loppu

\begin{Lause} \label{ketjusääntö} \index{osittaisderivaatta!c@ketjusäännöt|emph}
\index{ketjusääntö (os.-derivoinnin)|emph}
\vahv{(Ketjusääntö)} Jos yhdistetyssä funktiossa
$\,F(t)=f(x(t),y(t))\,$ $f(x,y)$ on jatkuva pisteen $(x(t),y(t))$ ympäristössä
$U_\delta(x(t),y(t))$, osittaisderivaatat $f_x$ ja $f_y$ ovat olemassa ko.\ ympäristössä ja
jatkuvia pisteessä $(x(t),y(t))$, ja derivaatat $x'(t)$ ja $y'(t)$ ovat olemassa, niin pätee
derivoimissääntö
\[
F'(t)=f_x(x(t),y(t))x'(t)+f_y(x(t),y(t))y'(t).
\]
\end{Lause}
\tod Olkoon $\eps>0$ riittävän pieni (tarkempi ehto hetimiten) ja tarkastellaan jonoa
$\seq{\Delta t_n}$, jolle pätee $0<|\Delta t_n|<\eps\ \forall n$ ja
$\Delta t_n \kohti 0\ (n\kohti\infty)$. Kirjoitetaan $x(t)=x$, $y(t)=y$ ja
$x(t+\Delta t_n)=x+\Delta x_n$, $y(t+\Delta t_n)=y+\Delta y_n$, $n=1,2,\ldots$
Koska $x'(t)$ ja $y'(t)$ ovat olemassa, niin
\[
\Delta x_n = x'(t)\Delta t_n+\ord{|\Delta t_n|}, \quad
\Delta y_n = y'(t)\Delta t_n+\ord{|\Delta t_n|}.
\]
Näin ollen $\eps>0$ voidaan valita niin, että toteutuu $(\Delta x_n)^2+(\Delta y_n)^2<\delta^2$
$\forall n$ kun $|\Delta t_n|<\eps\ \forall n$, jolloin
$(x+\Delta x_n,y+\Delta y_n) \in U_\delta(x,y)\ \forall n$. Olettaen näin pätee
Differentiaalilaskun väliarvolauseen perusteella
\begin{align*}
F(t+\Delta t_n)&-F(t) = f(x+\Delta x_n,y+\Delta y_n)-f(x,y) \\
        &= [f(x+\Delta x_n,y+\Delta y_n)-f(x,y+\Delta y_n)]+[f(x,y+\Delta y_n)-f(x,y)] \\
        &= f_x(\xi_n,y+\Delta y_n)\Delta x_n+f_y(x,\eta_n)\Delta y_n,
\end{align*}
missä $\xi_n\in[x,x+\Delta x_n]$ tai $\xi_n\in[x+\Delta x_n,x]$ ja vastaavasti
$\eta_n\in[y,y+\Delta y_n]$ tai $\eta_n\in[y+\Delta y_n,y]$. (Jos $\Delta x_n=0$, asetetaan
$\xi_n=x$, vast.\ $\eta_n=y$ jos $\Delta y_n=0$.) Jakamalla puolittain $\Delta t_n$:llä
seuraa $\Delta x_n/\Delta t_n \kohti x'(t)$ ja $\Delta y_n/\Delta t_n \kohti y'(t)$, kun
$n\kohti\infty$, ja jatkuvuusoletuksien nojalla $f_x(\xi_n,y+\Delta y_n) \kohti f_x(x,y)$ ja
$f_y(x,\eta_n) \kohti f_y(x,y)$. Näin ollen raja-arvo
$\lim_{\Delta t \kohti 0}[F(t+\Delta t)-F(t)]/\Delta t = F'(t)$ on olemassa ja väitetty
derivoimissääntö on pätevä. \loppu

Seuraavan lauseen jatkuvuusoletuksissa viittataan Määritelmään
\ref{jatkuvuus kompaktissa joukossa - Rn}. Todistus nojaa jatkuvuuden syvällisempään
logiikkaan (tasaiseen jatkuvuuteen, vrt.\ Luku \ref{jatkuvuuden logiikka}). Esitetään
todistuksesta vain helpotettu versio, joka perustuu hieman vahvennettuihin oletuksiin.
\begin{*Lause} \label{derivointi integraalin alla} \index{derivointi integraalin alla|emph}
\index{osittaisderivaatta!e@derivointi integraalin alla|emph}
(\vahv{Derivointi integraalin alla}) Jos
$f(x,t)$ on jatkuva ja $\partial_x f=f_x(x,t)$  olemassa ja jatkuva suorakulmiossa
$K_\delta=[c-\delta,c+\delta]\times[a,b]$ jollakin $\delta>0$,
niin derivoimissääntö \eqref{d-int a} on pätevä, kun $x=c$.
\end{*Lause}
\tod (helpotettu) Oletetaan lisäksi, että $f_x$ toteuttaa Lipschitz-ehdon
\begin{equation} \label{dint-a: L-ehto}
|f_x(x_1,t)-f_x(x_2,t)| \le L|x_1-x_2| \quad 
                        \forall\,(x_1,t),\,(x_2,t) \in K_\delta. \tag{$\star$}
\end{equation}
Jos $0<\abs{\Delta x}\le\delta$, niin oletusten ($f$ jatkuva, $f_x$ olemassa) ja
Differentiaalilaskun väliarvolauseen nojalla
\begin{multline*}
f(c+\Delta x,t)-f(c,t)-f_x(c,t)\Delta x\ =\ [f_x(\xi,t)-f_x(c,t)]\Delta x, \\[1mm]
        \text{missä}\,\ \xi\in(c-\abs{\Delta x},\,c+\abs{\Delta x})\subset(c-\delta,c+\delta).
\end{multline*}
Tämän ja Lipschitz-ehdon \eqref{dint-a: L-ehto} perusteella voidaan arvioida
\[
\left|\frac{f(c+\Delta x,t)-f(c,t)}{\Delta x}-f_x(c,t)\right| 
                \le L\abs{\xi-c} \le L\abs{\Delta x}, \quad t\in[a,b].
\]
Näin ollen, ja koska $f(c+\Delta x,t)$, $f(c,t)$ ja $f_x(c,t)$ ovat ($t$:n suhteen) 
Riemann-integroituvia välillä $[a,b]$ oletusten ja Analyysin peruslauseen nojalla, päätellään
\begin{align*}
&\int_a^b \frac{f(c+\Delta x,t)-f(c,t)}{\Delta x}\,dt 
              = \int_a^b f_x(c,t)\,dt + \ordoO{\abs{\Delta x}} \\
&\qimpl F'(c) = \lim_{\Delta x \kohti 0} \int_a^b \frac{f(c+\Delta x,t)-f(c,t)}{\Delta x}\,dt 
              = \int_a^b f_x(c,t)\,dt. \loppu
\end{align*}

\Harj
\begin{enumerate}

\item
Laske osittaisderivaatat kaikkien muuttujien suhteen:
\begin{align*}\ \
&\text{a)}\ \ xy^3+x\sin(xy) \qquad
 \text{b)}\ \ \ln\sin(x-2y) \qquad
 \text{c)}\ \ \Arcsin\frac{y}{x} \\
&\text{d)}\ \ x^y \qquad
 \text{e)}\ \ \log_y x \qquad
 \text{f)}\ \ z^{xy} \qquad
 \text{g)}\ \ z^{x^y} \qquad
 \text{h)}\ \ e^{xy(s^2+st)}
\end{align*}

\item
Missä pisteissä seuraavilla funktioilla on molemmat osittaisderivaatat?
\[
\text{a)}\ \ \sqrt{x^2+y^2} \qquad
\text{b)}\ \ (x+y)\abs{x+y} \qquad
\text{c)}\ \ \sqrt{|x^2-y^2|}
\]

\item
Laske \vspace{1mm}\newline
a) \ $(3\partial_x-2\partial_y)^2 (x^2+xy), \quad (x,y)=(0,0)$ \vspace{1mm}\newline
b) \ $(\partial_x+\partial_y)(\partial_y+\partial_z)\,xy^2z^3, \quad 
                                                   (x,y,z)=(1,1,1)$ \vspace{1mm}\newline
c) \ $\partial_x xy \partial_y xy^2, \quad (x,y)=(2,1)$ \vspace{1mm}\newline
d) \ $(x\partial_y-y\partial_x)^2 (xy-y^2), \quad (x,y)=(1,1)$ \vspace{1mm}\newline
e) \ $(\partial_1+\ldots+\partial_n)(x_1^2+\ldots+x_n^2), \quad 
                                   (x_1,\ldots,x_n)=(1,2,\ldots,n)$ \vspace{1mm}\newline
f) \ $\partial_1\cdots\partial_n \ln(x_1^2+\ldots+x_n^2), \quad 
                                    (x_1,\ldots,x_n)=(1,1,\ldots,1)$

\item
Laske funktion $f$ osittaisderivaatat annetussa pisteessä $P$. Jos on annettu $m$, laske myös
korkeamman kertaluvun osittaisderivaatat kertalukuun $m$ asti. Kohdissa j)--m) käytä
implisiittistä osittaisderivointia. \vspace{1mm}\newline
a) \ $f(x,y)=x-y+2,\,\ P=(3,2),\,\ m=2$ \vspace{1mm}\newline
b) \ $f(x,y)=xy+x^2,\,\ P=(2,0),\,\ m=2$ \vspace{1mm}\newline
c) \ $f(x,y)=\Arctan(y/x),\,\ P=(-1,1)$ \vspace{1.2mm}\newline
d) \ $f(x,y)=\sin(x\sqrt{y}),\,\ P=(\pi/3,4)$ \vspace{1mm}\newline
e) \ $f(x,y)=1/\sqrt{x^2+y^2},\,\ P=(-3,4)$ \vspace{1.1mm}\newline
f) \ $f(x,y,z)=xyz,\,\ P=(1,-1,2),\,\ m=3$ \vspace{1mm}\newline
g) \ $f(x,y,z)=\ln(1+e^{xyz}),\,\ P=(2,0,-1)$ \vspace{1mm}\newline
h) \ $f(x,y,z)=x^{y\ln z},\,\ P=(e,2,e)$ \vspace{1mm}\newline
i) \ $\,f(x_1,x_2,x_3,x_4)=(x_1-x_2^2)/(x_3+x_4^2),\,\ P=(3,1,-1,-2)$ \vspace{1mm}\newline
j) \ $\,z=f(x,y): \ x^2-2y^2-3yz^3-z=0,\,\ P:\ x=2,\ y=z=-1$ \vspace{1mm}\newline
k) \ $z=f(x,y): \ \cos xyz=\sin(x-y+2z),\,\ P:\ x=y=1,\ z=\pi/2$ \vspace{1mm}\newline
l) \ $\,z=f(x,y): \ 2xz^3-xyz=1,\,\ P:\ x=y=z=1,\,\ m=2$ \vspace{1mm}\newline
m) \ $u=f(x,y,z): \ xyzu+xyu^2+xu^3=1,\,\ P:\ x=u=1,\ z=y=-1$

\item
Laske funktion $f(x,y)$ osittaisderivaattojen avulla lausekkeet $A^2f$, $B^2f$ ja 
$(AB-BA)f$, kun
\begin{align*}
&\text{a)}\ \ A=\partial_x+\partial_y\,,\,\ B=\partial_x-\partial_y \quad\ 
 \text{b)}\ \ A=2\partial_x+\partial_y\,,\,\ B=\partial_x+3\partial_y \\
&\text{c)}\ \ A=y\partial_x\,,\,\ B=x\partial_y \qquad\qquad\ \
 \text{d)}\ \ A=x\partial_x+y\partial_y\,,\,\ B=y\partial_x-x\partial_y
\end{align*}

\item
Näytä, että pätee:
\begin{align*}
&\text{a)}\ \ \left[(\partial_x)^2+(\partial_y)^2\right] \ln(x^2+y^2)=0, \quad (x,y)\neq(0,0) \\
&\text{b)}\ \ \left[(\partial_x)^2+(\partial_y)^2+(\partial_z)^2\right] 
                                     \dfrac{1}{\sqrt{x^2+y^2+z^2}}=0, \quad (x,y,z)\neq(0,0,0)
\end{align*}

\item
Laske sekä suoraan että ketjusäännön avulla: \vspace{1mm}\newline
a) \ $\partial u/\partial t$, kun $u=\sqrt{x^2+y^2}\,,\ x=e^{st},\ y=1+s^2\cos t$ \newline
b) \ $\partial z/\partial x$, kun $z=\Arctan(u/v),\ u=2x+y,\ v=2x-y$ \newline
c) \ $d z/d t$, kun $z=txy^2,\ x=t+\ln(y+t^2),\ y=e^t$

\item
Oletetaan, että tunnetaan funktion $f(x,y)$ osittaisderivaattojen lausekkeet $f_x(x,y)$,
$f_y(x,y)$, $f_{xx}(x,y)$, jne. Laske näiden avulla:
\begin{align*}
&\text{a)}\ \ \frac{\partial}{\partial x} f(2x,3y) \qquad
 \text{b)}\ \ \frac{\partial}{\partial x} f(2y,3x) \qquad
 \text{c)}\ \ \frac{\partial}{\partial y} f(yf(x,t),f(y,t)) \\
&\text{d)}\ \ \frac{\partial^2}{\partial s\partial t} f(t\sin s,t\cos s) \qquad
 \text{e)}\ \ \frac{\partial^3}{\partial t^2\partial s} f(s^2-t,s+t^2)
\end{align*}

\item
Funktiosta $y(x)$ tiedetään, että $y$ toteuttaa välillä $(-a,a)$ differentiaaliyhtälön
$y'=f(x,y)$ ja että $y(0)=0$. Funktiosta $f(x,y)$ tiedetään, että $f$ on origon ympäristössä
säännöllinen ja $f(0,0) = 1$, $f_x(0,0) = -1$, $f_y(0,0) = 2$, $f_{xx}(0,0) = 4$,
$f_{xy}(0,0) = -2$, $f_{yy}(0,0) = -6$. Määritä funktion $y(x)$ kolmannen asteen Taylorin
polynomi $T_3(x,0)$.

\item \index{eksakti differentiaaliyhtälö}
Sanotaan, että differentiaaliyhtälö $f(x,y)+g(x,y)y'=0$ on \kor{eksakti}, jos $f(x,y)=F_x(x,y)$
ja $g(x,y)=F_y(x,y)$ jollakin $F$. Päättele, että eksaktin differentiaaliyhtälön yleinen
ratkaisu on $F(x,y)=C$. Sovella ratkaisuideaa:

a) \ $e^{x^2}(y'+2xy)=0 \qquad\qquad\ $
b) \ $3x^2+6xy^2+(6x^2y+4y^3)y'=0$ \newline
c) \ $ e^{-y}+(1-xe^{-y})y'=0 \qquad$
d) \ $2x(x+\cos^2y)+(2y-x^2\sin 2y)y'=0$

\item
Laske $f'(0)$:
\begin{align*}
&\text{a)}\ \ f(x)=\int_0^1 \frac{e^{xt}}{1+t^2}\,dt \qquad\qquad
 \text{b)}\ \ f(x)=\int_1^2 \ln t\,e^{-xt^2}\,dt \\
&\text{c)}\ \ f(x)=\int_0^\pi \sin xt\cos t\,dt \qquad\
 \text{d)}\ \ f(x)=\int_0^1 \ln(1+xt+t^2)e^{-xt}\,dt
\end{align*}

\item
Funktio $f$ määritellään origon ymäristössä kaavalla
\[
f(x,y)=\int_0^1 \frac{\sin xt\cos yt}{1+x+y+t}\,dt.
\]
Laske $f$:n osittaisderivaatat toiseen kertalukuun asti origossa.

\item \label{H-udif-1: poikkeus}
a) Olkoon $F(x)=\int_0^\infty x^3 e^{-x^2 t}\,dt,\ x\in\R$. Näytä, että $F$ on kaikkialla 
derivoituva, mutta $F'(0)$ ei ole laskettavissa derivoimalla integraalin alla. \newline
b) Totea, että derivoimissäännön \eqref{d-int a} mukaan
\[
\frac{d}{dx} \int_1^\infty \frac{e^{-xt}}{t}\,dt = -\frac{e^{-x}}{x}\,, \quad x>0.
\]
Varmista tulos vaihtamalla integroimismuuttujaksi $s=xt$.

\item
Laske seuraavat integraalit derivoimalla annettua funktiota $F(x),\ x>0$.
\begin{align*}
&\text{a)}\ \ I_n=\int_0^\infty t^n e^{-t}\,dt, \quad n\in\N, \quad 
                    F(x)=\int_0^\infty e^{-xt}\,dt \\
&\text{b)}\ \ I_n=\int_0^\infty \frac{1}{(t^2+1)^n}\,dt, \quad n=2,3,4, \quad
                    F(x)=\int_0^\infty \frac{1}{t^2+x^2}\,dt \\
&\text{c)}\ \ I_n(x)=\int_0^1 t^{x-1}(\ln t)^n\,dt,\quad n\in\N,\quad F(x)=\int_0^1 t^{x-1}\,dt
\end{align*}

\item \label{H-udif-2: perusteluja}
Mihin (yhden muuttujan) funktioihin Lauseen \ref{osittaisderivoinnin vaihtosääntö}
todistuksessa sovelletaan Differentiaalilaskun väliarvolausetta ja millä väleillä? --- Tarkista
väliarvolauseen soveltuvuus lauseen oletusten perusteella, ja tarkista myös päättelyn toimivuus
siinä tapauksessa, että $\Delta x_n=0$ tai $\Delta y_n=0$.

\item(*)
Tutki, mitkä Lauseen \ref{osittaisderivoinnin vaihtosääntö} oletuksista ovat voimassa, kun
$(x,y)=(0,0)$ ja
\[
f(x,y) = \begin{cases}
         0, &\text{kun}\ (x,y)=(0,0) \\[2mm] \dfrac{x^3y-xy^3}{x^2+y^2}\,, &\text{muulloin}
         \end{cases}
\]
Päteekö vaihtosääntö?  

\item (*) 
Näytä, että jos funktio $f(x,y)$ ja osittaisderivaatat $f_x$ ja $f_y$ ovat jatkuvia avoimessa
suorakulmiossa $A=(a,b)\times(c,d)$ ja $f_{xx}=f_{xy}=f_{yx}=f_{yy}=0$ $A$:ssa, niin
$f(x,y)=a+bx+cy,\ (x,y) \in A$ jollakin $(a,b,c)\in\R^3$ (eli $f$ on ensimmäisen asteen
polynomi).

\item (*)
Lentokone lentää ylöspäin pitkin avaruuskäyrää 
\[
S:\ y=x^2,\ z=\frac{1}{3}(2x+y^2),
\]
missä $z>0$ on korkeus maan pinnasta (yksiköt km). Ilman lämpötila lentoradan lähellä on
(yksikkö $^\circ$C)
\[
T(x,y,z)=-10(z^2-z+1)+(2x^2+3y)/(1+z^2).
\]
Ulkoilman lämpötilaa mitataan myös koneessa --- olkoon mittaustulos $T(t)$ hetkellä $t$ (min).
Eräällä hetkellä kone on pisteessä $(1,1,1)$ ja sen vauhti on $6$ km/min. Mikä on kyseisellä
hetkellä mittarilukemasta $T(t)$ laskettu ulkolämpötilan hetkellinen muuttumisnopeus $T'(t)$
(yksikkö $^\circ$C/min)?

\item (*)
Olkoon
\[
F(x,y)=\int_0^\infty \frac{e^{-xt}-e^{-yt}}{t}\,dt, \quad x>0,\ y>0.
\]
Laskemalla $\partial F/\partial x$ ja $\partial F/\partial y$ näytä, että $F(x,y)=\ln(y/x)$.

\item (*)
Laske $y(0)$ ja $y'(0)$, kun $y(x)$ määritellään pisteen $x=0$ ympäristössä kaavalla
\[
\int_0^1 \frac{e^{xyt}}{x+y+t}\,dt = \ln 2.
\]

\item (*) \label{H-udif-1: Gamman derivaatta}
Totea, että laskusäännön \eqref{d-int a} mukaan $\Gamma$-funktion
$\Gamma(x) =\int_0^\infty t^{x-1}e^{-t}\,dt$ derivaatta on
\[
\Gamma'(x)=\int_0^\infty t^{x-1}\ln t\,e^{-t}\,dt, \quad x>0.
\]
Varmista tämä derivoimissääntö suoraan derivaatan määritelmästä näyttämällä, että on olemassa
vain $x$:stä riippuva vakio $C(x)$ siten, että pätee
\[
\left|\,\frac{\Gamma(x+\Delta x)-\Gamma(x)}{\Delta x}
    -\int_0^\infty t^{x-1}\ln t\,e^{-t}\,dt\right|\,\le\,C(x)|\Delta x|,
\]           
kun $\,x>0\,$ ja $\,|\Delta x| \le x/2$. \kor{Vihje}: Todista ensin aputulos:
\[
\abs{e^x-1-x} \le \frac{1}{2}\,x^2 e^{\abs{x}}, \quad x\in\R.
\]

\end{enumerate} %Osittaisderivaatat
\section{Gradientti} \label{gradientti}
\alku
\index{gradientti|vahv}
\index{differentiaalioperaattori!d@gradientti (nabla $\nabla$)|vahv}

Palautettakoon mieliin Luvusta \ref{differentiaali}, että jos yhden muuttujan funktio $f$ on
derivoituva pisteessä $x$, niin pätee
\[
f(x+\Delta x)=f(x)+df(x,\Delta x) + o(\abs{\Delta x}),
\]
missä $df(x,\Delta x)= f'(x)\Delta x$ on $f$:n differentiaali. Vastaava tulos kahden muuttujan
funktiolle $f(x,y)$ on kehitelmä muotoa
\[
f(x+\Delta x,y+\Delta y)=f(x,y)+df(x,y,\Delta x,\Delta y)+\ord{h}, \quad 
                       h=\sqrt{(\Delta x)^2+(\Delta y)^2}.
\]
Sikäli kuin tämä on pätevä jossakin pisteen $(x,y)$ ympäristössä, sanotaan jälleen, että
$df(x,y,\Delta x,\Delta y)$ on $f$:n
\index{differentiaali}%
\kor{differentiaali} pisteessä $(x,y)$. Differentiaalin
lausekkeen johtamiseksi oletetaan, kuten ketjusäännössä edellä (Lause \ref{ketjusääntö}), että
$f$ on jatkuva pisteen $(x,y)$ ympäristössä $U_\delta(x,y)$ ja osittaisderivaatat $f_x$ ja $f_y$
ovat olemassa ko.\ ympäristössä  ja jatkuvia pisteessä $(x,y)$. Tällöin pätee
(vrt.\ Lauseen \ref{ketjusääntö} todistus) 
\begin{align*}
f(x+\Delta x,&y+\Delta y)-f(x,y) \\
             &= [f(x+\Delta x,y+\Delta y)-f(x,y+\Delta y)] + [f(x,y+\Delta y)-f(x,y)] \\
             &= f_x(\xi,y+\Delta y)\Delta x+f_y(x,\eta)\Delta y \\
             &= f_x(x,y)\Delta x + f_y(x,y)\Delta y + r(x,y,\Delta x,\Delta y),
\end{align*}
missä
\[
r(x,y,\Delta x,\Delta y) = [f_x(\xi,y+\Delta y)-f_x(x,y)]\Delta x 
                         + [f_y(x,\eta)-f_y(x,y)]\Delta y.
\]
Koska tässä on $\abs{\xi-x}\le\abs{\Delta x}$ ja $\abs{\eta-y}\le\abs{\Delta y}$ ja $f_x$ ja
$f_y$ ovat jatkuvia pisteessä $(x,y)$, niin $f_x(\xi,y+\Delta y)-f_x(x,y)=\ord{1}$ ja
$f_y(x,\eta)-f_y(x,y)=\ord{1}$, kun $h \kohti 0$. Siis 
(vrt.\ suuruusluokka-algebran säännöt Luvussa \ref{taylorin polynomien sovelluksia})
\[
r(x,y,\Delta x,\Delta y) = \ord{1}\Delta x + \ord{1}\Delta y = \ord{h}, \quad
                                           h=\sqrt{(\Delta x)^2+(\Delta y)^2}.
\]
On päätelty, että pätee kehitelmä
\begin{equation} \label{grad-2}
\boxed{\kehys\quad 
f(x+\Delta x,y+\Delta y)=f(x,y)+f_x(x,y)\Delta x+f_y(x,y)\Delta y+o(h). \quad}
\end{equation}
Etsitty differentiaalin lauseke (tehdyin oletuksin) on näin muodoin
\[ 
df(x,y,\Delta x,\Delta y)) = f_x(x,y)\Delta x + f_y(x,y)\Delta y.  
\]
\begin{Exa} \label{grad-ex1} Arvioi $f(1.01,1.98)$ differentiaalin
$df(1,2,0.01,-0.02)$ avulla, kun $f(x,y)=xy-2y^3$.
\end{Exa}
\ratk Tässä on $f_x(x,y) = y$ ja $f_y(x,y)=x-6y^2$, joten
\begin{align*}
f(1.01,1.98) &= f(1+0.01,2-0.02) \\
             &\approx f(1,2) + f_x(1,2)\cdot 0.01 + f_y(1,2)\cdot(-0.02) \\
             &= -14 + 2\cdot 0.01 + (-23)\cdot (-0.02) = -13.52.
\end{align*}
Tarkka arvo on $f(1.01,1.98)=-13.524984$. \loppu

Kun differentiaalikehitelmässä \eqref{grad-2} otetaan käyttöön vektorimerkintä
\[ 
\Delta \vec r = \Delta x\,\vec i + \Delta y\,\vec j, 
\]
ja käytetään myös muuttujien $x,y$ tilalla vektorimerkintää $\vec r = x\vec i + y\vec j$, niin
kehitelmä voidaan kirjoittaa muodossa
\begin{equation} \label{grad-3}
f(\vec r + \Delta \vec r) 
         = f(\vec r) + \nabla f(\vec r)\cdot\Delta\vec r + o(\abs{\Delta\vec r}),
\end{equation}
missä $\nabla f$ on $f$:n \kor{gradientti}, joka määritellään
\[
\boxed{\kehys\quad \nabla f(x,y)=f_x(x,y)\vec i+f_y(x,y)\vec j. \quad}
\]
Myös kolmen muuttujan funktiolle on johdettavissa differentiaalikehitelmä muotoa \eqref{grad-3}
samalla tavoin kuin edellä. Tällöin on 
$\Delta\vec r = \Delta x\,\vec i + \Delta y\,\vec j + \Delta z\,\vec k$, ja gradientti 
määritellään
\[
\boxed{\kehys\quad \nabla f(x,y,z)=f_x(x,y,z)\vec i+f_y(x,y,z)\vec j+f_z(x,y,x)\vec k. \quad}
\]

Ym.\ tulosten perusteella gradienttia voi pitää 'yleistettynä derivaattana'. Yhden muuttujan
derivaattaoperaattoria $\dif=d/dx$ vastaa siis kahden ja kolmen muuttujan tapauksessa 
vektorimuotoinen differentiaalioperaattori
\[
\nabla=\begin{cases}
\vec i\,\partial_x+\vec j\,\partial_y &(d=2), \\
\vec i\,\partial_x+\vec j\,\partial_y+\vec k\,\partial_z &(d=3).
\end{cases}
\]
Gradientille käytetään myös symbolia 'grad'. Lukutapoja ovat 'gradientti', 'grad' ja 'nabla'.
\begin{Exa} Funktion $f(x,y,z)=x^2+3y^2+xyz$ gradientti on
\begin{align*}
\nabla f &= (\vec i\,\partial_x + \vec j\,\partial_y + \vec k\,\partial_z)(x^2+3y^2+xyz) \\
         &= (2x+yz)\vec i +(6y+xz)\vec j+xy\vec k. \loppu
\end{align*}
\end{Exa}

\subsection{Monen muuttujan gradientti}
\index{gradientti!a@monen muuttujan|vahv}

Edellä johdetut differentiaalikehitelmät ovat helposti yleistettävissä koskemaan yleisempää
$n$ muuttujan funktioita. Tällöin on kätevintä siirtyä matriisialgebran merkintätapoihin, eli
indeksoidaan muuttujat,
\[
x \ext x_1,\quad y \ext x_2,\quad \ldots,
\]
ja kirjoitetaan $\vec r\,$:n paikalle pystyvektori $\mx\,$:
\[
\vec r \ext \begin{bmatrix} x_1 \\ x_2 \\ \vdots \\ x_n \end{bmatrix}=\mx,\quad 
\Delta\vec r \ext \Delta\mx.
\]
Näillä merkinnöillä differentiaalikehitelmä \eqref{grad-3} saa muodon
\[
f(\mx+\Delta\mx)
     =f(\mx)+\sum_{k=1}^n \frac{\partial f}{\partial x_k}(\mx)\Delta x_k+o(\abs{\Delta\mx}),
\]
missä $\abs{\cdot}$ tarkoittaa $\R^n$:n euklidista normia. Tulos voidaan kirjoittaa 
matriisialgebran kielellä muotoon
\begin{equation} \label{grad-4}
\boxed{\begin{aligned}
\ykehys\quad f(\mx+\Delta\mx)&=f(\mx)+\scp{\Nabla f(\mx)}{\Delta\mx}
                              +o(\abs{\Delta\mx}), \\[2mm]
             \Nabla f(\mx)   &=\left[\frac{\partial f}{\partial x_1}(\mx),
                                     \frac{\partial f}{\partial x_2}(\mx),\ldots,
                                     \frac{\partial f}{\partial x_n}(\mx)\right]^T.\Akehys\quad
       \end{aligned} }
\end{equation}

Kehitelmässä \eqref{grad-4} on termi
$df(\mx,\Delta\mx) = \scp{\Nabla f(\mx)}{\Delta\mx} = (\Nabla f(\mx))^T\Delta\mx$ 
\index{differentiaali}%
jälleen \kor{differentiaali}. Muuttujien lukumäärästä riippumatta voi differentiaalia käyttää 
funktion arvon muutosten arviointiin samaan tapaan kuin Esimerkissä \ref{grad-ex1} edellä.
\begin{Exa} Ristikkorakenteen (ks.\ Luku \ref{yhtälöryhmät}) kuormitus aiheuttaa yksittäisen 
sauvan päätepisteiden $P$ ja $Q$ siirtymät
\begin{align*}
&\vec u = \overrightarrow{PP'}=u_1\vec i+u_2\vec j+u_3\vec k, \\
&\vec v = \overrightarrow{QQ'}=v_1\vec i + v_2\vec j+v_3\vec k.
\end{align*}
Olettaen, että $\abs{\vec u},\abs{\vec v}\ll\abs{\vec a}$, $\vec a=\overrightarrow{QP}$,
määritä sauvan venymä likimäärin differentiaalin avulla.
\end{Exa}
\ratk Venymä on kuuden muuttujan funktio
\[
f(u_1,u_2,u_3,v_1,v_2,v_3) = \abs{\vec a+\vec u-\vec v}-\abs{\vec a}.
\]
Tässä on
\begin{align*}
\abs{\vec a+\vec u-\vec v\,}^2\,
             &=\,(\vec a+\vec u-\vec v)\cdot (\vec a+\vec u-\vec v) \\
             &=\,\abs{\vec a\,}^2+2\vec a\cdot(\vec u-\vec v)+\abs{\vec u-\vec v\,}^2,
\end{align*}
joten
\[
f(u_1,\ldots,v_3) 
 =\left[\abs{\vec a}^2+\sum_{k=1}^3 2a_k(u_k-v_k)+\sum_{k=1}^3 (u_k-v_k)^2\right]^{1/2}
                                                 -\ \abs{\vec a}.
\]
Tällöin on $f(\mo)=0$ ja
\[
\frac{\partial f}{\partial u_k}(\mo)
              =\frac{a_k}{\abs{\vec a}}\,,\quad \frac{\partial f}{\partial v_k}(\mo)
              =-\frac{a_k}{\abs{\vec a}}\,, \quad k=1,2,3,
\]
joten differentiaalikehitelmän \eqref{grad-4} mukaan
\[ 
f(u_1,\ldots,v_3) \approx \sum_{k=1}^3 \frac{a_k}{\abs{\vec a}}\,(u_k-v_k) 
                        = \underline{\underline{\vec t\cdot (\vec u-\vec v)}},
\]
missä $\vec t=\vec a/\abs{\vec a\,}$ on $\vec a$:n suuntainen yksikkövektori. \loppu

\subsection{Differentioituvuus}
\index{gradientti!b@differentioituvuus|vahv}

Kehitelmä \eqref{grad-4} on pätevä olettaen, että $f$ on jatkuva $\mx$:n ympäristössä
\[ 
U_\delta(\mx) = \{\mx' \in \R^n \mid |\mx'-\mx|<\delta\}, \quad \delta>0 
\]
ja lisäksi osittaisderivaatat $\partial f/\partial x_k,\ k=1 \ldots n\,$ ovat olemassa ko.\
ympäristössä ja jatkuvia pisteessä $\mx$. Tällaisista taustaoletuksista päästään eroon, kun
asetetaan seuraava yleisempi \kor{differentioituvuuden} ja samalla gradientin määritelmä.
\begin{Def} \label{differentioituvuus} \index{differentioituvuus|emph}
Funktio $f:D_f\kohti\R$, $D_f\subset\R^n$, on \kor{differentioituva} pisteessä $\mx\in D_f$, 
jos $f$ on määritelty pisteen $\mx$ jossakin ympäristössä ja on olemassa pystyvektori 
$\mathbf{h}\in\R^n$ siten, että pätee 
\[
f(\mx+\Delta\mx)= f(\mx)+\mathbf{h}^T\Delta\mx+o(\abs{\Delta\mx}), \quad 
                      \text{kun}\,\ \abs{\Delta\mx}\kohti 0.
\]
Tällöin $\mathbf{h}$ on $f$:n \kor{gradientti} pisteessä $\mx$, merkitään
$\mathbf{h}=\Nabla f(\mx)$.
\end{Def}
Jos $f$ on differentioituva, niin gradientti on laskettavissa osittaisderivaattojen avulla
kaavan \eqref{grad-4} mukaisesti:
\begin{Prop} Jos $f$ on differentioituva pisteessä $\mx \in \R^n$, niin osittaisderivaatat
$\partial f/\partial x_k,\ k=1\ldots n\,$ ovat olemassa pisteesä $\mx$, ja pätee
$[\Nabla f(\mx)]_k=\partial f(\mx)/\partial x_k,\ k=1 \ldots n$.
\end{Prop}
\tod Kun kehitelmässä \eqref{grad-4} asetetaan $\Delta\mx=\Delta x_k\me_k$ 
($\me_k=$ euklidinen yksikkövektori), niin kehitelmä voidaan kirjoittaa muotoon
\[
f(x_1,\ldots,x_{k-1},x_k+\Delta x_k,x_{k+1},\ldots,x_n) - f(x_1,\ldots,x_n) 
                  = [\Nabla f(\mx)]_k\Delta x_k+o(\abs{\Delta x_k}).
\]
Jakamalla $(\Delta x_k)$:lla ja antamalla $\Delta x_k\kohti 0$ seuraa
$[\Nabla f(\mx)]_k=\partial f(\mx)/\partial x_k$. \loppu

Yhteenvetona edellä esitetystä voidaan todeta, että funktion jatkuvuus pisteen $\mx$
ympäristössä, osittaisderivaattojen olemassaolo ko.\ ympäristössä ja jatkuvuus pisteessä $\mx$
rittävät yhdessä takaamaan differentioituvuuden; toisaalta differentioituvuus pisteessä $\mx$
takaa vain osittaisderivaattojen olemassaolon kyseisessä pisteessä, ei muualla. Vertaamalla
edellisen luvun tarkasteluihin nähdäänkin nyt, että differentioituvuus on usean muuttujan
derivoituvuuden luontevin määritelmä. Nimittäin ensinnäkin kehitelmän \eqref{grad-4}
perusteella on ilmeistä, että differentioituvuudesta seuraa välittömästi jatkuvuus ko.\
pisteessä, kuten yhden muuttujan tapauksessa (vrt.\ Luku \ref{derivaatta})). Toiseksi nähdään,
että myös osittaisderivoinnin ketjusääntöjä perusteltaessa voidaan ulkofunktiota koskevat
jatkuvuus- ym. oletukset korvata differentioituvuudella. Nimittäin jos tarkastellaan Lauseen
\ref{ketjusääntö} funktiota $F(t)=f(x(t),y(t))$, merkitään $x(t)=x$, $y(t)=y$,
$x(t+\Delta t)=x+\Delta x$, $y(t+\Delta t)=y+\Delta y$ ja oletetaan, että $f$ on 
differentioituva pisteessä $(x,y)$, niin kehitelmän \eqref{grad-2} perusteella
\[
\frac{F(t+\Delta t)-F(t)}{\Delta t} = f_x(x,y)\frac{\Delta x}{\Delta t}
                                    + f_y(x,y)\frac{\Delta y}{\Delta t} 
                                    + (\Delta t)^{-1}\ord{h}.
\]
Jos edelleen derivaatat $x'(t)$ ja $y'(t)$ ovat olemassa, niin tässä on
\[
h=\sqrt{(\Delta x)^2+(\Delta y)^2} = \sqrt{[x'(t)]^2+[y'(t)]^2}\Delta t + \ord{\abs{\Delta t}},
\]
joten
\[
(\Delta t)^{-1}\ord{h} = (\Delta t)^{-1}\ord{\abs{\Delta t}} \kohti 0, \quad 
                                                           \text{kun}\ \Delta t \kohti 0.
\]
Näin muodoin seuraa ketjusääntö $F'(t)=f_x(x,y)x'(t)+f_y(x,y)y'(t)$.

Yhden muuttujan derivoituvuuden ja useamman muuttujan differentioituvuuden vastaavuudesta 
kertoo myös seuraava Differentiaalilaskun väliarvolauseen yleistys. Todistus jätetään
harjoitustehtäväksi (Harj.teht.\,\ref{H-udif-3: differentioituvuus}c).
\begin{Lause} (\vahv{$\R^n$:n väliarvolause}) \label{väliarvolause-Rn}
\index{vzy@väliarvolauseet!d@$\R^n$:n|emph}
Olkoon $\ma,\mb\in\R^n,\ \ma\neq\mb$ ja olkoon $f(\mx)$ jatkuva pisteessä $\mx(t)=t\ma+(1-t)\mb$ 
jokaisella $t\in[0,1]$ ja differentioituva pisteessä $\mx(t)$ jokaisella $t\in(0,1)$. Tällöin
jollakin $\xi\in(0,1)$ pätee 
\[
f(\mb)-f(\ma)=\scp{\Nabla f\bigl(\mx(\xi)\bigr)}{\mb-\ma}.
\]
\end{Lause}

\subsection{Suunnatut derivaatat}
\index{osittaisderivaatta!f@suunnattu derivaatta|vahv}
\index{gradientti!c@suunnattu derivaatta|vahv}
\index{suunnattu derivaatta|vahv}

Gradienttia käytetään usean muuttujan funktioita tutkittaessa samaan tapaan kuin derivaattaa
yhden muuttujan tapauksessa. Gradientin avulla voidaan erityisesti laskea, millä tavoin
funktion arvot muuttuvat lähellä annettua pistettä $\mx$, kun ko.\ pisteestä siirrytään
jonkin $\R^n$:n yksikkövektorin osoittamaan suuntaan. Nimittäin jos $\me$ on tällainen vektori,
eli
\[ 
\me = (e_i), \quad \abs{\me}^2 = \sum_{i=1}^n e_i^2 = 1, 
\]
niin asettamalla $\Delta\mx = \Delta t\,\me$ kehitelmässä \eqref{grad-4} saadaan tulos
\begin{equation} \label{grad-6}
f(\mx+\Delta t\,\me)=f(\mx)+\partial_{\me}f(\mx)\Delta t+\ord{\abs{\Delta t}},
\end{equation}
missä 
\[
\boxed{\quad \partial_{\me}f(\mx)
          =\sum_{k=1}^n e_k\frac{\partial f}{\partial x_k}(\mx)=\scp{\Nabla f(\mx)}{\me}. \quad}
\]
Näin määritelty $\partial_{\me}f$ voidaan nimetä $f$:n \kor{suunnatuksi derivaataksi} suuntaan
$\me$, sillä määritelmän \eqref{grad-6} mukaan on
\[
\partial_{\me}f(\mx)=\lim_{\Delta t\kohti 0}\frac{f(\mx+\Delta t\,\me)-f(\mx)}{\Delta t} 
                    = g'_{\me}(0), \quad g_{\me}(t) = f(\mx + t\me). 
\]
Erityisesti jos valitaan $\me=\me_k$ (euklidinen yksikkövektori), niin 
$\partial_{\me}f = \partial f/\partial x_k$. Siis osittaisderivaatat ovat suunnattujen 
derivaattojen erikoistapauksia.

Kahden tai kolmen muuttujan tapauksessa voidaan kehitelmä \eqref{grad-6} kirjoittaa haluttaessa
muotoon
\[
f(\vec r+\Delta t\,\vec e\,)=f(\vec r\,)+\vec e\cdot\nabla f(\vec r\,)\Delta t+\ord{\abs{\Delta t}},
\]
jolloin suunnattu derivaatta on differentiaalioperaattori
\[
\partial_e = \vec e\cdot\nabla=\begin{cases}
\,e_x\partial_x+e_y\partial y &(d=2), \\
\,e_x\partial_x+e_y\partial y+e_z\partial z &(d=3). 
\end{cases}
\]

Jos monen muuttujan tapauksessa on $\Nabla f(\mx) \neq \boldsymbol{0}$, niin suunnaksi $\me$ 
voidaan valita $\me = \pm [\Nabla f(\mx)]/\abs{\Nabla f(\mx)}$, jolloin on vastaavasti 
$\partial_{\me}f(\mx) = \pm \abs{\Nabla f(\mx)}$. Koska toisaalta Cauchyn-Schwarzin epäyhtälön
perusteella on
\[ 
\abs{\partial_{\me}f(\mx)} =   \left|\,\sum_{k=1}^n e_k \frac{\partial f}{x_k}\,\right| 
                           \le \abs{\me}\abs{\Nabla f(\mx)} = \abs{\Nabla f(\mx)}, 
\]
niin on päätelty, että suunnatun derivaatan suurin arvo $=\abs{\Nabla f(\mx)}$ gradientin
suuntaan ja pienin arvo $=-\abs{\Nabla f(\mx)}$ vastakkaiseen suuntaan. Siis\,:
\[ \boxed{\begin{aligned}
\quad &\text{Funktio kasvaa voimakkaimmin gradientin suuntaan, vähenee}\quad\ykehys \\
      &\text{voimakkaimmin negatiivisen gradientin suuntaan, ja muuttuu} \\
      &\text{vähiten suuntiin $\me$, joille $\scp{\Nabla f}{\me}=0$}.\akehys
\end{aligned} } \] 
\begin{Exa} Laske funktion $f(x,y,z)=x^3-2x^2yz+4yz^2-2x-y+z$ derivaatta suuntaan 
$\vec e = (\vec i - 2\vec j - \vec k)/\sqrt{6}$ pisteessä $(1,-1,2)$ sekä määritä suunta
$\vec n$, johon funktio kasvaa nopeimmin ko.\ pisteessä.
\end{Exa}
\ratk Gradientti on
\begin{align*}
&\nabla f(x,y,z) = (3x^2-4xyz-2)\vec i + (-2x^2z+4z^2-1)\vec j + (-2x^2y+8yz+1)\vec k \\
&\impl\quad \nabla f(1,-1,2) = 9\vec i + 11\vec j - 13\vec k.
\end{align*}
Derivaatta suuntaan $\vec e$ on
\[ \vec e\cdot\nabla f(1,-1,2) = \frac{1}{\sqrt{6}}\,[1 \cdot 9 -2 \cdot 11 - 1 \cdot (-13)] 
                               = \underline{\underline{0}}. 
\]
Nopeimman kasvun suunta on
\begin{align*}
\vec n = \frac{\nabla f(1,-1,2)}{\abs{\nabla f(1,-1,2)}} 
      &= \frac{1}{\sqrt{371}}\,(9\vec i + 11\vec j -13\vec k) \\
      &\approx \underline{\underline{0.47\vec i + 0.57\vec j - 0.67\vec k}}. \loppu
\end{align*}
\begin{Exa} Kolmion kärjet ovat pisteissä $P_1=(1,0)$, $P_2=(0,2)$ ja $P_3=(2,3)$. Kolmion 
kutakin kärkeä liikutetaan sama (pieni) matka $\Delta s$ siten, että $P_1$ siirtyy suuntaan 
$\vec j$, $P_2$ suuntaan $-\vec i$ ja $P_3$ suuntaan $\vec e$. Miten suunta $\vec e$ on 
valittava, jotta kolmion pinta-ala \ a) muuttuisi mahdollisimman vähän, \ 
b) kasvaisi mahdollisimman paljon, \ c) pienenisi mahdollisimman paljon\,? 
\end{Exa}
\ratk Jos kolmion kärjet muutoksen jälkeen ovat $P_1=(x_1,y_1)$, $P_2=(x_2,y_2)$ ja 
$P_3=(x_3,y_3)$, niin pinta-ala on (vrt.\ Luku \ref{ristitulo})
\begin{align*}
f(\mx)=f(x_1,y_1,x_2,y_2,x_3,y_3) 
    &= \frac{1}{2}\begin{vmatrix} x_3-x_1 & y_3-y_1 \\ x_2-x_1 & y_2-y_1 \end{vmatrix} \\ 
    &= \frac{1}{2}\,(x_3-x_1)(y_2-y_1)-\frac{1}{2}\,(x_2-x_1)(y_3-y_1).
\end{align*}
Jos $\,\vec e = \cos\varphi\vec i + \sin\varphi\vec j$, niin kärkien siirtymää kuvaa vektori
\[
\Delta \mx = \Delta s\,[0,1,-1,0,\cos\varphi,\sin\varphi]^T.
\]
Funktion $f$ gradientti on
\begin{align*}
           &\Nabla f(\mx) = \frac{1}{2}\,[y_3-y_2,x_2-x_3,y_1-y_3,x_3-x_1,y_2-y_1,x_1-x_2]^T \\
\impl\quad &\Nabla f(1,0,0,2,2,3) = \frac{1}{2}\,[1,-2,-3,1,2,1]^T.
\end{align*}
Differentiaalin avulla arvioitu pinta-alan muutos on siis
\[
\Delta f \approx \scp{\Nabla f(1,0,0,2,2,3)}{\Delta\mx} 
         = \frac{\Delta s}{2}\,(1+2\cos\varphi+\sin\varphi).
\]
Pinta-ala muuttuu vähiten, kun $1+2\cos\varphi+\sin\varphi=0$, eli kun joko $\vec e=-\vec j$
($\varphi=3\pi/2$) tai $\vec e=(-4\vec i+3\vec j\,)/5$. Pinta-ala muuttuu eniten, kun 
\[
D_\varphi(1+2\cos\varphi+\sin\varphi)=-2\sin\varphi+\cos\varphi=0,
\]
eli kun $\vec e=\pm(2\vec i + \vec j\,)/\sqrt{5}$. Näihin suuntiin on vastaavasti
$\Delta f \approx \frac{\Delta s}{2}(1\pm\sqrt{5})$.

Vastaus:
\[
\text{a)}\,\ \vec e = -\vec j\,\ \text{tai}\,\ \vec e = \frac{1}{5}(-4\vec i+3\vec j), \,\
\text{b)}\,\ \vec e = \frac{1}{\sqrt{5}}(2\vec i + \vec j), \,\
\text{c)}\,\ \vec e = -\frac{1}{\sqrt{5}}(2\vec i + \vec j). \loppu
\]

\subsection{Tasa-arvopinnat}
\index{tasa-arvopinta|vahv}

Samaan tapaan kuin derivaatan avulla voidaan määrätä käyrän tangentti ja normaali 
(vrt.\ Luku \ref{derivaatta geometriassa}), voidaan osittaisderivaattojen avulla määrätä 
\index{tangenttitaso} \index{normaali(vektori)!c@pinnan}%
\kor{pinnan tangenttitaso} ja tämän tason normaali, eli \kor{pinnan normaali}
(suora tai vektori). Jos pinnan $S\subset\Ekolme$ yhtälö on 
\[ 
S: \ F(x,y,z)=0, 
\]
niin pinnan normaali (ja normaalin avulla tangenttitaso) annetussa pisteessä voidaan laskea 
nopeimmin ajattelemalla, että pinta on funktion $F$ \pain{tasa-arvo}p\pain{inta}. On ilmeistä,
että yksikkövektori $\vec t$ on $F$:n tasa-arvopinnan tangenttivektori pisteessä
$(x_0,y_0,z_0)$ täsmälleen, kun $F$:n suunnattu derivaatta $\vec t\,$:n suuntaan häviää ko.\
pisteessä, eli kun $\vec t \cdot \nabla F(x_0,y_0,z_0)=0$. Tämän perusteella on
$\vec n = \nabla F(x_0,y_0,z_0)$ pinnan normaalivektori ko.\ pisteessä. Samaan tulokseen
tullaan, jos tarkasteltava pinta on pisteen $(x_0,y_0,z_0)$ kautta kulkeva $F$:n
tasa-arvopinta, jonka yhtälö on $\,F(x,y,z) = c = F(x_0,y_0,z_0)$. On siis päätelty, että
\[ \boxed{ \begin{aligned}
           \ykehys\quad &\vec n = \nabla F(x_0,y_0,x_0)\ 
                            \text{on funktion}\ F\ \text{tasa-arvopinnan} \quad \\
                        &\text{normaalivektori pisteessä}\ (x_0,y_0,z_0). \akehys
           \end{aligned} } 
\]
\begin{Exa} Määritä pinnan
\[ 
S:\ F(x,y,z) = x^2+y^2+z^2-xy-yz-xz = 1. 
\]
tangenttitason yhtälö pisteessä $(1,1,2)$.
\end{Exa}
\ratk Funktion $F$ gradientti on
\[ 
\nabla F(x,y,z) = (2x-y-z)\vec i + (2y-x-z)\vec j + (2z-y-x)\vec k, 
\]
joten pinnan normaalivektori pisteessä $(1,1,2)$ on
\[ 
\vec n = \nabla F(1,1,2) = -\vec i - \vec j + 2\vec k. 
\]
Tangenttitason yhtälö pisteessä $(1,1,2)\vastaa\vec r_0$ on
\[
T:\ (\vec r-\vec r_0)\cdot\vec n = 0\ \ekv\ \underline{\underline{x+y-2z+2=0}}. \loppu
\]
\begin{Exa} \label{udif-3: pintaesim} Jos pinnan yhtälö on annettu muodossa $S:\ z=f(x,y)$,
niin kirjoittamalla yhtälö muotoon $F(x,y,z)=z-f(x,y)=0$ saadaan normaalivektoriksi pisteessä
$(x,y,z) \in S\,$:
\[
\vec n = \nabla F(x,y,z) = -f_x(x,y)\vec i-f_y(x,y)\vec j+\vec k. \loppu
\]
\end{Exa}
\begin{Exa} Määritä käyrän
\[
\begin{cases} \,x^2-y^2+z^2=1, \\ \,x^2+y^2-2z^2=0 \end{cases}
\]
tangenttivektori pisteessä $(1,1,1)$. \end{Exa}
\ratk Kyseessä on kahden pinnan leikkauskäyrä. Pinnat ovat funktioiden $F_1(x,y,z)=x^2-y^2+z^2$
ja $F_2(x,y,z)=x^2+y^2-2z^2$ tasa-arvopintoja, joiden normaalivektorit pintojen yhteisessä 
pisteessä $(x,y,z)=(1,1,1)$ ovat
\begin{align*}
\vec n_1\,&=\,\nabla F_1(x,y,z)\,
           =\,2x\,\vec i - 2y\,\vec j + 2z\,\vec k\,=\,2(\vec i-\vec j+\vec k), \\
\vec n_2\,&=\,\nabla F_2(x,y,z)\,
           =\,2x\,\vec i + 2y\,\vec j - 4z\,\vec k\,=\,2(\vec i+\vec j-2\vec k).
\end{align*}
Kysytty tangenttivektori $\vec t$ on näitä vastaan kohtisuora, eli esimerkiksi
\[
\vec t = \frac{1}{4}\,\vec n_1\times\vec n_2 
       = \underline{\underline{\vec i+3\vec j+2\vec k}}. \loppu
\]

\subsection{Parametrisen pinnan normaali}
\index{normaali(vektori)!c@pinnan|vahv}
\index{parametrinen pinta|vahv}

Tarkastellaan $\R^3$:n parametrisoitua pintaa (vrt.\ Luku \ref{parametriset käyrät})
\[
S:\ \vec r=\vec r(u,v)\ \ekv\ 
                        \begin{cases} \,x=x(u,v),\\ \,y=y(u,v), \\ \,z=z(u,v). \end{cases}
\]
Jos oletetaan, että tässä funktiot $x$, $y$ ja $z$ ovat differentioituvia pisteessä $(u,v)$,
niin on ilmeistä, että vektorit
\begin{align*}
\vec t_u &= \lim_{\Delta u\kohti 0} \frac{\vec r(u+\Delta u,v)-\vec r(u,v)}{\Delta u}
          =\frac{\partial\vec r}{\partial u}(u,v), \\
\vec t_v &= \lim_{\Delta v\kohti 0} \frac{\vec r(u,v+\Delta v)-\vec r(u,v)}{\Delta v}
          =\frac{\partial\vec r}{\partial v}(u,v)
\end{align*}
\index{tangenttivektori (pinnan)}%
ovat pinnan tangenttitason suuntaisia eli \kor{pinnan tangenttivektoreita} pisteessä 
$P\vastaa\vec r(u,v)$. Parametrisaatiolta on sopivaa odottaa, että nämä vektorit
ovat lineaarisesti riippumattomat, jolloin pinnan normaalivektori pisteessä $P$ on
\[
\boxed{\quad\kehys 
  \vec n=\frac{\partial \vec r}{\partial u}\times\frac{\partial \vec r}{\partial v} \quad
         \text{(parametrisen pinnan normaali)}. \quad}
\]
\begin{Exa} (Vrt.\ Esimerkki \ref{udif-3: pintaesim}.) Jos pinnan yhtälö on $S:\ z=f(x,y)$,
niin parametrisoidusta esityksestä
\[ 
S:\ \vec r(x,y)\,=\,x\,\vec i+y\,\vec j+f(x,y)\,\vec k
\]
saadaan $S$:n tangentti- ja normaalivektoreiksi pisteessä $(x,y,f(x,y))\,$:
\begin{align*}
\vec t_x\,&=\,\partial_x\vec r(x,y)\,=\,\vec i+f_x(x,y)\,\vec k, \\
\vec t_y\,&=\,\partial_y\vec r(x,y)\,=\,\vec j+f_y(x,y)\,\vec k, \\
\vec n\   &=\,\vec t_x \times \vec t_y\,=\,-f_x(x,y)\,\vec i -f_y(x,y)\,\vec j+\vec k. \loppu
\end{align*}
\end{Exa}
\begin{Exa} (Vrt.\ Esimerkki \ref{parametriset käyrät}:\,\ref{jäähdytystorni}.) Määritä
parametrisen pinnan
\[
S:\ \begin{cases} \,x=(2-2v)\cos u-v\sin u, \\ \,y=v\cos u+(2-2v)\sin u, \\ \,z=3v \end{cases}
\]
normaali pisteessä $(x,y,z)=(2,0,0)$.
\end{Exa}
\ratk Koska $(2,0,0)\vastaa \vec r=2\vec i=\vec r(0,0)$, niin lasketaan
\begin{align*}
\vec t_u(0,0) &\,=\, \frac{\partial\vec r}{\partial u}(0,0)
               \,=\, \frac{\partial x}{\partial u}(0,0)\vec i
                     +\frac{\partial y}{\partial u}(0,0)\vec j
                     +\frac{\partial z}{\partial u}(0,0)\vec k=2\vec j, \\
\vec t_v(0,0) &\,=\, \frac{\partial\vec r}{\partial v}(0,0)
               \,=\, -2\vec i +\vec j +3\vec k \\[2mm]
\impl\ \vec n &\,=\, \vec t_u\times\vec t_v
               \,=\,6\vec i +4\vec k 
               \,\uparrow\uparrow\, \underline{\underline{3\vec i+2\vec k}}. \loppu
\end{align*}

\Harj
\begin{enumerate}

\item
Laske funktion arvolle likiarvo annetussa pisteessä käyttäen differentiaalia.
\begin{align*}
&\text{a)}\ \ x^2y^3, \quad (3.1,0.9) \qquad\qquad\qquad\qquad\
 \text{b)}\ \ \Arctan(y/x), \quad (3.01,2.98) \\[0.5mm]
&\text{c)}\ \ 24\,(x^2+xy+y^2)^{-1}, \quad (2.1,1.9) \qquad
 \text{d)}\ \ \sin(\pi xy+\ln y), \quad (0.01,1.05) \\
&\text{e)}\ \ \sqrt{x+2y+3z}\,, \quad (1.9,1.8,1.1) \qquad\,\
 \text{f)}\ \ xe^{y+z^2}, \quad (2.05,-3.92,1.97)
\end{align*}

\item
Suorakulmaisen särmiön muotoisen laatikon särmät on mitattu $1\%$:n tarkkuudella. Arvioi,
kuinka suuri vaikutus ($\%$) mittausvirheillä on enintään mittausten perusteella
laskettuun laatikon a) tilavuuteen, b) pohjan pinta-alaan, c) lävistäjään.

\item
Pisteen $A$ etäisyys tarkkailupisteestä $B$ on pyritty selvittämään kolmiomittauksella
valitsemalla toinen tarkkailupiste $C$ ja mittaamalla kulmat $ABC$ ja $ACB$. Mittaustulokset
ovat: $\kulma ACB=45 \pm 0.1\aste$, $\kulma ABC=105 \pm 0.1\aste$ ja pisteiden $B$ ja $C$
välimatkaksi on mitattu $265 \pm 0.5$ m. Laske kysytty etäisyys $d$ ja arvioi virhe $\Delta d$
differentiaalin avulla (maksimivirhe).

\item
Tornin korkeus mitataan kulmamittauksella kahdesta pisteestä $A$ ja $B$, jotka ovat
samassa suunnassa tornista katsoen. Mitatut kulmat ovat $50 \pm 1\aste$ ja $35 \pm 1\aste$.
Lisäksi pisteiden $A$ ja $B$ välimatkaksi on mitattu $100 \pm 1$ m. Mikä on tornin
korkeuden laskettu arvo ja kuinka suuri voi mittausvirhe korkeintaan olla differentiaalin
perusteella arvioiden?

\item
Suunnistaja, joka on kartan origossa, on matkalla rastille, joka on kartan mukaan suunnassa
$\varphi=30\aste$ (polaarikoordinaatti). Sateen ja hikoilun vuoksi karttapaperi on kutistunut
$x$-suunnassa $1\%$ ja $y$-suunnassa $0.6\%$. Arvioi, kuinka monta $\%$ suunnistajan on syytä
korjata rastin suuntaa ja kartalta mitattua eäisyyttä päästäkseen oikeisiin arvoihin.

\item Nelikulmion pinta-ala on funktio $f(\mx) = f(x_1, y_1, x_2, y_2, x_3, y_3, x_4, y_4)$,
missä $(x_i, y_i)$, $i = 1,\ldots,4$ ovat nelikulmion kärkipisteet. Laske $\nabla f$ 
pisteessä ${\bf x} = (0,0,1,0,1,1,0,1)$ ja differentiaalin avulla likimäärin sen nelikulmion
pinta-ala, jonka kärjet ovat pisteissä $\,(0.03, -0.05)$, $(1.02,-0.02)$, $(0.96, 1.01)$, 
$(-0.01, 0.98)$.

\item \label{H-udif-3: differentioituvuus}
a) Olkoon $f(0,0)=0$ ja $f(x,y)=xy\sin(1/\sqrt{x^2+y^2}\,)$ muulloin. Näytä, että $f$ on
origossa differentioituva mutta $f_x$ ja $f_y$ eivät ole origossa jatkuvia.\vspace{1mm}\newline
b) Olkoon $f(x,y)=0$ kun $x=0$ tai $y=0$ ja $f(x,y)=\abs{xy}^\alpha(x^2+y^2)^\beta$ muulloin 
($\alpha,\beta\in\R$). Määritä $A\subset\R^2$ ja $B\subset\R^2$ siten, että $f$ on
differentioituva origossa täsmälleen kun $(\alpha,\beta) \in A$ ja differentioituva jokaisessa
pisteessä $(x,y)\in\R^2$ täsmälleen kun $(\alpha,\beta) \in B$. \vspace{1mm}\newline
c) Todista Lause \ref{väliarvolause-Rn} tarkastelemalla funktiota
$F(t)=f\bigl(t\ma+(1-t)\mb\bigr)$.

\item
Laske suunnattu derivaatta annetussa pisteessä sekä annettuun että funktion nopeimman kasvun
suuntaan: \vspace{1mm}\newline
a) \ $e^{xy+y},\ \ (0,0),\ \ \vec i+\vec j$ \vspace{0.8mm}\newline
b) \ $\sin(\pi xy)\cos(\pi y^2),\ \ (1,2),\ \ \vec i-2\vec j$ \vspace{0.3mm}\newline
c) \ $xy^2z^3,\ \ (-3,2,1),\ \ 6\vec i-2\vec j+3\vec k$ \vspace{0.8mm}\newline
d) \ $ xy^2+yz^3,\ \ (3,-2,-1),\ \ \vec 2i+\vec 2j-3\vec k$

\item
Seuraavassa on annettu funktio $f(x,y)$ ja piste $(x_0,y_0)$. Määritä kussakin tapauksessa \,
(i) gradientti $\nabla f(x_0,y_0)$, \, (ii) käyrän $f(x,y)=f(x_0,y_0)$ tangentin yhtälö
pisteessä $(x_0,y_0)$, \, (iii) pinnan $z=f(x,y)$ tangenttitason yhtälö pisteessä
$(x_0,y_0,z_0),\ z_0=f(x_0,y_0)$. \vspace{2mm}\newline
a) \ $ f(x,y)=x^2-y^2,\,\ (2,-1)$ \quad
b) \ $f(x,y)=x^2/(x+y),\,\ (1,1)$ \vspace{1mm}\newline
c) \ $f(x,y)=e^{x^2y+x^2},\,\ (2,0)$ \qquad
d) \ $f(x,y)=\ln(x^2+y^2),\,\ (1,-2)$

\item 
Määritä seuraavien pintojen tangenttitaso annetussa pisteessä $P$ sekä ne pintojen pisteet,
joissa tangenttitaso on $xy$-tason suuntainen. \vspace{2mm}\newline
a) \ $x^2-2xy-y^2+z^2-2x-2y-3z= 0, \quad P=(0,0,0)$ \vspace{1mm}\newline
b) \ $xy+yz-2xz+z^3=1, \quad P=(1,1,1)$ \vspace{1mm}\newline
c) \ $e^{x+2y-z}=x+y^2+2z+1, \quad P=(0,0,0)$

\item
Määritä seuraavien funktioiden tasa-arvopinnan normaali ja ko.\ pinnan tangenttitason yhtälö
annetussa pisteessä. \vspace{2mm}\newline
a) \ $f(x,y,z)=x^2y+y^2z+z^2x,\ \ (1,-1,1)$ \vspace{1mm}\newline
b) \ $f(x,y,z)=\cos(x+2y+3z),\ \ (\pi/2,\pi,\pi)$ \vspace{1mm}\newline
c) \ $f(x,y,z)=ye^{-x^2},\ \ (0,1,1)$ \vspace{1mm}\newline
d) \ $f(x,y,z)=\sin x\sin 2y\sin 3z\sin(3x-2y+z),\ \ (\pi/6,\pi/6,\pi/6)$

\item
Määritä avaruuskäyrän tangenttivektori annetussa pisteessä $P\,$:
\begin{align*}
&\text{a)}\ \ \begin{cases} 
              \,x^2-y^2+2z^2=13, \\ \,3x+2y-z=3,
              \end{cases} \quad P=(-2,3,-3) \\
&\text{b)}\ \ \begin{cases} 
              \,x^3-xyz^3+yz^5=1, \\ \,xy^2+yz+2xz^3=0,
              \end{cases} \quad P=(1,2,-1)
\end{align*}

%\item (*) \label{H-DD-2: tulon ja osamäärän gradientti}
%a) Näytä, että jos $n$ muuttujan funktiot $f$ ja $g$ ovat differentioituvia pisteessä 
%$\mx\in\R^n$, niin myös $u=fg$ on differentioituva $\mx$:ssä ja pätee tulon derivoimissääntö 
%$\Nabla u(\mx)=(g\Nabla f+f\Nabla g)(\mx)$. \ b) Näytä, että lisäehdolla $g(\mx) \neq 0$
%myös $v=f/g$ on differentioituvua $\mx$:ssä. Mikä on $\Nabla v(\mx)$:n lauseke?

\item (*)
Kepin päät ovat pisteissä $A=(1,1,-2)$ ja $B=(-1,-2,4)$. Keppiä liikutetaan niin, että sen päät
liukuvat pitkin käyriä 
\[
S_1: \begin{cases} \,x^3+y^2+z=0, \\ \,x+y^2+z^3=-5 \end{cases} \ \text{ja} \quad
S_2: \begin{cases} \,x^3+y^2+z=7, \\ \,x+y^2+z^3=67 \end{cases}
\]
(kepin pituus säilyy). Olkoon kepin päät pienen siirron jälkeen pisteissä $A' \in S_1$ ja
$B' \in S_2$. Laske suhteen $|AA'|/|BB'|$ raja-arvo, kun $|AA'| \kohti 0$.
 
\item (*)
Suora, jonka eräs piste on $A$, liikkuu siten, että suora on koko ajan $xy$-tason suuntainen,
piste $A$ liikkuu vakionopeudella $z$-akselia pitkin ylöspäin, ja samalla suora pyörii
vakiokulmanopeudella $z$-akselin ympäri vastapäivään, positiivisen $z$-akselin suunnasta 
katsoen. Suora on $x$ akselin suuntainen täsmälleen kun $A$ on jossakin pisteistä 
$(0,0,3n),\ n\in\Z$. Määritä suoran liikkuessaan piirtämän (viivoitin)pinnan parametrinen
esitys sekä ko. pinnan normaali ja tangenttitaso siinä pinnan ja suoran $x=1,\, y=2$ 
leikkauspisteessä, joka on lähinnä $xy$-tasoa.

\item (*) \index{zzb@\nim!Jyrkin lasku}
(Jyrkin lasku) Tasosta kohoaa tunturi. Sektorissa $\,A:\,0 \le y \le 2x$ on tunturin korkeus
tasosta (yksiköt km)
\[
h(x,y) = e^{(-x^2-2xy+y^2)/100}, \quad (x,y) \in A.
\]
Laskettelija Jyrki lähtee pisteestä $(3,6)$ ja laskee sukset suunnattuina joka hetki niin, että
lasku on jyrkin mahdollinen. Laske Jyrkin laskureitti!

\end{enumerate} %Gradientti
\section{Vektorikentät: Divergenssi ja roottori} \label{divergenssi ja roottori}
\sectionmark{Divergenssi ja roottori}
\alku
\index{osittaisdifferentiaalioperaattori!vektorioperaattorit|vahv}

Tässä luvussa tarkastellaan funktioita, jotka riippuvat kahdesta tai kolmesta fysikaalisesta
p\pain{aikka}muuttujasta ($x,y$ tai $x,y,z$), mahdollisesti lisäksi fysikaalisesta
\pain{aika}muuttujasta $t$. Funktiot voivat olla joko \pain{skalaarifunktioita}, eli
reaaliarvoisia (tai skalaariarvoisia, kirjaimellisesti 'skaalaajafunktioita') tai
vektoriarvoisia, eli
\index{vektorikenttä} \index{funktio A!i@vektorikenttä}%
\kor{vektorikenttiä} (engl.\ vector field). Jatkossa ajatellaan ainoastaan fysikaalisia
vektorikenttiä, joissa on joko kaksi (tasokenttä) tai kolme (avaruuskenttä) komponenttia.
Skalaarifunktiot, joita
\index{skalaarikenttä}%
kutsutaan myös \kor{skalaarikentiksi}, voivat siis olla tyyppiä $f(x,y)$, $f(x,y,z)$,
$f(x,y,t)$ tai $f(x,y,z,t)$ (joskus aikamuuttuja halutaan kirjoittaa ensimmäiseksi,
esim.\ $f(t,x,y,z)$) ja vektorikentät esim.\ tyyppiä
\[ 
\vec F(x,y,z,t) = F_1(x,y,z,t)\,\vec i + F_2(x,y,z,t)\,\vec j + F_3(x,y,z,t)\,\vec k. 
\]
Fysikaalisia vektorikenttiä ovat esim. virtaavan nesteen, kaasun, yms.\ nopeusvektoreiden 
muodostama \pain{virtauskenttä}, (esim.\ gravitaatioon liittyvä) \pain{voimakenttä}, 
sähkömagneettisiin ilmiöihin liittyvät \pain{sähkökenttä} ja \pain{ma}g\pain{neettikenttä}, ym.
Tällaiset fysikaaliset kentät ovat joko \pain{staattisia} tai \pain{d}y\pain{naamisia}, riippuen
siitä onko muuttujien joukossa myös aika ($t$) vai ei. Vektorikenttä kuvataan usein graafisesti
'nuolikenttänä' (kuvassa virtauskenttä).

\begin{figure}[H]
\setlength{\unitlength}{1cm}
\begin{center}
\begin{picture}(12,5.5)
\path(0,0.5)(5,0.5)(5,2.5)(7,2.5)(7,0.5)(12,0.5)
\path(0,5)(12,5)
\Thicklines
\multiput(0,1)(0,0.5){8}{\vector(1,0){1}}
\multiput(1.5,1)(0,0.5){8}{\vector(1,0){1}}
\multiput(3,3.5)(0,0.5){3}{\vector(1,0){1}}
\multiput(3,1)(0,0.5){3}{\vector(4,1){0.5}}
\multiput(3,2.5)(0,0.5){2}{\vector(4,1){1}}
\multiput(3.8,1.2)(0,0.5){2}{\vector(1,1){0.5}}
\put(3.8,2.2){\vector(2,1){0.5}}
\multiput(5,3)(0,0.5){4}{\vector(1,0){2}}
\multiput(8,3.5)(0,0.5){3}{\vector(1,0){1}}
\multiput(8,2.5)(0,0.5){2}{\vector(4,-1){1}}
\put(8,2){\vector(2,-1){0.5}}
\put(8.8,1.6){\vector(1,-4){0.15}}
\put(8.6,0.8){\vector(-4,-1){0.5}}
\put(8,1){\vector(-1,3){0.15}}
\multiput(9.5,2.5)(0,0.5){5}{\vector(1,0){1}}
\put(9.5,2){\vector(4,-1){1}}
\multiput(11,1)(0,0.5){8}{\vector(1,0){1}}
\path(5,0.5)(7,0.5)
\multiput(1,0.5)(1,0){11}{\line(-1,-1){0.25}}
\end{picture}
%\caption{Esimerkki virtauskentästä}
\end{center}
\end{figure}
\index{differentiaalioperaattori!c@osittaisderivoinnin}%
Vektorikenttään kohdistuvina differentiaalioperaattorit 
$\,\partial_x,\partial_y,\partial_z,\partial_t\,$ ymmärretään samalla tavoin kuin skalaarin ja
vektorin kertolaskussa, eli operaattorit kohdistuvat vektorikentän jokaiseen komponenttiin. 
Esim.
\[ 
\partial_x\vec F(x,y,z,t) 
        = \pder{F_1}{x}\,\vec i + \pder{F_2}{x}\,\vec j + \pder{F_3}{x}\,\vec k.
\]

\subsection{Gradienttikentät}
\index{gradienttikenttä|vahv}

Jos $f$ on skalaarifunktio (skalaarikenttä) edellä mainittua tyyppiä, niin funktion
gradientti ymmärretään vain paikkamuuttujiin $x,y$ tai $x,y,z$ kohdistuvana. Gradientti toimii
tällöin differentiaalioperaattorina, joka muuntaa skalaarifunktion vektoriarvoiseksi funktioksi
eli vektorikentäksi. Monet fysikaaliset vektorikentät ovat tällaisia \kor{gradienttikenttiä}.
Jos $\vec F$ on gradienttikenttä ja
\[
\vec F=-\nabla u
\]
(miinusmerkki fysikaalisista mukavuussyistä), niin sanotaan, että $u$ on kentän $\vec F$ 
\index{potentiaali (vektorikentän)} \index{skalaaripotentiaali}%
\kor{(skalaari)potentiaali}.
\begin{Exa} Fysikaalisia esimerkkejä potentiaaleista ja gradienttikentistä ovat esimerkiksi\,:

\vspace{3mm}
\begin{tabular}{ll}
\pain{Potentiaali} & \pain{Gradienttikenttä} \\ \\
$u=$ gravitaatiopotentiaali \hspace{2cm} & $\vec G=-\nabla u=$ gravitaatiokenttä \\ \\
$u=$ sähköpotentiaali & $\vec E=-\nabla u=$ sähkökenttä \\ \\
$u=$ lämpötila & $\vec J=-\lambda\nabla u=$ lämpövirran tiheys
\end{tabular}
\vspace{2mm}

\index{Fourierb@Fourier'n laki}%
Yhteyttä $\vec J=-\lambda\nabla u$ sanotaan lämpöopissa \pain{Fourier'n} \pain{laiksi}. Sen 
mukaan siis lämpöenergia virtaa lämpötilan negatiivisen gradientin $-\nabla u$ suuntaan ja 
virtatiheys (yksikkö W/m$^2$) on gradientin itseisarvoon verrannollinen. 
Verrannollisuuskerrointa sanotaan materiaalin \pain{lämmön}j\pain{ohtavuudeksi}. \loppu
\end{Exa}

\subsection{Divergenssi ja roottori. Laplacen operaattori}
\index{differentiaalioperaattori!e@divergenssi $\nabla\cdot$|vahv}
\index{differentiaalioperaattori!f@roottori $\nabla\times$|vahv}
\index{differentiaalioperaattori!g@Laplacen operaattori|vahv} 
\index{divergenssi|vahv} \index{roottori|vahv} \index{Laplacen operaattori|vahv}

Differentiaalioperaattori $\nabla=\partial_x\vec i+\partial_y\vec i+\partial_z\vec k$ voidaan 
myös ymmärtää operaattorina, jonka kohteena ovat vektoriarvoiset funktiot eli vektorikentät. 
Tällöin operaattori $\nabla$ operoi kohdefunktioon $\vec F$ joko pistetulon (skalaaritulon) tai
ristitulon (vektoritulon) välityksellä. Kun kohdefunktioksi oletetaan
\[
\vec F(x,y,z)=F_1(x,y,z)\vec i + F_2(x,y,z)\vec j + F_3(x,y,z)\vec k,
\]
niin tulevat näin määritellyksi kentän $\vec F$ \kor{divergenssi} (engl. divergence)
\[
\boxed{\kehys\quad \nabla\cdot\vec F
           =\partial_x F_1+\partial_y F_2+\partial_z F_3 \quad\text{(divergenssi)}\quad}
\]
ja \kor{roottori} (engl.\ curl)
\[
\boxed{\quad \nabla\times\vec F=\begin{vmatrix}
\vec i & \vec j & \vec k \\
\partial_x & \partial_y & \partial_z \\
F_1 & F_2 & F_3
\end{vmatrix} \quad\text{(roottori)}.\quad}
\]
Roottorin laskukaavan purettu muoto on
\[ \nabla \times \vec F = (\partial_y F_3 - \partial_z F_2)\vec i 
                        - (\partial_x F_3 - \partial_z F_1)\vec j
                        + (\partial_x F_2 - \partial_y F_1)\vec k. 
\]
Divergenssiä ja roottoria merkitään myös symboleilla div ja rot (tai $\Vect{\text{rot}}$). 
Tavallisia lukutapoja ovat 'nabla piste' ja 'nabla risti', vastaten gradientin $\nabla f$ 
lukutapaa 'nabla f'.
\begin{Exa} Funktion $\vec F(x,y,z)=x^2\vec i+y^2z\vec j-(2xz+yz^2)\vec k$ divergenssi ja 
roottori ovat
\begin{align*}
\nabla\cdot\vec F  &= 2x+2yz-2x-2yz=0, \\
\nabla\times\vec F &= (\partial_y F_3-\partial_z F_2)\vec i
                     +(\partial_z F_1-\partial_x F_3)\vec j
                     +(\partial_x F_2-\partial_y F_1)\vec k \\
                   &= -(y^2+z^2)\vec i+2z\vec j. \loppu
\end{align*}
\end{Exa}
Piste- ja ristitulojen kautta tulevat määritellyksi myös mm.\ operaattoritulot 
\[
(1)\ \ \nabla\cdot\nabla \qquad (2)\ \ \nabla(\nabla\cdot\,) \qquad
(3)\ \ \nabla\times(\nabla\times\,) \qquad (4)\ \ \nabla\times\nabla
\]
Näistä viimeinen on nollaoperaattori:
\[
\boxed{\kehys\quad \nabla\times\nabla=\vec 0. \quad }
\]
Nollaoperaattori on myös $\nabla\cdot(\nabla\times\,)$, ts.\ pätee (vrt.\ Luku \ref{ristitulo})
\[
\boxed{\kehys\quad \nabla\cdot\nabla\times\vec F = (\nabla\times\nabla)\cdot\vec F = 0. \quad}
\]
Operaattorien (1), (2) ja (3) välillä on yhteys
(Harj.teht.\,\ref{H-udif-4: nablaamissääntöjä}f --- vrt.\ Luku \ref{ristitulo})
\[
\boxed{\kehys\quad \nabla\times(\nabla\times\vec F)=\nabla(\nabla\cdot\vec F)
                                                   -(\nabla\cdot\nabla)\vec F. \quad}
\]
Operaattoria $\nabla\cdot\nabla$, joka voi operoida sekä skalaariin että vektoriin, sanotaan 
\kor{Laplacen operaattoriksi}. Tämä on hyvin yleinen differentiaalioperaattori erilaisissa 
fysiikan laeissa (ks.\ huomautukset jäljempänä), siksi sille on vakiintunut erillinen symboli 
$\Delta$. Myös merkintää $\nabla^2$ käytetään. Lukutapoja ovat 'Laplace' tai 'nabla toiseen'. 
Määritelmä on siis
\begin{align*}
\nabla\cdot\nabla &= (\vec i \, \partial_x+\vec j \, \partial_y +\vec k \, \partial_z)\cdot
                     (\vec i \, \partial_x+\vec j \, \partial_y +\vec k \, \partial_z) \\
                  &=\partial_x^2+\partial_y^2+\partial_z^2=\Delta.
\end{align*}

\subsection{Operaattorit $\nabla\star$ ja $\vec u\cdot\nabla$}
\index{differentiaalioperaattori!h@$\nabla\star$, $\vec u\cdot\nabla$|vahv}

Yleisiä 'nablaamissääntöjä' voidaan usein yksinkertaistaa käyttämällä operaattorimerkintää
$\nabla\star$, missä '$\star$' voi olla 'piste' ($\nabla\cdot$), 'risti' ($\nabla\times$) tai
'tyhjä' ($\nabla$). Viimeksi mainitussa tapauksessa tulkitaan kohde aina skalaarikentäksi,
ts.\ $\nabla\vec F=\nabla F$. Operaattorimerkintä $\nabla\star$ on kätevä esimerkiksi tulon
derivoimissäännön yleistyksessä: Jos $f=f(x,y,z)$ on skalaarikenttä ja $\vec F$
kolmiulotteinen vektorikenttä, niin pätee
\[
\boxed{\kehys\quad \nabla\star(f\vec F)=\nabla f\star\vec F + f\nabla\star\vec F. \quad}
\] 
(Tapauksessa $\star=$ 'tyhjä' tämä tulkitaan: $\nabla(fF)=F\nabla f+f\nabla F$.)
Perustelu jätetään harjoitustehtäväksi (Harj.teht.\,\ref{H-udif-4: nablaamissääntöjä}abc). 

\index{muuttuvakertoiminen diff.-oper.}%
\kor{Muuttuvakertoimisista} vektoridifferentiaalioperaattoreista on syytä mainita
sovelluksissa yleinen operaattori
\[
\vec u\cdot\nabla  = u_1\partial_x + u_2\partial_y + u_3\partial_z,
\]
missä $\vec u=\vec u(x,y,z)$ (tai $\vec u=\vec u(x,y,z,t)$) on vektorikenttä. Tämä on
skalaarinen operaattori, joka voi kohdistua skalaari- tai vektorikenttään. Edellisessä
(ei jälkimmäisessä) tapauksessa $\vec u\cdot\nabla f$ on laskettavissa myös kaksivaiheisesti:
\[
f \map \nabla f \map \vec u\cdot\nabla f.
\]
Muuttuvakertoimisia operaattoreita yhdisteltäessä on huomioitava, että derivointi kohdistuu 
tällöin myös kertoimiin (vrt.\ Luku \ref{osittaisderivaatat}).

\subsection{Vectorikentän lähde. Poissonin ja Laplacen yhtälöt}

Kun $\vec F$ on jokin fysikaalinen kenttä, kuten sähkö-, magneetti-, virtaus- ym.\ kenttä, 
sanotaan divergenssiä
\[
\nabla\cdot\vec F=\rho
\]
\index{lzy@lähde (vektorikentän)}%
kentän $\vec F$ \kor{lähteeksi}. Kentän ja lähteen $\rho$ välinen yhteys on tällaisissa 
sovellutuksissa luonnonlaki, jossa lähde tulkitaan kentän $\vec F$ aiheuttajaksi --- siitä nimi
'lähde'. Tyypillinen lähtökohta onkin, että lähde on tunnettu ja kenttä halutaan määrätä. 
Esimerkiksi gravitaatiokenttää määrättäessä on $\rho$ verrannollinen massatiheyteen ja 
sähkökentän tapauksessa varaustiheyteen.

Jos vektorikentän $\vec F$ lähde $\rho$ on tunnettu ja lisäksi tiedetään, että $\vec F$ on 
gradienttikenttä, eli on olemassa potentiaali $u$ siten, että $\vec F = -\nabla u$ 
(näin on esim.\ gravitaatio- ja sähkökentän tapauksessa), niin kentän ja lähteen välinen yhteys
$\nabla\vec F=\rho$ voidaan kirjoittaa muotoon
\[
-\Delta u=\rho.
\]
\index{Poissonin yhtälö}%
Tätä sanotaan \kor{Poissonin yhtälöksi}. Sellaisissa pisteissä, joissa kenttä $\vec F$ on 
\index{lzy@lähteetön vektorikenttä} \index{vektorikenttä!a@lähteetön}%
\kor{lähteetön}, eli $\nabla\cdot\vec F=0$, toteuttaa kentän potentiaali
\index{Laplacen yhtälö}% 
\kor{Laplacen yhtälön}\footnote[2]{Laplacen ja Poissonin yhtälöiden nimet viittaavat 
ranskalaisiin matemaatikko-fyysikkoihin \hist{P.S. de Laplace} (1749-1827) ja 
\hist{S.D. Poisson} (1781-1840), jotka tutkivat tämän tyyppisten yhtälöiden teoriaa. Yhtälöiden
käytännöllisen ratkaisemisen perustan loi samoihin aikoihin vaikuttanut \hist{J.B.J. Fourier} 
(1768-1830). Myös Fourier oli matemaatikko-fysiikko, joka tutki erityisesti 
lämmönjohtumisilmiöitä. Näistä tutkimuksista alkunsa saaneet \kor{Fourier-sarjat} ja 
\kor{Fourier-muunnokset} ovat tehneet Fourier'n nimestä kuolemattoman.
\index{Laplace, P. S. de|av} \index{Poisson, S. D.|av} \index{Fouriera@Fourier, F. B. J.|av} }
\[
\Delta u=0.
\]
\index{harmoninen funktio}%
Funktiota, joka toteuttaa Laplacen yhtälön, sanotaan \kor{harmoniseksi}.
 
Sovellustilanteissa Poissonin tai Laplacen yhtälö esiintyy tyypillisesti
\index{reuna-arvotehtävä}% 
\kor{reuna-arvo\-tehtävässä} (engl.\ boundary value problem), jossa tarkasteltavan (avoimen)
joukon $A$ --- tai niinkuin tällaisissa yhteyksissä useammin sanotaan, \kor{alueen} $A$ 
($A\subset\Rkaksi$ tai $A\subset\Rkolme$) 
\index{reuna (joukon, alueen)} \index{reunaehto}% 
--- \kor{reunalla} $\partial A$ asetetaan jokin \kor{reunaehto}. Sovellustilanteessa reunaehto
määräytyy fysikaalisin perustein. Esimerkiksi jos reunalla asetetaan nk.\ (homogeeninen) 
\index{Dirichletb@Dirichlet'n (reuna)ehto}%
\kor{Dirichlet'n ehto}, niin saadaan reuna-arvotehtävän perusmuoto:
\[
\begin{cases} 
-\Delta u=\rho &\text{$A$:n sisäpisteissä}, \\ \,u=0 &\text{reunalla $\partial A$.}
\end{cases}
\]
Tämä on osoitettavissa tietyin (funktiota $\rho$ ja reunaviivaa/pintaa $S=\partial A$ koskevin) 
säännöllisyysehdoin yksikäsitteisesti ratkeavaksi. Itse ratkaisemisen on yleensä tapahduttava
numeerisin (likimääräisin) keinoin, muutamia poikkeustilanteita lukuun ottamatta.
\begin{Exa}
Jos $A$ on $R$-säteinen kiekko ($A\subset\Rkaksi$) tai kuula ($A\subset\Rkolme$) ja $\rho=Q=$ 
vakio, niin reuna-arvotehtävän
\[
\begin{cases} 
-\Delta u=Q &\text{$A$:n sisäpisteissä}, \\ \,u=0 &\text{reunalla $\partial A$}
\end{cases}
\]
ratkaisu on kummassakin tapauksessa toisen asteen polynomi:
\begin{align*}
A\subset\Rkaksi: \qquad u(x,y)   &= \frac{Q}{4}(R^2-x^2-y^2). \\
A\subset\Rkolme: \quad  u(x,y,z) &= \frac{Q}{6}(R^2-x^2-y^2-z^2). \loppu
\end{align*}
\end{Exa}

Poissonin ja Laplacen yhtälöt ovat esimerkkejä yhtälöistä, joista käytetään yleisnimeä 
\index{osittaisdifferentiaaliyhtälö}
\index{differentiaaliyhtälö!r@osittaisdifferentiaaliyhtälö}%
\kor{osittaisdifferentiaaliyhtälö} (engl.\ partial differential equation, lyh.\ PDE).
Paitsi Poissonin ja Laplacen yhtälöissä, Laplacen operaattori esiintyy monissa muissakin 
fysiikan osittaisdifferentiaaliyhtälöissä. Mainittakoon tässä ainoastaan ajasta riippuvia 
ilmiöitä kuvaavat \kor{diffuusioyhtälö} ja \kor{aaltoyhtälö}, joiden perusmuodot ovat 
(sovelluksissa näistä esiintyy erilaisia variaatioita):
\index{diffuusioyhtälö} \index{aaltoyhtälö}%
\begin{align*}
\text{Diffuusioyhtälö}: \quad   &\,u_t=\Delta u. \\
\text{Aaltoyhtälö}: \qquad\,\   &u_{tt}=\Delta u.
\end{align*}
Diffuusioyhtälön tyyppiä on esimerkiksi (aikariippuva) 
\pain{lämmön}j\pain{ohtumis}y\pain{htälö}. \index{lzy@lämmönjohtumisyhtälö}%

\subsection{Pyörrekenttä.  Maxwellin yhtälöt}

Fysikaalisen vektorikentän roottoria
\[
\nabla\times\vec F=\vec \omega
\]
\index{pyzz@pyörrekenttä}%
sanotaan kentän $\vec F$ \kor{pyörrekentäksi}. Pyörrekentillä on fysikaalinen merkitys 
esimerkiksi virtausmekaniikassa, mutta erityisen keskeisellä sijalla ne ovat sähkömagneettisten
kenttien teoriassa. Tämä nähdään jo sähkömagneettisten kenttien perusyhtälöistä, 
\kor{Maxwellin yhtälöistä} (\vahv{J.C. Maxwell}, 1864): \index{Maxwellin yhtälöt}
\[
\left\{\begin{aligned}
\nabla\times\vec E &= -\frac{\partial \vec B}{\partial t}\,, \\
\nabla\times\vec H &=  \frac{\partial \vec D}{\partial t}+\vec J, \\[2mm]
\nabla\cdot\vec D &=\rho, \\[3mm]
\nabla\cdot\vec B &= 0.
\end{aligned}\right.
\]
Tässä esiintyvien kenttien nimet ovat:
\begin{align*}
\vec E: \quad &\text{sähkökenttä} \qquad\qquad \vec D: \quad \text{sähkövuon tiheys} \\
\vec H: \quad &\text{magneettikenttä} \qquad\, \vec B: \quad \text{magneettivuon tiheys} \\
\rho  : \quad &\text{varaustiheys} \qquad\qquad \vec J: \quad \text{virtatiheys}
\end{align*}
Maxwellin yhtälöitä ratkaistaessa on huomioitava kenttien $\vec E,\vec D$ ja $\vec H,\vec B$
keskinäinen riippuvuus, joka on väliaineelle ominainen. Yksinkertisimmillaan riippuvuudet ovat
ilmaistavissa kahden materiaalivakion $\epsilon,\mu$ avulla muodossa 
\[
\vec D=\epsilon\vec E, \quad \vec B=\mu\vec H.
\]

\index{pyzz@pyörteetön vektorikenttä} \index{vektorikenttä!b@pyörteetön}%
Vektorikenttää $\vec F$ sanotaan \kor{pyörteettömäksi} sellaisessa alueessa, jossa 
$\nabla\times\vec F=\vec 0$. Kaikki gradienttikentät, eli skalaaripotentiaalin avulla 
ilmaistavat kentät, ovat pyörteettömiä (koska $\nabla\times\nabla=\vec 0$). Maxwellin
yhtälöiden mukaan sähkökenttä $\vec E$ on pyörteetön \pain{stationaarisessa} tilanteessa, jossa
kentät ovat ajasta riippumattomia. Sekä lähteetön että pyörteetön on esimerkiksi 
gravitaatiokenttä tyhjässä (massattomassa) avaruudessa, ja yleisemminkin gradienttikenttä 
lähteettömässä alueessa.

\subsection{Roottori tasossa}
\index{differentiaalioperaattori!f@roottori $\nabla\times$|vahv}
\index{roottori|vahv}

Jos tason vektorikenttä tulkitaan ($z$-koordinaatista riippumattomaksi ja $xy$-tason
suuntaiseksi) avaruuden vektorikentäksi, niin kentän roottori on
\[
\nabla\times[F_1(x,y)\,\vec i+F_2(x,y)\,\vec j\,] =(\partial_x F_2-\partial_y F_1)\,\vec k.
\]
Tässä on tapana kirjoittaa
\[
\text{rot}\,\vec F = \frac{\partial F_2}{\partial x}-\frac{\partial F_1}{\partial y}
                   \quad (\text{tason vektorikenttä}),
\]
jolloin roottori tasossa tulee määritellyksi skalaarisena operaattorina (kuten div). 
\begin{Exa} Maxwellin yhtälöiden mukaan stationaarisessa tilanteessa magneettikentän $\vec H$
pyörrekenttä on sähkövirran tiheys $\vec J$. Jos oletetetaan tyypillinen sovellustilanne,
jossa virta kulkee suorassa, $z$-akselin suuntaisessa johtimessa, niin $\vec J$ on muotoa
\[
\vec J(x,y,z)=\begin{cases} 
              \,I(x,y)\vec k, &\text{kun}\ (x,y) \in A, \\ \,\vec 0, &\text{muulloin}.
              \end{cases}
\]
Tässä $A\subset\Rkaksi$ edustaa johtimen poikkileikkausta ja $I$ on (skalaarinen) virran tiheys
johtimessa. Magneettikenttä on $xy$-tason suuntainen ja $z$-koordinaatista riippumaton,
ts.\ muotoa $\vec H = H_1(x,y)\vec i + H_2(x,y)\vec j$, jolloin Maxwellin yhtälö 
$\nabla\times\vec H=\vec J$ pelkistyy johtimen poikkipinnalla skalaariseksi yhtälöksi
\[
\text{rot}\,\vec H = \partial_x H_2 - \partial_y H_1 = I(x,y).
\]
Johtimen ulkopuolella magneettikenttä on pyörteetön: $\text{rot}\,\vec H=0$. \loppu
\end{Exa}

\subsection{*Navier--Stokesin virtausyhtälöt}
\index{Navier--Stokesin yhtälöt|vahv}

Nesteen tai kaasun (esim.\ veden tai ilman) virtausta hallitsevat perusyhtälöt ovat nk.\
\index{szy@säilymislaki}%
\kor{säilymislakeja} (engl.\ conservation laws), jotka ilmaisevat matemaattisesti
(osittaisdifferentiaaliyhtälöinä) \pain{massan}, \pain{liikemäärän} ja \pain{ener}g\pain{ian}
säilymisen virtauksessa. Massan ja liikemäärän säilymislaeissa esiintyvät vektori- ja
skalaarikentät ovat
\begin{align*}
\vec u: \quad &\text{virtausnopeus} \\
     p: \quad &\text{paine} \\
  \rho: \quad &\text{(massa)tiheys}
\end{align*}
Jos virtaavan aineen massatiheys voidaan olettaa vakioksi (esim.\ vesi, monissa
virtaustilanteissa myös ilma) ja lisäksi virtaus on ulkoisista voimista vapaa, niin liikemäärän
ja massan säilymislait voidaan kirjoittaa \kor{Navier--Stokesin yhtälöinä}
(\vahv{C.-L. Navier, G.G. Stokes}, 1822) 
\[
\left\{ \begin{aligned}
\vec u_t-\nu\Delta\vec u+(\vec u\cdot\nabla)\vec u+\rho^{-1}\nabla p &= \vec 0, \\
                                               \nabla\cdot\vec u &=0.
\end{aligned} \right.
\]
Tässä $\nu$ on virtaavan aineen (kinemaattinen) \pain{viskositeetti}.
%\footnote[2]{Virtausmekaniikan yhtälöiden numeerinen ratkaiseminen
%tietokoneilla on nykyisin arkipäivää lentokoneiden aerodynamiikassa, laivojen
%hydrodynamiikassa, meteorologiassa, ym.\ sovelluksissa. Yhtälöihin liittyy silti myös avoimia
%matemaattisia ongelmia. Esimerkiksi Navier--Stokesin yhtälöiden nk.\ globaalia (ajassa ja
%paikassa rajoittamatonta) ratkeavuutta (tai vastaesimerkillä: ratkeamattomuutta) säännöllisestä
%alkutilanteesta ei ole pystytty osoittamaan matemaattisesti. --- Kyseessä on eräs tunnettu
%''miljoonan dollarin ongelma'', jonka ratkaisjalle on yhdysvaltalainen Clay-instituutti
%luvannut mainitun suuruisen palkkion.}

\Harj
\begin{enumerate}

\item
Laske divergenssi ja roottori: \vspace{2mm}\newline
a) \ $x\vec i+y\vec j+z\vec k \qquad$
b) \ $yz\vec i+xz\vec j+xy\vec k \qquad$
c) \ $xy^2\vec i-yz^2\vec j+zx^2\vec k$ \vspace{1mm}\newline
d) \ $f(x)\vec i+g(y)\vec j+h(z)\vec k \qquad$
e) \ $f(y,z)\vec i+g(x,z)\vec j+h(x,y)\vec k$
 
\item 
Olkoon
$\vec F(x,y,z)=(x y-z^2)\,\vec i + x y z \,\vec j + (x-y^2-z^2)\,\vec k$. \vspace{1mm}\newline
a) Laske $\,\nabla\cdot\vec F$, $\,\nabla(\nabla\cdot\vec F)$, $\,\nabla\times\vec F$, 
$\,\nabla\times(\nabla\times\vec F)\,$ ja $\,\Delta \vec F$ ja tarkista, että pätee
$\,\nabla\times(\nabla\times\vec F) 
            = \nabla(\nabla\cdot\vec F)-\Delta\vec F$. \vspace{1mm}\newline
b) Laske $(\vec F\cdot\nabla)\vec F$ ja $(\vec F\times\nabla)\cdot\vec F$.

\item \label{H-udif-4: nablaamissääntöjä}
Olkoon $\vec F$ ja $\vec G$ tason tai avaruuden (mahdollisesti ajasta riippuvia) vektorikenttiä
ja $f$ ja $g$ skalaarikenttiä. Perustele seuraavat säännöt
(oletetaan riittävä derivoituvuus): \vspace{1mm}\newline
a) \ $\nabla(fg)=g\nabla f+f\nabla g\qquad\qquad\qquad\ \ \ $
b) \ $\nabla\cdot(f\vec F)=\vec F\cdot\nabla f+f\,\nabla\cdot\vec F$ \vspace{1mm}\newline
c) \ $\nabla\times(f\vec F)=\nabla f\times\vec F+f\,\nabla\times\vec F\qquad\ $
d) \ $\D\partial_t(\nabla\star\vec F)=\nabla\star\vec F_t$ \vspace{1mm}\newline
e) \ $\nabla\cdot(\vec F\times\vec G)=(\nabla\times\vec F)\cdot\vec G
                                        -(\nabla\times\vec G)\cdot\vec F$ \vspace{1mm}\newline
f) \ $\nabla\times(\nabla\times\vec F)
                  =\nabla(\nabla\cdot\vec F)-\Delta\vec F$ \vspace{1mm}\newline
g) \ $\square(\vec F\circ\vec G) = (\square\vec F)\circ\vec G+\vec F\circ\square\vec G \,\ $
     ja $\,\ \square(f\vec F)=(\square f)\vec F+f\square\vec F$, \newline
     kun $\,\ \square\in\{\partial_x,\partial_y,\partial_z,\partial_t\}\,\ $ ja
     $\,\ \circ\in\{\,\cdot\,,\times\,\}$ \vspace{1mm}\newline
h) \ $(\vec u\,\square)\circ f\vec F 
      = \square f\,\vec u\circ\vec F+f\vec u\circ\,\square\vec F$,\, kun
     $\,\ \square\in\{\partial_x,\partial_y,\partial_z,\partial_t\}\,\ $ ja
     $\,\ \circ\in\{\,\cdot\,,\times\,\}$ \newline

\item
a) Näytä, että seuraavat funktiot ovat määrittelyjoukossaan harmonisia:
\begin{align*}
&xy,\ \ x^2-y^2,\ \ x^3-3xy^2,\ \ x^3y-xy^3,\ \ \ln(x^2+y^2),\ \ \Arctan\frac{y}{x} \\
&e^{ax}(A\sin ay+B\cos ay)\ \ (a,A,B\in\R)
\end{align*}
b) Monen muuttujan Laplacen operaattori määritellään
$\Delta=\partial_1^2+ \ldots + \partial_n^2$. Tutki, millaisille reaalifunktioille $f$ pätee: 
$f\bigl(\abs{\mx})=f(\sqrt{x_1^2 + \ldots + x_n^2}\,\bigr)$ on harmoninen muualla kuin origossa.

\item
Seuraavilla osittaisdifferentiaaliyhtälöillä väitetään olevan annettua muotoa olevia ratkaisuja.
Tarkista väittämät (funktiot $f$ ja $g$ oletetaan kahdesti derivoituviksi).
\vspace{1mm}\newline
a) \ $u_{tt}=c^2u_{xx}\,,\ \ u(x,t)=f(x+ct)+g(x-ct)$ \vspace{1mm}\newline
b) \ $u_{xx}+u_{yy}=2u_{xy}\,,\ \ u(x,y)=xf(x+y)+yg(x+y)$ \vspace{1mm}\newline
c) \ $u_t=u_{xx}\,,\ \ u(x,t)=e^{-\omega^2 t}(A\sin\omega x+B\sin\omega x)\ \ (\omega,A,B\in\R)$
\vspace{1mm}\newline
d) \ $u_t=u_{xx}\,,\ \ u(x,t)=t^{-1/2}e^{-x^2/4t}$ \vspace{1mm}\newline
e) \ $u_t=u_{xx}+u_{yy}\,,\ \ u(x,y,t)=t^{-1}e^{-(x^2+y^2)/4t}$

\item
Näytä, että osittaisdifferentiaaliyhtälön $u_{xy}=0$ yleinen ratkaisu $\R^2$:ssa on
$u(x,y)=f(x)+g(y)$, missä $f:\,\R\kohti\R$ ja $g:\,\R\kohti\R$ ovat mitä tahansa $\R$:ssä 
derivoituvia funktioita.

\item (*)
Todista (olettaen riittävä derivoituvuuus)\,:
\[
\nabla\times(\vec F\times\vec G)=(\nabla\cdot\vec G)\vec F+(\vec G\cdot\nabla)\vec F
                                -(\nabla\cdot\vec F)\vec G-(\vec F\cdot\nabla)\vec G.
\]

\item (*)
Olkoon $A\subset\R^2$ tasasivuinen kolmio, jonka kärjet ovat $(0,0)$, $(a,0)$ ja
$(a/2,\sqrt{3}a/2)$ $(a>0)$. Näytä, että reuna-arvotehtävällä
\[
\begin{cases} 
-\Delta u=Q &\text{$A$:n sisäpisteissä}, \\ \,u=0 &\text{reunalla $\partial A$}
\end{cases}
\]
($Q=$ vakio) on polynomiratkaisu.

\item (*)
Tarkastellaan Maxwellin yhtälöitä homogeenisessa väliaineessa, jossa $\rho=0$, $\vec J=\vec 0$
ja kenttien $\vec D,\vec E$ ja $\vec B,\vec H$ välillä on yhteydet $\vec D=\epsilon\vec E$ ja 
$\vec B=\mu\vec H$ ($\epsilon$ ja $\mu$ materiaalivakioita). Näytä, että kaikki mainitut
kentät, sikäli kuin riittävän säännöllisiä, toteuttavat vektorimuotoisen aaltoyhtälön
\[
\vec F_{tt}=c^2\Delta\vec F,
\]
missä $c=(\epsilon\mu)^{-1/2}$ (= valon nopeus väliaineessa). 

\end{enumerate} %Divergenssi ja roottori
\section{Operaattorit grad, div, rot ja $\Delta$ \\
         käyräviivaisissa koordinaatistoissa}
\label{div ja rot käyräviivaisissa}
\sectionmark{Operaatorit käyräviivaisissa}
\alku
\index{kzyyrzy@käyräviivaiset koordinaatistot!c@--differentiaalioperaattorit|vahv}
\index{osittaisdifferentiaalioperaattori!vektorioperaattorit|vahv}

Fysikaalisten symmetrioiden vuoksi vektorikenttä/skaalaarikenttä halutaan usein esittää 
käyräviivaisessa  polaari-, lieriö- tai pallokoordinaatistossa. Tällöin myös
vektoridifferentiaalioperaattorit on muunnettava karteesisesta ko.\ käyräviivaiseen 
koordinaatistoon. Tarkastellaan esimerkkinä muunnosta tason polaarikoordinaatistoon.

Ensinnäkin on tutkittava, miten operaattorit $\partial_x$ ja $\partial_y$ ilmaistaan
polaarikoordinaatiston vastaavien operaattorien $\partial_r$ ja $\partial_\varphi$ avulla.
Lähtökohtana on ketjusääntö (ks.\ Luku \ref{osittaisderivaatat}), jonka mukaan
\begin{align*}
&u(x,y)=v(r,\varphi)=v(r(x,y),\varphi(x,y)) \\
&\impl \ \begin{cases}
\partial_x u=\dfrac{\partial r}{\partial x}\,\partial_r v
            +\dfrac{\partial \varphi}{\partial x}\,\partial_\varphi v, \\[2mm]
\partial_y u=\dfrac{\partial r}{\partial y}\,\partial_r v
            +\dfrac{\partial \varphi}{\partial y}\,\partial_\varphi v.
\end{cases}
\end{align*}
Tässä on $\,r=r(x,y)=\sqrt{x^2+y^2}\,$, joten
\[
\frac{\partial r}{\partial x}=\frac{x}{r}=\cos\varphi,\quad 
\frac{\partial r}{\partial y}=\frac{y}{r}=\sin\varphi.
\]
Derivoimalla implisiittisesti $x$:n suhteen (implisiittinen osittaisderivointi!) saadaan tällöin
\begin{align*}
x=r\cos\varphi \ &\impl \ 1=\frac{\partial r}{\partial x}\cos\varphi
                           -r\sin\varphi\frac{\partial\varphi}{\partial x} \\
                 &\impl \ 1=\cos^2\varphi-r\sin\varphi\frac{\partial\varphi}{\partial x} \\
                 &\impl \ \frac{\partial\varphi}{\partial x}=-\frac{1}{r}\sin\varphi.
\end{align*}
Vastaavasti derivoimalla $y$:n suhteen saadaan
\begin{align*}
0=\frac{\partial r}{\partial y}\cos\varphi-r\sin\varphi\frac{\partial\varphi}{\partial y} 
&=\sin\varphi\cos\varphi-r\sin\varphi\frac{\partial\varphi}{\partial y} \\
  \impl \ \frac{\partial\varphi}{\partial y} 
&= \frac{1}{r}\cos\varphi.
\end{align*}
Näin on saatu muunnoskaavat
\[
\boxed{
\begin{aligned}
\quad \partial_x &= \cos\varphi\,\partial_r-\frac{1}{r}\sin\varphi\,\partial_\varphi, \quad \\
      \partial_y &= \sin\varphi\,\partial_r+\frac{1}{r}\cos\varphi\,\partial_\varphi.
\end{aligned}}
\]
\begin{multicols}{2} \raggedcolumns
Kun näiden lisäksi huomioidaan kantavektorien $\vec i,\vec j$ ja 
$\vec e_r,\vec e_\varphi$ väliset muunnoskaavat
\[ \begin{cases}
\,\vec i = \cos\varphi \, \vec e_r - \sin\varphi \, \vec e_\varphi, \\
\,\vec j = \sin\varphi \, \vec e_r + \cos\varphi \, \vec e_\varphi,
\end{cases} \]
niin operaattorille $\nabla$ saadaan ensin esitysmuoto 
\begin{figure}[H]
\setlength{\unitlength}{1cm}
\begin{center}
\begin{picture}(4,4)(0,1)
\put(0,0.5){\vector(1,0){4}} \put(0,0.5){\vector(0,1){3.5}}
\put(3.8,0){$x$} \put(0.2,3.8){$y$}
\path(0,0.5)(3,2.5)
\put(3,2.5){\vector(1,0){1}} \put(3,2.5){\vector(0,1){1}}
\put(3.8,2){$\vec i$} \put(3.2,3.3){$\vec j$}
\put(3,2.5){\vector(3,2){0.9}} \put(3,2.5){\vector(-2,3){0.6}}
\put(1.9,3.3){$\vec e_\varphi$} \put(3.8,3.2){$\vec e_r$}
\put(0,0.5){\arc{1}{-0.6}{0}} \put(3,2.5){\arc{1}{-0.6}{0}}
\put(0.6,0.6){$\scriptstyle{\varphi}$} \put(3.6,2.6){$\scriptstyle{\varphi}$}
\end{picture}
\end{center}
\end{figure}
\end{multicols}
\begin{align*}
\nabla \ =\  \vec i\,\partial_x + \vec j\,\partial_y
        &=\ (\cos\varphi\,\vec e_r - \sin\varphi\,\vec e_\varphi)
            (\cos\varphi\,\partial_r -\frac{1}{r}\sin\varphi\,\partial_\varphi) \\
        &+\ (\sin\varphi \,\vec e_r + \cos\varphi \,\vec e_\varphi)
           (\sin\varphi\, \partial_r+\frac{1}{r}\cos\varphi \, \partial_\varphi)
\end{align*}
ja sievennysten jälkeen
\[
\boxed{\kehys\quad 
   \nabla = \vec e_r\,\partial_r+\frac{1}{r}\,\vec e_\varphi\,\partial_\varphi \quad
                                                 \text{(polaarikoordinaatisto).} \quad}
\]
\begin{Exa} Laske funktion $u(x,y)=x/(x^2+y^2)$ gradientti polaarikoordinaatistossa.
\end{Exa}
\ratk Koska
\[
u(x,y)\,=\,\frac{x}{x^2+y^2}\,=\,\frac{r\cos\varphi}{r^2}
                            \,=\,\frac{\cos\varphi}{r}\,=\,v(r,\varphi),
\]
niin em.\ muunnoskaavan mukaan
\[
\nabla u \,=\, \pder{v}{r}\,\vec e_r + \frac{1}{r}\pder{v}{\varphi}\,\vec e_\varphi
         \,=\, \underline{\underline{
                -\frac{1}{r^2}(\cos\varphi\,\vec e_r + \sin\varphi\,\vec e_\varphi)}}. \loppu
\]

Annetun vektorikentän $\vec F$ divergenssin laskemiseksi oletetaan, että $\vec F$ tunnetaan
muodossa
\[
\vec F = F_r(r,\varphi,z)\,\vec e_r + F_\varphi(r,\varphi,z)\,\vec e_\varphi.
\]
Tällöin on
\begin{align*}
\nabla\cdot\vec F \,
    &=\,\left(\vec e_r\,\partial_r+\frac{1}{r}\,\vec e_\varphi\,\partial_\varphi\right)
        \cdot(F_r\vec e_r + F_\varphi\vec e_\varphi) \\[3mm]
    &=\,(\vec e_r\,\partial_r)\cdot(F_r\vec e_r)\,
     +\,(\vec e_r\,\partial_r)\cdot(F_\varphi\vec e_\varphi) \\
    &+\,\frac{1}{r}\,(\vec e_\varphi\,\partial_\varphi)\cdot(F_r\vec e_r)
     +\,\frac{1}{r}\,(\vec e_\varphi\,\partial_\varphi)\cdot(F_\varphi\vec e_\varphi).
\end{align*}
Tässä on huomioitava, että vektorit $\vec e_r$ ja $\vec e_\varphi$ eivät ole vakiovektoreita
vaan riippuvat koordinaatista $\varphi$. Tarvitaan siis myös derivaatat 
$\partial_\varphi\,\vec e_r$ ja $\partial_\varphi\,\vec e_\varphi\,$:
\begin{align*}
&\partial_\varphi\,\vec e_r = \partial_\varphi(\cos\varphi\,\vec i+\sin\varphi\,\vec j)
                            = -\sin\varphi\,\vec i+\cos\varphi\,\vec j, \\
&\partial_\varphi\,\vec e_\varphi =\partial_\varphi(-\sin\varphi\,\vec i+\cos\varphi\,\vec j)
                                  = -\cos\varphi\,\vec i-\sin\varphi\,\vec j,
\end{align*}
eli
\[
 \boxed{\kehys\quad \partial_\varphi \vec e_r = \vec e_\varphi, \quad 
                    \partial_\varphi \vec e_\varphi = -\vec e_r. \quad} 
\]
Käyttämällä näitä kaavoja ja tulon derivoimissääntöjä 
(ks.\ Harj.teht.\,\ref{divergenssi ja roottori}:\,\ref{H-udif-4: nablaamissääntöjä}h)
saadaan ym.\ lauseke puretuksi termeittäin:
\begin{align*}
(\vec e_r\,\partial_r)\cdot(F_r\vec e_r)\,
    &=\,\partial_r F_r\,\vec e_r\cdot\vec e_r+F_r\,\vec e_r\cdot\partial_\varphi\vec e_r
   \,=\,\partial_r F_r+F_r\,\vec e_r\cdot\vec e_\varphi  
   \,=\,\partial_r F_r, \\[2mm]
(\vec e_r\,\partial_r)\cdot(F_\varphi\vec e_\varphi)\,
    &=\,\partial_r F_\varphi\,\vec e_r\cdot\vec e_\varphi
                  +F_\varphi\,\vec e_r\cdot\partial_r\vec e_\varphi
   \,=\, 0, \\ 
\frac{1}{r}\,(\vec e_\varphi\,\partial_\varphi)\cdot(F_r\vec e_r)\,
    &=\,\frac{1}{r}\,\bigl[\partial_\varphi F_r\,\vec e_\varphi\cdot\vec e_r
                       +F_r\,\vec e_\varphi\cdot\partial_\varphi\,\vec e_r\bigr]
   \,=\, \frac{1}{r}\,F_r\,\vec e_\varphi\cdot\vec e_\varphi
   \,=\,\frac{1}{r}\,F_r\,, \\
\frac{1}{r}\,(\vec e_\varphi\,\partial_\varphi)\cdot(F_\varphi\vec e_\varphi)\,
    &=\,\frac{1}{r}\,\bigl[\partial_\varphi F_\varphi\,\vec e_\varphi\cdot\vec e_\varphi
                       +F_\varphi\,\vec e_\varphi\cdot\partial_\varphi\,\vec e_\varphi\bigr]\, \\
    &=\,\frac{1}{r}\,\bigl[\partial_\varphi F_\varphi-F_\varphi\,\vec e_\varphi\cdot\vec e_r\bigr]
   \,=\,\frac{1}{r}\,\partial_\varphi F_\varphi.
\end{align*}
Lopputulos:
\[
\boxed{\kehys\quad \nabla\cdot\vec F = \partial_r F_r + \dfrac{1}{r}\,F_r
                                       + \frac{1}{r}\,\partial_\varphi F_\varphi \quad
                                         \text{(polaarikoordinaatisto)}. \quad}
\]
Kaksi ensimmäistä termiä yhdistämällä saadaa tälle vaihtoehtoinen nk.\ \kor{säilymislakimuoto}
(engl.\ conservation form) 
\[
\boxed{\kehys\quad \nabla\cdot\vec F = \dfrac{1}{r}\,\partial_r(rF_r)
                                       +\frac{1}{r}\,\partial_\varphi F_\varphi \quad
                                        \text{(säilymislakimuoto)}. \quad}
\]
\begin{Exa} Vektorikentän $\,\vec F=r^{-1}\,\vec e_r$ divergenssi muualla kuin origossa on
\[
\nabla\cdot\vec F = \frac{1}{r}\,\partial_r\left(r\cdot\frac{1}{r}\right)=0,
\]
joten kenttä on lähteetön aluessa $A=\{(x,y)\in\Rkaksi \mid (x,y) \neq (0,0)\}$. \loppu
\end{Exa}

Seuraavassa taulukossa esitetään koottuna operaattorien $\nabla$, $\nabla\cdot$, $\nabla\times$
ja $\Delta$ (säilymislakimuotoiset) laskusäännöt lieriö- ja pallokoordinaatistoissa. 
\begin{center}
\begin{tabular}{|ll|}
\hline & \\
\multicolumn{2}{|l|}{$\quad$\vahv{Lieriökoordinaatisto}}  \\ & \\ $\quad\nabla$ 
&= \quad $\vec e_r\,\partial_r+\dfrac{1}{r}\,\vec e_\varphi\,\partial_\varphi
         +\vec e_z\,\partial_z$ \\ & \\ $\quad\nabla\cdot\vec F$ 
&= \quad $\dfrac{1}{r}\,\partial_r(rF_r)+\dfrac{1}{r}\,\partial_\varphi F_\varphi
                       +\partial_z F_z$ \\ & \\ $\quad\nabla\times\vec F$ 
&= \quad $\dfrac{1}{r}\,\begin{vmatrix}
                        \vec e_r & r\vec e_\varphi & \vec e_z \\
                        \partial_r & \partial_\varphi & \partial_z \\
                        F_r & rF_\varphi & F_z
                        \end{vmatrix}$ \\ & \\ $\quad\Delta$ 
&= \quad $\dfrac{1}{r}\,\partial_r(r\partial_r)+\dfrac{1}{r^2}\partial_\varphi^2
                       +\partial_z^2$ \\ & \\ \hline & \\
\multicolumn{2}{|l|}{$\quad$\vahv{Pallokoordinaatisto}}  \\ & \\ $\quad\nabla$ 
&= \quad $\vec e_r \, \partial_r+\dfrac{1}{r}\vec e_\theta \, \partial_\theta
                      +\dfrac{1}{r\sin\theta} \vec e_\varphi \, \partial_\varphi$ \\ & \\
         $\quad\nabla\cdot\vec F$ 
&= \quad $\dfrac{1}{r^2}\partial_r(r^2F_r)
         +\dfrac{1}{r\sin\theta}\partial_\theta(\sin\theta\,F_\theta)
         +\dfrac{1}{r\sin\theta}\partial_\varphi F_\varphi$ \\ & \\ $\quad\nabla\times\vec F$
&= \quad $\dfrac{1}{r^2\sin\theta}\,\begin{vmatrix}
                                    \vec e_r & r\vec e_\theta & r\sin\theta \, \vec e_\varphi \\
                                    \partial_r & \partial_\theta & \partial_\varphi \\
                                    F_r & rF_\theta & r\sin\theta \, F_\varphi
                                    \end{vmatrix}$ \\ & \\ $\quad\Delta$ 
&= \quad $\dfrac{1}{r^2}\partial_r(r^2\partial_r)
          +\dfrac{1}{r^2\sin\theta}\partial_\theta(\sin\theta \, \partial_\theta)
          +\dfrac{1}{r^2\sin^2\theta} \partial_\varphi^2 \quad$ \\ & \\
\hline
\end{tabular}
\end{center}
\index{gradientti!d@käyräv. koordinaateissa}
\index{divergenssi!a@käyräv. koordinaateissa}
\index{roottori!a@käyräv. koordinaateissa}
\index{Laplacen operaattori!a@käyräv. koordinaateissa}
\index{differentiaalioperaattori!d@gradientti (nabla $\nabla$)}
\index{differentiaalioperaattori!e@divergenssi $\nabla\cdot$}
\index{differentiaalioperaattori!f@roottori $\nabla\times$}
\index{differentiaalioperaattori!g@Laplacen operaattori}%
\vspace{2mm}
Taulukon mukaan Laplacen operaattorin muunnos tason karteesisesta koordinaatistosta 
polaarikoordinaatistoon on
\[
\Delta\,=\,\partial_x^2+\partial_y^2\,
        =\,\frac{1}{r}\,\partial_r(r\partial_r)+\frac{1}{r^2}\,\partial_\varphi^2\,
        =\,\partial_r^2+\frac{1}{r}\,\partial_r+\frac{1}{r^2}\,\partial_\varphi^2.
\]
Tähän päädytään sieventämällä edellä kuvatulla tavalla operaattorilauseke
\[
\Delta\,=\,\nabla\cdot\nabla\,
        =\, \left(\vec e_r\,\partial_r+\frac{1}{r}\,\vec e_\varphi\,\partial_\varphi\right)
           \cdot\left(\vec e_r\,\partial_r+\frac{1}{r}\,\vec e_\varphi\,\partial_\varphi\right)
\]
(Harj.teht.\,\ref{H-udif-5: Laplacen operaattori}).
\begin{Exa} Olkoon kenttä $\vec F$ on pallosymmetrinen, eli pallokoordinaatistossa muotoa 
$\vec F = F(r)\vec e_r$. Millainen on funktio $F(r)$, jos kenttä on origon ulkopuolella 
lähteetön?
\end{Exa}
\ratk Taulukon mukaan kentän divergenssi on $\nabla\cdot\vec F = r^{-2}\partial_r(r^2 F(r))$, 
joten origon ulkopuolella ($r>0$) on oltava
\[ 
\partial_r(r^2 F(r)) = 0 \,\ \impl\,\ r^2 F(r) = A \,\ \impl\,\ F(r) = \frac{A}{r^2}, \quad
                                                 A\in\R. \loppu
\]

Polaari- tai pallosymmetrisessä tilanteessa, jossa lähde $\rho$ ja vektorikentän potentiaali $u$
riippuvat vain radiaalikoordinaatista $r$, voidaan Poissonin yhtälö kirjoittaa tavallisena
differentiaaliyhtälönä (vrt.\ taulukko edellä)\,:
\begin{align*}
\text{Polaarisymmetria}:\,\quad &\frac{1}{r}\frac{d}{dr}\left(r\frac{du}{dr}\right)\ \ 
                                                =\, u''+\frac{1}{r}u'=-\rho(r). \\
\text{Pallosymmetria}:   \qquad &\frac{1}{r^2}\frac{d}{dr}\left(r^2\frac{du}{dr}\right)
                                                =\, u''+\frac{2}{r}u'=-\rho(r).
\end{align*}
Nämä ovat molemmat Eulerin tyyppiä (vrt.\ Luku \ref{vakikertoimiset ja Eulerin DYt}), joten ne
ovat ratkaistavissa kvadratuureilla. Nopeimmin ratkaiseminen käy suoraan säilymislakimuodosta.
\begin{Exa}
\begin{multicols}{2} \raggedcolumns
Lämpötila pyöreän vastuslangan poikkipinnalla toteuttaa yhtälön
\[
-\Delta u=Q=\text{vakio}.
\]
Määrää $u$, kun tiedetään, että ulkopinnalla $u(R)=0$.
\begin{figure}[H]
\setlength{\unitlength}{1cm}
\begin{center}
\begin{picture}(3,2)
\put(1,1){\circle{2}}
\dashline{0.2}(1,2)(3,2) \dashline{0.2}(1,0)(3,0)
\put(2.25,0.9){$2R$}
\put(2.5,0.7){\vector(0,-1){0.7}}
\put(2.5,1.3){\vector(0,1){0.7}}
\end{picture}
\end{center}
\end{figure}
\end{multicols}
\end{Exa}
\ratk \ Lähdetään säilymislakimuodosta (polaarisymmetria):
\begin{align*}
&-\frac{1}{r}\frac{d}{dr}(ru')=Q \ \impl \ \frac{d}{dr}(ru')=-Qr \\
&\impl \ ru'=-\frac{1}{2}Qr^2+A \ \impl \ u'=-\frac{1}{2}Qr+\frac{A}{r} \\
&\impl \ u(r)=-\frac{1}{4}Qr^2+A\ln r+B\quad (A,B\in\R) \\
&\impl \ A=0, \ B=\frac{1}{4}QR^2 \\ 
&\impl \ u(r)=\underline{\underline{\frac{1}{4}\,Q(R^2-r^2)}}.
\end{align*}
Karteesisessa koordinaatistossa esitettynä ratkaisu on
\[ 
u(x,y) = \frac{Q}{4}\,(R^2-x^2-y^2). 
\]
Ratkaisusta oli jätettävä pois logaritminen termi $A\ln r = \tfrac{1}{2}A\ln(x^2+y^2)$, syystä
että yhtälö $-\Delta u=Q$ ei muuten toteutuisi origossa (muualla kylläkin). \loppu

\Harj
\begin{enumerate}

\item
Laske vektorikentän
\[
\vec F=(x^2+y^2+z^2)(x\vec i+y\vec j+z\vec k)
\]
divergenssi $\nabla\cdot\vec F$ \ a) karteesisessa koordinaatistossa, \ b) muuntamalla $\vec F$
ensin pallokoordinaatistoon.

\item
Näytä, että seuraavat polaari- tai pallokoordinaatistossa määritellyt funktiot ovat 
määrittelyjoukossaan harmonisia.
\[
\text{a)}\ \ u(r,\varphi)=\frac{\cos\varphi}{r} \qquad
\text{b)}\ \ u(r,\varphi)=\frac{\sin 2\varphi}{r^2} \qquad
\text{c)}\ \ u(r,\theta,\varphi)=\frac{\sin\theta\cos\varphi}{r^2}
\]

\item
Vektorikentästä $\vec F=F_1\vec i+F_2\vec j+F_3\vec k$ tiedetään, että funktiot $F_i$ ovat
$\R^3$:ssa differentioituvia ja että kenttä on $\R^3$:ssa pyörteetön. Määritä kentän karteesiset
komponentit $F_i$, kun tiedetään lisäksi, että kenttä on pallokoordinaatistossa muotoa
\[
\vec F = r^3\sin^2\theta\,\vec e_r + F_\theta(r,\theta,\varphi)\,\vec e_\theta\,.
\]

\item \label{H-udif-5: Laplacen operaattori}
Sievennä Lapalacen operaattorin lauseke polaarikoordinaatistossa, ts.\ näytä, että
\[
\left(\vec e_r\,\partial_r+\frac{1}{r}\,\vec e_\varphi\,\partial_\varphi\right)
        \cdot\left(\vec e_r\,\partial_r+\frac{1}{r}\,\vec e_\varphi\,\partial_\varphi\right)
 \,=\, \dfrac{1}{r}\,\partial_r(r\partial_r)+\dfrac{1}{r^2}\partial_\varphi^2\,.
\]

\item 
Sähköjohtimessa kulkeva virta, jonka tiheys on $\vec J = J(r)\vec e_z$ (lieriökoordinaatisto),
aiheuttaa magneettikentän muotoa $\vec H = H(r)\vec e_\varphi$, missä $H(0) = 0$. Määrää $H(r)$
$J(r)$:n avulla Maxwellin yhtälöstä $\nabla\times\vec H=\vec J$.

\item 
Homogeenisesta materiaalista valmistetussa lieriön tai lieriökuoren muotoisen kappaleen
poikkipinnalla
\[
A = \{(x,y)\in\R^2 \mid R_1^2 \le x^2 + y^2 \le R_2^2 \}
\]
lämpötila $u$ toteuttaa $A$:n sisäpisteissä Poissonin yhtälön $-u_{xx}-u_{yy}=q$, missä $q$ 
riippuu vain polaarikoordinaatista $r$, sekä reunalla $\partial A$ annetut reunaehdot. Määrää
$u$ seuraavissa tapauksissa:
\vspace{1mm}\newline
a) \ $R_1=0,\ R_2=R,\ q(r)=Q= $ vakio, $u(R)=u_0$ \newline
b) \ $R_1=0,\ R_2=R,\ q(r)=Q= $ vakio, $u'(R)=-ku(R)$ \newline 
c) \ $R_1=0,\ R_2=R,\ q(r)=Q(1-r^2/R^2),\ u'(R)=0$ \newline
d) \ $R_1=R,\ R_2=2R,\ q(r)=0, u(R_1)=u_0,\ u'(R_2)=0$ \newline
e) \ $R_1=R,\ R_2=2R,\ q(r)=Q= $ vakio, $u(R_1)=u(R_2)=0$

\item 
Homogeenisesta materiaalista valmistetussa pallon tai pallokuoren muotoisessa kapplaeessa
\[
A = \{(x,y,z)\in\R^3 \mid R_1^2 \le x^2 + y^2 + z^2 \le R_2^2 \}
\]
lämpötila $u$ toteuttaa $A$:n sisäpisteissä Poissonin yhtälön $-\Delta u=q$, missä $q$ riippuu
vain pallokoordinaatista $r$, sekä reunalla $\partial A$ annetut reunaehdot. Määrää $u$ 
edellisen tehtävän tapauksissa.

\item (*)
Johda pallokoordinaatiston kantavektorien derivoimiskaavat
\begin{align*}
\partial_\theta\vec e_r  &= \vec e_\theta\,, \qquad\quad 
                            \partial_\theta\vec e_\theta=-\vec e_r\,, \qquad\,\
                            \partial_\theta\vec e_\varphi=\vec 0\,, \\
\partial_\varphi\vec e_r &= \sin\theta\vec e_\varphi\,, \quad
                            \partial_\varphi\vec e_\theta=\cos\theta\vec e_\varphi\,, \quad
                            \partial_\varphi\vec e_\varphi=-\sin\theta\vec e_r-\cos\theta\vec e_\theta 
\end{align*}
ja näiden avulla gradientin ja Laplacen operaattorin esitysmuodot pallokoordinaatistossa.

\item (*) \index{zzb@\nim!Keplerin laki}
(Keplerin laki) Avaruusalus, jonka massa $=m$, liikkuu tason keskeisvoimakentässä
$\vec F=\nabla(k/r)$ ($k=$ vakio, $r=$ etäisyys origosta) siten, että aluksen
polaarikoordinaatit ovat $r(t)$ ja $\varphi(t)$ hetkellä $t$. Kirjoita aluksen liikeyhtälö
$m\vec r\,''=\vec F$ differentiaaliyhtälösysteeminä funktioille $r(t)$ ja $\varphi(t)$ ja
näytä, että pätee \pain{Ke}p\pain{lerin} \pain{toinen} \pain{laki}
\[
[r(t)]^2\varphi'(t)=\text{vakio}.
\]

\end{enumerate} %Operaattorit grad, div, rot ja Laplace käyräviivaisissa koordinaatistoissa
\section[Epälineaariset yhtälöryhmät: Jacobin matriisi \\ ja Newtonin menetelmä]
{Epälineaariset yhtälöryhmät: Jacobin \\ matriisi ja Newtonin menetelmä}
\label{jacobiaani}
\sectionmark{Jacobin matriisi}
\alku

Tarkastellaan yleistä $m$ yhtälön ja $n$ tuntemattoman yhtälöryhmää muotoa
\[
\left\{ \begin{aligned}
&f_1(x_1,\ldots,x_n)\ =0 \\
&\quad\vdots \\
&f_m(x_1,\ldots,x_n)=0
\end{aligned} \right.
\]
missä funktiot $f_i\,,\ i=1 \ldots m$, ovat $n$ reaalimuuttujan reaaliarvoisia funktioita, eli 
funktioita tyyppiä $f_i\,:\ \DF_{f_i} \kohti \R,\ \DF_{f_i}\subset\R^n$. Yhtälöryhmä on 
\index{epzy@epälineaarinen yhtälöryhmä} \index{yhtzy@yhtälöryhmä!b@epälineaarinen}
\kor{epälineaarinen}, jos se ei ole lineaarinen, eli jos ainakin yksi funktioista $f_i$ on 
ei-affiininen kuvaus (vrt.\ Luku \ref{affiinikuvaukset}). Kun määritellään
\[
\mf = (f_1,\ldots,f_m), \quad \DF_\mf = \DF_{f_1}\cap\DF_{f_2}\cap\cdots\cap\DF_{f_m}\,,
\]
niin ym.\ yhtälöryhmän voi kirjoittaa kätevästi vektorimuodossa\,:
\[
\mf(\mx)=\mv{0}.
\]
Tässä siis $\mf$ on funktio (kuvaus) tyyppiä $\,\mf:\ \DF_\mf\kohti\R^m,\ \DF_\mf\subset\R^n$.
\index{funktio A!j@$n$ muuttujan vektoriarvoinen} \index{vektoric@vektoriarvoinen funktio}
\index{vektorimuuttujan funktio}%
Tämä on \kor{vektorimuuttujan vektoriarvoinen} funktio, jonka voi tulkita aiemmin Luvuissa 
\ref{lineaarikuvaukset}--\ref{affiinikuvaukset} määriteltyjen lineaari- ja affiinikuvausten
\index{vektorikenttä}%
yleistykseksi. Myös termiä \kor{vektorikenttä} (vrt.\ Luku \ref{divergenssi ja roottori}) voi
käyttää. Yhtälöryhmiä matriisialgebran keinoin käsiteltäessä tulkitaan $\mf$ tavallisesti
pystyvektoriksi. Kuten nähdään jatkossa, matriisialgebraa tarvitaan muun muassa, kun halutaan
määrittää algoritmisin keinoin ko.\ yhtälöryhmän \kor{ratkaisut}, eli joukko
\[
X=\{\mx\in\DF_\mf \mid \mf(\mx)=\mv{0}\}.
\]
Yhtälöryhmiä käytännössä (numeerisesti) ratkaistaessa on tavallisimmin $m=n$. 
\begin{Exa}
Epälineaaristen yhtälöryhmien
\[ 
\text{a)} \ \left\{ \begin{aligned} 
                    &3x+y+z+9=0 \\ &7x+y+3z+27=0 \\ &x^2+y^2+z^2-81=0 
                    \end{aligned} \right.  \qquad
\text{b)} \ \left\{ \begin{aligned} &x+y+z-16=0 \\ &x^2+y^2+z^2-81=0 \end{aligned} \right.
\]
vektorimuotoisia esitystapoja ovat
\begin{align*}
\text{a)}\ \mf(x,y,z)\,&=\,(\,3x+y+z+9,\ 7x+y+3x+27,\ x^2+y^2+z^2-81\,) \\
                       &=\,(0,0,0), \\
\text{b)}\ \mf(x,y,z)\,&=\,(x+y+z-16,\ x^2+y^2+z^2-81\,) \\
                       &=\,(0,0),
\end{align*}
tai pystyvektoreiden avulla
\begin{align*}
&\text{a)}\ \mf(x,y,z) = \begin{bmatrix} 
                         3x+y+z+9 \\ 7x+y+3z+27 \\ x^2+y^2+z^2-81 
                         \end{bmatrix} 
                       = \begin{bmatrix} 0\\0\\0 \end{bmatrix}, \\[3mm]
&\text{b)}\ \mf(x,y,z) = \begin{bmatrix} 
                         x+y+z-16 \\ x^2+y^2+z^2-81 
                         \end{bmatrix}\,=\,\begin{bmatrix} 0\\0 \end{bmatrix}.
\end{align*}
Algebrallisin (tai b-kohdassa geometrisin) keinoin voidaan todeta ratkaisuiksi
\[
\text{a)}\ X = \{\,(0,0,-9),\,(6,-6,3)\,\}, \quad \text{b)}\ X=\emptyset. \loppu
\]
\end{Exa}

\index{differentioituvuus}%
Vektoriarvoista funktiota $\mf$ sanotaan \kor{differentioituvaksi} pisteessä $\mx\in\DF_\mf$,
jos jokainen komponenttifunktio $f_i$ on ko.\ pisteessä differentioituva (Määritelmän 
\ref{differentioituvuus} mielessä). Jos $\mf$ on differentioituva, niin yhdistämällä 
komponenttifunktioiden differentiaalikehitelmät saadaan tulos
\[
\mf(\mx+\Delta\mx)=\mf(\mx)+\mJ\,\mf(\mx)\Delta\mx+\mr(\mx),
\]
missä $\mJ$ on $\mf$:n \kor{Jacobin}\footnote[2]{Saksalainen matemaatikko \vahv{K.G.J. Jacobi}  
eli vuosina 1804-1851. \index{Jacobi, K. G. J.|av}} \kor{matriisi} (engl. Jacobian)
\index{Jacobin matriisi} \index{differentiaalioperaattori!i@Jacobin matriisi}%
\[
\boxed{\quad \mJ\,\mf(\mx)=\begin{bmatrix} 
                           \dfrac{\ykehys\partial f_1}{\partial x_1}(\mx) & \ldots & 
                           \dfrac{\partial f_1}{\partial x_n}(\mx) \\ \vdots \\ 
                           \dfrac{\partial f_m}{\akehys\partial x_1}(\mx) & \ldots & 
                           \dfrac{\partial f_m}{\partial x_n}(\mx)
                           \end{bmatrix}\quad \text{(Jacobin matriisi)} \quad}
\]
ja jäännösfunktiolle $\mr$ pätee arvio
\[
\abs{\mr(\mx)} = o(\abs{\Delta\mx}), \quad \text{kun}\,\ \abs{\Delta x} \kohti 0.
\]
Tapauksessa $m=1$ Jacobin matriisi on sama kuin transponoitu gradientti, eli
$\mJ f = (\Nabla f)^T$. Tapauksessa $n=1$ Jacobin matriisi puolestaan palautuu
vektoriarvoisen yhden muuttujan funktion (tavalliseksi) derivaataksi.
Tapauksissa $m=2,3$ tämä on ennestään tuttu parametrisen käyrän
derivaattana (Luku \ref{derivaatta geometriassa}). Jacobin matriisi on siis jälleen
derivaattakäsitteen yleistys kahdellakin eri tavalla. Määritelmästä nähdään yleisemmin, että
Jacobin matriisi on esitettävissä joko gradienttien $\nabla f_i$ tai osittaisderivaattojen
$\partial_j\mf$ avulla muodoissa
\[
\mJ\mf \,=\, [\nabla f_1 \ldots \nabla f_m]^T \,=\, [\partial_1\mf \ldots \partial_n\mf].
\]
\begin{Exa} \label{yryhmä-esim}
Kun yhtälöryhmä
\[ \begin{cases}
    \,x^3-2xy^5-x-2 = 0 \\
   -x^3y+2xy^2+y^5+3 = 0
   \end{cases} \]
kirjoitetaan muotoon $\mf(x,y)=\mo$, $\ \mf=[f_1,f_2]^T$, niin Jacobin matriisi on
\[
\mJ\,\mf(x,y) = \begin{bmatrix}
\partial_x f_1 & \partial_y f_1 \\
\partial_x f_2 & \partial_y f_2
\end{bmatrix} \\
= \begin{bmatrix}
3x^2-2y^5-1 & -10xy^4 \\ -3x^2+2y^2 & -x^3+4xy+5y^4
\end{bmatrix}.
\]
Yhtälöryhmän eräs ratkaisu on $(x,y)=(2,1)$. Tässä pisteessä on
\[
\mJ\,\mf(2,1)=\begin{rmatrix} 9 & -20 \\ -10 & 5 \end{rmatrix}. \loppu
\]
\end{Exa}

Jos $\mf$ on differentioituva pisteessä $\ma$, niin approksimaatiota
\[
\mf(\mx) \approx \mf(\ma) + \mJ\mf(\ma)(\mx-\ma)
\]
\index{linearisaatio (funktion)}% \index{funktion approksimointi!ab@linearisaatiolla}%
sanotaan aiempaan tapaan (vrt.\ Luku \ref{derivaatta}) $\mf$:n \kor{linearisaatioksi} pisteessä
$\ma$. (Oikeammin kyse on 'affinisaatiosta'.) Koska approksimaation virhe on suuruusluokkaa
$\ord{\abs{\mx-\ma}}$ (differentioituvuuden määritelmän mukaisesti), niin johtopäätös on\,:
\[
\boxed{\quad\kehys \text{Differentioituva kuvaus}\ 
                 \approx\ \text{affiinikuvaus p\pain{aikallisesti}}. \quad}
\]
\jatko \begin{Exa} (jatko) Pisteen $(-1,1)\,\vastaa\,\ma=[-1,1]^T$ ympäristössä on likimain
\[
\mf(x,y)\,\approx\,\mf(-1,1)+\mJ(-1,1)(\mx-\ma)\,
     =\,\begin{rmatrix} 0\\3 \end{rmatrix}+\begin{rmatrix} 0&10\\-1&2 \end{rmatrix} 
                                           \begin{rmatrix} x+1\\y-1 \end{rmatrix}. \loppu
\]
\end{Exa}

\subsection{Newtonin menetelmä yhtälöryhmille}
\index{Newtonin menetelmä!a@yhtälöryhmälle|vahv}

Tarkastellaan yhtälöryhmää $\mf(\mx)=\mo$, missä $\mf$ on tyyppiä $\mf:\ \R^n\kohti\R^n$.
Newtonin menetelmässä tehdään yhtälöryhmässä linearisoiva approksimaatio pisteessä $\mx_k$,
eli ratkaistaan
\[
\mf(\mx_k)+\mJ(\mx_k)(\mx-\mx_k)=\mo.
\]
Kun ratkaisua merkitään $\mx=\mx_{k+1}$, on tuloksena iteraatiokaava
\[
\mJ(\mx_k)(\mx_{k+1}-\mx_k)=-\mf(\mx_k)
\]
eli
\[
\boxed{\kehys\quad \mx_{k+1}=\mx_k-\inv{\mJ(\mx_k)}\,\mf(\mx_k) \quad \text{(Newton)}. \quad}
\]
Perusajatus tässä on täsmälleen sama kuin yhden yhtälön ($n=1$) tapauksessa (vrt.\ Luku 
\ref{kiintopisteiteraatio}): kyseessä on
\index{kiintopisteiteraatio}%
\kor{kiintopisteiteraatio} muotoa
\[
\mx_{k+1}=\mF(\mx_k), \quad k=0,1,\ldots,
\]
missä $\mF(\mx)=\mx-\mJ(\mx)^{-1}\mf(\mx)$. 

Newtonin menetelmä suppenee myös yhtälöryhmien ratkaisussa samantapaisin oletuksin kuin yhden 
yhtälön tapauksessa. --- Kerrattakoon Luvusta \ref{usean muuttujan jatkuvuus}, että
vektorijonon $\seq{\mx_k}$ suppeneminen kohti vektoria $\mc\in\R^n$ tarkoittaa:
\[
\mx_k \kohti \mc \qekv \abs{\mx_k-\mc} \kohti 0 \quad (k\kohti\infty),
\]
missä $\abs{\cdot}$ on $\R^n$:n euklidinen normi. Kuten tapauksessa $n=1$, Newtonin menetelmän
\index{kvadraattinen!a@suppeneminen}%
suppeneminen on myös (riittävin oletuksin) \kor{kvadraattista} seuraavassa
konvergenssilauseessa lähemmin määriteltävällä tavalla. Lause on todistuksineen varsin
suoraviivainen yleistys Lauseesta \ref{Newtonin konvergenssi}, mutta todistuksen yksityiskohdat
ovat teknisempiä. (Todistus nojaa usean muuttujan Taylorin lauseeseen, ks.\ Luku
\ref{usean muuttujan taylorin polynomit} jäljempänä, vrt.\ myös
Harj.teht.\,\ref{taylorin lause}:\ref{H-dif-4: Newtonin konvergenssi}.) Sivuutetaan todistus.
\begin{Lause} \label{Newtonin konvergenssi -Rn}
Olkoon $\mf:\DF_\mf\kohti\R^n$, $\DF_\mf\subset\R^n$, $\mf=(f_i)$, ja olkoon $\mf(\mc)=\mo$, 
$\mc\in\R^n$. Olkoot edelleen osittaisderivaatat $\partial^\alpha  f_i(\mx)$ olemassa ja 
jatkuvia pisteen $\mc$ ympäristössä, kun $i=1\ldots n$ ja $\abs{\alpha}\leq 2$, ja olkoon 
Jacobin matriisi $\mJ\,\mf(\mc)$ säännöllinen. Tällöin Newtonin iteraatio 
$\mx_{k+1}=\mx_k-\inv{\mJ(\mx_k)}\,\mf(\mx_k)$ suppenee kohti $\mc$:tä, mikäli $\abs{\mx_0-\mc}$
on riittävän pieni, ja suppeneminen on kvadraattista:
\[
\abs{\mx_{k+1}-\mc}\leq C\abs{\mx_k-\mc}^2\quad (C=\text{vakio}).
\]
\end{Lause}
\begin{Exa} \label{Newton-2d: ex1} Etsi Newtonin menetelmällä ratkaisu yhtälöryhmälle 
\[ \begin{cases}
    \,x^3-2xy^5-x=2.04 \\
    -x^3y+2xy^2+y^5=-0.05
\end{cases} \]
pisteen $(-1,1)$ läheltä. 
\end{Exa}
\ratk Tässä on
\[
\mf(x,y)= \begin{bmatrix} x^3-2xy^5-x-2.04 \\ -x^3y+2xy^2+y^5+0.05 \end{bmatrix}
\]
ja Jacobin matriisi $\mJ(x,y)$ on sama kuin Esimerkissä \ref{yryhmä-esim}. Iteraatiokaavan
\[
[x_{k+1},\,y_{k+1}]^T = [x_k,\,y_k]^T -\mJ(x_k,y_k)^{-1}\mf(x_k,y_k)
\]
mukaan laskien saadaan
\begin{align*}
                       (x_0,y_0)\ &=\ (-1,\,1) \\
                       (x_1,y_1)\ &=\ (-0.942,1.004) \\
                       (x_2,y_2)\ &=\ (-0.918940,1.006156) \\
                       (x_3,y_3)\ &=\ (-0.914247,1.006622) \\
                       (x_4,y_4)\ &=\ (-0.914214,1.006633) \\
                       (x_5,y_5)\ &=\ (-0.914216,1.006633) \\
                                  &\vdots \quad \loppu
\end{align*}

Esimerkissä Newtonin iteraatio suppeni nopeasti, koska alkuarvaus oli lähellä ratkaisua. Ellei
hyvää alkuarvausta ole käytettävissä, voidaan ensin koettaa paikallistaa ratkaisut jollakin 
yksinkertaisemmalla menetelmällä. Esimerkiksi jos yhtälöryhmä ei ole kooltaan suuri, voidaan 
kokeilemalla tutkia, missä funktio
\[
F(\mx)=\sum_{i=1}^n [f_i(\mx)]^2
\]
saa pieniä arvoja. Vieläkin suoraviivaisempi on 'yrityksen ja erehdyksen' menetelmä: Iteroidaan
\index{suppenemisallas}%
vaihtelevilla alkuarvauksilla, kunnes sattumalta osutaan nk.\ \kor{suppenemisaltaaseen}, eli 
riittävän lähelle jotakin ratkisua. Mikäli iteraatio ei suppene esimerkiksi kymmenen kierroksen
jälkeen, vaihdetaan alkuarvausta.
\begin{Exa}
Ratkaise yhtälöryhmä
\[
\begin{cases}
\,x^2+y^2= 1 \\ \,e^x+\sin y=1
\end{cases}
\]
\end{Exa}
\ratk Tässä on
\[
\mf(x,y) = \begin{bmatrix} x^2+y^2-1 \\ e^x+\sin y-1 \end{bmatrix}, \qquad
\mJ\,\mf(x,y) = \begin{bmatrix} 2x & 2y \\ e^x & \cos y \end{bmatrix},
\]
joten Newtonin iteraatio saa muodon
\[
\begin{bmatrix} x_{k+1} \\ y_{k+1} \end{bmatrix} =
\begin{bmatrix} x_k \\ y_k \end{bmatrix} - 
\inv{\begin{bmatrix} 2x_k & 2y_k \\ e^{x_k} & \cos y^k \end{bmatrix}}
\begin{bmatrix} x_k^2+y_k^2-1 \\ e^{x_k}+\sin y_k-1\end{bmatrix}.
\]
Ratkaisuja on kaksi:
\begin{align*}
A:\quad &(x,y)=(\phantom{-}0.5538..,\, -0.8327..\,) \\
B:\quad &(x,y)=(-0.8089..,\phantom{-}\,0.5879..\,)
\end{align*}
Seuraavassa tuloksia alkuarvauksista riippuen. Merkintä * tarkoittaa epäonnistumista 
(singulaarinen Jacobin matriisi tai iteraation hajaantuminen).
\[
\begin{array}{lll}
(\phantom{-}1,\phantom{-}0)\kohti A\qquad & 
(\phantom{-}0\phantom{.0},\phantom{-}0)\kohti * \qquad & 
(-\phantom{1}5,\phantom{-}0\phantom{.0})\kohti B \\ 
(\phantom{-}1,\phantom{-}1)\kohti B\qquad & 
(\phantom{-}0.1,\phantom{-}0)\kohti A \qquad & 
(-10,\phantom{-}0\phantom{.0})\kohti B \\
(-1,\phantom{-}1)\kohti B\qquad & 
(-0.1,\phantom{-}0)\kohti * \qquad & 
(\phantom{-1}3,\phantom{-}3\phantom{.0})\kohti B \\
(-1,-1)\kohti B\qquad & 
(\phantom{-}5\phantom{.0},\phantom{-}0)\kohti * \qquad & 
(\phantom{-1}3,-3\phantom{.0})\kohti A \\
(\phantom{-}0,-1)\kohti A\qquad & 
(\phantom{-}0\phantom{.0},\phantom{-}5)\kohti B \qquad & 
(\phantom{-1}3,-1.5)\kohti A \\
(\phantom{-}1,-1)\kohti A\qquad & 
(\phantom{-}0\phantom{.0},-5)\kohti A \qquad & 
(\phantom{-1}3,-1.2)\kohti B \\
\end{array} \loppu
\]

\subsection{Suuret yhtälöryhmät. Jatkamismenettely}
\index{jatkamismenettely (algoritmi)|vahv}

Newtonin menetelmää käytetään yleisesti myös hyvin suurien epälineaaristen yhtälöryhmien
ratkaisuun. Tällaiset ongelmat syntyvät usein osana laajempaa laskentaprosessia, jolloin
on myös tavallista, että hyvä alkuarvaus on tarjolla laskennan aikaisemmista vaiheista.
Näin on  esimerkiksi ratkaistaessa differentiaaliyhtälösysteemeitä numeerisilla
askelmenetelmillä (ks.\ Luku \ref{DYn numeeriset menetelmät}). 

Jos halutaan vain ratkaista yksittäinen suurikokoinen yhtälöryhmä ilman ennakkotietoa
ratkaisun (ratkaisujen) sijainnista, voi toimivan alkuarvauksen löytäminen Newtonin
iteraatiolle olla hyvin vaikeaa. Tällaisessa tilanteessa varsin yleisesti käytetty on nk.
\kor{jatkamismenettely} (engl.\ continuation method), jossa ratkaistava ongelma 
parametrisoidaan muotoon
\[
P(t),\quad 0\leq t<1.
\]
Oletetaan, että $P(1)$ on ongelma, joka varsinaisesti halutaan ratkaista, ja että $P(0)$ on 
helppo ratkaista. Tällöin jatkamismenettelyn ideana on ratkaista peräkkäin
\[
P(t_0),P(t_1),\ldots,P(t_N),\quad t_N=1,
\]
missä (esimerkiksi) $t_k=k\Delta t$. Jos $\Delta t$ on pieni ja parametrisointi suoritettu 
hyvin, on probleeman $P(t_{k-1})$ ratkaisu hyvä alkuarvaus Newtonin iteraatiolle, kun 
ratkaistaan probleemaa $P(t_k)$. Näin voidaan askelittain edetä lopputilaan
$t_N=1$.\footnote[2]{Jatkamismenettelyyn perustuvan algoritmin suunnittelussa on yleensä
apua probleeman fysikaalisesta taustasta. Esimerkiksi rakenteiden mekaniikan ongelmassa, jossa
halutaan simuloida rakenteen suuria muodonmuutoksia kuormituksen alaisena, on luontevaa
menetelllä kuten koejärjestelyssä: Parametrisoidaan kuormituksen voimakkuus, eli lähdetään
kuormittamattomasta tilasta ja lisätään kuormitusta pienin askelin, kunnes päädytään haluttuun
lopputilaan. Parametrin $t$ voi tällöin kuvitella koejärjestelyyn liittyväksi 
aikamuuttujaksi. Aikamuuttujan lisääminen (aikaparametrisointi) on luonnollista
yleisemminkin silloin, kun fysiikasta peräisin olevaan ongelmaan etsitään ajasta riippumatonta
(tasapaino)ratkaisua.}
Ellei parempaa ideaa ole käytettävissä, voi yleisen yhtälöryhmän parametrisoida vaikkapa
muunnoksella muotoa
\[
\mf(\mx)=\mo\ \ext\ \\ \mg(\mx)+t\,[\,\mf(\mx)-\mg(\mx)\,]=\mo,
\]
missä $\mg(\mx)=\mo$ on jokin helposti ratkeava (esim.\ lineaarinen) yhtälöryhmä.
\begin{Exa} Yhtälöryhmälle %[ratk: (x,y)=(5,4)]
\[
\left\{ \begin{aligned}
        &x^3-2y^3+x+2y=10 \\ &x^3+y^3-2xy^2-3x-y=10
        \end{aligned} \right.
\]
antaa Newtonin algoritmi alkuarvauksilla $(x_0,y_0)=(1,1)$ ja $(x_0,y_0)=(-1,-1)$ ratkaisun
$(x,y)=(-2.25872,-2.42834)$. Kokeillaan, löytyykö jokin muu ratkaisu jatkamismenettelyllä:
Parametrisoidaan yhtälöryhmä muotoon
\[
\mf(t,\mx) = \begin{bmatrix}
             t(x^3-2y^3)+x+2y \\ t(x^3+2y^3-2xy^2)-3x-y
             \end{bmatrix}
           = \begin{bmatrix} 10\\10 \end{bmatrix}
\]
ja ratkaistaan tämä parametrin arvoilla $t_k=0.2k,\ k=0 \ldots 5$. Kun $k=0$, on ratkaisu
$(x,y)=(-6,8)$. Tästä eteenpäin ($k=1 \ldots 5$) käytetään edellisellä $k$:n arvolla saatua
ratkaisua alkuarvauksena Newtonin iteraatiolle. Tulos:
\begin{center}
\begin{tabular}{ll}
$t$   & $\text{Ratkaisu}$  \\ \hline \\
$0.2$ & $(10.5223,8.55714)$ \\
$0.4$ & $(7.60605,6.14808)$ \\
$0.6$ & $(6.30833,5.07754)$ \\
$0.8$ & $(5.53177,4.43769)$ \\
$1.0$ & $(5.00000,4.00000)$
\end{tabular}
\end{center}
Ratkaisu $(x,y)=(5,4)$ löytyy myös alkuarvauksia vaihtelemalla. Esim.\ $(x_0,y_0)=(3,3)$ tai
jopa $(x_0,y_0)=(0,0)$ johtavat tähän. \loppu
\end{Exa}

\subsection{Yksinkertaistettu Newtonin iteraatio}
\index{Newtonin menetelmä!a@yhtälöryhmälle|vahv}
 
Epälineearista yhtälöryhmää Newtonin menetelmällä ratkaistaessa on jokaisella 
iteraatiokierroksella suoritettava seuraavat laskuoperaatiot:
\begin{itemize}
\item[-] Funktioevaluaatio: $\ \mx_k\ \map\ \mf(\mx_k)=\my_k$.
\item[-] Jacobin matriisin \kor{päivitys} (engl.\ updating) 
         $\ \mx_k\ \map\ \mJ\mf(\mx_k)=\mA_k$. \index{pzy@päivitys (Jacobin matriisin)}
\item[-] Lineaarisen yhtälöryhmän ratkaisu: 
         $\ \my_k\map\mpu_k=\mA_k^{-1}\my_k\ (\mA_k\mpu_k=\my_k)$.
\item[-] Vähennyslasku: $\ \mx_k\,,\mpu_k\ \map\ \mx_k-\mpu_k=\mx_{k+1}$.
\end{itemize}
Sekä Jacobin matriisin päivittäminen että etenkin lineaarisen yhtälöryhmän ratkaiseminen ovat
suhteellisen raskaita operaatioita, jos yhtälöryhmän koko on suuri. Hyvin suuria yhtälöryhmiä
ratkaistaessa saatetaankin päivittämisessä hieman 'laiskotella', ts.\ päivitystä ei suoriteta
jokaisella iteraatiokierroksella. Tämä helpottaa myös lineaaristen yhtälöryhmien (suoraa) 
ratkaisemista, koska yhtälöryhmien kerroinmatriisi pysyy samana päivityksien välissä. Riittää
silloin laskea Jacobin matriisin $LU$-hajotelma jokaisen päivityksen jälkeen ja pitää se tallessa
seuraavaan päivitykseen asti (vrt.\ Luku \ref{Gaussin algoritmi}). Riippuen siitä, kuinka usein
päivitys suoritetaan, saadaan Newtonin menetelmälle erilaisia variaatioita, joista voidaan 
valita ongelmakohtaisesti tehokkain.

Em.\ variaatioista yksinkertaisimmassa lasketaan Jacobin matriisi vain kerran,
alkuarvauspisteessä. Kun merkitään $\mA=\mJ(\mx_0)$, niin iteraatiokaavaksi tulee
\[
\mx_{k+1}=\mx_k-\mA^{-1}\mf(\mx_k), \quad k=0,1\ldots
\]
%tai algoritmisemmin kirjoitettuna
%\begin{align*}
%&\mx_k\map\mb_k=\mf(\mx_k), \\
%&\mb_k\map\mpu_k\,:\ \mA\mpu_k=\mb_k, \\ 
%&\mx_{k+1}=\mx_k-\mpu_k\,, \quad k=0,1,\ldots
%\end{align*}
Tämä Newtonin menetelmän äärimmäinen yksinkertaistus ei ole aidon Newtonin menetelmän veroinen
käytännössä, sillä se suppenee (sikäli kuin suppenee) vain lineaarisesti 
(vrt.\ Harj.teht.\,\ref{kiintopisteiteraatio}:\,\ref{H-V-7: yksinkertaistettu Newton},
kun $n=1$). --- Sen sijaan teoreettisten tarkastelujen 'ajattelumenetelmäksi' tämä menetelmä
sopii hyvin, ks.\ seuraava luku. 
\begin{Exa} Esimerkin \ref{Newton-2d: ex1} tilanteessa on
\[
\mA=\mJ\mf(-1,1)= \begin{rmatrix} 0&10\\-1&2 \end{rmatrix}, \quad
\mA^{-1}=\frac{1}{10}\begin{rmatrix} 2&-10\\1&0 \end{rmatrix},
\]
joten ym.\ tavalla yksinkertaistetuksi Newtonin iteraatiokaavaksi tulee
\[
\begin{bmatrix} x_{k+1}\\y_{k+1} \end{bmatrix} 
   = \begin{bmatrix} x_k\\y_k \end{bmatrix}
         -\frac{1}{10}\begin{rmatrix} 2&-10\\1&0 \end{rmatrix}
         \left[\begin{array}{l} 
               x_k^3-2x_ky_k^5-x_k-2.04\\-x_k^3y_k+2x_ky_k^2+y_k^5-0.05 
               \end{array}\right].
\]
Iteraatio suppenee, mutta selvästi aitoa Newtonin iteraatiota hitaammin
(vrt.\ Esimerkki \ref{Newton-2d: ex1}):
\begin{align*}
\qquad (x_0,y_0)\ &=\ (-1,\,1) \\
       (x_1,y_1)\ &=\ (-0.942,1.004) \\
       (x_2,y_2)\ &=\ (-0.929318,1.005191) \\
       (x_3,y_3)\ &=\ (-0.923142,1.005780) \\
       (x_4,y_4)\ &=\ (-0.919696,1.006109) \\
       (x_5,y_5)\ &=\ (-0.917651,1.006304) \\
%      (x_6,y_6)\ &=\ (-0.916395,1.006424) \\
%      (x_7,y_7)\ &=\ (-0.915608,1.006500) \\
                  &\ \vdots \loppu
\end{align*}
%Iteraatio suppenee, mutta selvästi aitoa Newtonin iteraatiota hitaammin. \loppu
\end{Exa}

\Harj
\begin{enumerate}

\item
Laske seuraavissa tapauksissa $\mf$:n Jakobin matriisi ensimmäisessä annetussa pisteessä,
approksimoi $\mf$ affiinikuvauksella ja laske tämän avulla $\mf$:n arvo likimäärin toisessa
annetussa pisteessä. Vertaa tarkkaan arvoon. \vspace{1mm}\newline
a) \   $\mf(x,y)=(x^3+y^3,x^3-y^3),\,\ (1,1),\,\ (0.9,1.1)$ \newline
b) \   $\mf(x,y)=(e^x\sin\pi y,e^{-x}\cos\pi y),\,\ (0,1/3),\,\ (-0.04,0.30)$ \newline
c) \,\ $\mf(x,y,z)=(x^2y,x^2z,y^2-z^2),\,\ (1,3,3)\,\ (0.99,,3.02,2.97)$ \newline
d) \   $\mf(x,y,z)=(x^2+yz,y^2-x\ln z),\,\ (2,2,1),\,\ (1.98,2.01,1.03)$ \newline
e) \,\ $\mf(\mx)=(x_1x_2,\,x_1^2-x_3^2,\,x_2^2+x_3x_4,\,x_2x_4+x_4^2),\,\ (1,2,-2,-1),$ \newline 
       \phantom{e \,\ }  $(1.01,2.02,-2.02,-1.01)$
 
\item 
Yhtälöryhmällä
\[
[\mf(x,y,z)]^T \,=\, [\,x^3 - x^2 y + x z^3,\,xyz + z^2,\,x^3 + y^3 + z^3\,] \,=\, \mb^T
\]
on muuan helppo ratkaisu, kun $\mb^T=[1,2,3]$. Approksimoimalla $\mf$ affiinikuvauksella ko.\ 
ratkaisupisteen lähellä laske yhtälöryhmän ratkaisu likimäärin, kun $\mb=[1.04,1.98,3.02]^T$.

\item
Laske koordinaattimuunnoksien $(r,\varphi,z)\map(x,y,z)$ (lieriök. $\ext$ karteesinen) ja 
$(r,\theta,\varphi)\map(x,y,z)$ (pallok.\ $\ext$ karteesinen) Jacobin matriisit 
$\mJ(r,\varphi,z)$ ja $\mJ(r,\theta,\varphi)$. Missä pisteissä $\mJ$ on singulaarinen matriisi?

\item
Olkoon $\mf$ tyyppiä $\mf: \R^n\kohti\R^p$ ja $\mg$ tyyppiä $\mg: \R^p\kohti\R^m$. Johda
yhdistetyn funktion $\mF(\mx)=(\mg\circ\mf)(\mx)=\mg[\mf(\mx)]$ Jacobin matriisin laskusääntö
$\mJ\mF(\mx)=\mJ\mg[\mf(\mx)]\mJ\mf(\mx)$. \kor{Vihje}: Ketjusääntö!

\item
Etsi seuraaville yhtälöryhmille ratkaisu Newtonin menetelmällä lähtien alkuarvauksesta
$(x_0,y_0)=(1,1)$.
\[
\text{a)}\,\ \begin{cases} \,x^4+y^4=2xy^5\\ \,x+x^2+y^4=4 \end{cases} \quad
\text{b)}\,\ \begin{cases} \,x^3+3x=y^4+4y \\ \,\cos\pi x+xye^{-y}=0 \end{cases}
\]

\item
Etsi seuraavien yhtälöryhmien ratkaisut neljän merkitsevän numeron tarkkuudella käyttäen
kaksiulotteista Newtonin menetelmää. Laske a)- ja b)-kohdissa ratkaisu myös vaihtoehtoisella
tavalla, jossa eliminoidaan ensin toinen tuntemattomista ja käytetään 1-ulotteista Newtonin
menetelmää. \vspace{1mm}\newline
a) \ $y-e^x=0,\,\ x-\sin y=0$ \newline
b) \ $x^2+y^2=1,\,\ y=e^x\ $ (kaksi ratkaisua) \newline 
c) \ $y-\sin x=0,\,\ x^2+(y+1)^2=2\ $ (kaksi ratkaisua) \newline
d) \ $x^2-xy+2y^2=10,\,\ x^3y^2=2\ $ (neljä ratkaisua) \newline
e) \ $\sin x+\sin y=1,\,\ x^3=y^2<30\ $ (neljä ratkaisua)

\item 
Selvitä (kaksiulotteista) Newtonin menetelmää käyttäen, missä pisteissä Cartesiuksen lehden
$x^3 + y^3 = 3xy$ tangentin ja jonkin koordinaatiakselin välinen kulma on $45\aste$.

\item
Määritä likimäärin seuraavien yhtälöryhmien ratkaisut eliminoimalla ensin yksi tuntemattomista
ja käyttämällä sen jälkeen Newtonin menetelmää.
\[
\text{a)}\ \ \begin{cases} x^2+y^2+z^2=1\\z=xy\\6xz=1 \end{cases} \
\text{b)}\ \ \begin{cases} x^2+y^2+z^2=1\\y=\sin z\\z+z^2+z^3=x+y \end{cases} \
\text{c)}\ \ \begin{cases} x^2+y^4=1\\z=x^3y\\e^x=2y-z \end{cases}
\]

\item
Yhtälöryhmällä
\[
x+yze^x \,=\, x^3+y^3z \,=\, y+e^{xyz} \,=\, 2.1
\]
on ratkaisu pisteen $(0,1,2)$ lähellä. Iteroi tästä alkuarvauksesta kolme kertaa Newtonin
menetelmän yksinkertaistetulla versiolla, jossa Jacobin matriisia ei päivitetä. Vertaa oikean 
Newtonin iteraation antamiin tuloksiin.

\item (*)
Etsi jatkamismenettelyä ja Newtonin menetelmää käyttäen jokin ratkaisu yhtälöryhmälle
\[
\left\{ \begin{aligned}
        &3x^3-y^3+z^3+x-2z+40=0 \\ 
        &4x^3-3y^3-z^3-2x-3y-40=0 \\ 
        &x^3+y^3+z^3-5x-5y+5z-40=0
        \end{aligned} \right.
\] 
% Eräs atkaisu on (5,6,-6).)

\item (*)
Määritä suora, joka sivuaa kahdessa eri pisteessä käyrää
\[
K:\ x^2y^2-2xy^3+y^4-4x^2y+6xy^2-2y^3+4x^2-3y^2-12x+y=20.
\]
% Käyrän yhtälö: 4x+3y+24=(x-y-1)^2(y-2)^2. S: 4x+3y+24=0. Sivuamispisteet (-3,-4) ja (-7.5,2).

\item (*)
(Avaruuspysäköinti) Tähtien välisen avaruden pisteessä $P=(x,y,z)$ (pituusyksikkö = valovuosi)
on avaruusasema. Avaruusasemaan vaikuttaa lähimpien tähtien vetovoima $\vec G=-k\nabla u$,
missä $k$ on vakio ja
\[
u(x,y,z) = \sum_{i=1}^n \frac{m_i}{|P-P_i|}\,,
\]
missä $m_i$ on tähden $T_i$ massa ja $P_i$ sijainti. Määritä avaruuspysäköintiä varten kaikki
pisteet $P$, joissa $\vec G=\vec 0$, kun $m_i=iM,\ i=1 \ldots n\,$ ($M=$ vakio)
ja \vspace{1mm}\newline
a) \ $n=3,\,\ P_1=(0,0,0),\ P_2=(1,0,0),\ P_3=(0,1,0)$, \newline
b) \ $n=4,\,\ P_1=(0,0,0),\ P_2=(1,0,0),\ P_3=(0,1,0),\ P_4=(0,0,1)$. 
 
\end{enumerate} %Epälineaariset yhtälöryhmät: Jacobin matriisi ja Newtonin menetelmä
\section[Käänteisfunktiolause. Implisiittifunktiolause. \\ Kontraktiokuvauslause]
{Käänteisfunktiolause. Implisiittifunktiolause. \\ Kontraktiokuvauslause}
\label{käänteisfunktiolause} 
\sectionmark{Käänteisfunktiolause}
\alku

Tässä luvussa tarkastellaan peruskysymystä epälineaarisen yhtälöryhmän ratkeavuudesta, kun 
yhtälöitä ja tuntemattomia on yhtä monta. Olkoon $\mf(\mx)=(f_1(\mx),\ldots,f_n(\mx))$, \
$\mx\in\DF_\mf\subset\R^n$ ja tarkastellaan yhtälöryhmää
\begin{equation} \label{epälin yryhmä}
\mf(\mx)=\my, \tag{$\star$}
\end{equation}
missä $\my\in\R^n$. Sikäli kuin yhtälöryhmällä on yksikäsitteinen ratkaisu $\mx\in\DF_\mf$
jollakin $\my$, voidaan ratkaisu muodollisesti kirjoittaa
\[
\mx = \mf^{-1}(\my),
\]
\index{kzyzy@käänteisfunktio} \index{funktio B!i@käänteisfunktio}%
missä $\mf^{-1}(\my)=(g_1(\my),\ldots,g_n(\my))$ on $\mf$:n \kor{käänteisfunktio}. Jos
yhtälöryhmällä on yksikäsitteinen jokaisella $\my \in B$ ($B\subset\R^n$) ja
$A=\mf^{-1}(B)=\{\mx\in\DF_\mf\ | \ \mf(\mx) \in B\}$, niin $\mf: A \kohti B$ on bijektio.
\begin{Exa} \label{udif-7: esim 1} Olkoon $\mf(x,y)=(x+y,x^2-y^2),\ \mx=(x,y)\in\R^2$. Kun
merkitään $\my=(u,v)$, niin yhtälöryhmä \eqref{epälin yryhmä} on
\[
\begin{cases} \,x+y =u, \\ \,x^2-y^2 =v. \end{cases}
\]
Jälkimmäisen yhtälön muodosta $u(x-y)=v$ nähdään, että jos $u \neq 0$, niin yhtälöryhmällä on
yksikäsitteinen ratkaisu $x=\frac{1}{2}(u^2+v)/u$, $y=\frac{1}{2}(u^2-v)/u$. Suoralla
$S:\,x+y=0$ on $\mf(x,y)=(0,0)$, joten tällä suoralla $\mf$ ei ole 1-1. Päätellään, että jos
$A=\{(x,y)\in\R^2\ | \ x+y \neq 0\}\,$ ja $\,B=\{(u,v)\in\R^2\ |\ u \neq 0\}$, niin
$\mf:\,A \kohti B$ on bijektio ja
\[
\inv{\mf}(\my) \,=\, \left(\frac{u^2+v}{2u}\,,\,\frac{u^2-v}{2u}\right), \quad 
                     \my = (u,v) \in B. \loppu
\]
\end{Exa}

\subsection{Paikallinen käänteisfunktio. Käänteisfunktiolause}

Esimerkissä \ref{udif-7: esim 1} yhtälöryhmä \eqref{epälin yryhmä} ratkesi täydellisesti.
Tavallisemmin epälineaarisen yhtälöryhmän täydellinen ratkaiseminen on vaikeaa ellei
mahdotonta, eikä myöskään mitään lineaaristen yhtälöryhmien teoriaan
verrattavaa yleisempää ratkeavuusteoriaa ole. Sen sijaan voidaan suhteellisen yleisin ehdoin
selvittää kysymys yhtälöryhmän p\pain{aikallisesta} ratkeavuudesta seuraavan (sovelluksissa
tyypillisen, vrt.\ edellinen luku) ongelman asettelun mukaisesti: Jos $\ma \in \DF_\mf$ ja
$\mb=\mf(\ma)$, niin onko (ja millä ehdoilla) yhtälöryhmällä \eqref{epälin yryhmä}
yksikäsitteinen ratkaisu \pain{lähellä} \pain{$\ma$:ta}, kun $\my$ on samoin \pain{lähellä}
\pain{$\mb$:tä}\,?
\begin{Def} \label{paikallinen kääntyvyys} \index{paikallinen käänteisfunktio|emph}
\index{kzyzy@käänteisfunktio!a@paikallinen|emph}
Funktio $\mf:\DF_f\kohti\R^n$, $\DF_\mf\subset\R^n$, on \kor{paikallisesti kääntyvä} 
(engl.\ locally invertible) pisteessä $\ma\in D_\mf$, jos $\exists\delta>0$ siten, että $\mf$
on määritelty ympäristössä $U_\delta(\ma)=\{\mx\in\R^n \ | \ \abs{\mx-\ma}<\delta\}$ ja on
tähän ympäristöön rajoitettuna kääntäen yksikäsitteinen kuvaus. Käänteisfunktiota
\[
\mx=\inv{\mf}(\my) \ \ekv \ \my=\mf(\mx) \ \ja \ \mx\in U_\delta(\ma)
\]
sanotaan $\mf$:n \kor{paikalliseksi} (lokaaliksi) \kor{käänteisfunktioksi}.
\end{Def}
\jatko \begin{Exa} (jatko) Ratkaisemalla yhtälöryhmä $\mf(\mx)=\my$ todettiin esimerkin
funktio $\mf$ kääntyväksi rajoitettuna joukkoon $A=\{(x,y)\in\R^2\ | \ x+y \neq 0\}$.
Myös Määritelmän \ref{paikallinen kääntyvyys} mukaisesti $\mf$ on paikallisesti kääntyvä
jokaisessa pisteessä $\ma = (a,b) \in A$, sillä $A$ on avoin joukko, jolloin
$\forall (a,b) \in A\ \exists \delta>0$ siten, että $U_\delta(a,b) \subset A$, jolloin $\mf$ on
tähän ympäristöön rajoitettuna 1--1. (Tarkemmin pätee tässä: $U_\delta(a,b) \subset A$
täsmälleen kun $\delta \le d$, missä $d=|a+b|/\sqrt{2}$, on pisteen $P=(a,b)$ etäisyys suorasta
$S: x+y=0$.) \loppu
\end{Exa}

Yhden reaalimuuttujan funktion tapauksessa paikalliselle kääntyvyydelle voidaan helposti antaa
riittävät ehdot derivaatan avulla. Oletetaan:
\begin{itemize}
\item[(i)]   $f$ on derivoituva välillä $(a-\delta,a+\delta)$ jollakin $\delta>0$.
\item[(ii)]  $f'(x)$ on jatkuva pisteesä $x=a$.
\item[(iii)] $f'(a) \neq 0$.
\end{itemize}
Näiden oletuksien perusteella on joko $f'(x)>0$ tai $f'(x)<0$ välillä $[a-\rho,a+\rho]$
jollakin $\rho\in(0,\delta)$ (Propositio \ref{jatkuvan funktion jäykkyys}), jolloin $f$ on
ko.\ välillä aidosti monotoninen (Lause \ref{monotonisuuskriteeri}) ja siis 1-1. Ehdot
(i)--(iii) ovat siis riittävät $f$:n paikalliselle kääntyvyydelle pisteessä $a$. Koska $f$ on
näillä ehdoilla sekä aidosti monotoninen että jatkuva välillä $[a-\rho,a+\rho]$, niin
päätellään myös (Lause \ref{ensimmäinen väliarvolause}): Yhtälöllä $f(x)=y$ on yksikäsitteinen
ratkaisu välillä $[a-\rho,a+\rho]$ aina kun $|y-f(a)|\le\eps$, missä
$\eps \,=\, \min\{\,|f(a)-f(a-\rho)|,\,|f(a)-f(a+\rho)|\,\} \,>\, 0$. Edelleen ehdoilla
(i)---(iii) pätee myös (Lause \ref{käänteisfunktion derivoituvuus}, $\delta=\rho$): $f$:n
paikallinen käänteisfunktio $\inv{f}$ on derivoituva pisteessä $b=f(a)$ ja
$\dif\inv{f}(b)=1/f'(a)$.
\begin{Exa} Funktio $f(x)=x^2$ täyttää oletukset (i)--(iii) pisteissä $a \neq 0$, joten näissä
pisteissä $f$ on paikallisesti kääntyvä. --- Tämä on selvää muutenkin, sillä $f$ tiedetään
kääntyväksi väleillä $(-\infty,0]$ ja $[0,\infty)$. Välillä $(-\delta,\delta)$ ei $f$ ole 1-1
millään $\delta>0$, joten $f$ ei ole paikallisesti kääntyvä pisteessä $a=0$. Funktio
$f(x)=x^3$ on sen sijaan 1-1 välillä $(-\infty,\infty)$, joten $f$ on paikallisesti kääntyvä
jokaisessa pisteessä, myös pisteessä $a=0$, jossa $f'(a)=0$. \loppu
\end{Exa}
Esimerkistä nähdään, että tapaus $f'(a)=0$ on $f$:n paikallisen kääntyvyyden kannalta avoin
tapaus, eli $f$ voi olla paikallisesti kääntyvä pisteessä $a$ tai ei.

\vspace{3mm}

Edellä mainituilla väittämillä, perustuen oletuksiin (i)--(iii), on myös useamman muuttujan
funktioita koskevat vastineet, jotka seuraavassa kootaan yhdeksi lauseeksi.
Tämän \kor{Käänteisfunktiolauseen} todistus nojaa keskeisesti toiseen huomattavaan lauseeseen,
\kor{Kontraktiokuvauslauseeseen}, joka esitetään ja todistetaan jäljempänä.
Käänteiskuvauslauseen todistus esitetään vasta luvun lopussa.
\begin{*Lause} \label{käänteiskuvauslause} \index{Kzyzy@Käänteisfunktiolause|emph}
\vahv{(Käänteisfunktiolause)} Oletetaan, että funktio $\,\mf:\DF_\mf\kohti\R^n$,
$\DF_\mf\subset\R^n$ täyttää ehdot: (i) $\exists\delta>0$ siten, että $f$ on differentioituva
pisteen $\ma$ ympäristössä $U_\delta(\ma)\subset\DF_\mf$. (ii) Osittaisderivaatat
$\partial f_i(\mx)/\partial x_j,\ i,j=1 \ldots n\,$ ovat jatkuvia pisteessä $\mx=\ma$.
(iii) Jacobin matriisi $\mJ\mf(\mx)=(\partial f_i(\mx)/\partial x_j)$ on säännöllinen
pisteessä $\mx=\ma$.
Tällöin on olemassa $\rho\in(0,\delta)$ ja $\eps>0$ siten, että $\mf$ on 1-1 joukossa
$K=\{\mx\in\R^n \ | \ |\mx-\ma| \le \rho\}$ ja yhtälöryhmällä $\,\mf(\mx)=\my\,$ on ratkaisu
$\mx \in K$ aina kun $|\my-\mb|\le\eps$, missä $\mb=f(\ma)$. Lisäksi $\mf$:n paikallinen
käänteisfunktio $\inv{\mf}$ on differentioituva pisteessä $\mb$ ja
$\mJ\,\inv{\mf}(\mb)=\inv{[\mJ\,\mf(\ma)]}$.
\end{*Lause}
Kuviossa on joukot $\,B=\{\my\in\R^n\ | \ |\my-\mb|\le\eps\}\,$ ja $\,\inv{\mf}(B)$ rajattu
yhtenäisellä viivalla ja joukot $K$ ja $\mf(K)$ pisteviivalla. Lauseen väittämän mukaisesti
paikallinen käänteisfunktio $\mf^{-1}$ on määritelty $B$:ssä ja $\mf^{-1}(B) \subset K$.
\begin{figure}[H]
\setlength{\unitlength}{1cm}
\begin{center}
\begin{picture}(14,6)(0,0.5)
\curvedashes[2mm]{0,1,2}
\put(3,3){\bigcircle{6}}
\curvedashes{}
\put(12,3){\circle{3}}
\curvesymbol{$\scriptscriptstyle{\bullet}$}
\put(3,3){\bigcircle[-2]{4}}
\curve(3.9,4.2,7.5,5,11.06,4.18)
\curve(4.08,4.65,7.5,5.5,11.4,4.82)
\put(7.5,5.7){$\mf$} \put(7.5,4.6){$\mf$}
\put(2.9,2.9){$\bullet$} \put(11.9,2.9){$\bullet$}
\put(3,3){\vector(3,1){2.8}} \put(12,3){\vector(1,2){0.68}} \put(3,3){\vector(3,-2){1.65}}
\put(2.9,2.5){$\ma$} \put(11.8,2.5){$\mb$} \put(12.2,4.0){$\eps$} 
\put(5.4,3.9){$\delta$} \put(4.3,2.3){$\rho$}
\Thicklines
\put(11.22,4.89){\vector(2,-1){0.2}}
\put(10.87,4.27){\vector(2,-1){0.2}}
\thinlines
\renewcommand{\yscale}{1.3}
\put(12,3){\bigcircle[-2]{3}}
\renewcommand{\yscale}{1.9}
\renewcommand{\yscalex}{0.5}
\put(3,3){\bigcircle{2}}
\end{picture}
\end{center}
\end{figure}
\jatko\jatko \begin{Exa} (jatko) Esimerkissä $\mf$:n paikallinen kääntyvyys voidaan selvittää
myös ratkaisematta yhtälöryhmää \eqref{epälin yryhmä}\,: Lauseen \ref{käänteiskuvauslause}
mukaan $\mf$ on paikallisesti kääntyvä pisteessä $(a,b)$ aina kun $a+b \ne 0$, sillä
\[
\mJ\mf(a,b) = \begin{rmatrix} 1&1\\2a&-2b \end{rmatrix} 
              \impl\quad \det\mJ(a,b)=-2(a+b) \neq 0,\ \ \text{kun}\ a+b \neq 0. \loppu
\]
\end{Exa} \seur
\begin{Exa} (Vrt.\ Esimerkki \ref{yryhmä-esim} edellisessä luvussa.) \, Käänteisfunktiolauseen
mukaan yhtälöryhmällä
\[
\mf(x,y)=\my \ \ekv \ \begin{cases}
\ x^3-2xy^5-x= u \\
-x^3y+2xy^2+y^5= v
\end{cases}
\]
on pisteen $\,\ma=(2,1)$ lähiympäristössä yksikäsitteinen ratkaisu, kun $|\my-\mb|\le\eps$,
missä $\mb=(2,3)$ ja $\eps>0$ on riittävän pieni. Pisteen $\mx=\ma$ lähiympäristössä on
likimain
\[
\mf(\mx) \,\approx\, \begin{bmatrix} 2\\3 \end{bmatrix}
 + \begin{rmatrix} 9&-20\\-10&5\end{rmatrix} \begin{bmatrix} x-2\\y-1 \end{bmatrix},
\]
joten ehdot $\mf(\mx)=\my$ ja $\my \in B:\,|\my-\mb|\le\eps\,\ekv\,(u-2)^2+(v-3)^2 \le \eps^2$
vastaavat pisteen $\mx=\ma$ lähellä likimain ehtoa
\begin{align*}
[9(x-2)-20(y-1)]^2+[-10(x-2)+5(y-1)]^2    &\le \eps^2 \\
\qekv 181(x-2)^2-460(x-2)(y-1)+425(y-1)^2 &\le \eps^2.
\end{align*}
Tämän mukaan joukkoa $\mf^{-1}(B)$ rajaa likimain toisen asteen käyrä --- Kyseessä on
\index{ellipsi}%
(vino) \kor{ellipsi}, jonka keskipiste on $\ma=(2,1)$, vrt.\ kuvio edellä. \loppu
\end{Exa}

\subsection{Implisiittifunktiolause}

Käänteisfunktiolauseen hieman yleisempi muoto on \kor{Implisiittifunktiolause}, jossa 
tarkastellaan yhtälöryhmää muotoa
\[
\mF(\mx,\my)=\mo,
\]
missä $\mx\in\R^n$, $\my\in\R^p$ ja $\mF:D_\mF\kohti\R^p$, $D_\mF\subset\R^{n+p}$. Olkoon 
annettu piste $(\mx,\my)=(\ma,\mb)$, jossa $\mF(\ma,\mb)=\mo$. Kysytään: Voidaanko $\my$ 
ratkaista yhtälöryhmästä muodossa $\my=\mg(\mx)$ pisteen $(\ma,\mb)$ ympäristössä? 
--- Riittävä ehto ratkeavuudelle saadaan jälleen tutkimalla $\mF$:n Jacobin matriisia
\[
\mJ\,\mF(\mx,\my)=\begin{bmatrix}
\dfrac{\partial F_1}{\partial x_1} & \ldots & \dfrac{\partial F_1}{\partial x_n} & 
\dfrac{\partial F_1}{\partial y_1} & \ldots & \dfrac{\partial F_1}{\partial y_p} \\ \vdots \\
\dfrac{\partial F_p}{\partial x_1} & \ldots & \dfrac{\partial F_p}{\partial x_n} & 
\dfrac{\partial F_p}{\partial y_1} & \ldots & \dfrac{\partial F_p}{\partial y_p}
\end{bmatrix}.
\]
Kun kirjoitetaan tämä muotoon 
\[
\mJ\,\mF(\mx,\my)=[\mJ_\mx\mF(\mx,\my), \mJ_\my\mF(\mx,\my)],
\]
missä $\mJ_\mx\mF$ sisältää $\mJ\,\mF$:n ensimmäiset $n$ saraketta, niin ratkeavuuden kannalta
kriittinen on matriisi $\mJ_\my\mF(\ma,\mb)$ (kokoa $p \times p$). Jos tämä on säännöllinen,
niin ratkaiseminen yleensä onnistuu. Tarkemmin muotoiltuna tämä väittämä on mukaelma Lauseesta
\ref{käänteiskuvauslause}, ja myös todistus (jota ei esitetä) noudattaa tämän lauseen
todistuksen ajatuskulkua, ks.\ luvun loppu 
\begin{*Lause} \label{implisiittifunktiolause} \index{Implisiittifunktiolause|emph} 
\vahv{(Implisiittifunktiolause)} \ Oletetaan, että funktio $\,\mF(\mx,\my)$, \linebreak 
missä $\mx\in\R^n$, $\my\in\R^p$ ja $\mF:\DF_\mF\kohti\R^p$, $\DF_\mF\subset\R^{n+p}$, täyttää
ehdot: \ (i) $\exists\delta>0\,$ siten, että $\mF$ on differentioituva pisteen $(\ma,\mb)$
ympäristössä $U_\delta(\ma,\mb) \subset D_\mF$. \ \linebreak (ii) Osittaisderivaatat
$[\mJ_\mx\mF(\mx,\my)]_{ij}=\partial F_i(\mx,\my)/\partial x_j,\ i=1 \ldots p,\ j=1 \ldots n\,$
ja $\,[\mJ_\my\mF(\mx,\my)]_{ij}=\partial F_i(\mx,\my)/\partial y_j$, $i,j=1 \ldots p\ $ ovat
jatkuvia pisteessä $\,(\mx,\my)=(\ma,\mb)$. (iii) Matriisi $\mJ_\my\mF(\ma,\mb)$ on
säännöllinen. Tällöin on olemassa $\rho\in(0,\delta)$ ja $\eps>0$ siten, että yhtälöryhmä
$\mF(\mx,\my)=\mo$ ratkeaa ehdolla $|\my-\mb|\le\rho\,$ yksi-käsitteisesti muotoon
$\my=\mg(\mx)$ aina kun $|\mx-\ma|\le\eps$. Lisäksi $\mg$ on differentioituva pisteessä $\ma$
ja $\mJ\mg(\ma) = -[\mJ_\my\mF(\ma,\mb)]^{-1}\mJ_\mx\mF(\ma,\mb)$.
\end{*Lause}
Lauseen \ref{implisiittifunktiolause} erikoistapaaus on Lause \ref{käänteiskuvauslause},
kun $\mF(\mx,\my)=\mx-\mf(\my)$ $(p=n)$. Jos $\mg$:n differentioituvuus oletetaan, niin Jacobin
matriisin $\mJ\mg(\ma)$ laskusäännön voi johtaa implisiittisellä osittaisderivoinnilla 
(Harj.teht.\,\ref{H-udif-7: implisiittifunktion derivoimissääntö}).
\begin{Exa}
Missä tasokäyrän $S:\ F(x,y)=x^2-2xy-y^2-1=0\,$ pisteissä voidaan käyrän yhtälöstä ratkaista
paikallisesti a) $y$ $x$:n avulla, \, b) $x$ $y$:n avulla?
\end{Exa}
\ratk \ a) \, Ratkeavuusehto on $F_y(x,y)=-2x-2y\neq 0\ \ekv\ x+y\neq 0$. Koska 
$x+y=0 \ \impl \ F(x,y)=2x^2-1$, niin poikkeuspisteitä ovat ainoastaan 
$(1/\sqrt{2},-1/\sqrt{2})$ ja $(-1/\sqrt{2},1/\sqrt{2})$. Muissa pisteissä ratkaiseminen
onnistuu.

b) \, Ratkeavuusehto on $F_x(x,y)=2x-2y\neq 0\ \ekv\ y\neq x$. Tämä ehto toteutuu
kaikissa käyrän pisteissä (koska $y=x \ \impl \ F(x,y)=-2x^2-1<0$), joten myös
ratkaiseminen onnistuu kaikissa pisteissä. \loppu

\begin{Exa} Yhtälöryhmä
\[
\begin{cases} \,2x^2+xy-yz=0 \\ \,2xz-2y^2+z^2=0 \end{cases}
\]
ratkaistaan pisteen $(x,y,z)=(1,2,2)$ ympäristössä muotoon $y=y(x),\ z=z(x)$. Laske
$y'(1)$ ja $z'(1)$.
\end{Exa}
\ratk \ Pisteessä $(x,y,z)=(1,2,2)$ on
\[
\mJ_{y,z}\mF(x,y,z) = \begin{bmatrix} 
                      \partial_y F_1 & \partial_z F_1 \\ \partial_y F_2 & \partial_z F_2 
                      \end{bmatrix} 
                   = \begin{bmatrix} x-z & -y \\ -4y & 2x+2z \end{bmatrix}
                   = \begin{rmatrix} -1&-2\\-8&6 \end{rmatrix}.
\]
Tämä on säännöllinen matriisi, joten ratkaiseminen onnistuu väitetyllä tavalla. Koska 
$\partial_x F_1=4x+y=6$ ja $\partial_x F_2=2z=4$, kun $(x,y,z)=(1,2,2)$, niin saadaan
\[
\mJ\mg(1)=\begin{bmatrix} y'(1)\\z'(1) \end{bmatrix}
         = -\begin{rmatrix} -1&-2\\-8&6 \end{rmatrix}^{-1} \begin{bmatrix} 6\\4 \end{bmatrix}
         = \begin{bmatrix} 2\\2 \end{bmatrix}. 
\]
Samaan tulokseen tullaan, kun derivoidaan implisiittisesti yhtälöt
\[
2x^2+xy(x)-y(x)z(x)=0, \quad 2xz(x)-2[y(x)]^2+[z(x)]^2=0,
\]
asetetaan $(x,y,z)=(1,2,2)$ ja ratkaistaan derivaatat $y'(1)$ ja $z'(1)$ syntyvästä
(lineaarisesta) yhtälöryhmästä. \loppu

\subsection{*Kontraktiokuvauslause}

\index{kiintopisteiteraatio|(}%
Yhtälöryhmän $\mf(\mx)=\my$ paikallisessa ratkeavuusteoriassa lähtökohta on sama kuin
ratkaisua käytännössä etsittäessä, eli kirjoitetaan yhtälöryhmä muotoon $\mx=\mF(\mx)$,
jolloin kyse on siitä, onko $\mF$:llä y\pain{ksikäsitteinen}
\index{kiintopiste}%
\kor{kiintopiste} $\ma$:n lähellä. Kuten arvata saattaa, $\mF$ pyritään valitsemaan niin, että
kiintopiste löytyy kiintopisteiteraatiolla. Iteraation suppenemiselle, ja samalla kiintopisteen
yksikäsitteisyydelle annetun pisteen $\ma$ lähiympäristössä, antaa takeet
\begin{*Lause} \label{kontraktiokuvauslause} \index{Kontraktiokuvauslause|emph}
(\vahv{Kontraktiokuvauslause}\footnote[2]{Kontraktiokuvauslause tunnetaan myös nimellä 
\kor{Banachin kiintopistelause}, syystä että lause on pätevä euklidisia avaruuksia $\R^n$ 
yleisemmissä \kor{Banach-avaruuksissa}. Hyvin monet matemaattisten yhtälöiden ratkavuutta 
koskevat väittämät nojaavat Kontraktiokuvauslauseeseen sen yleisemmissä muodoissa. Tällainen
on esimerkiksi differentiaaliyhtälöiden ratkeavuutta koskeva Picardin-Lindelöfin lause
(Luku \ref{Picard-Lindelöfin lause}). \index{Banachin kiintopistelause|av}})
Olkoon $\mF:\DF_\mF\kohti\R^n,\ \DF_\mF\subset\R^n$ ja 
$K=\{\,\mx\in\R^n \mid \abs{\mx-\ma} \le \rho\,\}\subset\DF_\mF$ jollakin $\ma\in\R^n$ ja 
$\rho>0$. Tällöin jos \index{kontraktio(kuvaus)}%
\begin{itemize}
\item[(i)] $\mF$ on joukkoon $K$ rajoitettuna \kor{kontraktio(kuvaus)}, eli
\[
\abs{\mF(\mx)-\mF(\my)} \le L\abs{\mx-\my}, \quad \mx,\my\in K,\quad \text{missä}\,\ L<1,
\]
\item[(ii)] $\mF(K)\subset K$,
\end{itemize}
niin
\begin{itemize}
\item[(1)] $\mF$:llä on $K$:ssa täsmälleen yksi kiintopiste $\mc$.
\item[(2)] Kiintopisteiteraatio $\mx_{k+1}=\mF(\mx_k)$, \ $k=0,1,\ldots$ \ 
           \index{suppeneminen!b@kiintopisteiteraation|emph}%
           suppenee kohti $\mc$:tä jokaisella $\mx_0\in K$. \newline
\end{itemize}
\end{*Lause}
\tod Ensinnäkin oletuksesta $(ii)$ seuraa, että kiintopisteiteraatiolle $\mx_{k+1}=\mF(\mx_k)$
pätee
\[
\mx_0\in K \ \impl \ \mx_k\in K \quad \forall k,
\]
joten oletuksen (i) perusteella
\begin{align*}
\abs{\mx_{k+1}-\mx_k} &= \abs{\mF(\mx_k)-\mF(\mx_{k-1})} \\
&\le L\abs{\mx_k-\mx_{k-1}} \\
& \ \ \vdots \\
&\leq L^k \abs{\mx_1-\mx_0}.
\end{align*}
Tästä seuraa, että jos $l>k$, niin
\begin{align*}
\abs{\mx_l-\mx_k} &\le \abs{\mx_l-\mx_{l-1}}+\ldots + \abs{\mx_{k+1}-\mx_k} \\
&\le \sum_{i=k}^{l-1} L^i\abs{\mx_1-\mx_0} \\
&\le \abs{\mx_1-\mx_0}\sum_{i=k}^{\infty} L^i \\
&=L^k(1-L)^{-1}\abs{\mx_1-\mx_0}.
\end{align*}
Siis yleisemmin
\[
\abs{\mx_k-\mx_l}<L^{\min \{k,l\}} (1-L)^{-1}\abs{\mx_1-\mx_0}.
\]
\index{Cauchyn!ea@jono $\R^n$:ssä}%
Tämän mukaan $\{\mx_k\}$ on \kor{$\R^n$:n Cauchyn jono}, ts.\
\[
|\mx_k-\mx_l| \kohti 0, \quad \text{kun}\,\ \min\{k,l\} \kohti \infty.
\]
Kun tässä merkitään $\mx_k = ((\mx_k)_i,\ i=1 \ldots n)$, niin seuraa
\[
|(\mx_k)_i-(\mx_l)_i| \le |\mx_k-\mx_l| \kohti 0, \quad 
               \text{kun}\,\ \min\{k,l\} \kohti \infty, \quad i=1 \ldots n.
\]
Siis reaalilukujono $\{(\mx_k)_i,\ k=0,1,\ldots\}$ on Cauchyn jono jokaisella $i$. Cauchyn
kriteerin (Lause \ref{Cauchyn kriteeri}) mukaan on tällöin olemassa reaaliluvut
$c_i,\ i=1 \ldots n$ siten, että $\lim_k (\mx_k)_i = c_i,\,\ i=1 \ldots n$, jolloin on myös
$\lim_k|\mc_k-\mc|=0$, $\mc=(c_i)$ (vrt.\ Luku \ref{usean muuttujan jatkuvuus}). On siis
päätelty, että jono $\seq{\mx_k}$ (ja yleisemminkin jokainen $\R^n$:n Cauchyn jono) suppenee:  
\[
\mx_k\kohti \mc\in \R^n.
\]
Mutta koska $K$ on määritelmänsä perusteella suljettu (itse asiassa kompakti, ks.\
Määritelmä \ref{kompakti joukko - Rn}), niin $\mc\in K$. Tällöin koska $\mF$ on oletuksen (i)
perusteella $K$:ssa jatkuva, niin seuraa $\mF(x_k)\kohti\mF(\mc)\,$, jolloin iteraatiokaavan 
$\mx_{k+1}=\mF(\mx_k)$ perusteella on $\mc=\mF(\mc)$, eli $\mc$ on kiintopiste. On siis 
osoitettu, että ainakin yksi kiintopiste $\mc\in K$ on olemassa, ja että kiintopisteiteraatio
suppenee jokaisella $\mx_0\in K$ kohti jotakin $K$:ssa olevaa kiintopistettä. Lopuksi
näytetään, että kiintopisteitä on $K$:ssa vain yksi: Olkoot $\mc_1$ ja $\mc_2\,$ $\mF$:n
kiintopisteitä $K$:ssa, eli olkoon $\mc_1=\mF(\mc_1)$, $\mc_2=\mF(\mc_2)$ ja
$\mc_1\,,\mc_2\in K$. Oletuksesta (i) seuraa silloin
\[
\abs{\mc_1-\mc_2} =   \abs{\mF(\mc_1)-\mF(\mc_2)} 
                  \le L\abs{\mc_1-\mc_2} \qekv (1-L)\abs{\mc_1-\mc_2} \le 0.
\]
Koska tässä on $1-L>0$ (oletus (i)), niin on oltava $\abs{\mc_1-\mc_2}=0\ \ekv\ \mc_1=\mc_2$. 
Kontraktiokuvauslause on näin todistettu. \loppu
\index{kiintopisteiteraatio|)}

\subsection{*Käänteisfunktiolauseen todistus}
\index{Kzyzy@Käänteisfunktiolause|vahv}

Lähdetään differentiaalikehitelmästä
\begin{equation} \label{K-lause a}
\mf(\mx)=\mb+\mJ\mf(\ma)(\mx-\ma)+\mr(\mx), \quad \mb=\mf(\ma), \quad \mx\in K, \tag{a}
\end{equation}
missä $K=\{\mx\in\R^n \ | \ |\mx-\ma|\le\rho\}$, $\,\rho\in(0,\delta)$. Tämän perusteella on
\begin{equation} \label{K-lause b}
\mf(\mx_1)-\mf(\mx_2)=\mJ\mf(\ma)(\mx_1-\mx_2)+\mr(\mx_1)-\mr(\mx_2), \quad
                                                  \mx_1,\mx_2 \in K. \tag{b}
\end{equation}
Kun tässä sovelletaan vasemmalla puolella $\R^n$ väliarvolausetta
(Lause \ref{väliarvolause-Rn})  funktioihin $f_i=(\mf)_i$, niin seuraa
\begin{equation} \label{K-lause c}
f_i(\mx_1)-f_i(\mx_2) = [\nabla f_i(\boldsymbol{\xi}_i)]^T(\mx_1-\mx_2),
                                                     \quad i=1 \ldots n, \tag{c}
\end{equation}
missä $\boldsymbol{\xi}_i\in K$ (tarkemmin $\boldsymbol{\xi}_i=t_i(\mx_1-\mx_2),\ t_i\in(0,1)$).
Jatkuvuusoletuksien perusteella on tässä edelleen
\begin{equation} \label{K-lause d}
\nabla f_i(\boldsymbol{\xi}_i) = \nabla f_i(\ma)+\md_i, \quad |\md_i| \kohti 0 \,\ 
                                 \text{kun}\ \rho \kohti 0, \quad i = 1 \ldots n. \tag{d}
\end{equation}
(Tässä sivuutetaan hieman teknisiä yksityiskohtia.) Yhdistämällä \eqref{K-lause b},
\eqref{K-lause c} ja \eqref{K-lause d} nähdään, että
\[
\mr(\mx_1)-\mr(\mx_2)\,          =\,[\md_1 \ldots \md_n]^T(\mx_1-\mx_2),% \notag \\
\]
joten Cauchyn--Schwarzin epäyhtälön perusteella
\begin{equation} \label{K-lause e}
|\mr(\mx_1-\mr(\mx_2)| \,=\, \Bigl(\sum_{i=1}^n\scp{\md_i}{\mx_1-\mx_2}^2\Bigr)^{1/2}\,
                       \,\le\,L|\mx_1-\mx_2|, \tag{e}
\end{equation}
missä
\[
L=\bigl(\sum_{i=1}^n|\md_i|^2\bigr)^{1/2} \kohti 0, \quad \text{kun}\ \rho \kohti 0.
\]

Koska $\mJ\mf(\ma)$ on säännöllinen, niin yhtälö \eqref{K-lause b} on toisin kirjoitettuna
\begin{equation} \label{K-lause f}
\mx_1-\mx_2 = [\mJ\mf(\ma)]^{-1}[\mf(\mx_1)-\mf(\mx_2)]
             -[\mJ\mf(\ma)]^{-1}[\mr(\mx_1)-\mr(\mx_2)]. \tag{f}
\end{equation}
Kun tässä sovelletaan kolmioepäyhtälöä ja arviota \eqref{K-lause e}, niin seuraa
\begin{align} \label{K-lause g}
|\mx_1-\mx_2| 
          &\le |[\mJ\mf(\ma)]^{-1}[\mf(\mx_1)-\mf(\mx_2)]|
              +|[\mJ\mf(\ma)]^{-1}[\mr(\mx_1)-\mr(\mx_2)]|  \notag \\
          &\le C|\mf(\mx_1)-\mf(\mx_2)| + CL|\mx_1-\mx_2|, \quad \mx_1,\mx_2 \in K, \tag{g}
\end{align}
missä $\,C=\norm{[\mJ\mf(\ma)]^{-1}}$ (ks.\ 
Harj.teht.\,\ref{matriisialgebra}:\ref{H-m-1: matriisin normi}). Koska $\,C\,$ on vain
matriisista $[\mJ\mf(\ma)]^{-1}$ riippuva vakio ja $L \kohti 0$ kun $\rho \kohti 0$, niin
voidaan valita $\rho\in(0,\delta)$ siten, että pätee
\begin{equation} \label{K-lause h}
CL\le\frac{1}{2}\,, \tag{h}
\end{equation}
jolloin epäyhtälöistä \eqref{K-lause g} ja \eqref{K-lause h} seuraa
\begin{equation} \label{K-lause i}
|\mx_1-\mx_2| \,\le\, 2C|\mf(\mx_1)-\mf(\mx_2)| \quad \forall\ \mx_1,\mx_2 \in K. \tag{i}
\end{equation}
Tämän mukaan $\,\forall\,\mx_1,\mx_2 \in K$ pätee $\mf(\mx_1)=\mf(\mx_2)\,\impl\,\mx_1=\mx_2$,
joten $\mf$ on $K$:ssa 1-1, kun $\rho\in(0,\delta)$ on valittu niin, että ehto
\eqref{K-lause h} toteutuu. Käänteisfunktiolauseen 1.\ osaväittämä on näin todistettu.

Käänteisfunktiolauseen toisen väittämän todistamiseksi osoitetaan, että kun $\rho$ on valittu
em.\ tavalla, niin $\exists\,\eps>0$ siten, että Kontraktiokuvauslauseen ehdot
toteutuvat, kun $|\my-\mb|\le\eps$ ja ratkaistava yhtälöryhmä kirjoitetaan muotoon
\[
\mf(\mx)=\my\,\ \ekv\,\ \mx=\mF(\mx), \quad \mF(\mx)=\mx-[\mJ\mf(\ma)]^{-1}[\,\mf(\mx)-\my\,],
                                      \quad \mx \in K.
\]
--- Huomattakoon, että kiintopisteiteraatio $\mx_{k+1}=\mF(\mx_k)$ on tällöin sama kuin
edellisessä luvussa tarkasteltu yksinkertaistettu Newtonin iteraatio sovellettuna yhtälöryhmään
$\mf(\mx)-\my=\mo$ (alkuarvauksella $\mx_0=\ma$).

Deifferentiaalikehitelmän \eqref{K-lause a} perusteella $\mF$:n lauseke saadaan muotoon
\[
\mF(\mx) = \ma + [\mJ\mf(\ma)]^{-1}(\my-\mb) - [\mJ\mf(\ma)]^{-1}\mr(\mx), \quad \mx \in K.
\]
Kun käytetään arviota \eqref{K-lause e} tässä lausekkeessa, niin seuraa
\begin{align} \label{K-lause j}
\abs{\mF(\mx_1)-\mF(\mx_2)}\ 
    &=\ \left|[\mJ\mf(\ma)]^{-1}[\,\mr(\mx_1)-\mr(\mx_2)\,]\right| \notag \\[1mm]
    &\le\ CL\abs{\mx_1-\mx_2}, \quad \mx_1,\mx_2\in K \tag{j}
\end{align}
ja koska $\mr(\ma)=\mo$, niin seuraa myös
\begin{align} \label{K-lause k}
\abs{\mF(\mx)-\ma}\ 
    &=\ \left|[\mJ\mf(\ma)]^{-1}(\my-\mb)
         -[\mJ\mf(\ma)]^{-1}[\,\mr(\mx)-\mr(\ma)\,]\right| \notag\\[1mm]
    &\le\ C|\my-\mb|+CL|\mx-\ma|, \quad \mx\in K \tag{k}
\end{align}
($C$ kuten edellä). Valitaan nyt $\eps$ siten, että toteutuu
\begin{equation} \label{K-lause l}
0<\eps\le\frac{\rho}{2C}\,. \tag{l}
\end{equation}
Tällöin jos $\abs{\my-\mb}\le\eps$, niin epäyhtälöistä \eqref{K-lause j}, \eqref{K-lause k},
\eqref{K-lause h} ja \eqref{K-lause l} seuraa
\[
\begin{cases} 
\,\abs{\mF(\mx_1)-\mF(\mx_2)}\ 
  \le\ \frac{1}{2}\abs{\mx_1-\mx_2}, \quad \mx_1\,,\mx_2 \in K, \\
\,\abs{\mF(\mx)-\ma}\ \le \frac{\rho}{2}+\frac{\rho}{2} = \rho, \quad \mx \in K.
\end{cases}
\]
Näistä ensimmäisen ehdon mukaan $\mF$ on kontraktiokuvaus $K$:ssa ja toisen ehdon mukaan
$\mF(K) \subset K$, joten Kontraktiokuvauslauseen ehdot ovat täytetyt. Tämän lauseen mukaan
yhtälöryhmällä $\mf(\mx)=\my\,\ekv\,\mx=\mF(\mx)$ on $K$:ssa yksikäsitteinen ratkaisu, kun
$\rho$ ja $\eps$ on valittu em.\ tavalla ja $|\my-\mb|\le\eps$. Käänteisfunktiolauseen
toinen väittämä on näin todistettu.

Viimeisen osaväittämän todistamiseksi valitaan yhtälössä \eqref{K-lause f} 
$\,\mx_1=\mx=\mf^{-1}(\my)$, missä $\,0<|\my-\mb|\le\eps\,$ (jolloin $\mx\neq\ma$) ja
$\,\mx_2=\ma$, jolloin yhtälö saa muodon
\begin{equation} \label{K-lause m}
\mf^{-1}(\my)-\ma = [\mJ\mf(\ma)]^{-1}(\my-\mb)-[\mJ\mf(\ma)]^{-1}\mr(\mx). \tag{m}
\end{equation}
Epäyhtälön \eqref{K-lause i} mukaan on (samoilla $\mx$ ja $\my$)
\[
|\mx-\ma| \le 2C|\my-\mb|.
\]
Tämän sekä arvion $|[\mJ\mf(\ma)]^{-1}\mr(\mx)| \le C|\mr(\mx)|$ perusteella
\[
\frac{|[\mJ\mf(\ma)]^{-1}\mr(\mx)|}{|\my-\mb|} \le\,
\frac{C|\mr(\mx)|}{|\mx-\ma|}\,\cdot\,\frac{|\mx-\ma|}{|\my-\mb|}
\,\le\, 2C^2\,\frac{|\mr(\mx)|}{|\mx-\ma|}\,.
\]
Koska tässä $|\my-\mb| \kohti 0\,\impl\,|\mx-\ma| \kohti 0$ (edellisen epäyhtälön mukaan) ja
koska $|\mr(\mx)|/|\mx-\ma| \kohti 0$, kun $|\mx-\ma| \kohti 0$, niin
$|[\mJ\mf(\ma)]^{-1}\mr(\mx)|=\ord{|\my-\mb|}$. Yhtälö \eqref{K-lause m} voidaan siis
kirjoittaa
\[
\mf^{-1}(\my)=\mf^{-1}(\mb)+[\mJ\mf(\ma)]^{-1}(\my-\mb)+\mr(\my), \quad 
                                              |\mr(\my)|=\ord{|\my-\mb|}.
\]
Näin ollen $\mf^{-1}$ on differentioituva pisteessä $\mb$ ja 
$\mJ\mf^{-1}(\mb)=[\mJ\mf(\ma)]^{-1}$. \loppu

\Harj
\begin{enumerate}

\item
Tarkastellaan funktiota $\mf(x,y)=(2x^2-y^2,-x^2+2y^2)$. a) Määritä $\mf$:n arvojoukko
$\mf(\R^2)$. b) Millaisille joukoille $A\subset\R^2$ pätee: $f:\,A \kohti f(A)$ on bijektio?
c) Näytä, että $\mf$ on paikallisesti kääntyvä pisteessä $(a,b)$, jos $ab \neq 0$. Onko tämä
ehto myös välttämätön paikalliselle kääntyvyydelle? d) Määritä $\mf$:n paikallisten
käänteisfunktioiden lausekkeet pisteissä $(1,1)$, $(-1,1)$, $(1,-1)$ ja $(-1,-1)$.

\item
Olkoon $\mf(x,y)=(x-y+e^{x+2y},x+y+e^{x-2y})$ ja olkoon $\mf^{-1}=\mf$:n paikallinen 
käänteisfunktio origon ympäristössä. Edelleen olkoon \vspace{1mm}\newline
a) \ $K_a(x_0,y_0)=\{(x,y)\in\R^2 \mid \abs{x-x_0} \le a\ \ja\ \abs{y-y_0} \le a\}$, \newline
b) \ $K_a(x_0,y_0)=\{(x,y)\in\R^2 \mid (x-x_0)^2+(y-y_0)^2 \le a^2\}$, \vspace{1mm}\newline
missä $a>0$ on pieni luku. Approksimoimalla $\mf$ affiinikuvauksella määritä likimäärin joukot
$\mf\bigl(K_a(0,0)\bigr)$ ja $\mf^{-1}\bigl(K_a(1,1)\bigr)$. Piirrä kuviot!

\item
Näytä, että seuraavat funktiot ovat bijektioita kuvauksina $\mf:\,\R^2\kohti\R^2$. Määritä
käänteisfunktion lauseke, mikäli mahdollista. Määritä myös pisteet, joissa $\mJ\mf$ on 
singulaarinen. \vspace{1mm}\newline
a) \ $\mf(x,y)=(x^3+y^3,x^3-y^3) \quad$ 
b) \ $\mf(x,y)=((x+y)^3,(x-y)^3)$ \newline
c) \ $\mf(x,y)=(x-y,x+y^3) \qquad\ $
d) \ $\mf(x,y)=(e^x-y,x+x^3+y+y^3)$

\item
Olkoon $\mf=(e^x\cos y,e^x\sin y)$. \ a) Näytä, että $\mf$ on kaikkialla paikallisesti 
kääntyvä. \ b) Määritä $\mf(A),\ A=\R\times[0,2\pi]$. Onko $\,\mf:\,A\kohti\mf(A)$ 1-1\,? \
\linebreak c) Olkoon $A=\{(x,y)\in\R^2 \ | \ x^2+y^2<R^2\}$. Mikä on suurin $R$:n arvo, jolla
$\mf:\,A\kohti\mf(A)$ on bijektio?

\item
Missä pisteissä $(x,y)$ toteutuvat Käänteisfunktiolauseen oletukset funktiolle
$\mf(x,y)=(x^2+2y,y^2-2x)$\,? Määritä Jacobin matriisi $\mJ\mf^{-1}(u,v)$, $(u,v)=\mf(x,y)$
näissä pisteissä.

\item
Tasossa liikkuvan robotin käsivarsi koostuu origoon kiinnitetystä olkavarresta (janasta)
$OP$ ja pisteeseen $P$ kiinnitetystä kyynärvarresta $PQ$. Kummankin varren pituus $=1$.
Kättä ($Q$) ohjataan säätämällä olkavarren ja $x$-akselin välistä napakulmaa $\varphi$ ja
kyynärvarren ja olkavarren välistä kulmaa $\theta$. Kulmat ovat vaihdeltavissa väleillä 
$\varphi\in[0,\pi]$ ja $\theta\in[-5\pi/6,5\pi/6]$ ($\theta>0$ vastapäivään).
\vspace{1mm}\newline
a) \ Määritä robotin käden sijainti funktiona $\mf(\varphi,\theta)$. \newline
b) \ Mihin tason pisteisiin käsi ulottuu, eli mikä on $\mf$:n arvojoukko? 
     Kuvio! \newline
c) \ Totea, että $\mf$ ei ole 1--1. Missä pisteissä 
     $(\varphi,\theta)\in(0,\pi)\times(-5\pi/6,5\pi/6)$ toteutuvat
     Käänteisfunktiolauseen ehdot? \newline
d) \ Robotin käsi halutaan siirtää pisteestä $(1.4,0.2)$ pisteeseen $(1.42,0.22)$.
     Arvioi differentiaalin avulla, millaisilla kulman muutoksilla $\Delta\varphi$
     ja $\Delta\theta$ tämä liike saadaan aikaan, kun käsivarsi on asennossa, jossa
     $\theta<0$.

\item \label{H-udif-7: implisiittifunktion derivoimissääntö}
Johda implisiittifunktion $\my=\mg(\mx)$ (Lause \ref{implisiittifunktiolause}) Jacobin
matriisin laskusääntö implisiittisellä osittaisderivoinnilla yhtälöryhmästä
\[
\mF(\mx,\mg(\mx))=\mo \qekv F_i\bigl(\mx,g_1(\mx),\ldots,g_p(\mx)\bigr)=0,\quad i = 1 \ldots p.
\]

\item
Oletetaan, että yhtälöryhmä $F(x,y,u,v)=0,\ G(x,y,u,v)=0$ toteutuu, kun $(x,y,u,v)=(a,b,c,d)$,
ja että ko.\ pisteen lähellä yhtälöryhmä ratkeaa sekä muotoon $(u,v)=\mg(x,y)$ että muotoon
$(x,y)=\mh(u,v)$. Johda Implisiittifunktiolauseen viimeisestä osaväittämästä (ko.\ lauseen
oletuksin) tulos $\mJ\mg(a,b)\mJ\mh(c,d)=\mI$. Millä muulla tavalla tämä on pääteltävissä?

\item
Olkoon $f$ säännöllinen funktio tyyppiä $f:\,\R^2\kohti\R$. Jos yhtälöryhmä
%pisteessä $(x,y,z)=(a,b,c)$ on
\[
\begin{cases} \, f(f(x,y),z)=0 \\ \,f(x,f(y,z))=0 \end{cases}
\]
toteutuu, kun $(x,y,z)=(a,b,c)$, niin millaisella $f$:n osittaisderivaattoja $f_x$ ja $f_y$
koskevalla ehdolla voidaan taata, että pisteen $(a,b,c)$ ympäristössä yhtälöryhmä ratkeaa
muotoon $x=x(z),\ y=y(z)$\,? 

\item
Olkoon $a,b,c>0$. Näytä, että yhtälöryhmä
\[
\begin{cases}
\,\dfrac{x^2}{a^2}+\dfrac{y^2}{b^2}+\dfrac{z^2}{c^2}=1 \\[2mm]
\,\dfrac{(2x-a)^2}{a^2}+\dfrac{4y^2}{b^2}=1
\end{cases}
\]
määrittelee pisteen $P=(a,b,c)$ ympäristössä parametrisen käyrän \newline 
$S:\,x=f(z)\,\ja\,y=g(z)$. Mikä on $S$:n tangenttivektori pisteessä $P$\,?

\item
Näytä, että yhtälö tai yhtälöryhmä määrittelee annetun pisteen $P$ ympäristössä 
implisiittifunktion annettua muotoa sekä ratkaise lisätehtävä: \vspace{1mm}\newline
a) \ $x^6+2y^4-3x^2y=0,\,\ P=(1,1),\,\ y=f(x)$. \ Käyrän $y=f(x)$ tangentin yhtälö pisteessä 
$P$\,? \vspace{1mm} \newline
b) \ $\sin(yz)+y^2 e^z=x,\,\ P=(e\pi^2,\pi,1),\,\ z=f(x,y)$. \ Arvioi $f(27,3)$ \newline
differentiaalin avulla. \vspace{1mm}\newline
c) \ $ye^{xz}+\sin(x-y+z)=0,\,\ P=(0,0,0),\,\ z=f(x,y)$. \ Pinnan $z=f(x,y)$ tangenttitason
yhtälö origossa? \vspace{1mm}\newline
d) \ $x=z+y\sin z,\,\ P=(0,0,0),\,\ z=f(x,y)$. \ Laske $f_{xy}(0,0)$. \vspace{1mm}\newline
e) \ $x-y-z+1=0\,\ja\,x^2+y^2-2z=0,\,\ P=(1,1,1),\,\ (x,y)=\mf(z)$. \ 
Geometrinen tulkinta? \vspace{1mm}\newline
f) \ $2x^2+3y^2+z^2=47\,\ja\,x^2+2y^2-z=0,\,\ P=(-2,1,6),\,\ (y,z)=(f(x),g(x))$. Laske
$f'(-2)$ ja $g'(-2)$. \vspace{1mm}\newline
g) \ $y^5+xy+z^2=4\,\ja\,e^{xz}=y^2,\,\ (3,1,0),\,\ (x,y)=(f(z),g(z))$. \ Laske $f'(0)$ ja 
$g'(0)$. \vspace{1mm}\newline
h) \ $2x^2+3y^2+z^2=47\,\ja\,x^2+2y^2-z=0,\,\ P=(-2,1,6),\,\ (y,z)=(f(x),g(x))$. Laske
$f'(-2)$ ja $g'(-2)$.

\item
Näytä, että seuraavat yhtälöryhmät määrittelevät annettua muotoa olevan implisiittifunktion
$\mg$ annetun pisteen lähellä, ja laske Jacobin matriisi $\mJ\mg$ ko.\ pisteessä.
\begin{align*}
&\text{a)}\ \ \begin{cases} \,x-u^3-v^3=0 \\ \,y-uv+v^2=0, \end{cases} \quad
              (u,v)=\mg(x,y), \quad (x,y,u,v)=(2,0,1,1) \\
&\text{b)}\ \ \begin{cases} \,xe^y-u^3z+\sin v=0 \\ \,x^3u^2+yv-uz\cos v=0, \end{cases} \quad 
              \begin{aligned} &(u,v)=\mg(x,y,z), \\ &(x,y,z,u,v)=(1,0,1,1,0) \end{aligned} \\
&\text{c)}\ \ \begin{cases} 
              \,xy^2+zu+v^2=3 \\ \,x^3z+2y-uv=2 \\ \,xu+yv-xyz=1,
              \end{cases} \quad 
              \begin{aligned} &(x,y,z)=\mg(u,v), \\ &(x,y,z,u,v)=(1,1,1,1,1) \end{aligned}
\end{align*}

\item (*)
Näytä, että funktio $\mf(x,y)=(2x-xy,4y-xy)$ on paikallisesti kääntyvä pisteessä $(a,b)$
täsmälleen kun $a+2b \neq 4$.

\item (*)
Oletetaan, että yhtälö $F(x,y,z)=0$ määrittelee (Implisiittifunktiolauseen oletuksin)
implisiittifunktiot $x=x(y,z)$, $y=y(z,x)$ ja $z=z(x,y)$. Näytä, että pätee:
\[
\pder{x}{y}\cdot\pder{y}{z}\cdot\pder{z}{x}=-1.
\]
Tarkista säännön pätevyys, kun $F(x,y,z)=x^3y^2z-1$. Muotoile ja todista vastaava yleisempi
väittämä koskien yhtälön $F(x_1,\ldots,x_n)=0$ määrittelemiä implisiittifunktioita 
($n$ kpl, $n \ge 2$).

\end{enumerate} %Kontraktiokuvuauslause. Käänteisfunktiolause. Implisiittifunktiolause
\section{Usean muuttujan ääriarvotehtävät} \label{usean muuttujan ääriarvotehtävät}
\alku
\index{zyzy@ääriarvotehtävä|vahv}

Aloitetaan määritelmästä.
\begin{Def} \label{usean muuttujan ääriarvot}
\index{paikallinen maksimi, minimi, ääriarvo|emph}
\index{maksimi (funktion)!a@paikallinen|emph}
\index{minimi (funktion)!a@paikallinen|emph}
\index{zyzy@ääriarvo (paikallinen)|emph}
\index{laakapiste|emph}
Funktiolla $f:\DF_f\kohti\R$, $D_f\subset\R^n$, on pisteesä $\mc\in\DF_f$
\kor{paikallinen maksimi}, jos jollakin $\delta>0$ pätee
\[
0<\abs{\mx-\mc}<\delta \ \impl \ x\in\DF_f \text{ ja } f(\mx)<f(\mc),
\]
ja \kor{paikallinen minimi}, jos jollakin $\delta>0$ pätee
\[
0<\abs{\mx-\mc}<\delta \ \impl \ x\in\DF_f \text{ ja } f(\mx)>f(\mc).
\]
Jos $f$:llä on pisteessä $\mc$ paikallinen maksimi tai minimi, niin $f$:llä on 
\kor{paikallinen ääriarvo} pisteessä $\mc$. Jos $\mc\in\DF_f$ ei ole funktion $f$ paikallinen
ääriarvokohta, mutta jollakin $\delta>0$ pätee
\begin{align*}
\text{joko}\,:\quad &0<\abs{\mx-\mc}<\delta\ \impl \ x\in\DF_f \text{ ja } f(\mx) \le f(\mc),\\
\text{tai}\,: \quad &0<\abs{\mx-\mc}<\delta\ \impl \ x\in\DF_f \text{ ja } f(\mx) \ge f(\mc),
\end{align*}
niin sanotaan, että $\mc$ on $f$:n \kor{laakapiste}.
\end{Def}
\begin{Exa} \label{udif-8.1} Funktiolla $f(x,y)=ax^2+by^2$ on origossa paikallinen
ääriarvokohta, jos $ab>0$. Kyseessä on paikallinen minimi, jos $a>0$ ja $b>0$, ja paikallinen
maksimi, jos $a<0$ ja $b<0$. Jos on $ab=0$, niin origo on laakapiste. Tapauksessa $ab<0$ ei
origo ole ääriarvokohta eikä laakapiste. \loppu
\end{Exa}
Funktion paikallista ääriarvoa sanotaan myös
\index{suhteellinen ääriarvo}%
\kor{suhteelliseksi}, jolloin funktion maksimia/minimiä koko tarkastelun kohteena olevassa
joukossa (esim.\ koko määrittelyjoukossa) sanotaan
\index{maksimi (funktion)!b@absoluuttinen} \index{minimi (funktion)!b@absoluuttinen}
\index{absoluuttinen maksimi, minimi} \index{globaali ääriarvo}%
\kor{absoluuttiseksi}. Myös termiä \kor{globaali} ääriarvo käytetään, etenkin jos kyse on 
funktion koko määrittelyjoukosta.
\jatko \begin{Exa} (jatko) Jos joukko $A \subset \R^2$ sisältää origon, niin tapauksessa $ab>0$
on $f(0,0)=0$ funktion absoluuttinen minimi/maksimi $A$:ssa. \loppu
\end{Exa}  

Gradientin avulla voidaan helposti paikallistaa reaaliarvoisen funktion paikalliset minimit ja 
maksimit, jos funktio on differentioituva. Nimittäin Luvun \ref{gradientti} tarkastelujen 
perusteella seuraava tulos on varsin ilmeinen:
\begin{Prop} \label{ääriarvopropositio-Rn}
Jos $f:\DF_f\kohti\R$, $\DF_f\subset\R^n$, on differentioituva pisteessä $\mc\in\DF_f$ ja $\mc$
on$f$:n paikallinen ääriarvo- tai laakapiste, niin $\Nabla f(\mc)=\mo$.
\end{Prop}
\index{kriittinen piste}%
Gradientin nollakohtia sanotaan yleisemmin funktion \kor{kriittisiksi pisteiksi}.
\begin{Exa} \label{saari uudelleen}
Etsi funktion $f(x,y)=x^2+2y^2+2xy+4x+14y$ kriittiset pisteet. 
(Vrt. Luku \ref{kahden ja kolmen muuttujan funktiot}, Esimerkki \ref{saari}.)
\end{Exa} 
\ratk 
\[
\nabla f(x,y)=\vec 0 \qekv \begin{cases} 
                           \,\partial_x f(x,y)= 2x+2y+4=0, \\ 
                           \,\partial_y f(x,y)= 2x+4y+14=0 
                           \end{cases}\ \ekv \quad \begin{cases} \,x=3, \\ \,y=-5. \end{cases}
\]
Ainoa kriittinen piste on siis $(3,-5)$. \loppu

Esimerkissä kriittinen piste on funktion absoluuttinen minimi, kuten selvitettiin Luvussa 
\ref{kahden ja kolmen muuttujan funktiot} pelkin algebran keinoin. --- Kriitisen pisteen
ei yleisemmin tarvitse olla paikallinen ääriarvokohta tai laakapiste. Esimerkiksi funktion
\[
f(x,y)=x^2-y^2
\]
\index{satulapiste}%
kriittinen (origo) on nk.\ \kor{satulapiste} (engl. saddle point) --- termi ei kaivanne
lähempää määrittelyä. Funktion kriittisten pisteiden luokittelu ääriarvokohdiksi,
laakapisteiksi tai satulapisteiksi on ongelma, jonka (algebrallisella) ratkaisulla on
yleisempääkin mielenkiintoa. Asiaan palataan hieman myöhemmin. Tässä yhteydessä rajoitutaan
ääriarvotehtäviin, joissa kriittisten pisteiden luokittelulla ei ole keskeistä merkitystä.

\subsection{Rajoitettu optimointi} 
\index{rajoitettu optimointi|vahv}

Sovelluksissa esiintyvät usean muuttujan ääriarvotehtävät voidaan yleensä muotoilla 
\kor{rajoitetun optimoinnin} (engl.\ constrained optimization) ongelmina. Tyypillisessä 
rajoitetun optimoinnin ongelmassa halutaan etsiä reaaliarvoisen funktion
$f(\mx),\ \DF_f\subset\R^n$ minimi tai maksimi joukossa $A$, jonka määritelmä asetetaan
muodossa
\[ 
\mx \in A \qekv g_i(\mx) \ge 0, \quad i=1 \ldots m. 
\]
Sovelluksissa kyse on yleensä jonkin käytännön kannalta tärkeän (kuten taloudellisen hyödyn tai
tappion) maksimointi tai minimointi, eli
\index{optimointi} \index{rajoitusehto (optimoinnin)}%
\kor{optimointi}, tehtävän asettelussa väistämättömien \kor{rajoitusehtojen} voimassa ollessa.
--- Optimointialgoritmien suunnittelun (tai käytön)
kannalta on huomion arvoista, että epäyhtälörajoitteisiin voi sisältyä myös yhtälörajoitteita
muodossa
\[ 
g(\mx)= 0 \qekv \begin{cases} \ \ g(\mx) &\ge 0, \\ -g(\mx) &\ge 0. \end{cases} 
\]
\index{lineaarinen rajoite}%
Rajoitteita sanotaan \kor{lineaarisiksi}, jos funktiot $g_i$ ovat ensimmäisen asteen polynomeja
(affiinikuvauksia). Jos myös $f$ on tätä tyyppiä, on kyse
\index{lineaarinen optimointi!lineaarinen ohjelmointi}%
\kor{lineaarisen ohjelmoinnin} ongelmasta, vrt.\ luku \ref{affiinikuvaukset}.

Tarkastellaan esimerkkinä kahden muuttujan tilannetta, jossa optimointiongelman asettelu on
\[
f(x,y)=\min/\max\,! \quad \text{ehdoilla} \quad g_i(x,y) \ge 0, \quad i=1 \ldots m.
\]
Oletetaan jatkossa, että sekä $f$ että rajoitefunktiot $g_i$ ovat määriteltyjä ja 
differentioituvia $\R^2$:ssa. Ongelman voi purkaa kahteen osaan: Ratkaistaan ensin 
\kor{rajoittamaton} ongelma käymällä läpi $f$:n kaikki kriittiset pisteet. Kriittisistä 
\index{kzyypzy@käypä (kriittinen piste)}%
pisteistä \kor{käypiä} ovat ne, joissa asetetut rajoitusehdot ovat voimassa. Oletetaan, että
näitä on äärellinen määrä, jolloin voidaan suorittaa valinta: Valitaan pisteistä se, jossa $f$
saa pienimmän/suurimman arvonsa. 

Optimointitehtävää ei ole vielä ratkaistu, vaan hankalampi vaihe on vasta edessä: On määrättävä
erikseen funktion minimi/maksimi tarkasteltavan joukon $A$
\index{reuna (joukon, alueen)}%
\kor{reunalla} $\partial A$, joka määritellään
\[ 
(x,y) \in \partial A \qekv g_i(x,y)=0 \quad \text{jollakin}\ i. 
\]
Kuvassa on koko optimointiongelma kuvattuna yhden rajoitusehdon ($m=1$) tapauksessa.
\begin{figure}[H]
\setlength{\unitlength}{1cm}
\begin{center}
\begin{picture}(11,7)(0,-2)
\closecurve(1,0.4,6,1.8,5,4,0.3,3.5)
\put(1.12,2.48){$\bullet$} \put(3.12,1.45){$\bullet$} \put(6,1.8){$\bullet$} 
\put(3.15,-0.84){$\bullet$}
\put(0,-2){$\vec\nabla f=\vec 0$} \put(0.2,-1.5){\vector(1,4){1}} 
\put(0.2,-1.5){\vector(1,1){3}} 
\put(0.2,-1.5){\vector(4,1){3}}
\put(2,2.5){$g(x,y)>0$} \put(6.5,4.4){$g(x,y)=0$}
\put(7.8,1.9){\vector(-1,0){1.6}} 
\put(8,1.7){\parbox{3cm}{$f=\min/\max\, ! \,$, kun $g(x,y)=0$}}
\put(3.5,-0.85){(ei käypä)}
\curve(5,4,5.8,4.5,6,4.5,6.3,4.5)
\end{picture}
\end{center}
\end{figure}

\index{sidottu ääriarvotehtävä} \index{zyzy@ääriarvotehtävä!a@sidottu ääriarvotehtävä}%
Optimointiongelmaa yhden tai useamman yhtälörajoitteen alaisena sanotaan \kor{sidotuksi} 
ääriarvotehtäväksi. Kahden muuttujan tapauksessa ongelma ratkeaa luontevasti, jos oletetaan,
että jokainen rajoitusehto määrittelee parametrisen käyrän muotoa
\[ 
g_i(x,y)=0 \ \ekv \ x=x_i(t), \ y=y_i(t),\ t \in \R, \quad i = 1 \ldots m, 
\]
ja oletetaan vielä, että reuna $\partial A$ voidaan pilkkoa käyränkaariksi
\[ 
\partial A_i = \{(x,y) \in \R^2 \mid x = x_i(t),\ y = y_i(t),\ t \in [a_i,b_i]\}, 
\]
jotka yhdessä peittävät $\partial A$:n ja joiden yhteisiä pisteitä ovat enintään päätepisteet
$(x_i(a_i),y_i(a_i))$ ja $(x_i(b_i),y_i(b_i))$. Optimointiongelman jäljellä oleva osa on näillä
oletuksilla palautettu yhden muuttujan rajoitetuiksi optimointiongelmiksi:
\[ 
F_i(t) = f\bigl(x_i(t),y_i(t)\bigr) = \text{min/max\ !} \quad \text{kun} \quad 
                                      a_i \le t \le b_i\,, \quad i = 1 \ldots m.
\]
Nämä ongelmat ratkeavat normaaliin tapaan, eli etsimällä funktioiden $F_i$ kriittiset pisteet
(= derivaatan nollakohdat) avoimilta väleiltä $(a_i,b_i)$ ja tutkimalla erikseen päätepisteet
$a_i,b_i$. Sikäli kuin funktioilla $F_i$ on käypiä kriittisiä pisteitä äärellinen määrä, on 
löydetty äärellinen pistejoukko, josta optimoitavan funktion ääriarvot välttämättä löytyvät.
\begin{Exa} \label{esim: rajoitettu optimointi} Ratkaise rajoitetut optimointitehtävät
\begin{align*} 
&f(x,y) = 6x^2-4xy+9y^2+16x-22y = \text{min\ !} \quad \text{ja} \quad f(x,y) = \text{max\ !} \\
&\text{kun} \quad x \ge 0,\quad  x^2+(y-1)^2 \le 1.
\end{align*}
\end{Exa}
\ratk Kriittisiä pisteitä on yksi:
\[ 
\nabla f(x,y) = \vec 0 \qekv \begin{cases} 
                             \ 12x-4y+16=0, \\ -4x+18y-22=0 \end{cases} 
                       \qekv \begin{cases} \,x=-1, \\ \,y=1. \end{cases} 
\]
Tämä ei ole käypä, joten siirrytään tutkimaan funktiota tarkasteltavan joukon $A$ reunalla 
$\partial A$:
\[ 
(x,y) \in \partial A \qekv x=0 \quad \text{tai} \quad x^2+(y-1)^2=1. 
\]
Tämä jakautuu kahteen osaan:
\begin{align*}
&\partial A_1\,: \quad x=0\,\ \ja\,\ x^2 + (y-1)^2 \le 1 \qekv x=0\,\ \ja\,\ y \in [0,2]. \\
&\partial A_2\,: \quad x^2 + (y-1)^2 = 1\,\ \ja\,\ x \ge 0.
\end{align*}
Reunan osa $\partial A_1$ on jana, jolla
\[ 
F_1(y) = f(0,y) = 9y^2-22y. 
\]
Funktion $F_1$ kriittinen piste $y=11/9$ on käypä, joten merkitään muistiin $f$:n arvo tässä 
pisteessä sekä janan $\partial A_1$ päätepisteissä:
\[
f(0,11/9) = -121/9 = -13.4444.., \quad f(0,0)=0, \quad f(0,2) = -8. 
\]
Siirrytään reunan osalle $\partial A_2$. Tämä on puoliympyrän kaari, joka parametrisoituu 
muodossa
\[ 
x = \sin t, \quad y=1-\cos t, \quad t \in [0,\pi]. 
\]
Tutkittava funktio on näin ollen
\begin{align*}
 F_2(t) &= f(\sin t,1-\cos t) \\
        &= 6\sin^2t+4\sin t\cos t+9\cos^2 t+12\sin t+4\cos t-13, \quad t \in [0,\pi].
\end{align*}
Funktion $F_2$ kriittiset pisteet löytyvät ratkaisemalla yhtälö
\[ 
F_2'(t)= \ldots = -3\sin 2t + 4\cos 2t + 12\cos t - 4\sin t = 0. 
\]
Numeerisin keinoin saadaan ainoaksi käyväksi ratkaisuksi
\[ 
t = 0.927295 .., \quad\ (x(t),y(t)) = (0.800000..,0.400000..) = (4/5,2/5). 
\]
Tässä pisteessä on $F_2(0.927295..) = 8$. Koska reunan osien $\partial A_1$ ja $\partial A_2$
päätepisteet ovat samat, niin on päätelty, että $f$:n pienin ja suurin arvo löytyvät joukosta
$\{-121/9,\,0,\,-8,\,8\}$. Optimointitehtävien ratkaisut ovat siis
\begin{align*}
&f_{\text{min}} = -121/9 = -13.4444.. \quad 
                               \text{pisteessä}\ (x,y) = (0,11/9) = (0,\,1.1111..), \\
&f_{\text{max}} = 8 \quad \text{pisteessä}\ (x,y) = (4/5,\,2/5) = (0.8,\,0.4). \loppu
\end{align*}

\subsection{Lagrangen kertojien menetelmä}
\index{Lagrangen!c@kertojien menetelmä|vahv}

Edellä esitettyä ratkaisutapaa voi periaatteessa soveltaa useammankin muuttujan rajoitettuun
optimointitehtävään. Ongelmaksi kuitenkin muodostuvat sidotut ääriarvotehtävät, joita joudutaan
ratkaisemaan optimoitaessa kohdefunktiota tarkasteltavan joukon reunalla. Useamman kuin kahden
muuttujan tapauksessa tällaisten tehtävien ratkaiseminen parametrisoimalla, eli vapaiden
muuttujien lukumäärää vähentämällä, on usein hankalaa. Kolmen tai useamman muuttujan sidotuissa 
ääriarvotehtävissä käytetäänkin yleensä toista, paljon elegantimpaa ja yleispätevämpää
menetelmää, jota keksijänsä \hist{J. L. Lagrangen} (1736-1813) mukaan sanotaan 
\kor{Lagrangen kertojien} (engl.\ Lagrange multipliers) \kor{menetelmäksi}. 

Tarkastellaan aluksi jälleen kahden muuttujan tilannetta. Olkoon etsittävä funktion $f(x,y)$ 
paikalliset ääriarvokohdat käyrällä $S$, jonka määrittelee sidosehto $g(x,y)=0$. Olkoon 
$P=(x_0,y_0) \in S$ eräs tällainen ääriarvokohta ja olkoon $\vec t\,$ $S$:n tangenttivektori
pisteessä $P$. Tällöin on oltava
\begin{multicols}{2} \raggedcolumns
\[
\vec t\cdot\nabla f(x_0,y_0)=0.
\]
\begin{figure}[H]
\setlength{\unitlength}{1cm}
\begin{center}
\begin{picture}(4,2)
\curve(0,0,1,0.5,3.5,1.2)
\put(0.9,0.4){$\bullet$}
\put(1,0.5){\vector(3,1){2}}
\put(2.8,1.3){$\vec t$}
\put(0.9,0){$P$}
\put(3.7,1.15){$g(x,y)=0$}
\end{picture}
\end{center}
\end{figure}
\end{multicols}
Toisaalta koska käyrä $g(x,y)=0$ funktion $g$ tasa-arvokäyrä, niin pätee myös 
(vrt.\ Luku \ref{gradientti})
\[
\vec t\cdot\nabla g(x_0,y_0)=0.
\]
Näin ollen, sikäli kuin $\nabla g(x_0,y_0)\neq\vec 0$ (mikä jatkossa oletetaan --- pisteet, 
joissa $g(x_0,y_0)=0$ ja $\nabla g(x_0,y_0)=\vec 0$, on tutkittava erikseen), on oltava jollakin
$\lambda\in\R$
\[
\nabla f(x_0,y_0)+\lambda\nabla g(x_0,y_0)=\vec 0.
\]
Kysyttyjä paikallisia ääriarvopisteitä voidaan siis hakea ratkaisemalla yhtälöryhmä
\[
\begin{cases}
\,f_x(x,y)+\lambda g_x(x,y)=0, \\
\,f_y(x,y)+\lambda g_y(x,y)=0, \\
\,g(x,y)=0.
\end{cases}
\]
Tässä on kolme tuntematonta ja kolme yhtälöä, joten ainakin muodollinen puoli on kunnossa.

Em.\ yhtälöryhmään päädytään helpommin muistettavalla tavalla, kun määritellään funktio
\[
F(x,y,\lambda)=f(x,y)+\lambda g(x,y).
\]
Tässä on siis rajoitusehdossa esiintyvä funktio $g$ tuotu mukaan kertoimella $\lambda$ ---
\kor{Lagrangen kertojalla} --- painotettuna. Ym.\ yhtälöryhmään päädytään nyt yksinkertaisesti,
kun haetaan $F$:n kriittiset pisteet ehdoista
\[
\begin{cases}
\,\partial_x F(x,y,\lambda)=0, \\
\,\partial_y F(x,y,\lambda)=0, \\
\,\partial_\lambda F(x,y,\lambda)=0.
\end{cases}
\]
Em.\ geometrinen perustelu toimii myös kolmen muuttujan tapauksessa. Nimittäin jos 
$P=(x_0,y_0,z_0)$ on funktion $f(x,y,z)$ paikallinen ääriarvopiste pinnalla $S:\,g(x,y,z)=0$,
niin jokaiselle pinnan tangenttivektorille $\vec t$ pisteessä $P$ on oltava voimassa
\[
\vec t\cdot\nabla f(x_0,y_0,z_0)=0.
\]
Tällöin $\nabla f(x_0,y_0,z_0)$ on pinnan $S$ normaali, joten jollakin $\lambda\in\R$
\[
\nabla f(x_0,y_0,z_0)+\lambda\nabla g(x_0,y_0,z_0)=\vec 0
\]
(olettaen, että $\nabla g(x_0,y_0,z_0)\neq\vec 0\,$). Ääriarvopisteet löytyvät tällöin funktion
\[
F(x,y,z,\lambda)=f(x,y,z)+\lambda g(x,y,z)
\]
kriittisten pisteiden joukosta.
\begin{Exa}
Hae funktion $f(x,y,z)=xy+yz-xz$ ääriarvot yksikköpallon pinnalla.
\end{Exa}
\ratk Rajoitusehto on $x^2+y^2+z^2=1$, joten haetaan funktion
\[
F(x,y,z,\lambda)=xy+yz-xz+\lambda(x^2+y^2+z^2-1)
\]
kriittisiä pisteitä:
\[
\begin{cases}
\,\partial_x F =y-z+2\lambda x = 0, \\
\,\partial_y F =x+z+2\lambda y = 0, \\
\,\partial_z F =y-x+2\lambda z = 0, \\
\,\partial_\lambda F =x^2+y^2+z^2-1 = 0.
\end{cases}
\]
Kolme ensimmäistä yhtälöä on matriisimuodossa
\[
\begin{bmatrix} 2\lambda & 1 & -1 \\ 1 & 2\lambda & 1 \\ -1 & 1 & 2\lambda \end{bmatrix}
\begin{bmatrix} x \\ y \\ z \end{bmatrix}=\mo.
\]
Tälle haetaan ei-triviaalia ratkaisua, joten kerroinmatriisin on oltava singulaarinen. 
Kerroinmatriisin determinantti on
\[
\det(\mA)=8\lambda^3-6\lambda-2=2(\lambda-1)(2\lambda+1)^2,
\]
joten on oltava joko $\lambda=1$ tai $\lambda=-1/2$. Kun $\lambda=1$, tulee yhtälöryhmäksi
\[
\begin{rmatrix} 2 & 1 & -1 \\ 1 & 2 & 1 \\ -1 & 1 & 2 \end{rmatrix}
\begin{bmatrix} x \\ y \\ z \end{bmatrix}=\mo \ \underset{\text{(Gauss)}}{\ekv} \ 
\begin{rmatrix} 2 & 1 & -1 \\ 0 & \frac{3}{2} & \frac{3}{2} \\ 0 & 0 & 0 \end{rmatrix}
\begin{bmatrix} x \\ y \\ z \end{bmatrix}=\mo.
\]
Tämän ratkaisu yhdessä rajoitusehdon $x^2+y^2+z^2=1$ kanssa on
\[
(x,y,z)=t(1,-1,1),\quad t=\pm 1/\sqrt{3}\,,
\]
ja funktion $f$ arvo näissä pisteissä on
\[
f\left(\pm\frac{1}{\sqrt{3}},\mp\frac{1}{\sqrt{3}},\pm\frac{1}{\sqrt{3}}\right)
=\underline{\underline{-1}}.
\]
Jos $\lambda=-1/2$, niin yhtälöryhmäksi tulee
\[
\begin{rmatrix} -1 & 1 & -1 \\ 1 & -1 & 1 \\ -1 & 1 & -1 \end{rmatrix}
\begin{bmatrix} x \\ y \\ z \end{bmatrix}=\mo \ \ekv \
\begin{rmatrix} -1 & 1 & -1 \\ 0 & 0 & 0 \\ 0 & 0 & 0 \end{rmatrix}
\begin{bmatrix} x \\ y \\ z \end{bmatrix}=\mo.
\]
Rajoitusehdon toteuttavia ratkaisuja ovat $\,(x,y,z)=(t-s,t,s)$, missä $t,s\in\R$
toteuttavat
\[
(t-s)^2+t^2+s^2=1 \ \ekv \ 2(t^2+s^2-ts)=1.
\]
Funktion $f$ arvoksi näissä pisteissä tulee
\begin{align*}
f(x,y,z) &= (t-s)t+ts-(t-s)s \\
&= t^2+s^2-ts=\underline{\underline{1/2}}.
\end{align*}
Siis
\begin{alignat*}{2}
f_{\min} &= -1  \quad & &\text{pisteissä } \ \pm\frac{1}{\sqrt{3}}(1,-1,1), \\
f_{\max} &= 1/2 \quad & 
         &\text{käyrällä}\ S:\,(x,y,z)=(t-s,t,s),\ \ (t,s)\in\R^2,\ t^2+s^2-ts=1/2.
\end{alignat*}
Maksimipisteet muodostavat yksikköpallon isoympyrän tasolla $T: x-y+z=0$. \loppu

Lagrangen kertojien menetelmä toimii, poikkeustilanteita lukuunottamatta, myös useamman 
rajoitusehdon tapauksessa. Jos on haettava funktion $f(x_1,\ldots,x_n)$ ääriarvot 
rajoitusehdoilla
\[
g_i(x_1,\ldots,x_n)=0,\quad i=1\ldots m<n,
\]
niin nämä löydetään funktion
\[
F(\mx,\boldsymbol{\lambda})=f(\mx)+\sum_{i=1}^m \lambda_i g_i(\mx)
\]
kriittisten pisteiden joukosta. Esimerkiksi tapauksessa $n=3$, $m=2$ on tutkittava funktiota
\[
F(x,y,z,\lambda,\mu)=f(x,y,z)+\lambda g_1(x,y,z)+\mu g_2(x,y,z).
\]
Menetelmän idean voi tässäkin tapauksessa perustella geometrisesti: Rajoitusehtojen voi olettaa
määrittelevän avaruuskäyrän $K$, joka on pintojen
\[
S_1: \ g_1(x,y,z)=0,\quad S_2: \ g_2(x,y,z)=0
\]
leikkaus. Jos pintojen normaalit käyrän pisteessä $(x_0,y_0,z_0)$ ovat
\[
\vec n_i=\nabla g_i(x_0,y_0,z_0),\quad i=1,2,
\]
niin käyrän $K$ tangentti $\vec t$ on näitä kumpaakin vastaan kohtisuora. Ääriarvoehto on samaa
muotoa kuin yhden rajoitusehdon tapauksessa, eli on oltava
\[
\vec t\cdot\nabla f(x_0,y_0,z_0)=0.
\]
Koska tässä $\vec t$ siis on kohtisuorassa vektoreita $\vec n_1$ ja $\vec n_2$ vastaan, niin
sikäli kuin $\vec n_1$ ja $\vec n_2$ ovat lineaarisesti riippumattomat, on oltava
$\nabla f(x_0,y_0,z_0)=\lambda\vec n_1+\mu\vec n_2$ jollakin $(\lambda,\mu)\in\R^2$ eli
\[
\nabla f(x_0,y_0,z_0)=\lambda\nabla g_1(x_0,y_0,z_0)+\mu\nabla g_2(x_0,y_0,z_0).
\]
\begin{Exa}
Minimoi $\,f(x,y,z)=x+2y-z\,$ ehdoilla $x^2+y^2+z^2=12$ ja $x+y+z=1$. 
\end{Exa}
\ratk Etsitään funktion 
\[ 
F(x,y,z,\lambda,\mu)=x+2y-z+\lambda(x^2+y^2+z^2-12)+\mu(x+y+z-1)
\]
kriittiset pisteet:
\[
\begin{cases}
 \,2\lambda x+\mu = -1 \\ 
 \,2\lambda y+\mu = -2 \\ 
 \,2\lambda z+\mu = 1 \\
 \,z^2+y^2+z^2 = 12 \\ 
 \,x+y+z = 1
\end{cases} 
\impl \ \ \begin{cases} \,2\lambda(x+y+z)+3\mu=-2 \\ \,x+y+z=1 \end{cases}
\impl \ \ 2\lambda=-3\mu-2.
\]
Yhtälöryhmän kolmesta ensimmäisestä ja neljännestä yhtälöstä seuraa myös
\[ 
4\lambda^2(x^2+y^2+z^2)=(\mu+1)^2+(\mu+2)^2+(\mu-1)^2=48\lambda^2 . 
\]
Tässä on edellisen tuloksen perusteella $48\lambda^2 = 12(3\mu+2)^2$, joten seuraa
\begin{align*}
105\mu^2+140\mu+42=0\,\ 
                 &\impl\quad\ (\mu,\lambda)=\begin{cases}
                                         (-0.877485, \ \phantom{-}0.316228), \\
                                         (-0.455848, \ -0.316228)
                                         \end{cases} \\
                 &\impl\,\ (x,y,z)=\begin{cases}
                                   (-0.19371, \ -1.77485, \ \phantom{-}2.96856), \\
                                   (\phantom{-}0.86038, \ \phantom{-}2.44152, \ -2.30190).
                                   \end{cases}
\end{align*}
Näistä ensimmäinen piste antaa minimiarvon $f_{\min}=\underline{\underline{-6.71197}}$.
\loppu

Lagrangen kertojien menetelmää käytettäessä ei syntyvän epälineaarisen yhtälöryhmän
ratkaiseminen luonnollisesti aina onnistu käsinlaskulla. Ratkaisut etsitään silloin
numeerisin keinoin, tavallisimmin Newtonin menetelmällä. Numeeriseen ratkaisemiseen
yhdistettynä Lagrangen kertojien menetelmä näyttääkin itse asiassa parhaat puolensa, 
sillä ratkaisualgoritmista tulee tällä tavoin hyvin suoraviivainen ja helposti 
(konevoimin) toteutettava. 
\begin{Exa} Esimerkissä \ref{esim: rajoitettu optimointi} ratkaistiin sidosehdon 
parametrisoinnilla ääriarvotehtävä
\begin{align*}
&f(x,y)=6x^2-4xy+9y^2+16x-22y = \text{max !} \\
&\text{ehdolla} \quad g(x,y)=x^2+(y-1)^2-1=0.
\end{align*}
Lagrangen kertojaa käytettäessä etsitään funktion
\[
F(x,y,\lambda) = 6x^2-4xy+9y^2+16x-22y + \lambda[x^2+(y-1)^2-1]
\]
kriittiset pisteet eli ratkaistaan yhtälöryhmä
\[
\begin{cases}
\,6x-2y+x\lambda-8=0, \\ -2x+9y+(y-1)\lambda-11=0, \\ \,x^2+(y-1)^2-1=0.
\end{cases}
\]
Kun ratkaisussa käytetään Newtonin menetelmää, saadaan iteraatiokaavaksi
\[
\begin{bmatrix} x_{k+1} \\ y_{k+1} \\ \lambda_{k+1} \end{bmatrix}
= \begin{bmatrix} x_k \\ y_k \\ \lambda_k \end{bmatrix}
- \begin{bmatrix} 
  \lambda_k+6 & -2 & x_k \\ -2 & \lambda_k+9 & y_k-1 \\ 2x_k & 2y_k-2 & 0
  \end{bmatrix}^{-1}
  \begin{bmatrix}
  6x_k-2y_k+x_k\lambda_k-8 \\ -2x_k+9y_k+(y_k-1)\lambda_k-11 \\ x_k^2+(y_k-1)^2-1
  \end{bmatrix}.
\]
Alkuarvauksella $(x_0,y_0,\lambda_0)=(1,0.5,0)$ saadaan
\begin{center}
\begin{tabular}{llll}
$k$ & $x_k$      & $y_k$      & $\lambda_k$  \\ \hline \\
$0$ & $1.000000$ & $0.500000$ & $\phantom{-1}0.0000$ \\
$1$ & $1.000000$ & $0.750000$ & $-12.5000$ \\
$2$ & $0.815287$ & $0.136146$ & $-14.9283$ \\
$3$ & $0.818006$ & $0.376563$ & $-14.8590$ \\
$4$ & $0.800355$ & $0.399765$ & $-14.9999$ \\
$5$ & $0.800000$ & $0.400000$ & $-15.0000$
\end{tabular}
\end{center}
\end{Exa}
Suppeneminen oli hieman onnekasta, sillä alkuarvaus osoittautui Lagrangen kertojan osalta
kehnoksi. Parempi alkuarvaus olisi voitu laskea esim.\ seuraavasti (yleisempi menettely: ks.\
Harj.teht.\,\ref{H-udif-8: lambdan alkuarvaus}):
\begin{align*}
&\vec a=\nabla f(x_0,y_0)=26\vec i-17\vec j, \quad \vec b=\nabla g(x_0,y_0)=2\vec i-\vec j, \\
&\vec a+\lambda\vec b \approx \vec 0\,\ \impl\,\ 
\lambda \approx -\frac{\vec a\cdot\vec b}{|\vec b\,|^2} = -13.8 = \lambda_0. \loppu
\end{align*}

\Harj
\begin{enumerate}

\item
Määritä funktion suurin ja pienin arvo annetuilla rajoitusehdoilla. Käytä
sidosehdoissa (mikäli niitä on) parametrisointia. \vspace{1mm}\newline
a) \ $x-x^2+y^2, \quad 0 \le x \le 2,\,\ 0 \le y \le 1$ \newline
b) \ $xy-2x, \quad -1 \le x \le 1,\,\ 0 \le y \le 1$ \newline
c) \ $xy-x^3y^2, \quad 0 \le x,y \le 1$ \newline
d) \ $xy(a-2x-y), \quad \abs{x} \le a,\,\ \abs{y} \le a$ \newline
e) \ $xy(1-x-2y), \quad x,y \ge 0,\,\ x+y \le 1$ \newline 
f) \ $x^2-2y^2-xy-x, \quad x,y \ge -1,\,\ x+y \le 1$ \newline
g) \ $xy-x^2, \quad x^2+y^2 \le 1$ \newline
h) \ $3+x-x^2-y^2, \quad 2x^2+y^2 \le 1$ \newline
i) \ $y^2-2x^2, \quad x^2-y \le 1,\,\ x+y \le 1$ \newline 
j) \ $xy+x-y, \quad x^2+y^2 \le 1,\,\ x+y+1 \ge 0,\,\ x-y-1 \le 0$ \newline
k) \ $x^3+3y^3, \quad 0 \le xy \le 1,\,\ x^2+y^2 \le 4$ \newline
l) \ $\sin x+\sin y-\sin(x+y), \quad 0 \le x,y \le \pi$ \newline
m) \ $\sin x\cos y, \quad x,y \ge 0,\,\ x+y \le 2\pi$ \newline
n) \ $\sin x\sin y\sin(x+y), \quad x,y \ge 0,\,\ x+y \le \pi$ \newline
o) \ $(x-3y)e^y, \quad \abs{x} \le 1,\,\ \abs{y} \le 1$ \newline
p) \ $xe^{-x^2-y^2}, \quad x^2+y^2 \le R^2$ \newline
q) \ $(x+2y)e^{-x^2-y^2}, \quad (x,y)\in\R^2$ \newline
r) \ $xy^2e^{-xy}, \quad x,y \ge 0$ \newline
s) \ $(2x-y)(1+x^2+y^2)^ {-1}, \quad (x,y)\in\R^2$ \newline
t) \ $(1+2x+3y)(1+x^2+y^2)^{-1}, \quad (x,y)\in\R^2$ \newline
u) \ $xy^2+yz^2, \quad x^2+y^2+z^2 \le 1$ \newline
v) \ $xy+yz, \quad x^2+y^2+z^2 \le 1$ \newline
x) \ $xy^2z^3, \quad x^2+y^2+z^2 \le 1$ \newline
y) \ $xyze^{-x^2-2y^2-3z^2}, \quad (x,y,z)\in\R^3$ \newline
z) \ $(x+y+z)e^{-x^2-2y^2-3z^2}, \quad (x,y,z)\in\R^3$ \newline
å) \ $(x+2y+3z+4u)e^{-x^2y^2z^2u^2}, \quad (x,y,z,u)\in\R^4$

\item
Halutaan määrätä funktion $f(x,y,z)=xy+z^2$ minimi- ja maksimiarvo pallopinnalla
$S:\,x^2+y^2+z^2=1$. Ratkaise tehtävä \ a) parametrisoimalla $S$ pallonpintakoordinaateilla, \
b) Lagrangen kertojien avulla. \ c) Mitkä ovat $f$:n minimi- ja maksimiarvot ehdolla
$x^2+y^2+z^2 \le 1$\,?

\item
Määritä Lagrangen kertojien avulla funktion
\[
\text{a)}\ \ f(x,y,z)=x^2+y^2+z^2 \qquad \text{b)}\ \ f(x,y,z)=xy
\]
pienin ja suurin arvo pallon $S_1:\,x^2+y^2+z^2+4x+4y=0$ ja lieriön $S_2:\, x^2+y^2+2x+2y=6$ 
leikkauskäyrällä.

\item
Ratkaise seuraavat sidotut ääriarvotehtävät Lagrangen kertojien avulla. \vspace{1mm}\newline
a) \ $x^2+y^2=$ min!/max! käyrällä $S:\,x^2+8xy+7y^2=45$ \newline
b) \ $x^2+y^2=$ min!/max! käyrällä $S:\,x^4+x^2y^2+5y^4=95$ \newline
c) \ $x^3y^5=$ max! ehdolla $x+y=8$ \newline
d) \ $x+2y-3x=$ min!/max! pallopinnalla $x^2+y^2+z^2=1$ \newline
e) \ $x^2+y^2+z^2=$ min! ehdolla $xyz^2=2$ \newline
f) \ $xyz=$ min!/max! pallopinnalla $x^2+y^2+z^2=12$ \newline
g) \ $x=$ min!/max! käyrällä $S:\,z=x+y\,\ja\,x^2+2y^2+2z^2=8$ \newline
h) \ $x^2+y^2+z^2=$ min!/max! käyrällä $S:\,z^2=x^2+y^2\,\ja\,x-2z=3$ \newline
i) \ $z=$ min!/max! käyrällä $S:\,x^2+y^2=8\,\ja\,x+y+z=1$ \newline
j) \ $x^3+y^3+z^3=$ min!/max! käyrällä $S:\,x+y+z=3\,\ja\,x^2+y^2+z^2=27$ \newline
k) \ $\ma^T\mx=$ min!/max! ehdolla $\abs{\mx}=1\,\ (\ma,\mx\in\R^n)$ \newline
l) \ $x_1^{2m} + \ldots + x_n^{2m}=$ min! ehdolla $x_1 + \ldots + x_n=a \neq 0\,\ (m\in\N)$

\item
Ratkaise Lagrangen kertojaa käyttäen: \vspace{1mm}\newline
a) \ \ Mikä on origon lyhin etäisyys käyrästä $S:\,x^2+8xy+7y^2=45\,$? \newline
b) \   Mikä pinnan $S:\,xy+2yz+3xz=0$ piste $Q$ on lähinnä pistettä 
       $\phantom{\text{b) \ \ }} P=(-2,2,-2)\,$?

\item
Tetraedrin yksi kärki on origossa ja muut kolme kärkeä ovat origosta suuntiin $\vec i$,
$\vec i-\vec j+\vec k$ ja $-\vec i+\vec j+\vec k$. Lisäksi origon etäisyys tetraedrin
vastakkaisesta sivutahkosta $=3$. Mikä on tetraedrin pienin tilavuus näillä ehdoilla?

\item
a) Näytä, että suorakulmaisista särmöistä, joilla on sama sivutahkojen yhteenlaskettu
pinta-ala, kuutio on tilavuudeltaan suurin. \vspace{1mm}\newline
b) Kanneton suorakulmaisen särmiön muotoinen laatikko valmistetaan kahdesta eri materiaalista
siten, että pohjamateriaali on (pintayksikköä kohti) viisi kertaa kalliimpaa kuin laatikon
seinämien materiaali. Miten laatikon mitat on valittava, kun halutaan minimoida 
materiaalikustannukset ehdolla, että laatikon tilavuus $=a^3$\,?

\item
a) Ellipsoidin muotoinen jalokivi
\[
A: \quad \frac{x^2}{a^2}+\frac{y^2}{b^2}+\frac{z^2}{c^2}\,\le\,1
\]
halutaan hioa suorakulmaisen särmiön muotoiseksi. Miten särmiön mitat on valittava, jotta
kiveä menisi hukkaan mahdollisimman vähän? \vspace{1mm}\newline
b) Näytä, että jos $\alpha,\beta$ ja $\gamma$ ovat kolmion kulmat, niin
\[
\sin\frac{\alpha}{2}\sin\frac{\beta}{2}\sin\frac{\gamma}{2}\,\le\,\frac{1}{8}\,.
\]

\item
Laske (tarvittaessa numeerisin keinoin) funktion $f(x,y)=(x-y^2)e^{xy}$ pienin ja suurin
arvo joukossa \ a) $A:\,x^2+y^2 \le 4$, \ b) $A:\,x^4+y^4 \le 16$.

\item (*) \index{aritmeettinen keskiarvo} \index{geometrinen keskiarvo}
          \index{keskiarvo!b@aritmeettinen, geometrinen}
Todista väittämä: Lukujen $a_k \ge 0,\ k=1 \ldots n\,$ \kor{geometrinen keskiarvo} on enintään
yhtä suuri kuin \kor{aritmeettinen keskiarvo}, ts.\ 
\[
\sqrt[n]{a_1 \cdots a_n}\,\le\,\frac{1}{n}(a_1+ \ldots + a_n).
\]
\kor{Vihje}: Aseta sopiva sidosehto!

\item (*)
Etsi funktion $f(x,y,z)=(x+2y)e^{xz}+xyz+y^3$ ääriarvokohdat pallopinnalla
$S:\,(x-3)^2+(y+3)^2+(z+4)^2=16$. Menetelmä: Lagrangen kertojat ja Newton!

\item (*) \label{H-udif-8: lambdan alkuarvaus}
Yleisessä Lagrangen kertojien menetelmässä etsitään funktion
$F(\mx,\boldsymbol{\lambda})=f(\mx)+\sum_{i=1}^m \lambda_i g_i(\mx),\ \mx\in\R^n,\ m<n$
kriittistä pistettä $(\mc,\boldsymbol{\mu})$. Oletetaan, että eräässä $\mc$:n ympäristössä
$U_\delta(\mc)$ vektorit $\Nabla g_i(\mx),\ i=1 \ldots m\,$ ovat lineaarisesti riippumattomat
ja että on käytettävissä alkuarvaus $\mx_0 \in U_\delta(\mc)$. Näytä, että jos alkuarvaus
$\boldsymbol{\lambda}_0$ määrätään $\mx_0$:n avulla laskemalla
\[
|\Nabla f(\mx_0)-\sum_{i=1}^m\lambda_i\Nabla g_i(\mx_0)|^2 = \text{min!} 
                \ \ \impl\ \ (\lambda_i)=\boldsymbol{\lambda}_0,
\]
($|\cdot| = \R^n$:n euklidinen normi), niin pätee:
$\mx_0=\mc\ \impl\ \boldsymbol{\lambda}_0=\boldsymbol{\mu}$. Millaiseen 
$\boldsymbol{\lambda}_0$:n laskukaavaan tämä menettely johtaa tapauksessa $m=1$\,? Sovella
menettelyä sidottuun ääriarvotehtävään
\begin{align*}
&f(x,y,z)=x^2+y^2+z^2 = \text{min!} \\
&\text{ehdoilla} \quad x^3y+xy^3+2=0, \quad x^3y+y^3z+xz^3+1=0,
\end{align*}
kun alkuarvaus on $(x_0,y_0,z_0)=(1,-1,1)$. Määritä $F$:n kriittinen piste likimäärin yhdellä
Newtonin iteraation askeleella.

\end{enumerate}
 %Usean muuttujan ääriarvotehtävät
\section{Usean muuttujan Taylorin polynomit} \label{usean muuttujan taylorin polynomit}
\alku \sectionmark{Usean muuttujan Taylor}
\index{Taylorin polynomi!a@usean muuttujan|vahv}

Tarkastellaan kahden muuttujan funktiota $f=f(x,y)$ pisteitä $(x_0,y_0)$ ja $(x,y)$
yhdistävällä janalla $S$. Tällä janalla $f$ voidaan tulkita yhden muuttujan funktiona
\[
g(t)=f(x_0+h_xt,y_0+h_yt),\quad t\in [0,1],
\]
missä on merkitty
\[
h_x=x-x_0,\quad h_y=y-y_0.
\]
\begin{figure}[H]
\setlength{\unitlength}{1cm}
\begin{center}
\begin{picture}(6,3)
\path(0,0)(6,3) \put(3,1.5){\vector(2,1){0.1}}
\put(0.9,0.4){$\bullet$} \put(0.55,0.8){$t=0$} \put(0.8,0){$(x_0,y_0)$}
\put(4.9,2.4){$\bullet$} \put(4.55,2.8){$t=1$} \put(4.8,2){$(x,y)$}
\put(3,1.1){$t$} \put(2.5,1.5){$S$} \curve(2.4,1.6,2.2,1.4,2,1)
\end{picture}
\end{center}
\end{figure}
Oletetaan, että funktion $f$ osittaisderivaatat $\partial^\alpha$ ovat kertalukuun 
$\abs{\alpha}=n+1$ asti olemassa ja jatkuvia janalla $S$. Tällöin voidaan osittaisderivoinnin
ketjusäännöistä päätellä, että funktio $g$ on $n+1$ kertaa jatkuvasti derivoituva välillä 
$[0,1]$, jolloin Taylorin lauseen mukaan pätee jokaisella $t\in (0,1]$
\[
g(t)=\sum_{k=0}^n \frac{1}{k!}g^{(k)}(0)\,t^k+\frac{1}{(n+1)!}g^{(n+1)}(\xi)\,t^{n+1},
\]
missä $\xi\in (0,t)$. Kun erityisesti valitaan $t=1$, on tulos
\begin{equation} \label{udif-9: eq1}
f(x,y)=g(1)=\sum_{k=0}^n\frac{1}{k!}g^{(k)}(0)+\frac{1}{(n+1)!}g^{(n+1)}(\xi). \tag{$\star$}
\end{equation}
Tässä voidaan derivaatta $g'(t)$ laskea ketjusäännön perusteella muodossa
\begin{align*}
g'(t) &= h_x f_x(x_0+h_xt,y_0+h_yt)+h_y f_y(x_0+h_xt,y_0+h_yt) \\
      &=[\,(h_x\partial_x+h_y\partial_y)f\,](x_0+h_xt,y_0+h_yt).
\end{align*}
Soveltamalla tätä sääntöä uudelleen nähdään, että
\[
g^{(k)}(t)=[\,(h_x\partial_x+h_y\partial_y)^kf\,](x_0+h_xt,y_0+h_yt). 
\]
Tässä $h_x$ ja $h_y$ tulkitaan derivoitaessa vakioiksi (koska derivointi kohdistuu muuttujaan
$t$). Tuloksen perusteella $f(x,y)$:n lauseke \eqref{udif-9: eq1} voidaan tulkita
\kor{kahden muuttujan Taylorin kaavana}
\[
\boxed{\begin{aligned}\ykehys
f(x,y)   &= T_n(x,y,x_0,y_0)+R_n(x,y), \quad \text{missä} \\
\quad T_n(x,y,x_0,y_0) 
         &= \sum_{k=0}^n \frac{1}{k!}\,[\,(h_x\partial_x+h_y\partial_y)^kf\,](x_0,y_0), \\
R_n(x,y) &=\frac{1}{(n+1)!}\,[\,(h_x\partial_x + h_y\partial_y)^{n+1}f\,](x(\xi),y(\xi)),\\[2mm]
h_x      &= x-x_0,\quad h_y=y-y_0, \\
x(\xi)   &= x_0+h_x\xi,\quad y(\xi)=y_0+h_y\xi,\quad \xi\in (0,1). \akehys\quad
\end{aligned}}
\]
Tässä $T_n(x,y,x_0,y_0)$ on $f$:n \kor{Taylorin polynomi} astetta $n$ pisteessä $(x_0,y_0)$ ja
\index{jzyzy@jäännöstermi (Lagrangen)}%
$R_n$ on \kor{jäännöstermi}. Huomioimalla, että (binomikaavan perusteella, vrt.\
Luku \ref{osittaisderivaatat})
\[
(h_x\partial_x+h_y\partial_y)^k f 
        = \sum_{l=0}^k \binom{k}{l}h_x^{k-l}h_y^l\partial_x^{k-l}\partial_y^l f, \quad
          \binom{k}{l}=\frac{k!}{(k-l)!\,l!}\,,
\]
saadaan polynomille $T_n(x,y,x_0,y_0)$ konkreettisempi esitysmuoto
\[
\boxed{\quad T_n(x,y,x_0,y_0)=\sum_{k=0}^n\sum_{l=0}^k \frac{1}{(k-l)!\, l!}\,
   \frac{\partial^k f}{\partial x^{k-l}\partial y^l}(x_0,y_0)\,(x-x_0)^{k-l}(y-y_0)^l. \quad}
\]
Tästä ja helposti todennettavasta derivointikaavasta
\[
\left[\frac{\partial^{i+j}}{\partial x^i\partial y^j}(x-x_0)^m(y-y_0)^n\right]_
                                  {\left|\begin{array}{l} 
                                   \scriptstyle{x=x_0} \\ 
                                   \scriptstyle{y=y_0} \end{array}\right.}
=\begin{cases} m!\,n!, &\text{jos $i=m$ ja $j=n$}, \\ 0, &\text{muulloin} \end{cases}
\]
nähdään, että polynomi $p(x,y)=T_n(x,y,x_0,y_0)$ toteuttaa ehdot
\[
\partial^\alpha p(x_0,y_0)=\partial^\alpha f(x_0,y_0),\quad \abs{\alpha}\leq n.
\]
Taylorin polynomi myös määräytyy yksikäsitteisesti näistä ehdoista, kuten yhden muuttujan
tapauksessa (vrt.\ Luku \ref{taylorin lause}).
\begin{Exa}
Määrää funktion
\[
f(x,y)=e^{-x}+xy+y^2+\cos x\,\cos y
\]
Taylorin polynomi $T_2(x,y,0,0)$.
\end{Exa}
\ratk Derivoimalla todetaan
\begin{align*}
&f(0,0)=2, \quad f_x(0,0)=-1, \quad f_y(0,0)=0, \\
&f_{xx}(0,0)=0, \quad f_{xy}(0,0)=f_{yx}(0,0)=f_{yy}(0,0)=1,
\end{align*}
joten
\begin{align*}
T_2(x,y,0,0) &= 2 +(-1) \cdot x + 0 \cdot y
                  +\frac{1}{2!0!} \cdot 0 \cdot x^2 
                  +2\cdot\frac{1}{1!1!} \cdot 1 \cdot xy
                  +\frac{1}{0!2!} \cdot 1 \cdot y^2 \\
             &= 2-x+xy+\frac{1}{2}y^2. \loppu
\end{align*}

Taylorin kaava siis pätee, kunhan $f$:n osittaisderivaatat ovat kertalukuun $n+1$ asti jatkuvia
tarkastelupisteen $(x,y)$ ja pisteen $(x_0,y_0)$ välisellä yhdysjanalla. Taylorin kaava
on siten kahdenkin muuttujan  tapauksessa johtoajatukseltaan yksiulotteinen. Jos halutaan, että
kaava pätee tarkastelupisteen liikkuessa jossakin joukossa $A$ pisteen $(x_0,y_0)$
ympäristössä, niin onkin oletettava, $f$:n riittävän säännöllisyyden lisäksi, että $A$ on
\index{tzy@tähden muotoinen}%
pisteen $(x_0,y_0)$ suhteen nk. \kor{tähden muotoinen} (engl.\ star-shaped). Tämä tarkoittaa,
että jokaiseen pisteeseen $(x,y)\in A$ on pisteestä $(x_0,y_0)$ 'näköyhteys', eli pisteiden
välinen yhdysjana on kokonaisuudessaan joukossa $A$. Näin ajatellen voidaan Taylorin kaavasta
johtaa seuraava yleinen polynomiapproksimaatiotulos kahden muuttujan funktiolle.
\begin{Lause} \label{Taylor-R2} \index{Taylorin lause!c@kahden muuttujan|emph}
Jos funktion $f=f(x,y)$ osittaisderivaatat ovat kertalukuun $n+1$ asti jatkuvia pisteen 
$(x_0,y_0)$ suhteen tähden muotoisessa ja kompaktissa joukossa $A$, niin jokaisella 
$(x,y)\in A$ pätee
\[
f(x,y)=T_n(x,y,x_0,y_0)+\Ord{h^{n+1}}, \quad h=\sqrt{(x-x_0)^2+(y-y_0)^2}.
\]
Jos $f$:n osittaisderivaatat ovat $A$:ssa jatkuvia  kertalukuun $n$ asti, niin 
\[
f(x,y)=T_n(x,y,x_0,y_0)+\ord{h^n}, \quad (x,y) \in A.
\]
\end{Lause}
\tod Jos $f$:n osittaisderivaatat ovat kertalukuun $n+1$ jatkuvia kompaktissa joukossa $A$,
niin ne ovat myös rajoitettuja (ks.\ Lause \ref{weierstrass - Rn}):
\[
\abs{\partial^\alpha f(x,y)}\leq C_\alpha\quad\forall\,(x,y) \in A,\quad \abs{\alpha}\leq n+1.
\]
Valitsemalla $\,C=\max_{\abs{\alpha}=n+1} C_\alpha\,$ seuraa Taylorin kaavan jäännöstermille arvio
\begin{align*}
\abs{R_n(x,y)} &\le \frac{C}{(n+1)!}\sum_{l=0}^{n+1} 
                                    \binom{n+1}{l}\abs{x-x_0}^{n+1-l}\abs{y-y_0}^l \\
               &=\frac{C}{(n+1)!}\,\left(\abs{x-x_0}+\abs{y-y_0}\right)^{n+1} \\
               &\le \frac{C(\sqrt{2})^{n+1}}{(n+1)!}\,h^{n+1}=\Ord{h^{n+1}}, \quad (x,y) \in A.
\end{align*}
Astetta heikompien säännöllisyysoletusten ollessa voimassa päätellään vastaavasti
(vrt.\ Lause \ref{Taylorin approksimaatiolause})
\begin{align*}
f(x,y) &= T_{n-1}(x,y,x_0,y_0)+R_{n-1}(x,y) = T_n(x,y,x_0,y_0)+\ord{h^n}. \loppu
\end{align*}

Tapauksessa $n=1$ Lauseen \ref{Taylor-R2} jälkimmäinen väittämä on: Jos $f$, $f_x$,
$f_y$ ovat jatkuvia pisteen $(x_0,y_0)$ (tähden muotoisessa) ympäristössä, niin
\begin{align*}
f(x,y) &= T_1(x,y,x_0,y_0)+\ord{h} \\
&= f(x_0,y_0)+f_x(x_0,y_0)(x-x_0)+f_y(x_0,y_0)(y-y_0)+\ord{h}.
\end{align*}
Tulos on ennestään tuttu, sillä tehdyin oletuksin $f$ on differentioituva pisteessä $(x_0,y_0)$
(vrt.\ Luku \ref{gradientti}). Lauseen \ref{Taylor-R2} jälkimmäistä väittämää voidaan käyttää
hyväksi myös määrättäessa $f$:n Taylorin polynomia: Väittämän mukaan riittää löytää mikä
tahansa polynomi $p(x,y)$, jolle pätee
\[
f(x,y)-p(x,y)=\ord{h^{n}},\quad h=\sqrt{(x-x_0)^2+(y-y_0)^2}.
\]
Tällainen polynomi on välttämättä yksikäsitteinen --- siis $\ p(x,y)=T_n(x,y,x_0,y_0)$
(vrt.\ Propositio \ref{Taylor-prop}).
\jatko \begin{Exa} (jatko) Esimerkin funktiolle pätee
\begin{align*}
f(x,y)&=1-x+\frac{1}{2}x^2+\Ord{\abs{x}^3}+xy+y^2
                          +(1-\frac{1}{2}x^2)(1-\frac{1}{2}y^2)+\ordoO{x^4+y^4} \\
&= 2-x+xy+\frac{1}{2}y^2+\Ord{\abs{x}^3+\abs{y}^3} \\
&= 2-x+xy+\frac{1}{2}y^2+\Ord{h^3},
\end{align*}
joten myös tällä perusteella on
\[
T_2(x,y,0,0)=2-x+xy+\frac{1}{2}y^2. \loppu
\]
\end{Exa}

\subsection{Monen muuttujan Taylorin polynomit}

Em.\ tarkastelut yleistyvät helposti kolmen ja useamman muuttujan funktioihin. Kolmen muuttujan
funktion $f=f(x,y,z)$ Taylorin polynomi määritellään
\begin{align*}
T_n(x,y,z,x_0,y_0,z_0) 
    &= \sum_{k=0}^n \frac{1}{k!}(h_x\partial_x+h_y\partial_y+h_z\partial_z)^kf(x_0,y_0,z_0), \\
h_x &= x-x_0,\quad h_y=y-y_0,\quad h_z=z-z_0.
\end{align*}
Yleisemmin jos $f=f(\mx)=f(x_1,\ldots,x_d)$, niin $f$:n Taylorin polynomi astetta $n$ pisteessä 
$\ma=(a_1,\ldots,a_d)\in\R^d$ on
\[ \boxed{
\quad T_n(\mx,\ma)
  =\sum_{k=0}^n\frac{1}{k!}\bigl[\bigl(\sum_{i=1}^d h_i\partial_i\bigr)^kf\bigr](\ma),\quad
                               h_i=x_i-a_i,\quad i=1\ldots d. \quad}
\]
Mainittakoon ilman tarkempia perusteluja, että tämä lauseke on purettavissa seuraavaan muotoon,
joka ei ulkonäöltään juuri poikkea yhden muuttujan tilanteesta
(vrt.\ myös tapaus $d=2$ edellä):
\[ \boxed{ \begin{aligned}
\quad &T_n(\mx,\ma)=\sum_{\abs{\alpha}\le n}\frac{\ygehys 1}{\alpha !}\,
                             \partial^\alpha f(\ma)\,(\mx-\ma)^\alpha, \quad \\
      &\alpha ! = \prod_{i=1}^d \alpha_i\,!\,, \quad 
                  \mx^\alpha = \prod_{i=1}^d x_i^{\alpha_i}, \\
      &\alpha   = (\alpha_1, \ldots, \alpha_d), \quad \mx = (x_1, \ldots, x_d). \akehys
\end{aligned} } \]
Kaavassa summaus käy yli kaikkien erilaisten moni-indeksien 
(eli järjestettyjen indeksijoukkojen), joille $\abs{\alpha} = \alpha_1 + \ldots \alpha_d \le n$.

Taylorin polynomi  $p(\mx)=T_n(\mx,\ma)$ määräytyy aina myös yksikäsitteisesti polynomina 
astetta $n$, joka toteuttaa ehdot
\[
\partial^\alpha p(\ma)=\partial^\alpha f(\ma),\quad \abs{\alpha}\leq n.
\]
Sikäli kuin $f$ ja $f$:n osittaisderivaatat kertalukuun $n$ asti ovat jatkuvia pisteen $\ma$
ympäristössä, niin riittää myös löytää (keinolla millä hyvänsä) polynomi $p$ astetta $n$,
jolle pätee
\[ 
f(\mx) = p(\mx) + \ord{\abs{\mx-\ma}^n}.
\]
Tällöin on $p(\mx)=T_n(\mx,\ma)$.

\subsection{Polynomi $T_2(\mx,\ma)$ -- Hessen matriisi}

Monen muuttujan Taylorin polynomeja lasketaan harvemmin astelukua $n=2$ pidemmälle.
Em.\ kaavan mukaan $f$:n toisen asteen Taylorin polynomi on
\[
T_2(\mx,\ma) 
= f(\ma)+\sum_{i=1}^d \partial_i f(\ma)(x_i-a_i)
        + \frac{1}{2} \sum_{i=1}^d\sum_{j=1}^d \partial_i\partial_j f(\ma)(x_i-a_i)(x_j-a_j).
\]
Tämän voi esittää matriisimuodossa
\[
T_2(\mx,\ma)=f(\ma)+[\Nabla f(\ma)]^T(\mx-\ma)+\frac{1}{2}(\mx-\ma)^T\mH(\ma)(\mx-\ma),
\]
missä $\Nabla f(\ma)$ on gradientti ja $\mH(\ma)$ on $f$:n toisen kertaluvun
osittaisderivaatoista koostuva nk.\
\index{Hessen matriisi}%
\kor{Hessen matriisi} (engl.\ Hessian)\,:
\[
[\,\mH(\ma)\,]_{ij}=\frac{\partial^2 f}{\partial x_i\partial x_j}(\ma),\quad i,j=1\ldots d.
\]
\begin{Exa}
Määrää funktion
\[
f(x_1,\ldots,x_d)=\prod_{i=1}^d e^{c_i x_i}
\]
Taylorin polynomi $T_2(\mx,\mo)$.
\end{Exa}
\ratk Koska
\[
f(\mo)=1,\quad \frac{\partial f}{\partial x_i}(\mo)=c_i\,,\quad i=1\ldots d, \quad
\frac{\partial^2 f}{\partial x_i\partial x_j} = c_i c_j\,,\quad i,j=1\ldots d,
\]
niin
\begin{align*}
T_2(\mx,\mo) &= 1+\left(\sum_{i=1}^d x_i\,
                \frac{\partial}{\partial x_i}\right)f(\mo)+\frac{1}{2}\left(\sum_{i=1}^d 
                         x_i\,\frac{\partial}{\partial x_i}\right)^2f(\mo) \\
             &= 1 + \sum_{i=1}^d c_i x_i+\frac{1}{2}\sum_{i,j=1}^d c_i c_j x_i x_j\,.
\end{align*}
Kun merkitään $\mc=[c_1,\ldots,c_d]^T$ ja $\mH=(c_ic_j)=\mc\mc^T$ 
($=f$:n Hessen matriisi origossa), niin tulos saadaan muotoon
\begin{align*}
T_2(\mx,\mo) \,&=\, 1 + \mc^T\mx + \frac{1}{2} \mx^T\mH\mx \\
             \,&=\, 1 + \mc^T\mx + \frac{1}{2} (\mc^T\mx)^2.
\end{align*}
Lyhyempää tietä samaan tulokseen tullaan, kun huomataan käyttää reaalifunktion
$\exp(x)$ ominaisuuksia:
\begin{align*}
f(\mx) = \exp({\mc^T\mx}) &= 1+\mc^T\mx+\frac{1}{2}(\mc^T\mx)^2+\Ord{\abs{\mc^T\mx}^3} \\
                          &= 1+\mc^T\mx+\frac{1}{2}(\mc^T\mx)^2+\Ord{\abs{\mx}^3}. \loppu
\end{align*}

\Harj
\begin{enumerate}

\item
Laske seuraavien funktioiden Taylorin polynomit annettua astetta $n$ annetussa pisteessä.
Käytä tilaisuuden tullen apuna yhden muuttujan Taylorin polynomeja.
\vspace{3mm} \newline
a) \ $\D f(x,y)=2x^4-5y^3+2xy^2,\,\ (0,0),\,\ n=3$ \vspace{3mm}\newline
b) \ $\D f(x,y)=\sqrt{x^3y}\,,\,\ (1,4),\,\ n=4$ \vspace{1.5mm}\newline
c) \ $\D f(x,y)=\frac{1}{2+x-2y}\,,\,\ (2,1),\,\ n=3$ \vspace{1.5mm}\newline
d) \ $\D f(x,y)=y^2\ln x,\,\ (1,0),\,\ n=4$ \vspace{3mm}\newline
e) \ $\D f(x,y)=\sin(xy),\,\ (0,0),\,\ n\in\N$ \vspace{3mm}\newline
f) \ $\D f(x,y)=e^{xy}\sin y,\,\ (0,0),\,\ n=3$ \vspace{3mm}\newline
g) \ $\D f(x,y)=\cos(x+\sin y),\,\ (0,0),\,\ n=4$ \vspace{1.5mm}\newline
h) \ $\D f(x,y)=\int_0^{x+y^2-1} e^{-t^2}\,dt+\int_0^{\pi/2} \cos(xt^2+yt)\,dt,\,\ 
                                             (0,1),\,\ n=2$ \vspace{1.5mm}\newline
i) \ $\D x^2+2y^2+3z^2=6\ \ekv\ z=f(x,y),\,\ (1,1,1),\,\ n=2$ \vspace{3mm}\newline
j) \ $\D x+xy^2+2z-\sin z=0\ \ekv\ z=f(x,y),\,\ (0,0,0),\,\ n=2$ \vspace{3mm}\newline
k) \ $\D f(x,y,z)=e^{xyz},\,\ (0,0,0),\,\ n\in\N$ \vspace{1.5mm}\newline
l) \ $\D f(x,y,z)=\frac{1}{x+y+z}\,,\,\ (1,0,0),\,\ n=3$ \vspace{1.5mm}\newline
m) \ $\D f(\mx)=\cos(x_1 + \ldots + x_n),\,\ \mo,\,\ n=2$ \vspace{1.5mm}\newline
n) \ $\D f(\mx)=\frac{1}{x_1 + \ldots + x_n}\,,\,\ (1,\ldots,1),\,\ n=2$

\item
Laske funktion $f(x,y)=\ln(1+x^2+y^2)$ osittaisderivaatta $f_{xxxxxxyy}(0,0)$ Taylorin
polynomien avulla.

\item
Arvioi, kuinka suuri on seuraavien approksimaatioiden virhe enintään kiekossa
$A:\,x^2+y^2 \le a^2\ (a>0)$.
\begin{align*}
&\text{a)}\ \ e^{x+2y} \approx 1+x+2y \qquad\quad\,
 \text{b)}\ \ e^{x+2y} \approx 1+x+2y+\frac{1}{2}\,x^2+2xy+2y^2 \\
&\text{b)}\ \ \cos(xy)\approx 1-\frac{1}{2}\,x^2y^2 \qquad
 \text{d)}\ \ 2xy+2\cos(x-y) \approx 2+x^2+y^2
\end{align*}

\item
Määritä funktion toisen asteen Taylorin polynomi ja sen avulla Hessen matriisi
origossa:
\begin{align*}
&\text{a)}\ \ x\sin(x+y) \qquad
 \text{b)}\ \ xy\cos(x-y) \qquad\quad
 \text{c)}\ \ e^{x+y}+e^{x-y} \\
&\text{d)}\ \ \frac{1-x+x^2}{1+y-y^2} \qquad\
 \text{e)}\ \ \cos(x+y-2z) \qquad
 \text{f)}\ \ \frac{e^{xy}+e^{yz}+e^{xz}}{1+x+y+z}
\end{align*}

\item
Yhtälö $x^y = y^x\ \ekv\ y \ln x = x \ln y$ määrittelee käyrän, mahdollisesti useampia, pisteen
$(a, a)$ $(a > 0)$ ympäristössä. Tutki asiaa approksimoimalla funktiota 
$f(x, y) =  y \ln x - x \ln y$ ensimmäisen, toisen ja kolmannen asteen Taylorin polynomilla 
pisteessä $(a, a)$.

\end{enumerate} %Usean muuttujan Taylorin polynomit

\chapter{Matriisin ominaisarvot ja neliömuodot}

Matriiseja ja matriisialgebraa on aiemmin (Luku \ref{matriisit}) tarkasteltu lähinnä
lineaaristen yhtälöryhmien teorian ja ratkaisemisen näkökulmasta. Tässä luvussa avataan
matriisilaskuun ja sen sovelluksiin kokonaan uusi näkökulma, kun tarkastelun kohteeksi otetaan
matriisiin liitettävät erityiset luvut ja vektorit, nimeltään \kor{ominaisarvot} ja
\kor{ominaisvektorit}.

Luvussa \ref{matriisin ominaisarvot} määritellään matriisin ominaisarvot ja -vektorit, eli
yhdistettynä \kor{ominaisparit}, sekä esitetään ominaisarvoteorian keskeisimmät väittämät.
Osoittautuu, että yleisen reaalisen matriisin ominaisarvot ja -vektorit voivat olla
komp\-leksisia. Symmetrisen matriisin ominaisarvo-ongelma sen sijaan ratkeaa kokonaan
$\R^n$:ssä, ja osoittautuu, että tässä tapauksessa ominaisvektoreista voidaan myös muodostaa
$\R^n$:n ortonormeerattu kanta. Tähän perustuva matriisin \kor{diagonalisointi} on keskeisellä
sijalla Luvussa \ref{diagonalisointi}, jossa symmetrisen matriisin ominais\-arvoteoriaa
sovelletaan $\R^n$:n \pain{neliömuotoihin}. Tällä tavoin ensinnäkin ratkeaa (ainakin osittain)
Luvussa \ref{usean muuttujan ääriarvotehtävät} avoimeksi jätetty ongelma: Onko funktion
$f(\mx)$ kriittinen piste paikallinen minimi, maksimi vai ei kumpikaan? Neliömuotoihin liittyen
ratkaistaan Luvussa \ref{diagonalisointi} myös toisen asteen käyrien ja ja toisen asteen
pintojen geometrisen luokittelun ongelmat matriisien ominaisarvoteorian avulla.    

Luvussa \ref{pinnan kaarevuus} osoitetaan, että symmetriset matriisit ja niiden
ominaisarvoteoria ovat avaimia myös \kor{pinnan kaarevuuden} käsitteeseen. Viimeisessä
osaluvussa esitellään vielä lyhyesti käsite \kor{tensori}, jollaiseksi esim.\ pinnan kaarevuus
on ymmärrettävissä. Tensori ja sen erilaiset esiintymismuodot johdattavat pohtimaan uudelleen
myös vektorin ja skalaarin käsitteitä.
 %Matriisin ominaisarvot ja neliömuodot
\section{Matriisin ominaisarvot} \label{matriisin ominaisarvot}
\alku

Aloitetaan määritelmästä.
\begin{Def} \index{matriisin ($\nel$neliömatriisin)!i@$\nel$ominaisarvo, -vektori|emph}
Jos $\mA$ on neliömatriisi kokoa $n\times n$, niin kompleksiluku $\lambda\in\C$ on $\mA$:n 
\kor{ominaisarvo} (engl.\ eigenvalue, ruots.\ egenvärde, saks.\ Eigenwert), ja vektori
$\mx\in\C^n$, $\mx\neq\mo$, on ominaisarvoon $\lambda$ liittyvä \kor{ominaisvektori}
(engl.\ eigenvector), jos pätee
\[
\mA\mx=\lambda\mx.
\]
\end{Def}
Määritelmässä voi $\mA$ olla yhtä hyvin reaalinen kuin kompleksinen matriisi. Määritelmä siis 
\pain{ei} yksinkertaistu muotoon $\lambda\in\R$, $\mx\in\R^n$ siinäkään tapauksessa, että $\mA$
on reaalinen.

Määritelmästä nähdään heti, että ominaisvektorit eivät ole yksikäsitteiset: Jos $\mx$ on 
ominaisvektori, niin samoin on $\alpha\mx$ jokaisella $\alpha\in\C$, $\alpha\neq 0$. 
\index{normeeraus (vektorin)}%
Ominaisvektorit voidaan siis \kor{normeerata} mielivaltaisella tavalla, esimerkiksi ehdolla 
$\abs{\mx}=1$.

\index{ominaisarvo, -vektori, -pari}%
Matriisin ominaisarvot ja -vektorit, eli \kor{ominaisparit} (engl. eigenpair) määrittelevää 
yhtälöä $\mA\mx=\lambda\mx$ sanotaan \kor{ominaisarvoyhtälöksi}. Ominaisparien laskeminen 
aloitetaan määrämällä tästä yhtälöstä ensin kaikki ominaisarvot. Ominaisarvojen joukkoa, jota
\index{spektri (matriisin)} \index{matriisin ($\nel$neliömatriisin)!i@$\nel$spektri}%
sanotaan $\mA$:n \kor{spektriksi}, merkitään jatkossa symbolilla $\sigma(\mA)\,$:
\[
\sigma(\mA)=\{\mA\text{:n ominaisarvot}\} \subset\C.
\]
Spektrin avulla saadaan matriisin säännöllisyydelle jälleen uusi luonnehdinta:
\[
\boxed{\kehys\quad \mA\ \text{säännöllinen} \qekv 0\not\in\sigma(\mA). \quad}
\]

Kun ominaisarvoyhtälö kirjoitetaan muotoon
\[
(\mA-\lambda\mI)\mx=\mo
\]
($\mI=\text{yksikkömatriisi}$), niin matriisialgebran ja determinanttiopin mukaan tällä on 
ei-triviaali ratkaisu, eli $\lambda\in\sigma(\mA)$, täsmällen kun
\[
\det(\mA-\lambda\mI)=0.
\]
Tämä on siis \kor{ominaisarvoehto}. Koska $\lambda$ esiintyy tässä ainoastaan 
$(\mA-\lambda\mI)$:n diagonaalilla, niin alideterminanttisäännön 
(Lause \ref{alideterminanttisääntö}) perusteella nähdään, että ominaisarvoehto purkautuu
yhtälöksi muotoa
\[
p(\lambda)=0,
\]
missä $p(\lambda)$ on polynomi astetta $n$, tarkemmin
\[
p(\lambda)=(-1)^n\lambda^n+\{\text{alempiasteisia termejä}\}.
\]
\index{karakteristinen yhtälö (polynomi)!b@neliömatriisin}
\index{matriisin ($\nel$neliömatriisin)!j@$\nel$karakteristinen polynomi}%
Ominaisarvoehtoa $p(\lambda)=0$ sanotaan $\mA$:n \kor{karakteristiseksi yhtälöksi} ja
polynomia $p(\lambda)=\det(\mA-\lambda\mI)$ $\mA$:n \kor{karakteristiseksi polynomiksi}.
Matriisin $\mA$ ominaisarvot ovat siis $\mA$:n karakteristisen polynomin juuret, ja näin ollen
$\mA$:n spektri $\sigma(\mA)$ koostuu vähintään yhdestä ja enintään $n$:stä kompleksiluvusta 
(Algebran peruslause). Jos $\mA$ on reaalinen matriisi, niin karakteristinen polynomi
$p(\lambda)$ on reaalikertoiminen, jolloin juuret ovat joko reaalisia tai muodostavat 
konjugaattipareja. Jos $\lambda\in\sigma(\mA)$ on $p(\lambda)$:n $m$-kertainen juuri, niin
lukua $m_a(\lambda)=m$ sanotaan ko. ominaisarvon
\index{kertaluku!e@ominaisarvon} \index{algebrallinen kertaluku}%
\kor{algebralliseksi kertaluvuksi}. Algebran peruslauseen mukaan on
\[
\sum_{\lambda\in\sigma(\mA)} m_a(\lambda)=n.
\]
Jos $m_a(\lambda)=1$, on kyseessä
\index{yksinkertainen!b@ominaisarvo}%
\kor{yksinkertainen} (engl. simple) ominaisarvo. 

Kun ominaisarvo $\lambda\in\sigma(\mA)$ on löydetty (mahdollisesti numeerisin apukeinoin),
saadaan vastaavat ominaisvektorit selville ominaisarvoyhtälöstä $(\mA-\lambda\mI)\mx=\mo$, joka
ominaisarvon määritelmän mukaisesti on singulaarinen yhtälöryhmä $\C^n$:ssä. Kun tämä 
yhtälöryhmä saatetaan Gaussin algoritmilla singulaariseen perusmuotoonsa, saadaan kaikki 
ratkaisut selville. Yleisesti jos $\mx_1 \in \C^n$ ja $\mx_2 \in \C^n$ ovat ominaisarvoyhtälön
ratkaisuja, niin samoin on $\alpha_1\mx_1+\alpha_2\mx_2\ \forall \alpha_1,\alpha_2 \in \C$, 
joten tiettyyn ominaisarvoon liittyvät ominaisvektorit muodostavat $\C^n$:n aliavaruuden. Tätä
aliavaruutta sanotaan ko.\ ominaisarvoon liittyväksi
\index{ominaisavaruus}%
\kor{ominaisavaruudeksi} (engl.\ eigenspace).
Jos ominaisavaruuden dimensio on $\nu$, eli jos samaan ominaisarvoon
$\lambda$ voidaan liittää täsmälleen $\nu$ lineaarisesti riippumatonta ominaisvektoria, niin
sanotaan, että luku $\nu$ (aina $\nu\geq 1$) on ko.\ ominaisarvon
\index{geometrinen kertaluku}%
\kor{geometrinen kertaluku}, ja merkitään
\[
m_g(\lambda)=\nu.
\]
Jokaiseen ominaisarvoon $\lambda\in\sigma(\mA)$ liittyy siis kaksi kertalukua, algebrallinen
kertaluku $m_a(\lambda)$ ja geometrinen kertaluku $m_g(\lambda)$, jotka molemmat ovat
luonnollisia lukuja. Voidaan myös osoittaa, että aina pätee
\[ 
m_g(\lambda)\leq m_a(\lambda), \quad \lambda\in\sigma(\mA).
\]
\begin{Exa} \label{eig-ex1}
Määritä matriisin
\[
\mA=\begin{rmatrix} 2 & 0 & 4 \\ 0 & 6 & 0 \\ 4 & 0 & 2 \end{rmatrix}
\]
ominaisarvot ja ominaisvektorit.
\end{Exa}
\ratk Ominaisarvojen determinanttiehto on
\begin{align*}
\det(\mA-\lambda\mI) &= \begin{vmatrix} 
                        2-\lambda & 0 & 4 \\ 0 & 6-\lambda & 0 \\ 4 & 0 & 2-\lambda 
                        \end{vmatrix} \\ 
                     &= (6-\lambda)(\lambda^2-4\lambda-12) = -(6-\lambda)^2(\lambda+2) = 0
\end{align*}
(determinantti purettu toisen vaaka- tai pystyrivin mukaan), joten ominaisarvot ja niiden 
algebralliset kertaluvut ovat
\[ 
\lambda_1=6, \quad m_a(\lambda_1)=2, \qquad \lambda_2=-2, \quad m_a(\lambda_2)= 1. 
\]
Ominaisvektorit saadaan ratkaisemalla ominaisarvoyhtälö kummallekin ominaisarvolle erikseen:
\begin{align*}
(\mA-\lambda_1\mI)\mx=\mo \ &\ekv \ \begin{rmatrix} 
                                    -4 & 0 & 4 \\ 0 & 0 & 0 \\ 4 & 0 & -4 
                                    \end{rmatrix}
\begin{bmatrix} x_1 \\ x_2 \\ x_3 \end{bmatrix} = \mo \\ \\
&\ekv \ \begin{rmatrix} -4 & 0 & 4 \\ 0 & 0 & 0 \\ 0 & 0 & 0 \end{rmatrix}
\begin{bmatrix} x_1 \\ x_2 \\ x_3 \end{bmatrix} = \mo. \\ \\
(\mA-\lambda_2\mI)\mx=\mo \ &\ekv \ \begin{rmatrix} 
                                    4 & 0 & 4 \\ 0 & 8 & 0 \\ 4 & 0 & 4 
                                    \end{rmatrix}
\begin{bmatrix} x_1 \\ x_2 \\ x_3 \end{bmatrix} = \mo \\ \\
&\ekv \ \begin{rmatrix} 4 & 0 & 4 \\ 0 & 8 & 0 \\ 0 & 0 & 0 \end{rmatrix}
\begin{bmatrix} x_1 \\ x_2 \\ x_3 \end{bmatrix} = \mo.
\end{align*}
Nähdään, että  ominaisarvoon $\lambda_1$ liittyy kaksi lineaarisesti riippumatonta 
ominaisvektoria, esimerkiksi
\[
\mx_1=\begin{rmatrix} 1 \\ 0 \\ 1 \end{rmatrix},\quad 
\mx_2=\begin{rmatrix} 0 \\ 1 \\ 0 \end{rmatrix},
\]
ja ominaisarvoon $\lambda_2$ liittyy yksi lineaarisesti riippumaton ominaisvektori, esimerkiksi
\[
\mx_3=\begin{rmatrix} -1 \\ 0 \\ 1 \end{rmatrix}.
\]
Ominaisarvojen geometriset kertaluvut ovat näin ollen
\[
m_g(\lambda_1)=2,\quad m_g(\lambda_2)=1. \loppu
\]
\begin{Exa} \label{eig-ex2}
Laske matriisin
\[
\mA=\begin{rmatrix} 1 & 0 & -2 \\ 0 & 1 & -2 \\ 0 & 1 & -1 \end{rmatrix}
\]
ominaisarvot ja ominaisvektorit.
\end{Exa}
\ratk Karakteristinen polynomi on
\[
p(\lambda)=(1-\lambda)(\lambda^2+1),
\]
joten ominaisarvot ovat
\[
\lambda_1=1,\quad \lambda_2=i,\quad \lambda_3=-i.
\]
Samaan tapaan laskien kuin edellisessä esimerkeissä saadaan ominaisvektoreiksi
\[
\mx_1=\begin{bmatrix} 1 \\ 0 \\ 0 \end{bmatrix},\quad
\mx_2=\begin{bmatrix} 1+i \\ 1+i \\ 1 \end{bmatrix},\quad
\mx_3=\begin{bmatrix} 1-i \\ 1-i \\ 1 \end{bmatrix}.
\]
Ominaisarvojen kertaluvut ovat
\[
m_a(\lambda_i)=m_g(\lambda_i)=1,\quad i=1,2,3. \loppu
\]

\begin{Exa} \label{eig-ex3} Laske diagonaalimatriisin $\mA=\text{diag}\, (d_i, \ i=1\ldots n)$
ominaisarvot ja ominaisvektorit. 
\end{Exa}
\ratk Karakteristinen yhtälö on
\[
\det(\mA-\lambda\mI)=\prod_{i=1}^n (d_i-\lambda)=0.
\]
Jos $d_i\neq d_j$, kun $i\neq j$, niin ominaisarvot ja niiden algebrallisest kertaluvut ovat
\[
\lambda_i=d_i,\quad m_a(\lambda_i)=1,\quad i=1\ldots n.
\]
Jos $d_i=d \ \forall i$ (jolloin $\mA=d\mI$), niin $\mA$:lla on yksi $n$-kertainen ominaisarvo:
\[
\lambda=d,\quad m_a(\lambda)=n.
\]
Jos yleisemmin luku $d$ esiintyy $\mA$:n diagonaalilla täsmälleen $m$ kertaa, niin ominaisarvon
$\lambda=d$ algebrallinen kertaluku on $m_a(\lambda)=m$. Myös geometrinen kertaluku on 
$m_g(\lambda)=m$, sillä ominaisarvoyhtälöstä nähdään, että ominaisvektoreita ovat $\R^n$:n 
peruskantavektorit $\me_i$ jokaisella $i$, jolla $d_i=d$. Erityisesti jos $d_i=d \ \forall i$,
niin jokainen vektori $\mx\in\R^n$, $\mx\neq \mo$, on ominaisvektori. \loppu

Tähänastisissa esimerkeissä ovat omainaisarvojen algebralliset ja geometriset kertaluvut olleet
samat. Seuraavasta esimerkistä nähdään, että näin ei ole aina.
\begin{Exa} \label{eig-ex4} Määrää $\mA$:n ominaisarvot ja ominaisvektorit, kun
\[
\mA=\begin{bmatrix} 
    a & 1 \\ & a & 1 \\ & & a & \ddots \\ & & & \ddots & 1 \\ & & & & a 
    \end{bmatrix}
\]
eli
\[
A=(a_{ij})_{i,j=1}^n,\quad a_{ij}
 = \begin{cases} 
   \,a, &\text{kun } i=j, \\ \,1, &\text{kun } j=i+1, \\ \,0, &\text{muulloin}.
   \end{cases}
\]
\end{Exa}
\ratk Kyseessä on kolmiomatriisi, jonka karakteristinen yhtälö on
\[
\det(\mA-\lambda\mI)=(a-\lambda)^n=0.
\]
Tämän mukaan $\lambda=a$ on $n$-kertainen ominaisarvo ($m_a(\lambda)=n$). Kun $\lambda=a$, niin
ominaisarvoyhtälöt ovat
\[
(\mA-\lambda\mI)\mx=\mo \ \ekv \ \left\{ \begin{aligned}
x_2 &= 0 \\
x_3 &= 0 \\
& \ \, \vdots \\
x_n &= 0
\end{aligned} \right.
\]
joten $\mA$:lla on vain yksi lineaarisesti riippumaton ominaisvektori $\mx=[1,0,\ldots,0]^T$. 
Siis ainoan ominaisarvon geometrinen kertaluku on $m_g(\lambda)=1$. \loppu

Todetaan tässä yhteydessä vielä matriisin ominaisvektoreiden lineaarista riippuvuutta
rajoittava yleinen tulos.
\begin{Lause} \label{ominaisvektorien lineaarinen riippumattomuus}
\index{lineaarinen riippumattomuus|emph} Matriisin eri ominaisarvoihin liittyvät ominaisvektorit
ovat keskenään lineaarisesti riippumattomat.
\end{Lause}
\tod Olkooon $\mA\ma_i=\lambda_i\ma_i,\ i=1 \ldots m$, missä $\ma_i\neq\mo$ ja 
$\lambda_i\neq\lambda_j$, kun $i \neq j$. Kun merkitään
\[
\mB=(\mA-\lambda_1\mI)(\mA-\lambda_2\mI) \cdots (\mA-\lambda_{m-1}\mI) 
   = \prod_{k=1}^{m-1}(\mA-\lambda_k\mI),
\]
niin oletusten perusteella
\[
\mB\ma_j = \prod_{k=1}^{m-1}(\lambda_j-\lambda_k)\,\ma_j 
         = \begin{cases} 
            \,\mo, \quad &j=1 \ldots m-1, \\ 
            \,\prod_{k=1}^{m-1}(\lambda_m-\lambda_k)\,\ma_m, \quad &j=m. 
           \end{cases} 
\]
Oletetaan nyt, että
\[
\sum_{j=1}^m x_j\ma_j = \mo.
\]
Tällöin seuraa
\[
\mo=\mB\sum_{j=1}^m x_j\ma_j\,=\,\sum_{j=1}^m x_j\mB\ma_j\,
                              =\,\prod_{k=1}^{m-1}(\lambda_m-\lambda_k)x_m\ma_m\,,
\]
joten on oltava $x_m=0$. Siis
\[
\sum_{j=1}^{m-1} x_j\ma_j = \mo.
\]
Jos $m=2$, niin tämä yhtälö toteutuu vain kun $x_1=0$, jolloin on päätelty, että $x_1=x_2=0$.
Jos $m>2$, niin kertomalla yhtälö puolittain matriisilla
\[
\mB=\prod_{k=1}^{m-2}(\mA-\lambda_k\mI)
\]
päätellään samalla tavoin kuin edellä, että $x_{m-1}=0$, jne. Siis vektorit 
$\ma_1, \ldots, \ma_m$ ovat lineaarisesti riippumattomat:
\[
\sum_{j=1}^m x_j\ma_j = \mo \qimpl x_m = x_{m-1} = \ldots = x_1 = 0. \loppu
\]

Lauseella \ref{ominaisvektorien lineaarinen riippumattomuus} on seuraava ilmeinen seuraamus:
\begin{Kor} Jos $\mA$ on kokoa $n\times n$ ja kaikki $\mA$:n ominaisarvot ovat yksinkertaisia,
niin $\mA$:n ominaisvektoreista voidaan muodostaa $\C^n$:n kanta.
\end{Kor}

\subsection{Symmetrinen matriisi}
\index{neliömatriisi!a@symmetrinen|vahv}

Reaalisen ja symmetrisen matriisin ominaisarvoteoria voidaan pelkistää seuraavaksi kauniiksi
lauseeksi. Lauseen täydellinen todistus sopii paremmin lineaarialgebran laajemman esittelyn
yhteyteen, joten rajoitutaan tässä todistamaan vain kaksi lauseeseen sisältyvää (helpompaa)
osaväittämää.
\begin{Lause} \label{symmetrisen matriisin ominaisarvoteoria} Olkoon $\mA$ reaalinen ja 
symmetrinen matriisi kokoa $n\times n$. Tällöin $\mA$:n ominaisarvot ovat reaaliset,
ominaisarvojen algebralliset ja geometriset kertaluvut ovat samat, ja $\mA$:n ominaisvektoreista
voidaan muodostaa $\R^n$:n ortonormeerattu kanta.
\end{Lause}
\tod (osittainen) \ \, Jos
\[
\mA\mx=\lambda\mx,\quad \lambda\in\C,\ \mx\in\C^n,\ \mx\neq\mo,
\]
niin
\[
\scp{\mA\mx}{\mx}=\lambda\scp{\mx}{\mx}=\lambda\abs{\mx}^2,
\]
missä $\scp{\cdot}{\cdot}$ on $\C^n$:n euklidinen skalaaritulo ja $\abs{\cdot}$ on euklidinen
normi:
\[
\scp{\mx}{\my}=\sum_{i=1}^n x_i\overline{y}_i,\quad \abs{\mx}=\sqrt{\scp{\mx}{\mx}}.
\]
Konjugoimalla saatu yhtälö puolittain saadaan
\[
\overline{\scp{\mA\mx}{\mx}}=\overline{\lambda\abs{\mx}^2}=\overline{\lambda}\abs{\mx}^2.
\]
Mutta $\mA$:n oletetun reaalisuuden ja symmetrian perusteella on
(vrt.\ Luku \ref{matriisialgebra})
\[
\overline{\scp{\mA\mx}{\mx}}=\scp{\mx}{\mA\mx}=\scp{\mA\mx}{\mx},
\]
joten $\lambda\abs{\mx}^2=\overline{\lambda}\abs{\mx}^2$. Tässä on $\abs{\mx}^2\in\R_+$, joten
$\lambda=\overline{\lambda}$, eli $\lambda\in\R$.

Olkoot $\lambda$ ja $\mu$ $\mA$:n kaksi eri suurta ominaisarvoa ja $\mx$ ja $\my$ vastaavat
ominaisvektorit:
\[
\mA\mx=\lambda\mx\,\ \ja\,\ \mA\my=\mu\my.
\]
Jo todetun perusteella on $\lambda,\mu\in\R$, joten seuraa
\begin{align*}
\scp{\mA\mx}{\my} &= \scp{\lambda\mx}{\my}=\lambda\scp{\mx}{\my}, \\
\scp{\mx}{\mA\my} &= \scp{\mx}{\mu\my}=\mu\scp{\mx}{\my}.
\end{align*}
Mutta koska $\mA$ on reaalinen ja symmetrinen, niin
\[
\scp{\mA\mx}{\my}=\scp{\mx}{\mA\my},
\]
joten seuraa
\[
(\lambda-\mu)\scp{\mx}{\my}=0 \ \impl \ \scp{\mx}{\my}=0.
\]
Yhdistämällä tähänastiset päätelmät on todistettu, että $\mA$:n ominaisarvot ovat reaaliset, ja
että eri ominaisarvoihin liittyvät ominaisvektorit ovat ortogonaaliset. Todistuksen loppuosa
sivuutetaan. \loppu

Lauseen \ref{symmetrisen matriisin ominaisarvoteoria} tuloksesta seuraa, että reaalisen ja 
symmetrisen matriisin tapauksessa ei kompleksilukuja tarvita myöskään ominaisvektoreissa. 
Nimittäin jos $\mx\in\C^n$ on ominaisvektori ja $\mx=\mpu+i\mpv$, $\mpu,\mpv\in\R^n$, niin 
$\mA$:n, $\lambda$:n, $\mpu$:n ja $\mpv$:n reaalisuuden perusteella
\begin{align*}
\mA\mx=\lambda\mx \ &\ekv \ \mA\mpu+i\mA\mpv=\lambda\mpu+i\lambda\mpv \\
&\ekv \ \mA\mpu=\lambda\mpu \ \ja \ \mA\mpv=\lambda\mpv.
\end{align*}
Reaalisen ja symmetrisen matriisin ominaisarvoyhtälö $\mA\mx=\lambda\mx$ voidaan siis ratkaista
kokonaan $\R^n$:ssä.
\begin{Exa} \label{eig-ex5} Symmetrisen matriisin
\[
\mA=\begin{rmatrix} 2 & 0 & 4 \\ 0 & 6 & 0 \\ 4 & 0 & 2 \end{rmatrix}
\]
ortonormeeratut ominaisvektorit ovat (vrt.\ Esimerkki \ref{eig-ex1})
\[
\mx_1=\frac{1}{\sqrt{2}}[1,0,1]^T, \quad 
\mx_2=[0,1,0]^T, \quad 
\mx_3=\frac{1}{\sqrt{2}}[-1,0,1]^T. \loppu
\]
\end{Exa}

\subsection{Diagonalisoituva matriisi}
\index{diagonalisointi!a@matriisin|vahv}

Olkoon $\mA$ reaalinen ja symmetrinen matriisi kokoa $n\times n$ ja olkoon $\mC$ reaalinen
matriisi, jonka sarakkeina ovat $\mA$:n ortonormeeratut ominaisvektorit:
\[
\mC=[\mc_1\ldots\mc_n],\quad 
         \mA\mc_i=\lambda_i\mc_i, \quad i=1\ldots n, \quad \scp{\mc_i}{\mc_j}=\delta_{ij}.
\]
Tässä voidaaan ominaisarvoyhtälöt $\mA\mc_i=\lambda_i\mc_i$ koota matriisiyhtälöksi
\[
[\mA\mc_1\ldots\mA\mc_n] \,=\, [\lambda_1\mc_1\ldots\lambda_n\mc_n]
                         \,=\, [\mC(\lambda_1\me_1)\ldots\mC(\lambda_n\me_n)]
\]
eli (vrt.\ Luku \ref{matriisialgebra})
\[
\mA\mC=\mC\boldsymbol{\Lambda},
\]
missä $\boldsymbol{\Lambda}=[\lambda_1\me_1\ldots\lambda_n\me_n]=\text{diag}(\lambda_i)$. Koska
$\mC$ on ortogonaalinen, on $\inv{\mC}=\mC^T$, joten seuraa
\[
\boxed{\kehys\quad \mC^T\mA\mC=\boldsymbol{\Lambda}
                          \qekv \mA=\mC\boldsymbol{\Lambda}\mC^T. \quad}
\]
Tässä tuloksessa siis $\mC$:n sarakkeina ovat $\mA$:n \pain{ortonormeeratut} ominaisvektorit 
jossakin järjestyksessä ja $\boldsymbol{\Lambda}$ on diagonaalimatriisi, jonka lävistäjällä ovat
$\mA$:n ominaisarvot $\mC$:n määräämässä järjestyksessä siten, että $\mA\mc_i=\lambda\mc_i$,
$\lambda_i=[\boldsymbol{\Lambda}]_{ii}$. Annetaan tässä yhteydessä hieman yleisempi määritelmä.
\begin{Def} (\vahv{Similariteettimuunnos}) \label{similaarimuunnos}
\index{similaarisuus|emph} \index{ortogonaalisuus!d@similariteettimuunnoksen|emph}
\index{neliömatriisi!g@diagonalisoituva|emph}
Jos $\mA$ ja $\mB$ ovat 
samankokoisia neliömatriiseja, ja on olemassa säännöllinen matriisi $\mC$ siten, että 
$\mB=\mC^{-1}\mA\mC$, niin sanotaan, että $\mA$ ja $\mB$ ovat \kor{similaariset}, ja kuvausta
$\mA \map \mB$ sanotaan \kor{similariteettimuunnokseksi}. Similariteettimuunnos on 
\kor{ortogonaalinen}, jos $\mC$ on ortogonaalinen matriisi. Jos $\mA$ on similaarinen 
diagonaalimatriisin kanssa, niin sanotaan, että $\mA$ on \kor{diagonalisoituva}.
\end{Def}
Reaalinen ja symmetrinen matriisi on siis määritelmän mukaisesti diagonalisoituva, ja tässä
tapauksessa matriisi $\mC$ on myös valittavissa niin, että diagonalisoiva similariteettimuunnos
on ortogonaalinen. Yleisemmin voidaan $\mC$:n sarakkeiksi valita mitkä tahansa $n$ lineaarisesti
riippumatonta ominaisvektoria. Tällaisen ominaisvektorisysteemin olemassaolo on myös
välttämätön ehto sille, että matriisi on diagonalisoituva:
\begin{Lause} \label{diagonalisoituva matriisi} Matriisi $\mA$ kokoa $n \times n$ on 
diagonalisoituva täsmälleen kun $\mA$:lla on $n$ lineaarisesti riippumatonta ominaisvektoria.
Tällöin jos matriisin $\mC$ sarakkeina ovat mitkä tahansa $\mA$:n lineaarisesti riippumattomat
ominaisvektorit ja diagonaalimatriisin $\boldsymbol{\Lambda}$ lävistäjäalkioina ovat vastaavat
ominaisarvot, niin $\mA=\mC\boldsymbol{\Lambda}\mC^{-1}$.
\end{Lause}
\tod Jos $\mA$:lla on lineaarisesti riippumattomat ominaisvektorit $\mc_i\,,\ i=1 \ldots n$,
niin muodostamalla nämä sarakkeina matriisi $\mC=[\mc_1 \ldots \mc_n]$ todetaan kuten edellä,
että $\mA\mC=\mC\boldsymbol{\Lambda}\ \ekv\ \mA=\mC\boldsymbol{\Lambda}\mC^{-1}$, missä
$\boldsymbol{\Lambda}=\text{diag}\,(\lambda_i)$ ja $\lambda_i=$ ominaisvektoriin $\mc_i$ 
liittyvä ominaisarvo. Tämä todistaa väittämän osan \fbox{$\Leftarrow$}\,. Osan \fbox{$\impl$} 
todistamiseksi olkoon $\mA=\mC^{-1}\mD\mC$, missä $\mC$ on säännöllinen ja 
$\mD=\text{diag}\,(d_i)$ on diagonaalimatriisi. Tällöin pätee
\[
\mD\me_i=d_i\me_i \qekv (\mC^{-1}\mD\mC)(\mC^{-1}\me_i)=d_i\mC^{-1}\me_i, \quad i=1 \ldots n,
\]
joten matriisilla $\mA=\mC^{-1}\mD\mC$ on ominaisvektorit $\mc_i=\mC^{-1}\me_i$ ja vastaavat
ominaisarvot $d_i$. Vektorit $\mc_i$ ovat lineaarisesti riippumattomat, sillä
\[
\sum_{i=1}^n x_i\mc_i=\mo \qekv \mC\sum_{i=1}^n x_i\mc_i=\sum_{i=1}^n x_i\me_i=\mo 
                          \qekv x_i=0,\ i= 1 \ldots n.
\]
Siis $\{d_i,\ i=1 \ldots n\}\subset\sigma(\mA)$ ja $\mA$:lla on $n$ lineaarisesti riippumatonta
ominaisvektoria vastaten näitä ominaisarvoja. Tällöin on myös oltava 
$\{d_i,\ i=1 \ldots n\}=\sigma(\mA)$, sillä muu johtaisi ristiriitaan Lauseen 
\ref{ominaisvektorien lineaarinen riippumattomuus} kanssa. \loppu
\begin{Exa} Matriisilla
\[
\mA=\begin{bmatrix} 1&1\\0&1 \end{bmatrix}
\]
on vain yksi ominaisvektori (Esimerkki \ref{eig-ex4}), joten $\mA$ ei ole diagonalisoituva
(Lause \ref{diagonalisoituva matriisi}). \loppu
\end{Exa}
\begin{Exa}
Määrittele diagonalisoiva muunnos $\mA\map\mD=\inv{\mC}\mA\mC$ matriisille
\[
\mA=\begin{rmatrix} 2 & 0 & 4 \\ 0 & 6 & 0 \\ 4 & 0 & 2 \end{rmatrix}.
\]
\end{Exa}
\ratk saadaan suoraan edellisen luvun Esimerkeistä \ref{eig-ex1} ja \ref{eig-ex5}:
\[
\begin{rmatrix} 6&0&0\\0&6&0\\0&0&-2 \end{rmatrix}=
\begin{rmatrix} 
\frac{1}{\sqrt{2}}&0&\frac{1}{\sqrt{2}}\\0&1&0\\-\frac{1}{\sqrt{2}}&0&\frac{1}{\sqrt{2}} 
\end{rmatrix}
\begin{rmatrix} 2&0&4\\0&6&0\\4&0&2 \end{rmatrix} 
\begin{rmatrix} 
\frac{1}{\sqrt{2}}&0&-\frac{1}{\sqrt{2}}\\0&1&0\\ \frac{1}{\sqrt{2}}&0&\frac{1}{\sqrt{2}} 
\end{rmatrix}.
\]
Tässä on valittu ortogonaalinen muunnos. Mahdollista on myös valita $\mC$:n sarakkeiksi
esim.\ vektorit $\mc_1=[1,1,1]^T$, $\mc_2=[1,0,-1]^T$ ja $\mc_3=[0,1,0]^T$, sillä nämäkin
ovat $\mA$:n lineaarisesti riippumattomia ominaisvektoreita (vastaten ominaisarvoja 
$\lambda_1=6$, $\lambda_2=-2$ ja $\lambda_3=6$, ks.\ edellisen luvun Esimerkki \ref{eig-ex1}).
Kysytyksi muunnokseksi saadaan tällä valinnalla
\[
\begin{rmatrix} 6&0&0\\0&-2&0\\0&0&6 \end{rmatrix}=
\begin{rmatrix} 1&1&0\\1&0&1\\1&-1&1 \end{rmatrix}^{-1}
\begin{rmatrix} 2&0&4\\0&6&0\\4&0&2 \end{rmatrix}
\begin{rmatrix} 1&1&0\\1&0&1\\1&-1&1 \end{rmatrix} \loppu
\]

\Harj
\begin{enumerate}

\item
Todista: \vspace{1mm}\newline
a) Jos $\lambda\in\sigma(\mA)$, niin $\lambda^p\in\sigma(\mA^p)\ \forall p\in\N$. \newline
b) Jos $\mA$ on säännöllinen ja $\lambda\in\sigma(\mA)$, niin 
   $\lambda^{-1}\in\sigma(\mA^{-1})$.
\newline
c) Jos $\mA$ on reaalinen, niin $\sigma(\mA^T)=\sigma(\mA)$. \newline
d) Jos $1\in\sigma(\mA)$ tai $2\in\sigma(\mA)$, niin $\mB=2\mI-3\mA+\mA^2$ on singulaarinen.

\item 
Määritä seuraavien matriisien ominaisarvot ja -vektorit sekä ominaisarvojen algebralliset ja
geometriset kertaluvut.
\begin{align*}
&\text{a)}\ \ \begin{rmatrix} 1&-2\\-2&1 \end{rmatrix} \qquad
 \text{b)}\ \ \begin{rmatrix} 1&-1\\1&1 \end{rmatrix} \qquad
 \text{c)}\ \ \begin{rmatrix} 4&4\\-1&0 \end{rmatrix} \\[2mm]
&\text{d)}\ \ \begin{rmatrix} 3&1&1\\1&3&1\\1&1&3 \end{rmatrix} \qquad
 \text{e)}\ \ \begin{rmatrix} -2&1&1\\0&0&4\\0&-1&-4 \end{rmatrix} \qquad
 \text{f)}\ \ \begin{rmatrix} 1&2&1\\0&2&-2\\1&0&3 \end{rmatrix} \\[2mm]
&\text{g)}\ \ \begin{rmatrix} 2&2&3\\4&1&0\\1&-2&0 \end{rmatrix} \qquad
 \text{h)}\ \ \begin{rmatrix} 1&0&0&1\\0&2&0&0\\9&0&3&0\\1&0&0&4 \end{rmatrix} \qquad
 \text{i)}\ \ \begin{rmatrix} 1&0&0&4\\0&1&3&0\\0&2&1&0\\1&0&0&1 \end{rmatrix}
\end{align*}

\item
Olkoon
\[
\mA=\begin{rmatrix} 1&1&0\\0&2&0\\2&1&-1 \end{rmatrix}, \quad
\mB=\begin{bmatrix} 1&t&t\\t&1&t\\t&t&1 \end{bmatrix}, \quad
\mC=\begin{bmatrix}
    \alpha&1&1&\beta \\ 4&\alpha&\beta&2 \\ 4&\beta&\alpha&2 \\ \beta&3&3&\alpha    
    \end{bmatrix}.
\]
a) Matriisin $\mA$ ominaisvektoreita ovat $[1,1,1]^T$, $[1,0,1]^T$ ja $[0,0,1]^T$. Laske tätä
tietoa hyväksi käyttäen $\mA^{11}\mx$, kun $\mx=[2,1,1]^T$. \vspace{1mm}\newline
b) Matriisin $\mB$ ominaisarvot ovat reaalifunktioita $t\map\lambda_i(t)$. Hahmottele näiden
kuvaajat! \vspace{1mm}\newline
c) Matriisilla $\mC$ on ominaisarvo $\lambda$ ja vastaava ominaisvektori $[0,1,-1,0]^T$.
Millainen ehto tästä seuraa luvuille $\alpha,\beta$ ja $\lambda$\,?

\item
a) Näytä, että jos neliömatriisille pätee $\mA^2=\mA$, niin $\sigma(\mA)\subset\{0,1\}$. Näytä
esimerkillä, että vaihtoehdot $\sigma(\mA)=\{0\}$, $\sigma(\mA)=\{1\}$ ja $\sigma(\mA)=\{0,1\}$
ovat kaikki mahdollisia. \vspace{1mm}\newline
b) Lineaarikuvaus $\mx\map\mA\mx$, missä $\mA$ on kokoa $3 \times 3$, vastaa projektiota
tasolle $T:\,x+3y-2z=0$ suoran $S:\,x=y=z$ sunnassa. Näytä, että $\sigma(\mA)=\{0,1\}$ ja
että $\mA$:lla on kolme lineaarisesti riippumatonta ominaisvektoria. Mitkä ovat ominaisarvojen
kertaluvut? Päteekö myös $\mA^2=\mA$\,?
\vspace{1mm} \newline
c) Olkoon $\mA=\ma\ma^T$, missä $\ma\in\R^n,\ n \ge 2$ ja $\abs{\ma}=1$. Näytä, että 
$\sigma(\mA)=\{0,1\}$. Mitkä ovat ominaisarvojen geometriset kertaluvut?

\item
Matriisilla kokoa $3 \times 3$ on ominaisuus: Matriisin ominaisarvot ovat $\lambda_1=-1$, 
$\lambda_2=1$, $\lambda_3=2$ ja vastaavat ominaisvektorit ovat $[1,1,1]^T$,
$[1,1,-1]^T$ ja $[0,1,2]^T$. Montako erilaista nämä ehdot täyttävää matriisia on olemassa?
Määritä yksi!

\item
Tutki, ovatko seuraavat matriisit $\mA$ diagonalisoituvia. Myönteisessä tapauksessa laske 
matriisille jokin (reaalinen tai kompleksinen) tulohajotelma muotoa $\mA=\mC\mD\mC^{-1}$, missä
$\mD$ on diagonaalinen.
\begin{align*}
&\text{a)}\ \ \begin{rmatrix} 1&0\\4&-3 \end{rmatrix} \qquad
 \text{b)}\ \ \begin{rmatrix} 3&1\\5&-1 \end{rmatrix} \qquad
 \text{c)}\ \ \begin{rmatrix} -3&2\\-2&1 \end{rmatrix} \qquad
 \text{d)}\ \ \begin{rmatrix} 1&2\\-4&-3 \end{rmatrix} \\[2mm]
&\text{e)}\ \ \begin{rmatrix} 1&1&0\\0&2&0\\-2&5&-1 \end{rmatrix} \qquad
 \text{f)}\ \ \begin{rmatrix} -2&-1&1\\1&1&0\\0&1&0 \end{rmatrix} \qquad
 \text{g)}\ \ \begin{rmatrix} 0&2&0\\-2&2&2\\1&1&0 \end{rmatrix} \\[2mm]
&\text{h)}\ \ \begin{rmatrix} -1&-1&0\\1&-2&3\\1&-2&-6 \end{rmatrix} \qquad
 \text{i)}\ \ \begin{rmatrix} 1&0&0&9\\0&1&0&0\\0&0&1&0\\1&0&0&1 \end{rmatrix} \qquad
 \text{j)}\ \ \begin{rmatrix} 1&0&0&3\\0&1&3&0\\0&1&1&0\\1&0&0&1 \end{rmatrix}
\end{align*}

\item
a) Näytä, että similaarisuus on samaa kokoa olevien neliömatriisien välinen
ekvivalenssirelaatio. Millainen on ekvivalenssiluokka, joka sisältää yksikkömatriisin 
$\mI$\,? \vspace{1mm}\newline
b) Olkoon $\mA$ ja $\mB$ lineaarikuvauksen $A:\,\R^n\kohti\R^n$ matriisit avaruuden $\R^n$
kahdessa eri kannassa. Näytä, että $\mA$ ja $\mB$ ovat similaariset. 

\item
Määritä seuraavien matriisien ominaisarvot ja muodosta ominaisvektoreista $\R^2$:n, $\R^3$:n tai
$\R^4$:n ortonormeerattu, positiivisesti suunnistettu kanta.
\begin{align*}
&\text{a)}\ \ \begin{rmatrix} 0&2\\2&0 \end{rmatrix} \qquad
 \text{b)}\ \ \begin{rmatrix} 1&1\\2&2 \end{rmatrix} \qquad
 \text{c)}\ \ \begin{rmatrix} 5&2\\2&8 \end{rmatrix} \qquad
 \text{d)}\ \ \begin{rmatrix} 0&-2\\-2&1 \end{rmatrix} \\[2mm]
&\text{e)}\ \ \begin{rmatrix} 0&-1&-1\\-1&0&1\\-1&1&0 \end{rmatrix} \qquad
 \text{f)}\ \ \begin{rmatrix} 4&2&2\\2&1&1\\2&1&1 \end{rmatrix} \qquad
 \text{g)}\ \ \begin{rmatrix} 1&2&0\\2&1&1\\0&1&1 \end{rmatrix} \\[2mm]
&\text{h)}\ \ \begin{rmatrix} 3&1&1\\1&3&1\\1&1&3 \end{rmatrix} \qquad
 \text{i)}\ \ \begin{rmatrix} 2&0&0&2\\0&1&0&0\\0&0&1&0\\2&0&0&1 \end{rmatrix} \qquad
 \text{j)}\ \ \begin{rmatrix} 1&0&0&2\\0&1&2&0\\0&2&1&0\\2&0&0&1 \end{rmatrix}
\end{align*}

\item (*) \index{Cayleyn--Hamiltonin lause}
\kor{Cayleyn--Hamiltonin lauseen} mukaan neliömatriisi toteuttaa oman karakteristisen
yhtälönsä, ts.\ jos $p(\lambda)$ on matriisin karakteristinen polynomi, niin $p(\mA)=\mo$. \ 
a) Todista lause siinä tapauksessa, että $\mA$ on diagonalisoituva. \ b) Näytä, että lause on
tosi myös Esimerkin \ref{eig-ex4} ei-diagonalisoituvalle matriisille

\end{enumerate} %Matriisin ominaisarvot
\section{Neliömuotojen luokittelu} \label{diagonalisointi}
\alku
\index{neliömuoto|vahv}

Yleinen $\R^n$:n \kor{neliömuoto} on funktio $f(\mx)=\mx^T\mA\mx,\ \mx\in\R^n$, missä $\mA$ on
reaalinen ja symmetrinen matriisi kokoa $n\times n$ 
(ks.\ Esimerkki \ref{osittaisderivaatat}:\,\ref{neliömuoto}). Tässä luvussa tarkastellaan
neliömuodon $f$ luokittelua sen mukaan, minkä merkkisiä arvoja $f$ saa muualla kuin origossa.
Yleisessä tapauksessa luokitteluongelman ratkaisu nojaa edellisessä luvussa esitettyyn
symmetrisen matriisin ominais\-arvoteoriaan ja tähän perustuvaan matriisin diagonalisointiin
(Lauseet \ref{symmetrisen matriisin ominaisarvoteoria} ja \ref{diagonalisoituva matriisi}).
Tämä ratkaisu myös tuottaa neliömuodon \kor{pääakselikoordinaatiston}, jossa neliömuoto
(eli matriisi $\mA$) diagonalisoituu.

Jatkossa luokitellaan ensin neliömuodot ja ratkaistaan luokitteluongelma, rajoittuen
ei-triviaaleihin tapauksiin $\mA\neq\mo$ ja $n \ge 2$. Tämän jälkeen tarkastellaan kahta
neliömuotojen teorian sovellusta: $n$ muuttujan funktion kriittisen pisteen luokittelua
mahdollisena paikallisena ääriarvokohtana (Sovellus \#1) ja toisen asteen käyrien ja pintojen
geometrista luokittelua (Sovellus \#2). 

Jos $\mx\in\R^n$, $\mx\neq\mo$ ja kirjoitetaan $\me=\mx/|\mx|$, niin neliömuodolle
$f(\mx)=\mx^T\mA\mx$ pätee
\begin{equation} \label{neliömuodon skaalaus}
f(\mx)=f(|\mx|\me)=|\mx|^2\me^T\mA\me=|\mx|^2f(\me). \tag{$\star$}
\end{equation}
Tämän mukaan nelömuodon arvojen etumerkkejä tutkittaessa riittää määrittää $f$:n arvojoukko
$\R^n$:n yksikköpallolla
\[
B=\{\mx\in\R^n \ | \ \abs{\mx}=1\}.
\]
Arvojoukon $f(B)$ mukaisesti neliömuoto luokitellaan seuraavasti.
\begin{Def} \label{neliömuodon luokittelu} \index{neliömuoto!a@luokittelu|emph}
\index{positiividefiniittisyys!c@neliömuodon|emph}
\index{negatiivisesti!b@definiitti (neliömuoto)|emph}
\index{indefiniitti (neliömuoto)|emph}
\index{semidefiniitti (neliömuoto)|emph}
\index{puolidefiniitti (neliömuoto)|emph}
Neliömuoto $f(\mx)=\mx^T\mA\mx$ on
\begin{itemize}
\item[-] \kor{positiivisesti definiitti}, jos $f(\mx)>0 \quad \forall \mx\in B$,
\item[-] \kor{negatiivisesti definiitti}, jos $f(\mx)<0 \quad \forall \mx\in B$,
\item[-] \kor{positiivisisesti semidefiniitti} (puolidefiniitti), jos 
         $f(\mx)\geq 0 \quad \forall \mx\in B$ ja $f(\mx)=0$ jollakin $\mx\in B$,
\item[-] \kor{negatiivisisesti semidefiniitti}, jos $f(\mx)\leq 0 \quad \forall \mx\in B$ ja
         $f(\mx)=0$ jollakin $\mx\in B$,
\item[-] \kor{indefiniitti} (ei-definiitti), jos $f(\mx)> 0$ ja $f(\my)<0$ joillakin 
         $\mx,\my\in B$.
\end{itemize}\end{Def} 
Yksinkertaisimmassa eli kahden muuttujan tapauksessa neliömuodon definiitisyysongelma
ratkeaa helposti. Kun merkitään $\mx=(x,y)$ ja $\mA=(a_{ij})$, niin yleinen kahden muuttujan
neliömuoto on funktio
\[
f(x,y)=a_{11}x^2+2a_{12}xy+a_{22}y^2,\quad a_{ij}\in\R.
\]
Definiittisyyslaji saadaan määrätyksi esimerkiksi laskemalla suoraan $f$:n arvojoukko 
yksikköympyrällä $B=\{(x,y)\in\R^2 \ | \ x^2+y^2=1\}$. Tämä käy parametrisoimalla $B$ muodossa
$x=\cos t,\ y=\sin t,\ t\in [0,2\pi]$, jolloin tutkimuskohteeksi tulee funktio
\[
f(\cos t,\sin t)=F(t)=a_{11}\cos^2 t +2a_{12}\cos t\sin t+a_{22}\sin^2 t.
\]
$F$:n arvojouko $[F_{\text{min}},F_{\text{max}}]$ välillä $[0,2\pi]$ määrää $f$:n definiittisyyden
lajin.

Toinen laskutapa on määrätä $f$:n nollakohdat origon kautta kulkevilla suorilla
$S_t:\ y=tx,\ t\in\R$ ja (erikseen) suoralla $S_y:\ x=0\,$:
\begin{alignat*}{2}
y = tx\ &:\quad &f(x,tx)&=(a_{11}+2a_{12}t+a_{22}t^2)x^2=F(t)x^2 \\
x = 0 \ &:\quad &f(0,y) &= a_{22}y^2.
\end{alignat*}
Sen mukaan, onko $f(x,y)=0$ vain origossa, pitkin yhtä suoraa, vai pitkin kahta origossa
leikkaavaa suoraa, on pääteltävissä, että $f$ on vastaavasti definiitti, semidefiniitti tai
indefiniitti.
\begin{Exa} Luokittele neliömuoto $f(x,y)=x^2+axy+4y^2$, kun \newline
a) $a=3$, \ b) $a=4$, \ c) $a=5$.
\end{Exa}
\ratk
\begin{align*}
\text{a)}\quad f(\cos t,\sin t) 
         &= \cos^2 t + 3\cos t\sin t + \sin^2 t \\
         &= (\cos t+\tfrac{3}{2}\sin t)^2 + \tfrac{1}{4}\sin^2 t >0\quad\forall t\in [0,2\pi] \\
         &\impl \ f \text{ on positiivisesti definiitti}. \\[2mm]
\text{b)}\quad\qquad\,\  f(x,y) 
         &=(x+2y)^2 \\
         &\impl \ f \text{ on positiivisesti semidefiniitti}. \\[2mm]
\text{c)}\quad\qquad  f(0,y)\,\ 
         &= 4y^2>0,\text{ kun } y\neq 0, \\
 f(x,tx) &=(1+5t+4t^2)x^2=0,\,\ \text{kun}\,\ t=-1\,\ \text{tai}\,\ t=-1/4 \\
         &\impl \ f \text{ on indefiniitti}. \loppu
\end{align*}

\subsection{Neliömuodon diagonalisointi}
\index{neliömuoto!b@diagonalisointi|vahv}
\index{diagonalisointi!b@neliömuodon|vahv}

Useamman kuin kahden muuttujan tapauksessa nelömuodon luokitteluongelma ratkeaa luontevimmin
symmetrisen matriisin ominaisarvoteorian avulla: Suoritetaan matriisin $\mA$ diagonalisaatio
\[
\mD=\mC^T\mA\mC \ \ekv \ \mA=\mC\mD\mC^T,
\]
missä $\mD=\text{diag}\,(\lambda_i)$ koostuu $\mA$:n ominaisarvoista ja $\mC$ 
ortonormeeratuista ominaisvektoreista. Tällöin $f(\mx)$ saa muodon
\[
f(\mx)=\mx^T\mC\mD\mC^T\mx=(\mC^T\mx)^T\mD(\mC^T\mx).
\]
Siis jos tehdään $\R^n$:ssä muuttujan vaihto
\[
\my=\mC^T\mx\,\ \ekv\,\ \mx=\mC\my,
\]
niin $f$ saa muodon
\[
f(\mx)=g(\my)=\my^T\mD\my=\sum_{i=1}^n\lambda_iy_i^2.
\]
Neliömuotoa $g(\my)$ sanotaan --- varsin hyvällä syyllä --- $f$:n 
\kor{diagonalisoiduksi muodoksi}. Huomattakoon, että $\mC$:n ollessa ortogonaalinen säilyttää
muunnos $\mx\map\my=\mC^T\mx$ vektorin pituuden:
\[
\abs{\my}^2=\my^T\my=\mx^T\mC\mC^T\mx=\mx^T\mx=\abs{\mx}^2.
\]
Määritelmän \ref{neliömuodon luokittelu} mukaisesti neliömuodon luokitteluongelma ratkeaa näin
$\mA$:n ominaisarvojen avulla seuraavasti:
\vspace{2mm}
\[
\boxed{
\begin{alignedat}{2}
\quad&f(\mx)=\mx^T\mA\mx \ : && \rule{0mm}{7mm} \\ \\
&f\text{ positiivisesti definiitti }&&\ekv \ \lambda>0\quad\forall \lambda\in\sigma(\mA). \\ \\
&f\text{ negatiivisesti definiitti } &&\ekv \ \lambda<0\quad\forall \lambda\in\sigma(\mA).\\ \\
&f\text{ positiivisesti semidefiniitti } &
  &\ekv \ \lambda\geq 0\quad\forall \lambda\in\sigma(\mA)\text{ ja } 0\in\sigma(\mA). \\ \\
&f\text{ negatiivisesti semidefiniitti } &
  &\ekv \ \lambda\leq 0\quad\forall \lambda\in\sigma(\mA)\text{ ja } 0\in\sigma(\mA). \\ \\
&f\text{ indefiniitti } &
  &\ekv \ \lambda,\mu\in\sigma(\mA)\text{ joillakin } \lambda>0\text{ ja }\mu<0. \quad
                                              \rule[-5mm]{0mm}{2mm}
\end{alignedat}}
\]
Kun huomioidaan myös skaalaustulos \eqref{neliömuodon skaalaus}, niin em.\ luokittelujen
nojalla pätee erityisesti
\begin{itemize}
\item[-] $f$ positiivisesti definiitti $\,\qimpl f(\mx) \ge \lambda|\mx|^2\ \forall \mx\in\R^n$,
\item[-] $f$ negatiivisesti definiitti $\qimpl f(\mx) \le -\lambda|\mx|^2\ \forall \mx\in\R^n$,
\end{itemize}
missä $\lambda>0$ on pienin luvuista $|\lambda_i|$, $\lambda_i\in\sigma(\mA)$.
\jatko \begin{Exa} (jatko) Esimerkissä on
\[
f(x,y) = \mx^T\mA\mx = \begin{bmatrix} x&y \end{bmatrix}
                       \begin{rmatrix} 1&\frac{a}{2}\\\frac{a}{2}&4 \end{rmatrix}
                       \begin{bmatrix} x\\y \end{bmatrix}.
\]
Esimerkin luokitteluihin tapauksissa a) $a=3$, \ b) $a=4$, \ c) $a=5$ päädytään myös
laskemalla
\[
\text{a)}\,\ \sigma(\mA)=\{\frac{1}{2}(5 \pm 3\sqrt{2})\}, \quad
\text{b)}\,\ \sigma(\mA)=\{0,5\}, \quad
\text{c)}\,\ \sigma(\mA)=\{\frac{1}{2}(5 \pm \sqrt{34}\}. \loppu
\]
\end{Exa} 
\begin{Exa} Luokittele $\R^3$:n neliömuoto $\,f(x,y,z)=x^2+3y^2+z^2+4xz$.
\end{Exa}
\ratk
\begin{align*}
f(x,y,z) &= \begin{bmatrix} x & y & z \end{bmatrix} \begin{rmatrix}
1 & 0 & 2 \\ 0 & 3 & 0 \\ 2 & 0 & 1 \end{rmatrix}\begin{bmatrix} x \\ y \\ z \end{bmatrix} \\
&= \mx^T\mA\mx,\quad \mx=\begin{bmatrix} x & y & z \end{bmatrix}^T.
\end{align*}
Tässä on $\sigma(2\mA)=\{-2,6\}$ (edellisen luvun Esimerkki \ref{eig-ex1}), joten
$\sigma(\mA)=\{-1,3\}$ ja $f$ on siis indefiniitti. \loppu

\subsection{Neliömuodon pääakselikoordinaatisto}
\index{neliömuoto!c@pääakselikoordinaatisto|vahv}
\index{pzyzy@pääakselikoordinaatisto|vahv}

Neliömuodon $\mx^T\mA\mx$ definiittisyyslaji voidaan päätellä pelkästään $\mA$:n
ominais\-arvojen avulla, mutta kiinnostava on myös tieto, millaisessa koordinaatistossa
neliömuoto diagonalisoituu. Tämän tiedon välittävät $\mA$:n ortonormeeratut ominaisvektorit,
jotka määräävät diagonalisoivan koordinaattimuunnoksen $\my=\mC^T\mx$. Tämän mukaisesti
vektorin $\my$ alkiot ovat vektorin $\mx$ koordinaatit kannassa $\{\mc_1, \ldots, \mc_n\}$,
jonka muodostavat $\mC$:n sarakkeet. Jatkossa oletetaan, että kanta $\{\mc_1, \ldots, \mc_n\}$
on myös p\pain{ositiivisesti} \pain{suunnistettu} (tarvittaessa yhden vektorin suunnan vaihto
vastakkaiseksi), jolloin on $\det\mC=1$ ja kannan vaihdossa $\{\me_i\} \ext \{\mc_i\}$ on kyse
\pain{koordinaatiston} \pain{kierrosta} (ks.\ Luku \ref{lineaarikuvaukset}).

Kierrettyä koordinaatistoa, jossa neliömuoto $f(\mx)=\mx^T\mA\mx$ diagonalisoituu, sanotaan 
ko.\ neliömuodon \kor{pääakselikoordinaatistoksi}. Kyseisen koordinaatiston kantana toimivat
siis $\mA$:n ortonormeeratut ja positiivisesti suunnistetut ominaisvektorit.
\jatko \begin{Exa} (jatko) Esimerkissä pääakselikoordinaatiston (ortonormeeratut ja
positiivisesti suunnistetut) kantavektorit ovat
\[
\mc_1=\frac{1}{\sqrt{2}}\begin{bmatrix} 1&0&1 \end{bmatrix}^T, \quad
\mc_2=\begin{bmatrix} 0&1&0 \end{bmatrix}^T, \quad
\mc_3=\frac{1}{\sqrt{2}}\begin{bmatrix} -1&0&1 \end{bmatrix}^T
\]
vastaten $\mA$:n ominaisarvoja $\lambda_1=\lambda_2=3$ ja $\lambda_3=-1$ (vrt.\ edellinen luku).
Kun muunnettuja koordinaatteja merkitään $\my=(\xi,\eta,\zeta)$, niin koordinaatiston
kiertomuunnos on
\[
\my=\mC^T\mx \qekv \begin{bmatrix} \xi \\ \eta \\ \zeta \end{bmatrix} =
                   \begin{rmatrix} \frac{1}{\sqrt{2}}&0&\frac{1}{\sqrt{2}} \\ 0&1&0 \\
                   -\frac{1}{\sqrt{2}}&0&\frac{1}{\sqrt{2}} \end{rmatrix}
                   \begin{bmatrix} x\\y\\z \end{bmatrix},
\]
ja tässä pääakselikoordinaatistossa on siis
\[
f(x,y,z)=g(\xi,\eta,\zeta)=3\xi^2+3\eta^2-\zeta^2. \loppu
\]
\end{Exa}

\subsection{Sovellus 1: Funktion kriittisen pisteen luokittelu}
\index{kriittinen piste!a@luokittelu|vahv}

Funktion $f(\mx)$ kriittisen pisteen luokittelu mahdollisena paikallisena ääriarvokohtana
(vrt.\ Propositio \ref{ääriarvopropositio-Rn}) on kysymys, jote ei toistaiseksi ole tarkastelu.
Palataan nyt tähän kysymykseen neliömuotijen teorian valossa. Olkoon $\mc$ funktion $f$
kriittinen piste, ts.\ $\Nabla f(\mc)=\mo$. Tällöin jos $f$ on $\mc$:n ympäristössä riittävän
säännöllinen, niin usean muuttujan Taylorin kaavan (ks.\ Luku
\ref{usean muuttujan taylorin polynomit}) perusteella pätee
\[
f(\mx) = f(\mc)+\tfrac{1}{2}(\mx-\mc)^T\mH(\mc)(\mx-\mc)+\ord{\abs{\mx-\mc}^2},
\]
missä $\mH(\mc)$ on funktion Hessen matriisi pisteessä $\mc\,$:
\[
[\,\mH(\mc)\,]_{ij}=\frac{\partial^2 f}{\partial x_i\partial x_j}(\mc),\quad i,j=1\ldots n.
\]
Sen seikan selvittämiseksi, onko $\mc$ $f$:n paikallinen ääriarvokohta vai ei, on siis
ilmeisesti ensisijaisesti tutkittava, millaisia arvoja funktio
\[
g(\mx)=\frac{1}{2}(\mx-\mc)^T\mH(\mc)(\mx-\mc)
\]
saa pisteen $\mc$ lähiympäristössä. Kun siirretään origo pisteeseen $\mc$ 
koordinaattimuunnoksella $\mx'=\mx-\mc$, niin tutkimuskohteeksi tulee $\R^n$:n neliömuoto
$h(\mx')=g(\mx'+\mc) = (\mx')^T\mA\mx'$, $\mA=\frac{1}{2}\mH(\mc)$.

Olkoon neliömuoto $h(\mx')$ positiivisesti definiitti. Tällöin pätee mainituin oletuksin jollakin
$\delta>0$
\[
f(\mx) \ge f(\mc)+\lambda\abs{\mx-\mc}^2-c_\delta\abs{\mx-\mc}^2, \quad
                               \text{kun}\,\ 0<\abs{\mx-\mc}<\delta,
\]
missä $\lambda>0$ ja $c_\delta\kohti 0$, kun $\delta\kohti 0$. Siis jos valitaan $\delta>0$ 
siten, että $c_\delta\leq \lambda/2$, niin ko.\ $\delta$:n arvolla pätee
\[
f(\mx) \ge f(\mc)+\frac{1}{2}\lambda\abs{\mx-\mc}^2, \quad 0<\abs{\mx-\mc}<\delta.
\]
Näin ollen $\mc$:ssä on $f$:n paikallinen minimi. Vastaavalla tavalla päätellään, että $\mc$
on $f$:n paikallinen maksimipiste, jos $h(\mx')$ on negatiivisesti definiitti, ja että $\mc$ ei ole 
$f$:n paikallinen ääriarvo- eikä laakapiste, jos $h(\mx')$ on indefiniitti. Päätelmät
yhdistettyinä ovat siis seuraavat.
\begin{Prop} \label{ääriarvokriteerit Rn:ssä}
Jos funktiolle $f:A\kohti\R,\ A\subset\R^n$, on pisteen $\mc\in A$ jossakin ympäristössä
voimassa approksimaatio
\[
f(\mx)=f(\mc)+g(\mx-\mc)+\ord{\abs{\mx-\mc}^2}, \quad \mx \in U_\delta(\ma) \subset A,
\]
missä $g$ on neliömuoto, niin pätee
\begin{enumerate}
\item $g$ positiivisesti definiitti $\ \, \impl \ \mc$ on $f$:n paikallinen minimipiste.
\item $g$ negatiivisesti definiitti $\ \impl \ \mc$ on $f$:n paikallinen maksimipiste.
\item $g$ indefiniitti $\impl \ \mc$ ei ole $f$:n paikallinen ääriarvo- eikä laakapiste.
\end{enumerate}
\end{Prop}
Huomattakoon, että neliömuodon indefiniittisyys on mahdollinen vain kahden tai useamman 
muuttujan tilanteessa, sillä jos $n=1$, niin $\mx^T\mA\mx=Ax^2$, joten neliömuoto on joko
positiivisesti definiitti ($A>0$), negatiivisesti definiitti ($A<0$) tai $=0$ ($A=0$). Ehto
$f''(c)>0$ on yhden muuttujan analyysista tuttu paikallisen minimin (riittävä) ehto. Tämän
vastine $n$ muuttujan tilanteessa on siis ehto
\[
\mx^T\mA\mx>0\quad\forall\mx\in\R^n,\ \mx\neq\mo\ ;\quad 
            \mA=\left(\frac{\partial^2 f}{\partial x_i\partial x_j}(\mc)\right),
\]
eli toisen derivaatan $f''(c)$ vastine $n$ muuttujan ääriarvotarkastelussa on kaikkien toisen
kertaluvun osittaisderivaattojen muodostama Hessen matriisi.
\begin{Exa}
Tutki, onko seuraavilla funktioilla paikallista ääriarvoa origossa: \vspace{1mm}\newline
a) \ $f(x,y)=\cos x+2y-2e^y,\,\ $ b) \ $f(x,y)=e^{-x^2}e^{-y^2}+3\sin xy$.
\end{Exa}
\ratk \ a) $f(x,y)=-1+(-\frac{1}{2}x^2-y^2)+\ldots$ \ Neliömuoto-osa on negatiivisesti
definiitti. Siis $f$:llä on origossa paikallinen maksimi. \vspace{1mm}\newline
b) $\displaystyle{f(x,y)=1+(-x^2-y^2+3xy)+\ldots}$ \ Neliömuoto-osa on indefiniitti. Siis
origo ei ole $f$:n paikallinen ääriarvopiste eikä myöskään laakapiste. \loppu
\begin{Exa}
Selvitä funktion $\,f(x,y)=x^2+2y^2+2xy+4x+14y\,$ kriittisen pisteen $(3,-5)$ laatu 
(vrt.\ Luvun \ref{usean muuttujan ääriarvotehtävät} Esimerkki \ref{saari uudelleen}).
\end{Exa}
\ratk Siirretään kriittinen piste ensin origoon:
\begin{align*}
&x=3+x',\ y=-5+y' \\
&g(x',y')=f(3+x',-5+y') = (x')^2+2x'y'+2(y')^2-29.
\end{align*}
Koska $g$:n nelömuoto-osa on positiivisesti definiitti, on $f$:llä on pisteessä $(3,-5)$ 
paikallinen minimi. Funktion muodosta nähdään, että tämä on myös absoluuttinen minimi. \loppu

Propositio \ref{ääriarvokriteerit Rn:ssä} jättää avoimeksi tapauksen, jossa neliömuoto $g(\mx')$
on puolidefiniitti. Kuten tapauksessa $n=1$ ($f''(c)=0$), on mahdollisen ääriarvokohdan laatu 
ratkaistava tapauskohtaisesti.
\begin{Exa}
Määrittele kriittisen pisteen $(0,0)$ laatu, kun
\[
f(x,y)=x^2+2xy+y^2+a(x+y)^3+bx^n,\quad a,b\in\R,\quad n\in\{3,4,\ldots\}.
\]
\end{Exa}
\ratk Koska $f$:n neliömuoto-osa
\[
g(x,y)=x^2+2xy+y^2=(x+y)^2
\]
on positiivisesti puolidefiniiti, on tutkimuksia jatkettava ainakin suoralla $y=-x$\,:
\[
f(x,-x)=bx^n.
\]
Päätellään: Jos $b=0$, on origo laakapiste. Jos $b>0$ ja $n$ on parillinen, on origo $f$:n 
paikallinen minimipiste. Muulloin, eli jos $b<0$ ja $n$ on parillinen, tai jos $b \neq 0$ ja
$n$ on pariton, saa $f$ origon jokaisessa ympäristössä sekä positiivisia että negatiivisia
arvoja, joten origo ei ole paikallinen ääriarvo- eikä laakapiste. \loppu

\subsection{Sovellus 2: Toisen asteen käyrät ja pinnat}
\index{toisen asteen käyrä|vahv} \index{toisen asteen pinta|vahv}

Palautettakoon mieliin Luvusta \ref{suorat ja tasot}, että yleisen \pain{toisen} \pain{asteen}
\pain{kä}y\pain{rän} yhtälö tasossa on muotoa
\[
Ax^2+By^2+2Cxy+Dx+Ey+F=0,
\]
missä $A,\ldots,F\in\R$ ja ainakin yksi luvuista $A,B,C$ on nollasta poikkeava, ja yleisen 
\pain{toisen} \pain{asteen} p\pain{innan} yhtälö avaruudessa on muotoa
\[
Ax^2+By^2+Cz^2+2Dxy+2Eyz+2Fxz+Gx+Hy+Iz+J=0,
\]
missä $A,\ldots,J\in\R$ ja ainakin yksi luvuista $A,\ldots,F$ on nollasta poikkeava. Jos 
merkitään
\begin{align*}
&\text{Taso:}\qquad\,\ \mA=\begin{bmatrix} A&C\\C&B \end{bmatrix}, \quad 
                       \mx=\begin{bmatrix} x\\y \end{bmatrix}, \quad
                       \ma=\begin{bmatrix} D\\E \end{bmatrix}, \quad c=F, \\[3mm]
&\text{Avaruus:} \quad \mA=\begin{bmatrix} A&D&F\\D&B&E\\F&E&C \end{bmatrix}, \quad 
                       \mx=\begin{bmatrix} x\\y\\z \end{bmatrix}, \quad 
                       \ma=\begin{bmatrix} G\\H\\I \end{bmatrix}, \quad c=J,
\end{align*}
niin toisen asteen käyrän ja pinnan yhtälöt saadaan muotoon
\[
\mx^T\mA\mx+\ma^T\mx+c=0.
\]
Suoritetaan tässä koordinaattimuunnos
\[
\mx=\mC\my+\mb,
\]
missä $\my=[\xi,\eta]^T$ (taso) tai $\my=[\xi,\eta,\zeta]^T$ (avaruus), ja $\mC$ ja $\mb$ 
valitaan niin, että yhtälö saa muunnetuissa koordinaateissa mahdollisimman yksinkertaisen 
muodon. Ensinnäkin valitaan $\mC$ niin, että neliömuoto $\mx^T\mA\mx$ diagonalisoituu, eli 
valitaan $\mC$:n sarakkeiksi $\mA$:n ortonormeeratut ja positiivisesti suunnistetut 
ominaisvektorit. Tällöin yhtälö saadaan muotoon
\[
\my^T\mD\my+\md^T\my+d=0
\]
missä $\mD$ on diagonaalinen matriisi (diagonaalilla $\mA$:n ominaisarvot) ja 
\[
\md=\mC^T(2\mA\mb+\ma), \quad d=\mb^T(\mA\mb+\ma)+c.
\]
Jos $\mA$ on säännöllinen matriisi, niin määrätään $\mb$ siten, että $\md=\mo$, muussa 
tapauksessa saatetaan $\mb$:n valinnalla termi $\md^T\my+d$ mahdollisimman yksinkertaiseen
muotoon. (Jos kyseessä on avaruuden pinta ja $\lambda=0$ on $\mA$:n kaksinkertainen ominaisarvo,
voi yhtälön yksinkertaistumiseen vaikuttaa myös $\mC$:n valinnalla, ks.\ 
Harj.teht.\,\ref{H-eig-3: pintojen erikoisluokka}.) Yhtälön näin pelkistyttyä voidaan sen
määräämä käyrä tai pinta luokitella geometrisesti.

Jos kyseessä on tasokäyrä, niin em.\ muunnokset johtavat yhteen seuraavista perusmuodoista:
\begin{itemize}
\item[1.] $\displaystyle{\quad \frac{\xi^2}{a^2}+\frac{\eta^2}{b^2}=q, \quad 
                                            q\in\{1,0,-1\}, \quad a,b>0.}$ \\[2mm]
\item[2.] $\displaystyle{\quad \frac{\xi^2}{a^2}-\frac{\eta^2}{b^2}=q, \quad 
                                            q\in\{1,0,-1\}, \quad a,b>0.}$ \\[2mm]
\item[3.] $\displaystyle{\quad \eta=a\xi^2 \quad \text{tai} \quad \xi=a\eta^2, \quad 
                                            a\in\R,\ a\neq 0.}$ \\[2mm]
\item[4.] $\displaystyle{\quad \xi^2=a \quad\,\ \text{tai} \quad \eta^2=a, \quad\,\ a\in\R.}$
\end{itemize}
\index{ellipsi} \index{hyperbeli} \index{paraabeli}%
Tapauksessa 1 käyrä on \kor{ellipsi}, jos $q=1$, muulloin kyseessä on piste ($q=0$) tai tyhjä
joukko ($q=-1$). Tapauksessa 2 käyrä on \kor{hyperbeli}, jos $q=\pm 1$, muuten kysessä on kaksi
toisensa leikkaavaa suoraa ($q=0$). Tapauksessa 3 käyrä on \kor{paraabeli}, ja tapauksessa 4 on
kyseessä joko kaksi yhdensuuntaista suoraa ($a>0$), yksi suora ($a=0$), tai tyhjä joukko 
($a<0$). Tapaukseen 1 päädytään kun $\mA$ on definiitti (ominaisarvot samanmerkkiset),
tapaukseen 2 kun $\mA$ on indefiniitti (ominaisarvot erimerkkiset) ja tapauksiin 3 ja 4 kun
$\mA$ on puolidefiniitti (yksi ominaisarvo $=0$).
\begin{Exa}
Etsi sellainen koordinaatisto, jossa funktion
\[
f(x,y)=13x^2-8xy+7y^2-42x+36y
\]
tasa-arvokäyrät saavat toisen asteen käyrän perusmuodon. Kuvio!
\end{Exa}
\ratk Kirjoitetaan tasa-arvokäyrän $S:\ f(x,y)=c\,$ yhtälö ensin muotoon
\[
\begin{bmatrix} x&y \end{bmatrix}
\begin{rmatrix} 13 & -4 \\ -4 & 7 \end{rmatrix}
\begin{bmatrix} x\\y \end{bmatrix}
+ \begin{bmatrix} -42&36 \end{bmatrix} \begin{bmatrix} x\\y \end{bmatrix} - c = 0.
\]
Matriisin $\ \displaystyle{\mA=\begin{rmatrix} 13 & -4 \\ -4 & 7 \end{rmatrix}}\ $ ominaisarvot
ja ortonormeeratut, positiivisesti suunnistetut ominaisvektroit ovat
\[
\lambda_1=15,\,\ \mc_1=\frac{1}{\sqrt{5}}\begin{bmatrix} 2 \\ -1 \end{bmatrix},\quad
\lambda_2= 5,\,\ \mc_2=\frac{1}{\sqrt{5}}\begin{bmatrix} 1 \\ 2 \end{bmatrix}.
\]
Suoritetaan koordinaattimuunnos kahdessa vaiheessa. Ensin kierto neliömuodon
pääakselikoordinaatistoon $(x',y')$ (edellä $\mb=\mo$)\,: 
\[
\begin{bmatrix} x\\y \end{bmatrix} =
\frac{1}{\sqrt{5}}\begin{rmatrix} 2&1\\-1&2 \end{rmatrix}
\begin{bmatrix} x'\\y' \end{bmatrix} \qekv
\begin{bmatrix} x'\\y' \end{bmatrix} =
\frac{1}{\sqrt{5}}\begin{rmatrix} 2&-1\\1&2 \end{rmatrix}
\begin{bmatrix} x\\y \end{bmatrix}.
\]
Tulos:
\begin{align*}
f(x,y)=c &\qekv 15(x')^2+5(y')^2-\frac{120}{\sqrt{5}}\,x'+\frac{30}{\sqrt{5}}\,y'-c=0 \\
         &\qekv 15\left(x'-\frac{4}{\sqrt{5}}\right)^2
                +5\left(y'+\frac{3}{\sqrt{5}}\right)^2-57-c=0.
\end{align*}
Toinen vaihe on origon siirto, koordinaateiksi
\begin{align*}
\xi &= x'-\frac{4}{\sqrt{5}}\,, \quad \eta=y'+\frac{3}{\sqrt{5}}\,. \\[2mm]
\text{Lopputulos:} \qquad f(x,y) = c \quad
    &\ekv\quad 3\xi^2+\eta^2=\frac{1}{5}(57+c). \hspace{4.2cm}
\end{align*}
Tämän mukaan tasa-arvokäyrä on joko ellipsi ($c>-57$), piste ($c=-57$) tai tyhjä joukko
($c<-57$). Ellipsien
yhteisiä pääakseleita ovat suorat $\xi=0$ ja $\eta=0$, ja niiden yhteisessä keskipisteessä
\[
(\xi,\eta)=(0,0)\,\ \ekv\,\ (x',y')=(4/\sqrt{5},-3/\sqrt{5})\,\ \ekv\,\ (x,y)=(1,-2)
\]
$f$ saavuttaa absoluuttisen minimiarvonsa $f_{\text{min}}=-57$. \loppu
\vspace{4mm}
\input{plots/koordkierto.tex}

Toisen asteen pinnat voidaan em.\ koordinaattimuunnoksen antaman perusmuodon avulla luokitella
seuraaviin yhdeksään päätyyppiin (yhtälöissä voi koordinaattien järjestystä vaihdella).
\index{ellipsoidi} \index{hyperboloidi} \index{paraboloidi}
\index{lieriö} \index{kartio}
\index{yksivaippainen hyperboloidi} \index{kaksivaippainen hyperboloidi}
\index{elliptinen lieriö} \index{hyperbolinen lieriö} \index{parabolinen lieriö}
\index{elliptinen paraboloidi} \index{hyperbolinen paraboloidi}%
\begin{itemize}
\item[1.] \kor{Ellipsoidi:} \hspace{38mm}
          $\displaystyle{\frac{\xi^2}{a^2}+\frac{\eta^2}{b^2}+\frac{\zeta^2}{c^2}=1.}$
\item[2.] \kor{Yksivaippainen hyperboloidi:} 
          $\displaystyle{\,\ \quad\frac{\xi^2}{a^2}+\frac{\eta^2}{b^2}-\frac{\zeta^2}{c^2}=1.}$
\item[3.] \kor{Kaksivaippainen hyperboloidi:} 
          $\displaystyle{\quad\,\frac{\xi^2}{a^2}+\frac{\eta^2}{b^2}-\frac{\zeta^2}{c^2}=-1.}$
\item[4.] \kor{Kartio:} \hspace{43mm}
          $\displaystyle{\frac{\xi^2}{a^2}+\frac{\eta^2}{b^2}-\frac{\zeta^2}{c^2}=0.}$
\item[5.] \kor{Elliptinen paraboloidi:} \hspace{16mm}\, 
          $\displaystyle\zeta={\frac{\xi^2}{a^2}+\frac{\eta^2}{b^2}}\,.$
\item[6.] \kor{Hyperbolinen paraboloidi:} \hspace{11mm} 
          $\displaystyle{\zeta=\frac{\xi^2}{a^2}-\frac{\eta^2}{b^2}}\,.$
\item[7.] \kor{Elliptinen lieriö:} \hspace{26mm} 
          $\displaystyle{\frac{\xi^2}{a^2}+\frac{\eta^2}{b^2}=1.}$
\item[8.] \kor{Hyperbolinen lieriö:} \hspace{20mm} 
          $\displaystyle{\frac{\xi^2}{a^2}-\frac{\eta^2}{b^2}=1.}$
\item[9.] \kor{Parabolinen lieriö:} \hspace{22mm} 
          $\displaystyle{\kehys\eta=a\xi^2. \phantom{\frac{\xi^2}{a^2}}}$
\end{itemize}
\vspace{3mm}
\index{satulapinta}%
Hyperbolinen paraboloidi on toiselta nimeltään \kor{satulapinta}.
Tapauksissa 1--4 $\mA$ on säännöllinen matriisi, ja tapauksessa 1 ominaisarvot ovat lisäksi
samanmerkkiset. Tapauksissa 5--8 on $\mA$:lla yksinkertainen ja tapauksessa 9 kaksinkertainen
ominaisarvo $\lambda=0$.
Näiden päätyyppien lisäksi toisen asteen pinnan yhtälö voi esittää kahta toisensa leikkaavaa
tasoa, kahta yhdensuuntaista tasoa, yhtä tasoa, avaruussuoraa, pistettä, tai ei mitään.
\begin{Exa} Luokittele eri $c$:n arvoilla toisen asteen pinta
\[
5x^2+2y^2+2z^2+2xy-4yz-2xz+10x+2y-2z+c=0.
\]
\end{Exa}
\ratk Matriisin
\[
\mA=\begin{rmatrix} 5&1&-1\\1&2&-2\\-1&-2&2 \end{rmatrix}
\]
karakteristinen yhtälö on
\[
\begin{vmatrix} 5-\lambda &1&-1\\1&2-\lambda &-2\\-1&-2&2-\lambda \end{vmatrix} 
           = -\lambda^3+9\lambda^2-18\lambda=0.
\]
Ominaisarvot ovat $\lambda_1=0$, $\lambda_2=3$ ja $\lambda_3=6$. Näitä vastaavat, 
ortonormeeratut ja positiivisesti suunnistetut ominaisvektorit ovat matriisin $\mC$ sarakkeiksi
koottuna:
\[
\mC= \begin{rmatrix} 
      0 & \frac{1}{\sqrt{3}} & \frac{2}{\sqrt{6}} \\[1mm]
      \frac{1}{\sqrt{2}} & -\frac{1}{\sqrt{3}} & \frac{1}{\sqrt{6}} \\[1mm]
      \frac{1}{\sqrt{2}} & \frac{1}{\sqrt{3}} & -\frac{1}{\sqrt{6}}
     \end{rmatrix}.
\]
Tehdään ensin koordinaatiston kiertomuunnos $\mC\mx'=\mx\ \ekv\ \mx'=\mC^T\mx$, missä 
$\mx=[x,y,z]^T,\ \mx'=[x',y',z']^T$, jolloin pinnan yhtälö saadaan muotoon
\begin{align*}
0\,&=\,3(y')^2+6(z')^2+\frac{5}{\sqrt{3}}\,y'+\frac{22}{\sqrt{6}}\,z'+c \\
   &=\,3\left(y'+\frac{1}{\sqrt{3}}\right)^2+6\left(z'+\frac{2}{\sqrt{6}}\right)^2+c-5.
\end{align*}
Tästä päästään perusmuotoon origon siirrolla:
\[
\xi=x', \quad \eta=y'+\frac{1}{\sqrt{3}}\,, \quad 
\zeta=z'+\frac{2}{\sqrt{6}} \qimpl 3\eta^2+6\zeta^2=5-c.
\]
Tästä nähdään, että jos $c<5$, niin pinta on \pain{elli}p\pain{tinen} \pain{lieriö}. Lieriön
symmetria-akseli on suora $S:\,\eta=\zeta=0$. Suora $S$ kulkee pisteen 
$\,(\xi,\eta,\zeta)=(0,0,0)$ eli pisteen $(x,y,z)=(-1,0,0)$ kautta ja sen suuntavektori 
alkuperäisessä koordinaatistossa on $\mt=[0,1,1]^T\vastaa\vec j+\vec k$ (ominaisarvoa 
$\lambda_1=0$ vastaava $\mA$:n ominaisvektori, joka määrää $\xi$-akselin suunnan). Tapauksessa
$c=5$ yhtälö määrittelee vain suoran $S$ ja tapauksessa $c>5$ ei mitään. --- Suora $S$ on itse
asiassa tasojen $\,T_1\,:\ x-y+z+1=0\,$ ja $\,T_2\,:\ 2x+y-z+2=0\,$ leikkaussuora, sillä 
alkuperäinen yhtälö on sama kuin
\[
(x-y+z+1)^2+(2x+y-z+2)^2=5-c. \loppu
\]

\pagebreak

\Harj
\begin{enumerate}

\item
Luokittele seuraavat neliömuodot määrittämällä a)-kohdassa arvojoukko yksiköympyrällä ja
b)- ja c)-kohdissa sijoituksella $y=tx$. \vspace{1mm}\newline
a)\,\ $3x^2-16xy+16y^2 \qquad$
b)\,\ $4x^2-16xy+16y^2 \qquad$
c)\,\ $5x^2-16xy+16y^2$

\item
Määritä nelömuodon pääakselikoordinaatisto ja diagonalisoitu muoto: \vspace{1mm}\newline
a)\,\ $f(x,y)=xy \qquad$
b)\,\ $f(x,y)=4x^2-16xy+16y^2$ \newline
c)\,\ $f(x,y)=-3x^2-3y^2+4xy \qquad$
d)\,\ $f(x,y,z)=2x^2+3y^2+2z^2-6xz$ \newline
e)\,\ $f(x,y,z,u)=x^2+y^2+z^2+u^2+2yz+2xu$

\item
Määritä neliömuoto $f(x,y,z)$, jonka pääakselikoordinaatiston kantavektorit ovat
\[
\mc_1=\frac{1}{7} \begin{rmatrix} 3\\-2\\6 \end{rmatrix}, \quad
\mc_2=\frac{1}{7} \begin{rmatrix} -2\\6\\3 \end{rmatrix}, \quad
\mc_3=\frac{1}{7} \begin{rmatrix} -6\\-3\\2 \end{rmatrix}
\]
ja jonka diagonalisoitu muoto pääakselikoordinaatistossa on
\[
g(\xi,\eta,\zeta)=49\xi^2-98\eta^2+147\zeta^2.
\]

\item 
Tutki, onko origo funktion paikallinen ääriarvokohta:
\begin{align*}
&\text{a)}\ \ f(x,y)=\cos(x-2y)-\sin xy \qquad
 \text{b)}\ \ f(x,y)=e^x y-xe^y+\sin(x-y) \\
&\text{c)}\ \ f(x,y)=1-x+2y+e^{\frac 12 xy} +\sin(x-2y)+2\cos(x+y)\\[1mm]
&\text{d)}\ \ f(x,y)=1+x+2y+(1+x+2y)^{-1}+2\sin(xy)-12\cos y \\
&\text{e)}\ \ f(x,y)=e^{x^2+y^2}\sqrt[3]{7+42xy} \\[1mm]
&\text{f)}\ \ f(x,y,z)=x+y+z+e^{-x-y-z}+(x+y)\sin(x+y)+\cos(y+z)\\[0.5mm]
&\text{g)}\ \ f(x,y,z)=x+y+z-e^{x+y+z}-(x+y)\sin(x+y)+\cos(y+z)
\end{align*}

\item
Luokittele origo funktion kriittisenä pisteenä eri parametrin $a$ arvoilla:
\[
\text{a)}\ \ f(x,y)= \frac{1+x^2+2y^2}{1+axy} \qquad
\text{b)}\ \ f(x,y)= \sin xy +a\cos(x+y)
\]

\item
Etsi seuraavien funktioiden kaikki kriittiset pisteet ja luokittele ne.
\begin{align*}
&\text{a)}\ \ x^2+2y^2-4x+4y \qquad
 \text{b)}\ \ xy-x+y \qquad
 \text{c)}\ \ x^3+y^3-3xy \\[2mm]
&\text{d)}\ \ x^4+y^4-4xy \qquad\qquad\ 
 \text{e)}\ \ \cos(x+y) \qquad
 \text{f)}\ \ \frac{x}{y}+\frac{8}{x}-y \\ 
&\text{g)}\ \ \frac{1}{x-y+x^2+y^2} \qquad\quad\
 \text{h)}\ \ \frac{xy}{1+x+y} \qquad\,\
 \text{i)}\ \ (x-3y)e^{x^2+y^2}
\end{align*}
 
\item
Etsi karteesinen koordinaatisto, jossa seuraavien funktioiden tasa-arvokäyrät ovat toisen
asteen käyrän yksinkertaisinta perusmuotoa. Hahmottele tasa-arvokäyrien kulku.
\vspace{1mm}\newline
a) \ $16x^2+9y^2+24xy \qquad$
b) \ $2x^2+3y^2+2xy \qquad$
c) \ $xy+y^2$ \newline
d) \ $xy+x+2y+1 \qquad$
e) \ $x^2+4xy+4y^2+6x+8y$ \newline
f) \ $2x^2+3y^2+2xy-4x-4y \qquad$
g) \ $3x^2+4xy+y^2+2x$

\item
Etsi seuraaville toisen asteen pinnoille koordinaatisto, jossa pinnan yhtälö pelkistyy
perusmuotoonsa. Luokittele pinnat. \vspace{1mm}\newline
a) \ $xy+yz+xz=0 \qquad$
b) \ $xy-yz+xz=0 \qquad$
c) \ $xy+yz+x+y+z=0$ \newline
d) \ $xy+yx-xz+2x-3y+z+1=0$ \newline
e) \ $x^2+y^2+z^2+8xy+6yz-24y+8z=0$ \newline
f) \ $x^2+y^2+3z^2+2xy+4yz+4xz+x+y+z-18=0$ 

\item (*)
Näytä, että eräällä $m\in\N$ pätee: Jos origon läpi kuljetaan pitkin käyriä
\[
y=kx^n,\ k\in\R,\ n\in\N,\ n \neq m \quad \text{tai} \quad x=ky^n,\ k\in\R,\ n\in\N,
\]
niin funktiolla $f(x,y)=y^2+3x^{50}y+2x^{100}$ on kullakin käyrällä origossa paikallinen minimi.
Näytä, että origo ei kuitenkaan ole $f$:n paikallinen minimipiste eikä edes laakapiste.

\item \label{H-eig-3: pintojen erikoisluokka}
Toisen asteen pinnan yhtälö $\,\mx^T\mA\mx+\mb^T\mx+c=0,\ \mx^T=[x,y,z]$, on saatettu
koordinatiston kierrolla muotoon
\[
\xi^2+G\xi+H\eta+I\zeta+J=0
\]
($\mA$:lla kaksinkertainen ominaisarvo $\lambda=0$). Näytä, että uudella koordinaatiston
kierrolla ja sen jälkeen origon siirolla yhtälö saadaan perusmuotoon $\eta'=a(\xi')^2$ tai
$(\xi')^2=a$ ($a\in\R$). Sovella menettelyä: \vspace{1mm}\newline
a) \ $x^2+x+2y-2z-4=0 \qquad$
b) \ $(x+y+2z)^2+x+y-z=0$ \newline
c) \ $x^2+y^2+4z^2-2xy+4xz-4yz-1=0$ \newline
d) \ $x^2+4y^2+z^2+4xy+2xz+4yz-4x-3y-4z+4=0$ \newline
e) \ $9x^2+4y^2+z^2-12xy-6xz+4yz+6x-4y+2z+2=0$

\item (*)
Määritä pääakselikoordinaatisto ja luokittele pinta: \vspace{1mm}\newline
a) \ $17x^2+76y^2+54z^2-60xy+12xz+48yz+4x+16y+36z+6=0$ \newline
b) \ $x^2-68y^2+18z^2+36xy+60xz-96yz-22x-53y+26z+121=0$ \newline
c) \ $13x^2+40y^2+45z^2-36xy+24xz+12yz-34x-17y+44z+124=0$ \newline
d) \ $32x^2-20y^2+135z^2-24xy+156xz-132yz+52x-44y+90z+15=0$ \newline
e) \ $48x^2+117y^2+31z^2-36xy-60xz+96yz-216x-66y+86z+300=0$

\end{enumerate}



 %Neliömuodon diagonalisointi
\section{Pinnan kaarevuus} \label{pinnan kaarevuus}
\alku
\index{kaarevuus (pinnan)|vahv}

Palautettakoon mieliin käyräteoriasta, että \pain{kä}y\pain{rän} \pain{kaarevuus} kertoo, miten
nopeasti käyrän yksikkötangenttivektori, tai vaihtoehtoisesti kaaretumissuuntaan osoittava
päänormaalivektori, muuttuu käyrää pitkin kuljettaessa, ks.\ Luku \ref{käyrän kaarevuus},
kaavat \eqref{kaarevuuskaava a}--\eqref{kaarevuuskaava b}. 
\begin{multicols}{2} \raggedcolumns
Jos halutaan määrätä tasokäyrän kaarevuus annetussa käyrän pisteessä $P$, niin laskukaavojen
kannalta mukavin on koordinaatisto, jonka origo on $P$:ssä ja $x$-akseli käyrän
tangentin suuntainen (kuvio). Olkoon käyrän yhtälö tässä koordinaatistossa $y=f(x)$, missä $f$
on kahdesti jatkuvasti derivoituva pisteen $x=0$ lähellä. Tällöin
käyrän yksikkönormaalivektori $P$:n lähellä on
\[
\vec n(x) = \frac{1}{\sqrt{1+[f'(x)]^2}}\,[-f'(x)\vec i+\vec j\,].
\]
\begin{figure}[H]
\setlength{\unitlength}{1cm}
\begin{center}
\begin{picture}(4,4)(0,0.5)
\curve(0,0,1,1.5,2,1.4)\curve(2,1.4,3,1.5,4,3)
\put(0.9,1.4){$\bullet$}
\put(1,1.5){\vector(3,2){2.4}} \put(1,1.5){\vector(-2,3){1.6}}
\put(1,1.5){\vector(3,2){1.2}} \put(1,1.5){\vector(-2,3){0.8}}
\put(1.1,1.1){$P$} \put(3.1,3.3){$x$} \put(-0.4,4){$y$}
\put(1.9,2.5){$\vec i$} \put(0.4,2.8){$\vec j$}
\put(0.5,0.93){\vector(-3,2){1.2}} \put(-0.7,1.9){$\vec n(x)$}
\end{picture}
\end{center}
\end{figure}
\end{multicols}

Kuvion tilanteessa käyrä on koordinaatistossa $(x,y)$ alaspäin kaartuva $P$:n lähellä
($f''(0)<0$), joten tässä tapauksessa käyrän päänormaalivektori $P$:n lähellä $=-\vec n(x)$.
Kun huomioidaan, että $f'(0)=0$, seuraa derivoimalla
\[
\frac{d\vec n}{dx}(0)=-f''(0)\,\vec i.
\]
Valitussa koordinaatistossa siis käyrän (merkkinen) kaarevuus $P$:ssä on $\kappa=f''(0)$
(vrt.\ Luvun \ref{käyrän kaarevuus} kaavat \eqref{kaarevuuskaava b} ja
\eqref{tasokäyrän kaarevuus} päänormaalivektorille $\vec n$\,).

\vspace{2mm}
\begin{multicols}{2} \raggedcolumns
Tarkastellaan nyt pintaa, jonka yhtälö on $z=f(x,y)$. Vastaavasti kuin edellä oletetaan tässä
koordinaatisto valituksi siten, että $xy$-taso on pinnan tangenttitaso origossa (kuvio).
Oletetaan myös, että $f$:n osittaisderivaatat ovat toiseen kertalukuun asti jatkuvia pisteen
$(x,y)=(0,0)$ lähellä.
\begin{figure}[H]
\begin{center}
\import{kuvat/}{kuvaDD-6.pstex_t}
\end{center}
\end{figure}
\end{multicols}
Oletettuun tilanteeseen voidaan ajatella päästyn siten, että on valittu annetun pinnan jokin
piste $P$, siirretty koordinaatiston origo ko.\ pisteeseen, ja sen jälkeen valittu tässä 
pisteessä (kierretty) koordinaatisto, jossa $xy$-taso on pinnan tangenttitaso $P$:ssä.

Pinnan normaali pisteessä $(x,y,f(x,y))$ on (vrt.\ Luku \ref{gradientti})
\[
\vec n(x,y)=\frac{-f_x\vec i-f_y\vec j+\vec k}{\sqrt{1+f_x^2+f_y^2}}\,.
\]
Kordinaatiston valinnan perusteella on $f_x(0,0)=f_y(0,0)=0$, joten seuraa
\begin{align*}
\frac{\partial\vec n}{\partial x}(0,0) &= -a_{11}\vec i-a_{12}\vec j, \\
\frac{\partial\vec n}{\partial y}(0,0) &= -a_{12}\vec i-a_{22}\vec j,
\end{align*}
missä
\[
\mA=\begin{bmatrix} a_{11} & a_{12} \\ a_{12} & a_{22} \end{bmatrix}
   =\begin{bmatrix} f_{xx}(0,0) & f_{xy}(0,0) \\ f_{xy}(0,0) & f_{yy}(0,0) \end{bmatrix}.
\]

\index{kaarevuusmatriisi}%
Matriisia $\mA$ sanotaan (tehdyin oletuksin) pinnan \kor{kaarevuusmatriisiksi} tarkastellussa
pisteessä ja mainitulla tavalla valituissa koordinaateissa $(x,y)$ pinnan tangenttitasossa.
Luvut $a_{11}$ ja $a_{22}$ ilmaisevat pinnan kaarevuuden koordinaattiakselien suunnassa: Origon 
lähellä käyrät $S_1: \ y=0, \ z=f(x,0)$ ja $S_2: \ x=0, \ z=f(0,y)$ ovat likimain ympyräviivoja,
joiden keskipisteet ovat pisteissä $(0,0,1/a_{11})$ ja $(0,0,1/a_{22})$. Luku $a_{12}$ on pinnan
\index{kierevyys (pinnan)}%
\kor{kierevyys} origossa. Kierevyys kertoo, kuinka nopeasti (ja mihin suuntaan) pinnan normaali
'kaatuu sivulle' koordinaattiviivojen suuntaan kuljettaessa (vrt.\ avaruuskäyrän kierevyys:
Harj.teht.\,\ref{käyrän kaarevuus}:\ref{H-dif-3: kierevyys}).

Kaarevuusmatriisi luonnollisesti riippuu tarkastelun kohteena olevasta pinnan pisteestä, mutta
se riippuu myös valitusta koordinaatistosta $(x,y)$ pinnan tangenttitasossa. Onkin syytä tutkia,
miten kaarevuusmatriisi muuttuu, kun koordinaattiakseleita $x,y$ kierretään $z$-akselin ympäri,
eli kun tehdään koordinaattimuunnos
\[
\mC\begin{bmatrix} \xi \\ \eta \end{bmatrix}=\begin{bmatrix} x \\ y \end{bmatrix} \ \ekv \ 
\begin{bmatrix} \xi \\ \eta \end{bmatrix} = \mC^T \begin{bmatrix} x \\ y \end{bmatrix},
\]
missä $\mC$ on ortogonaalinen matriisi (kokoa $2 \times 2$). Vastaus tähän kysymykseen saadaan,
kun ajatellaan funktio $f(x,y)$ approksimoiduksi origon lähellä Taylorin kaavan mukaisesti:
\[
f(x,y)=\tfrac{1}{2}\,a_{11}\,x^2+\tfrac{1}{2}\,a_{22}\,y^2+a_{12}\,xy+\ldots
\]
Vain kehitelmän näkyvillä termeillä on merkitystä kaarevuuden kannalta, sillä jäännöstermin 
toisen kertaluvun osittaisderivaatat häviävät origossa. Päätellään siis, että kaarevuusmatriisi
muuntuu koordinaatiston kierrossa samalla tavoin kuin neliömuoto. Toisin sanoen, kierretyssä
$(\xi,\eta)$-koordinaatistossa kaarevuusmatriisi on
\[
\mB=\mC^T\mA\mC.
\]
Jos tässä erityisesti valitaan $\mC$:n pystyriveiksi $\mA$:n ortonormeeratut ja positiivisesti
suunnistetut ominaisvektorit, niin $\mB$ on diagonaalinen:
\[
\mB=\begin{bmatrix} \kappa_1 & 0 \\ 0 & \kappa_2 \end{bmatrix}.
\]
\index{pzyzy@pääkaarevuus(koordinaatisto)}%
Sanotaan, että $\,\mB$:n lävistäjäalkiot $\kappa_1,\kappa_2$ ovat pinnan \kor{pääkaarevuudet} 
(engl.\ principal curvatures) tarkasteltavassa pisteessä, ja että koordinaatisto $(\xi,\eta)$ on
ko.\ pisteessä \kor{pääkaarevuuskoordinaatisto}. Kun valitaan $\zeta=z$, niin koordinaatistossa
$(\xi,\eta,\zeta)$ pinnan yhtälö on siis origon ympäristössä muotoa
\[
\zeta=\frac{1}{2}\,\kappa_1\xi^2+\frac{1}{2}\,\kappa_2\eta^2\,+\,\ldots
\]
Riittävän säännöllinen pinta on siis jokaisen pisteensä $P$ lähiympäristössä likimain joko 
(a) \pain{elli}p\pain{tinen} p\pain{araboloidi} ($\kappa_1\kappa_2>0$), (b) p\pain{arabolinen} 
\pain{lieriö} tai \pain{taso} ($\kappa_1\kappa_2=0$; taso jos $\kappa_1=\kappa_2=0$)
tai (c) \pain{h}yp\pain{erbolinen} p\pain{arabolidi}
($\kappa_1\kappa_2<0$), vrt.\ toisen asteen pintojen luokittelu edellä. Tämän
jaottelun mukaisesti sanotaan, että $P$ on pinnan
\index{elliptinen piste (pinnan)} \index{hyperbolinen piste (pinnan)}
\index{parabolinen piste (pinnan)}% 
(a) \kor{elliptinen}, (b) \kor{parabolinen}, tai (c) \kor{hyperbolinen} piste. 

Jos pinnan piste $P$ on pääkaarevuuskoordinaatiston $(\xi,\eta,\zeta)$ origo em.\ tavalla ja 
pintaa tarkastellaan positiivisen $\zeta$-akselin (eli pinnan normaalin) suunnasta, niin 
elliptisessä tapauksessa pinta kaartuu joka suuntaan joko tarkkailijaa kohti (jos $\kappa_1>0$
ja $\kappa_2>0$) tai tarkkailijasta poispäin (jos $\kappa_1<0$ ja $\kappa_2<0$). Hyperbolisessa
tapauksessa pinta kaartuu toiseen pääkaarevuussuuntaan tarkkailijaa kohti ja toiseen 
tarkkailijasta poispäin, eli $P$ on nk.\
\index{satulapiste}%
\kor{satulapiste}. Parabolisessa tapauksessa pinta
taas kaartuu oleellisesti vain toiseen pääkaarevuussuuntaan. Kaikissa tapauksissa kaartuminen
tapahtuu pääkaarevuussuuntiin siten, että $\xi\zeta$-tason ja pinnan leikkausviiva on origon
lähellä likimain ympyräviiva, jonka keskipiste on $Q_1=(0,0,1/\kappa_1)$, ja vastaavasti 
$\eta\zeta$-tason ja pinnan leikkausviiva on likimain ympyräviiva, jonka keskipiste on 
$Q_2=(0,0,1/\kappa_2)$. Lukuja $R_i=1/\abs{\kappa_i}$ (voi olla myös $R_i=\infty$) sanotaan
\index{pzyzy@pääkaarevuussäde} \index{kaarevuussäde} \index{kaarevuuskeskiö}%
tämän mukaisesti \kor{pääkaarevuussäteiksi} ja pisteitä $Q_i$ \kor{kaarevuuskeskiöiksi}.
Kaarevuuskeskiöt voi mieltää geometrisesti niin, että pinnan normaali kiertyy likimain 
kaarevuuskeskiön ympäri, kun normaalin ja pinnan leikkauspiste siirtyy tarkasteltavasta 
pisteestä $P$ vastaavaan pääkaarevuussuuntaan. Tämän mukaisesti piste $P$ on siis elliptinen
täsmälleen kun kaarevuuskeskiöt ovat pinnan samalla puolella ja äärellisen matkan päässä ja 
hyperbolinen täsmälleen kun kaarevuuskeskiöt ovat eri puolilla pintaa ja äärellisen matkan 
päässä.
\begin{Exa} Liikuttaessa pallopinnalla
\[
S:\ (x-x_0)^2+(y-y_0)^2+(z-z_0)^2=R^2
\]
mihin tahansa suuntaan kiertyy pallon keskipisteestä $Q=(x_0,y_0,z_0)$ poispäin osoittava 
normaalivektori $\vec n$ keskipisteen ympäri. Molemmat kaarevuuskeskiöt ovat siis pisteessä $Q$.
Päätellään, että jos karteesisen koordinaatiston $(\xi,\eta,\zeta)$ origo on pallopinnalla ja 
$\xi\eta$-taso on pallopinnan tangenttitaso, niin jokainen tällainen koordinaatisto on 
pääkaarevuuskoordinaatisto. Jos normaalivektori $\vec n$ on valittu mainitulla tavalla, niin 
kaarevuusmatriisi on
\[
\mA=\begin{rmatrix} -\frac{1}{R} & 0 \\[1mm] 0 & -\frac{1}{R} \end{rmatrix}.
\]
Pinnan jokainen piste on elliptinen. \loppu
\end{Exa}
\begin{Exa} \vahv{Torus}. $xy$-tason ympyräviiva
\[
K=\{(x,y)\in\R^2 \mid (x-R)^2+y^2=a^2\}, \quad \text{missä} \quad 0<a<R,
\]
\index{torus}%
pyörähtää avaruudessa $y$-akselin ympäri. Laske näin syntyvän pinnan, \kor{toruksen},
pääkaarevuudet sen $xy$-tasolla olevassa pisteessä $P=(x,y,0)$.
\end{Exa}
\begin{figure}[H]
\setlength{\unitlength}{1cm}
\begin{center}
\begin{picture}(10,6)(-1,-1.5)
\put(-1,0){\vector(1,0){10}} \put(8.8,-0.4){$x$}
\put(0,-1){\vector(0,1){5.8}} \put(0.2,4.6){$y$}
\put(4,0){\bigcircle{4}}
\put(0,3){\line(4,-3){4}} \put(2.4,0){\line(0,1){1.2}}
\put(2.4,1.2){\vector(-4,3){0.8}} \put(2.4,1.2){\vector(3,4){0.6}}
\put(2.65,2){$\vec\tau$} \put(1.35,1.4){$\vec n$}
\put(2.3,-0.4){$x$} \put(3.85,-0.5){$R$} \put(3.2,0.75){$a$} \put(-0.4,2.9){$c$}
\put(1.4,2.1){$d$}
\put(3.9,-0.1){$\bullet$} \put(-0.1,2.9){$\bullet$} \put(2.3,1.1){$\bullet$}
\put(2.65,1.1){$P$} \put(4.05,0.3){$Q_1$} \put(0.15,3.15){$Q_2$}
%\put(1.5,0){\line(0,1){3.765}} \put(7,0){\line(0,1){3.3179}}
%\put(1.4,-0.5){$a$} \put(6.9,-0.5){$b$}
%\put(3,1.5){$A$} \put(8,3.7){$y=f(x)$}
\end{picture}
\end{center}
\end{figure}
\ratk \ Geometrisesti voidaan päätellä, että pääkaarevuussuunnat pisteessä $P$ ovat $\vec\tau$
ja $\vec k$, missä $\,\vec\tau\,$ on $xy$-tason suuntainen $S$:n yksikötangenttivektori ja
$\vec k$ on $xy$-tason normaali. Nimittäin kun pisteestä $P$ liikutaan suuntaan $\,\vec\tau\,$,
kiertyy pinnan normaali pisteen $Q_1=(R,0,0)$ ympäri. Suuntaan $\,\vec k\,$ liikuttaessa
normaali taas kiertyy $y$-akselilla olevan pisteen $Q_2=(0,c,0)$ ympäri (ks.\ kuvio), koska
$y$-akseli on pinnan symmetria-akseli. Pisteet $Q_1,\,Q_2$ ovat siis kaarevuuskeksiöt. Kun
pinnan normaali $\vec n$ valitaan osoittamaan pisteestä $Q_1$ poispäin, niin näin valitussa
pääkaarevuuskoordinaatistossa $(\xi,\eta,\zeta)$ on $Q_1=(0,0,-a)$ ja (ks.\ kuvio)
\[
Q_2 = \begin{cases} (0,0,d), &\text{jos $x<R$}, \\ (0,0,-d), &\text{jos $x>R$}, \end{cases}
\]
missä $d$ on pisteen $P$ etäisyys $y$-akselista suuntaan $\pm\vec n$. Tapauksessa $x<R$ 
päätellään yhdenmuotoisten kolmioiden avulla että (ks.\ kuvio)
\[
\frac{a+d}{R}=\frac{a}{R-x} \qimpl d=\frac{ax}{R-x}\,,
\]
joten pääkaarevuudet ovat tässä tapauksessa
\[
\underline{\underline{\kappa_1=-\frac{1}{a}\,, \quad 
                      \kappa_2=\frac{1}{a}\left(\frac{R}{x}-1\right)}}.
\]
Nämä todetaan päteviksi myös tapauksessa $x \ge R$. Piste $P=(x,y,0)$ on tämän mukaisesti
elliptinen kun $x>R$, parabolinen kun $x=R$, ja hyperbolinen kun $x<R$. \loppu

\begin{Exa}
Määritä pinnan $\,S:\ z=\sin xy\,$ pääkaarevuudet ja pääkaarevuuskoordinaatisto origossa.
\end{Exa}
\ratk \ Pinnan tangenttitaso origossa on $xy$-taso, ja kaarevuusmatriisi origossa on
\[
\mA=\begin{bmatrix} 0 & 1 \\ 1 & 0 \end{bmatrix}.
\]
Tämän ominaisarvot ovat $\lambda_1=+1$ ja $\lambda_2=-1$ sekä vastaavat ominaisvektorit
$[1,1]^T$ ja $[-1,1]^T$. Siis pääkaarevuudet ovat $\kappa_1=+1$ ja $\kappa_2=-1$, ja
pääkaarevuuskoordinaatiston kantavektorit ovat
\[
\vec e_1 = (\vec i + \vec j)/\sqrt{2}, \quad \vec e_2 = (-\vec i + \vec j)/\sqrt{2}.
\]
Tässä koordinaatistossa pinnan yhtälö on sarjamuodossa
\begin{align*}
z &= xy - \frac{(xy)^3}{3!} + \ldots \\
  &= \frac{1}{2}(\xi^2-\eta^2) - \frac{1}{48}(\xi^2-\eta^2)^3 + \ldots
\end{align*}
Pinta on siis origon lähellä likimain hyperbolinen paraboloidi (satulapinta), eli origo on
pinnan hyperbolinen piste. \loppu
\begin{Exa}
Määritä pinnan $S:\ z=xy-y^2\,$ pääkaarevuudet ja -suunnat pisteessä $P=(1,1,0)$.
\end{Exa}
\ratk \ Jos $z(x,y)=xy-y^2$, niin $\,z_x(1,1)=1$ ja $z_y(1,1)=-1$, joten pinnan
yksikkönormaalivektori pisteessä $(1,1,0)$ on
\[
\vec n=\frac{1}{\sqrt{3}}(-\vec i+\vec j+\vec k).
\]
Yksikkötangenttivektoreita ovat
\[
\vec t_1 = \frac{1}{\sqrt{2}}(\vec i+\vec j), \quad
\vec t_2 = \vec n\times\vec t_1=\frac{1}{\sqrt{6}}(-\vec i+\vec j-2\vec k),
\]
ja $\{\vec t_1,\vec t_2,\vec n\}$ on ortonormeerattu, positiivisesti suunnistettu systeemi. 
Suoritetaan koordinaattimuunnos $(x,y,z)\hookrightarrow (\xi,\eta,\zeta)\,$ koordinaatistoon 
$\{P,\vec t_1,\vec t_2,\vec n\}\,$:
\[
\begin{rmatrix} 
      \frac{1}{\sqrt{2}} & -\frac{1}{\sqrt{6}} & -\frac{1}{\sqrt{3}} \\[2mm]
      \frac{1}{\sqrt{2}} &  \frac{1}{\sqrt{6}} &  \frac{1}{\sqrt{3}} \\[2mm]
                       0 & -\frac{2}{\sqrt{6}} &  \frac{1}{\sqrt{3}}
      \end{rmatrix}
\begin{bmatrix} \xi \\[2mm] \eta \\[2mm] \zeta \end{bmatrix} 
= \begin{bmatrix} x-1 \\[2mm] y-1 \\[2mm] z \end{bmatrix} \qekv
\begin{cases}
    \,x &= \frac{\xi}{\sqrt{2}}-\frac{\eta}{\sqrt{6}}-\frac{\zeta}{\sqrt{3}}+1, \\[2mm]
    \,y &= \frac{\xi}{\sqrt{2}}+\frac{\eta}{\sqrt{6}}+\frac{\zeta}{\sqrt{3}}+1, \\[2mm]
    \,z &= \phantom{\frac{\xi}{\sqrt{2}}}-\frac{2\eta}{\sqrt{6}}+\frac{\zeta}{\sqrt{3}}\,.
\end{cases}
\]
Sijoitus yhtälöön $\,z=y(x-y)\,$ antaa
\begin{align*}
&-\frac{2}{\sqrt{6}}\,\eta+\frac{1}{\sqrt{3}}\,\zeta
  =\left(\frac{1}{\sqrt{2}}\,\xi+\frac{1}{\sqrt{6}}\,\eta+\frac{1}{\sqrt{3}}\,\zeta+1\right)
   \left(-\frac{2}{\sqrt{6}}\,\eta-\frac{2}{\sqrt{3}}\,\zeta\right) \\[2mm]
&\ekv \quad 3\sqrt{3}\,\zeta+\sqrt{6}\,\xi\zeta+2\sqrt{2}\,\eta\zeta+\zeta^2
            +\sqrt{3}\,\xi\eta+\eta^2=0.
\end{align*}
Tämä määrää implisiittisesti funktion $\zeta=\zeta(\xi,\eta)$, jolle 
$\zeta(0,0)=\zeta_\xi(0,0)=\zeta_\eta(0,0)=0$. Implisiittisesti derivoimalla saadaan lasketuksi
toisen kertaluvun osittaisderivaatat (vrt.\ Harj.teht.\,\ref{H-eig-3: kaarevuusmatriisi}):
\[
\zeta_{\xi\xi}(0,0)=0,\quad \zeta_{\xi\eta}(0,0)=-\frac{1}{3}\,,\quad 
\zeta_{\eta\eta}(0,0)=-\frac{2}{3\sqrt{3}}\,.
\]
Kaarevuusmatriisi valitussa koordinaatistossa on siis
\[
\mA=\begin{rmatrix} 0 & -\frac{1}{3} \\ -\frac{1}{3} & -\frac{2}{3\sqrt{3}} \end{rmatrix}.
\]
$\mA$:n ominaisarvot ovat 
\[
\lambda_1=\frac{1}{3\sqrt{3}}\,, \quad \lambda_2=-\frac{1}{\sqrt{3}}\,,
\]
ja ominaisvektorit ovat suorilla
\[
\lambda_1:\ \xi+\sqrt{3}\,\eta=0, \quad \lambda_2:\ \sqrt{3}\,\xi-\eta=0.
\]
Näiden suorien suuntavektoreita ovat
\begin{align*}
&\vec v_1 = \sqrt{3}\,\vec t_1-\vec t_2 = \frac{2}{\sqrt{6}}(2\vec i+\vec j+\vec k), \\
&\vec v_2 = \vec t_1+\sqrt{3}\,\vec t_2 = \sqrt{2}\,(\vec j-\vec k).
\end{align*}
Siis pääkaarevuudet ja -suunnat alkuperäisessä koordinaatistossa ovat
\begin{alignat*}{2}
\kappa_1 &=\underline{\underline{\frac{1}{3\sqrt{3}}}}\,,\quad 
         &&\text{suunta }\ \underline{\underline{2\vec i+\vec j+\vec k}}, \\
\kappa_2 &=\underline{\underline{-\frac{1}{\sqrt{3}}}}\,,\quad 
         &&\text{suunta }\ \underline{\underline{\vec j-\vec k}}.
\end{alignat*}
Piste $P$ on pinnan hyperbolinen piste. \loppu

\Harj
\begin{enumerate}

\item 
Määritä pinnan pääkaarevuudet ja pääkaarevuuskoordinaatisto annetussa pisteessä:
\vspace{1mm}\newline
a) \ $z=(x+y)(x-2y),\,\ (0,0,0) \quad\ \ $
b) \ $z=2\sin xy + \cos(x+y),\,\ (0,0,1)$ \newline
c) \ $z=1-x+2y+e^{x-2y},\,\ (0,0,2) \quad\ $
d) \ $x^2+2y^2-2xy+z^2=1,\,\ (0,0,-1)$ \newline
e) \ $\cos(x+y)+\sin(x+y+z)-\cos z=0,\,\ (0,0,0)$

\item
Käyrä $K: y=5\sin x$ pyörähtää avaruudessa $x$-akselin ympäri. Määritä syntyvän pyörähdyspinnan 
pääkaarevuussäteet ja -suunnat pisteessä \vspace{1mm}\newline
a) \ $(\pi/6,5/2,0), \quad$
b) \ $(\pi/2,4,-3), \quad$ 
c) \ $(3\pi/4,-5/2,-5/2)$.

\item \label{H-eig-3: kaarevuusmatriisi}
Tarkastellaan pintaa $\,S: F(x,y,z)=0$, missä $F$:n osittaisderivaatat ovat jatkuvia toiseen
kertalukuun asti. Näytä, että jos taso $T: z=z_0$ on $S$:n tangenttitaso pisteessä
$P=(x_0,y_0,z_0) \in S$ ja $F_z(P) \neq 0$, niin $S$:n kaarevuusmatriisi pisteessä $P$ on
\[
\mA=-\frac{1}{F_z(P)} \begin{bmatrix} F_{xx}(P)&F_{xy}(P)\\F_{xy}(P)&F_{zz}(P) \end{bmatrix}.
\]

\item (*)
Määritä pinnan pääkaarevuudet ja pääkaarevuuskoordinaatisto annetussa pisteessä:
\vspace{1mm}\newline
a) \ $9x^2+4y^2+z^2=14,\,\ (1,-1,1) \quad\ $
b) \ $2x^2-2y^2-z^2=1,\,\ (1,0,-1)$ \newline
c) \ $xy+2yz-3xz=0,\,\ (1,1,1) \qquad\ $
d) \ $xy^2z^3=1,\,\ (1,1,1)$

\item (*)
Suora $S_t$ liikkuu avaruudessa parametrin $t\in\R$ mukaan siten, että suora kulkee pisteen
$(0,0,2t)$ kautta ja suoran suuntavektori on $\vec a(t)=\cos t\,\vec i+\sin t\,\vec j$. Määritä
suoran synnyttämän viivoitinpinnan pääkaarevuudet ja pääkaarevuussuunnat origossa.

\end{enumerate} %Pinnan kaarevuus
\section{*Tensorit, vektorit ja skalaarit}  \label{tensorit}
\sectionmark{*Tensorit}
\alku

Kuten edellä on todettu, on neliömuodolla $\mx^T\mA\mx$ se ominaisuus, että koordinaatistoa 
kierrettäessä neliömuodon matriisi muuntuu similaarisesti:
\[
\mx\hookrightarrow\my=\mC^T\mx \ \impl \ \mA\hookrightarrow\mB=\mC^T\mA\mC.
\]
On myös nähty, että pinnan kaarevuusmatriisilla on tämä sama ominaisuus, kun koordinaatiston 
kierto tapahtuu pinnan normaalin ympäri. Yleisesti sanotaan sellaista oliota, jolla on 
jokaisessa koordinaatistossa matriisin olomuoto, ja joka muuntuu koordinaatistoa kierrettäessä
\index{tensori}%
ym.\ säännön mukaan, \kor{tensoriksi}.\footnote[2]{Tensori on matematiikassa tässä esitettyä
yleisempi käsite. Tensorit liitetään, ei ainoastaan karteesisen koordinaatiston kiertoihin,
vaan yleisempien käyräviivaisten kordinaatistojen välisiin muunnoksiin. Muunnoksiin liittyvä
differentiaalilaskenta, nk.\ \kor{tensorilaskenta}, on oma matematiikan lajinsa. Tensoreilla ja
tensorilaskennalla on paljon käyttöä fysiikassa.}
Neliömuoto, matriisinsa kautta nähtynä, on siis tensori, itse asiassa \kor{symmetrinen tensori}.
Myös pinnan kaarevuus on symmetrinen tensori, ja kaarevuusmatriisin sijasta puhutaankin yleensä
\index{kaarevuustensori}%
\kor{kaarevuustensorista}. Näin puhuen kaarevuus tulee selvemmin ymmärretyksi pinnan
paikalliseksi geometriseksi ominaisuudeksi, joka vain näyttäytyy erilaisena eri
koordinaatistoissa.

Jos $f=f(x,y,z)$ on kahdesti jatkuvasti derivoituva, niin tietyssä pisteessä laskettu Hessen 
matriisi
\[
\mA=\begin{bmatrix} 
    f_{xx} & f_{xy} & f_{xz} \\ f_{xy} & f_{yy} & f_{yz} \\ f_{xz} & f_{yz} & f_{zz} 
    \end{bmatrix}
\]
muuntuu koordinaatistoa kierrettäessä ym. similaarisuussäännön mukaisesti. --- Tämä nähdään 
approksimoimalla $f$:ää toisen asteen Taylorin polynomilla. Silloin voidaan ajatella, että $f$:n
'toinen derivaatta' ei ole mihinkään yksittäiseen koordinaatistoon sidottu toisten 
osittaisderivaattojen matriisi, vaan pikemmin \pain{tensori}, jonka ilmiasuja nämä matriisit 
ovat (!).

Myös \pain{lineaarikuvaus} $\,A:\R^n\kohti\R^n$ voidaan ymmärtää tensorina, sillä jos
\[
A:\ \mx \map \my \ \ekv \ \my=\mA\mx,
\]
niin nähdään, että koordinaattimuunnoksessa
\[
\mx=\mC\mx',\quad \my=\mC\my'\quad (\inv{\mC}=\mC^T)
\]
lineaarikuvaus saa muodon
\[
\my'=\mB\mx',\quad \mB=\mC^T\mA\mC.
\]
\index{zza@\sov!Magneettinen permeabiliteetti}%
\begin{Exa} \vahv{Magneettinen permeabiliteetti}.
Sähkömagnetiikassa magneettivuon tiheyden ($\vec B$) ja magneettikentän voimakkuuden ($\vec H$)
välinen yksinkertaisin materiaalilaki on
\[
\vec B=\mu\vec H,
\]
missä materiaalivakio $\mu$ on väliaineen \pain{ma}g\pain{neettinen} p\pain{ermeabiliteetti}.
Tämä laki on sama kaikissa koordinaatistoissa. Yleisemmin, jos materiaali on epäisotrooppinen,
voi $\vec B$:n ja $\vec H$:n riippuvuus olla yleisempi lineaarikuvaus muotoa
\[
\mB=\boldsymbol{\mu}\mH \qekv \begin{bmatrix} B_1 \\ B_2 \\ B_3 \end{bmatrix} =
\begin{bmatrix} 
\mu_{11}&\mu_{12}&\mu_{13} \\ \mu_{21}&\mu_{22}&\mu_{23} \\ \mu_{31}&\mu_{32}&\mu_{33} 
\end{bmatrix}
\begin{bmatrix} H_1 \\ H_2 \\ H_3 \end{bmatrix}.
\]
Tällöin $\boldsymbol{\mu}$ on tensori, eli matriisi $(\mu_{ij})$ muuntuu koordinaatistoa 
kierrettäessä similaarisesti. \loppu
\end{Exa}
\begin{Exa}
Kahdesti jatkuvasti derivoituvasta funktiosta $f=f(x,y)$ tiedetään, että $f_{xx}(0,0)=24$, 
$f_{yy}(0,0)=-24$ ja $f_{xy}(0,0)=-7$. Laske $f_{\xi\xi}(0,0)$, $f_{\xi\eta}(0,0)$ ja 
$f_{\eta\eta}(0,0)$ koordinaatistossa $(\xi,\eta)$, jonka kantavektorit ovat
\[
\vec e_\xi=\frac{1}{5}(4\vec i + 3\vec j),\quad \vec e_\eta=\frac{1}{5}(-3\vec i +4\vec j).
\]
\end{Exa}
\ratk Koordinaattimuunnos $(x,y)\hookrightarrow (\xi,\eta)$ määräytyy ehdosta
\[
%       &x\vec i+y\vec j=\xi\vec e_\xi+\eta\vec e_\eta
%                =\frac{1}{5}(4\xi-3\eta)\vec i+\frac{1}{5}(3\xi+4\eta)\vec j \\
\begin{bmatrix} x\\y \end{bmatrix}=\frac{1}{5}\begin{rmatrix} 4&-3\\3&4 \end{rmatrix}
\begin{bmatrix} \xi \\ \eta \end{bmatrix}=\mC\begin{bmatrix} \xi \\ \eta \end{bmatrix}.
\]
Koska $f$:n toisen kertaluvun osittaisderivaatat muodostavat tensorin, on
\begin{align*}
\begin{bmatrix} 
f_{\xi\xi}(0,0) & f_{\xi\eta}(0,0) \\ f_{\xi\eta}(0,0) & f_{\eta\eta}(0,0) 
\end{bmatrix}
&=\mC^T\begin{bmatrix} f_{xx}(0,0)&f_{xy}(0,0) \\ f_{xy}(0,0)&f_{yy}(0,0) \end{bmatrix}\mC \\
&=\frac{1}{25}\begin{rmatrix} 4&3\\-3&4 \end{rmatrix} 
              \begin{rmatrix} 24&-7\\-7&-24 \end{rmatrix}
              \begin{rmatrix} 4 & -3 \\ 3 & 4 \end{rmatrix} \\
&=\begin{rmatrix} 0&-25\\-25&0 \end{rmatrix}.
\end{align*}
Siis $f_{\xi\xi}(0,0)=f_{\eta\eta}(0,0)=0$, $f_{\xi\eta}=-25$. --- Eräs oletukset toteuttava 
funktio on
\[
f(x,y)=(4x+3y)(3x-4y)=-25\xi\eta. \loppu
\]

\subsection{Vektorit}
\index{vektoria@vektori (geometrinen)!a@tason \Ekaksi|vahv}
\index{vektoria@vektori (geometrinen)!b@avaruuden \Ekolme|vahv}

Jos tensori on olio, joka 'näyttää matriisilta' jokaisessa (karteesisessa) koordinaatistossa,
niin tason tai avaruuden vektori voidaan vastaavasti tulkita olioksi, jonka olomuoto on lukupari
tai lukukolmikkko jokaisessa koordinaatistossa. Kuten tensori, tämä pari tai kolmikko muuntuu
koordinaatistoa kierrettäessä systemaattisella tavalla. Esimerkiksi jos 
$\vec v=x\vec i + y\vec j + z\vec k$ on avaruusvektori, niin muunnossääntö 
$\mx \vastaa (x,y,z) \map (x',y',z') \vastaa \mx'$ siirryttäessä kierrettyyn koordinaatistoon,
jonka kantavektorien koordinaatit kannassa $\{\vec i,\vec j,\vec k\}$ ovat 
ortogonaalimatriisin $\mC$ sarakkeet, on (vrt.\ Luku \ref{lineaarikuvaukset})
\[
\mC\mx'=\mx \qekv \mx'=\mC^T\mx.
\]
Siis ei ainoastaan tensorin, vaan myös vektorin 'perimmäinen olemus' paljastuu vasta 
koordinaatistoa kierrettäessä (!). --- Havaitaan myös, että symmetrisellä tensorilla ja 
vektorilla on se yhteinen piirre, että löytyy koordinaatisto, tai koordinaatistoja, joissa 
tensorin/vektorin ilmiasu matriisina/$\R^n$:n alkiona saa mahdollisimman yksinkertaisen muodon.
Symmetrisen tensorin tapauksessa tämä yksinkertaisin muoto on diagonaalimatriisi. Tason tai 
avaruuden vektorin yksinkertaisin muoto on lukupari $(a,0)$ tai lukukolmikko $(a,0,0)$, missä 
$a>0$ on vektorin pituus. Tällaiseen koordinaatistoon siirryttäessä  palataan itse asiassa 
vektorin alkuperäiseen geometriseen ideaan --- vektorihan oli alunperin suunnattu jana. Vektorin
suunnan pysyvyys koordinaatistoa kierrettäessä varmistetaan em.\ muunnossäännöllä. Sen mukaan
siis vektorin ilmiasu $\R^n$:n alkiona \pain{ei} ole pysyvä, vaan nimenomaan muuttuu 
koordinaatistoa kierrettäessä, muunnossäännön ilmaisemalla systemaattisella tavalla.
\begin{Exa} Voidaan kuvitella (etenkin tapauksissa $n=2$ ja $n=3$), että euklidinen avaruus 
$\R^n$ on pisteavaruuden $E^n$ koordinaattiavaruus jossakin valitussa (karteesisessa) 
koordinaatistossa. Jos nyt valittua pistettä $P \in E^n$ edustaa vektori $\mx$ (paikkavektori),
niin koordinaatistoa kierrettäessä $P$:n koordinaatit muuntuvat kaavan $\mx'=\mC^T\mx$
mukaisesti. Muunnoskaava varmistaa, että piste 'pysyy paikallaan', eli pisteen paikkavektori on
vektori juuri tämän vaatimuksen vuoksi. Vastaavasti jos tarkastellaan funktion $f(\mx)$
muuntumista koordinaatiston kierrossa, niin muunnossäännöllä $g(\mx')=f(\mC\mx')$ tarkoitetaan,
että 'funktio pysyy paikallaan', ts.\ funktion arvo tietyssä (paikallaan pysyvässä) pisteessä
ei muutu. \loppu \end{Exa}
\begin{Exa} Olkoon $f=f(x,y,z)$ on differentioituva funktio. Kierretään koordinaatistoa 
muunnoksella $\mC\mx'=\mx\ \ekv\ \mx'=\mC^T\mx\ (\mx=[x,y,z]^T,\ \mx'=[x',y',z']^T)$ ja 
merkitään $g(\mx')=f(\mx)=f(\mC\mx')$. Olkoon $\nabla f(0,0,0)= a_1\vec i+a_2\vec j+a_3\vec k$.
Tällöin jos $\ma=[a_1,a_2,a_3]^T$, niin differentioituvuuden määritelmän mukaisesti on
\[
f(x,y,z)=f(0,0,0)+\ma^T\mx + o(\abs{\mx}).
\]
Ketjusääntöjen perusteella myös $g$ on differentioituva origossa, joten jollakin 
$\mb=[b_1,b_2,b_2]^T$ on
\[
g(x',y',z')=g(0,0,0)+\mb^T\mx' + o(\abs{\mx'})=f(0,0,0)+\mb^T\mx' + o(\abs{\mx'}).
\]
Kun näissä hajotelmissa oletetaan, että $\mC\mx'=\mx$, niin $\abs{\mx'}=\abs{\mx}$ ja 
$g(x',y',z')=f(x,y,z)$, joten on oltava
\[
\ma^T\mx=\mb^T\mx'=\mb^T\mC^T\mx=(\mC\mb)^T\mx \qimpl \mC\mb=\ma.
\]
Tulos merkitsee, että funktion gradientti (tulkittuna pystyvektorina) muuntuu koordinaatistoa
kierrettäessä kuten vektori. Samaan tulokseen tullaan myös laskemalla $g$:n gradientti suoraan
ketjusääntöjen avulla kaavasta $g(\mx')=f(\mC\mx')$. \loppu
\end{Exa}
Funktion gradientti on jo aiemminkin tulkittu vektoriksi, mutta tällöin tulkinta perustui vain
gradientin 'ulkonäköön' kiinteässä koordinaatistossa. Nyt voidaan siis vahvistaa, että tulkinta
kestää myös koordinaatiston kierron:
\[
\boxed{\quad\kehys \text{Gradientti \kor{on} vektori}. \quad}
\]

\subsection{Skalaarit}
\index{skalaari|vahv}

Jos tensorin ja vektorin todellinen luonne paljastuu vasta koordinaatiston kierrossa, niin 
samoin on \pain{skalaarin} laita. Skalaari on olio, jonka ilmiasu jokaisessa koordinaatistossa
on luku (tässä reaaliluku), ja nimenomaan aina \pain{sama} luku. Esimerkiksi jos 
$f=f(\mx),\ \mx\in\R^n$, on reaaliarvoinen funktio, niin $f$:n  arvo tietyssä 
(kiinteäksi ajatellussa) pisteessä on skalaari. Vektoriin liittyvä skalaari on vektorin pituus,
koska pituus on reaaliluku, joka ei muutu koordinaatiston kierrossa. Kahteen vektoriin 
$\mx,\my\in\R^n$ liittyvä skalaari on (nimensä mukaisesti) \pain{skalaaritulo}, sillä jos 
$\mx'=\mC^T\mx$ ja $\my'=\mC^T\my$, niin
\[
(\mx')^T\my' = (\mC^T\mx)^T\mC^T\my = \mx^T\mC\mC^T\my = \mx^T\mI\my = \mx^T\my.
\]
Tensoreihin liittyvä mielenkiintoinen skalaari on tensorin \kor{jälki}:
\begin{Def} \label{tensorin jälki} \index{jzy@jälki (tensorin)|emph}
Jos $T$ on $\R^n$:n tensori, jonka matriisi annetussa ortonormeeratussa koordinaatistossa on
$\mA=(a_{ij})$, niin $T$:n \kor{jälki} (engl.\ trace) on
\[
tr\,T = \sum_{i=1}^n a_{ii}\,.
\]
\end{Def}
\begin{Lause} \label{jälki on skalaari} Tensorin jälki on skalaari. \end{Lause}
\tod Olkoon $\mC$ ortogonaalimatriisi kokoa $n \times n$ ja $\mB=\mC^T\mA\mC$. Lause väittää,
että
\[
\sum_{i=1}^n [\mB]_{ii} = \sum_{i=1}^n [\mA]_ {ii}\,.
\]
Lähdetään matriisitulon määritelmästä ja vaihdetaan summausjärjestystä\,:
\begin{align*}
\sum_{i=1}^n [\mB]_{ii} 
         &= \sum_{i=1}^n\sum_{j=1}^n[\mC^T]_{ij}\,[\mA\mC]_{ji} \\
         &= \sum_{i=1}^n\sum_{j=1}^n[\mC^T]_{ij}\sum_{k=1}^n[\mA]_{jk}\,[\mC]_{ki} \\
         &= \sum_{j=1}^n\sum_{k=1}^n[\mA]_{jk}\left(\sum_{i=1}^n[\mC]_{ki}\,[\mC^T]_{ij}\right).
\end{align*}
Tässä on $\mC$:n ortogonaalisuuden perusteella
\[
\sum_{i=1}^n[\mC]_{ki}\,[\mC^T]_{ij} = [\mC\mC^T]_{kj} = [\mI]_{kj} = \delta_{kj}\,,
\]
joten
\[
\sum_{i=1}^n [\mB]_{ii} = \sum_{j=1}^n\sum_{k=1}^n[\mA]_{jk}\delta_{kj} 
                        = \sum_{j=1}^n[\mA]_{jj}\,. \loppu
\]

\begin{Exa} Tarkastellaan funktion $u=u(x,y)\,$ 'toista derivaattaa', eli tensoria $T$, jonka
matriisi on
\[
\mA = \begin{bmatrix} u_{xx}(x,y) & u_{xy}(x,y) \\ u_{xy}(x,y) & u_{yy}(x,y) \end{bmatrix}.
\]
Kun koordinaatistoa kierretään muunnoksella $\mC[\xi,\eta]^T=[x,y]^T$ ja merkitään 
$v(\xi,\eta)=u(x(\xi,\eta),y(\xi,\eta))$, niin $T$:n matriisi kierretyssä koordinaatistossa on
\[
\begin{bmatrix} 
v_{\xi\xi}(\xi,\eta) & v_{\xi\eta}(\xi,\eta) \\ v_{\xi\eta}(\xi,\eta) & v_{\eta\eta}(\xi,\eta)
\end{bmatrix}
= \mC^T \begin{bmatrix} 
        u_{xx}(x,y) & u_{xy}(x,y) \\ u_{xy}(x,y) & u_{yy}(x,y) 
        \end{bmatrix} \mC.
\]
Lauseen \ref{jälki on skalaari} mukaan pätee
\[
(\partial_\xi^2+\partial_\eta^2)v(\xi,\eta)\,=\,(\partial_x^2+\partial_y^2)u(x,y). \loppu
\]
\end{Exa}
Esimerkistä voidaan vetää se (yleisemminkin $\R^n$:ssä pätevä) merkittävä johtopäätös, että
Laplacen operaattori $\Delta$ 'näyttää samalta' kaikissa karteesisissa koordinaatistoissa.
Koska $\Delta$ myös operoi sekä skalaareihin että vektoreihin kuten skalaari, niin voidaan
kahdessa eri merkityksessä sanoa\,:
\[
\boxed{\quad\kehys \text{$\Delta$ \kor{on skalaarinen operaattori}}. \quad}
\]

\Harj
\begin{enumerate}

\item
Kahdesti jatkuvasti derivoituvasta funktiosta $f(x,y)$ tiedetään: $f_{xx}(0,0)=0$,
$f_{yy}(0,0)=-2$ ja $f_{xy}(0,0)=2$. Laske $f_{\xi\xi}(0,0)$ ja $f_{\eta\eta}(0,0)$ sellaisessa
kierretyssä koordinaatistossa, jossa $f_{\xi\eta}(0,0)=0$.

\item
Eräässä materiaalissa magneettikentän tiheyden ja magneettikentän voimakkuuden välillä on
riippuvuus $\mB=\boldsymbol{\mu}\mH$, missä $\boldsymbol{\mu}$ on tensori, jonka
matriisiesitys on
\[
\boldsymbol{\mu}= \begin{bmatrix} \mu&0&0 \\ 0&2\mu&0 \\ 0&0&3\mu \end{bmatrix}
\]
koordinaatistossa, jonka kantavektorit ovat
\[
\vec e_1=\frac{1}{5}(3\vec i+ 4\vec j\,), \quad 
\vec e_2=\frac{1}{25}(12\vec i-9\vec j+20\vec k\,), \quad 
\vec e_3=\frac{1}{25}(16\vec i-12\vec j-15\vec k\,). 
\]
Millainen on $\boldsymbol{\mu}$:n esitysmuoto peruskoordinaatiston kannassa
$\{\vec i,\vec j,\vec k\}$\,?

\item
Näytä, että jos $\vec a$ ja $\vec F$ ovat vektorikenttiä (eli vektoriarvoisia funktioita),
niin \, a) $\nabla\cdot\vec F$ on skalaarikenttä, \ b) $(\vec a\cdot\nabla)\vec F$ on
vektorikenttä.

\item (*)
a) Näytä, että differentiaalioperaattori
\[
A=a\partial_x^2+b\partial_y^2+c\partial_x\partial_y,
\]
missä $a$, $b$ ja $c$ ovat skalaareita ($a,b,c\in\R$), säilyy muuttumattomana $\Rkaksi$:n
koordinaatiston kierrossa täsmälleen kun $a=b$ ja $c=0$,
eli kun $A=a\Delta$. \vspace{1mm}\newline
b) Millaiseksi osittaisdifferentialiyhtälö $u_{xx}-u_{yy}=0$ muuntuu kierretyssä 
koordinaatistossa, jonka kantavektorit ovat
\[
\vec e_1=\frac{1}{\sqrt{2}}(\vec i+\vec j\,), \quad 
\vec e_2=\frac{1}{\sqrt{2}}(-\vec i+\vec j\,)\,?
\]
c) Jos $\vec a=\vec i+\vec j$, niin millaisen muodon differentiaalioperaattori
\[
(\vec a\cdot\nabla)^2 = (\partial_x)^2+(\partial_y)^2+2\partial_x\partial_y
\]
saa b-kohdan kierretyssä koordinaatistossa?

\end{enumerate} %*Tensorit, vektorit ja skalaarit

\chapter{Usean muuttujan integraalilaskenta}

Tämän luvun sisällön muodostavat erilaiset \pain{määrät}y\pain{n} \pain{inte}g\pain{raalin}
käsitteen laajennukset useampiin ulottuvuuksiin ja erilaisiin geometrisiin tilanteisiin.
Tarkasteltavia integraalien lajeja ovat kahden tai useamman muuttujan funktioihin 
liitettävät \kor{taso-} ja \kor{avasuusintegraalit}, \kor{viiva-} eli 
\kor{käyräintegraalit} ja \kor{pintaintegraalit}. Sovelluksina käsitellään mm.\ pinta-alojen,
tilavuuksien, kaarenpituuksien, \kor{paino\-pisteiden}, \kor{hitausmomenttien}, ym.\ laskemista.

Luvussa \ref{tasointegraalit} tarkastellaan aluksi tasointegraalien \kor{mittateoreettisia}
perusteita. Osoittautuu, että tasointegraalissa, samoin muissakin integraalien lajeissa, on
kyse integroimisesta jonkin \kor{mitan} suhteen. Tasointegraaliin liittyvä mitta on $\R^2$:n
osajoukkoihin liitettävä \kor{pinta-alamitta}, joka Luvussa \ref{tasointegraalit} määritellään
tarkemmin \kor{Jordanin} mittana. Paitsi tämän mitan ominaisuuksia ja suhdetta integraaliin,
Luvussa \ref{tasointegraalit} tarkastellaan myös tasointegraalin laskemista numeerisesti 
'suoraan määritelmästä'.

Usean muuttujan integraalilaskun keskeistä sisältöä on väistämättä integraalien klassinen
laskutekniikka, jonka perusta on \kor{Fubinin lause}
(Luku \ref{tasointegraalien laskutekniikka}). Lause palauttaa taso- ja avaruusintegraalit
peräkkäisiksi eli \kor{iteroiduiksi} 1-ulotteisiksi integraaleiksi, jotka voidaan suotuisissa
oloissa laskea suljetussa muodossa. Fubinin lauseen käyttöä tarkastellaan Luvuissa
\ref{tasointegraalien laskutekniikka}--\ref{avaruusintegraalit}. 

Integraalien klassisessa laskutekniikkassa auttaa usein myös \kor{muuttujan vaihto}, esimerkiksi
siirtyminen tason tai avaruuden käyräviivaisiin koordinaatistoihin. Muuttujan vaihdon
laskutekniikkaa tarkastellaan Luvussa \ref{muuttujan vaihto integraaleissa}.

Taso- ja avaruusintegraalien moninaisia sovelluksia käydään läpi Luvussa
\ref{pinta- ja tilavuusintegraalit}. Lopuksi Luvuissa 
\ref{viivaintegraalit}--\ref{pintaintegraalit} tarkastellaan taso- ja avaruuskäyriin sekä
$\R^3$:n kaareviin pintoihin liittyviä viiva- ja pintaintegraaleja sovelluksineen. 

 %Usean muuttujan integraalilaskenta
\section{Pinta-alamitta ja tasointegraalit} \label{tasointegraalit}
\alku \sectionmark{Tasointegraalit}
\index{tasointegraali|vahv}

Tasoalueen p\pain{inta-alan} määräämistä määrätyn integraalin avulla on tarkasteltu aiemmin
Luvussa \ref{pinta-ala ja kaarenpituus}. Tässä luvussa tarkastelun kohteena ovat yleisemmät
\kor{tasointegraalit} muotoa
\begin{equation} \label{integraalimerkintä}
\int_A f\, d\mu=I(f,A,\mu), \tag{$\star$}
\end{equation}
missä $A\subset\R^2$, $f$ on $A$:ssa määritelty (kahden muuttujan) funktio ja $\mu$ on $\R^2$:n
\kor{pinta-alamitta}. Integraali $I(f,A,\mu)$ on tarkemmin \pain{reaaliluku},
joka riippuu funktiosta $f$, joukosta $A$ ja mitasta $\mu$, kuten ilmenee lukutavasta:
$f$\pain{:n} \pain{inte}g\pain{raali} y\pain{li} \pain{$A$:n} \pain{mitan} $\mu$
\pain{suhteen}. --- Kyseessä on ennestään tutun \pain{määrät}y\pain{n} \pain{inte}g\pain{raalin}
yleistys; nimittäin kuten jäljempänä havaitaan, merkintä \eqref{integraalimerkintä} on pätevä
myös, kun $f$ on välillä $A=[a,b]$ määritelty (integroituva) funktio,
$I(f,A,\mu)=\int_a^b f(x)\,dx$ ja $\mu$ on $\R$:n \kor{pituusmitta}.

\subsection{Jordanin pinta-alamitta}
\index{pinta-alamitta!a@tason|vahv}
\index{Jordan-mitta!a@pinta-alamitta|vahv}

Koska tasointegraali määritellään suhteessa pinta-alamittaan, on tämä ensin määriteltävä.
\index{mitta, mitallisuus}%
\kor{Mitta} (engl.\ measure) on yleisemmin \kor{joukkofunktio}, joka liittää \kor{mitalliseen}
(engl.\ measurable) joukkoon (tässä $\R^2$:n osajoukkoon) reaaliluvun:
\[
\mu:\mathcal{M}\Kohti\R,\quad \mathcal{M}=\{\text{mitalliset joukot $A\subset\R^2$}\}.
\]
Pinta-alamitaksi oletetaan jatkossa \kor{Jordan-mitta}\footnote[2]{Jordan-mittaa sanotaan myös
\kor{Peanon--Jordanin} mitaksi. Termit viittaavat matemaatikoihin \hist{Camille Jordan}
(ransk.\ 1838-1922) ja \hist{Giuseppe Peano} (ital.\ 1858-1932). \index{Jordan, C.|av}
\index{Peano, G.|av}}, joka jatkossa määritellään niin,
että seuraavat kolme aksioomaa toteutuvat. Aksioomista ensimmäisessä asetetaan
\index{perussuorakulmio}%
koordinattiakselien mukaan suunnatun nk.\ \kor{perussuorakulmion} mitta. Muut kaksi ovat
yleisempienkin \kor{positiivisten} mittojen aksioomia (tai aksioomien seurauksia). 
\begin{itemize}
\item[A1.] \kor{Perussuorakulmion mitta}: \ 
           $T=\{(x,y)\in\R^2 \mid x\in[a_1,b_1]\ \ja\ y\in[a_2,b_2]\}$: \\[0.25cm] 
           $\mu(T)=(b_1-a_1)(b_2-a_2)$.
\item[A2.] \kor{Additiivisuus}: \ $A,B\in\mathcal{M} \ \ja\ A \cap B=\emptyset$ \\[0.25cm]
           $\impl \ A \cup B \in \mathcal{M} \ \ja \ \mu(A \cup B)=\mu(A)+\mu(B)$.
           \index{additiivisuus!b@mitan}
\item[A3.] \kor{Positiivisuus}: \ $\mu(A)\geq 0\quad\forall A\in\mathcal{M}$.
           \index{positiivisuus (mitan)}
\end{itemize}
Huomattakoon, että aksiooman A1 mukaisesti mitta $\mu$ on lähtökohtaisesti
valitusta (karteesisesta) koordinaatistosta riippuva. Mittoja käsittelevälle matematiikan
lajille, \kor{mittateorialle}, tällainen lähtökohta on tyypillinen, jolloin mitan
riippumattomuus koordinaatistosta (mikäli tosi) on aina erikseen osoitettava.

Tyhjä joukko $\emptyset$ katsotaan aina mitalliseksi, jolloin aksiooman A2 mukaan on oltava
$\mu(\emptyset)=0$. Yleisemmin jos $A \in \mathcal{M}$ ja $\mu(A)=0$, sanotaan että $A$ on
\index{nollamittaisuus}%
\kor{nollamittainen}. Nollamittaiset joukot ovat mittateoriassa tärkeällä
sijalla.\footnote[2]{Nollamittaisuuteen liittyy erikoinen matematiikan termi
\kor{melkein kaikkialla} (engl.\ almost everywhere), joka tarkoittaa: muualla kuin
nollamittaisessa (osa)joukossa. \index{melkein kaikkialla|av}}
\begin{Exa} Koordinaattiakselien suuntaiset janat $A=\{a\}\times[b,c]$ ja
$B=[a,b]\times\{c\}$ voidaan tulkita suorakulmion erikoistapauksiksi, jolloin aksiooman A1
mukaan $\mu(A)=\mu(B)=0$. \loppu
\end{Exa}
\begin{Exa} Jos $A=[a_1,b_1]\times[a_2,b_2]$ ja $B=[b_1,c_1]\times[a_2,b_2]$, niin
$A \cup B=[a_1,c_1]\times[a_2,b_2]$, joten aksiooman A1 mukaan on
$\mu(A \cup B)=\mu(A)+\mu(B)$. \loppu
\end{Exa}
Jälkimmäisessä esimerkissä additiivisuusaksiooma A2 toteutuu, vaikka $A \cap B\neq\emptyset$.
Tämä johtuu siitä, että (edellisen esimerkin mukaan) $A \cap B$ on nollamittainen.
--- Yleisemminkin voi aksiooomassa A2 ehdon $\,A \cap B = \emptyset\,$ lieventää ehdoksi
$\mu(A \cap B)=0$ (ks.\ Jordan-mitan määrittely jäljempänä ja
Harj.teht.\,\ref{H-uint-1: väittämiä}c).

\subsection{Integraali yli rajoitetun joukon}
\index{Riemannin!a@integraali|vahv}
\index{Riemann-integroituvuus!c@tasossa|vahv}

Olkoon $A\subset\R^2$ rajoitettu joukko, jolloin se sisältyy johonkin perussuorakulmioon:
\[
A \subset T = [a_1,b_1]\times [a_2,b_2]
\]
(ks.\ kuvio alla). Olkoon $f$ $A$:ssa määritelty rajoitettu funktio ja olkoon $f_0=$ $f$:n
\index{nollajatko (funktion)}%
\kor{nollajatko} $A$:n ulkopuolelle:
\[
f_0(x,y)=\begin{cases} \,f(x,y), &\text{kun}\ (x,y)\in A, \\ \,0, &\text{muulloin}. \end{cases}
\]
Tällöin sovitaan ensinnäkin, että
\[
\boxed{\kehys\quad \int_A f\, d\mu=\int_T f_0\, d\mu, \quad A \subset T. \quad}
\]
\begin{figure}[H]
\begin{center}
\input{kuvat/kuvaUint-1.pstex_t}
\end{center}
\end{figure}
Kuten määrätyssä integraalissa, integraalin $\int_T f_0\,d\mu$ määrittelyn lähtökohtana on
\index{jako (suorakulmioihin)}%
$T$:n \kor{jako} (ositus), tässä tapauksessa perussuorakulmion muotoisiin osajoukkoihin:
\begin{align*}
&\mathcal{T}_h=\{T_{kl}, \ k=1,\ldots, m, \ l=1,\ldots, n\}, \\[1mm]
&T_{kl} = [x_{k-1},x_k]\times [y_{l-1},y_l], \quad
\begin{cases} \,a_1 = x_0<x_1<\ldots x_m=b_1, \\ \,a_2 = y_0<y_1<\ldots < y_n=b_2. \end{cases}
\end{align*}
\begin{figure}[H]
\begin{center}
\input{kuvat/kuvaUint-2.pstex_t}
\end{center}
\end{figure}
Jaon (eli suorakulmiojoukon $\mathcal{T}_h$) indeksoinnissa käytetty parametri $h$ on jälleen
\index{tiheysparametri}%
\kor{tiheysparametri}. Tämän määritelmä kahdessa ulottuvuudessa on
\[
h=\max\{h_x,\,h_y\}, \quad h_x=\max_k\{x_k-x_{k-1}\}, \quad h_y=\max_l\{y_l-y_{l-1}\}.
\]
Integraali $\int_T f_0\,d\mu$ määritellään nyt määrätyn integraalin tapaan eli
\index{Riemannin!b@summa}%
\kor{Riemannin summien} avulla raja-arvoprosessilla
(vrt.\ Määritelmät \ref{Riemannin integraali} ja \ref{raja-arvo Lim})
\[
\int_T f_0\,d\mu = \Lim_{h\kohti 0} \sum_{k=1}^m \sum_{l=1}^n f_0(\xi_{kl},\eta_{kl})\mu(T_{kl}),
\]
missä $(\xi_{kl},\eta_{kl})\in T_{kl}$. Vaihtoehtoinen, etenkin teoreettisissa tarkasteluissa
kätevä integroituvuuden kriteeri saadaan, kun määritellään jakoon $\mathcal{T}_h$ liittyen
\index{Riemannin!c@ylä- ja alasumma}%
\begin{align*}
&-\text{\kor{yläsumma}}: \quad \overline{\sigma}_h(f_0,\mathcal{T}_h)
                          =\sum_{k=1}^m\sum_{l=1}^n M_{kl}\,\mu(T_{kl}), \quad
                             M_{kl}=\sup_{(x,y)\in T_{kl}} f_0(x,y), \\
&-\text{\kor{alasumma}}: \quad \underline{\sigma}_h(f_0,\mathcal{T}_h)
                          =\sum_{k=1}^m\sum_{l=1}^n m_{kl}\,\mu(T_{kl}), \quad
                             m_{kl}=\inf_{(x,y)\in T_{kl}} f_0(x,y).
\end{align*}
Tällöin $f$ on integroituva yli $A$:n täsmälleen kun pätee
(vrt.\ Lause \ref{Riemann-integroituvuus})
\[
\sup_{\mathcal{T}_h} \underline{\sigma}_h(f_0,\mathcal{T}_h) 
                     = \underline{I}(f,\mu,A) = \overline{I}(f,\mu,A)
                     = \inf_{\mathcal{T}_h} \overline{\sigma}_h(f_0,\mathcal{T}_h),
\]
\index{Riemannin!d@ylä- ja alaintegraali}%
ja integraalin arvo = näiden \kor{ala}- ja \kor{yläintegraalien} yhteinen arvo $I(f,A,\mu)$.

Molemmista määrittelytavoista nähdään, että $\int_T f_0\,d\mu$ ei riipu suorakulmion $T$
valinnasta sikäli kuin $T \supset A$ (Harj.teht.\,\ref{H-uint-1: T:n valinta}). --- Tämä tulos
ennakoitiin jo edellä, kun sovittiin: $\,\int_A f\,d\mu = \int_T f_0\,d\mu,\ T \supset A$.

\subsection{Lineaarisuus. Additiivisuus. Vertailuperiaate}

Jos merkitään rajoitetun joukon A yli integroituvien funktioiden joukkoa symbolilla
$\mathcal{R}_A$, niin seuraavat yleiset ominaisuudet ovat yhteisiä tasointegraaleille ja 
yksiulotteisille määrätyille integraaleille (ks.\ Lauseet
\ref{integraalin additiivisuus}--\ref{integraalien vertailuperiaate}). Nämä ovat voimassa myös
jatkossa määriteltäville muille integraalityypeille.
\index{lineaarisuus!b@integroinnin}
\index{additiivisuus!a@integraalin}
\index{vertailuperiaate!a@integraalien}%
\[ 
\boxed{ \begin{aligned}
\ykehys\quad &\text{Lineaarisuus}\,: \qquad 
f,g \in \mathcal{R}_A\ \impl\ \alpha f + \beta g \in \mathcal{R}_A \quad
                                         \forall\ \alpha,\beta \in \R \quad \text{ja} \\
&\phantom{\text{Lineaarisuus}\,:} \qquad 
\int_A (\alpha f + \beta g)\,d\mu = \alpha\int_A f\,d\mu + \beta\int_A g\,d\mu. \\[2mm]
&\text{Additiivisuus}\,: \quad 
f\in\mathcal{R}_A\ \ja\ f\in\mathcal{R}_B\ \ja\ A \cap B = \emptyset \\
&\phantom{\text{Additiivisuus}\,:} \quad \impl\ \int_{A \cup B} f\,d\mu 
                                              = \int_A f\,d\mu + \int_B f\,d\mu. \\[2mm]
&\text{Vertailuperiaate}\,: \quad 
f,g \in \mathcal{R}_A\ \ja\ f \le g\,\ A\text{:ssa}\ \impl\ \int_A f\,d\mu \le \int_A\,g\,d\mu. 
        \quad\Akehys \end{aligned} } 
\]
Lineaarisuus ja vertailuperiaate seuraavat suoraan määritelmästä, kuten yksi\-ulotteisessa
tapauksessa (ks.\ Luku \ref{riemannin integraali}). Additiivisuus on johdettavissa
yksinkertaisimmin lineaarisuudesta: Kun merkitään $f$:n nollajatkoja joukkojen $A$, $B$ ja
$A \cup B$ suhteen vastaavasti symboleilla $f_A$, $f_B$ ja $f_0$, niin pätee
\[
A \cap B = \emptyset \qimpl f_0=f_A+f_B.
\]
Näin ollen jos $T=[a_1,b_1]\times[a_2,b_2] \supset A \cup B$, niin lineaarisuuden nojalla
\[
\int_T f_0\,d\mu = \int_T f_A\,d\mu + \int_T f_B\,d\mu
   \qekv \int_{A \cup B} f\,d\mu = \int_A f\,d\mu + \int_B f\,d\mu.
\]

\subsection{Rajoitetun joukon mitta}

Em.\ määritelmissä on integraalin riippuvuus mitasta varsin yksinkertainen, sillä määritelmissä
tarvitaan vain suorakulmioiden $T_{kl}$ mittoja aksiooman A1 mukaisesti. Mutta kun integraali on
kerran määritelty käyttäen hyväksi mitan yksinkertaisimpia ominaisuuksia, voidaankin määritellä
yleisemmän joukon $A$ mitta käyttäen hyväksi integraalia (!).
Menettely on seuraava: Tarkastellaan funktiota
\[
f(x,y)=1,\quad (x,y)\in A.
\]
\index{karakteristinen funktio}%
Tämän nollajatkoa sanotaan $A$:n \kor{karakteristiseksi funktioksi} ja merkitään
\[
\chi_A(x,y)=\begin{cases} 
            \,1, &\text{kun}\ (x,y) \in A, \\ \,0, &\text{kun}\ (x,y) \notin A. 
            \end{cases}
\]
Sikäli kuin $\chi_A$ on Riemann-integroituva yli suorakulmion $T\supset A$, on em.\ 
määritelmien mukaan
\[
\int_A d\mu=\int_T \chi_A\,d\mu, \quad T=[a_1,b_1]\times[a_2,b_2]\supset A.
\]
\begin{Def} (\vahv{Jordan-mitta}) \label{Jordan-mitta} \index{pinta-alamitta!a@tason|emph}
\index{Jordan-mitta!a@pinta-alamitta|emph} \index{mitta, mitallisuus!a@Jordan-mitta|emph} 
Rajoitettu joukko $A\subset\R^2$ on \kor{Jordan-mitallinen} täsmälleen kun $A$:n
karakteristinen funktio $\chi_A$ on Riemann-integroituva yli suorakulmioiden
$T=[a_1,b_1]\times[a_2,b_2]\supset A$ ja $A$:n \kor{Jordan-mitta} on tällöin
\[
\boxed{\kehys\quad \mu(A)=\int_A d\mu. \quad}
\]
\end{Def}
Mitta-aksioomien A1--A3 toteutuminen on tämän määritelmän ja em.\ yleisten integraalin
ominaisuuksien perusteella ilmeistä.

Kaikki rajoitetut joukot eivät ole mitallisia (ks.\ Harj.teht.\,\ref{H-uint-1: esimerkkejä}a),
mutta jokaiselle rajoitetulle joukolle $A\subset\R^2$ voidaan määritellä
\index{ulkomitta (Jordanin)} \index{siszy@sisämitta (Jordanin)}
\begin{align*}
&\text{\kor{ulkomitta}:} \quad 
            \overline{\mu}(A)=\inf_{\mathcal{T}_h}\overline{\sigma}_h(\chi_A,\mathcal{T}_h), \\
&\text{\kor{sisämitta}:} \quad 
            \underline{\mu}(A)=\sup_{\mathcal{T}_h}\underline{\sigma}_h(\chi_A,\mathcal{T}_h),
\end{align*}
missä $\inf$ ja $\sup$ lasketaan yli kaikkien perussuorakulmion $T\supset A$ jakojen. Joukko
$A$ on mitallinen täsmälleen kun $\overline{\mu}(A)=\underline{\mu}(A)$.
\begin{figure}[H]
\begin{center}
\input{kuvat/kuvaUint-3.pstex_t}
\end{center}
\end{figure}
Kuviossa on
\begin{align*}
\mu(\raisebox{-0.1cm}{\epsfig{file=kuvat/kuvaUint-4a.eps}}) 
                      &= \underline{\sigma}_h(\chi_A,\mathcal{T}_h)
                       =\sum_{T_{kl}\in\mathcal{T}_h:\,T_{kl}\subset A} \mu(T_{kl}), \\
\mu(\raisebox{-0.1cm}{\epsfig{file=kuvat/kuvaUint-4b.eps}}) 
                      &= \overline{\sigma}_h(\chi_A,\mathcal{T}_h)
                         -\underline{\sigma}_h(\chi_A,\mathcal{T}_h).
\end{align*}
\begin{Exa} $A=$ jana, jonka päätepisteet ovat $(0,0)$ ja $(2,1)$. Näytä: $\mu(A)=0$.
\end{Exa}
\begin{multicols}{2}
\ratk Jaetaan $T=[0,2]\times[0,1] \supset A$ suorakulmioihin kokoa $h \times h/2$, missä
$2/h=n\in\N$. Tällöin on (vrt.\ kuvio)
\[
\overline{\sigma}_h(\chi_A,\mathcal{T}_h) = \frac{2}{h}\cdot\frac{h^2}{2} = h.
\]
Näin ollen $\overline{\mu}(A) \le h=2/n\ \forall n\in\N$, joten $\overline{\mu}(A)=0$ ja siis
$\mu(A)=0$. \loppu
\begin{figure}[H]
\setlength{\unitlength}{1cm}
\begin{center}
\begin{picture}(4,4)
\put(0,1){\vector(1,0){5}} \put(0,1){\vector(0,1){2.5}}
\put(5.3,0.9){$x$} \put(-0.1,3.8){$y$}
\put(3.95,0.6){$\scriptstyle{2}$} \put(-0.4,2.9){$\scriptstyle{1}$}
\put(0.45,0.6){$\scriptstyle{h}$} \put(-0.6,1.2){$\scriptstyle{h/2}$}
\put(0,3){\line(1,0){4}} \put(4,1){\line(0,1){2}}
\multiput(0,1)(0.5,0.25){8}{\line(1,0){0.5}}
\multiput(0,1.25)(0.5,0.25){8}{\line(1,0){0.5}}
\multiput(0,1)(0.5,0.25){8}{\line(0,1){0.25}}
\multiput(0.5,1)(0.5,0.25){8}{\line(0,1){0.25}}
\thicklines
\put(0,1){\line(2,1){4}}
\end{picture}
\end{center}
\end{figure}
\end{multicols}

Mainittakoon vielä seuraava lause, jonka osittainen todistus jätetään harjoitustehtäväksi
(Harj.teht.\,\ref{H-uint-1: integroituvuus L-ehdolla}). Lauseen jatkuvuusoletus
(jota harjoitustehtävässä on vahvistettu) viittaa Määritelmään
\ref{jatkuvuus kompaktissa joukossa - Rn}.
\begin{*Lause} \label{jatkuvan funktion integroituvuus tasossa}
\index{Riemann-integroituvuus!b@jatkuvan funktion|emph}
\index{Riemann-integroituvuus!c@tasossa|emph}
Jos $f$ on jatkuva suorakulmiossa $T=[a,b]\times[c,d]$ ja $A \subset T$ on mitallinen, niin
$f$ on Riemann-integroituva yli $A$:n.
\end{*Lause} 

\subsection{Siirto-, peilaus- ja kiertoinvarianssi}
\index{Jordan-mitta!a@pinta-alamitta|vahv}
\index{siirtoinvarianssi (mitan)|vahv}
\index{peilausinvarianssi (mitan)|vahv}
\index{kiertoinvarianssi (mitan)|vahv}

Jotta Jordan-mitta vastaisi tavanomaista geometriasta tunnettua pinta-alaa, on mitan oltava
\kor{siirto}-, \kor{peilaus} ja  \kor{kiertoinvariantti}. Tällä tarkoitetaan, että jos $A$ on
mitallinen ja $A'$ on $A$:n kanssa (geometrisesti)
\index{yhtenevyys}%
\kor{yhtenevä}, eli $A'=\mf(A)$, missä joillakin $a,b,\theta\in\R$ on
\[
\mf(x,y)=(a+x\cos\theta \mp y\sin\theta,\,b+x\sin\theta \pm y\cos\theta),
\]
\begin{figure}[H]
\begin{center}
\input{kuvat/kuvaUint-7.pstex_t}
\end{center}
\end{figure}
niin $\mu(A')=\mu(A)$. Kuvaus $\mf:A \kohti A'$ on affiinimuunnoksen erikoistapaus, joka
\index{siirto (translaatio)} \index{kierto!a@geom.\ kuvaus} \index{peilaus}%
koostuu \kor{siirrosta} (siirtovektori $\vec r_0=a\vec i + b\vec j\,$), \kor{kierrosta}  
(kiertokulma $\theta$) ja mahdollisesta \kor{peilauksesta} (merkin vaihtelu). Yhdistettyä
siirtoa ja kiertoa (kuva) sanotaan
\index{euklidinen liike}%
\kor{euklidiseksi liikkeeksi}.  

Siirto- ja peilausinvarianssi ovat mitan $\mu$ määritelmästä varsin ilmeisiä. Kierron suhteen 
sen sijaan voisi määritelmän koordinaattiriippuvuuden epäillä aiheuttavan ongelmia. Tällaisia
ei todellisuudessa ole, vaan pätee
(ks.\ Harj.teht.\,\ref{H-uint-1: kiertoinvarianssi a},\ref{H-uint-1: kiertoinvarianssi b})
\begin{Lause}
Jordanin pinta-alamitta on siirto-, peilaus- ja kiertoinvariantti, ts. geometrisesti yhtenevien
joukkojen mitat ovat samat.
\end{Lause}

\subsection{Keskipistesääntö}
\index{keskipistesääntö|vahv}

Kuten yhden muuttujan integraaleja, myös tasointegraaleja voi laskea numeerisesti suoraan 
määritelmästä käsin, eli approksimoimalla integraalia Riemannin summalla:
\[ 
\int_A f\, d\mu = \int_T f_0\, d\mu 
    \,\approx\, \sum_{k=1}^m \sum_{l=1}^n f_0(\xi_{kl},\eta_{kl})\mu(T_{kl}), \quad T \supset A.
\]
Jos tässä valitaan $(\xi_{kl},\eta_{kl})=T_{kl}$:n keskipiste, niin approksimaatiota sanotaan
\index{yhdistetty!a@keskipistesääntö}%
\kor{yhdistetyksi keskipistesäännöksi}. Kuten yhdessä dimensiossa (vrt.\ Luku 
\ref{numeerinen integrointi}), tämä on Riemannin summiin perustuvista approksimaatioista
yleensä tarkin.
\begin{Exa} Jos $A = [a,b] \times [c,d]$, niin integraali $\int_A x\,d\mu$ saadaan lasketuksi
jakamalla $A\ $ $m \times n$ samankokoiseen suorakulmioon ja käyttämällä yhdistettyä
keskipistesääntöä:
\begin{align*}
\int_A x\,d\mu\ 
   &\approx\ \sum_{k=1}^m\sum_{l=1}^n 
    \left[a+\frac{(k-\tfrac{1}{2})(b-a)}{m}\right]\frac{b-a}{m}\cdot\frac{c-d}{n} \\
   &=\, (b-a)(c-d)\left[a + \frac{b-a}{m^2}\sum_{k=1}^m (k-\tfrac{1}{2})\right] \\
   &=\, (b-a)(c-d)[a+\tfrac{1}{2}(b-a)] \,=\, \tfrac{1}{2}(a+b)\mu(A).
\end{align*}
Koska tulos ei riipu parametreista $m,n$, on oltava $\int_A x\,d\mu = \tfrac{1}{2}(a+b)\mu(A)$.
Vastaavalla tavalla laskien saadaan $\int_A y\,d\mu = \tfrac{1}{2}(c+d)\mu(A)$. \loppu
\end{Exa}
Esimerkissä saatiin jo yhdellä suorakulmiolla ($m=n=1,\ T_{11}=A$) tarkka tulos. Yleisemmin
keskipistesääntö integroi suorakulmiossa tarkasti ensimmäisen asteen polynomin:
\begin{samepage}
\begin{multicols}{2}
\begin{align*}
f(x,y) &= a_0 + a_1 x + a_2 y,\\
& \quad (a_0,a_1,a_2 \in \R) \\
       &\impl \ \int_T f\, d\mu=f(x_0,y_0)\mu(T).
\end{align*}
\vfill\null
\columnbreak

\begin{figure}[H]
\setlength{\unitlength}{1cm}
\begin{center}
\begin{picture}(-3,0.5)(4,2)
\path(0,0)(4,0)(4,2)(0,2)(0,0)
\put(1.9,0.9){$\bullet$} \put(2.2,0.9){$(x_0,y_0)$}
\put(1.9,2.2){$T$}
\end{picture}
\end{center}
\end{figure}
\end{multicols}
\end{samepage}

Tämän ominaisuuden ja kahden muuttujan Taylorin kaavan
(Luku \ref{usean muuttujan taylorin polynomit}) perusteella keskipistesäännölle on johdettavissa 
virhearvio samaan tapaan kuin yhdessä dimensiossa (Harj.teht.\,\ref{H-uint-1: kp-virhearvio}).
\begin{Exa} \label{keskipistesääntö tasossa} Olkoon $A = [0,1] \times [0,1]$ ja laskettavana
integraali $\int_A x^2 y^2\,d\mu$. Kun $A$ jaetaan neliöihin kokoa $h \times h$ ja kussakin 
neliössä integroimispisteeksi valitaan (a) vasen alanurkka, (b) keskipiste, niin integraalille
saadaan seuraavat likiarvot (tarkka arvo $=1/9\,$; taulukkoon on merkitty myös summattavien 
termien lukumäärä $N=h^{-2}$)\,:
\begin{center}
\begin{tabular}{llll}
$h$    & $N$    & Likiarvo (a)  & Likiarvo (b) \\  \hline & \\
0.1    & $10^2$ & 0.081225000.. & 0.110556250.. \\
0.01   & $10^4$ & 0.107813722.. & 0.111105555.. \\
0.001  & $10^6$ & 0.110778138.. & 0.111111055.. \\
0.0001 & $10^8$ & 0.111077781.. & 0.111111110.. \qquad\qquad\loppu
\end{tabular}
\end{center}
\end{Exa}
Jos joukko $A$ on yleisempi joukko kuin suorakulmio, niin Riemannin summiin perustuva
numeerinen approksimaatio
\[ 
\int_A f\,d\mu = \int_T f_0\,d\mu 
                 \approx \sum_{k=1}^m \sum_{l=1}^n f_0(\xi_{kl},\eta_{kl})\mu(T_{kl}) 
\]
\begin{multicols}{2} \raggedcolumns
ei välttämättä ole kovin hyvä, syystä että $f$:n nollajatko $f_0$ on (yleensä) epäjatkuva 
sellaisissa suorakulmioissa $T_{kl}$, jotka leikkaavat reunaviivaa $\partial A$ (kuvio).
%\begin{multicols}{2} \raggedcolumns
Tällöin arvio
\[
\int_{T_{kl}} f_0\, d\mu\approx f_0(\xi_{kl},\eta_{kl})\mu(T_{kl})
\]
on pääsääntöisesti kehno, valittiinpa $(\xi_{kl},\eta_{kl}) \in T_{kl}$ miten hyvänsä.
\begin{figure}[H]
\begin{center}
\input{kuvat/kuvaUint-10.pstex_t}
\end{center}
\end{figure}
\end{multicols}

\begin{multicols}{2} \raggedcolumns
Jos halutaan parantaa algoritmia, niin on syytä käyttää tarkempia approksimaatioita
integraaleille
\[
\int_{T_{kl}} f_0\, d\mu=\int_{T_{kl}\cap A} f\, d\mu
\]
silloin kun $T_{kl}\cap\partial A\neq\emptyset$. Esimerkiksi jakoa voidaan paikallisesti
tihentää ja käyttää samaa algoritmia uudelleen ($T$:n tilalla $T_{kl}$, $A$:n tilalla
$A \cap T_{kl}$).
\begin{figure}[H]
\begin{center}
\input{kuvat/kuvaUint-11.pstex_t}
\end{center}
\end{figure}
\end{multicols}

\subsection{$\R$:n pituusmitta}
\index{pituusmitta (Jordanin)|vahv}
\index{Jordan-mitta!b@$\R$:n pituusmitta|vahv}
\index{mitta, mitallisuus!a@Jordan-mitta|vahv}

Määriteltäessä $\R$:n (Jordanin) \kor{pituusmitta} kirjoitetaan aksiooman A1 tilalle
\begin{itemize}
\item[A1.] \kor{Suljetun välin mitta}: $\,\ A=[a,b]: \,\ \mu(A)=b-a$.
\end{itemize}
Muut aksioomat säilyvät ennallaan. Jos $A\subset\R$ on rajoitettu joukko, niin määritellään
\[
\int_A f\,d\mu = \int_T f_0\,d\mu, \quad T=[a,b] \supset A,
\]
missä $f_0$ on $f$:n nollajatko, ja tässä edelleen
\[
\int_T f_0\,d\mu = \Lim_{h \kohti 0} \sum_{k=1}^n f_0(\xi_k)\mu(T_k),
\]
missä $\,T_k=[x_{k-1},x_k]$, $\,\xi_k \in T_k\,$ ja $\,a=x_0<x_1 < \ldots <x_n=b$. Koska
$\mu(T_k)=x_k-x_{k-1}$, niin tässä $\int_T f_0\,d\mu=\int_a^b f_0(x)\,dx\,$ (Määritelmä
\ref{Riemannin integraali}). Määrätyssä (Riemannin) integraalissa on siis kyse integroimisesta
$\R$:n pituusmitan suhteen. Rajoitetun joukon $A\subset\R$ Jordan-mitta määritellään kuten
edellä, eli integraalin avulla:
\[
\mu(A)=\int_A d\mu.
\]
\begin{Exa} Jos $A=[0,1]\cup[2,4]$ niin integraalin additiivisuuden (tai suoraan määritelmän)
perusteella
\[
\int_A f\,d\mu = \int_{[0,1]} f\,d\mu+\int_{[2,4]} f\,d\mu 
               = \int_0^1 f(x)\,dx+\int_2^4 f(x)\,dx.
\]
Erityisesti on $\mu(A)=\int_A d\mu = 1+2=3$. \loppu
\end{Exa}

\Harj
\begin{enumerate}

\item \label{H-uint-1: T:n valinta}
Näytä, että jos $T_1$ ja $T_2$ ovat joukon $A\subset\R^2$ sisältäviä perussuorakulmioita, niin
samoin on $T_1 \cap T_2$. Päättele tästä, että määritelmä $\int_A f\,d\mu=\int_T f_0\,d\mu$
antaa saman tuloksen perussuorakulmion $T \supset A$ valinnasta riippumatta.

\item \label{H-uint-1: esimerkkejä}
a) Näytä, että joukko $A=([0,1]\cap\Q)\times[0,1]$ ei ole Jordan-mitallinen. \newline
b) Anna esimerkki joukosta $A$, jolle pätee $A\subset[0,1]\times[0,1]$, $\overline{\mu}(A)=1$
ja $\underline{\mu}(A)=0.99$.

\item \label{H-uint-1: väittämiä}
Todista seuraavat rajoitettuja joukkoja $A,B\subset\R^2$ koskevat
väittämät:  \vspace{1mm}\newline 
a) Jos $A \subset B$, niin $\underline{\mu}(A)\le\underline{\mu}(B)$ ja
$\overline{\mu}(A)\le\overline{\mu}(B)$. \vspace{1mm}\newline
b) Jos $\mu(B)=0 $, niin $\underline{\mu}(A \cup B)=\underline{\mu}(A)$ ja
$\overline{\mu}(A \cup B)=\overline{\mu}(A)$. \vspace{1mm}\newline
c) Jos $\mu(A \cap B)=0$, niin 
$\underline{\mu}(A \cup B)=\underline{\mu}(A)+\underline{\mu}(B)$ ja
$\overline{\mu}(A \cup B)=\overline{\mu}(A)+\overline{\mu}(B)$.

\item 
Laske joukon $A = \{(x,y)\in\R^2 \mid x \in [0, 1]\,\ja\, 0 \le y \le x^2\}$ Jordanin 
ulko-ja sisämitalle approksimaatiot jakamalla $T = [0, 1] \times [0, 1]$ suorakulmioihin kokoa
$h \times h^2\ (h^{-1}\in\N)$ ja näiden perusteella mitta $\mu(A)$.

\item
Olkoon $A\subset\R^2$ rajoitettu joukko ja $f$ ja $g$ määriteltyjä ja rajoitettuja $A$:ssa.
Näytä, että jos $f(x,y) \le g(x,y)\ \forall (x,y) \in A$, niin
$\overline{I}(f,\mu,A)\,\le\,\underline{I}(g,\mu,A)$. 

\item \label{H-uint-1: kiertoinvarianssi a}
a) Olkoon $A=$ suorakulmainen kolmio, jonka kärjet ovat pisteissä $(0,0)$, $(a, 0)$ ja $(0, b)$.
Näytä Jordan-mitan määritelmästä, että $\mu(A) = \frac{1}{2} |a| |b|$. \vspace{1mm}\newline
b) Vedoten a-kohdan tulokseen ja mitan additiivisuuteen päättele, että suorakulmion mitta on 
kiertoinvariantti.

\item 
Laske yhdistettyä keskipistesääntöä käyttäen integraali $\int_A (1+x-2 y)\,d\mu$, kun 
$A\subset\R^2$ on $T$:n muotoinen alue, jonka nurkkapisteet ovat $(2,0)$, $(3,0)$, $(3,3)$, 
$(5,3)$, $(5,4)$, $(0,4)$, $(0,3)$ ja $(2,3)$.

\item \label{H-uint-1: toisen asteen integraalit}
Olkoon $A=[-a/2,a/2]\times[-b/2,b/2]$. Jakamalla $A$ samankokoisiin suorakulmioihin ja 
käyttämällä yhdistettyä keskipistesääntöä näytä oikeaksi:
\[
\int_A x^2\,d\mu=\frac{1}{12}\,a^3b, \quad
\int_A y^2\,d\mu=\frac{1}{12}\,ab^3, \quad
\int_A \abs{xy}\,d\mu=\frac{1}{16}\,a^2b^2.
\]

\item 
Olkoon $A = [0,1] \times [0,1]$. Laske seuraavat integraalit jakamalla $A$ neliöihin kokoa 
$h \times h$ ($h^{-1}\in\N$), käyttämällä yhdistettyä keskipistesääntöä, ja laskemalla tuloksen
raja-arvo, kun $h \to 0$.
\[
\text{a)}\ \ \int_A xy^2\,d\mu \qquad
\text{b)}\ \ \int_A x^2y^2\,d\mu \qquad
\text{c)}\ \ \int_A e^{x+y}\,d\mu
\]

\item (*)
Luvussa \ref{pinta-ala ja kaarenpituus} väitettiin, että jos
\[
A = \{\,(x,y) \in \R^2 \mid x \in [a,b]\ \ja\ 0 \le y \le f(x)\,\},
\]
missä $f$ on rajoitettu, ei-negatiivinen ja Riemann-integroituva välillä $[a,b]$, niin
$\,\mu(A)=\int_a^b f(x)\,dx$. Todista väittämä uudelleen näyttämällä, että
\[ 
\underline{\mu}(A) = \underline{\int_a^b} f(x)\,dx, \quad 
\overline{\mu}(A) = \overline{\int_a^b} f(x)\,dx 
\]
ja vetoamalla Lauseeseen \ref{Riemann-integroituvuus}.

\item (*) \label{H-uint-1: integroituvuus L-ehdolla}
Todista Lauseen \ref{jatkuvan funktion integroituvuus tasossa} väittämä, kun jatkuvuuden
sijasta oletetaan, että $f$ toteuttaa Lipschitz-jatkuvuusehdon
\[
\abs{f(x_1,y_1)-f(x_2,y_2)} \le L(\abs{x_1-x_2}+\abs{y_1-y_2}), \quad 
                                              (x_1,y_1),\,(x_2,y_2) \in T.
\]
\kor{Vihje}: Vertaa ylä- ja alasummia $\overline{\sigma}(f_0,\mathcal{T}_h)$ ja
$\underline{\sigma}(f_0,\mathcal{T}_h)$ ($\mathcal{T}_h=T$:n ositus).

\item (*) \label{H-uint-1: fplus ja fmiinus}
Olkoon $A\subset\R^2$ rajoitettu joukko ja $f$ määritelty $A$:ssa. Määritellään
\[
f_+(x,y) = \begin{cases} 
            \,f(x,y), &\text{kun}\ f(x,y)>0, \\ 
            \,0,      &\text{muulloin}. 
           \end{cases}
\]
a) Näytä: $\ \overline{I}(f_+,A)-\underline{I}(f_+,A) \le \overline{I}(f,A)-\underline{I}(f,A)$.
\vspace{1mm}\newline
b) Päättele: Jos $f$ on Riemann-integroituva yli $A$:n, niin samoin ovat $f_+$, $f_-=f-f_+$ ja
$|f|$. 

\item (*) \label{H-uint-1: kiertoinvarianssi b}
Lähtien tehtävän \ref{H-uint-1: kiertoinvarianssi a} tuloksesta todista: Jos $A\subset\R^2$ on
rajoitettu, niin $\underline{\mu}(A)$ ja $\overline{\mu}(A)$ ovat kiertoinvariantteja.
Päättele, että jos $A$ on Jordan-mitallinen, niin $\mu(A)$ on kiertoinvariantti.

\item (*)
Integraalille $\int_A f\,d\mu$, missä $A$ on kolmio, jonka kärjet ovat $(0,0)$, $(1,0)$ ja
$(0,1)$, ja $f(x,y)=x+2y$, lasketaan likiarvo jakamalla $T=[0,1]\times[0,1] \supset A$
neliöihin kokoa $h \times h$ ja laskemalla $\int_T f_0\,d\mu$ yhdistetyllä keskipistesäännöllä.
a) Näytä, että tuloksena on ylälikiarvo, jonka virhe $=\tfrac{3}{2}h+\Ord{h^2}$. \newline
b) Mikä on integraalin tarkka arvo? 

\item (*) \label{H-uint-1: kp-virhearvio}
Olkoon $f$:n osittaisderivaatat toiseen kertalukuun asti jatkuvia suorakulmiossa
$A=[-a/2,a/2]\times[-b/2,b/2]$. Käyttämällä Taylorin lausetta, integraalien vertailuperiaatetta
ja tehtävän \ref{H-uint-1: toisen asteen integraalit} tuloksia näytä oikeaksi keskipistesäännön
virhearvio
\[
\Bigl|\int_A f\,d\mu - f(0,0)\,ab\,\Bigr|\,\le\,\frac{ab}{24}(M_{11}a^2+M_{22}b^2)
                                                +\frac{1}{16}\,M_{12}\,a^2b^2,
\]
missä $M_{11}$, $M_{22}$ ja $M_{12}$ ovat osittaisderivaattojen $f_{xx}$, $f_{yy}$ ja $f_{xy}$
itseisarvojen maksimiarvot $A$:ssa. Näytä edelleen, että jos $A$ jaetaan suorakulmioihin
kokoa $h_1 \times h_2\ (a/h_1\in\N,\ b/h_2\in\N)$, niin yhdistetylle keskipistesäännölle
pätee virhearvio
\[
\abs{E(f)}\,\le\,\frac{ab}{24}\left(M_{11}h_1^2+M_{22}h_2^2+\frac{3}{2}\,M_{12}\,h_1h_2\right).
\]
Mikä $E(f)$ on tarkasti, jos $f$ on toisen asteen polynomi ja tunnetaan osittaisderivaattojen
$f_{xx}$, $f_{yy}$ ja $f_{xy}$ (vakio)arvot $F_{11}$, $F_{22}$ ja $F_{12}\,$?

\end{enumerate} %Pinta-alamitta ja tasointegraalit
\section{Tasointegraalien laskutekniikka} \label{tasointegraalien laskutekniikka}
\alku
\index{tasointegraali|vahv}

Tasointegraali on mahdollistaa palauttaa yhden muuttujan integrointitehtäviksi, jolloin 
integraali voidaan suotuisissa oloissa laskea käyttäen hyväksi integraalifunktioita, eli 
suljetussa muodossa. Tässä luvussa tarkastellaan tällaista laskutekniikkaa.

\index{x@$x$-projisoituva (joukko)}%
Tarkastellaan joukkoa $A\subset\R^2$ joka on nk. \kor{$x$-projisoituva}, eli muotoa
\[
A=\{(x,y)\in\R^2 \ | \ x\in [a,b], \ g_1(x)\leq y\leq g_2(x)\},
\]
missä oletetaan, että $g_1$ ja $g_2$ ovat välillä $[a,b]$ rajoitettuja. Olkoon 
$T=[a,b]\times [c,d]$ suorakulmio, joka sisältää $A$:n, ja olkoon $\mathcal{T}_h$ jokin $T$:n
jako.
\begin{figure}[H]
\begin{center}
\import{kuvat/}{kuvaUint-13.pstex_t}
\end{center}
\end{figure}
Olkoon $f$ määritelty $T$:ssä ja rajoitettu, ja tarkastellaan integraalin $\int_A f\, d\mu$
määrittävää Riemannin summaa. Olkoon $\mathcal{T}_h=\{T_{kl},\ k=1 \ldots m,\ l=1 \ldots n\}$
ja valitaan
\[
(\xi_{kl},\eta_{kl})=(\xi_k,\eta_l)\in T_{kl},\quad k=1 \ldots m,\ l=1 \ldots n.
\]
Tällöin summa muodon
\[
\sum_{T_{kl}\in\mathcal{T}_h} f_0(\xi_{kl},\eta_{kl})\mu(T_{kl})
= \sum_{k=1}^m\sum_{l=1}^n f_0(\xi_k,\eta_l)(x_k-x_{k-1})(y_l-y_{l-1}).
\]
Sikäli kuin funktio $y \map f_0(\xi_k,y)$ on Riemann-integroituva välillä $[c,d]$, on
sisemmällä summalla raja-arvo
\[
\sum_{l=1}^n f_0(\xi_k,\eta_l)(y_l-y_{l-1})\ 
                  \underset{h\kohti 0}{\Kohti}\ \int_c^d f_0(\xi_k,y)\,dy
                         = \int_{g_1(\xi_k)}^{g_2(\xi_k)} f(\xi_k,y)\,dy.
\]
Kun merkitään
\[
F(x)=\int_{g_1(x)}^{g_2(x)} f(x,y)\, dy,\quad x\in [a,b],
\]
niin päätellään, että sikäli kuin $F$ on edelleen Riemann-integroituva välillä $[a,b]$, niin
jaon $\mathcal{T}_h$ ollessa riittävän tiheä on likimäärin
\[
\sum_{T_{kl}\in\mathcal{T}_h} f_0(\xi_k,\eta_l)\mu(T_{kl}) 
            \approx \sum_{l=1}^n F(\xi_k)(x_k-x_{k-1}) \approx \int_a^b F(x)\,dx.
\]
Näyttää siis ilmeiseltä, että funktioiden $g_1$, $g_2$ ja $f$ ollessa riittävän säännöllisiä
pätee
\[
h\kohti 0 \ \impl \ 
            \sum_{T_{kl}\in\mathcal{T}_h} f_0(\xi_k,\eta_l)\mu(T_{kl})\kohti\int_a^b F(x)\,dx.
\]
Näin on johdettu (tai ainakin tehty uskottavaksi) laskukaava
\begin{equation} \label{tasopalautus-x}
\boxed{ \begin{aligned}
\ykehys\quad&A=\{(x,y)\in\R^2 \ | \ x\in [a,b], \ g_1(x)\leq y\leq g_2(x)\} \quad\\
            &\impl \quad \int_A f\,d\mu 
                         = \int_a^b\left[\int_{g_1(x)}^{g_2(x)} f(x,y)\,dy\right]dx. \quad
           \end{aligned} } 
\end{equation}
Tämän mukaan siis tasointegraali yli $x$-projisoituvan joukon on laskettavissa peräkkäisinä
\index{iteroitu integraali}%
yksiulotteisina, nk.\ \kor{iteroituina integraaleina}. Kyseessä on \kor{Fubinin lauseen} 
nimellä tunnetun väittämän eräs muoto. Lause muotoillaan täsmällisemmin ja todistetaan luvun 
lopussa.

\index{y@$y$-projisoituva joukko}%
Jos $A\subset\R^2$ on \kor{$y$-projisoituva}, ts. (vrt. kuvio vasemmalla)
\[
A=\{(x,y)\in\R^2 \ | \ y\in [a,b], \ g_1(y)\leq x \leq g_2(y)\}, \\[3mm]
\]
%\vspace{1mm}
\begin{multicols}{2} \raggedcolumns
\begin{figure}[H]
\begin{center}
\input{kuvat/kuvaUint-14.pstex_t}
\end{center}
\end{figure}
\begin{figure}[H]
\begin{center}
\input{kuvat/kuvaUint-15.pstex_t}
\end{center}
\end{figure}
\end{multicols}
niin integrointikaavassa \eqref{tasopalautus-x} on vaihdettava $x$ ja $y$, jolloin kaava saa
muodon
\begin{equation} \label{tasopalautus-y}
\int_A f\,d\mu=\int_a^b\left[\int_{g_1(y)}^{g_2(y)} f(x,y)\,dx\right]dy.
\end{equation}
Jos $A$ ei ole kumpaakaan projisoituvaa tyyppiä, niin kaavoja \eqref{tasopalautus-x} ja 
\eqref{tasopalautus-y} päästään yleensä käyttämään, kun $A$ ensin jaetaan sopiviin osiin 
(vrt.\ kuvio edellä) ja käytetään integraalin additiivisuutta:
\[
\int_A f\,d\mu=\sum_{i=1}^m \int_{A_i} f\,d\mu,\quad A
              =\left(\bigcup_{i=1}^m A_i\right)\cup B, \quad \mu(A_i\cap A_j)
              =\mu(B)=0,\ i\neq j.
\]

Iteraatiokaavojen \eqref{tasopalautus-x}--\eqref{tasopalautus-y} perusteella voidaan 
tasointegraalikin laskea suotuisissa oloissa integraalifunktioiden avulla. Olkoon esimerkiksi
$A$ $x$-projisoituva ja oletetaan, että on löydettävissä funktion $f(x,y)$ integraalifunktio
muuttujan $y$ suhteen, ts. funktio $F(x,y)$, jolle pätee
\[
\frac{\partial}{\partial y} F(x,y)=f(x,y),\quad (x,y)\in A.
\]
Tällöin kaavan \eqref{tasopalautus-x} mukaan
\[
\int_A f(x,y)\,d\mu = \int_a^b\left[\sijoitus{y=g_1(x)}{y=g_2(x)}F(x,y)\right]dx
                    =\int_a^b G(x)\,dx,
\]
missä
\[
G(x)=F(x,g_2(x))-F(x,g_1(x)).
\]
Jos edelleen on löydettävissä funktion $G$ integraalifunktio $H$, niin integraali saadaan
lasketuksi suljetussa muodossa:
\[
\int f\,d\mu=\sijoitus{a}{b} H(x)=H(b)-H(a).
\]
Laskun onnistuminen tällä tavoin riippuu siis paitsi funktiosta $f$, myös funktioista $g_1$ ja
$g_2$, eli $A$:n reunaviivan muodosta. Onnistuminen on taattu esimerkiksi silloin, kun $f$, 
$g_1$ ja $g_2$ ovat polynomeja, sillä tällöin myös $F$, $G$ ja $H$ ovat polynomeja.

Ajatellen iteraatiokaavoja \eqref{tasopalautus-x}--\eqref{tasopalautus-y} käytetään 
tasointegraaleille merkintätapoja
\[
\int_A f\, dxdy\quad\text{tai} \quad \iint_A f\, dxdy.
\]
Näissä siis kirjoitetaan $d\mu=dxdy$, mikä merkintä viittaa 'differentiaalisen suorakulmion' 
$\,[x,x+dx]\times[y,y+dy]$ pinta-alaan. Merkintää käytetään jatkossa.
\begin{Exa} Olkoon $A=[0,1]\times[0,1]$. Laske $\int_A \sqrt{\abs{x-y}}\,dxdy$. \end{Exa}
\ratk \ Kaavan \eqref{tasopalautus-x} mukaan
\begin{align*}
\int_A \sqrt{\abs{x-y}}\,dxdy 
&= \int_0^1 \left[ \int_0^1 \sqrt{\abs{x-y}}\,dy \right] dx \\
&= \int_0^1 \left[ \int_0^x \sqrt{x-y}\,dy + \int_x^1 \sqrt{y-x}\, dy \right] dx \\
&= \int_0^1 \left[ -\frac{2}{3}\sijoitus{y=0}{y=x} (x-y)^{3/2}
                   +\frac{2}{3}\sijoitus{y=x}{y=1} (y-x)^{3/2} \right] dx \\
&= \int_0^1 \frac{2}{3} \left[ x^{3/2}+(1-x)^{3/2} \right] dx \\
&= \sijoitus{0}{1} \frac{4}{15} \left[ x^{5/2}-(1-x)^{5/2} \right] 
 = \underline{\underline{\frac{8}{15}}}\,. \loppu
\end{align*}
\begin{Exa}
$A=\text{kolmio}$, jonka kärjet ovat $(0,0)$, $(1,0)$ ja $(0,1)$. Laske
\begin{multicols}{2} \raggedcolumns
\[
I=\int_A xy^2\,dxdy.
\]
\begin{figure}[H]
\begin{center}
\input{kuvat/kuvaUint-16.pstex_t}
\end{center}
\end{figure}
\end{multicols}
\end{Exa}
\ratk
\begingroup
\allowdisplaybreaks
\begin{align*}
I &= \int_0^1\left[\int_0^{1-x} xy^2\,dy\right] dx \\
  &= \int_0^1 \left[\sijoitus{y=0}{y=1-x} \frac{1}{3}xy^3\right] dx \\
  &= \int_0^1 \frac{1}{3}x(1-x)^3\,dx \\
  &= \sijoitus{0}{1} -\frac{1}{12}x(1-x)^4 + \int_0^1 \frac{1}{12}(1-x)^4\,dx
     \qquad \text{(osittaisintegrointi)} \\
  &= \int_0^1 \frac{1}{12}(1-x)^4\,dx = \sijoitus{0}{1} -\frac{1}{60}(1-x)^5
   = \underline{\underline{\frac{1}{60}}}\,. \loppu
\end{align*}%
\endgroup
\begin{Exa} Laske $\mu(A)$, kun $A=\text{$R$-säteinen kiekko}$.
\end{Exa}
\ratk Kaavan \eqref{tasopalautus-x} mukaan
\begin{align*}
\mu(A) = \int_A d\mu 
      &= \int_{-R}^R \left[\int_{-\sqrt{R^2-x^2}}^{\sqrt{R^2-x^2}} dy\right] dx \\
      &= \int_{-R}^R \left[\sijoitus{-\sqrt{R^2-x^2}}{\sqrt{R^2-x^2}} y \right] dx
       = 2\int_{-R}^R \sqrt{R^2-x^2}\,dx.
\end{align*}
Sijoitus $x=R\sin t$, $dx=R\cos t\, dt$, $t\in [-\frac{\pi}{2},\frac{\pi}{2}]\,$ antaa 
\begin{align*}
2\int_{-R}^R \sqrt{R^2-x^2}\,dx &= 2R^2\int_{-\pi/2}^{\pi/2}\cos^2 t\,dt \\
                               &= R^2\sijoitus{-\pi/2}{\pi/2}(t+\sin t\cos t) 
                               = \pi R^2. \loppu
\end{align*}
Esimerkin 3 yleistyksenä nähdään kaavasta \eqref{tasopalautus-x}, että $x$-projisoituvan joukon
pinta-alamitta on laskettavissa kaavalla
\[
\mu(A)=\int_a^b [g_2(x)-g_1(x)]\,dx,
\]
joka on tuttu jo Luvusta \ref{pinta-ala ja kaarenpituus}. Kaavan käyttö vastaa pinta-alan
laskemista 'siivuttamalla' $A$ osiin muotoa 
$\Delta A=\{(x,y)\in A \ | \ x\in [x,x+\Delta x]\}$, jolloin on
$\mu(\Delta A) \approx [g_2(x)-g_1(x)]\Delta x$ pienillä $\Delta x$:n arvoilla ja olettaen,
että $g_1$ ja $g_2$ ovat jatkuvia pisteessä $x$. (Jatkuvuusoletus on lievennettävissä, vrt.\
Luku \ref{pinta-ala ja kaarenpituus}.)

\begin{figure}[H]
\begin{center}
\input{kuvat/kuvaUint-17.pstex_t}
\end{center}
\end{figure}

\subsection{Epäoleelliset tasointegraalit}
\index{tasointegraali!a@epäoleellinen|vahv}
\index{epzyoi@epäoleellinen integraali|vahv}

Jos tasointegraalissa $\int_A f\,dxdy$ joko $A$ ei ole rajoitettu tai $f$ ei ole $A$:ssa
rajoitettu, voidaan tasointegraalin määrittelyä laajentaa vastaavaan tapaan kuin yksiulotteisen
Riemannin integraalin tapauksessa, vrt.\ Luku \ref{integraalin laajennuksia}. Jos määritelmän
laajennus tuottaa integraalille yksikäsitteisen (reaali)arvon, sanotaan jälleen, että näin
\index{suppeneminen!c@integraalin}%
määritelty \kor{epäoleellinen} integraali \kor{suppenee}. Valaistakoon asiaa ensin esimerkeillä.
\begin{Exa} Olkoon $A=[0,1]\times[0,1]$ ja $B=[0,\infty)\times[0,\infty)$. Laske
\[
\text{a)}\,\ \int_A \frac{y^2}{\sqrt{x}}\,dxdy, \qquad
\text{b)}\,\ \int_B \frac{x}{1+x^2+y^2}\,dxdy.
\]
\end{Exa}
\ratk a) \ Koska $f$ ei ole $A$:ssa rajoitettu (ei edes määritelty, kun $x=0$), niin kyseessä
on epäoleellinen integraali. Tämä voidaan (ainakin yrittää) laskea raja-arvona
\[
\int_A f\,dxdy=\lim_{\eps\kohti 0^+} \int_{A_\eps} f\,dxdy,
\]
missä $A_\eps=[\eps,1]\times [0,1]$. Tässä $\int_{A_\eps} f\,dxdy$ on tavanomainen Riemannin
integraali, joten iteraatiokaava \eqref{tasopalautus-x} soveltuu:
\begin{align*}
\int_{A_\eps} f\,dxdy 
&= \int_\eps^1 \left[ \int_0^1 \frac{y^2}{\sqrt{x}}\,dy \right]dx
 = \int_\eps^1 \frac{1}{\sqrt{x}}\left[ \int_0^1 y^2\,dy \right]dx \\
&= \int_\eps^1 \frac{1}{\sqrt{x}}\left[ \sijoitus{y=0}{y=1} \frac{y^3}{3} \right]dx
 = \int_\eps^1 \frac{1}{3\sqrt{x}}\,dx\,
 = \,\sijoitus{\eps}{1} \frac{2}{3}\sqrt{x}\,\kohti\,\underline{\underline{\frac{2}{3}}}\,,
                                              \quad \text{kun}\ \eps\kohti 0^+.
\end{align*}
Integraali siis suppenee. Oikea integraalin arvo olisi saatu myös luottaen suoraan kaavaan
\eqref{tasopalautus-x} eli ohittamalla raja-arvoprosessi:
\[
\int_A f\,dxdy = \int_0^1\left[\int_0^1 \frac{y^2}{\sqrt{x}}\,dy\right]dx 
               = \int_0^1 \frac{1}{\sqrt{x}}\,dx \cdot \int_0^1 y^2\,dy
               = 2 \cdot \frac{1}{3} = \frac{2}{3}\,.
\]
b) Myös tämä integraali suppenee, ja suoraan kaavaan \eqref{tasopalautus-y} perustuva lasku
\begin{align*}
\int_A \frac{x}{(1+x^2+y^2)^2}\,dxdy 
       &= \int_0^\infty\left[\int_0^\infty \frac{x}{(1+x^2+y^2)^2}\,dx\right]dy \\
       &= \int_0^\infty\left[\,\sijoitus{x=0}{x=\infty}-\frac{1}{2(1+x^2+y^2)}\right]dy \\
       &= \int_0^\infty \frac{1}{2(1+y^2)}\,dy
        = \sijoitus{y=0}{y=\infty}\frac{1}{2}\Arctan y = \frac{\pi}{4}
\end{align*}
antaa oikean tuloksen. Asian voi varmistaa laskemalla ensin
\[
\int_{T_{ab}} \frac{x}{(1+x^2+y^2)^2}\,dxdy 
               = \int_0^b\left[\int_0^a \frac{x}{(1+x^2+y^2)^2}\,dx\right]dy,
\]
missä $T_{ab}=[0,a]\times[0,b]$, ja toteamalla, että tämän integraalin arvo $\kohti\pi/4$ aina
kun $a\kohti\infty$ ja $b\kohti\infty$. \loppu

Esimerkin perusteella kaavat \eqref{tasopalautus-x}--\eqref{tasopalautus-y} näyttävät soveltuvan
suoraan myös epäoleellisiin integraaleihin. Useimmiten näin onkin käytännössä, mutta kyse on
vain pääsäännöstä, kuten seuraava esimerkki osoittaa.
\begin{Exa} Laske säännöillä \eqref{tasopalautus-x} ja \eqref{tasopalautus-y} integraali
\[
I=\int_A f(x,y)\,dxdy, \quad A=[0,\infty)\times[0,1], \quad f(x,y)=xy(2-xy)e^{-xy}.
\]
\end{Exa}
\ratk
\begin{align*}
\text{Sääntö \eqref{tasopalautus-x}}: \quad 
I &= \int_0^\infty \left[ \sijoitus{y=0}{y=1} xy^2e^{-xy}\right]dx
   =\int_0^\infty xe^{-x}=1. \\[2mm]
\text{Sääntö \eqref{tasopalautus-y}}: \quad 
I &= \int_0^1 \left[\lim_{a\kohti\infty}\sijoitus{x=0}{x=a} x^2ye^{-xy}\right]dy
   =\int_0^1 \left[\lim_{a\kohti\infty} a^2ye^{-ay}\right]dy=0. \loppu
\end{align*}
Esimerkin ristiriidan voi pelkistää seuraavaan päättelyketjuun, jossa kysymysmerkillä varustettu
päätelmä (ja vain se!) on väärä:
\begin{align*}
1 &= \int_0^\infty \left[\int_0^1 f(x,y)\,dy\right]dx \\
  &= \lim_{a\kohti\infty} \int_0^a \left[\int_0^1 f(x,y)\,dy\right]dx \\
  &= \lim_{a\kohti\infty} \int_0^1 \left[\int_0^a f(x,y)\,dx\right]dy \\
  &= \int_0^1 \lim_{a\kohti\infty} \left[\int_0^a f(x,y)\,dx\right]dy \qquad \text{(?)} \\
  &= \int_0^1 \left[\int_0^\infty f(x,y)\,dx\right]dy = 0.
\end{align*}

Esimerkissä ongelmaksi osoittautuu integroitavan funktion \pain{merkinvaihtelu}: Jos tämä
sallitaan, eivät iteraatiokaavat \eqref{tasopalautus-x}--\eqref{tasopalautus-y} välttämättä anna
ristiriidattomia tuloksia. Ongelma ratkaistaankin poistamalla merkinvaihtelun mahdollisuus
epäoleellisen integraalin määritelmässä seuraavasti: 

\index{positiivinen osa (funktion)} \index{negatiivinen osa (funktion)}%
Määritellään funktion $f(x,y)$ \kor{positiivinen osa} $f_+$ ja \kor{negatiivinen osa} $f_-$
asettamalla $f$:n määrittelyjoukossa $D_f$
\[
\begin{aligned}
f_+(x,y) &= \begin{cases} 
              \,f(x,y), &\text{kun}\ f(x,y)>0, \\ 
              \,0, &\text{muulloin}, 
            \end{cases} \\
f_-(x,y) &= \begin{cases} -f(x,y), &\text{kun}\ f(x,y)<0, \\ 
                          \,0, &\text{muulloin}. 
            \end{cases}
\end{aligned}
\]
Tällöin jos $A\subset D_f$ on rajoitettu joukko ja $f$ on Riemann-integroituva yli $A$:n, niin
samoin ovat $f_+$ ja $f_-$ (Harj.teht.\,\ref{tasointegraalit}:\ref{H-uint-1: fplus ja fmiinus}).
Koska on myös $f=f_+-f_-$, niin seuraa
\[
\int_A f\,d\mu=\int_A f_+\,d\mu-\int_A f_-\,d\mu.
\]
Jos $A$ ei ole rajoitettu tai $f$ ei ole $A$:ssa rajoitettu, niin otetaan tämä integraalin
\pain{määritelmäksi}, eli asetetaan vaatimus, että $f_+$ ja $f_-$ ovat molemmat integroituvia
yli $A$:n. Funktion merkinvaihtelun mahdollisuus on näin poistettu integraalin laajennetusta
määritelmästä.
\jatko \begin{Exa} (jatko). Tässä on
\[
f_+(x,y)=\begin{cases} 
          \,xy(2-xy)e^{-xy}, &\text{kun } 0 \le xy \le 2, \\ \,0, &\text{muulloin}, 
         \end{cases}
\]
joten kun $A=[0,\infty)\times [0,1]$, saadaan
\begin{align*}
\int_A f_+\,dxdy &= \int_0^1 \left[\int_0^\infty f_+(x,y)\,dx\right]dy \\
                 &= \int_0^1 \left[\int_0^{2/y} xy(2-xy)e^{-xy}\,dx\right]\,dy \\
                 &=\int_0^1 \left[\sijoitus{x=0}{x=2/y} x^2ye^{-xy}\right]dy\,
                \,=\int_0^1 4e^{-2}y^{-1}\,dy.
\end{align*}
Saatu integraali ei suppene, joten $f_+$ ei ole integroituva yli $A$:n, eikä määritelmän mukaan
siis myöskään $f$. \loppu
\end{Exa}
Todettakoon lopuksi, että em.\ määritelmänkään nojalla iteraatiokaavat
\eqref{tasopalautus-x}--\eqref{tasopalautus-y} eivät sovellu epäoleellisiin integraaleihin
aivan yleispätevästi, mutta poikkeukset näistä (pää)säännöistä ovat melko eksoottisia,
ks.\ Harj.teht.\,\ref{Riemann vs Fubini}.

\subsection{Fubinin lause}

Tarkoituksena on perustella tasointegraalin iteraatiokaava \eqref{tasopalautus-x}. Tässä riittää
rajoittua erikoistapaukseen, jossa $A$ on suorakulmio, eli laskukaavaan
\[
\int_T f\,d\mu = \int_a^b\left[\int_c^d f(x,y)\,dy\right]dx, \quad T = [a,b] \times [c,d].
\]
Nimittäin jos tässä on
\[ 
T \supset A = \{(x,y) \in \R^2 \mid x \in [a,b]\ \ja\ g_1(x) \le y \le g_2(x) \}, 
\]
niin kyse on kaavasta \eqref{tasopalautus-x}, kun valitaan $f=f_0=$ $f$:n nollajatko $A$:n 
ulkopuolelle.
\begin{Lause} \label{Fubini} \index{Fubinin lause|emph} 
\vahv{(Tasointegraalin iteraatiokaava -- Fubinin\footnote[2]{Italialainen matemaatikko
\hist{Guido Fubini} eli vuosina 1897-1943. \index{Fubini, G.|av}} lause)}
Olkoon $f$ määritelty, rajoitettu ja 
Riemann-integroituva suorakulmiossa $T=[a,b]\times [c,d]$. Olkoon edelleen $f(x,y)$ muuttujan
$y$ suhteen Riemann-integroituva välillä $[c,d]$ jokaisella $x \in [a,b]$ ja muuttujan $x$ 
suhteen Riemann-integroituva välillä $[a,b]$ jokaisella $y \in [c,d]$. Tällöin funktiot
\[ 
F(x) = \int_c^d f(x,y)\,dy, \quad G(y) = \int_a^b f(x,y)\,dx 
\]
ovat Riemann-integroituvia väleillä $[a,b]$ ja $[c,d]$, ja pätee
\[ 
\int_T f(x,y)\,dxdy = \int_a^b F(x)\,dx = \int_c^d G(y)\,dy. 
\]
\end{Lause}
\tod Olkoon $\mathcal{T}_h = \{T_{kl},\ k=1\ldots m,\ l=1\ldots n\},\ 
T_{kl}=[x_{k-1},x_k]\times[y_{l-1},y_l]$, suorakulmion $T$ jako, ja olkoon
\[ 
m_{kl} = \inf_{(x,y) \in T_{kl}} f(x,y), \quad M_{kl} = \sup_{(x,y) \in T_{kl}} f(x,y). 
\]
Tällöin $m_{kl} \le f(x,y) \le M_{kl}\ \forall (x,y) \in T_{kl}$, joten integraalien 
vertailuperiaatteen nojalla
\[ 
m_{kl}(y_l-y_{l-1}) \le \int_{y_{l-1}}^{y_l} f(x,y)\,dy 
                    \le M_{kl}(y_l-y_{l-1}) \quad \forall x \in [x_{k-1},x_k]. 
\]
Summaamalla tämä yli indeksin $l$ seuraa
\[ 
m_k\,=\,\sum_{l=1}^n m_{kl}(y_l-y_{l-1}) \le F(x) 
         \le \sum_{l=1}^n M_{kl}(y_l-y_{l-1})\,=\,M_k \quad \forall x \in [x_{k-1},x_k], 
\]
mistä puolestaan seuraa
\begin{align*}
\sum_{k=1}^m\sum_{l=1}^n m_{kl}(x_k-x_{k-1})(y_l-y_{l-1})\,
                &=\,\sum_{k=1}^m m_k(x_k-x_{k-1}) \\
                &\le \underline{\int_a^b}F(x)\,dx\,\le \overline{\int_a^b}F(x)\,dx \\
                &\le\,\sum_{k=1}^m M_k(x_k-x_{k-1}) \\
                &=\,\sum_{k=1}^m\sum_{l=1}^n M_{kl}(x_k-x_{k-1})(y_l-y_{l-1}),
\end{align*}
eli
\[ 
\underline{\sigma}(f,\mathcal{T}_h) \le \underline{\int_a^b}F(x)\,dx 
                    \le \overline{\int_a^b}F(x)\,dx \le \overline{\sigma}(f,\mathcal{T}_h), 
\]
missä $\underline{\sigma}(f,\mathcal{T}_h)$ ja $\overline{\sigma}(f,\mathcal{T}_h)$ ovat
jakoon $\mathcal{T}_h$  liittyvät Riemannin ala- ja yläsummat. Ottamalla tässä vasemmalla
supremum ja oikealla infimum kaikkien jakojen $\mathcal{T}_h$ suhteen seuraa tasointegraalin
määritelmän nojalla (vrt.\ edellinen luku)
\[ 
\int_T f\,d\mu \le \underline{\int_a^b}F(x)\,dx \le \overline{\int_a^b}F(x)\,dx 
                                                \le \int_T f\,d\mu. 
\]
Tämän mukaan $F$:n ylä- ja alaintegraalit välillä $[a,b]$ ovat samat, joten $F$ on välillä
$[a,b]$ Riemann-integroituva (Lause \ref{Riemann-integroituvuus}), ja epäyhtälöketjusta seuraa
myös väitetty (ensimmäinen) laskukaava. Toinen laskukaava perustellaan vastaavasti.  
%\[ 
%\underline{\int_a^b}F(x)\,dx = \overline{\int_a^b}F(x)\,dx 
%                             = \int_a^b F(x)\,dx = \int_T f\,d\mu. \loppu 
%\]

Mainittakoon, että Fubinin lauseessa ei välttämättä tarvita oletuksia funktion $f$ 
integroituvuudesta erikseen muuttujien $x$ ja $y$ suhteen, vaan riittää, että $f$ on 
integroituva yli $T$:n. Tällöin on väittämässä funktiot $F$ ja $G$ määriteltävä joko ylä- tai
alaintegraaleina (mikä tahansa neljästä vaihtoehdosta), muuten lauseen väittämä säilyttää 
pätevyytensä, ja todistuskin pysyy olennaisesti samana (!). Jos siis oletetaan ainoastaan, että
$f$ on Riemann-integroituva yli $T$:n, niin pätee esimerkiksi laskukaava
\[ 
\int_T f\,dxdy = \int_a^b\left[\overline{\int_c^d} f(x,y)\,dy\right]dx. 
\]
Huomautettakoon lopuksi, että Lauseiden \ref{Analyysin peruslause} ja
\ref{jatkuvan funktion integroituvuus tasossa} perusteella Fubinin lauseen ehdot toteutuvat
oletetussa muodossa, jos $f$ on jatkuva suorakulmiossa $T=[a,b]\times[c,d]$.

\Harj
\begin{enumerate}

\item
Olkoon $f(x,y)=e^{x^4-2xy-y^2}$ ja $A=[-1,1]\times[0,2]$. Laske $\int_A f_{xy}(x,y)\,dxdy$.

\item
Laske annetun funktion $f(x,y)$ integraali yli annetun joukon $A$: \vspace{1mm}\newline
a) \ $xy+y^2,\,\ A:\ 0 \le x \le 1\ \ja\ 0 \le y \le 1-x$ \newline
b) \ $(x+y)e^{x+y},\,\ A:\ x,y \ge 0\ \ja\ x+y \le 1$ \newline
c) \ $\cos y,\,\ A:\ 0 \le x \le \pi\ \ja\ \abs{y} \le x$ \newline
d) \ $y^2e^{xy}\,\ A:\ 0 \le x \le 2\ \ja\ x \le y \le 2$ \newline
e) \ $xy^2,\,\ A:\ 0 \le x \le 1\ \ja\ x^2 \le y \le \sqrt{x}$ \newline
f) \ $x\cos y,\,\ A:\ 0 \le x \le 1\ \ja\ 0 \le y \le 1-x^2$ \newline
g) \ $\sqrt{a^2-y^2}\,,\,\ A=$ kolmio, jonka kärjet $(0,0)$, $(0,a)$ ja $(a,a)$ \ ($a>0$)
\newline
h) \ $xy(1+x^4)^{-1},\,\ A=$ kolmio, jonka kärjet $(0,0)$, $(1,0)$ ja $(1,1)$ 

\item
Määritä pinta-ala $\mu(A)$, kun $A$ määritellään annetuilla ehdoilla ($a>0$).
\vspace{1mm}\newline 
a) \ $ax \ge y^2,\,\ x^3 \le ay^2 \qquad\qquad\,$
b) \ $ax \ge y^2,\,\ x^2+y^2 \le a^2$ \newline
c) \ $a^2y \le x^3,\,\ y \ge 3x-2a \qquad\ $
d) \ $x^2-y^2 \le a^2,\,\ |y| \le x/2$ \newline
e) \ $x,y \ge 0,\ ax \ge y^2,\ ay \ge x^2,\ 8xy^2 \le a^3$

\item
Määritä ja luokittele funktion
\[
f(x,y)=\int_A (10-xu-yv^2)^2\,dudv, \quad A=[0,1]\times[0,1]
\]
kriittiset pisteet.

\item
Suppeneeko vai hajaantuuko integraali $\int_A(x-y)^{-1}\,dx$, kun \newline
a) $A=[0,1]\times[0,1]$, \ b) $A=[0,1]\times[-1,0]$\,?

\item
Laske seuraavat integraalit kaikilla arvoila $\alpha\in\R$, joilla integraali suppenee.
\vspace{1mm}
a) \ $\int_A x^\alpha y^\alpha\,dxdy, \quad A=[1,\infty)\times[1,\infty)$ \newline
b) \ $\int_A |x-2y|^\alpha\,dxdy, \quad A=[1,2]\times[0,1]$ \vspace{1mm}\newline
c) \ $\int_A |x-2y|^\alpha\,dxdy, \quad A=[1,2]\times[1,2]$ \vspace{1mm}\newline
d) \ $\int_A x^\alpha ye^{-xy}\,dxdy, \quad A=[0,1]\times[0,\infty)$

\item
Seuraavat integraalit ovat muotoa $\int_A f(x,y)\, dxdy$. Määritä ensin $A$ ja laske sitten
integraalin arvo valitsemalla sopiva integroimisjärjestys!
\begin{align*}
&\text{a)}\ \ \int_0^1\left[\int_x^1 e^{-y^2}\,dy\right]dx \qquad\quad
 \text{b)}\ \ \int_0^{\pi/2}\left[\int_y^{\pi/2} \frac{\sin x}{x}\,dx\right]dy \\
&\text{c)}\ \ \int_0^1\left[\int_{x^2}^x xy^{-1}e^y\,dy\right]dx \qquad
 \text{d)}\ \ \int_0^1\left[\int_x^1 \frac{y^\alpha}{x^2+y^2}\,dy\right]dx\ \ (\alpha>0)
\end{align*}

\item
Tiedetään, että $\int_\R e^{-x^2}\,dx=\sqrt{\pi}$.\footnote[2]{Ks.\ Propositio \ref{Gamma(1/2)}
jäljempänä.} Laske tällä perusteella
\begin{align*}
&\text{a)}\,\ \int_{\R^2} |x|e^{-x^2-y^2}\,dxdy \qquad\quad\ \
 \text{b)}\,\ \int_{\R^2} (x^2+y^2)e^{-x^2-y^2}\,dxdy \\
&\text{c)}\,\ \int_{\R^2} |x+y|e^{-x^2-y^2}\,dxdy \qquad
 \text{d)}\,\ \int_{\R^2} |x-y^2|e^{-|x|-y^2}\,dxdy
\end{align*}

\item (*)
a) Näytä integraaliin vertaamalla, että 2-ulotteinen sarja
\[
\sum_{i=1}^\infty\sum_{j=1}^\infty \frac{1}{(i+j)^\alpha}
\]
suppenee täsmälleen kun $\alpha>2$. \ b) Näytä, että approksimaation
\[
\sum_{i=1}^\infty\sum_{j=1}^\infty \frac{1}{(i+j)^\alpha} \,\approx\,
\sum_{i=1}^N\sum_{j=1}^N \frac{1}{(i+j)^\alpha} \quad (\alpha>2,\ N\in\N)
\]
virhe on luokkaa $\Ord{N^{2-\alpha}}$. \ c) Näytä, että tarkemmin pätee
\[
\sum_{i=1}^\infty\sum_{j=1}^\infty \frac{1}{(i+j)^\alpha} \,=\,
\sum_{i=1}^N\sum_{j=1}^N \frac{1}{(i+j)^\alpha} +
\frac{2-2^{2-\alpha}}{(\alpha-2)(\alpha-1)}\,N^{2-\alpha} + \Ord{N^{1-\alpha}}.
\]

\item (*) \label{Riemann vs Fubini} \index{zzb@\nim!Riemann vastaan Fubini}
(Riemann vastaan Fubini) Järjestetään rationaaliluvut välillä $\,[0,1]\,$ jonoksi
$\,\{ x_n,\ n = 0, 1, 2, \ldots \}\,$ ja määritellään joukossa $A = [0,1] \times [0, \infty)$
funktio $f$ seuraavasti:
\[
f(x,y) = \begin{cases} 
         \,e^{n-y}, &\text{kun}\ \ x = x_n\ \ \text{ja}\ \ y \ge n, \\ \,0, &\text{muulloin}.
         \end{cases}
\]
a) Näytä, että $f$ on Riemann-integroituva yli $A$:n (laajennettu määritelmä).
b) Näytä, että laskukaavoista \eqref{tasopalautus-x}--\eqref{tasopalautus-y} jälkimmäinen
toimii, edellinen ei, ts. 
\[
\int_A f\,dx dy\, = \int_0^\infty \left[\int_0^1 f(x, y)\,dx \right] dy\, \neq \, 
                    \int_0^1 \left[\int_0^\infty f(x, y)\,dy \right] dx.
\]

\end{enumerate} %Tasointegraalien laskutekniikka
\section{Avaruusintegraalit} \label{avaruusintegraalit}
\alku
\index{avaruusintegraali|vahv}

Integraalia muotoa
\[
\int_A f\,d\mu,\quad A\subset\R^3,\quad f=f(x,y,z),
\]
\index{tilavuusmitta} \index{Jordan-mitta!c@tilavuusmitta}
\index{mitta, mitallisuus!a@Jordan-mitta}%
missä $\mu$ on $\R^3$:n \kor{tilavuusmitta}, sanotaan \kor{avaruusintegraaliksi} 
(myös tilavuusintegraaliksi, engl.\ volume integral). Tilavuusmitta ja -integraali määritellään
samalla periaattella kuin tasossa, vain sillä erotuksella, että perussuorakulmion tilalla on
suorakulmainen perussärmiö, jonka mitaksi oletetaan (tilavuusmitan aksiooma)
\[
T=[a_1,b_1] \times [a_2,b_2] \times [a_3,b_3]: \quad \mu(T)=(b_1-a_1)(b_2-a_2)(b_3-a_3).
\]
Jordan-mittana (ulko- ja sisämittojen avulla, vrt. Luku \ref{tasointegraalit}) määritelty 
tilavuusmitta on siirto-, peilaus- ja kiertoinvariantti, ts.\ se on valitusta (karteesisesta)
koordinaatistosta riippumaton.

Kuten tasointegraalit, avaruusintegraalitkin voidaan laskea suoraan numeerisesti integraalin
määritelmästä. Jos $\mathcal{T}_h$ on särmiön $T\supset A$ jako osasärmiöihin, joiden särmät
ovat enintään $h$:n pituiset, vaatii jakoon liittyvän Riemannin summan laskeminen yleisesti
$N\sim h^{-3}$ laskuoperaatiota, kun vastaava luku tasossa on  $N\sim h^{-2}$ ja yhdessä
dimensiossa $N\sim h^{-1}$. Tarkkuus riippuu kaikissa tapauksissa tiheysparametrista $h$
olennaisesti samalla tavalla (esimerkiksi $\text{virhe}\sim h^2$), joten samaan tarkkuuteen
pyrittäessä työmäärä kasvaa voimakkaasti integraalin dimension kasvaessa. (Sama ilmiö vaivaa
kaikkia numeerisen integroinnin menetelmiä --- ja numeerisia laskentamenetelmiä yleisemminkin.)
Riemannin summiin perustuvista menetelmistä
\index{keskipistesääntö} \index{yhdistetty!a@keskipistesääntö}%
tarkin on jälleen (yhdistetty) \kor{keskipistesääntö}, joka integroi kussakin osasärmiössä
tarkasti ensimmäisen asteen polynomin.
\begin{Exa} Olkoon $A = [0,1] \times [0,1] \times [0,1]$ ja laskettava
$\int_A 3x^2 y^2 z^2\,d\mu$. Kun $A$ jaetaan suorakulmaisiin särmiöihin kokoa 
$h \times h \times h$ ja kunkin särmiön yli integroidaan keskipistesäännöllä, niin laskettavaksi
tulee summa, jossa on $N=h^{-3}$ termiä. Tällä tavoin saadaan integraalille seuraavat likiarvot
(tarkka arvo $=1/9$, vrt.\ Esimerkki \ref{tasointegraalit}:\,\ref{keskipistesääntö tasossa})\,:
\begin{center}
\begin{tabular}{lll}
$h$    & $N$       &  Likiarvo  \\  \hline  \\
0.1    & $10^3$    &  0.110279859.. \\
0.01   & $10^6$    &  0.111102777.. \\
0.001  & $10^9$    &  0.111111027.. \\
0.0001 & $10^{12}$ &  0.111111110.. \qquad\qquad\loppu
\end{tabular}
\end{center}
\end{Exa}

Myös avaruusintegraalin voi palauttaa peräkkäisiksi yksiulotteisiksi integraaleiksi, jolloin 
integraalin voi suotuisissa oloissa laskea suljetussa muodossa. Ensinnäkin jos 
$T=[a_1,b_1]\times[a_2,b_2]\times[a_3,b_3]$ on suorakulmainen särmiö, ja merkitään
$Q=[a_1,b_1]\times[a_2,b_2]$, niin seuraten Fubinin lauseen todistuksessa käytettyä ajatusta
saadaan iteraatiokaava
\[ 
\int_T f\,d\mu = \int_Q F(x,y)\,dxdy, \quad F(x,y) = \int_{a_3}^{b_3} f(x,y,z)\,dz. 
\]
Kun tässä edelleen käytetään Fubinin lauseen laskukaavaa, on tuloksena kolminkertainen 
iteraatiokaava
\[
\int_T f\,d\mu = \int_{a_1}^{b_1}\left\{\int_{a_2}^{b_2}\left[\int_{a_3}^{b_3} 
                                                     f(x,y,z)\,dz\right]dy\right\}dx.
\]
\jatko \begin{Exa} (jatko) Esimerkin integraali purkautuu kolmen integraalin tuloksi:
\begin{align*}
\int_{A} f\,d\mu &= \int_0^1\left\{\int_0^1\left[\int_0^1 3x^2 y^2 z^2\,dz\right]dy\right\}dx \\
                 &= 3\int_0^1 x^2\left\{\int_0^1 y^2\left[\int_0^1 z^2\,dz\right]dy\right\}dx \\
                 &= 3\int_0^1 x^2\,dx \int_0^1 y^2\,dy \int_0^1 z^2\,dz
                  = 3\cdot\frac{1}{3}\cdot\frac{1}{3}\cdot\frac{1}{3} = \frac{1}{9}\,. \loppu
\end{align*}
\end{Exa}

Jos $A \subset \R^3$ on yleisempi joukko kuin suorakulmainen särmiö, niin määritellään kuten
tasossa
\[ 
\int_A f\,d\mu = \int_T f_0\,d\mu, 
\]
missä $f_0$ on $f$:n nollajatko suorakulmaiseen särmiöön $T \supset A$. Soveltamalla tässä 
ym.\ iteraatiokaavaa saadaan erilaisia $A$:n muotoon sovitettuja integraalin purkukaavoja.
Esimerkiksi jos $A$ esitetään muodossa
\[
A=\{(x,y,z)\in\R^3 \ | \ x\in B\subset\R \ \ja \ (y,z)\in C(x)\subset\R^2\},
\]
niin saadaan purkukaava
\begin{equation} \label{siivutuskaava}
\int_A f\,d\mu=\int_B\left[\int_{C(x)} f(x,y,z)\,dydz\right]dx, \tag{$\star$}
\end{equation}
missä sisemmän tasointegraalin voi edelleen purkaa edellisen luvun menetelmin. Purkusäännön
\eqref{siivutuskaava} voi havainnollistaa geometrisesti 'siivutusperiaatteena', jossa $A$
jaetaan osiin
\[
\Delta A=\{(x,y,z)\in A \ | \ x\in\Delta B\},
\]
jolloin on likimäärin (vrt. kuvio)
\[
\int_{\Delta A} f\,d\mu\approx\left[\int_{C(x)} f(x,y,z)\,dydz\right]\mu(\Delta B)\quad
                                                     (\mu=\text{pituusmitta}).
\]
\begin{figure}[H]
\begin{center}
\import{kuvat/}{kuvaUint-23.pstex_t}
\end{center}
\end{figure}
Ajatellen, että avaruusintegraali viime kädessä palautuu peräkkäisiksi yksiulotteisiksi 
integraaleiksi, käytetään avaruusintegraaleille usein merkintöjä
\[
\int_A f\,dxdydz\quad\text{tai}\quad \iiint f\,dxdydz,
\]
missä siis on merkitty $d\mu=dxdydz$.\footnote[2]{Sovelluksissa (etenkin fysiikassa)
avaruusintegraalin mittamerkintä on usein $dV$.}
%Tälle 'differentiaaliselle tilavuudelle' käytetään myös
%usein merkintää $dV$ (etenkin fysiikassa).
\begin{Exa}
$A=\text{tetraedri}$, jonka kärjet ovat $(0,0,0)$, $(1,0,0)$, $(0,1,0)$ ja $(0,0,1)$. Laske
$\int_A f\,dxdydz$, kun $f(x,y,z)=xyz$.
\end{Exa}
\ratk Purkukaavan \eqref{siivutuskaava} nojalla on ensinnäkin
\[
\int_A f\,dxdydz = \int_0^1\left[\int_{C(x)} xyz\,dydz\right]dx,
\]
missä $yz$-tason joukko $C(x)$ märäytyy ehdoista $y \ge 0$, $z \ge 0$ ja
\[
x+y+z \le 1\,\ \ekv\,\ y+z \le 1-x,
\]
eli $C(x)$ on kolmio, jonka kärjet ovat pisteissä $(0,0,0)$, $(0,1-x,0)$ ja $(0,0,1-x)$.
Soveltaen sisempään integraaliin edellisen luvun purkusääntöjä saadaan
\begin{align*}
\int_A f\,dxdydz 
&= \int_0^1\left\{\int_0^{1-x}\left[\int_0^{1-x-y} xyz\,dz\right]dy\right\}dx \\
&= \int_0^1\left\{\int_0^{1-x}\left[\sijoitus{z=0}{z=1-x-y}
                                   \frac{1}{2}xyz^2\right]dy\right\}dx \\
&= \int_0^1\left[\int_0^{1-x} \frac{1}{2}xy(1-x-y)^2\,dy\right]dx \\
(\text{os.\ int.}) \quad 
&= \int_0^1\left[\sijoitus{y=0}{y=1-x} -\frac{1}{6}xy(1-x-y)^3 + \int_0^{1-x}
                 \frac{1}{6} x(1-x-y)^3\,dy \right]dx   \\
&= \int_0^1\left[\sijoitus{y=0}{y=1-x} -\frac{1}{24}x(1-x-y)^4 \right]dx \\
&= \int_0^1 \frac{1}{24} x(1-x)^4\,dx \\
(\text{os.\ int.}) \quad
&= \sijoitus{0}{1} -\frac{1}{120}x(1-x)^5 + \int_0^1 \frac{1}{120} (1-x)^5\,dx \\
&= \sijoitus{0}{1} -\frac{1}{720}(1-x)^6 = \underline{\underline{\frac{1}{720}}}\,. \loppu
\end{align*}

Purkusääntö \eqref{siivutuskaava} on erityisen kätevä silloin kun $f$ ei riipu muuttujista
$y,z$. Tällöin integraali palautuu suoraan 1-ulotteiseksi edellyttäen, että joukkojen $C(x)$
pinta-alamitta on helposti määrättävissä:
\[
\int_A f\,dxdydz=\int_B f(x)\,\mu(C(x))\,dx,\quad f=f(x)\quad (\mu=\text{pinta-alamitta}).
\]
\begin{Exa} \label{3-pallon tilavuus} Laske $R$-säteisen pallon tilavuus.
\end{Exa}
\ratk Käytetään purkukaavaa \eqref{siivutuskaava}, jolloin 
\[ 
B=[-R,R], \quad C(x)=\{(y,z)\in\R^2 \ | \ y^2+z^2\leq R^2-x^2\}.
\]
Koska $C(x)$ on kiekko, jonka säde $=\sqrt{R^2-x^2}$, niin $\mu(C(x))=\pi(R^2-x^2)$.
Siis
\[
\mu(A) = \int_A\,d\mu = \int_{-R}^R \pi(R^2-x^2)\,dx 
                      = \sijoitus{-R}{R}\pi\left(R^2 x - \frac{1}{3}\,x^3\right)
                      = \frac{4}{3}\,\pi R^3. \loppu
\]
\begin{Exa} \label{puolipallon momentti} Laske $\int_A x\ dxdydz$, kun 
$A=\{(x,y,z) \ | \ x^2+y^2+z^2\leq R^2, \ x\geq 0\}$ (puolipallo).
\end{Exa}
\ratk Purkukaavassa \eqref{siivutuskaava} on tässä $B = [0,R]$ ja $C(x)$ on sama kuin
edellisessä esimerkissä, joten saadaan
\begin{align*}
\int_A f\,dxdydz &= \int_B f(x)\,\mu(C(x))\,dx \\
&=\int_0^R x\cdot\pi(R^2-x^2)\,dx \\
&=\pi\sijoitus{0}{R} \left(\frac{1}{2}R^2x^2-\frac{1}{4}x^4\right)
 =\underline{\underline{\frac{1}{4}\pi R^4}}. \loppu
\end{align*}
\begin{Exa}
Suorien $x=0$, $y=0$ ja käyrän $y=e^{-x}$ rajaama tasoalue pyörähtää $x$-akselin ympäri,
jolloin syntyy avaruuden $\R^3$ joukko $A$. Laske tilavuus $\mu(A)$.
\end{Exa} 
\ratk $A$ ei ole rajoitettu, joten kyseessä on epäoleellinen integraali. Kaavassa
\eqref{siivutuskaava} on $B=[0,\infty)$ ja $C(x)$ on kiekko, jonka säde $=e^{-x}$, joten
\[
\mu(A)=\int_A dxdydz = \int_0^\infty \pi(e^{-x})^2\,dx
                     = \int_0^\infty \pi e^{-2x}\,dx 
                     = \sijoitus{0}{\infty} -\frac{\pi}{2}e^{-2x} 
                     = \underline{\underline{\frac{\pi}{2}}}\,. \loppu
\]
\begin{Exa} Kaksi $R$-säteistä lieriötä leikkaa kohtisuorasti toisensa. Mikä on molempien
lieriöiden sisään jäävän joukon $A$ tilavuusmitta?
\end{Exa}
\ratk Olkoon toisen lieriön akseli $z$-akseli ja toisen $y$-akseli, jolloin
\[
A=\{(x,y,z)\in\R^3 \mid x^2+y^2 \le R^2\ \ja\ x^2+z^2 \le R^2\}.
\]
Kun leikataan tämä $yz$-tason suuntaisilla tasoilla, niin todetaan, että
\begin{align*}
A    &= \{(x,y,z)\in\R^3 \mid x\in [-R,R]\ \ja\ (y,z)\in C(x)\}, \quad \text{missä} \\[2mm]
C(x) &= \{(y,z)\in\R^2 \ | \ x^2+y^2\leq R^2 \ \ja \ x^2+z^2\leq R^2\} \\
     &= \{(y,z)\in\R^2 \ | \ y^2\leq R^2-x^2 \ \ja \ z^2\leq R^2-x^2\}.
\end{align*}
Poikkileikkaus $C(x)$ on tämän mukaan neliö, jonka sivun pituus $=2\sqrt{R^2-x^2}$.
Kaavan \eqref{siivutuskaava} mukaan on siis
\[
\mu(A) = \int_{-R}^R \mu(C(x))\,dx
       = \int_{-R}^R 4(R^2-x^2)\,dx
       = 4\sijoitus{-R}{R}(R^2x-\frac{1}{3}x^3)
       = \underline{\underline{\frac{16}{3}R^3}}. \loppu
\]

\subsection{Avaruusintegraalit $\R^n$:ssä}
\index{avaruusintegraali!a@$n$-ulotteinen|vahv}

Yleinen \kor{$n$-ulotteinen avaruusintegraali} on muotoa $I(f,A,\mu)=\int_A f\,d\mu$,
\index{tilavuusmitta!a@$n$-ulotteinen} \index{Jordan-mitta!d@$n$-ulotteinen tilavuusmitta}
\index{mitta, mitallisuus!a@Jordan-mitta}%
missä $A\subset\R^n$, $f=f(\mx)=f(x_1,\ldots,x_n)$, ja $\mu$ on \kor{$n$-ulotteinen 
tilavuusmitta} (Jordan-mitta). Tämän ominaisuuksiin kuuluu, että $n$-ulotteisen suorakulmaisen
särmiön mitta on (aksiooma)
\[ 
\mu(T)=\prod_{i=1}^n (b_i-a_i), \quad 
                 T=[a_1,b_1] \times [a_2,b_2] \times \cdots \times [a_n,b_n]. 
\]
Integraalin vaihtoehtoiset merkintätavat
\[
\int_A f\,dx_1\cdots dx_n,\quad \iint\cdots\int f\,dx_1\cdots dx_n.
\]
viittaavat jälleen siihen tosiseikkaan, että integraali on palautettavissa peräkkäisiksi 
yksiulotteisiksi integraaleiksi. Iteraatiokaavan perusmuoto tapauksessa $A=T$ on
\[
\int_T f\,d\mu=\int_{a_1}^{b_1}\left[\int_{a_2}^{b_2} \cdots \left[\int_{a_n}^{b_n} 
                                             f(x_1,\ldots,x_n)\,dx_n\right]\cdots\right]dx_1.
\]
\begin{Exa}
Laske neliulotteisen $R$-säteisen pallon (kuulan) tilavuus $\mu(A)$.
\end{Exa}
\ratk Kun kirjoitetaan
\[
A=\{(x_1,x_2,x_3,x_4)\in\R^4 \ | \ x_1\in [-R,R] \ \ja \ (x_2,x_3,x_4)\in C(x_1)\},
\]
niin (vrt. purkukaava \eqref{siivutuskaava} edellä)
\[
\int_A d\mu=\int_{-R}^R \mu(C(x_1))\,dx_1,
\]
missä $\mu$ on $\R^3$:n tilavuusmitta. Tässä $C(x_1)\subset\R^3$ on pallo, jonka säde on 
$\sqrt{R^2-x_1^2}$, joten
\begin{align*}
\mu(A) &= \int_{-R}^R \frac{4}{3}\pi(R^2-x_1^2)^{3/2}\,dx_1\quad (\text{sij. } x_1=R\sin t) \\
       &= \frac{4\pi}{3}R^4\int_{-\pi/2}^{\pi/2} \cos^4 t\,dt
        = \frac{4\pi}{3}R^4\cdot\frac{3\pi}{8}        
        = \underline{\underline{\frac{1}{2}\pi^2 R^4}}. \loppu
\end{align*}

Mitä tulee $n$-ulotteisen Jordan-mitan teoreettisiin ominaisuuksiin, todettakoon ainoastaan
(ilman todistusta), että nämä ominaisuudet ovat vastaavat kuin tasossa. Esimerkiksi mitta on
siirto- ja kiertoinvariantti ja määriteltävissä ylä- ja alaintegraalien avulla, vrt.\ Luku
\ref{tasointegraalit}. Pätee myös Lauseen \ref{jatkuvan funktion integroituvuus tasossa}
yleistys:
\begin{*Lause} \label{jatkuvan funktion integroituvuus Rn:ssä}
\index{Riemann-integroituvuus!b@jatkuvan funktion|emph} Jos $f$ on jatkuva särmiössä
$T=[a_1,b_1] \times [a_2,b_2] \times \cdots \times [a_n,b_n]$ ja $A \subset T$ on mitallinen,
niin $f$ on Riemann-integroituva yli $A$:n.
\end{*Lause} 

\subsection{*Suuntaissärmiön $K\subset\R^n$ tilavuus}
\index{tilavuus!c@$\R^n$:n suuntaissärmiön|vahv}

Palautettakoon mieliin Luvusta \ref{affiinikuvaukset}, että avaruuden $\R^n$ $n$-ulotteinen
suuntaissärmiö, jonka kärki on pisteessä $\mx_0$ ja kärjestä lähtevät särmävektorit ovat
$\ma_i$, on joukko
\[
K=\{\mx\in\R^n \mid \mx=\mx_0+\sum_{i=1}^n t_i\ma_i,\ t_i\in[0,1],\ i=1 \ldots n\}.
\]
Jos kirjoitetaan $\ma_i=\sum_{j=1}^n a_{ij}\me_j$ ja muodostetaan kertoimista $a_{ij}$ matriisi
$\mA$, niin tapauksissa $n=2$ ja $n=3$ tiedetään, että $K$:n pinta-ala ($n=2$) tai tilavuus
($n=3$) on laskettavissa kaavalla $\mu(K)=\abs{\text{det}\mA}$ (ks.\ Luku \ref{ristitulo}).
Näytetään, että tämä pätee yleisesti.
\begin{Prop} \label{suuntaissärmiön tilavuuskaava} Avaruuden $\R^n$ $n$-ulotteisen 
suuntaissärmiön $K$ tilavuusmitta on $\mu(K)=\abs{\text{det}\mA}$, missä matriisin $\mA$ riveinä
ovat $K$:n särmävektoreiden $\ma_i$ kertoimet $a_{ij}$ kannassa $\{\me_1,\ldots,\me_n\}$.
\end{Prop}
\tod Pidetään tunnettuna, että $\R^n$:n tilavuusmitta on kiertoinvariantti. Tällöin 
kantavektori $\me_n$ voidaan valita vektoreiden $\ma_i,\ i=1 \ldots n-1$ virittämää hypertasoa
(aliavaruutta) vastaan kohtisuoraksi, jolloin on $a_{in}=0$, kun $i=1 \ldots n-1$. Kierto
ei vaikuta myöskään determinantin $\det\mA$ arvoon, sillä kierto vastaa muunnosta
$\mA^T=[\ma_1,\ldots,\ma_n] \ext [\mC\ma_1,\ldots,\mC\ma_n]=\mC\mA^T$ eli $\mA \ext \mA\mC^T$,
missä $\det\mC=1$. Oletetussa koordinaatistossa voidaan suuntaissärmiö esittää muodossa
\[
K=\{\mx\in\R^n \mid \mx=\mx_0+t_n a_{nn}\me_n+\my,\,\ t_n\in[0,1]\,\ja\,\my \in A(t_n)\},
\]
missä $A(t_n)\subset\R^{n-1}$ on $(n-1)$-ulotteinen suuntaissärmiö, jonka särmävektorit ovat 
$\mb_i=\sum_{j=1}^{n-1} a_{ij}\me_j,\ i=1 \ldots n-1$, ja kärki on pisteessä
$\my_0(t_n)=t_n\sum_{j=1}^{n-1} a_{nj}\me_j$. Koska $\R^{n-1}$:n tilavuusmitta on 
siirtoinvariantti, niin joukon $A(t_n)$ mitta on $t_n$:stä riippumaton; merkitään
$\mu(A(t_n))=V_{n-1}$. Purkukaavaa \eqref{siivutuskaava} vastaten $K$:n mitta on tällöin
\[
\mu(K)=\int_0^{\abs{a_{nn}}}\left[\int_{A(t_n)} dx_1 \ldots dx_{n-1}\right]dx_n
      =\abs{a_{nn}}V_{n-1}\,.
\]
Jos merkitään $\mu(K)=V_n$, niin on siis saatu palautuskaava
\[
V_n=\abs{a_{nn}}V_{n-1}\,.
\]
Toisaalta jos merkitään $D_n=\abs{\text{det}\mA}$ ja huomioidaan, että $a_{in}=0$ kun 
$i \neq n$, niin alideterminanttisäännön (Lause \ref{alideterminanttisääntö}) perusteella
saadaan vastaava palautuskaava
\[
D_n=\abs{a_{nn}}D_{n-1}\,.
\]
Koska palautuskaavat ovat samaa muotoa ja koska $V_n=D_n$ kun $n=2,3$, niin päätellään, että
$V_n=D_n\ \forall n$. \loppu

\Harj
\begin{enumerate}

\item
Laske tilavuus $\mu(A)$, kun $A\subset\R^3$ on määritelty annetuilla ehdoilla. 
\vspace{1mm}\newline
a) \ $0 \le x \le 1\ \ja\ 0 \le y \le x\ \ja\ 0 \le z \le 1-x^2$ \newline
b) \ $0 \le y \le 1\ \ja\ 0 \le x \le y\ \ja\ 0 \le z \le 1-x^2$ \newline
c) \ $x,y \ge 0\ \ja\ x+y \le 1\ \ja\ 0 \le z \le 1-x^2-y^2$ \newline
d) \ $1 \le x \le 2\ \ja\ 0 \le y \le x\ \ja\ 0 \le z \le 1/(x+y)$ \newline
e) \ $0 \le z \le 1-x^2-2y^2$ \newline
f) \ $0 \le x \le \sqrt[4]{\pi}\ \ja\ 0 \le y \le x\ \ja\ 0 \le z \le x^2\sin(y^4)$ \newline
g) \ $x^2 \le z \le 1-y^2$ \newline
h) \ $x^2+y^2 \le 8\ \ja\ y-4 \le z \le 8-x$

\item
a)--e) Laske edellisen tehtävän tapauksissa a)--e) integraalit $\,\int_A x\,d\mu$,
$\,\int_A y\,d\mu$ ja $\int_A z\,d\mu$.

\item
Laske $\int_A f(x,y,z)\,dxdydz$ annetuilla $f$ ja $A$: \vspace{1mm}\newline
a) \ $f:\ xyz,\ \ A:\ 0 \le x \le 1\ \ja -2 \le y \le 0\ \ja\ 1 \le z \le 4$ \newline
b) \ $f:\ x^2+y^2+z^2,\ \ A:\ 0 \le x,y,z \le a\ \ (a>0)$ \newline
c) \ $f:\ (1-x-y)^5,\ \ A:\ x,y,z \ge 0\ \ja\ x+y+z \le 1$ \newline
d) \ $f:\ xyz^4,\ \ A:\ x,y,z \ge 0\ \ja\ x+y+z \le 1$ \newline
e) \ $f:\ 3+2xy,\ \ A:\ z \ge 0\ \ja\ x^2+y^2+z^2 \le 4$ \newline
f) \ $f:\ x,\ \ A:\ x,y,z \ge 0\ \ja\ (x/a)+(y/b)+(z/c) \le 1\ \ (a,b,c>0)$ \newline
g) \ $f:\ xy+z^2,\ \ A:\ 0 \le z \le 1-\abs{x}-\abs{y}$ \newline
h) \ $f:\ yz^2e^{-xyz},\ \ A:\ 0 \le x,y,z \le 1$ \newline
i) \ $\,f:\ y,\ \ A:\ 0 \le x,y,z \le 1\ \ja\ 1-y \le z \le 2-x-y$ \newline
j) \ $\,f:\ (x+y+z)^{-3},\ \ A:\ 1 \le z \le 2\ \ja\ 0 \le y \le z\ 
                                                \ja\ 0 \le x \le y+z$ \newline
k) \ $f:\ \cos x\cos y\cos z,\ \ A:\ x,y,z \ge 0\ \ja\ x+y+z \le \pi$

\item
Perustuen joukon $A\subset\R^3$ esitysmuotoon
\[
A=\{(x,y,z)\in\R^3 \ | \ (x,y)\in B\subset\R^2 \ \ja \ z\in C(x,y)\subset\R\,\}
\]
johda purkukaava
\[
\int_A f\,d\mu=\int_B\left[\int_{C(x,y)} f(x,y,z)\,dz\right]dxdy.
\]
Sovella kaavaa, kun $A=\{\,(x,y,z)\in\R^3\ | \ x,y,z \ge 0\ \ja\ x^2+y+z \le 1\,\}$ \ ja
$f(x,y,z)=xy^2$.

\item
Tetredria $K\subset\R^3$ rajoittavat tasot $z=0$, $x=2y$, $x=-y$ ja $y+z=a$ ($a>0$). Laske
$\int_K z^2\,dxdydz$.

\item
Teltan pohja on $xy$-tasolla oleva kiekko, jonka säde on $R=2$ m, ja teltan jokainen
$xz$-tason suuntainen poikkileikkaus on tasakylkinen kolmio, jonka korkeus on $h=2$ m. Laske
teltan tilavuus.

\item \label{H-uint-3: ellipsoidin tilavuus}
Näytä, että ellipsoidin $\,S: x^2/a^2+y^2/b^2+z^2/c^2=1$ sisään jäävän joukon $A\subset\R^3$
tilavuus on $\mu(A)=4\pi abc/3$ 
(vrt.\ Harj.teht.\,\ref{pinta-ala ja kaarenpituus}:\ref{H-int-8: ellipsin pinta-ala}).

\item
Olkoon $a,b>0,\ a \neq b$. Laske lieriöiden $S_1:\,x^2+y^2=a^2$ ja $S_2:\,x^2+z^2=b^2$ sisään
jäävän joukon $A\subset\R^3$ tilavuus $\mu(A)$ yksiulotteisena integraalina. Voiko integraalin
laskea suljetussa muodossa?

\item
Joukko $\{(x,y,z)\in\R^3 \mid \sqrt{x^2+y^2} \le z \le a\}$ edustaa täydessä viinilasissa
olevaa viiniä. Lasiin upotetaan varovasti $R$-säteinen kuula, jolloin viiniä valuu ulos,
kunnes kuula uppoaa kokonaan tai pysähtyy lasiin. Millä suhteen $R/a$ arvolla viiniä valuu
ulos eniten?

\item
Olkoon $A=[0,1]\times[0,1]\times\ldots\times[0,1]\subset\R^n$. Laske 
$\int_A f(\mx)\,dx_1 \ldots dx_n$, kun \vspace{1mm}\newline
a) \ $f(\mx)=\sum_{i=1}^n x_i, \quad$
b) \ $f(\mx)=\prod_{i=1}^n x_i, \quad$
c) \ $f(\mx)=e^{-(x_1+x_2+\,\cdots\,+x_n)}$.

\item
Olkoon $K\subset\R^n$ $n$-simpleksi, jonka kärjet ovat origo ja pisteet $(a,0,\ldots,0)$,
$(0,a,0,\ldots,0),\ \ldots , (0,\ldots,0,a)$ ($a>0$). Näytä, että $K$:n $n$-ulotteinen tilavuus
on $\mu(K)=a^n/n!\,$.

\item
Suuntaissärmiön $K\subset\R^4$ särmävektorit ovat $\,[1,1,1,1]^T$, $\,[0,1,1,0]^T$, \newline
$[0,0,1,1]^T$ ja $[1,1,0,1]^T$. Laske $K$:n neliulotteinen tilavuus $\mu(K)$.

\item (*)
Seuraavissa integraaleissa integroidaan erään joukon $A\subset\R^3$ yli. Määritä ensin $A$ ja
laske sitten integraalin arvo valitsemalla sopiva integroimisjärjestys.
\[
\text{a)}\ \int_0^1\left[\int_z^1\left(\int_0^x e^{x^3}\,dy\right)dx\right]dz \quad
\text{b)}\ \int_0^1\left[\int_0^{1-x}\left(\int_y^1 
                                   \frac{\sin(\pi z)}{z(2-z)}\,dz\right)dy\right]dx
\]

\item (*)
Kolmen $R$-säteisen lieriön akselit leikkaavat toisensa kohtisuorasti samassa pisteessä.
Näytä, että kaikkien kolmen lieriön sisään jäävän joukon $A$ tilavuusmitta on
$\mu(A)=8(2-\sqrt{2})R^3$.

\item (*) 
a) Laske $\R^n$:n $R$-säteisen pallon (kuulan) $n$-ulotteinen tilavuus, kun $n=5$ ja $n=6$. \
b) Millä $n$:n arvolla yksikköpallon ($R=1$) tilavuus on suurin? Perustele!

\end{enumerate} %Avaruusintegraalit
\section[Taso- ja avaruusintegraalien muuntaminen]{Taso- ja avaruusintegraalien \\ 
muuntaminen} 
\label{muuttujan vaihto integraaleissa}
\sectionmark{Integraalien muuntaminen}
\alku
\index{muuttujan vaihto (sijoitus)!b@integraalissa|vahv}

Olkoon $A,B\subset\R^n$, $n\in\N$ ja olkoon $\mpu:B \kohti A$ bijektio (tai 'melkein bijektio',
ks.\ huomautukset jäljempänä).
\begin{figure}[H]
\begin{center}
\import{kuvat/}{kuvaUint-26.pstex_t}
\end{center}
\end{figure}
Halutaan laskea integraali $\int_A f\,d\mu = \int_A f\,dx_1\ldots dx_n$ muunnetussa muodossa
\[
\int_A f\,d\mu=\int_B g\,d\mu', \quad g(\mt)=f(\mpu(\mt)), \quad \mt \in B.
\]
Tässä $g$ on kuvauksen $\mpu$ välittämä $f$:n vastine $B$:ssä, ja $\mu'$ on toinen, toistaiseksi
tuntematon $\R^n$:n mitta. Kyse on siis muuttujan vaihdosta eli \pain{si}j\pain{oituksesta}
\[
\mx=\mpu(\mt),\quad \mt\in B.
\]
Muunnoksessa on ensinnäkin määrättävä funktio $g$. Sikäli kuin lähtökohtana on
(niinkuin yleensä) tunnettu kuvaus $\mpu$, tämä on suoraviivainen toimenpide. Myös mitan
$\mu'$ formaali määrittely on helppoa. Nimittäin jos $\mpu(\Delta B)=\Delta A \subset A$ ja
valitaan $f(\mx)=1,\ \mx\in \Delta A$ (jolloin $g(\mt)=1,\ \mt\in \Delta B$), on mitan ja
integraalin yhteyden (vrt. Luku \ref{tasointegraalit}) ja oletetun integraalien välisen
yhteyden perusteella
\[
\mu'(\Delta B)=\int_{\Delta B} d\mu'=\int_{\Delta A} d\mu=\mu(\Delta A).
\]
Siis $\mu'(\Delta B)=\mu(\Delta A)$, eli $\Delta B$:n mitta määräytyy vastinjoukon
$\Delta A=\mpu(\Delta B)$ Jordan-mittana. --- Tässä on huomattava, että kuvauksen $\mpu$ on
oltava (ainakin lähes) injektio, jotta $\mu'$ olisi todella mitta. Nimittäin koska mitta on
additiivinen, niin $B$:n mitallisille osajoukoille on oltava voimassa
\[
\Delta B_1\cap\Delta B_2=\emptyset \qimpl
       \mu'(\Delta B_1\cup\Delta B_2)=\mu'(\Delta B_1)+\mu'(\Delta B_2).
\]
Jos $\mpu$ ei olisi injektio, niin joillakin $\mx_1,\mx_2 \in B$, $\mx_1\neq\mx_2$ olisi
$\mpu(\mx_1)=\mpu(\mx_2)$, tai yleisemmin $\mpu(\Delta B_1)=\mpu(\Delta B_2)=\Delta A$, missä
$\Delta B_1,\Delta B_2 \subset B$ ja $\Delta B_1 \cap \Delta B_2 =\emptyset$. Tällöin
$\mu'(\Delta B_1\cup\Delta B_2)=\mu(\Delta A)$ ja
$\mu'(\Delta B_1)+\mu'(\Delta B_2)=2\mu(\Delta A)\neq\mu(\Delta A)$, ellei ole
$\mu(\Delta A)=0$. Päätellään siis, että $\mpu$:n epäinjektiivisyys voidaan sallia
\pain{enintään} (mitan $\mu'$ suhteen) \pain{nollamittaisessa} j\pain{oukossa}.

Sikäli kuin $\mpu$ on injektio, tai mainitulla tavalla 'melkein', on integraalin
muuntamiskysymys siis periaatteessa ratkaistu. Toistaiseksi ei kuitenkaan ole selvää,
millä tavoin mitan $\mu'$ avulla voidaan käytännössä laskea (muuten kuin palaamalla 
alkuperäisiin muuttujiin). Tämä laskutekninen kysymys onkin kaikkein keskeisin, sikäli 
kuin muunnoksesta halutaan jotakin hyötyä.

\subsection{Muuntosuhde}

Tutkitaan muunnettua integraalia $\int_B g\,d\mu'$. Oletetaan, että $B$ on rajoitettu ja
$g(\mt)$ rajoitettu $B$:ssä, jolloin on
(vrt.\ Luvut \ref{tasointegraalit} ja \ref{avaruusintegraalit})
$\int_B g\,d\mu'=\int_T g_0\,d\mu'$, missä $T \supset B$ on $n$-ulotteinen suorakulmainen
perussärmiö ja $g_0=g$:n nollajatko $B$:n ulkopuolelle. Oletetaan jatkossa, että
$\mpu:\ T \kohti \mpu(T)$ on (lähes) injektio ja että $\mpu(T')$ on Jordan-mitallinen aina
kun $T' \subset T$ on Jordan-mitallinen. Tällöin jos $\mathcal{T}_h=\{T_k\}\,$ on $T$:n jako
osasärmiöihin, joiden särmien pituus on enintään $h$, niin vastaavasti kuin Jordan-mitan suhteen
integroitaessa pätee summakaava
\[
\int_T g_0\,d\mu' \,=\, \Lim_{h \kohti 0} \sum_k g_0(\mt_k)\mu'(T_k)
                  \,=\, \Lim_{h \kohti 0} \sum_k g_0(\mt_k)\mu(\mpu(T_k)),
\]
missä $\mt_k \in T_k$.
\begin{figure}[H]
\begin{center}
\import{kuvat/}{kuvaUint-27.pstex_t}
\end{center}
\end{figure}
Oletetaan nyt, että kuvaukselle $\mpu$ on määriteltävissä $T$:ssä jatkuvana funktiona
$J(\mt)$ nk.\ \kor{muuntosuhde} (mittasuhde, suurennussuhde) siten, että jokaisella
$T_k\in\mathcal{T}_h$ ja jokaisella $\mt_k \in T_k$ pätee\footnote[2]{Muuntosuhteen
määrittely voidaan yleisemmin rajoittaa särmiöihin, joille pätee $T_k \cap U=\emptyset$,
missä $\mu'(U)=0$. Tällöin muuntosuhde ei välttämättä ole koko $T$:ssä jatkuva tai edes
määritelty.}
\[
\left|\frac{\mu(\mpu(T_k))}{\mu(T_k)}-J(\mt_k)\right| \,\le\, \eps(h) \kohti 0 \quad 
                                                             \text{kun}\,\ h \kohti 0,
\]
missä $\eps(h)$ riippuu vain tiheysparametrista $h$ (ei $k$:sta eikä yleisemmin
$\mathcal{T}_h$:sta).

Tämän perusteella voidaan em.\ summakaavassa kirjoittaa likimäärin
$\mu(\mpu(T_k) \approx J(\mt_k)\mu(T_k)$. Olettaen, että $|g_0(\mt)| \le C,\ \mt \in T$
($g$ oli rajoitettu), saadaan tälle approksimatiolle tehtyjen oletusten perusteella virhearvio
\[
\left|\sum_k g_0(\mt_k)\mu(\mpu(T_k))-\sum_k g_0(\mt_k)J(\mt_k)\mu(T_k)\right|
\le C\eps(h)\sum_k\mu(T_k) = C\eps(h)\mu(T).
\]
Oletukseen $\,\lim_{h \kohti 0}\eps(h)=0\,$ perustuen ja integraalin summakaavaa uudelleen
soveltaen seuraa
\[
\Lim_{h \kohti 0} \sum_k g_0(\mt_k)\mu(\mpu(T_k)) 
          \,=\, \Lim_{h \kohti 0} \sum_k g_0(\mt_k)J(\mt_k)\mu(T_k)
          \,=\, \int_T g_0(\mt)J(\mt)\,d\mu.
\]
Tässä on edelleen $\int_T g_0(\mt)J(\mt)\,d\mu = \int_B g(\mt)J(\mt)\,d\mu$
(koska $g_0J=(gJ)_0$), joten on päätelty:
\[
\int_A f\,d\mu = \int_B g\,d\mu' = \int_B g(\mt)J(\mt)\,d\mu.
\]
Integraalia $\int_A f\,d\mu$ muunnettaessa on siis muunnosten $f(\mx)=f(\mpu(\mt))=g(\mt)$
ja $A \ext B$, $\mpu(B)=A$, lisäksi suoritettava paikalliseen muuntosuhteeseen $J(\mt)$
perustuva \pain{mittamuunnos} $d\mu'=J(\mt)\,d\mu$, eli muunnoskaava on
\begin{equation} \label{muuntokaava}
\boxed{\kehys\quad \int_A f(\mx)\,dx_1\ldots dx_n=\int_B g(\mt)J(\mt)\,dt_1\ldots dt_n. \quad}
\tag{$\star$}
\end{equation}
%Siis jos muuntosuhde $J(\mt)$ pystytään määräämään (laskutekniikasta jäljempänä), niin
%integraali $\int_A f\,d\mu,\ A\subset\R^n$ muunnetaan muodollisesti sijoituksilla
%\[
%f(\mx) = f(\mpu(\mt)) = g(\mt), \quad d\mu = J(\mt)\,dt_1\cdots dt_n, \quad 
%                                      A \ext B, \quad \mpu(B)=A. \Akehys
%\]
\begin{Exa} Olkoon $A=\{(x,y)\in\R^2 \mid x\in[1,2]\ \ja\ x \le y \le 2x\}$. Laske
$\int_A xy\,dxdy\,$ käyttäen sijoitusta $(x,y)=\mpu(\mt)=\mpu(t,s)=(t,st)$.
\end{Exa}
\ratk Ensinnäkin todetaan, että $\,A=\mpu(B)$, missä $\,B=[1,2]\times[1,2]$, ja että 
$\mpu:\,B \kohti A$ on bijektio. Muuntosuhteen laskemiseksi tarkastellaan suorakulmiota 
$\Delta T=[t,t+\Delta t]\times[s,s+\Delta s]$, missä $t,s \ge 1$ ja 
$0 < \Delta t,\Delta s \le h$. Tällöin on
\begin{align*}
\Delta A\,=\,\mpu(\Delta T)\,
         &=\{(x,y)\in\R^2 \mid t \le x \le t+\Delta t\ \ja\ sx \le y \le (s+\Delta s)x\}\\[1mm]
\impl\quad\mu(\Delta A) 
         &= \int_t^{t+\Delta t}\left(\int_{sx}^{(s+\Delta s)x}\,dy\right)dx\,
          =\,\int_t^{t+\Delta t} (\Delta s)x\,dx \\
         &=\,\frac{1}{2}\Delta s\left[(t+\Delta t)^2-t^2\right]\,
          =\,t\,\Delta t\Delta s + \frac{1}{2}(\Delta t)^2\Delta s.
\end{align*}
Koska $\mu(\Delta A)/\mu(\Delta T)=t+\frac{1}{2}\Delta t = t + \Ord{h}$, niin muuntosuhteen
määritelmän perusteella on $J(t,s)=t$. Siis kaavan \eqref{muuntokaava} mukaan
\[
\int_A f(x,y)\,dxdy = \int_B f(t,st)\,t\,dtds = \int_1^2 t^3\,dt\cdot\int_1^2 s\,ds 
                    = \frac{15}{4}\cdot\frac{3}{2} = \frac{45}{8}\,.
\]
Tarkistus (ilman muuttujan vaihtoa):
\[
\int_A xy\,dxdy = \int_1^2\left[\int_x^{2x} xy\,dy\right]dx
                = \int_1^2\left[\sijoitus{y=x}{y=2x} \frac{1}{2}xy^2\right]dx
                = \int_1^2 \frac{3}{2}x^3\,dx = \frac{45}{8}\,. \loppu
\]
Kuten esimerkissä, integraalin muuntamisen keskeisin laskutekninen ongelma on muuntosuhteen
määrääminen. Jatkossa ratkaistaan tämä ongelma differentiaalilaskennan keinoin. Aloitetaan
yksiulotteisesta integraalista.


\subsection{Muuntosuhde $\R$:ssä}
\index{muuntosuhde integraalissa!a@$\R$:ssä|vahv}

Olkoon $A=[a,b]$ ja $B=[c,d]$ suljettuja välejä ja $u:B \kohti A$ bijektio. Oletetaan, että
$u$ on jatkuvasti derivoituva välillä $B$. Tällöin jos $T_k=[t_{k-1},t_k]\subset B$, missä 
$t_k-t_{k-1} \le h$, niin jollakin $\xi_k\in T_k$ on
(Lause \ref{toinen väliarvolause}).
\[ 
\mu(u(T_k)) = |u(t_k)-u(t_{k-1})| = |u'(\xi_k)|(t_k-t_{k-1}) = |u'(\xi_k)|\mu(T_k).
\]
Tässä $u'$ (ja näin ollen myös $|u'|$) on jatkuva, joten $|u'(\xi_k)|-|u'(t)|=\ord{1}$
$\forall t \in T_k$, kun $h \kohti 0$.\footnote[2]{Päättely nojaa tässä jatkuvuuden
syvällisempään logiikkaan: Tehdyin oletuksin $|u'|$ on tasaisesti jatkuva välillä $[a,b]$
(Lause \ref{kompaktissa joukossa jatkuva on tasaisesti jatkuva}, ks.\ myös Lause
\ref{tasaisen jatkuvuuden käsite}).} Näin ollen jokaisella $t \in T_k$
\[
\mu(u(T_k)) \,=\, |u'(t)|\mu(T_k)+\ord{1}\mu(T_k), \quad \text{kun}\ h \kohti 0.
\]
Muuntosuhteen määritelmän mukaan on siis $J(t)=\abs{u'(t)}$, jolloin muunnoskaava 
\eqref{muuntokaava} saa muodon
\begin{equation} \label{muuntokaava-R1a}
\int_a^b f(x)\,dx=\int_B f(u(t))\abs{u'(t)}\,dt, \quad u(B)=[a,b]. \tag{a}
\end{equation}
Koska oletettiin, että $u:\ B \kohti [a,b]$ on jatkuva bijektio, niin $u$ on välillä $B$ joko
aidosti kasvava tai aidosti vähenevä. Tällöin jos $u(\alpha)=a$ ja $u(\beta)=b$, niin
\begin{alignat*}{2}
&u\text{ kasvava } (u'(x)\geq 0) \ &&\impl \ B=[\alpha,\beta], \\
&u\text{ vähenevä } (u'(x)\leq 0) \ &&\impl \ B=[\beta,\alpha].
\end{alignat*}
Kummassakin tapauksessa kaava \eqref{muuntokaava-R1a} voidaan kirjoittaa muotoon
\begin{equation} \label{muuntokaava-R1b}
\int_a^b f(x)\,dx=\int_B f\bigl(u(t)\bigr)\abs{u'(t)}\,dt=\int_\alpha^\beta f(u(t))u'(t)\,dt.
\tag{b}
\end{equation}
Tässä (määrätyn integraalin vaihtosääntöön perustuvassa) laskukaavassa on kyse
\pain{si}j\pain{oituksesta} \pain{määrät}y\pain{ssä} \pain{inte}g\pain{raalissa}, vrt.\ Luku
\ref{analyysin peruslause}.
Kaava \eqref{muuntokaava-R1b} on (tietyin edellytyksin, ks.\ mainittu luku) pätevä, vaikka $u$
ei olisikaan bijektio. Sen sijaan kaava \eqref{muuntokaava-R1a} \pain{ei} ole pätevä, jos $u'$
vaihtaa merkkinsä välillä $B$.

\begin{multicols}{2} \raggedcolumns
\begin{Exa} Jos integraalissa $\int_0^1 x\,dx$ tehdään sijoitus $x=u(t)=3t-2t^2$,
niin $u'(t)=3-4t$, $u(0)=0$, $u(1)=1$, ja määrätyn integraalin sijoituskaava
\eqref{muuntokaava-R1b} on pätevä:
\begin{align*}
\int_0^1 x\,dx &= \int_0^1 f(u(t))u'(t)\,dt \\
               &= \int_0^1 (9t-18t^2+8t^3)\,dt \\
               &= \sijoitus{0}{1}(\tfrac{9}{2}t^2-6t^3+2t^4) = \frac{1}{2}\,.
\end{align*}
\end{Exa}
\begin{figure}[H]
\setlength{\unitlength}{1cm}
\begin{center}
\begin{picture}(4,5)(-1,-0.5)
\put(0,0){\vector(1,0){3}} \put(2.8,-0.5){$t$}
\put(0,0){\vector(0,1){3}} \put(0.2,2.8){$x$}
\curve(0,0,1,2,2,2)
\dashline{0.1}(0,2)(2,2) \dashline{0.1}(1,0)(1,2)
\put(2,0){\line(0,-1){0.1}} \put(1.9,-0.5){$1$} \put(0.9,-0.5){$\tfrac{1}{2}$}
\put(0,2){\line(-1,0){0.1}} \put(-0.4,1.85){$1$}
\put(1.5,2.4){$x=u(t)$}
\end{picture}
\end{center}
\end{figure}
\end{multicols}
Sen sijaan kaava \eqref{muuntokaava-R1a} antaa väärän tuloksen:
\[
\int_0^1 f(u(t))\abs{u'(t)}\,dt
     =\int_0^{3/4}(9t-18t^2+8t^3)\,dt - \int_{3/4}^1(9t-18t^2+8t^3)\,dt 
     = \frac{49}{64}\,.
\]
Kummallakin tavalla saadaan oikea tulos, jos muunnetuksi integroimisväliksi valitaan 
$B=[0,\tfrac{1}{2}]$, jolloin $u:B\kohti[0,1]$ on bijektio (vrt.\ kuvio). \loppu

Esimerkistä nähdään, että määrätyn integraalin muunnoskaava \eqref{muuntokaava} antaa
yleisesti väärän tuloksen, jos muunnoskuvaus ei ole injektiivinen. Ongelma on hankalampi
useammassa dimensiossa, missä epäinjektiivisyyttä ei aina ole helppo havaita. Useammassa
dimensiossa ei ongelman poistamiseksi myöskään ole mitään yksinkertaista, määrätyn integraalin
sijoituskaavaan verrattavaa oikotietä.

\subsection{Muuntosuhde $\R^n$:ssä}
\index{muuntosuhde integraalissa!b@$\R^n$:ssä|vahv}

Tarkastellaan ensin kaksiulotteista muunnosta
\[
\begin{cases}
\,x=u(t,s), \\
\,y=v(t,s), &\mt=(t,s)\in T, \ \mpu=(u,v),
\end{cases}
\]
missä $T$ on perussuorakulmio. Oletetaan, että $u$ ja $v$ ovat jatkuvasti derivoituvia $T$:ssä
(osittaisderivaatat olemassa ja jatkuvia $T$:ssä, Määritelmä
\ref{jatkuvuus kompaktissa joukossa - Rn}). Tällöin suorakulmion
$\Delta T=[t_0+\Delta t]\times [s_0+\Delta s]\subset T$ kuvautumista voi tutkia likimäärin
linearisoivalla approksimaatiolla (vrt. Luku \ref{jacobiaani})
\[
\begin{cases}
\,x(t,s)\approx u(t_0,s_0)+u_t(t_0,s_0)(t-t_0)+u_s(t_0,s_0)(s-s_0), \\
\,y(t,s)\approx v(t_0,s_0)\,+v_t(t_0,s_0)(t-t_0)+v_s(t_0,s_0)(s-s_0).
\end{cases}
\]
Tämä on affiinikuvaus, jonka mukaisesti suorakulmio kuvautuu suunnikkaaksi. Suunnikkaan
virittävät vektorit
\begin{align*}
\vec a &= \Delta t\bigl[u_t(t_0,s_0)\vec i + v_t(t_0,s_0)\vec j\,\bigr], \\
\vec b &= \Delta s\bigl[u_s(t_0,s_0)\vec i + v_s(t_0,s_0)\vec j\,\bigr].
\end{align*}
\begin{figure}[H]
\begin{center}
\import{kuvat/}{kuvaUint-28.pstex_t}
\end{center}
\end{figure}
Kun määritellään kuvauksen $\mpu$ Jacobin matriisi (vrt.\ Luku \ref{jacobiaani})
\[
\mJ\mpu(t,s)=\begin{bmatrix} u_t & u_s \\ v_t & v_s \end{bmatrix},
\]
niin nähdään, että suunnikkaan pinta-ala on
\[
\mu(\Delta A)=\abs{\vec a \times \vec b}=\abs{\det[\mJ\mpu(t_0,s_0)]}\,|\Delta t||\Delta s|.
\]
Koska $\mu(\Delta T)=|\Delta t||\Delta s|$, niin muuntosuhde on tämän perusteella
$J(t,s)=|\det[\mJ\mpu(t,s)]|$. Oletetuin säännöllisyyehdoin tämä osoittautuu oikeaksi
tulokseksi (sivuutetaan tarkemmat perustelut), joten tasointegraalin muuntosuhde saadaan
\index{Jacobin determinantti}%
\kor{Jacobin determinantin} $\det[\mJ\mpu(t,s)]$ avulla:
\[
\boxed{\kehys\quad 
       \text{Muuntosuhde = kuvauksen $\mpu$ Jacobin determinantin itseisarvo.} \quad}
\]

Em.\  tulos on pätevä myös yleisemmin $n$-ulotteiselle avaruusintegraalille. Nimittäin jos
$\mpu:\,T \kohti \mpu(T)$ on epälineaarinen kuvaus ($T\subset\R^n$ perussärmiö), niin tämän
linearisaatio pisteessä $\mt$ on $\R^n$:n affiinikuvaus, joka kuvaa suorakulmaisen särmiön
suuntaissärmiöksi. Ko.\ suuntaissärmiön särmävektoreista muodostettu determinantti = $\mpu$:n
Jacobin (matriisin) determinantti pisteessä $\mt$. Koska determinantin itseisarvo on särmiön
tilavuus (Propositio \ref{suuntaissärmiön tilavuuskaava}), niin ym.\ sääntö on pätevä. Siis
yleistä $n$-ulotteista integraalia muunnettaessa mitan muunnoskaava on
\[
\boxed{\kehys\quad d\mu=\abs{\det(\mJ\mpu)}\,dt_1\cdots dt_n. \quad}
\]
\begin{Exa}
$A=\text{suunnikas}$, jonka kärjet ovat $(0,0)$, $(1,0)$, $(2,1)$ ja $(3,1)$. 
Laske $\,\int_A xy\,dxdy$.
\end{Exa}
\begin{multicols}{2} \raggedcolumns
\ratk Tehdään muuttujan vaihto
\[
\begin{cases}
\,x=u(t,s)=t+2s, \\
\,y=v(t,s)=s,
\end{cases}
\]
\begin{figure}[H]
\begin{center}
\import{kuvat/}{kuvaUint-29.pstex_t}
\end{center}
\end{figure}
\end{multicols}
jolloin $\mpu: B \kohti A$, $B=[0,1]\times [0,1]$ on bijektio ja
\begin{align*}
\mJ=\begin{bmatrix} 1 & 2 \\ 0 & 1 \end{bmatrix} \ &\impl \ J=\abs{\det\mJ}=1 \\
\impl \ \int_A xy\,dxdy &= \int_0^1\left[\int_0^1 (t+2s)s\,ds\right]dt \\
                        &= \int_0^1 \left(\frac{1}{2}t+\frac{2}{3}\right)\,dt
                         = \underline{\underline{\frac{11}{12}}}\,. \loppu
\end{align*}
\begin{Exa} Laske integraali
\[ 
\int_A e^{-x-y-z}\,dxdydz, \quad 
         A = \{(x,y,z) \in \R^3 \mid x+y \ge 0\ \ja\ y+z \ge 0\ \ja\ x+z \ge 0\,\}. 
\]
\end{Exa}
\ratk Tehdään $A$:n muotoon sopiva muuttujan vaihdos
\[
\begin{cases} \,u = x+y \\ \,v = y+z \\ \,w = x+z \end{cases} 
    \ekv\quad \begin{bmatrix} x\\y\\z \end{bmatrix} =
              \frac{1}{2} \begin{rmatrix} 1&-1&1\\1&1&-1\\-1&1&1 \end{rmatrix}
              \begin{bmatrix} u\\v\\w \end{bmatrix},
\]
jolloin muuntosuhde on
\[
J = \left(\frac{1}{2}\right)^3 \begin{detmatrix} 1&-1&1\\1&1&-1\\-1&1&1 \end{detmatrix} 
  = \frac{1}{2}\,.
\] 
Kuvaus $(u,v,w) \in B \map (x,y,z) \in A$ on ilmeinen bijektio, kun
$B = [0,\infty)\times[0,\infty)\times[0,\infty)$, joten muunnoskaavaa \eqref{muuntokaava}
ja Fubinin sääntöä soveltaen on tulos
\begin{align*} 
\int_A &e^{-x-y-z}\,dxdydz = \int_B e^{-\frac{1}{2}(u+v+w)}\,\tfrac{1}{2}\,dudvdw \\ 
       &= \frac{1}{2}\int_0^\infty\int_0^\infty\int_0^\infty 
           e^{-\frac{1}{2}u}e^{-\frac{1}{2}v}e^{-\frac{1}{2}w}\,dudvdw
        = \frac{1}{2}\left[\int_0^\infty e^{-\frac{1}{2}u}\,du\right]^3 
        = \underline{\underline{4}}. \loppu
\end{align*}

\subsection{Integraalit käyräviivaisissa koordinaatistoissa}
\index{tasointegraali!c@käyräv.\ koordinaateissa|vahv}
\index{avaruusintegraali!b@käyräv.\ koordinaateissa|vahv}
\index{muuntosuhde integraalissa!c@käyräv.\ koordinaatistoissa|vahv}
\index{kzyyrzy@käyräviivaiset koordinaatistot!d@--integraalit|vahv}

Integraalien muunnoskaavat tulevat fysikaalisissa sovelluksissa käyttöön useimmin silloin, kun
karteesisesta koordinaatistosta halutaan siirtyä käyräviivaiseen napa-, lieriö- tai 
pallokoordinaatistoon. Jos tasossa siirrytään napakoordinaatteihin, niin muunnoksen
\[
x=r\cos\varphi=u(r,\varphi), \quad y=r\sin\varphi=v(r,\varphi)
\]
Jacobin matriisi on
\[
\mJ=\begin{rmatrix} \cos\varphi & -r\sin\varphi \\ \sin\varphi & r\cos\varphi \end{rmatrix}.
\]
Tämän determinantin itseisarvo on $J=r$, joten muunnoskaavaksi tulee
\[
\boxed{\kehys\quad \int_A f(x,y)\,dxdy=\int_B g(r,\varphi)\,rdrd\varphi. \quad}
\]
Esimerkiksi jos $A=$ origokeskinen, $R$-säteinen kiekko, niin kaava on pätevä, kun 
$B=[0,R]\times[0,2\pi]$. Muunnoksen lievästä epäinjektiivisyydestä ei tässä ole haittaa, koska
se rajoittuu nollamittaiseen osajoukkoon ($r=0$, $\varphi=0$ tai $\varphi=2\pi$).
\begin{Exa} Laske $\displaystyle{\int_{\R^2} e^{-x^2-y^2}\,dxdy}$.
\end{Exa}
\ratk Siirrytään napkoordinaatistoon. Koordinaattimuunnos $\mpu: B \kohti \R^2$ on (melkein)
bijektio, kun valitaan $B=\{(r,\varphi)\in\R^2 \mid r \ge 0,\ 0 \le \varphi \le 2\pi\}$, joten
\begin{align*}
\int_{\R^2} e^{-x^2-y^2}\,dxdy 
&= \int_B e^{-r^2}r\,dr d\varphi \\
&= \int_0^{2\pi} \left[\int_0^\infty e^{-r^2}r\,dr\right]d\varphi
 = 2\pi \sijoitus{0}{\infty} \left(-\frac{1}{2}e^{-r^2}\right)
 = \underline{\underline{\pi}}. \loppu
\end{align*}
Esimerkin integraali laskettuna karteesisessa koordinaatistossa on
\begin{align*}
\int_{\R^2} e^{-x^2-y^2}\,dxdy &= \int_{\R^2} e^{-x^2}\cdot e^{-y^2}\,dxdy \\
&= \left[\int_{-\infty}^\infty e^{-x^2}\,dx\right]
   \left[\int_{-\infty}^\infty e^{-y^2}\,dy\right] 
 = \left[\int_{-\infty}^\infty e^{-x^2}\,dx\right]^2.
\end{align*}
Saatiin siis hauska tulos:
\begin{Prop} \label{Gamma(1/2)} $\ \int_{-\infty}^\infty e^{-x^2}\,dx=\sqrt{\pi}$.
\end{Prop}
Siirryttäessä kolmessa dimensiossa lierökoordinaatistoon on muuntosuhde sama kuin tasossa
siirryttäessä polaarikoordinaatistoon, eli $J(r,\varphi,z)=r$. Pallokoordinaatistoon
siirryttäessä saadaan muuntosuhteeksi
\[ 
J=\abs{\det\mJ\,}=\left|\begin{array}{ccc} 
                  \sin\theta\cos\varphi & r\cos\theta\cos\varphi & -r\sin\theta\sin\varphi \\
                  \sin\theta\sin\varphi & r\cos\theta\sin\varphi & r\sin\theta\cos\varphi \\
                  \cos\theta            & -r\sin\theta           & 0
                  \end{array}\right| = r^2\sin\theta. 
\]
Integraalin muunnoskaavat lieriö- ja pallokoordinaatistoon ovat näin ollen                                     
\[
\boxed{\begin{aligned}
\ykehys\quad &\text{Lieriökoord.\,:} \qquad 
              \int_A f(x,y,z)\,dxdydz = \int_B g(r,\varphi,z)\,r\,drd\varphi dz \\
             &\text{Pallokoord.\,:} \qquad\    
              \int_A f(x,y,z)\,dxdydz 
                = \int_B g(r,\theta,\varphi)\,r^2\sin\theta\,drd\theta d\varphi \quad\akehys 
\end{aligned}}
\]
\begin{Exa} \label{integraali yli pallokuoren} Laske $I=\int_A(x^2+y^2)\,dxdydz$, kun $A$ on
\index{pallokuori}%
\kor{pallokuori} eli kahden pallopinnan väliin jäävä alue:
\[
A=\{\,(x,y,z)\in\R^3\ |\ R_1^2 \le x^2+y^2+z^2 \le R_2^2\,\}.
\]
\end{Exa} 
\ratk Lasku käy helpoiten pallokoordinaattien avulla:
\begin{align*}
I &= \int_0^{2\pi}\left\{\int_0^\pi\left[\int_{R_1}^{R_2}
         r^2\sin^2\theta\cdot r^2\sin\theta\,dr\right]d\theta\right\}d\varphi \\
  &= \int_0^{2\pi}d\varphi\cdot\int_0^\pi\sin^3\theta\,d\theta\cdot\int_{R_1}^{R_2}r^4\,dr
   = 2\pi\cdot\frac{4}{3}\cdot\frac{1}{5}(R_2^5-R_1^5)
   = \underline{\underline{\frac{8\pi}{15}(R_2^5-R_1^5)}}. \loppu
\end{align*}

\Harj
\begin{enumerate}

\item
Muunna integraali $\int_A f(x,y)\,dxdy$ annettua muunnosta $\mpu$ käyttäen laskemalla ensin 
tarkasti suhde $\mu(\Delta A)/\mu(\Delta T)$, missä $\Delta T=[t,t+\Delta t,s,s+\Delta s]$ ja
$\Delta A=\mpu(\Delta T)$, ja muuntosuhde tämän raja-arvona, kun 
$\max\{\Delta t,\Delta s\} \kohti 0\,$: 
a) \ $A:\ x\in[1,2]\ \ja\ -x^2 \le y \le 3x^2,\,\ \mpu(t,s)=(t,st^2)$ \newline
b) \ $A:\ x\in[1,\infty)\ \ja\ 1/x^2 \le y \le 2/x^2,\,\ \mpu(t,s)=(t,s/t^2)$

\item
Laske sopivalla muunnoksella: \vspace{1mm}\newline
a) \ $\int_A (x+2y)^4(x-2y)^6\,dxdy,\,\ A:\ \abs{x+2y} \le 1\ \ja\ \abs{x-2y} \le 2$ \newline
b) \ $\int_A (2x+3y)^2(x-5y)^2\,dxdy,\,\ A:\ \abs{2x+3y} \le 4\ \ja\ \abs{x-5y} \le 3$
 
\item
Laske seuraavat integraalit sijoituksella muotoa $x=as\,\cos t,\ y=bs\,\sin t$.
\begin{align*}
&\text{a)}\ \ \int_{\R^2} e^{-3x^2-4y^2}\,dxdy \qquad
 \text{b)}\ \ \int_{\R^2} \frac{1}{(1+x^2+4y^2)^2}\,dxdy \\
&\text{c)}\ \ \int_A \ln\left(1+\frac{x^2}{4}+\frac{y^2}{9}\right)dxdy, \quad
              A:\ x,y \ge 0\ \ja\ 9x^2+4y^2 \le 36
\end{align*}

\item
a) Joukko $A\subset\R^2$ sijaitsee koordinaattineljänneksessä $x,y>0$ ja rajoittuu käyriin
$y^3=ax^2$, $y^3=bx^2$, $x^4=cy^3$ ja $x^4=dy^3$, missä $0<a<b$ ja $0<c<d$. Laske pinta-ala
$\mu(A)$ sijoituksella $y^3=ux^2,\ x^4=vy^3$. \vspace{1mm}\newline
b) Ratkaise Harj.teht.\,\ref{avaruusintegraalit}:\ref{H-uint-3: ellipsoidin tilavuus} muuttujan
vaihdolla $(x,y,z)=(au,bv,cw)$.

%\item 
%Kappaletta, jonka tiheys $\rho$ on vakio, rajoittavat $xy$-taso, lieriö $x^2+y^2=a^2$ ja 
%paraboloidi $x^2+y^2=az$ ($a>0$). Laske kappaleen massa ja hitausmomentit koordinaatiakselien 
%suhteen käyttäen lieriökoordinaatteja.

\item 
Laske integraali $\int_A f\,dxdydz$ lieriö- tai pallokoordinaatteihin
siirtymällä: \vspace{1mm}\newline
a)\ \ $A:\ x^2+y^2 \le z \le \sqrt{2-x^2-y^2}, \quad f(x,y,z)=z$ \newline
b)\ \ $A:\ x^2+y^2+z^2 \le 12,\ z \ge x^2+y^2, \quad f(x,y,z)=1$ \newline
c)\ \ $A:\ x^2+y^2+z^2 \le R^2,\ x,y,z \ge 0, \quad f(x,y,z)=x$ \newline
d)\ \ $A:\ x^2 + y^2 + z^2 \le R^2,\ x,y,z\ge 0, \quad f(x,y,z=xyz$ \newline
e)\ \ $A:\ 1 \le x^2+y^2 \le 4,\ 0 \le y \le x,\ 0 \le z \le 1, \quad f(x,y,z)=x^2+y^2$ \newline
f)\,\ \ $A:\ 4 \le x^2+y^2+z^2 \le 9,\ z \ge 0, \quad f(x,y,z)=z$ \newline
g)\ \ $A:\ x^2+y^2 \ge R^2,\ x^2+y^2+z^2 \le R^2, \quad f(x,y,z)=x^2+y^2$

\item 
Puolikartio $z=\sqrt{x^2+y^2}$ jakaa origokeskiset pallot kahteen osaan. Laske näiden
osien tilavuuksien suhde pallokoordinaattien avulla.

\item (*)
Olkoon $f(x,y)=e^{(y-x)/(y+x)}$ ja $A$ kolmio, jonka kärjet ovat $(0,0)$, $(1,0)$ ja $(0,1)$.
Laske integraali $\int_A f\,dxdy$ \ a) polaarikoordinaattien avulla, \linebreak
b) sijoituksella $u=y-x,\ v=y+x$.

\item (*)
Laske tasointegraali $\D \int_0^\infty\left[\int_0^x \frac{e^{-x-2y}}{x+2y}\,dy\right]dx$.

\item (*)
Laske tilavuus $\mu(A)$, kun $A\subset\R^3$ määritellään ehdoilla
\[
\frac{x^2}{a^2}+\frac{y^2}{b^2}+\frac{z^2}{c^2} \le 1\ \ja\ z+y \ge b \quad (a,b,c>0).
\]

\item (*)
Olkoon $K$ suuntaissärmiö, jonka yksi kärki on origossa ja origosta lähtevien särmien toiset
päätepisteet ovat $(2,1,1)$, $(-1,2,2)$ ja $(0,-2,1)$. Laske $\int_K xye^z\,dxdydz\,$ sopivalla
muuttujan vaihdolla.

\end{enumerate}  %Taso-ja avaruusintegraalien muuntaminen
\section[Integraalien sovellukset: Tiheys ja kokonaismäärä]
{Integraalien sovellukset: Tiheys ja  \\ kokonaismäärä} 
\label{pinta- ja tilavuusintegraalit}
\sectionmark{Integraalien sovellukset}
\alku

Fysikaalisissa yhteyksissä käytetään tasointegraaleista usein nimeä 
p\pain{intainte}g\pain{raali} ja $\R^3$:n avasuusintegraalista nimeä 
\pain{tilavuusinte}g\pain{raali}. 
\begin{multicols}{2}
Tasointegraalista tulee 'pintaintegraali', kun taso ajatellaan $\R^3$:n (tai $\Ekolme$:n)
avaruustasoksi ja pinta-alamitta ko.\ tasolla määritellyksi Jordan-mitaksi. Fysikaalisesti $A$
voi olla esimerkiksi levymäisen kappaleen sivupinta tai yleisemmän kolmiulotteisen kappaleen
ulkopinnan tasomainen osa.  
\begin{figure}[H]
\begin{center}
\import{kuvat/}{kuvaUint-19.pstex_t}
\end{center}
\end{figure}
\end{multicols}
Ym.\ tavalla  ymmärrettynä tasointegraalista tulee myös matemattisena käsitteenä erikoistapaus 
yleisemmästä \kor{pintaintegraalista} (engl.\ surface integral), jossa pinta voi olla kaareva.
Pintaintegraaleja tässä yleisemmässä merkityksessä käsitellään edempänä Luvussa 
\ref{pintaintegraalit}. Myös yksiulotteinen integraali voidaan tulkita 'avaruudellisesti'
ajattelemalla, että kyseessä on avaruussuoralla määritelty pituusmitta. Näin ymmärrettynä 
yksiulotteinen integraali on erikoistapaus avaruuden (myös tason) käyrään liitettävästä 
\kor{viiva-} eli \kor{käyräintegraalista}. Näitä käsitellään seuraavassa luvussa.

Fysikaalisissa sovelluksissa sanotaan yleensä \pain{alueeksi} (tilavuusintegraalin tapauksessa
joskus 'tilavuudeksi') joukkoa, jonka yli integroidaan. Termi \kor{alue} (engl.\ domain tai 
region) on myös matematiikassa tietyn tyyppisistä joukoista käytetty termi. Jatkossa 
ymmärettäköön alue kuitenkin 'fysikaalisena joukkona'.
 
\subsection{Tiheys ja kokonaismäärä}
\index{tiheys(funktio)|vahv} \index{kokonaismäärä|vahv}

Olkoon $A\subset\R^d$ joukko tai fysikaalisemmin alue, missä $d\in\{1,2,3\}$. Tyypillisessä
integraalien sovellustilanteessa tunnetaan fysikaalisen suureen \kor{tiheys} 
(tiheysfunktio, jakauma) alueessa $A$ ja tehtävänä on laskea suureen \kor{kokonaismäärä}
kaavasta \index{kokonaismäärä!a@integraalikaava}%
\[
\boxed{\kehys\quad \text{Kokonaismäärä $A$:ssa = tiheyden integraali yli $A$:n.} \quad}
\]
Jos tiheysfunktio $=f$ ja kokonaismäärää merkitään symbolilla $F$, niin laskukaava on siis
\[
F=\int_A f\,d\mu,
\]
missä $\mu$ on $\R^d$:n Jordan-mitta. Tiheysfunktio ja kokonaismäärä voivat olla myös
vektoriarvoisia: Esimerkiksi jos $A\subset\R^3$ ja tiheysfunktio on $A$:ssa määritelty
vektorikettä $\vec f(x,y,z)=f_1(x,y,z)\vec i+f_2(x,y,z)\vec j+f_3(x,y,z)\vec k$, niin 
kokonaismäärä on
\[ 
\vec F=\int_A \vec f\,d\mu=\vec i\int_A f_1\,dxdydz+\vec j\int_A f_2\,dxdydz
                                                   +\vec k\int_A f_3\,dxdydz.
\]

Sovellusesimerkkejä em.\ ajattelusta ovat vaikkapa seuraavat:
\begin{center}
\begin{tabular}{|l|l|}
\hline
\ykehys\ tiheys & kokonaismäärä \\ \hline & \\
$\rho=\text{massatiheys}$ $[\text{kg}/\text{m}^d]$ 
& $m=\text{kokonaismassa}$ $[\text{kg}]$ \\ & \\
$\sigma=\text{varaustiheys}$ $[\text{C}/\text{m}^d]$ 
& $Q=\text{kokonaisvaraus}$ $[\text{C}]$ \\ & \\
$\vec f=\text{voimatiheys}$ $[\text{N}/\text{m}^d]$ 
& $\vec F=\text{kokonaisvoima}$ $[\text{N}]$ \\ & \\ \hline 
\end{tabular}
\end{center}
Jos fysikaalinen suure on jakautunut tasomaiselle (tai yleisemmälle, ks.\ Luku
\index{tiheys(funktio)!a@pinta-, viivatiheys}%
\ref{pintaintegraalit}) pinnalle, niin tiheyttä sanotaan \kor{pintatiheydeksi}
(esim. pintavaraustiheys, yksikkö $\text{C}/\text{m}^2$). Suoralle 
(tai yleisemmälle käyrälle, ks.\ Luku \ref{viivaintegraalit}) jakautuneen suureen
yhteydessä puhutaan vastaavasti \kor{viivatiheydestä} (esim.\ massan viivatiheys
langassa, yksikkö kg/m).
\begin{Exa} Ilman tiheys korkeudella $x$ maanpinnasta olkoon $\rho(x)=\rho_0 e^{-x/a}$, missä
$\rho_0=1\ \text{kg}/\text{m}^3$ ja $a=10$ km. Kuinka paljon ilmaa (yksikkö = kg) on
kuvitellussa, pystysuorassa ja $100$ km korkeassa putkessa, jonka poikkipinta-ala 
$=1\ \text{m}^2\,$?
\end{Exa}
\ratk Ilman massatiheys (viivatiheys) kuvitellussa putkessa on 
\[
f(x)=1\ \text{m}^2 \cdot \rho(x)=e^{-x/a}\,\frac{\text{kg}}{\text{m}},
\]
joten kokonaismassa on
\begin{align*}
m &= 1\,\frac{\text{kg}}{\text{m}} \cdot \int_0^{100\,\text{km}} e^{-x/a}\,dx \\
  &\approx 1\,\frac{\text{kg}}{\text{m}} \cdot \int_0^\infty e^{-x/a}\,dx
   = 1\,\frac{\text{kg}}{\text{m}} \cdot a  
   = 1\,\frac{\text{kg}}{\text{m}} \cdot 10000\,\text{m}
   = \underline{\underline{10000\,\text{kg}}}. \loppu
\end{align*}
\begin{Exa}
Neliön muotoisella metsäaukiolla $A=[0,a]\times[0,a]$ on lumi jakautunut epätasaisesti siten,
että massan pintatiheys on
\[
\rho(x,y)=36\rho_0\left[1+\left(\frac{x+y}{a}\right)^2\right], \quad 
                                     \rho_0=1 \ \text{kg}/\text{m}^2.
\]
Suuriko on lumen kumulatiivinen sademäärä $[\text{cm}]$, jos tasan jakautuneelle lumelle pätee 
$1\ \text{kg}/\text{m}^2 \vastaa 1 \ \text{cm}$?
\end{Exa}
\ratk Lumen kokonaismassa on
\begin{align*}
m &= \int_A \rho\,dxdy 
   = 36\rho_0\int_0^a\int_0^a \left[1+\left(\frac{x+y}{a}\right)^2\right]\,dxdy \\
  &= 36\rho_0\int_0^a\left\{\sijoitus{y=0}{y=a} 
                     \left[y+\frac{a}{3}\left(\frac{x+y}{a}\right)^3\right]\right\}\,dx \\
  &= 36\rho_0\int_0^a\left[a+\frac{1}{3a^2}(x+a)^3-\frac{1}{3a^2}x^3\right]dx \\
  &= 36\rho_0\sijoitus{0}{a} \left[ax+\frac{1}{12a^2}(x+a)^4-\frac{1}{12a^2}x^4\right] \\
  &= 78\rho_0a^2 \vastaa \underline{\underline{78 \ \text{cm}}} \quad 
                                       (\rho_0a^2 \vastaa 1 \ \text{cm}). \loppu
\end{align*}
\begin{Exa} Puolipallon muotoisessa hiekkakasassa $A:\,x^2+y^2+z^2 \le R^2,\ z \ge 0$
on hiekan massatiheys
\[
\rho(x,y,z) = \rho_0+\rho_1(x,y,z)=\rho_0+0.036\rho_0(1-z/R),
\]
missä $\rho_0$ (= vakio) on kuivan hiekan ja $\rho_1$ on hiekkaan sitoutuneen veden
tiheys. Montako prosenttia hiekkakasan koko massasta on vettä?
\end{Exa}
\ratk Kuivan hiekan massa on
\begin{align*}
m_0 &= \int_A \rho_0\,dxdydz = \rho_0\mu(A)=\frac{2}{3}\pi \rho_0R^ 3 
     \approx 0.667\pi\rho_0 R^3
\intertext{ja veden (vrt.\ Esimerkki \ref{avaruusintegraalit}:\,\ref{puolipallon momentti})}
m_1 &= \int_A \rho_1\,dxdydz = 0.036\rho_0\left(\frac{2}{3}\pi R^3-\frac{1}{4}\pi R^3\right)
     = 0.015\pi\rho_0 R^3.
\end{align*}
Veden suhteellinen osuus on siis $0.015/0.682 \approx 0.022$. Vastaus: $2.2\,\%$. \loppu

\subsection{Kokonaismäärän aksioomat}
\index{kokonaismäärä!b@aksioomat|vahv}

Em.\ esimerkeissä pidettiin tiheyden ja kokonaismäärän välistä integraalikaavaa
'annettuna' eli fysikaalisena lähtöoletuksena. Integraalikaava on kuitenkin mahdollista 
johtaa paljon yksinkertaisemmista oletuksista, joita voidaan pitää edellä käytettyjen
matemaattisten mallien yhteisinä perusaksioomina. Olkoon tiheysfunktio $f$ ja merkitään
kokonaismäärää $A$:ssa ($A\subset\R^d$) symbolilla $F(A)$. Oletetaan:
\index{additiivisuus!c@kokonaismäärän} \index{vertailuperiaate!c@kokonaismäärien}%
\begin{itemize}
\item[A1.] \pain{Additiivisuus}: \ Jos $\mu(A \cap B)=0$, niin $F(A \cup B)=F(A)+F(B)$.
\item[A2.] \pain{Vertailu}p\pain{eriaate}: \ Jos jokaisella $\mx \in A$ pätee
           $m \le f(\mx) \le M$, niin \newline $m\,\mu(A) \le F(A) \le M\mu(A)$.
\end{itemize}
Jos nyt $A\subset\R^d$ ja halutaan määrätä kokonaismäärä $F(A)$, niin tulkitaan
tiheysfunktio $f$ määritellyksi $A$:n ulkopuolella nollajatkona $f_0$. Tällöin jos 
$T \supset A$ on suljettu väli\,/\,perussuorakulmio\,/\,suorakulmainen perussärmiö ja
$\mathcal{T}_h$ on $T$:n jako, niin oletuksista A1--A2 (kun $f=f_0$) seuraa
\[
\underline{\sigma}_h(f_0,\mathcal{T}_h) \le F(A) \le \overline{\sigma}_h(f_0,\mathcal{T}_h),
\]
missä $\underline{\sigma}_h(f_0,\mathcal{T}_h)$ ja $\overline{\sigma}_h(f_0,\mathcal{T}_h)$
ovat jakoon $\mathcal{T}_h$ liittyvät ala- ja yläsummat 
(vrt.\ Luvut \ref{riemannin integraali} ja \ref{tasointegraalit}). Koska tämä arvio on pätevä
jokaisella $\mathcal{T}_h$, niin sikäli kuin $f$ on integroituva yli $A$:n, on oltava
$F(A) = \int_A f\,d\mu$. Integraalikaava siis seuraa oletetuista (sovelluksissa yleensä
helposti hyväksyttävistä) aksioomista A1--A2.
\begin{Exa} Olkoon $A=[a,b]$ ja $F(A)$ käyrän $y=f(x)$ ja $x$-akselin väliin
jäävän tasoalueen pinta-ala. Tällöin jos $f(x) \ge 0\ \forall x \in A$ ja $f$ on 
integroituva välillä $A$, niin aksioomista A1--A2 seuraa integraalikaava 
$F(A)=\int_A f\,dx$. --- Vrt.\ Luvun \ref{pinta-ala ja kaarenpituus} tarkastelut.
\end{Exa} 

\subsection{Momentti}
\index{momentti (voiman)|vahv}
\index{kokonaismäärä!c@momentti|vahv}

Jos $\vec f=f_1\vec i+f_2\vec j+f_3\vec k$ on voimatiheys $\R^3$:ssa, niin ko.\ voimien
\pain{momenttitihe}y\pain{s} avaruuden pisteen 
$P_0\vastaa \vec r_0=x_0\vec i+y_0\vec j+z_0\vec k$ suhteen pisteessä $(x,y,z)$ on 
$\,\vec m=(\vec r-\vec r_0)\times\vec f$, missä $\vec r=x\vec i+y\vec j+z\vec k$.
Tämän mukaisesti alueeseen $A\subset\R^3$ jakautuneiden voimien \pain{kokonaismomentti}
pisteen $P_0$ suhteen on
\[
\vec M = \int_A \vec m\,d\mu
       = \int_A(\vec r-\vec r_0)\times\vec f\,d\mu,
\]
missä $\mu$ on $\R^3$:n tilavuusmitta. Kaava on pätevä myös, jos $\vec f$ on pintatiheys
avaruustasolla $T$ ($A \subset T$) tai viivatiheys avaruussuoralla $S$ ($A \subset S$),
jolloin $\mu$ on vastaavasti $2$-ulotteinen tai $1$-ulotteinen Jordan-mitta. Esimerkiksi
jos $\vec f$ on pintatiheys $xy$-tason alueessa $A$ ja $P_0=$ origo, niin kokonaismomentin
laskukaava on
\begin{align*}
\vec M &= \int_A (x\vec i+y\vec j\,)\times(f_1\vec i+f_2\vec j+f_3\vec k\,)\,dxdy \\
       &= \vec i\int_A yf_3(x,y)\,dxdy-\vec j\int_A xf_3(x,y)\,dxdy
                                      +\vec k \int_A [xf_2(x,y)-yf_1(x,y)]\,dxdy.
\end{align*}

\begin{Exa} \vahv{Sulkuportti}. \index{zza@\sov!Sulkuportti}
Kanavan sulkuportti on neliön muotoinen, sivun pituus 4 metriä. Portti aukeaa oven tavoin. 
Portin toisella puolen on vettä koko portin korkeudella (4\, m), toisella puolella ei ole
vettä. Määritä porttiin kohdistuva kokonaisvoima ja porttia auki vääntävä momentti $M$.
\end{Exa}
\begin{multicols}{2} \raggedcolumns
\ratk
Porttiin kohdistuu normaalin suuntainen paine (ks.\ kuvio)
\begin{align*}
&\vec f(x,y)=\rho_0g(a-y)\vec k, \quad (x,y)\in A, \\
&A=[0,a]\times [0,a], \quad a=4\,\text{m}, \\
&\rho_0g=1000\,G/\text{m}^3, \quad G \approx 9.8\,\text{N}.
\end{align*}
Porttiin kohdistuva kokonaisvoima on paineen (pintatiheyden) integraali:
\begin{figure}[H]
\begin{center}
\import{kuvat/}{kuvaUint-21.pstex_t}
\end{center}
\end{figure}
\end{multicols}
\begin{align*}
\vec F=\int_A\vec f\, dxdy 
&= 1000\,G\,\text{m}^{-3}\vec k\int_0^a\left[\int_0^a (a-y)\,dy\right]dx \\
&=500\,G\,\text{m}^{-3}a^3\vec k=\underline{\underline{32000\,G\vec k}}.
\end{align*}
Momentin (porttia kiinni vääntävänä positiivinen) $\vec j$-komponentti on
\begin{align*}
M_y &= -\int_A xf_3(x,y)\,dxdy 
     = -1000\,G\,\text{m}^{-3}\int_0^a\left[\int_0^a x(a-y)\,dy\right]dx \\
    &=-1000\,G\,\text{m}^{-3}\cdot\frac{1}{4}a^4=-64000\,G\,\text{m}.
\end{align*}
Siis luukkua auki vääntävä momentti on  
\[
M = -M_y \approx \underline{\underline{6.3\cdot 10^5\,\text{Nm}}}. \loppu
\]

\subsection{Painopiste. Keskiö}
\index{painopiste|vahv}
\index{keskizzb@keskiö (joukon)|vahv}
\index{kokonaismäärä!d@painopiste, keskiö|vahv}

Tavallisin esimerkki avaruuteen jakautuneesta voimasta on g\pain{ravitaatio},
jonka tiheys on
\[
\vec f(x,y,z)=\rho(x,y,z)g\vec e,\quad (x,y,z)\in A.
\]
Tässä $A$ voi edustaa esimerkiksi kiinteää kappaletta, $\rho=\text{massatiheys}$ 
$[\text{kg}/\text{m}^3]$, $g$ on gravitaation kiihtyvyys (maan pinnalla 
$g\approx 9.8\,\text{m}/\text{s}^2$), ja $\vec e=$ gravitaatiovoiman suuntavektori 
(yksikkövektori). Kokonaisvoiman
\[
\vec G=\int_A\vec f\,dxdydz=mg\vec e,\quad m=\int_A\rho\,dxdydz
\]
ohella kiinnostava on gravitaatiovoimien kokonaismomentti, joka pisteen 
$P_0 \vastaa \vec r_0=x_0\vec i+y_0\vec j+z_0\vec k$ suhteen on
\[
\vec M = \int_A(\vec r-\vec r_0)\times\vec e\,\rho g\,dxdydz 
       = g\left(\int_A (\vec r-\vec r_0)\rho\,dxdydz\right)\times\vec e.
\]
Pistettä $P_0$ sanotaan kappaleen p\pain{aino}p\pain{isteeksi} (engl. center of gravity),
jos painovoimien momentti $P_0$:n suhteen $=\vec 0$ riippumatta vektorista $\vec e$ 
(eli riippumatta kappaleen asennosta suhteessa painovoimakenttään). Näin on täsmälleen
kun
\[
\vec 0 = \int_A (\vec r-\vec r_0)\rho\,dxdydz 
       = \int_A \rho\vec r\,dxdydz - \vec r_0 \int_A \rho\,dxdydz,
\]
joten painopisteen paikkavektori on
\[
\boxed{\kehys\quad \vec r_0 = \frac{1}{m(A)}\int_A \rho\vec r\,dxdydz, \quad 
                                             m(A)=\int_A \rho\,dxdydz. \quad}
\]
Tässä $m(A)=$ kappaleen massa. Painopisteen koordinaatit ovat siis
\[
x_0=\frac{1}{m(A)}\int_A x\rho(x,y,z)\,dxdydz,\quad\text{jne.}
\]
Jos $\rho=\rho_0=$ vakio, niin painopisteen paikkavektori on
\[
\vec r_0 = \frac{1}{\mu(A)}\int_A \vec r\,dxdydz \quad \text{($\rho=$ vakio)}.
\]
Tämän matemaattinen yleistys on joukon $A \subset \R^n$ \kor{keskiö}, joka määritellään
\[
\mx_0 = \frac{1}{\mu(A)}\int_A \mx\,d\mu 
      = \frac{1}{\mu(A)}\sum_{i=1}^n\left(\int_A x_i\,d\mu\right)\me_i,
\]
missä $\mu$ on $n$-ulotteinen Jordan-mitta. Koska funktiot $f_i(\mx)=x_i$ epäilemättä ovat
integroituvia jokaisen mitallisen joukon yli
(Lause \ref{jatkuvan funktion integroituvuus Rn:ssä}), niin joukon $A \subset \R^n$ keskiö on 
määritelty aina kun $A$ on mitallinen ja $\mu(A) \neq 0$.
\begin{Exa} Määritä seuraavien joukkojen keskiöt. 
\begin{align*}
&\text{a)}\,\ A=\{(x,y,z)\in\R^3 \ | \ x^2+y^2\leq R^2 \ \ja \ x\geq 0 \ 
                                                          \ja \ z\in [-H/2,H/2]\,\} \\
&\text{b)}\,\ A=\{(x,y,z)\in\R^3 \ | \ x^2+y^2+z^2\leq R^2 \ \ja \ x\geq 0\,\}
\end{align*}
\end{Exa}
\ratk a)\ Tässä on $\mu(A)=\frac{1}{2}\pi R^2H$ (kuten integroimalla helposti selviää), ja
symmetriasyistä $\int_A y\,dxdydz=\int_A z\,dxdydz=0$, joten $y_0=z_0=0$. Keskiön
$x$-koordinaatti on lieriökoordinaateilla laskien
\begin{align*}
x_0  = \frac{1}{\mu(A)}\int_A x\,dxdydz
    &= \frac{2}{\pi R^2H}\int_0^R\left\{\int_{-\pi/2}^{\pi/2}\left[\int_{-H/2}^{H/2}
         r\cos\varphi \cdot r\,dz\right]d\varphi\right\}dr \\
    &= \frac{2}{\pi R^2H}\cdot\int_0^R r^2\,dr\cdot\int_{-\pi/2}^{\pi/2}\cos\varphi\,d\varphi
                                             \cdot\int_{-H/2}^{H/2} dz \\
    &= \frac{2}{\pi R^2H} \cdot \frac{1}{3}R^3 \cdot 2 \cdot H
     =\underline{\underline{\frac{4}{3\pi}R}}.
\end{align*}
b)\ \, Tässäkin on $y_0=z_0=0$, ja
(ks.\ Esimerkit \ref{avaruusintegraalit}:\,\ref{3-pallon tilavuus}--\ref{puolipallon momentti})
\[
x_0=\frac{1}{\mu(A)}\int_A x\,dxdydz = \frac{3}{2\pi R^3}\cdot\frac{\pi R^4}{4} 
                                     = \underline{\underline{\frac{3}{8}\,R}}. \loppu
\]

\subsection{Pappuksen sääntö}
\index{Pappuksen (Pappoksen) sääntö|vahv}

Olkoon $A$ $\,xy$-tason alue, jolle pätee $y \ge 0\ \forall (x,y) \in A$.
Halutaan laskea sen pyörähdyskappaleen $V$ tilavuus, joka syntyy $A$:n pyörähtäessä
$x$-akselin ympäri. Olkoon $B=\{x\in\R \mid (x,y)\in A\ \text{jollakin}\ y\in\R\}$.
Tällöin $V$ voidaan esittää muodossa
\begin{align*}
&V = \{(x,y,z)\in\R^3 \mid x \in B\,\ja\,(y,z) \in C(x)\}, \\
&\text{missä}\ \ C(x) = \{(y,z)\in\R^2 \mid \sqrt{y^2+z^2} \in D(x)\}, \\[1mm]
&\text{missä}\ \ D(x) =\{y\in\R \mid (x,y) \in A\} \subset [0,\infty).
\end{align*}
Tämän perusteella on ensinnäkin (vrt.\ Luku \ref{avaruusintegraalit})
\[
\mu(V)=\int_B \mu\bigl(C(x)\bigr)\,dx.
\]
Koska tässä $C(x)$ on ympyräsymmetrinen $yz$-tason origon suhteen, niin
napakoordinaattimuunnoksella $y=r\cos\varphi,\ z=r\sin\varphi$ saadaan
pinta-ala $\mu\bigl(C(x)\bigr)$ lasketuksi muodossa
\[
\mu\bigl(C(x)\bigr)=\int_{r \in D(x)}\int_{\varphi=0}^{2\pi} r\,drd\varphi
                   =2\pi\int_{D(x)} r\,dr
                   =2\pi\int_{D(x)} y\,dy.
\]
Näin ollen
\[
\mu(V) = 2\pi\int_B\left[\int_{D(x)} y\,dy\right]dx = 2\pi\int_A y\,dxdy.
\]
Tässä voidaan vielä kirjoittaa $\int_A y\,dxdy = y_0\mu(A)$, missä $\mu(A)=$ $A$:n
pinta-ala ja $y_0=$ $A$:n keskiön $y$-koordinaatti. Näin muodoin saadaan \kor{Pappuksen}
(Pappoksen) \kor{sääntönä}\footnote[2]{Pappuksen sääntöä on sanottu myös
\kor{Guldinin säännöksi} sveitsiläisen matemaatikon \hist{Paul Guldin}in (1577-1643) mukaan.
Säännön keksi kuitenkin kreikkalainen \hist{Pappos} jo 300-luvulla. \index{Guldin, P.|av}
\index{Guldinin sääntö|av} \index{Pappos|av}} tunnettu laskukaava
\[
\boxed{\kehys\quad \mu(V)=\mu(A) \cdot s, \quad s=2\pi y_0 \quad 
                                          \text{(Pappuksen sääntö)}. \quad}
\]
Tässä $s=\,$ $A$:n keskiön pyörähdyksessä kulkema matka.
\begin{Exa} Kun ympyräviiva $\,K:\ x^2+(y-R)^2=a^2$, missä $R \ge a$, pyörähtää 
$x$-akselin ympäri, niin syntyvän pyörähdyspinnan (toruksen) sisään jäävän alueen
tilavuus on Pappuksen säännön mukaan $\mu(V)=\pi a^2 \cdot 2\pi R = 2\pi^2 a^2 R$.
\loppu
\end{Exa}  

\subsection{Massamitta. Jordan-mitan yleistykset}
\index{massamitta|vahv}
\index{mitta, mitallisuus!b@massamitta|vahv}

Jos $A \subset \R^3$ edustaa kolmiulotteista kappaletta, jonka massatiheys $\rho$ ei ole vakio,
niin kappaleen painopiste määritellään
\[
\vec r_0=\frac{1}{m(A)}\int_A \vec r\,\rho\,dxdydz, \quad m(A) = \int_A \rho\,dxdydz.
\]
Kun tässä kirjoitetaan $dm=\rho\,dxdydz$, niin painopisteen lauseke voidaan esittää hieman
elegantimmin muodossa
\[
\vec r_0=\frac{1}{m(A)}\int_A \vec r\,dm, \quad m(A)=\int_A dm.
\]
Näin tulee määritellyksi massatiheyteen $\rho$ liitettävä \pain{massamitta} $m$. Kyseessä on 
todellakin (additiivinen ja dimensiottomana positiivinen) mitta, joka siis $A$:n tilavuuden
sijasta mittaa $A$:n sisältämää kokonaismassaa. 

Yleisemmin jos $\rho(\mx)$ on $\R^n$:ssä määritelty ei-negatiivinen funktio, joka on 
integroituva yli joukon $A \subset \R^n$, niin $A$:lle voidaan  määritellä mitta
\[ 
\mu_\rho(A) = \int_A \rho\,d\mu = \int_A d\mu_\rho. 
\]
\index{tiheys(funktio)}%
Funktiota $\rho$ sanotaan tällöin \kor{mitan} $\mu_\rho$ \kor{tiheysfunktioksi}. Rajoitettu
joukko $A \subset \R^n$ katsotaan $\mu_\rho$-mitalliseksi, kun $\rho$ on integroituva 
(tavallisessa tai laajennetussa mielessä) yli $A$:n. --- Jordanin pituus-, pinta-ala- ja
tilavuusmitat voidaan siis nähdä yleisempien mittojen erikoistapauksina: Jordanin mitan 
mukainen joukon 'massasisältö' saadaan asettamalla mitan tiheysfunktioksi $\rho=1$.

\subsection{Hitausmomentti} 
\index{hitausmomentti|vahv}
\index{kokonaismäärä!e@hitausmomentti|vahv}

Jos $A\subset\R^3$ edustaa kiinteää kappaletta ja $S$ on avaruussuora, niin kappaleen
\pain{hitausmomentti} suoran $S$ suhteen määritellään
\[
I_S=\int_A r^2\rho\,dxdydz = \int_A r^2\,dm,
\]
missä $m$ on edellä määritelty massamitta ($\rho=$ massatiheys) ja $r=r(x,y,z)$ on pisteen 
$(x,y,z)$ etäisyys suorasta $S$. Hitaustiheys kappaleessa on siis
$f=r^2\rho$.\footnote[2]{Hitausmomentti määrittelee fysikaalisesti kappaleen pyörimishitauden
pyörimisakselin $S$ suhteen. Jos $\theta(t)$ on pyörimiskulma ajan hetkellä $t$, niin
pyörimisliikkeen (Newtonin) yhtälö on $I_S\theta''=M_S$, missä $M_S$ on kappaleeseen vaikuttava
momentti $S$:n suuntaan.} 
\begin{multicols}{2} \raggedcolumns
Jos suora kulkee pisteen $P_0\vastaa\vec r_0$ kautta ja sen suuntavektori on yksikkövektori 
$\vec e$, niin (vrt.\ kuvio)
\[
r(x,y,z)=\abs{\vec e\times(\vec r-\vec r_0)}.
\]
\begin{figure}[H]
\begin{center}
\import{kuvat/}{kuvaUint-25.pstex_t}
\end{center}
\end{figure}
\end{multicols}
Olkoon edellä origo kappaleen painopiste ja merkitään $R=$ origon etäisyys suorasta $S$, jolloin
\[ 
\vec e\times(\vec r - \vec r_0) = \vec e\times\vec r - R\vec n, \quad \abs{\vec n\,}=1 
\]
($\vec n=$ vakiovektori), ja näin ollen
\[
r^2\ =\ (\vec e\times\vec r - R\vec n)\cdot(\vec e\times\vec r - R\vec n)\
     =\ \abs{\vec e\times\vec r\,}^2 - 2R\,\vec n\cdot\vec e\times\vec r + R^2.
\]
Kun sijoitetaan tämä em.\ integraalikaavaan, niin saadaan
\[ 
I_S = \int_A \abs{\vec e\times\vec r\,}^2\,dm - 2R\vec n\cdot\vec e\times\int_A \vec r\,dm 
                                              + R^2\int_A dm. 
\]
Tässä ensimmäinen termi $=$ kappaleen hitausmomentti origon kautta kulkevan, suoran $S$ 
suuntaisen suoran $S'$ suhteen, toinen termi $=0$, koska origo on kappaleen painopiste, ja 
kolmas termi $=mR^2$, missä $m=m(A)$ on kappaleen massa. Hitausmomentille on näin saatu 
\kor{Steinerin sääntönä} tunnettu palautuskaava
\index{Steinerin sääntö}%
\[
\boxed{\kehys\quad I_S = I_{S'} + mR^2 \quad \text{(Steinerin sääntö).} \quad } 
\]
Tässä siis $S$ ja $S'$ ovat yhdensuuntaiset suorat, $S'$ kulkee kappaleen painopisteen kautta,
$R=$ suorien välinen etäisyys, ja $m=$ kappaleen massa.
\begin{Exa} Homogeenisen, $R$-säteisen teräskuulan massa $=m$. Laske kuulan hitausmomentti 
a) kuulan keskipisteen kautta kulkevan suoran $S$, \ b) kuulaa sivuavan suoran $S'$ suhteen.
\end{Exa}
\ratk a)\ Symmetriasyistä hitausmomentti on $S$:n suuntavektorista $\vec e$ riippumaton. Kun
valitaan $\vec e=\vec k$ ja huomioidaan, että $m=\rho_0\cdot\frac{4}{3}\pi R^3$, missä $\rho_0$
on kuulan massatiheys, niin (ks.\ edellisen luvun Esimerkki \ref{integraali yli pallokuoren})
\[
I_S = \int_A (x^2+z^2)\rho_0\,dxdydz = \rho_0\cdot\frac{8}{15}\pi R^5
                                     =\underline{\underline{\frac{2}{5}\,mR^2}}.
\]
b)\ Origo on kappaleen painopiste, joten a-kohdan ja Steinerin säännön mukaan
\[ 
I_{S'} = \frac{2}{5}\,mR^2 + mR^2 = \underline{\underline{\frac{7}{5}\,mR^2}}. \loppu 
\]

\subsection{*Hitaustensori}
\index{hitaustensori|vahv}
\index{tensori!a@hitaustensori|vahv}

Tarkastellaan vielä kappaleen hitausmomentin $I_S$ laskemista matriisialgebran keinoin. Jos
suoran $S$ suuntavektori (yksikkövektori) on $\vec e = e_1\vec i + e_2\vec j + e_3\vec k$, niin
\begin{align*}
\vec e\times\vec r &= \vec e \times(x\vec i + y\vec j + z\vec k) \\
                   &= (e_2 z - e_3 y)\vec i + (e_3 x - e_1 z)\vec j + (e_1 y - e_2 x)\vec k,
\end{align*}
joten
\begin{align*}
I_S = \int_A \abs{\vec e\times\vec r\,}^2\,dm 
   &= \int_A\left[(e_2 z - e_3 y)^2 + (e_3 x - e_1 z)^2 + (e_1 y - e_2 x)^2\right]\,dm \\
   &= \sum_{i=1}^3\sum_{j=1}^3 I_{ij}e_ie_j = \me^T\mI\,\me,
\end{align*}
missä $\me^T = [e_1,e_2,e_3]$ ja $\mI = (I_{ij})$ on symmetrinen nk.\ \pain{hitausmatriisi},
jonka alkiot ovat
\begin{align*}
I_{11}&=\int_A (y^2+z^2)\,dm, \quad I_{22}=\int_A (x^2+z^2)\,dm, \quad 
                                    I_{33}=\int_A (x^2+y^2)\,dm, \\
I_{12}&=I_{21}=-\int_A xy\,dm, \quad I_{13}=I_{31}=-\int_A xz\,dm, \quad 
                                     I_{23}=I_{32}=-\int_A yz\,dm.
\end{align*}
Hitausmomentin määritelmän mukaan lävistäjäalkiot $I_{ii}$ ovat hitausmomentteja 
koordinaattiakselien suhteen. Kun näissä alkioissa kirjoitetaan 
$y^2+z^2=-x^2+(x^2+y^2+z^2)$ jne, niin nähdään, että hitausmomentille $I_S$ pätee myös
laskukaava
\begin{equation} \label{hitausmomentin tensorikaava}
\boxed{\kehys\quad I_S = J_0 - \me^T\mJ\,\me, \quad } \tag{$\star$}
\end{equation}
missä
\[ 
J_0 = \int_A (x^2+y^2+z^2)\,dm,\quad \mJ 
    = \int_A \begin{rmatrix} x^2&xy&xz\\xy&y^2&yz\\xz&yz&z^2 \end{rmatrix} dm. 
\]
Matriisin $\mJ$ alkioita sanotaan \pain{hitaustuloiksi}. Jos siis halutaan laskea kappaleen
hitausmomentti annetun suoran $S$ suhteen, niin ensin on laskettava jossakin koordinaatistossa
hitaustulomatriisi $\mJ$ eli integraalit 
\[ 
J_{11} = \int_A x^2\, dm, \quad J_{12} = \int_A xy\,dm, \quad 
                                J_{13}=\int_A xz\,dm, \quad \text{jne} 
\]
($6$ erilaista). Tämän jälkeen hitausmomentti määräytyy kaavasta
\eqref{hitausmomentin tensorikaava}, missä $\me$ on suoran $S$ suuntainen yksikkövektori
ja $J_0 = J_{11}+J_{22}+J_{33}$.

Jos edellä yksikkövektori $\vec e$ ilmaistaan kierretyssä koordinaatistossa 
$\{\vec i',\vec j', \vec k'\}$ koordinaativektorina $\me'$, niin $\me=\mC\me'$, missä $\mC$ on
ortogonaalinen matriisi (vrt.\ Luku \ref{lineaarikuvaukset}). Kaavassa
\eqref{hitausmomentin tensorikaava} tämä muunnos merkitsee, että $\me$:n tilalle tulee
$\me'$ ja $\mJ$:n tilalle $\mC^T\mJ\mC$ ($J_0$ ei muutu). Hitaustulomatriisi, samoin
hitausmatriisi, muuntuu siis koordinaatistoa kierrettäessä kuten (symmetrinen) tensori,
vrt.\ Luku \ref{tensorit}. Sanotaankin, että kyseessä on \kor{hitaustensori}, jolloin
\pain{hitaus} (pyörimishitaus, vrt.\ alaviite edellä) tulee ymmärretyksi koordinaatiston
kierrosta riippumattomana kappaleen ominaisuutena suhteessa valittuun koordinaatiston origoon
(pyörimiskeskus). Päätellään edelleen, että koordinaatiston kierrolla löytyy
aina \kor{päähitauskoordinaatisto}, jossa hitausmatriisi on diagonaalinen (!). Nimittäin tämä
löytyy ratkaisemalla hitausmatriisin ominaisarvo-ongelma (vrt.\ Luku \ref{diagonalisointi}).
Ominaisarvoja, eli hitausmatriisin
\index{pzyzy@päähitausmomentti}%
lävistäjäalkioita päähitauskoordinaatistossa, sanotaan \kor{päähitausmomenteiksi}.
\begin{Exa} Määritä homogeenisen kuution (sivun pituus $a$, massatiheys $\rho_0$, massa 
$m=\rho_0 a^3$) hitausmomentti lävistäjän suhteen.
\end{Exa}
\ratk Olkoon $A=[-a/2,a/2]\times[-a/2,a/2]\times[-a/2,a/2]$, jolloin symmetrian perusteella on
$I_{ij}=0$ kun $i \neq j$ (päähitauskoordinaatisto). Päähitausmomentit ovat myös symmetrian 
perusteella samansuuruiset, eli jokaisella $i=1,2,3$ on
\[
I_{ii} = \rho_0\int_A (y^2+z^2)\,dxdydz = 2\rho_0\int_A z^2\,dxdydx 
       = 2\rho_0 a^2\int_{-a/2}^{a/2} z^2\,dz = \frac{1}{6}\,ma^2.
\]
Koska hitausmatriisi on diagonaalinen ja lävistäjäalkiot ovat samansuuruiset, niin 
hitausmatriisi on sama kaikissa koordinaatistoissa, joiden origo on kuution
keskipisteessä (painopisteessä). Siis vastaus: $I_S=\underline{\underline{\frac{1}{6}ma^2}}$.
\loppu

\subsection{*Todennäköisyysmitta}
\index{mitta, mitallisuus!c@todennäköisyysmitta|vahv}

Matematiikan lajissa nimeltä \kor{todennäköisyyslaskenta} tarkastellaan nk.\ 
\index{satunnaismuuttuja}%
\kor{satunnaismuuttujia} $\mx\in\R^n$. Jokaiseen satunnaismuuttujaan liittyy ko.\ muuttujalle
ominainen \kor{todennäköisyysmitta} $P$, jonka tiheysfunktiota $f$ ($f(\mx) \ge 0\ \forall \mx$)
\index{todennäköisyysmitta, -tiheys} \index{tiheys(funktio)!b@todennäköisyystiheys}%
sanotaan \kor{todennäköisyystiheydeksi} tai \kor{-jakaumaksi}. Todennäköisyysmitalta 
edellytetään, että koko $\R^n$ on $P$-mitallinen ja
\[ 
P(\R^n) = \int_{R^n} dP = \int_{\R^n} f\,d\mu = 1. 
\]
Joukon $A \subset  \R^n$ todennäköisyysmittaa sanotaan $A$:n \kor{todennäköisyydeksi}. 
Todennäköisyysmitan avulla määriteltyä $\R^n$:n painopistettä
\[ 
\mx_0 = \int_{\R^n} \mx\,dP = \int_{R^n} \mx\,f(\mx)\,d\mu 
\]
\index{odotusarvo}%
sanotaan (tarkasteltavan satunnaismuuttujan) \kor{odotusarvoksi}.
\begin{Exa} Olkoon $A\subset\R^n$ mitallinen ja $\mu(A)>0$ ($\mu=\R^n$:n tilavuusmitta) ja 
olkoon satunnaismuuttujan $\mx$ todennäköisyystiheys \kor{tasajakauma}
\index{tasajakauma}%
\[
f(\mx)= \begin{cases} 
        1/\mu(A), \quad &\text{jos}\ \mx\in A, \\ 0, \quad &\text{muulloin}.
        \end{cases}
\]
Tällöin on
\[
P(B) = \int_B\,dP = \int_B f(\mx)\,d\mu = \frac{\mu(A \cap B)}{\mu(A)}\,, \quad B\subset\R^n.
\]
Odotusarvo $=A$:n keskiö. \loppu
\end{Exa}

\Harj
\begin{enumerate}

\item
Pullapitkossa, joka sijaitsee $x$-akselilla välillä $[0,a]$, $a=44$ cm, ovat rusinat
jauhautuneet siten, että rusinoiden viivatiheys on
\[
\rho(x) = 1.5\rho_0\left[1+\frac{x}{a}\left(1-\frac{x}{a}\right)\right], \quad 
                                              x\in[0,a], \quad \rho_0=\frac{1}{\text{cm}}\,.
\]
Montako (saman kokoista) rusinaa pullaan on pantu? 

\item 
Pisteen $P_0 \vastaa \vec r_0$ keskietäisyys joukon $A\subset R^2$ pisteistä voidaan määritellä
integraalina $d=\frac{1}{\mu(A)} \int_A \abs{\vec r - \vec r_0}\,d\mu.$ Laske pisteen $(2,0)$ 
keskietäisyys neliöstä $A=[0,1]\times [0,1].$

\item 
Kolmion muotoisen levyn pinnalla $\,A:\,\ x\ge 0\ \ja\ x+|y| \le a$ vaikuttaa pinnan normaalin
suuntainen paine $p=p_0(1-x^2/a^2),\,p_0=$ vakio. Laske paineesta aiheutuva kokonaisvoima sekä
momentti origon suhteen.

\item
Rakennuksen nurkkauksessa on tetraedrin muotoinen hiekkakasa
\[
A = \{(x,y,z)\in\R^3 \mid x,y,z \ge 0\ \ja\ x+y+z \le a\}.
\]
Kasassa massatiheys on
\[
\rho(x,y,z) = \rho_0+\frac{\rho_0}{30}\left(1-\frac{x+y+z}{a}\right)^2,
\]
missä $\rho_0$ (= vakio) on kuivan hiekan massatiheys ja loppuosa tiheydestä edustaa
hiekkaan sitoutunutta vettä. Jos kasan koko massa on $100$ kg, niin montako kiloa kasassa on
vettä?

\item
Kohdissa a)--k) määritä joukon $A$ keskiö, muissa kohdissa kappaleen $A\subset\R^3$ painopiste,
kun massatiheys $\rho$ on annettu ($a>0,\ \rho_0=$ vakio). \vspace{1mm}\newline
a) \ $\ A\subset\R^2:\,\ x,y \ge 0\ \ja\ x+y \le 1$ \newline
b) \ $\ A\subset\R^2:\,\ 0 \le x \le 1\ \ja\,-x^2 \le y \le x^3$ \newline
c) \ $\ A\subset\R^2:\,\ 0 \le x \le 1\ \ja\ x^m \le y \le \sqrt[m]{x}\ \ (m\in\N,\ m \ge 2)$
\newline
d) \ $\ A\subset\R^2:\,\ x=t-\sin t\ \ja\ 0 \le y \le 1-\cos t,\ \ t\in[0,2\pi]$ \newline
e) \ $\ A\subset\R^2:\,\ x^2+y^2 \le a^2\ \ja\ 0 \le y \le x$ \newline
f) \ $\,\ A\subset\R^3:\,\ x,y,z \ge 0\ \ja\ x+2x+3y \le 6$ \newline
g) \ $\ A\subset\R^3:\,\ 0 \le x \le 1\ \ja\ 0 \le y \le x^2\ \ja\ 0 \le z \le xy^2$ \newline
h) \ $\ A\subset\R^3:\,\ 0 \le x,y,z \le 1\ \ja\ x+y+z \le 2$ \newline
i) \ $\,\ A\subset\R^3:\,\ 0 \le x,y,z \le 4\ \ja\ (x-1)^2+(y-2)^2+(z-3)^2 \ge 1$ \newline
j) \ $\,\ A\subset\R^3:\,\ x,y,z \ge 0\ \ja\ x^2+y^2+z^2 \le a^2$ \newline
k) \ $\ A\subset\R^3:\,\ 0 \le y \le x\ \ja\ z \ge 0\ \ja\ x^2+y^2+z^2 \le a^2$ \newline
l) \ $\,\ A:\,\ x,y,z \ge 0\ \ja\ x+y+z \le a,\,\ \rho=\rho_0 z/a$ \newline
m) \ $A:\,\ 0 \le x \le a \ \ja\ a\abs{y} \le x^2\ \ja \abs{z} \le a,\,\ \rho=\rho_0 x/a$ 
\newline
n) \ $\ A:\,\ 0 \le x,y,z \le a,\,\ \rho=\rho_0(x^2+y^2+z^2)/a^2$ \newline
o) \ $\ A:\,\ 0 \le x \le a\ \ja\ y^2+z^2 \le 4x^2,\,\ \rho=\rho_0 x/a$ \newline
p) \ $\ A:\,\ 0 \le x \le a\pi/2\ \ja\ y^2+z^2 \le a^2\cos^2(x/a),\,\ \rho=\rho_0 (x/a)^2$
\newline
q) \ $\ A:\,\ x^2+y^2 \le a^2\ \ja\ 0 \le z \le 2a,\,\ \rho=\rho_0 z$ \newline
r) \ $\ A:\,\ x,y,z \ge 0\ \ja\ x^2+y^2+z^2 \le a^2,\,\ \rho=\rho_0 (x^2+y^2)/a^2$ \newline
s) \ $\ A:\,\ x \ge 0\ \ja\ y^2+z^2 \le a^6/(a^2+x^2)^2,\,\ \rho=\rho_0 x/a$ \newline
t) \ $\ A:\,\ x,y \ge 0\ \ja\ 0 \le z \le e^{-(x+y)/a},\,\ \rho=\rho_0(x^2+y^2)/a^2$

\item 
Kolmion muotoiseen patoluukkuun, jonka kärjet ovat pisteissä $A=(0,0,0)$, $B=(0,0,-L)$ ja 
$C=(0,L,0)$, vaikuttaa veden paine $\vec p\,(y,z)=\rho g z \vec i$. Luukkua ollaan juuri 
avaamassa (hallitusti), jolloin luukkuun vaikuttavat painekuorman lisäksi aukeamista hillitsevä
momentti $\vec M=M\vec k$ ja saranoiden tukivoimat $\vec F_A = F_A \vec i$ ja
$\vec F_B=F_B\vec i$. Laske painekuormituksen kokonaisvoima ja momentti saranan $A$ suhteen ja
näiden avulla edelleen (voima- ja momenttitasapainosta) tuntemattomat suureet $M,\, F_A$ ja
$F_B$.

\item
Seuraavassa on annettu tasoalue $A$. Määritä Pappuksen sääntöä hyväksi käyttäen a-kohdassa
$A$:n keskiö ja muissa kohdissa tilavuus $\mu(V)$, missä $V$ on pyörähdyskappale, joka
syntyy $A$:n pyörähtäessä $x$-akselin ympäri. \vspace{1mm}\newline
a) \ $x^2+y^2 \le R^2,\,\ y \ge 0$. \newline
b) \ Kolmio, jonka kärjet ovat $(2,1)$, $(1,2)$ ja $(4,2)$. \newline
c) \ Suunnikas, jonka kärjet ovat $(3,2)$, $(1,3)$, $(4,6)$ ja $(6,5)$. \newline
d) \ Käyrien $y=x^2$ ja $y=\sqrt{x}$ väliin jäävä alue välillä $x\in[0,1]$. \newline 
e) \ Käyrien $x=\sin y$ ja $x=-\sin y$ väliin jäävä alue välillä $y\in[0,\pi]$. \newline
f) \ Positiivisen $\,y$-akselin ja käyrän $x=e^{-y}$ väliin jäävä alue.

\item
Koordinaattitasojen ja tasojen $z=a$ ja $x+y+z=3a$ ($a>0$) rajaaman kappaleen hitausmomentti
$z$-akselin suhteen on $I_z=kma^2$, missä $m=$ kappaleen massa. Laske kerroin $k$, kun
kappaleen massatiheys on a) $\rho=\rho_0$, b) $\rho=\rho_0(x+y+z)/a\,$ ($\rho_0=$ vakio).

\item
a) Homogeenista kappaletta rajaavat taso $z=a$ ja paraboloidi $\,x^2+y^2=az$, $a>0$. 
Laske lieriökoordinaateilla
kertoimet $k_x$, $k_y$ ja $k_z$ kaavoissa $I_x=k_xma^2$, $I_y=k_yma^2$ ja $I_z=k_zma^2$, missä
$I_x,I_y,I_z$ ovat kappaleen hitausmomentit koordinaattiakselien suhteen ja $m=$ kappaleen
massa. \vspace{1mm}\newline
b) Millä $\alpha$:n arvolla ($\alpha\in\R$) homogeenisen kappaleen 
$A:\ 0 \le z/a \le (r/a)^\alpha$ (lieriökoordinaatit, $a>0$) hitausmomentit kaikkien kolmen
karteesisen koordinaatiakselin suhteen ovat samat? \vspace{1mm}\newline
c) $R$-säteisen kuulan massatiheys pallokoordinaatistossa on $\rho(r)=\rho_0(r/R)^\alpha$
($\rho_0=$ vakio) ja massa $=m$. Millä $\alpha$:n arvolla kuulan hitausmomentti kuulan
keskipisteen kautta kulkevan suoran suhteen on $I_S=\frac{1}{2}mR^2$\,?

\item 
Koivupuusta valmistettu kappale (homogeeninen, tiheys $=500$ kg/m$^3$) on muodoltaan 
suorakulmainen särmiö, jonka särmien pituudet ovat $20$, $30$ ja $60$ cm. Laske kappaleen 
hitausmomentit \ a) särmien, b) lävistäjien suhteen.

\item
$R$-säteisen homogeenisen kuulan keskipiste $=(3a,-2a,6a)$ ja massa $=m$. Määritä jokin
päähitauskoordinaatisto ja päähitausmomentit, kun pyörimiskeskus $=$ origo. \kor{Vihje}: Ks.\
Harj.teht.\,\ref{lineaarikuvaukset}:\,\ref{H-m-6: kiertoja}a.

\item 
Jos $x$ ja $y$ ovat riippumattomia satunnaislukuja, niin todennäköisyys sille, että $(x,y)$ osuu
joukkoon $A\subset\R^2$ on $P(A)=\mu(A \cap B)$ missä $\mu$ on $\R^2$:n pinta-alamitta ja
$B=[0,1]\times[0,1]$. Millä todennäköisyydellä on \newline
a) $x^2+y^2 < 1$, \,\ b) $x+y\ge 3/2$, \,\ c) $x+y=1$\,?

\item (*)
a) Koordinaattitasot ja taso $x+y+z=a>0$ rajaavat homogeenisen kappaleen, jonka massa $=m$.
Määritä kappaleen päähitauskoordinaatisto ja päähitausmomentit, kun pyörimiskeskus
$=$ origo. \vspace{1mm}\newline
b) Kappaleen massa $=m$ ja painopiste on origossa. Näytä, että jos kappaleen hitaustulomatriisi
origon suhteen $=\mJ$, niin hitaustulomatriisi pisteen $P=(x_0,y_0,z_0)$ suhteen on
$\mJ+\ma\ma^T$, missä $\ma^T=[x_0,y_0,z_0]$. \vspace{1mm}\newline
c) Ratkaise a)-kohdan ongelma, kun pyörimiskeskus on painopiste.

\item (*) \index{zzb@\nim!Monumentti}
(Monumentti) Tuntemattoman matemaatikon muistomerkki on tehty umpiraudasta
noudattaen seuraavia ohjeita koordinaatistossa, jossa positiivinen $x$-akseli osoittaa itään
ja positiivinen $y$-akseli pohjoiseen. Maan pinnan taso on $xy$-taso ja pituusyksikkö on metri.
\begin{enumerate}
\item[1.] Idästä katsoen monumentin profiili on suorien $z=0,\, y=2$ ja käyrän $z=y^2$ 
          rajaama alue. 
\item[2.] Päältä katsoen monumentin profiili on vinoneliö, jonka kärjet ovat pisteissä 
          $A=(-\frac{3}{2},0)$, $B=(1,0)$, $C=(0,2)$ ja $D=(\frac{5}{2},2)$.
\item[3.] Monumentin jokainen pystysuora, itä-länsisuuntainen poikkileikkaus on kolmio, jonka 
          kärki on maan pinnalla janalla $BC$ ja tämän kärjen vastainen sivu on vaakasuora.
\end{enumerate}
a) Paljonko monumentti painaa? ($\rho=7800$ kg/m$^3$) \newline
b) Määritä monumentin painopiste. \newline
c) Monumentti tuetaan kiinnittämällä kärki $A$ maahan upotettuun betonipainoon. Kuinka suuri
on tämän vastapainon massan vähintään oltava, jotta monumentti ei kaadu?

\end{enumerate} %Integraalien sovellukset: Tiheys ja kokonaismäärä
\section{Viivaintegraalit} \label{viivaintegraalit}
\alku
\index{viivaintegraali|vahv}

Tässä luvussa tarkastellaan integraaleja muotoa
\[
I(f,S,\mu)=\int_S f\,d\mu,
\]
missä $S\subset\R^2$ tai $S\subset\R^3$ on (tason tai avaruuden) \kor{käyrä} ja $\mu$ on käyrään
liitettävä \kor{kaarenpituusmitta}. Tällaisia integraaleja sanotaan \kor{viivaintegraaleiksi}
(tai käyräintegraaleiksi). Kaarenpituusmitta pitkin suoraa on tavallinen Jordanin pituusmitta. 
Tasokäyrän (kaaren) $S:\,y=f(x)\ \ja\ x\in[a,b]$ tapauksessa kaarenpituusmitta
on määritelty aiemmin (Luku \ref{kaarenpituus}) ja tulkittu myös integraalin avulla
(Luku \ref{pinta-ala ja kaarenpituus}). Seuraavassa yleistetään tämä käsite ja tutkitaan, miten 
viivaintegraaleja --- ja itse kaarenpituusmittoja --- käytännössä lasketaan.

Otetaan sama lähtökohta kuin Luvussa \ref{parametriset käyrät}, eli oletetaan käyrä $S$ 
parametrisoiduksi reaalimuuttujan $t$ avulla välillä $[a,b]$. Tällöin esimerkiksi avaruuskäyrä
on määritelty $\R^3$:n (tai $E^3$:n) pistejoukkona
\[
S=\{(x,y,z) \ | \ x=x(t), \ y=y(t), \ z=z(t), \ t\in [a,b]\}.
\]
Funktiot $x(t)$, $y(t)$, $z(t)$ (tasokäyrän tapauksessa $x(t)$, $y(t)$) oletetaan jatkossa 
ainakin jatkuviksi välillä $[a,b]$, jolloin käyrässä ei ole 'katkoksia'. Funktioiden
säännöllisyysoletuksia lisätään jatkossa tarpeen mukaan.\footnote[2]{Huomautettakoon, että 
mikäli funktiot $x(t)$, $y(t)$ ja $z(t)$ ovat pelkästään jatkuvia, ei käyrän $S$ 'ulkonäöstä'
voi tehdä pitkälle meneviä johtopäätöksiä. Esimerkiksi on konstruoitavissa välillä $[0,1]$ 
jatkuvat funktiot $x(t)$ ja $y(t)$ siten, että
$S=\{(x(t),y(t)) \ | \ t\in [0,1]\}=[0,1]\times [0,1]$ (!).
Ensimmäisenä tällaisen 'täyteiskäyrän' kostruoi \hist{G. Peano} vuonna 1890. 
\index{Peanon käyrä|av}} 

Jos käyrän parametrisointi $t\in[a,b]\mapsto\vec r\,(t)$ on injektio puoliavoimella välillä
\index{yksinkertainen!c@käyrä, kaari} \index{Jordan-käyrä} \index{suljettu käyrä}%
$[a,b)$, niin kyseessä on \kor{yksinkertainen käyrä} eli \kor{Jordan-käyrä}. Jos lisäksi
$\vec r\,(a)=\vec r\,(b)$, niin käyrä on \kor{suljettu}.
\begin{figure}[H]
\begin{center}
\input{kuvat/kuvaUint-30.pstex_t}
\end{center}
\end{figure}
Jatkossa esitettävässä laskutekniikassa käyrän parametrisoinnilta edellytetään vain riittävä
säännöllisyys. Käyrän ei siis tarvitse olla Jordan-käyrä vaan se voi olla itseään leikkaava tai
jopa itsensä päälle kiertyvä.
\begin{Exa}
Yksikköympyrä tai sen kaari voidaan parametrisoida muodossa
\[
S=\{\,(x,y)\in\R^2 \ | \ x=\cos t, \ y=\sin t, \ t\in [a,b]\,\}.
\]
Tämä on Jordan-käyrä (ympyräkaari), jos $b-a<2\pi$, ja suljettu Jordan-käyrä, jos $b-a=2\pi$.
Yleisemmin kyseessä on parametrinen käyrä, jonka voi tulkita esim.\ edustavan tasaista
ympyräliikettä (aikaparametrisointi). \loppu
\end{Exa}
\begin{Exa} Ruuviviiva on avaruuden (ei-suljettu) Jordan-käyrä muotoa
\[
S=\{\,(x,y,z)\in\R^3 \ | \ x=R\cos t, \ y=R\sin t, \ z=ct, \ t\in [a,b]\,\},
\]
missä $R>0$ ja $c \neq 0$. \loppu
\end{Exa}

\subsection{Kaarenpituusmitta}
\index{kaarenpituusmitta|vahv}

Yleistetään aluksi kaarenpituusmitan käsite Luvusta \ref{kaarenpituus} parametriselle 
avaruuskäyrälle $S = \{(x(t),y(t),z(t)) \mid t \in [a,b]\}$. (Tasokäyrä tulee käsitellyksi
erikoistapauksena $z(t)=0$.) Olkoon $\{t_k, \ k=0,\ldots n\}$ välin $[a,b]$ jako 
($a=t_0<t_1<\ldots<t_n=b,\ n\in\N$), olkoon 
$\mathcal{P}=\{P_k=(x(t_k),y(t_k),z(t_k)),\ k=0,\ldots,n\}$, ja olkoon edelleen 
$S_{\mathcal{P}}$ janoista $P_{k-1}P_k$, $k=1\ldots n\,$ koostuva murtoviiva. Tällöin jos
kaarenpituusmitalle oletetaan perusaksioomat
\begin{itemize}
\item[-] aksiooma 1: $\quad$ janan $AB$ $\text{mitta}=\abs{\overrightarrow{AB}}$,
\item[-] aksiooma 2: $\quad$ mitan additiivisuus murtoviivalla,
\end{itemize}
niin $S_{\mathcal{P}}$:n pituusmitta on
\[
\mu(S_{\mathcal{P}})=\sum_{k=1}^n\abs{\vec r\,(t_k)-\vec r\,(t_{k-1})}.
\]
Kuten aiemmin Luvussa \ref{kaarenpituus}, käyrän $S$ mitta määritellään tällöin
\[
\mu(S)=\sup_{\mathcal{P}} \mu(S_{\mathcal{P}}).
\]
Mitta on määritelty täsmälleen kun reaalilukujoukko $\{\mu(S_{\mathcal{P}})\}$ on rajoitettu,
\index{suoristuva (käyrä)}%
eli $S$ on mitallinen eli \kor{suoristuva} täsmälleen tällä ehdolla.

Ym.\ määritelmästä nähdään, että kaarenpituusmitta muistuttaa Jordan-mittoja sikäli, että 
perusaksioomissa tarvitaan ainoastaan yksinkertaisen perusjoukon (tässä janan) mitta ja 
additiivisuusperiaate. Jordan-mittojen konstruktiota muistuttaa myös kaarenpituuden 
määrittelyssä käytetty approksimaatioperiaate, jonka mukaan käyrää approksimoidaan 
perusjoukkojen äärellisillä yhdistelmillä. 

\subsection{Kaarenpituusparametrisaatio}
\index{kaarenpituusparametrisaatio|vahv}

Olkoon $S$ suoristuva Jordan-käyrä. Tällöin jos $S(t)$, $t\in [a,b]$ on käyrän $S$ osa, joka
vastaa parametrin arvoja välillä $[a,t]$, niin kaarenpituusmitan määritelmästä ja kuvauksen
$t\mapsto\vec r(t)$ oletetusta injektiivisyydestä seuraa
\[
t_1,t_2\in [a,b] \ \ja \ t_1<t_2 \ 
               \impl \ \mu(S(t_2))-\mu(S(t_1)) \ge \abs{\vec r\,(t_2)-\vec r\,(t_1)} > 0.
\]
\begin{multicols}{2} \raggedcolumns
Näin ollen jos merkitään
\[
s(t)=\mu(S(t)),
\]
niin $s(t)$ on välillä $[a,b]$ aidosti kasvava. 
\begin{figure}[H]
\begin{center}
\import{kuvat/}{kuvaUint-31.pstex_t}
\end{center}
\end{figure}
\end{multicols}
Siis jokaista $s\in [0,\mu(S)]$ vastaa yksikäsitteinen $t\in [a,b]$ siten, että
$\mu(S(t))=s$. Tämän kääntäen yksikäsitteisen vastaavuuden perusteella käyrä $S$ voidaan yhtä
hyvin parametrisoida kaarenpituuden $s$ avulla, eli voidaan kirjoittaa
\[
S=\{P\in E^d \ | \ P\vastaa \vec r=\vec r\,(s), \ 0\leq s\leq\mu(S)\}.
\]
Tätä sanotaan \kor{kaarenpituusparametrisaatioksi}. Sen laskeminen käytännössä on ongelma
sinänsä, eikä siitä olekaan hyötyä viivaintegraaleja käytännössä laskettaessa. Sen sijaan
käyräteorettisten tarkastelujen 'ajatteluparametrisaationa' se on luonnollinen, syystä että
määritelmän perusteella pätee
\[
\abs{\dvr(s)}=1.
\]
Tämän ominaisuuden vuoksi kaarenpituusparametriaatio on 'helpoin' parametrisaatio silloin kun
halutaan tutkia käyrän (parametrisaatiosta riippumattomia) geometrisia ominaisuuksia, vrt.\
Lukujen \ref{derivaatta geometriassa} ja \ref{käyrän kaarevuus} tarkastelut.
Kaarenpituusparametrisaatiota ajatellaan myös yleisessä kaarenpituusmitan merkinnässä
\[
d\mu=ds \quad \text{(kaarenpituusmitta)}.
\]
Tämä merkintä omaksutaan jatkossa.

\pagebreak
\subsection{Viivaintegraalien laskutekniikka}

Viivaintegraaleja käytännössä laskettaessa lähdetään käyrän tunnetuksi oletetusta 
parametrisaatiosta $t\mapsto\vec r\,(t)$ (joka harvoin on kaarenpituusparametrisaatio) ja 
pyritään tämän avulla 'oikomaan' integraali tavalliseksi määrätyksi integraaliksi yli 
välin $[a,b]$. Tämä tarkoittaa muunnosta
\[
\int_S f\,ds=\int_{[a,b]} g\,d\mu',
\]
missä $g(t)=f(x(t),y(t),z(t))$ ja $\mu'$ on kaarenpituusmitan vastine välillä $[a,b]$. Mitan
$\mu'$ määrittämiseksi menetellään samalla tavoin kuin muuttujan vaihdossa 
(vrt.\ Luku \ref{muuttujan vaihto integraaleissa}). Jos $\Delta S(t)$ on parametriväliä
$[t,t+\Delta t]\subset [a,b]$ vastaava käyrän $S$ osa, niin sikäli kun on määritettävissä
\kor{muuntosuhde} \index{muuntosuhde integraalissa!d@viivaintegraalissa}
\[
J(t)=\lim_{\Delta t\kohti 0^+} \frac{\mu(\Delta S(t))}{\Delta t}\,,
\]
ja lisäksi $J(t)$ on riittävän säännöllinen (esim.\ jatkuva tai paloittain jatkuva), niin
\[
\int_S f\,ds=\int_a^b g(t)J(t)\,dt.
\]
Viivaintegraali on näin palautettu tavalliseksi määrätyksi integraaliksi Jordan-mitan suhteen.

Muuntosuhteen $J(t)$ määräämiseksi oletetaan, että $x(t)$, $y(t)$ ja $z(t)$ ovat jatkuvasti 
derivoituvia välillä $[a,b]$. (Oletus on hiukan lievennettävissä, ks.\ kommentit jäljempänä). 
Olkoon $\Delta S$ käyrän osa välillä $[t,t+\Delta t] \subset [a,b]$ ($\Delta t>0$) ja olkoon
$\{t_j,\ j=0 \ldots n\}$ välin $[t,t+\Delta t]$ jako, ts.\ 
$t=t_0 < t_1 < \ldots < t_n = t+\Delta t$. Tällöin Differentiaalilaskun väliarvolauseen mukaan
\[
x(t_j)-x(t_{j-1})=x'(\xi_j)(t_j-t_{j-1}),\quad \xi_j\in (t_{j-1},t_j).
\]
Koska $x'$ on jatkuva välillä $[a,b]$ ja $\,\xi_j\in[t,t+\Delta t]$, niin tässä
\[
x'(\xi_j) = x'(t) + \ord{1}, \quad \text{kun}\ \Delta t \kohti 0. 
\]
[Tässä nojataan jatkuvuuden syvällisempään logiikkaan (tasaiseen jatkuvuuteen), vrt.\ vastaava
päättely Luvussa \ref{muuttujan vaihto integraaleissa}.] Arvioimalla samalla tavoin
$y(t_j)-y(t_{j-1})$ ja $z(t_j)-z(t_{j-1})$ päätellään, että
\vspace{1mm}
\begin{align*}
&\sqrt{[x(t_j)-x(t_{j-1})]^2+[y(t_j)-y(t_{j-1})]^2+[z(t_j)-z(t_{j-1})]^2} \\[1mm]
&\quad = \sqrt{[x'(t)]^2+[y'(t)]^2 + [z'(t)]^2}\,(t_j-t_{j-1}) 
                                   + \ord{1}(t_j-t_{j-1}), \quad \text{kun}\ \Delta t \kohti 0.
\end{align*}
\vspace{1mm}
Summaamalla yli $j$:n ja käyttämällä kaarenpituuden määritelmää seuraa
\[
\mu(\Delta S)=\abs{\dvr(t)}\Delta t+\ord{\Delta t}, \quad \text{kun}\ \Delta t \kohti 0.
\]
Muuntosuhteen määritelmän nojalla on $J(t)=|\dvr(t)|$. Siis kaarenpituusmitan muunnoskaava 
avaruuskäyrälle on
\[
ds = \abs{\dvr(t)}\,dt = \sqrt{[x'(t)]^2+[y'(t)]^2+[z'(t)]^2}\,dt,
\]
ja viivaintegraalille on saatu laskukaava
\begin{equation} \label{viivaintegraalin kaava}
\boxed{\quad \int_S f\,ds=\int_a^b g(t)\abs{\dvr(t)}\,dt, \quad 
                     g(t)=f(\vec r\,(t)). \quad} \tag{$\star$}
\end{equation}
Kaava on pätevä hieman heikomminkin kuin oletetuin edellytyksin. Esimerkiksi seuraavat oletukset
ovat yhdessä riittävät:
\begin{itemize}
\item[(i)]   $x(t)$, $y(t)$, $z(t)$ ovat jatkuvia välillä $[a,b]$.
\item[(ii)]  $x'(t)$, $y'(t)$ $z'(t)$ ovat olemassa ja jatkuvia avoimilla väleillä 
             $(c_{i-1},c_i)$, $i=1\ldots m$, missä $a=c_0<c_1<\ldots<c_m=b$ $(m\in\N)$.
\item[(iii)] $g(t)\abs{\dvr(t)}$ on (laajennetussa mielessä) Riemann-integroituva välillä
             $[a,b]$.
\end{itemize}
Sovelluksissa ei näitä lievennyksiä pitemmälle ole juuri tarvetta mennä.

Jos kaavassa \eqref{viivaintegraalin kaava} asetetaan $f=1$ $(\impl g=1)$, on tuloksena
kaarenpituuden laskukaava
\[
\boxed{\kehys\quad \mu(S)=\int_a^b \abs{\dvr(t)}\,dt. \quad}
\]
Huomattakoon, että jos käytetään kaarenpituusparametrisaatiota $(t=s)$, niin $a=0$, $b=\mu(S)$
ja $\abs{\dvr(s)}=1$, jolloin kaava saa muodon
\[
\mu(S)=\int_0^{\mu(S)}ds=\sijoitus{0}{\mu(S)} s=\mu(S).
\]
Tämä on luonnollisesti vain tautologia, josta ei ole käytännön hyötyä kaarenpituuden
laskemisessa.
\begin{multicols}{2}
\begin{Exa}
Ohuesta, tasapaksusta ja homogeenisesta langasta valmistettu kierrejousi on avaruuskäyrä 
$x=R\cos t$, $y=R\sin t$, $z=Ht/(12\pi)$,  $0 \le t \le 12\pi$. Laske a) jousen pituus, 
b) hitausmomentti suoran $x=R$, $y=0$ suhteen, kun jousen massa $=m$.
\begin{figure}[H]
\begin{center}
\import{kuvat/}{kuvaUint-33.pstex_t}
\end{center}
\end{figure}
\end{Exa}
\end{multicols}
\ratk a)
\begin{align*}
\mu(S) &= \int_0^{12\pi} \sqrt{[x'(t)]^2+[y'(t)]^2+[z'(t)]^2}\,dt \\
       &= \int_0^{12\pi} \sqrt{R^2+\left(\frac{H}{12\pi}\right)^2}\,dt \\[2mm]
       &= \underline{\underline{\sqrt{(12\pi R)^2+H^2}}}\,.
\end{align*}

b) Olkoon langan massatiheys pituusyksikköä kohti $=\rho_0$, jolloin a-kohdan mukaan 
$m = \rho_0\sqrt{(12\pi R)^2+H^2}$. Hitausmomentti suoran $S:\ x=R,\ y=0$ suhteen on
\begin{align*}
I_S &= \int_S \left[(x-R)^2+y^2\right]\,\rho_0\,ds \\
    &= \int_0^{12\pi} \rho_0 R^2 \left[(1-\cos t)^2 + \sin^2 t\right]
                       \sqrt{R^2+\left(\frac{H}{12\pi}\right)^2}\,dt \\
    &= \frac{mR^2}{12\pi}\int_0^{12\pi}(2-2\cos t)\,dt \\[2mm]
    &= \underline{\underline{2mR^2}}.
\end{align*}

Tähän tulokseen olisi voitu tulla ilman integrointiakin: Koska langan koko massa on
vakioetäisyydellä $R\,$ $z$-akselista, niin hitausmomentti $z$-akselin suhteen on
$I_z=mR^2$, jolloin Steinerin säännöstä (ks.\ edellinen luku) seuraa $I_S=mR^2+mR^2=2mR^2$.
\loppu 

\subsection{Tasokäyrät}

Tasokäyrän tapauksessa on $\,\abs{\dvr(t)}=\sqrt{[x'(t)]^2+[y'(t)]^2}$.
Jos käyrä on annettu muodossa $y=f(x)$ ja valitaan $t=x$, niin saadaan 
(ennestään tuttu, vrt.\ Luku \ref{pinta-ala ja kaarenpituus}) tulos
$\,\abs{\dvr(x)}=\sqrt{1+[f'(x)]^2}$. Jos taas käyrä on annettu polaarimuodossa
$\,r=f(\varphi)$, niin on luonnollista valita parametriksi $t=\varphi$, jolloin
\begin{align*}
x'(\varphi) &= \frac{d}{d\varphi}[f(\varphi)\cos\varphi]
             = f'(\varphi)\cos\varphi-f(\varphi)\sin\varphi, \\
y'(\varphi) &= \frac{d}{d\varphi}[f(\varphi)\sin\varphi]
             = f'(\varphi)\sin\varphi+f(\varphi)\cos\varphi,
\end{align*}
ja näin ollen
\[
\abs{\dvr(\varphi)}=\sqrt{[x'(\varphi)]^2+[y'(\varphi)]^2}
                   =\sqrt{[f(\varphi)]^2+[f'(\varphi)]^2}.
\]
Laskettaessa viivaintegraaleja tasokäyrien yli voidaan siis käyrän esitysmuodosta riippuen
käyttää mittamuunnoksia
\index{muuntosuhde integraalissa!d@viivaintegraalissa}
\index{kzyyrzy@käyräviivaiset koordinaatistot!d@--integraalit}%
\[ \boxed{
\quad ds=\begin{cases}
\sqrt{[x'(t)]^2+[y'(t)]^2}\,dt &(\,x=x(t), \ y=y(t)\,) \quad\ykehys \\
\sqrt{1+[f'(x)]^2}\,dx &(\,y=f(x)\,) \\
\sqrt{[f(\varphi)]^2+[f'(\varphi)]^2}\,d\varphi &(\,r=f(\varphi)\,) \akehys
\end{cases} } \]
\begin{Exa}
Laske puoliympyrän kaaren $S:\ x^2+y^2=R^2,\ y \ge 0\ $ pituus $\mu(S)$ kolmella eri tavalla.
\end{Exa}
\ratk
(a) \ \, $x=R\cos t$, $\,y=R\sin t$, $\,t\in [0,\pi]$
\[
\impl \mu(S)=\int_0^\pi \sqrt{[x'(t)]^2+[y'(t)]^2}\,dt
            =\int_0^\pi R\,dt=\underline{\underline{\pi R}}.
\]
(b) \ \, $y=\sqrt{R^2-x^2}=f(x)$, $\,x\in [-R,R]$
\begin{align*}
\impl \ \mu(S) &= \int_{-R}^R \sqrt{1+[f'(x)]^2}\,dx=\int_{-R}^R \frac{R}{\sqrt{R^2-x^2}}\,dx \\
               &= \int_{-\pi/2}^{\pi/2} R\,dt
                = \underline{\underline{\pi R}}. \qquad [\,\text{sijoitus}\ x=R\sin t\,]
\end{align*}
(c) \ \, $r=R=f(\varphi)$, $\,\varphi\in [0,\pi]$
\[
\impl \ \mu(S)=\int_0^\pi \sqrt{[f(\varphi)]^2+[f'(\varphi)]^2}\,d\varphi
              =\int_0^\pi R\,d\varphi=\underline{\underline{\pi R}}. \loppu
\]

\begin{multicols}{2}
\begin{Exa}
Logaritmisen spiraalin $r=e^{-\varphi}$ kaarenpituus välillä $\varphi\in [0,\infty)$ on
\begin{align*}
\mu(S) &= \int_0^\infty \sqrt{[f(\varphi)]^2+[f'(\varphi)]^2}\,d\varphi \\
       &= \int_0^\infty \sqrt{2}\,e^{-\varphi}\,d\varphi\
        =\ \underline{\underline{\sqrt{2}}}. \loppu
\end{align*}
\begin{figure}[H]
\begin{center}
\epsfig{file=kuvat/kuvaUint-32.eps}
\end{center}
\end{figure}
%\end{Exa}
%\end{multicols}
\end{Exa}
\end{multicols}
\begin{Exa}
Laske viivaintegraali $\int_S f\,ds$, kun \ a) $f(x,y)=y$, \ b) $f(x,y)=x$ ja
$S=\{\,(x,y)\in\R^2 \mid y=x^3/3 \ \ja \ x\in [0,1]\,\}$. 
\end{Exa}
\ratk
\begin{align*}
a) \qquad \int_S y\,ds &= \int_0^1 \frac{1}{3}x^3\sqrt{1+x^4}\,dx \\
                       &=\sijoitus{0}{1} \frac{1}{18}(1+x^4)^{3/2}
                        =\underline{\underline{\frac{1}{18}(2\sqrt{2}-1)}}. \\[2mm]
b) \qquad \int_S x\,ds &= \int_0^1 x\sqrt{1+x^4}\,dx \qquad 
                              [\,\text{sijoitus}\ x^2=t, \ x\,dx=\tfrac{1}{2}\,dt\,] \\
                       &=\frac{1}{2}\int_0^1 \sqrt{1+t^2}\,dt \qquad 
                              [\,\text{sijoitus}\ t=\sinh u, \ dt=\cosh u\,du\,] \\
                       &=\frac{1}{2}\int_{0}^{\ln (\sqrt{2}+1)} \cosh^2 u\,du \\
                       &=\frac{1}{8}\int_{0}^{\ln (\sqrt{2}+1)} (e^{2u}+e^{-2u}+2)\,du \\
                       &=\frac{1}{16}\sijoitus{0}{\ln (\sqrt{2}+1)}(e^{2u}-e^{-2u}+4u) 
                        =\underline{\underline{
                         \frac{1}{4}\bigl[\sqrt{2}+\ln(\sqrt{2}+1)\bigr]}}. \loppu
\end{align*}

\Harj
\begin{enumerate}

\item
Rautalanka, jonka pituus $=10\pi$, on taivuteltu noudattamaan pisteestä $(3,0,0)$ lähtien \
a) ruuviviivaa $\,S:\, x=3\cos t,\ y=3\sin t,\ z=4t,\ t \ge 0$, \ b) käyrää
$\,S:\, x=3\cos^2t,\ y=4\sin^2t,\ z=5\sin t\cos t,\ t \ge 0$. Laske langan toinen päätepiste.

\item
Laske viivaintegraali $\int_S f\,ds$ annetuilla $f$ ja $S$\,: \vspace{1mm}\newline
a) \ $f(x,y)=xy, \quad S:\ x=2\cos t,\ y=\sin t,\ t\in[0,\frac{\pi}{2}]$ \newline
b) \ $f(r,\varphi)=r^2\varphi, \quad S:\ r=\varphi,\ \varphi\in[0,4\pi]$ \ (polaarik.) \newline
c) \ $f(x,y,z)=x^2z, \quad S:\ r=1,\ z=\varphi,\ \varphi\in[0,4\pi]$ \ (lieriök.) \newline
d) \ $f(x,y,z)=x^2z^3, \quad S:\ r=1,\ \theta=2\pi/3,\ \varphi\in[0,3\pi]$ \ (pallok.)

\item
Käyrän kaaren $S:\,t\in[a,b]\map\vec r\,(t)$ (matemaattinen) keskiö lasketaan kaavalla
$\vec r_0 = \frac{1}{\mu(S)} \int_S \vec r\,ds$. Sovella kaavaa: \vspace{1mm}\newline
a) \ $\vec r=R\cos t\vec i+R\sin t\vec j,\ t\in[0,\pi]\ $ (puoliympyrä) \newline
b) \ $\vec r=R(t-\sin t)\vec i+R(1-\cos t)\vec j,\ t\in[0,2\pi]\ $ (sykloidin kaari) \newline
c) \ $\vec r=R\cos^3 t\,\vec i+R\sin^3 t\,\vec j,\ t\in[0,\pi/2]\ $ (asteroidin kaari)

\item
Suoran langan pätepisteet ovat $(0,0)$ ja $(2a,0)$ ja langan massatiheys pituusyksikköä kohti
on $\rho(x)=\rho_0(1+x/a)$ ($\rho_0=$ vakio, $a>0$). Lanka taivutetaan noudattamaan origosta
lähtien ympyräviivaa $x^2+y^2=2ay,\ x \ge 0$. Määritä taivutetun langan painopiste.

\item
Lenkkeilijä lähtee pisteestä $(1,0)$ (pituusyksikkö = km) kiertämään pururataa $S: x^2+y^2=1$
vastapäivään. Missä pisteessä lenkkeilijä on tunnin juostuaan, jos hänen juoksuvauhtinsa on \ 
$v(s)=v_0e^{-0.05s}$, missä $v_0=12$ km/h ja $s=$ juostu matka kilometreina?

\item (*)
Esitä seuraavat käyrät kaarenpituusparametrisaation avulla. \vspace{1mm}\newline
a) \ $x=R(t-\sin t),\ y=R(1-\cos t),\ t\in[0,2\pi]$ \newline
b) \ $y=x^2,\ x \ge 0$

\item (*) \index{zzb@\nim!Laskiainen, 3.\ lasku}
(Laskiainen, 3.\ lasku) Näytä, että harjoitustehtävässä
\ref{toisen kertaluvun dy}:\ref{H-dy-3: nopein liuku} lumilautailijan lyhin laskuaika
pisteeseen $P=(a,0)$ ($a>0$) on $t_{min}=\sqrt{2\pi a/g}$.

\item (*)
Lieriöt $L_1: x^2+y^2=R^2$, $L_2: (y-R)^2+z^2=R^2$ ja pallo $K: (x-R)^2+y^2+z^2=R^2$ leikkaavat
toisensa pitkin kolmea avaruuskäyrää. Näytä, että näiden leikkauskäyrien pituudet 
ovat \vspace{1mm}\newline
a) \,\ $\mu(L_1 \cap L_2) = \int_0^{\pi/2} \sqrt{f(\sin t)}\,dt, \quad
                                   f(x)=[1-x^2(1-x)^2]/(2x-x^2)$, \newline
b) \,\ $\mu(L_1 \cap K) = \mu(L_2 \cap K) = \int_{\pi/6}^{\pi/2} \sqrt{g(\sin t)}\,dt, \quad
                                   g(x)=(2x-x^2)/(2x-1)$.

\item (*) \index{zzb@\nim!Kzz@Köysi katolla}
(Köysi katolla) $xy$-tasolle on pystytetty pitkä rakennus, jonka vesikatto on pinnalla
$z = 4-x^2/2$ ja räystäät suorilla $x=\pm 2,\ z=2$.
Rakennuksen yli on heitetty (hyvin taipuisa) köysi siten, että köyden kumpikin pää ulottuu 
$xy$-tasolle. \ a) Mikä on tällaisen köyden minimipituus? \ b) Mikä on köyden
minimipituus sillä lisäehdolla, että köysi tulee räystäiden yli pisteissä $(2, 2, 2)$ ja
$(-2, -2, 2)\,$? \
c) Millaisena tasokäyränä köysi näkyy b-kohdassa, kun sitä katsellaan kaukaa positiivisen
$z$-akselin suunnasta? 

\end{enumerate}
 %Viivaintegaalit
\section{Pintaintegraalit} \label{pintaintegraalit}
\alku
\index{pintaintegraali|vahv}


\kor{Pintaintegraalilla} tarkoitetaan integraalia muotoa
\[
I(f,A,\mu)=\int_A f\,d\mu,
\]

missä $A$ on pinnan $S \subset \R^ 3$ osa ja $\mu$ 
\index{pinta-alamitta!b@kaarevan pinnan}
\kor{pinta-alamitta pinnalla} $S$.
Jatkossa käytetään kaarevan pinnan pinta-alamitan yhteydessä merkintää $d\mu=dS$. Tämän mitan
suhde tason (Jordanin) pinta-alamittaan on samantyyppinen kuin kaarenpituusmitan suhde $\R$:n
pituusmittaan. Pintaintegraali on näin ajatellen tasointegraalin yleistys, ja etenkin
sovelluksissa termiä saatetaan käyttää tasointegraaleistakin puhuttaessa
(vrt.\ Luku \ref{pinta- ja tilavuusintegraalit}).

Kuten käyrän kaarenpituusmitta, voidaan myös kaarevan pinnan pinta-alamitta määritellä pintaa
'oikaisevien' approksimaatioiden kautta. Luonteva menettely on approksimoida pintaa 
\pain{kolmion} \pain{muotoisilla} \pain{taso}p\pain{innoilla}. Kun kolmiot sijoitetaan verkoksi,
jonka solmupisteet ovat pinnalla, niin verkkoa tihennettäessä tulee kolmioiden yhteenlasketun
pinta-alan lähestyä $A$:n pinta-alamittaa $\mu(A)$. Jos tämä sovitaan mitallisuuden 
määritelmäksi, niin pinnan kolmioapproksimaatioita voi käyttää pinta-alan numeeriseen 
laskemiseen 'suoraan määritelmästä'. Tällaisilla approksimaatioilla on muutakin käyttöä, 
esimerkiksi esitettäessä pintoja graafisesti (tietokonegrafiikka) tai jopa konstruoitaessa 
todellisia pintoja (esim.\ kaarevat peilipinnat).
\begin{figure}[H]
\begin{center}
\import{kuvat/}{kuvaUint-37.pstex_t}
\end{center}
\end{figure}
Suoran 'numeronmurskauksen' vaihtoehtona pinta-alaa ja pintaintegraaleja on jälleen mahdollista
käsitellä myös differentiaalilaskennan keinoin, jolloin pintaintegraalille saadaan laskukaava
tasointegraalina. Fubinin lauseen avulla tämä palautuu edelleen yksiulotteisiksi integraaleiksi,
jotka suotuisissa (käytännössä kylläkin harvinaisissa) oloissa voidaan laskea suljetussa
muodossa. Pintaintegraalin laskukaavaa johdettaessa on luonnollinen lähtökohta pinnan 
\index{parametrinen pinta}%
parametrisointi eli esittäminen \kor{parametrisena pintana}
(vrt.\ Luku \ref{parametriset käyrät}). Jos parametrisointi esitetään vektorimuodossa 
\[
\vec r = \vec r\,(u,v) = x(u,v)\vec i + y(u,v)\vec j + z(u,v)\vec k
\]
ja oletetaan, että $(x(u,v),y(u,v),z(u,v)) \in A\ \ekv\ (u,v) \in B \subset \R^2$, niin 
pintaintegraali on muunnettavissa muotoon
\[
\int_A f\,dS=\int_B g\,d\mu', \quad g(u,v)=f(x(u,v),y(u,v),z(u,v)),
\]
missä $\mu'$ on pinnan $S$ pinta-alamitan vastine parametritasossa. Muunnos parametritasoon on
jälleen muotoa
\[
d\mu'=J\,dudv,
\]
missä $J=J(u,v)$ on (yleensä laskettavissa oleva) muuntosuhde. Kun muuntosuhde tunnetaan, on
laskukaava valmis:
\[
\int_A f\,dS=\int_B gJ\,dudv.
\]

Muuntosuhteen laskemiseksi tutkitaan suorakulmion 
\[
\Delta B=[u_0,u_0+\Delta u]\times [v_0,v_0+\Delta v] \subset B
\]
muuntumista. Oletetaan funktiot $x(u,v), y(u,v), z(u,v)$ riittävän säännöllisiksi niin, että 
voidaan käyttää linearisoivaa approksimaatiota
\[
\vec r\,(u,v) \approx \vec r\,(u_0,v_0)+(u-u_0)\frac{\partial\vec r}{\partial u}(u_0,v_0)
                                       +(v-v_0)\frac{\partial\vec r}{\partial v}(u_0,v_0).
\]
Tämän mukaisesti $\Delta B$ kuvautuu pintaa pisteessä $P_0\vastaa\vec r\,(u_0,v_0)$ sivuavalle
tangenttitasolle. Kuva on tämän tason suunnikas, jonka virittävät vektorit
\[
\vec v_1=\Delta u\,\partial_u\vec r,\quad \vec v_2=\Delta v\,\partial_v\vec r.
\]
Suunnikaan pinta-ala on 
$|\vec v_1\times\vec v_2|=|\partial_u\vec r\times\partial_v\vec r\,||\Delta u||\Delta v|$,
joten päätellään, että muuntosuhde on\footnote[2]{Muuntosuhteen laskukaava on tarkemmin
perusteltavissa olettaen, että funktioiden $x(u,v), y(u,v), z(u,v)$ osittaisderivaatat ovat
jatkuvia perussuorakulmiossa $T \supset B$,}
\[
J=\left|\frac{\partial\vec r}{\partial u}\times\frac{\partial\vec r}{\partial v}\right|.
\]
Jos pinta on annettu muodossa $z=f(x,y)$, eli
$\,\vec r=\vec r\,(x,y)=x\vec i+y\vec j+f(x,y)\vec k$, niin
\begin{align*}
\partial_x\vec r\times\partial_y\vec r 
               &= \begin{vmatrix} 
                  \vec i & \vec j & \vec k \\ 1 & 0 & f_x \\ 0 & 1 & f_y 
                  \end{vmatrix} 
                = -f_x\vec i-f_y\vec j+\vec k \\
\impl \ J(x,y) &= \sqrt{1+[f_x(x,y)]^2+[f_y(x,y)]^2}.
\end{align*}
Koska tässä $\vec n=(-f_x\vec i-f_y\vec j+\vec k\,)/J=n_x\vec i+n_y\vec j+n_z\vec k$ on pinnan
yksikkönormaalivektori, niin tuloksen voi esittää myös muodossa $J(x,y)=1/\abs{n_z(x,y)}$.

$R$-säteisellä pallopinnalla on
\begin{align*}
\vec r = \vec r\,(\theta,\varphi)
      &= R\sin\theta\cos\varphi\,\vec i+R\sin\theta\sin\varphi\,\vec j+R\cos\theta\,\vec k \\
\impl\ J(\theta,\varphi) 
      &= \abs{\partial_\theta\vec r\times\partial_\varphi\vec r\,}
       = R^2\sin\theta.
\end{align*}
Yleisemmällä pyörähdyspinnalla, joka syntyy käyrän $K:\,y=f(x),\ x\in[a,b]$ pyörähtäessä
$x$-akselin tai käyrän $K:\,y=f(z),\ z\in[a,b]$ $z$-akselin ympäri (ks.\ kuviot alla) voidaan
muuntosuhde laskea vastaavaan tapaan
(Harj.teht.\,\ref{H-Uint-9: pyörähdyspintojen muuntosuhteet}). Tulokset koottuna:
\index{muuntosuhde integraalissa!e@pintaintegraalissa}
\index{kzyyrzy@käyräviivaiset koordinaatistot!d@--integraalit}%
\vspace{2mm}
\begin{center}
\begin{tabular}{lll}
Pinta & Muuttujat & $\quad$Muuntosuhde \\ \hline \\
$\vec r=\vec r\,(u,v)$ & $u,v$ & $\quad \abs{\partial_u\vec r \times \partial_v\vec r\,}$ \\ \\
$z=f(x,y)$ & $x,y$ & $\quad \sqrt{1+[f_x(x,y)]^2+[f_y(x,y)]^2}$ \\ \\
pallopinta $\,r=R$& $\theta,\varphi$ & $\quad R^2\sin\theta$ \\ \\
\parbox{3.5cm}{pyörähdyspinta: 
$y=f(x) \hookrightarrow\ \curvearrowleft \negthickspace\negthickspace \longrightarrow_x$} & 
$x,\varphi$ & $\quad |f(x)|\sqrt{1+[f'(x)]^2}$ \\ \\
\parbox{3.5cm}{pyörähdyspinta: 
$z=f(r) \hookrightarrow$\ \raisebox{-0.2cm}{$\circlearrowleft$}$\negthinspace \negthinspace 
\negmedspace \negthinspace \uparrow^z$} & $r,\varphi$ & $\quad r\sqrt{1+[f'(r)]^2}$

\end{tabular}
\end{center}
\begin{multicols}{2}
\begin{figure}[H]
\vspace{1cm}
\begin{center}
\import{kuvat/}{kuvaDD-2.pstex_t}
\end{center}
\end{figure}
\begin{figure}[H]
\begin{center}
\import{kuvat/}{kuvaDD-4.pstex_t}
\end{center}
\end{figure}
\end{multicols}
\begin{Exa}
$R$-säteiselle pallopinnalle on jakautunut tasaisesti massa $m$. Laske hitausmomentti 
$z$-akselin suhteen.
\end{Exa}
\ratk
\begin{align*}
I &= \int_A \rho_0(x^2+y^2)\,dS \\
&= \int_B \rho_0 R^2\sin^2\theta\cdot R^2\sin\theta\,d\theta d\varphi 
   \qquad \left(\,B=[0,\pi]\times[0,2\pi]\,\right) \\
&= \rho_0 R^4 \int_0^{2\pi} d\varphi \cdot \int_0^\pi \sin^3 \theta\,d\theta \\
&= 2\pi\rho_0R^4\sijoitus{\theta=0}{\theta=\pi} (-\cos\theta+\frac{1}{3}\cos^3\theta) \\
&= \frac{8\pi}{3}\rho_0 R^4.
\end{align*}
Koska
\begin{align*}
m =\int_A \rho_0\,dS
 &= \rho_0 R^2 \int_0^{2\pi} d\varphi \cdot \int_0^{\pi} \sin\theta\,d\theta \\
 &= \rho_0\cdot 4\pi R^2,
\end{align*}
niin $\displaystyle{I=\underline{\underline{\frac{2}{3}mR^2}}}$. \loppu
\begin{Exa}
Laske $\int_A f\,dS$, kun $f(x,y,z)=xy\ $ ja
\[
A = \left\{(x,y,z)\in\R^3 \mid z=x^2+\frac{1}{2}y^2 \; \ja \; (x,y)\in B\right\},
    \quad B=[0,1]\times [0,1].
\]
\end{Exa}
\ratk
\begin{align*}
\int_A f\,dS &= \int_B xy\sqrt{1+4x^2+y^2}\,dxdy \\
&=\int_0^1 x\left(\int_0^1 y\sqrt{1+4x^2+y^2}\,dy\right)dx \\
&=\int_0^1 x\left[\sijoitus{y=0}{y=1}\frac{1}{3}(1+4x^2+y^2)^{3/2}\right]dx \\
&=\int_0^1 \left[\frac{1}{3}x(2+4x^2)^{3/2}-\frac{1}{3}x(1+4x^2)^{3/2}\right]\,dx
\end{align*}
\begin{align*}
\quad &=\ \sijoitus{0}{1}\left[\frac{1}{60}(2+4x^2)^{5/2}-\frac{1}{60}(1+4x^2)^{5/2}\right] \\
      &= \underline{\underline{\frac{1}{60}\bigl(36\sqrt{6}-25\sqrt{5}-4\sqrt{2}+1\bigr)}}.
\loppu \\
\end{align*}

\begin{multicols}{2} \raggedcolumns
\begin{Exa}
Katkaistun suoran ympyräkartion korkeus $=h$, pohjan säde $=R$ ja katkaisukohdassa säde $=r$.
Laske vaipan ala $\mu(A)$.
\end{Exa}
\begin{figure}[H]
\begin{center}
\import{kuvat/}{kuvaUint-38.pstex_t}
\end{center}
\end{figure}
\end{multicols}
\ratk Tässä on kyseessä pyörähdyspinta, joka syntyy, kun jana
\[
K:\,y=f(x)=r+(R-r)\frac{x}{h}, \quad x\in [0,h]
\]
pyörähtää $x$-akselin ympäri. Parametreilla $(x,\varphi) \in B=[0,1]\times[0,2\pi]$ laskien
saadaan
\begin{align*}
\mu(A) &= \int_B f(x)\sqrt{1+[f'(x)]^2}\,dxd\varphi \\
&= 2\pi \int_0^h\sqrt{1+\left(\frac{R-r}{h}\right)^2}\,\left[\,r+(R-r)\frac{x}{h}\,\right]dx \\
&=\pi(R+r)h\,\sqrt{1+\left(\frac{R-r}{h}\right)^2} \\[3mm]
&=\underline{\underline{\pi(R+r)\sqrt{h^2+(R-r)^2}}}. \Akehys \loppu
\end{align*}

\begin{Exa}
Laske viivoitinpinnan
\[
\vec r\,(u,v)=v[\,\cos (\omega u)\vec i+\sin (\omega u)\vec j\,]+u\vec k, \quad
(u,v)\in [0,H]\times [0,L]
\]
pinta-ala.
\end{Exa}
\begin{multicols}{2} \raggedcolumns
\ratk
\begin{align*}
\partial_u\vec r\times\partial_v\vec r
          &= \begin{detmatrix} 
             \vec i & \vec j & \vec k \\
             -\omega v\sin (\omega u) & \omega v\cos (\omega u) & 1 \\
             \cos (\omega u) & \sin (\omega u) & 0 
             \end{detmatrix} \\
          &= -\sin (\omega u)\vec i+\cos (\omega u)\vec j -\omega v\vec k \\[2mm]
\impl \ J &= \abs{\partial_u\vec r\times\partial_v\vec r\,}=\sqrt{1+\omega^2 v^2}.
\end{align*}
\begin{figure}[H]
\begin{center}
\import{kuvat/}{kuvaUint-39.pstex_t}
\end{center}
\end{figure}
\end{multicols}

\begin{align*}
\mu(A) &= \int_0^H\left(\int_0^L \sqrt{1+\omega^2 v^2}\,dv\right)du \\
&= H\int_0^L \sqrt{1+\omega^2 v^2}\,dv \\
&\qquad [\ \text{sijoitus}\ \ \omega v=\sinh t,\quad dv=\inv{\omega}\cosh t\,dt,\quad 
                              \alpha=\text{arsinh}\, (\omega L)\ ] \\
&=H\inv{\omega}\int_0^\alpha \cosh^2 t\,dt \\
&=H\inv{\omega}\sijoitus{0}{\alpha} \left(\frac{1}{2}\sinh t\cosh t+\frac{1}{2}\,t\right) \\
&=\underline{\underline{
    \frac{1}{2}\left[\ln (\omega L +\sqrt{\omega^2 L^2+1})
                                   +\omega L\sqrt{\omega^2 L^2+1}\right]H\inv{\omega}}}. \loppu
\end{align*}

\subsection{Avaruuskulma}
\index{avaruuskulma|vahv}

Olkoon $V\subset\R^3$ avaruuden osajoukko, käytännössä esim.\ kiinteä kappale, pinta tai pinnan
osa. Määritellään pallokoordinaatistossa joukko $B\subset[0,\pi]\times[0,2\pi]$ ehdolla
\[
B=\{\,(\theta,\varphi)\ | \ P=(r,\theta,\varphi) \in V\ \text{jollakin}\ r>0\,\}.
\]
Tällöin $V$ näkyy origosta suuntaan $\vec e_r(\theta,\varphi)$ täsmälleen kun
$(\theta,\varphi) \in B$. Olkoon $A$ yksikköpallon $S$ osa, joka vastaa
pallonpintakoordinaattien $(\theta,\varphi)$ joukkoa $B$. Tällöin sanotaan, että $V$ näkyy
origosta \kor{avaruuskulmassa} (engl.\ solid angle)
\[
\Omega=\int_A dS=\int_B \sin\theta\,d\theta d\varphi.
\]
Tämän suurin arvo (kun $A=S$) on $4\pi$.
\begin{Exa}
Missä avaruuskulmassa lieriöpinta
\[
A=\{(x,y,z)\in\R^2 \; | \; x^2+y^2=R^2, \; 0\leq z\leq H\}
\]
näkyy origosta?
\end{Exa}
\ratk Kyseisellä pinnalla pallonpintakoordinaatit saavat arvot
\[
(\theta,\varphi) \in B=[\theta_0,\pi/2]\times[0,2\pi],\quad 
                 \cos\theta_0=\frac{R}{\sqrt{R^2+H^2}}\,,
\]
joten
\[
\Omega = \int_B \sin\theta\,d\theta d\varphi
       = \int_0^{2\pi} d\varphi\cdot\int_{\theta_0}^{\pi/2} \sin\theta\,d\theta
       = 2\pi\cos\theta_0
       =\underline{\underline{2\pi\,\frac{H}{\sqrt{R^2+H^2}}}}\,. \loppu
\]

\Harj
\begin{enumerate}

\item \label{H-Uint-9: pyörähdyspintojen muuntosuhteet} 
Halutaan laskea pinta-alamitan muuntosuhde $J$ pyörähdyspinnalle, joka syntyy, kun
a) $xy$-tason käyrä $K: y=f(x)$ pyörähtää $x$-akselin ympäri, b) $yz$-tason käyrä
$K: y=f(z)$ pyörähtää $z$-akselin ympäri. Laske muuntosuhde
parametrisaatioille \vspace{1mm}\newline
a) \ $y=f(x)\cos\varphi,\,\ z=f(x)\sin\varphi$, \newline 
b) \ $x=r\cos\varphi,\,\ y=r\sin\varphi,\,\ z=f(r)$.

\item
Laske sen pinnan ala, jonka lieriö $\,L:\, x^2+y^2=a^2$ ($a>0$) leikkaa \vspace{1mm}\newline
a) kartiosta $\,z^2=x^2+y^2$, \newline
b) satulapinnasta $\,az=xy$, \newline
c) paraboloidista $\,az=x^2+y^2$. 

\item
Laske pinta-ala $\mu(A)$, kun $A$ on \vspace{1mm}\newline
a) joukon $V=\{(x,y,z)\in\R^3 \mid x^2+y^2 \le z \le \sqrt{x^2+y^2}\,\}$ 
   reunapinta $\partial V$, \newline
b) parametrinen pinta $\,x=8u^2,\ y=v^2,\ z=4uv,\ (u,v)\in[0,1]\times[0,3]$.

\item
a) Laske puolipallon (pinnan) keskiö. \vspace{1mm}\newline
b) Pinnalle $S:\,x^2+y^2=R^2,\ 0 \le z \le H$ on jakautunut tasaisesti massa $m$. Laske
lieriökoordinaateilla hitausmomentit koordinaattiakselien suhteen.

\item \index{Guldinin sääntö}
Olkoon $f(x) \ge 0$, kun $x\in[a,b]$ ja $S$ pinta, joka syntyy, kun käyrä
$K:\,y=f(x)\ \ja\ x\in[a,b]$ pyörähtää $x$-akselin ympäri. Todista \kor{Guldinin sääntö}:
$S$:n pinta-ala = $K$:n pituus kertaa $K$:n keskiön pyörähdyksessä kulkema matka.

\item
Laske, missä avaruuskulmassa kohde näkyy origosta: \vspace{1mm}\newline
a) \ $R$-säteinen kiekko tasolla $z=a>0$, keskipiste $z$-akselilla. \newline
b) \ $R$-säteinen kuula, jonka keskipisteen etäisyys origosta $=a \ge R$. \newline
c) \ Puolikartion rajaama joukko $A:\,z\ge\sqrt{x^2+y^2}+a,\ a \ge 0$. \newline
d) \ Pararaboloidi $\,z=x^2+y^2+a, \ a\in\R$.

\item (*) \index{Vivian'in ikkuna} 
\kor{Vivian'in ikkunaksi} sanotaan sitä pintaa $A,$ jonka lieriö $x^2+y^2=Rx$ erottaa pallosta
$x^2 + y^2 + z^2 = R^2$. Laske ko.\ pinnan ala.

\item (*)
Olkoon $A\subset\R^3$ lieriöiden
\[
S_1:\ x^2+y^2=R^2, \quad S_2:\ y^2+z^2=R^2, \quad S_3:\ x^2+z^2=R^2
\]
sisään jäävä joukko. Laske $A$:n reunapinnan $\partial A$ pinta-ala.

\item (*) \index{zzb@\nim!N:s ydinvoimala} 
(N:s ydinvoimala) Voimalaitoksen jäähdytystornin vaipan ulkopinnan parametriesitys on
(vrt.\ Esimerkki \ref{parametriset käyrät}:\,\ref{jäähdytystorni})
\[
\begin{cases}
\,x=a[(2-2v)\cos u-v\sin u], \\
\,y=a[v\cos u +(2-2v)\sin u], \\
\,z=3av,
\end{cases}
\]
missä $a$=50 m ja $(u,v)\in B= [0,2\pi]\times [0,1]$. Vaippa on valmistettu betonista ja sen
paksuus on 20 cm. Laske tarvittavan betonin määrälle likiarvo käyttäen pintaintegraalia.

\item (*)
a) Levy
\[
A=\{\,(x,y,z) \ | \ \frac{x^2}{a^2}+\frac{y^2}{b^2} \le 1,\ z=c\,\} \quad (a,b,c>0)
\]
näkyy origosta avaruuskulmassa $\Omega$. Johda $\Omega$:lle laskukaava muotoa
$\Omega=\int_0^{2\pi} f(\varphi)\,d\varphi$. \ b) Laske, missä avaruuskulmassa avaruusneliö
\[
K=\{\,(x,y,z) \ | \ 0 \le x,y \le 1,\ z=1\,\}
\]
näkyy origosta.

\end{enumerate} %Pintaintegraalit

\chapter{Gaussin ja Stokesin lauseet}

\kor{Gaussin lause} ja \kor{Stokesin lause} ovat usean muuttujan differentiaali- ja 
integraalilaskun keskeisiä tuloksia. Lauseet tunnetaan enemmän käyttökelpoisuutensa kuin 
matemaattisen suuruutensa vuoksi, ja niihin viitataankin usein arkisissa yhteyksissä nimillä 
'Gaussin kaava' ja 'Stokesin kaava'. Hieman yksinkertaistaen näissä kaavoissa on kyse 
integraalikaavan
\[
\int_a^b f'(x)\,dx=f(b)-f(a)
\]
yleistämisestä koskemaan useamman (käytännössä yleensä kahden tai kolmen) muuttujan 
vektoriarvoisia funktioita eli \pain{vektorikenttiä}. Gaussin lauseeseen (kaavaan) vedotaan 
hyvin usein silloin, kun erilaiset fysiikan \kor{säilymislait} halutaan kirjoittaa 
osittaisdifferentiaaliyhtälöitten muotoon. Myös Stokesin lauseella on tällaista käyttöä etenkin
sähkömagnetiikassa.

Luvussa \ref{polkuintegraalit} tarkastellaan ensin viivaintegraaleille sukua olevia
\kor{polkuintegraaleja}. Näillä on käyttöä Gaussin ja Stokesin lauseiden yhteydessä ja
yleisemminkin fysikaalisten vektorikenttien sovelluksissa. Luvussa \ref{gaussin lause}
johdetaan Gaussin lause tasossa ja avaruudessa ja esitetään lauseen yleistetty muoto.
Lähtökohtana ovat taso- ja avaruusintegraaleja koskevat \kor{Greenin kaavat}. Luvussa
\ref{Gaussin lauseen sovelluksia} tarkastellaan esimerkkien valossa Gaussin lauseen käyttöä,
kun halutaan johtaa fysiikan osittaisdifferentiaaliyhtälöitä tai fysikaalisten vektorikenttien
jatkuvuusehtoja materiaalirajapinnoilla. 

Luvussa \ref{stokesin lause} johdetaan Stokesin lause ensin tasoon rajoittuen. Yleistettäessä
tulos koskemaan avaruuden pintoja tarvitaan pinnan \kor{suunnistuvuuden} käsite. Luvussa 
\ref{pyörteetön vektorikenttä} ratkaistaan Stokesin lauseen avulla fysiikassa keskeinen 
vektorikenttiä koskeva kysymys: Millä ehdoilla pyörteetön kenttä on gradienttikenttä eli 
lausuttavissa skalaaripotentiaalin avulla? 
 %Gaussin ja Stokesin lauseet
\section{Vektorikentät ja polkuintegraalit} \label{polkuintegraalit}
\alku
\index{polkuintegraali|vahv}

Luvussa \ref{viivaintegraalit} tarkasteltiin viivaintegraaleja, joissa tason tai avaruuden
käyrän yli integroidaan kaarenpituusmitan suhteen. Tässä luvussa tarkastelun kohteena on toinen
viivaintegraalien luokka, jolle käytetään jatkossa nimeä \kor{polkuintegraalit}
(engl.\ path integral, suom.\ myös \kor{tieintegraali}). Polkuintegraaleille on ominaista, että
integrointi käyrää pitkin tapahtuu tiettyyn \pain{suuntaan}, siksi nimitys \kor{polku}, jonka
voi tulkita suunnatuksi käyräksi. Suunnan ohella polkuintegraaleille on tyypillistä, että
integrointiin liittyvä mitta \pain{ei} ole kaarenpituusmitta vaan muu yksi\-ulotteinen mitta,
joka on tapauskohtainen. Polun suunta vaikuttaa polkuintegraaliin niin, että jos vain suunta
vaihtuu, eli polku pysyy muuten (käyränä) samana, niin polkuintegraalin arvo vaihtuu
vastaluvukseen. Polun parametrisoinnin kautta tämä vastaa määrätyn integraalin vaihtosääntöä,
ks.\ esimerkit jäljempänä.

Jatkossa rajoitutaan sellaisiin polkuintegraaleihin, jotka sovelluksissa liitetään yleensä
fysikaalisiin vektorikenttiin (kuten voima-, sähkö- ja magneettikenttiin). Vektorikenttiin
liittyvillä polkuintegraaleilla on jatkossa käyttöä myös Gaussin ja Stokesin lauseiden
yhteydessä.\footnote[2]{Matemaattisissa teksteissä erilaisten viivaintegraalien nimet eivät ole
täysin vakiintuneet. Esim.\ saatetaan puhua 'vektorikenttien viivantegraaleista', kun
tarkoitetaan polkuintegraaleja tämän tekstin merkityksessä.}

Erotukseksi käyrästä polku merkitään jatkossa symbolilla $p$ tai tarkemmin
\[
p: A \kohti B,
\]
jolloin merkintä kertoo sekä polun (käyrän) päätepisteet että polun suunnan. Luontevasti
polun suunnan määrittää parametrisointi: suunta on joko parametrin kasvusuunta tai vastakkainen
suunta. Jos polku on parametrisoitu välillä $t\in[a,b]$, niin polku voidaan merkitä tarkemmin
kuten parametrinen käyrä:   
\[ 
p:\ t \in [a,b]\ \map\ \vec r\,(t).
\]
Parametrisoinnin ei tarvitse olla 1--1, joten polku (kuten parametrinen käyrä) voi leikata
itsensä tai kiertyä itsensä päälle.
\begin{figure}[H]
\begin{center}
\import{kuvat/}{kuvaUint-34.pstex_t}
\end{center}
\end{figure}

Jatkossa tarkastelun kohteena ovat tason tai avaruuden polut muotoa $t\in[a,b]\map(x(t),y(t))$
tai $t\in[a,b]\map(x(t),y(t),z(t))$. Funktioiden $x(t)$, $y(t)$, $z(t)$ oletetaan olevan joko
jatkuvasti derivoituvia välillä  $[a,b]$ tai toteuttavan vastaavat, hieman heikommat
säännöllisyysehdot, vrt. Luku \ref{viivaintegraalit}. 

Kuten aiemmin, voidaan laskea \kor{polun pituus} integraalina
\[
\mu(p)=\int_a^b \abs{\vec r\,'(t)}\,dt.
\]
Tässä on kuitenkin kyse jo ennestään tutusta viivaintegraalista, jossa mitta on
kaarenpituusmitta eikä polun suunnalla ole väliä.

Polkuintegraaleja (jatkon kannalta myös merkittävimpiä) ovat
\begin{equation} \label{polkuintegraaleja}
\int_p f\,dx, \quad \int_p f\,dy, \quad \int_p f\,dz, \tag{$\star$}
\end{equation}
missä $f(x,y,z)$ (tasossa $f(x,y)$) on tunnettu funktio. Näihin integraaleihin liittyvä
mitta on tavallinen ($1$-ulotteinen) Jordan-mitta. Jos tunnetaan polun parametrisointi
välillä $t\in[a,b]$, niin esimerkiksi $\int_p f\,dx$ lasketaan yksinkertaisesti kirjoittamalla
$dx=x'(t)dt$ (kuten muuttujan vaihdossa). Jos vielä oletetaan, että $t=a$ vastaa polun
alkupistettä ja $t=b$ loppupistettä, niin saadaan laskukaava
\[
\int_p f\,dx = \int_a^b f(x(t),y(t),z(t)\,x'(t)\,dt.
\]
Parametrisoinnin vaihto vastaa tässä kaavassa (toista) muuttujan vaihtoa, joten parametrisointi
ei vaikuta integraalin arvoon.
\begin{Exa} Tason polku $p$ kulkee pisteestä $(0,0)$ pisteeseen $(1,1)$ pitkin käyrää $y=x^2$
ja polku $-p$ pitkin samaa käyrää pisteesät $(1,1)$ pisteeseen $(0,0)$. Laske
$\int_p xy\,dy$ ja $\int_{-p} xy\,dy$.
\end{Exa}
\ratk Valitaan molemmissa integraaleissa parametriksi $t=x$\,:
\begin{align*}
&\int_p xy\,dy     = \int_0^1 x \cdot x^2 \cdot 2x\,dx = \int_0^1 2x^4\,dx
                  = \underline{\underline{\frac{2}{5}}}\,, \\
&\int_{-p} xy\,dy  = \int_1^0 2x^4\,dx = -\int_0^1 2x^4\,dx 
                  = \underline{\underline{-\frac{2}{5}}}\,. \loppu
\end{align*}

Esimerkin jälkimmäisessä integraalissa käytettiin määrätyn integraalin vaihtosääntöä. --- Itse
asiassa kun vaihtosääntö huomioidaan, niin määrätty integraali on itsekin tulkittavissa
polkuintegraaliksi: $\int_a^b f(x)\,dx = \int_p f\,dx$, missä $p$ on $\R$:n polku, jonka
alkupiste on $a$ ja loppupiste $b$ (!).

\subsection{Polkuintegraali $\int_p \vec F \cdot d\vec r$}
\index{polkuintegraali!a@työintegraali|vahv}
\index{tyzz@työintegraali|vahv}

Fysiikan sovelluksissa polkuintegraalit liittyvät usein vektorikenttiin. Tyypillinen esimerkki
on \pain{voimakentässä} $\vec F(x,y,z)$ liikkuva (pistemäinen) kappale, jonka liikerata
tunnetaan jollakin aikavälillä $t\in [a,b]$. Tällöin voimakentän kappaleeseen tekemä
\pain{t}y\pain{ö} lasketaan polkuintegraalina pitkin kappaleen kulkemaa liikerataa. Liikerata
tulkitaan siis poluksi $p:t \map \vec r\,(t)$, $t\in [a,b]$. Jos $\vec F=\text{vakio}$, niin
fysiikan lakien mukaan työ $=|\vec F| \cdot s$, missä $s=$ polulla kuljettu matka voiman
vaikutussuunnassa, eli
\[
W=\vec F\cdot[\vec r\,(b)-\vec r\,(a)]\quad (\vec F\text{ vakio}).
\]
Vakiovoimakentän tekemän työn kannalta polkua siis 'mittaa' vektori
\[
\vec\mu(p)=\vec r\,(b)-\vec r\,(a).
\]
Jos tämä tulkitaan polun (vektoriarvoiseksi) mitaksi, niin nähdään, että tämäkin mitta on
additiivinen: Jos $p_1$ ja $p_2$ ovat $p$:n osapolkuja vastaten parametrin arvoja väleillä
$[a,t_0]$ ja $[t_0,b]$, $t_0\in (a,b)$, niin $\vec \mu(p)=\vec \mu(p_1)+\vec\mu(p_2)$,
vrt.\ kuvio.
\begin{figure}[H]
\begin{center}
\import{kuvat/}{kuvaUint-35.pstex_t}
\end{center}
\end{figure}
Entä jos voimakenttä $\vec F$ ei ole vakio, mutta on jatkuva? Tällöin menetellään niinkuin
integraaleissa yleensä: Otetaan käyttöön välin $[a,b]$ jako $\{t_k, \ k=0,\ldots,n\}$,
$a=t_0<t_1<\ldots t_n=b$, jolloin polku $p$ jakautuu peräkkäisiksi osapoluiksi $\Delta p_k$,
$k=1\ldots n\,$ vastaten parametrin arvoja väleillä $[t_{k-1},t_k]$. Koska funktio
$t \map \vec r\,(t)$ on jatkuva, niin osapolkujen $\Delta p_k$ päätepisteet 
$\vec r\,(t_{k-1})=\vec r_{k-1}$ ja $\vec r\,(t_k)=\vec r_k$ tulevat yhä lähemmäksi toisiaan
jaon tihetessä. Tällöin voima $\vec F$ on osapolulla $\Delta p_k$ likimain vakio 
(koska $\vec F$ oli jatkuva), joten voiman tekemä työ tällä osapolulla on likimain
\[
\Delta W_k\approx\vec F(\vec r_{k-1})\cdot\vec\mu(\Delta p_k),\quad 
\vec\mu(\Delta p_k)=\vec r_k-\vec r_{k-1}\,.
\]
Kun jaon tiheysparametri $h=\max_k|\vec r_k-\vec r_{k-1}|\kohti 0$, saadaan voimakentän
tekemälle kokonaistyölle integraalilauseke
\[
W=\Lim_{h\kohti 0}\sum_{k=1}^n\Delta W_k=\int_p \vec F\cdot d\vec\mu.
\]
Tämä on siis tulkittava polkuintegraaliksi vektorimitan $\vec\mu$ suhteen (!). Integraali saa
hieman konkreettisen muodon, kun käytetään merkintää $d\vec\mu=d\vec r$, jolloin työn
integraalikaava siis on
\[
\boxed{\kehys\quad W=\int_p \vec F\cdot d\vec r \quad (\text{työintegraali}). \quad}
\]
Määritelmän mukaisesti $W$:lle saadaan likiarvoja summien avulla. Esimerkiksi
\[
W\approx\sum_{k=1}^n \vec F(\vec r_{k-1})\cdot (\vec r_k-\vec r_{k-1}).
\]
Näin laskettaessa polusta $p$ ei tarvitse tehdä voimakkaita säännöllisyysoletuksia. Esimerkiksi
työ $W$ voidaan laskea, vaikka $p$ ei olisi suoristuva (ts.\ pituusmitta ei määritelty). Tämä
johtuu siitä, että työintegraali mittaa vain siirtymää voiman vaikutussuunnassa, ei
kaarenpituutta.

Jos oletetaan parametrisointi $p: t\in[a,b] \map \vec r\,(t)$, niin työintegraalissa voidaan
kirjoitta $d\vec r=\dvr(t)dt$, jolloin saadaan laskukaava
\[
W=\int_a^b \vec F(\vec r(t))\cdot \dvr(t)\,dt.
\]
Jos taas voimakenttä esitetään koordinaattimuodossa
\[
\vec F(x,y,z)=F_1(x,y,z)\vec i + F_2(x,y,z)\vec j+F_3(x,y,z)\vec k,
\]
niin kirjoittamall $d\vec r=dx\,\vec i+dy\,\vec j+dz\,\vec k$ työintegraali purkautuu
polkuintegraalien \eqref{polkuintegraaleja} summaksi:
\[
W=\int_p (F_1\,dx+F_2\,dy+F_3\,dz).
\]
Riittävän säännöllisellä polulla työintegraalin voi ilmaista kolmannellakin tavalla, sillä
\[
d\vec r = \frac{\dvr(t)}{|\dvr(t)|}|\dvr(t)|dt = \vec t\,ds,
\]
missä $\vec t$ on polun suuntainen yksikkötangenttivektori. Tämän mukaan siis työintegraali
voidaan haluttaessa liittää myös kaarenpituusmittaan laskukaavalla
\[
W = \int_p \vec F\cdot\vec t\,ds.
\]
Kuten kaavan johdosta ilmenee, tässä on $\vec t\,ds=\dvr(t)\,dt$, joten riippuvuus
kaarenpituusmitasta on näennäinen.
\begin{Exa}
Määritä voimakentän
\[
\vec F=(x^2-y)\vec i-2xy\vec j
\]
tekemä työ kappaleen liikkuessa polulla
\[
p: \ x(t)=2\cos t, \ y(t)=\sin t, \ t\in [0,2\pi].
\]
\end{Exa}
\ratk $\quad dx=-2\sin t\,dt,\quad dy=\cos t\,dt, \quad t\in [0,2\pi]$
\begin{align*}
\impl \quad W &= \int_0^{2\pi} [\,(4\cos^2 t-\sin t)(-2\sin t)-(4\cos t\sin t)\cos t\,]\,dt \\
              &= \int_0^{2\pi} (-12\cos^2 t\sin t+2\sin^2 t)\,dt \\
              &= \sijoitus{0}{2\pi} (4\cos^3 t-\cos t\sin t+t) 
               = \underline{\underline{2\pi}}. \loppu
\end{align*}

\subsection{Gradienttikenttä ja työintegraali}
\index{polkuintegraali!a@työintegraali|vahv}
\index{tyzz@työintegraali|vahv}
\index{gradienttikenttä|vahv}

Jos voimakenttä $\vec F$ on g\pain{radienttikenttä}, eli lausuttavissa skalaaripotentiaalin $u$
avulla muodossa
\[
\vec F=-\nabla u,
\]
niin työintegraali voidaan laskea hyvin yksinkertaisesti. Nimittäin tässä tapauksessa on
derivoinnin ketjusäännön (Luku \ref{osittaisderivaatat}) perusteella
\[
\vec F(\vec r(t))\cdot \dvr(t) = -\nabla u(\vec r(t))\cdot \dvr(t)
                               = -\frac{d}{dt} u(\vec r(t)),
\]
joten
\[
W = \int_a^b \vec F(\vec r(t))\cdot \dvr(t)\,dt 
  = -\sijoitus{a}{b} u(\vec r(t))
  = u(\vec r(a))-u(\vec r(b)).
\]
Siis gradienttikentän työintegraali määräytyy pelkästään polun päätepisteistä:
\[ 
\boxed{ \begin{aligned} \quad\ygehys 
                 &\text{Gradienttikentän tekemä työ} \\
                 &= \text{potentiaaliero polun alku- ja loppupisteiden välillä}. \quad\agehys
           \end{aligned} } 
\]

\jatko \begin{Exa} (jatko). Esimerkissä polun alku- ja loppupisteet ovat samat. Koska
$W\neq 0$, ei esimerkin kenttä $\vec F$ ole gradienttikenttä. Sen sijaan jos esimerkiksi
\[
\vec F=(x^2-y^2)\vec i-2xy\vec j,
\]
niin ilman enempää laskemista selviää, että $W=0$, sillä
\[
\vec F=-\nabla(-\frac{1}{3}x^3+xy^2). \loppu
\]
\end{Exa}

\subsection{Vektoriarvoiset polkuintegraalit}
\index{polkuintegraali!b@vektoriarvoinen p.-integraali|vahv}

Fysiikan sovelluksissa (esimerkiksi sähkömagnetiikassa) esiintyy myös vektoriarvoisia 
polkuintegraaleja muotoa
\[
\int_p f\,d\vec r\quad\text{tai}\quad\int_p \vec F\times d\vec r.
\]
Nämä voidaan laskea parametrisoinnin avulla samalla periaatteella kuin työintegraalikin.
\jatko \begin{Exa} (jatko) Laske $\int_p x\,d\vec r\,$ ja $\int_p \vec r \times d\vec r\,$,
kun $p$ on esimerkin polku.
\end{Exa}
\ratk
\begin{align*}
\int_p x\,d\vec r\,
&= \int_0^{2\pi} x(t)[x'(t)\vec i+y'(t)\vec j\,]\,dt \\
&= \int_0^{2\pi} 2\cos t\,(-2\sin t\,\vec i+\cos t\,\vec j\,)\,dt \\
&= -\vec i\int_0^{2\pi} 4\cos t\sin t\,dt + \vec j\int_0^{2\pi} 2\cos^2 t\,dt \\
&= -\vec i\sijoitus{0}{2\pi} 2\sin^2 t 
   +\vec j\sijoitus{0}{2\pi}(t+\cos t\sin t)
 = \underline{\underline{2\pi\vec j}}, \\
\int_p \vec r\times d\vec r\, 
&= \int_0^{2\pi} [x(t)\vec i+y(t)\vec j\,]\times[x'(t)\vec i+y'(t)\vec j\,]\,dt \\
&= \int_0^{2\pi}(2\cos t\,\vec i+\sin t\,\vec j\,)\times
               (-2\sin t\,\vec i+\cos t\,\vec j\,)\,dt \\
&= \int_0^{2\pi} (2\cos^2 t+2\sin^2 t)\vec k\, dt
 = \vec k \int_0^{2\pi} 2\,dt
 =\underline{\underline{4\pi\vec k}}. \loppu
\end{align*}

\Harj
\begin{enumerate}

\item
Laske polkuintegraali $\int_p (9x^2y\,dx-11xy^2\,dy)$, kun polku $p$ kulkee origosta 
pisteeseen $(1,1)$ \ a) pitkin käyrää $x(t)=t^2,\ y(t)=t^3$, \ b) pitkin suoraa, \ 
c) pitkin käyrää $\vec r\,(t)=t\vec i+t^\alpha\vec j,\ \alpha>0$.

\item
Laske polkuintegraali $\int_p (xdy-ydx)$, kun polku $p$ kulkee pisteestä $(1,0)$ pitkin
logaritmista spiraalia $r=e^{-\varphi}$ origoon.

\item
Laske polkuintegraali $\int_p [(y-x)\,dx+xy\,dy]$, kun polku polku $p$ on määritelty 
seuraavasti:

a) Pisteestä $(1,0)$ pisteeseen $(-1,0)$ yksikköympyrää pitkin vastapäivään

b) Pisteestä $(1,0)$ pisteeseen $(-1,0)$ yksikköympyrää pitkin myötäpäivään

c) Murtoviiva $ABCD$, missä $A=(1,0)$, $B=(1,1)$, $C=(-1,1)$ ja $D=(-1,0)$

d) Pisteestä $(1,0)$ yksikköympyrää pitkin takaisin lähtöpisteeseen vastapäivään kiertäen

e) Origosta pitkin $x$-akselia pisteeseen $(\pi,0)$ ja takaisin origoon pitkin käyrää $y=\sin x$

f) Pisteestä $(a,0)$ takaisin lähtöpisteeseen kiertäen vastapäivään ellipsiä
   $x^2/a^2+y^2/b^2=1$ ($a,b>0$).

\item 
Laske seuraavat polkuintegraalit.

a) $\int_p \vec F \cdot d\vec r$, kun $\vec F(x,y,z)=\sqrt y\,\vec i+2x\vec j+3y\vec k$ ja
polku kulkee origosta pisteeseen $(3,9,27)$ pitkin käyrää
$\vec r\,(t)=t\vec i+t^2\vec j+t^3\vec k$.

b) $\int_p \vec F\,\cdot\,d\vec r$, kun $\vec F(x,y,z)=x^3\vec i+y^2\vec j+z\vec k$ ja polku
kulkee origosta pisteeseen $(1,1,2)$ pitkin käyrää $S:\,x=y,\ z=x^2+y^2$.

c) $\int_p \vec F \times d\vec r$, kun $\vec F(x,y,z)=xyz\vec i+y^2\vec k$ ja polku $p$ seuraa
tason $x=y$ ja pinnan $z=x^2$ leikkauskäyrää origosta pisteeseen $(2,2,4)$.

\item 
Polun $p$ alkupiste on $(-1,1,-1)$ ja loppupiste $(1,2,3)$. Laske näillä tiedoilla seuraavat
polkuintegraalit (työintegraalit) kirjoittamalla integraalit ensin muotoon 
$\int_p \nabla u \cdot d\vec r$. \vspace{1mm}\newline
a) \ $\int_p (yz\,dx + zx\,dy + xy\,dz) \qquad\qquad\ \ $
b) \ $\int_p (yz^2\,dx+xz^2\,dy+2xyz\,dz)$ \newline
c) \ $\int_p [e^x y\,dx+(e^x+z^2)\,dy+2yz\,dz] \quad\ $
d) \ $\int_p \sin\frac{\pi(x+y+z)}{6}\,(dx+dy+dz)$

\item
Määritellään $g(u,v)=\int_p (y\,dx+2x\,dy)$, missä $p$ kulkee origosta pisteeseen $(u,v)$
suoraa pitkin. Mikä on $g$:n maksimiarvo yksikköympyrällä?

\item (*)
Kappale, jonka massa $=m$, liikkuu aikavälillä $[t_1,t_2]$ pitkin polkua $p:\,t\map\vec r\,(t)$
pisteestä $P_1$ pisteeseen $P_2$. Liikkeen aikana kappaleeseen vaikuttaa voimakenttä $\vec F$.
Johda liikeyhtälöstä $m\vec r\,''=\vec F$ energiaperiaate
\[
\frac{1}{2}\,m(v_2^2-v_1^2)=\int_p \vec F \cdot d\vec r, \quad 
           \text{missä}\ v_i=\abs{\dvr(t_i)},\ i=1,2.
\]
\end{enumerate} %Vektorikentät ja polkuintegraalit
\section{Gaussin lause} \label{gaussin lause}
\alku

Lähdetään tarkastelemaan integraalikaavan
\[
\int_a^b f'(x)\,dx=f(b)-f(a)
\]
yleistämistä, ensin yhdessä dimensiossa. Olkoon $A\subset\R$ äärellinen yhdistelmä 
pistevieraita, suljettuja välejä:
\[
A=\bigcup_{i=1}^n A_i,\quad A_i=[a_i,b_i],\quad A_i\cap A_j=\emptyset,\; i\neq j.
\]
Tällöin ym.\ kaava voidaan kirjoittaa muotoon
\begin{align*}
\int_A f'\,dx &= \sum_{i=1}^n [f(b_i)-f(a_i)] \\
&= \sum_{x\in\partial A} \omega(x)f(x),
\end{align*}
missä $\partial A$ on $A$:n reuna (koostuu pisteistä $a_i$, $b_i$, $i=1\ldots n$) ja $w(x)$ saa
arvoja $\pm 1$ seuraavan säännön mukaan:
\[
\omega(x)=\begin{cases} +1, &\text{jos } (x-\delta,x)\subset A\,\text{ jollakin } \delta>0, \\
                        -1, &\text{jos } (x,x+\delta)\subset A\,\text{ jollakin } \delta>0.
\end{cases}
\]
\begin{figure}[H]
\setlength{\unitlength}{1cm}
\begin{center}
\begin{picture}(10,1.5)
\multiput(0,1)(6,0){2}{\line(1,0){4}} \multiput(2,1)(6,0){2}{\line(0,1){0.1}} 
\multiput(1.9,1.2)(6,0){2}{$x$}
\multiput(0,0)(7.5,0){2}{\multiput(0.5,0.9)(0.2,0){7}{\line(1,1){0.2}}}
\multiput(1,1.2)(7.5,0){2}{$A$}
\put(1,0){$\omega(x)=+1$} \put(7,0){$\omega(x)=-1$}
\end{picture}
\end{center}
\end{figure}
Siirrytään nyt kahteen dimensioon. Olkoon $f=f(x,y)$ määritelty ja jatkuvasti derivoituva 
joukossa $A\subset\R^2$ (josta tehdään hetimiten yksinkertaistavia olettamuksia) ja 
tarkastellaan integraalia
\[
\int_A \frac{\partial f}{\partial x}\,dxdy.
\]
Jatkossa oletetaan, että joukko $A$ on suljettu ja jaettavissa äärellisen moneen yksinkertaista
muotoa olevaan osaan $A_i$ siten, että osat koskettavat toisiaan enintään reunoillaan, ts.
\[
A=\bigcup_{i=1}^n A_i,\quad A_i\cap A_j=\partial A_i\cap \partial A_j, \; i\neq j.
\]
Osat $A_i$ oletetaan edelleen kaikki $y$-projisoituviksi. Tarkemmin sanoen, oletetaan, että
jokainen $A_i$ on muotoa
\[
A_i=\{\,(x,y)\in\R^2 \ | \ a_i(y)\leq x\leq b_i(y) \; \ja \; y\in [c_i,d_i]\,\},
\]
missä funktiot $a_i(y)$, $b_i(y)$ ovat välillä $[c_i,d_i]$  jatkuvia.
\begin{figure}[H]
\begin{center}
\import{kuvat/}{kuvapot-1.pstex_t}
\end{center}
\end{figure}
Joukkoa $A$ koskeva oletus siis on, että $A$:n ositus ym. tavalla on mahdollinen. Ajatellen 
sovelluksissa kohdattavia 'käytännön joukkoja' ei oletus ole kovin rajoittava, vrt.\ kuvio.
\begin{figure}[H]
\begin{center}
\import{kuvat/}{kuvapot-2.pstex_t}
\end{center}
\end{figure}
Kun $A$:n ositus em. tavalla on tehty, voidaan tarkastelun kohteena oleva integraali purkaa
osiin additiivisuusperiaatteella (sillä $\mu(A_i\cap A_j)=0$, $i\neq j$)\,:
\[
\int_A \frac{\partial f}{\partial x}\,dxdy
           =\sum_{i=1}^n \int_{A_i}\frac{\partial f}{\partial x}\,dxdy.
\]
Tässä on Fubinin lauseen mukaan
\begin{align*}
\int_{A_i} \frac{\partial f}{\partial x}\,dxdy 
&= \int_{c_i}^{d_i}\left(\int_{a_i(y)}^{b_i(y)}\frac{\partial f}{\partial x}\,dx\right)dy \\
&=\int_{c_i}^{d_i} [f(b_i(y),y)-f(a_i(y),y)]\,dy = \int_{S_i} \omega_i(x,y)f(x,y)\,dy,
\end{align*}
missä $S_i\subset \partial A_i$ ja funktio $w_i$ määritellään kuten kuviossa.
\begin{figure}[H]
\begin{center}
\import{kuvat/}{kuvapot-3.pstex_t}
\end{center}
\end{figure}
Kun saatu tulos
\[
\int_{A_i} \frac{\partial f}{\partial x}\,dxdy=\int_{S_i}\omega_i(x,y)f(x,y)\,dy
\]
summataan yli $i$:n ja huomataan, että
\[
i\neq j \; \ja \; (x,y)\in S_i\cap S_j \; \impl \; \omega_i(x,y)+\omega_j(x,y)=0,
\]
niin nähdään, että oikealla puolella viivaintegraalit yli osajoukkojen $A_i$ yhteisten 
(eli $A$:n sisään jäävien) reunaviivojen kuomoutuvat. Käyttäen kummallakin puolella integraalin 
additiivisuusperiaatetta saadaan tulos näin ollen muotoon
\[
\int_A \frac{\partial f}{\partial x}\,dxdy=\int_{\partial A} \omega^{(x)}f\,dy,
\]
missä $\omega^{(x)}(x,y)=0\,$ $x$-akselin suuntaisilla reunaviivan $\partial A$ osilla ja muuten
$\omega^{(x)}(x,y)$ saa arvoja $\pm 1$ kuvion mukaisesti.
\begin{figure}[H]
\begin{center}
\import{kuvat/}{kuvapot-4.pstex_t}
\end{center}
\end{figure}
Olettaen, että $A$ on jaettavissa samalla tavoin $x$-projisoituviin osiin, saadaan vastaavasti
integraalikaava
\[
\int_A \frac{\partial f}{\partial y}\,dxdy=\int_{\partial A} \omega^{(y)}f\,dx,
\]
missä $\omega^{(y)}(x,y)=0\ $ $y$-akselin suuntaisilla reunaviivan $\partial A$ osilla ja muuten
$\omega^{(y)}(x,y)$ saa arvoja $\pm 1$ kuvion mukaisesti.
\begin{figure}[H]
\begin{center}
\import{kuvat/}{kuvapot-5.pstex_t}
\end{center}
\end{figure}
Huomattakoon, että saaduissa integraalikaavoissa reunaviivan $\partial A$ yli laskettavat
integraalit ovat p\pain{olkuinte}g\pain{raale}j\pain{a} (vrt. edellinen luku). Nämä purkautuvat
äärellisiksi summiksi, joissa kukin termi on määrätty integraali yli suljetun välin, mitan
ollessa tavallinen $\R$:n pituusmitta. Kaavoissa ei siis edellytetä edes reunaviivan
$\partial A$ suoristuvuutta. Olettamalla reunaviivalle lisää säännöllisyyttä saadaan tulokset
kuitenkin helpommin muistettavaan muotoon. 

Em.\ ositukset jakavat reunaviivan $\partial A$ osiin, jotka ovat joko muotoa
\[
S_i=\{\,(x,y)\in\R^2 \ | \ x=f_i(y) \ \ja \ y\in [a_i,b_i]\,\}
\]
tai muotoa
\[
S_i=\{\,(x,y)\in\R^2 \ | \ y=g_i(x) \ \ja \ x\in [c_i,d_i]\,\}.
\]
Oletetaan tässä, että funktiot $f_i$ ovat jatkuvia suljetuilla väleillä $[a_i,b_i]$ ja että
derivaatat $f'_i$ ovat jatkuvia avoimilla väleillä $(a_i,b_i)$. Vastaavasti oletetaan, että
funktiot $g_i$ ovat jatkuvia suljetuilla väleillä $[c_i,d_i]$ ja derivaatat $g'_i$ jatkuvia
avoimilla väleillä $(c_i,d_i)$. (Nämäkään lisäoletukset eivät ole käytännön kannalta kovin
rajoittava, vrt.\ osituskuvio edellä.) Jatkossa sanottakoon joukkoa $A$, joka totetuttaa kaikki
\index{perusalue}%
tehdyt oletukset \kor{perusalueeksi}. Perusalue on siis joukko, joka on jaettavissa äärellisen
moneen $x$-projisoituvaan osaan, samoin äärellisen moneen $y$-projisoituvaan osaan siten, että
näiden osien reunaviivalla $\partial A$ sijaitsevat reunakäyrät toteuttavat tehdyt
säännöllisyysoletukset. Perusalue on kompakti joukko
(vrt.\ Luku \ref{usean muuttujan jatkuvuus}).
 
Jos $A$ on perusalue, niin reunalla $\partial A$ on, erillisiä pisteitä lukuunottamatta, 
\index{ulkonormaali}%
määritelty reunan \kor{ulkonormaali}, eli reunaa vastaan kohtisuora, joukosta $A$ poispäin
osoittava yksikkövektori, jota merkittäköön
\[
\vec n(x,y)=n_x(x,y)\vec i+n_y(x,y)\vec j,\quad \abs{\vec n}=1.
\]
\begin{figure}[H]
\begin{center}
\import{kuvat/}{kuvapot-6.pstex_t}
\end{center}
\end{figure}
Niissä pisteissä $(x_i,y_i)\in\partial A$, joissa ulkonormaali $\vec n$ on määritelty, on
funktion $\omega^{(x)}$ määritelmän mukaisesti
\[
\omega^{(x)}(x,y) = \begin{cases} 
                      +1,     &\text{jos } n_x(x,y)>0, \\ 
                      \ \ 0,  &\text{jos}\ n_x(x,y)=0, \\
                      -1,     &\text{jos } n_x(x,y)<0,
                    \end{cases} 
\]
ja $\omega^{(y)}(x,y)$ määräytyy vastaavasti $n_y(x,y)$:n perusteella. Kun nämä yhteydet otetaan
lukuun, niin voidaan kirjoittaa (vrt.\ kuvio),
\[
\omega^{(x)}dy=n_x\,ds,\quad \omega^{(y)}dx=n_y\,ds,
\]
\begin{multicols}{2} \raggedcolumns
\begin{figure}[H]
\begin{center}
\import{kuvat/}{kuvapot-7.pstex_t}
\end{center}
\end{figure}
\begin{align*}
\\
\abs{n_x} &= \sin\theta=\frac{dy}{ds} \\
\abs{n_y} &= \cos\theta=\frac{dx}{ds}
\end{align*}
\end{multicols}
\index{Greenin kaavat}%
jolloin saadaan helposti muistettavat \kor{Greenin tasokaavat}
\[
\boxed{\begin{aligned}
\quad \int_A \frac{\ykehys\partial f}{\partial x}\,dxdy &= \int_{\partial A} n_xf\,ds, \\
      \int_A \frac{\partial f}{\akehys\partial y}\,dxdy &= \int_{\partial A} n_yf\,ds.
\end{aligned} \qquad \text{(Greenin tasokaavat)}\quad }
\]
Korostettakoon vielä, että näissä kaavoissa kaarenpituusmitta on otettu käyttöön vain
muistisäännön vuoksi. Todellisuudessa integraalit kaavojen oikealla puolella ovat
polkuintegraaleja.

Tarkastellaan seuraavaksi $A$:ssa määriteltyä vektorikenttää
\[
\vec F(x,y)=F_1(x,y)\vec i+F_2(x,y)\vec j,
\]
joka olkoon jatkuvasti derivoituva. Soveltaen Greenin tasokaavoja saadaan
\begin{align*}
\int_A \nabla\cdot\vec F\,dxdy &= \int_A \frac{\partial F_1}{\partial x}\,dxdy
                                 +\int_A \frac{\partial F_2}{\partial y}\,dxdy \\
                               &= \int_{\partial A} (n_xF_1+n_yF_2)\,ds
                                = \int_{\partial A} \vec n\cdot\vec F\, ds. 
\end{align*}
Näin on johdettu
\begin{Lause} \vahv{(Gaussin lause tasossa}) \label{Gaussin lause tasossa}
\index{Gaussin lause (kaava)|emph} Jos $A \subset \R^2$ on perusalue ja
$\vec F$ on $A$:ssa määritelty, jatkuvasti derivoituva vektorikenttä, niin pätee
\[
\boxed{\kehys\quad \int_A \nabla\cdot\vec F\,dxdy=\int_{\partial A}\vec n\cdot\vec F\,ds. \quad }
\]
\end{Lause}

Lauseen \ref{Gaussin lause tasossa} laskukaavalla, jota sanotaan jatkossa
\kor{Gaussin tasokaavaksi}, on vastine myös kolmessa (ja useammassakin) dimensiossa. Olkoon
$V\subset\R^3$ ja oletetaan, että $V$ on jaettavissa äärellisen moneen $xy$-projisoituvaan,
$yz$-projisoituvaan, ja $xz$-projisoituvaan osaan samaan tapaan kuin edellä. Esimerkiksi
$xy$-projisoituvat osat $V_i$ ovat tällöin muotoa
\[ 
V_i = \{\,(x,y,z) \in \R^3\ | \ (x,y) \in B_i\ \ja\ a_i(x,y) \le z \le b_i(x,y)\,\}. 
\]
Tässä joukot $B_i \subset \R^2$ oletetaan edelleen tason perusalueiksi, ja lisäksi oletetaan,
että funktiot $a_i$ ja $b_i$ ovat $B_i$:ssa jatkuvia ja $B_i$:n sisäpisteissä jatkuvasti 
derivoituvia (osittaisderivaatat jatkuvia). Joukkoa $V$, joka toteuttaa nämä ja vastaavat
oletukset koskien $yz$- ja $xz$-projisoituvia osituksia, sanotaan $\R^3$:n perusalueeksi. Jos
$V$ on nämä ehdot täyttävä, niin vastaavaan tapaan kuin tasossa nähdään oikeaksi
integraalikaava
\[
\int_V \frac{\partial f}{\partial z}\,dxdydz = \int_{\partial V} \omega^{(z)} f\,dxdy,
\]
missä $\omega^{(z)}$ saa arvoja $0,\pm 1$ vastaavalla periaatteella kuin tasossa. Tässä
voidaan kirjoittaa $dxdy=\abs{n_z}dS$, missä $n_z$ on $\partial V$:n yksikkönormaalivektorin
$\,z$-komponentti ja $dS$ viittaa $\partial V$:n pinta-alamittaan 
(vrt.\ Luku \ref{pintaintegraalit}). Kun huomioidaan myös $\omega^{(z)}$:n merkinvaihtelu,
niin integraalikaavalle saadaan muoto
\[
\int_V \frac{\partial f}{\partial z}\,dxdydz = \int_{\partial V} n_z f\,dS,
\]
missä $\vec n$ on $\partial V$:n ulkonormaali (yksikkövektori). Tämä on yksi kolmesta
\index{Greenin kaavat}%
\kor{Greenin avaruuskaavasta} --- muut kaksi ovat ilmeisiä. Yhdistämällä nämä kaavat seuraa
\begin{Lause} \vahv{(Gaussin lause avaruudessa)} \label{Gaussin lause avaruudessa}
\index{Gaussin lause (kaava)|emph} Jos $V\subset\R^3$ on perusalue ja
$\vec F$ on $V$:ssä määritelty, jatkuvasti derivoituva vektorikenttä, niin pätee
\[
\boxed{\quad \int_V \nabla\cdot\vec F\,dxdydz
                    =\int_{\partial V} \vec n\cdot\vec F\,dS. \quad}
\]
\end{Lause}
Myös tässä \kor{Gaussin avaruuskaavassa} pinnan $\partial A$ pinta-alamitta palvelee vain
muistisääntönä. Todellisuudessa kaavan oikea puoli koostuu tasointegraaleista, kuten kaavan
johdosta ilmenee.

Gaussin avaruuskaavassa kirjoitetaan oikea puoli usein muotoon
\[
\int_{\partial A} \vec n\cdot\vec F\,dS = \int_{\partial A} \vec F \cdot d\vec a,
\]
missä $d\vec a=\vec n\,dS$ on 'vektoroitu' mitta. Vastaavasti voidaan tasokaavassa kirjoittaa
\[
\vec n\cdot\vec F\,ds=\vec F\cdot d\vec n,\quad d\vec n=\vec n\,ds.
\]
\begin{Exa} Olkoon $V = \{(x,y,z) \in \R^3 \mid x^2+y^2+z^2 \le R^2\}$ ja 
$\vec F(x,y,z) = x\vec i + 2y\vec j + 3z\vec k$. Laske integraali 
$\int_{\partial V} \vec n\cdot\vec F\,dS$ kahdella eri tavalla.
\end{Exa}
\ratk a) Pinnalla $\partial V$ on $\vec n = (x\vec i + y\vec j + z\vec k)/R$, joten
pallonpintakoordinaattien avulla suoraan laskien saadaan
\begin{align*}
\int_{\partial V} 
\vec n\cdot &\vec F\,dS = \int_{\partial V} R^{-1}(x^2+2y^2+3z^2)\,dS \\
            &= \int_0^\pi\int_0^{2\pi} R^{-1}
               (R^2\sin^2\theta\cos^2\varphi+2R^2\sin^2\theta\sin^2\varphi
                                            +3\cos^2\theta)\,R^2\sin\theta\,d\theta d\varphi \\
            &=   R^3\int_0^\pi \sin^3\theta\,d\theta\,\int_0^{2\pi}\cos^2\varphi\,d\varphi
               +2R^3\int_0^\pi \sin^3\theta\,d\theta\,\int_0^{2\pi}\sin^2\varphi\,d\varphi \\
            &\phantom{=\ R^3\int_0^\pi 
                            \sin^2\theta\,d\theta\,\int_0^{2\pi}\cos^2\varphi\,d\varphi}
               +3R^3\int_0^\pi \cos^2\theta\sin\theta\,d\theta\,\int_0^{2\pi}\,d\varphi \\
            &= R^3\left(\frac{4}{3}\cdot\pi + 2\cdot\frac{4}{3}\cdot\pi 
                                            + 3\cdot\frac{2}{3}\cdot 2\pi\right)
             = \underline{\underline{8\pi\,R^3}}.
\end{align*}

b) Gaussin (avaruus)kaavan mukaan
\begin{align*}
\int_{\partial V} \vec n\cdot\vec F\,dS = \int_V \nabla\cdot\vec F\,dxdydz 
                                       &= \int_V 6\ dxdydz \\ 
                                       &= 6\mu(V) = 6\cdot\frac{4}{3}\,\pi\,R^3 
                                        = \underline{\underline{8\pi\,R^3}}. \loppu
\end{align*}

\subsection{Yleistetty Gaussin lause}
\index{Gaussin lause (kaava)!a@yleistetty|vahv}

Samaan tapaan kuin edellä johdettaessa Gaussin tasokaava Greenin kaavoista voidaan
päätellä, että Gaussin tasokaava on voimassa myös, jos pistetulon paikalla on kaavan
kummallakin puolella ristitulo, eli kaava pysyy voimassa muunnoksin
\[
\nabla\cdot\vec F \ext \nabla\times\vec F, \quad \vec n\cdot\vec F \ext \vec n\times\vec F.
\]
Edelleen kaava pätee myös muunnoksin
\[
\nabla\cdot\vec F \ext \nabla f, \quad \vec n\cdot\vec F \ext f\vec n.
\]
Gaussin avaruuskaava (Lause \ref{Gaussin lause avaruudessa}) pätee samoin muunnoksin. 
Käyttämällä merkintöjä $d\vec n=\vec n\,ds$ (taso) $d\vec a=\vec n\,dS$ (avaruus) ja 
yleiskertomerkkiä $\ast$ voi tulokset yhdistää muotoon
\[
\boxed{ \begin{aligned}
\ykehys \int_A \nabla\ast\vec F\,dxdy 
                &= \int_{\partial A} d\vec n\ast\vec F \quad \text{(taso)}, \\
\quad \int_V \nabla\ast\vec F\,dxdydz 
                &= \int_{\partial V} d\vec a\ast\vec F \quad \text{(avaruus)}. \akehys\quad
\end{aligned} }
\]
Tapauksessa $\ast=$ 'tyhjä' (skalaarin ja vekorin kertolasku) on tässä vektorikenttä $\vec F$
tulkittava skalaarikentäksi: $\vec F\hookrightarrow f$. Tulokset, jotka siis pätevät Gaussin 
lauseen oletuksin, tunnetaan \kor{yleistettynä Gaussin lauseena}.
\begin{multicols}{2} \raggedcolumns
\begin{Exa}
Veteen upotetun kappaleen $V\subset\R^3$ reunapinnalla $\partial V$ vaikuttaa paine
\[
\vec f(x,y,z)=-\rho_0gz\,\vec n
\]
($\rho_0=$ veden tiheys). Mikä on kappaleeseen kohdistuva kokonaisvoima $\vec F\,$?
\begin{figure}[H]
\begin{center}
\import{kuvat/}{kuvapot-10.pstex_t}
\end{center}
\end{figure}
\end{Exa}
\end{multicols}
\ratk Yleistetyn Gaussin lauseen ($\ast=$ 'tyhjä') mukaan
\begin{align*}
\vec F &= -\rho_0 g \int_{\partial V} z\,\vec n\,dS = -\rho_0 g \int_V \nabla z\,dV \\
       &= -\rho_0 g\,\vec e_z \int_V dV = \underline{\underline{-m(V)g\,\vec e_z}},
\end{align*}
missä $m(V)=$ kappaleen syrjäyttämän vesimäärän massa. Tulos tunnetaan \newline
\pain{Arkhimedeen} \pain{lakina}. \loppu

\Harj
\begin{enumerate}

\item 
a) Laske $\int_A (\partial f/\partial y)\,dxdy$, kun $A$ on origokeskinen yksikkökiekko ja
$f(x,y)=2x-3y^2+(x^2+y^2-1)\sin(1+xy)$. \vspace{1mm}\newline
b) Laske $\int_{\partial A} \vec F \cdot d\vec n$, kun $A$ on ellipsin $\,S:\,x^2/a^2+y^2/b^2=1$
sisään jäävä alue ja $\vec F=(x+e^y)\vec i+(\sin x+2y)\vec j$.

\item 
Olkoon $V=\{(x,y,z)\in\R^3 \mid \abs{x}\le 1\,\ja\,\abs{y}\le 1\,\ja\,\abs{z}\le 1\}$ ja
$\vec F(x,y,z)=x\,\vec i + y^2\,\vec j-\,\vec k$. Laske integraalit 
$\int_{\partial V} d\vec a\cdot\vec F$ ja $\int_{\partial V} d\vec a\times\vec F$ \ 
a) suoraan pintaintegraaleina, \ b) avaruusintegraaleina käyttäen yleistettyä Gaussin lausetta.

\item 
Olkoon $V=\{(x,y,z)\in \R^3 \mid x^2+y^2+z^2\le 1\,\ja\,x\ge 0\,\ja\,y\ge 0\}$.
Laske $\int_{\partial V} [(x+y)\,\vec i -2xz\,\vec j+(y-z)\,\vec k]\times d \vec a$ \ 
a) suoraan, \ b) tilavuusintegraaliksi muuntamalla. 

\item
Laske $\int_{\partial V} \vec F \cdot d\vec a$ ja $\int_{\partial V} \vec F \times d\vec a$, 
kun $\vec F(x,y,z)=3xz^2\vec i-x\vec j-y\vec k\,$ ja
$V=\{(x,y,z)\in\R^3 \mid 0 \le x \le 1 \,\ja\, y \ge 0 \,\ja\, y^2+z^2 \le 1\}$.

\item
Olkoon $\vec F$ säännöllinen vektorikenttä ja $u$ säännöllinen skalaarikenttä perusalueessa
$V\subset\R^3$. Näytä, että $\int_{\partial V} \nabla\times\vec F \cdot d\vec a=0$ ja
$\int_{\partial V} \nabla u\times d\vec a=\vec 0$. 

\item 
Jos $A\subset \R^2$ ja $V\subset \R^3$ ovat perusalueita, niin mikä geometrinen merkitys on
seuraavilla integraaleilla? \vspace{1mm}\newline
$\D
\text{a)}\ \ \frac{1}{2} \int_{\partial A} \vec r \cdot d\vec n \qquad
\text{b)}\ \ \frac{1}{3} \int_{\partial V}\vec r \cdot d\vec a \qquad
\text{c)}\ \ \frac{1}{2\mu(V)} \int_{\partial V} (x^2+y^2+z^2)\,d\vec a$

\item \label{H-pot-1: osittaisintegrointi}
Olkoon $u$, $f$ ja $\vec F$ riittävän säännöllisiä skalaari- ja vektorikenttiä perus\-alueessa
$V\subset\R^3$. Näytä, että pätee
\[
\int_V \nabla u\ast\vec F\,dxdydz = \int_{\partial V} u\,d\vec a\ast\vec F -
\int_V u\nabla\ast\vec F\,dxdydz,
\]
missä $\vec F=f$, jos $\ast=$ 'tyhjä'. Mikä on kaavan $2$-ulotteinen vastine? Entä 
$1$-ulotteinen vastine, kun $V=[a,b]\,$?

\item (*) \index{zzb@\nim!Zeppeliini}
(Zeppeliini) Tyynellä säällä ilman paine $p(x,y,z)$ ja tiheys $\rho(x,y,z)$ toteuttavat
tasapainolain
\[
\frac{\partial p}{\partial z} = -g\rho,
\]
missä $z=$ korkeus maan pinnasta ja $g=$ maan vetovoiman kiihtyvyys. Jos $V\subset\R^3$ on
ilmassa leijuva ilmalaiva, niin päteekö Arkhimedeen laki?

\end{enumerate} %Gaussin lause
\section{Gaussin lauseen sovelluksia} \label{Gaussin lauseen sovelluksia}
\alku

Fysiikassa Gaussin avaruuskaava (Lause \ref{Gaussin lause avaruudessa}) esitetään usein
muodossa
\[
\int_V \nabla\cdot\vec F\,dV=\int_{\partial V} \vec F\cdot d\vec a,\quad V\subset\R^3,
\]
missä on merkitty $dV=dxdydz$. Jos $A \subset S$, missä $S$ on avaruuden pinta, niin
pintaintegraali
\begin{equation} \label{vuokaava}
\phi=\int_A \vec F\cdot d\vec a = \int_A \vec F\cdot\vec n\,dS \tag{$\star$}
\end{equation}
\index{vuo, vuontiheys}%
on vektorikentän $\vec F$ \kor{vuo} (engl. flux) $A$:n läpi. Ajatellen tätä yhteyttä
sanotaan itse vektorikenttää fysikaalisissa sovelluksissa usein \pain{vuontihe}y\pain{deksi}.
Seuraavassa kahdessa sovellusesimerkissä johdetaan Gaussin kaavan avulla fysikaalista ilmiötä
kuvaava \pain{säil}y\pain{mislaki} osittaisdifferentiaaliyhtälön muodossa. Ensimmäinen esimerkki
on virtausmekaniikasta ja toinen lämpöopista.

\subsection{Sovellusesimerkki: Massan säilymislaki virtauksessa}
\index{massan säilymislaki|vahv}
\index{szy@säilymislaki|vahv}
\index{zza@\sov!Massan säilymislaki virtauksessa|vahv}

Tarkastellaan virtaavaa nestettä tai kaasua, jonka nopeus hetkellä $t$ pisteessä $(x,y,z)$ on
$\vec v = \vec v(t,x,y,z)$ (vektorikenttä, yksikkö m/s) ja tiheys on $\rho=\rho(t,x,y,z)$
(skalaarikenttä, yksikkö kg/m$^3$). Olkoon $A \subset S$, missä $S$ on avaruuden säännöllinen
(tai ainakin paloittain säännöllinen) pinta. Tarkastellaan $A$:n pientä, lähes tasomaista palaa
$\Delta A$, jonka pinta-ala $=\Delta S$. Jos oletetaan, että $\rho$ ja $\vec v$ ovat
$\Delta A$:n ympäristössä lähes vakioita (jatkuvuusoletus!), niin voidaan päätellä, että
(lyhyellä) aikavälillä $[t,t+\Delta t]$ pinnanpalan $\Delta A$ läpi menevät ne hiukkaset, jotka
ovat hetkellä $t$ joukossa
\[
\Delta V 
= \{\,(x,y,z)\vastaa\vec r \mid \vec r=\vec r_0-\tau\vec v,\ r_0\in A,\ \tau\in[0,\Delta t]\,\}.
\]
\begin{figure}[H]
\setlength{\unitlength}{1cm}
\begin{center}
\begin{picture}(11,2)(0,1)
\thicklines
\put(2,1){\line(1,0){4}} \put(1,2){\line(1,0){4}} \put(2,1){\line(-1,1){1}} 
\put(6,1){\line(-1,1){1}}
\put(1,2){\line(2,1){1}} \put(6,1){\line(2,1){1}} \put(5,2){\line(2,1){1}} 
\put(7,1.5){\line(-1,1){1}}
\put(2,2.5){\line(1,0){4}} 
\thinlines
\put(6,1.7){\vector(2,1){1.8}} \put(6,1.7){\vector(1,0){3}}
\put(8.05,2.6){$\vec n$} \put(9.2,1.6){$\vec v$}
\put(3,1.5){\line(1,-1){1}} \put(6.4,1.4){\line(1,-1){0.6}}
\put(4.2,0.4){$\Delta V$} \put(7.2,0.7){$\Delta A$}
\end{picture}
\end{center}
\end{figure}
Olkoon $\vec n$ virtaussuuntaan osoittava $S$:n yksikkönormaalivektori, ts.\
$\vec n\cdot\vec v \ge 0$. Jos oletetaan $\vec v$ vakioksi ja samoin $\vec n$ vakioksi
$\Delta A$:ssa (eli pinta tasoksi), niin $\Delta V$ on näillä oletuksilla suuntaissärmiö, jonka
poikkipinta-ala virtausta vastaan kohtisuorassa tasossa on
$\Delta S\,\vec v\cdot\vec n/|\vec v\,|$. Jos edellen myös $\rho$
oletetaan vakioksi, niin $\Delta V$:ssä olevan nesteen/kaasun massa on tehdyin oletuksin
\begin{align*}
\Delta m &=\rho\mu(\Delta V)=\rho\,[\Delta S\,\vec v\cdot\vec n/|\vec v\,|]\,|\vec v\,|\Delta t
                           = \rho\vec v\cdot\vec n\,\Delta S\Delta t \\
         &\impl\quad \frac{\Delta m}{\Delta S\Delta t} = \rho\vec v\cdot\vec n.
\end{align*}
Voidaan olettaa, että tehdyt likmääräistykset ovat voimassa lähes kaikissa $A$:n pisteissä,
lukuun ottamatta mahdollista (pinta-alamitan suhteen) nollamittaista osajoukkoa. Muissa kuin
näissä poikkeuspisteissä voidaan mainittujen oletusten myös olettaa toteutuvan tarkasti
raja-arvoina, kun $\Delta t \kohti 0$ ja $\Delta A$ kutistuu pisteeksi. Näin olettaen voidaan
\pain{massavuo} pinnan $A$ läpi (yksikkö kg/s) laskea integraalikaavalla \eqref{vuokaava},
missä \pain{massavuon} \pain{tihe}y\pain{s} (yksikkö kg/m$^2$/s) määritellään
\[
\vec F = \rho\vec v.
\]
Oletetaan nyt, että $S$ on suljettu pinta, joka sulkee sisäänsä perusalueen $V\subset\R^3$ 
($S=\partial V$). Tällöin nesteen/kaasun kokonaismassa $V$:ssä hetkellä $t$ on
\[
m(t) = \int_V \rho(t,x,y,z)\,dV.
\]
Oletetaan, että $m(t)$ voi muuttua vain $V$:n reunapinnan läpi tapahtuvan virtauksen vuoksi
(eli muilla tavoilla massaa ei 'häviä' tai 'synny'). Tällöin on voimassa tasapainolaki
\[
m'(t) = -\int_{\partial V} \rho\vec v\cdot d\vec a,
\]
missä $d\vec a=\vec n dS$ osoittaa $V$:stä poispäin, eli $\vec n=V$:n ulkonormaali. Kun tässä
oikealla käyttetään Gaussin kaavaa ja vasemmalla kirjoitetaan (vrt.\ Luku 
\ref{osittaisderivaatat})
\[
m'(t) = \frac{d}{dt}\int_V \rho(t,x,y,z)\,dV 
      = \int_V \frac{\partial \rho}{\partial t}\,(t,x,y,z)\,dV,
\]
niin tasapainolaki saa muodon
\begin{equation} \label{säilymislaki a}
\int_V \bigl[\rho_t+\nabla\cdot(\rho\vec v)\bigr]\,dV=0. \tag{a}
\end{equation}
Olkoon nyt $t\in\R$ kiinteä ajanhetki, $(x,y,z)\in\R^3$ kiinteä piste, ja oletetaan, että $\rho$
ja $\vec v$ ovat jatkuvasti derivoituvia (osittaisderivaatat jatkuvia) pisteen
$(t,x,y,z)\in\R^4$ ympäristössä. Tällöin jos tasapainolaissa \eqref{säilymislaki a} valitaan
$V=B_\eps=\eps$-säteinen pallo (kuula), jonka keskipiste on $(x,y,z)$, niin
\begin{align*}
0\ &=\ \int_{B_\eps}\bigl[\rho_t+\nabla\cdot(\rho\vec v)\bigr]\,dV \\
   &=\ \bigl[\rho_t+\nabla\cdot(\rho\vec v)\bigr](t,x,y,z)\,\mu(B_\eps)
         + o(1)\mu(B_\eps), \quad \text{kun}\ \eps \kohti 0.
\end{align*}
Tämän mukaan on tarkasteltavassa pisteessä $(t,x,y,z)$ oltava voimassa
\begin{equation} \label{säilymislaki b}
\boxed{\kehys\quad \rho_t + \nabla\cdot(\rho\vec v)=0. \quad} \tag{b}
\end{equation}
Tämä tunnetaan virtausmekaniikan (differentiaalisena) \pain{massan}
\pain{säil}y\pain{mislakina}. --- Huomattakoon, että koska tätä johdettaessa tehdyt
jatkuvuusoletukset eivät ole fysiikan sanelemia, niin säilymislain integraalimuoto
\eqref{säilymislaki a} (joka pätee heikommin säännöllisyysehdoin jokaiselle perusalueelle $V$)
on luonnonlakina alkuperäisempi kuin differentiaalinen muoto \eqref{säilymislaki b}.
Vrt.\ vastaava asetelma Esimerkissä \ref{analyysin peruslause}:\,\ref{liikelaki}.


\subsection{Sovellusesimerkki: Lämmön johtuminen}
\index{szy@säilymislaki|vahv}
\index{zza@\sov!Lzy@Lämmön johtuminen|vahv}

Tarkastellaan lämmön johtumisen ongelmaa, kun avaruuden $\R^3$ täyttää fysikaalisilta 
ominaisuuksiltaan tunnettu materiaali. Ongelmaan liittyvät perussuureet ovat
\pain{läm}p\pain{ötila} $u=u(t,x,y,z)$, \pain{läm}p\pain{övuon} \pain{tihe}y\pain{s} 
$\vec J=\vec J(t,x,y,z)$ (yksikkö W/m$^2$), ja \pain{läm}p\pain{ölähteen} \pain{tihe}y\pain{s} 
$\rho(t,x,y,z)$ (yksikkö W/m$^3$). Materiaalikertoimina oletetaan vielä tunnetuiksi materiaalin 
\pain{ominaisläm}p\pain{ö} $c=c(x,y,z,u)$ ja \pain{lämmön}j\pain{ohtavuus} 
$\lambda=\lambda(x,y,z,u)$. (Tässä oletetaan, että materiaalikertoimet voivat riippua 
paikkamuuttujien lisäksi lämpötilasta.)

Jos $V\subset\R^3$, niin $V$:n sisältämän \pain{läm}p\pain{öener}g\pain{ian} kokonaismäärä
hetkellä $t$ on (ominaislämmön määritelmä)
\[
E(t)=\int_V cu\,dV.
\]
Jos oletetaan, että $V$:n energiataseeseen vaikuttavat vain lämpölähde $V$:ssä ja lämmön
johtuminen $V$:n reunapinnan läpi, niin energian tasapainoyhtälö $V$:ssä on
\[
E'(t)= \int_V \rho\,dV - \int_{\partial V} \vec J\cdot d\vec a.
\]
Kun tässä kirjoitetaan
\[
E'(t)=\frac{d}{dt}\int_V cu\,dV = \int_V cu_t\,dV, \qquad
      \int_{\partial V} \vec J\cdot d\vec a = \int_V \nabla\cdot\vec J\,dV,
\]
niin tasapainoyhtälö saa muodon
\[
\int_V (cu_t+\nabla\cdot\vec J-\rho)\,dV = 0.
\]
Samanlaisella päättelyllä (ja oletuksilla) kuin edellisessä esimerkissä seuraa tästä 
differentiaalinen \pain{ener}g\pain{ian} \pain{säil}y\pain{mislaki}
\[
cu_t+\nabla\cdot\vec J = \rho.
\]
Jos vielä oletetaan lämmönjohtumisen \pain{Fourier'n} \pain{laki}
\index{Fourierb@Fourier'n laki}%
\[
\vec J = -\lambda \nabla u,
\]
\index{lzy@lämmönjohtumisyhtälö}%
niin energian säilymislaki saa \pain{lämmön}j\pain{ohtumis}y\pain{htälönä} tunnetun muodon
\[
\boxed{\kehys\quad cu_t-\nabla\cdot(\lambda\nabla u)=\rho. \quad}
\]
Jos $\rho=0$ ja $c$ ja $\lambda$ ovat vakioita (homogeeninen materiaali, $c$:llä ja
$\lambda$:lla ei lämpötilariippuvuutta), niin lämmönjohtumisyhtälö yksinkertaistuu muotoon
\[
u_t=k\Delta u, \quad k=\lambda/c.
\]
 
\subsection{*Vektorikenttien epäjatkuvuudet}
\index{vektorikenttä!d@epäjatkuva|vahv}
\index{epzy@epäjatkuvuus (vektorikentän)|vahv}

Olkoon vektorikenttä $\vec F$ määritelty $\R^3$:ssa ja jatkuvasti derivoituva. 
Tällöin kentän lähde $\rho=\nabla\cdot\vec F$ on $\R^3$:ssa jatkuva, ja Gaussin lauseen mukaan
pätee
\begin{equation} \label{kenttä ja lähde}
\int_V \rho\,dV = \int_{\partial V}\vec F\cdot d\vec a \quad (V\subset\R^3\,\ \text{perusalue}).
\end{equation}
Toisaalta, jos mainittujen säännöllisyysoletusten lisäksi oletetaan ainoastaan
\eqref{kenttä ja lähde}, niin aiemman päättelyn mukaisesti seuraa, että on oltava
$\nabla\cdot\vec F=\rho$. Siis jos $\vec F$ on $\R^3$:ssa jatkuvasti derivoituva ja $\rho$
jatkuva, niin pätee
\[
\text{Ehto \eqref{kenttä ja lähde} voimassa} \qekv \nabla\cdot\vec F = \rho\ \ \R^3:\text{ssa}.
\]

Entä jos $\vec F$ ei ole jatkuvasti derivoituva, tai $\rho$ ei ole jatkuva, mutta $\vec F$:n ja
$\rho$:n välillä on ehdon \eqref{kenttä ja lähde} mukainen yhteys? --- Tällöin on luontevaa
sopia, että ehto \eqref{kenttä ja lähde} \pain{määrittelee} $\rho$:n kentän $\vec F$ lähteeksi.
Fysikaalisissa sovelluksissa ajatellaan tällöin, että ehto \eqref{kenttä ja lähde} itse asiassa
ilmaisee luonnonlain sen alkuperäisessä muodossa (vrt.\ massan säilymislain johto edellä).

Vektorikentän ja lähteen välisen riippuvuuden esittäminen integraalimuotoisena 'Gaussin lakina'
\eqref{kenttä ja lähde} on erityisen hyödyllistä silloin, kun halutaan johtaa fysikaalisten
vektorikenttien j\pain{atkuvuusehto}j\pain{a} \pain{materiaalira}j\pain{a}p\pain{innoilla}.
Tarkastellaan esimerkkinä tällaisten ehtojen asettamista tasolla (tasomaisella 
materiaalirajapinnalla). Olkoon $T$ avaruustaso, joka jakaa $\R^3$:n kahteen avoimeen osaan 
$V_1$ ja $V_2$ (eli $\R^3$ jakautuu pistevieraisiin osiin $V_1$, $V_2$ ja $T$), ja olkoon
$\vec F$ paloittain jatkuvasti derivoituva vektorikenttä muotoa
\[
\vec F(x,y,z) = \begin{cases} \,\vec F_1(x,y,z), \quad \text{kun}\ (x,y,z) \in V_1, \\
                              \,\vec F_2(x,y,z), \quad \text{kun}\ (x,y,z) \in V_2,
                \end{cases}
\]
missä $\vec F_1$ ja $\vec F_2$ ovat molemmat koko $\R^3$:ssa määriteltyjä ja jatkuvasti
derivoituvia. Olkoon vastaavasti $\,\rho\,$ paloittain jatkuva funktio muotoa
\[
\rho(x,y,z) = \begin{cases} \,\rho_1(x,y,z), \quad \text{kun}\ (x,y,z) \in V_1, \\
                            \,\rho_2(x,y,z), \quad \text{kun}\ (x,y,z) \in V_2,
                \end{cases}
\]
missä $\,\rho_1$ ja $\,\rho_2$ ovat molemmat koko $\R^3$:ssa määriteltyjä ja jatkuvia.
\begin{figure}[H]
\setlength{\unitlength}{1cm}
\begin{center}
\begin{picture}(11,4)(0,0.5)
\thicklines
\put(2.2,4){$T$} \put(3,1){$V_1$} \put(6,1){$V_2$} \put(1,2){$\vec F_1,\ \ \rho_1$}
\put(4.5,3.5){$\vec F_2,\ \ \rho_2$}
\put(2,4){\line(1,-1){4}} \put(4.5,1.5){\vector(1,1){1}} \put(5.5,2){$\vec n$} 
\end{picture}
\end{center}
\end{figure}

Em.\ säännöllisyysoletuksien lisäksi oletetaan vielä, että kenttää $\,\vec F\,$ ja funktiota
$\,\rho\,$ sitoo ehto \eqref{kenttä ja lähde}, ts.\ $\,\rho\,$ on kentän $\,\vec F\,$ lähde.
Tällöin jos ko.\ ehdossa valitaan $V \subset V_1$ tai $V \subset V_2$ niin seuraa, että on 
oltava $\nabla\cdot\vec F_i=\rho_i$ $\,V_i$:ssä, $i=1,2$, eli pätee
\begin{equation} \label{lähdey-1}
 \nabla\cdot\vec F(x,y,z) = \rho(x,y,z), \quad (x,y,z) \in V_1 \cup V_2.
\end{equation}
Tutkitaan seuraavaksi, mitä ehdosta \eqref{kenttä ja lähde} seuraa, kun joukko $V$ valitaan
siten, että $V \cap\,T \neq \emptyset$. Tätä silmällä pitäen tarkastellaan pistettä 
$(x,y,z) \in T$ ja määritellään tätä pistettä ympäröivä 'pillerirasia' $B_{h,d}$ ehdoilla:
(i) $B_{h,d}$ on suorakulmainen särmiö, jonka keskipiste $\,=(x,y,z)$ ja kaksi sivutahkoa ovat
tason $T$ suuntaiset. (ii) $B_{h,d}$:n tasoa $T$ vastaan kohtisuoran särmän pituus $\,=d$. 
(iii) Suorakulmio $T_h = B_{h,d} \cap T$ on $d$:stä riippumaton ja $T_h$:n suurimman sivun
pituus $=h$. 
\begin{figure}[H]
\setlength{\unitlength}{1cm}
\begin{center}
\begin{picture}(11,4)(0,1)
\thicklines
\put(6,1){\line(-4,3){5}} \put(6,1){\line(4,1){4}}
\put(4.5,1.625){\line(0,1){1}} \put(4.5,1.625){\line(-4,3){1.5}} \put(3,2.75){\line(0,1){1}}
\put(4.5,2.625){\line(4,1){2}} \put(4.5,2.625){\line(-4,3){1.5}} \put(3,3.75){\line(4,1){2}}
\put(6.5,3.125){\line(-4,3){1.5}}
\put(4.5,2.125){\line(4,1){2}} \put(6.5,2.625){\line(0,1){0.5}} \put(4.5,1.625){\line(4,1){0.5}}
\put(4.75,3.2){\vector(0,1){1.8}} \put(4.9,5){$\vec n$} \put(9,2){$T$}
\end{picture}
\end{center}
\end{figure}
Valitaan ehdossa \eqref{kenttä ja lähde} $V=B_{h,d}$, missä $T_h = B_{h,d} \cap T$ on kiinteä ja
$d \kohti 0$. Tällöin seuraa
\[
\lim_{d \kohti 0} \int_{\partial B_{h,d}} \vec F\cdot d\vec a
        = \int_{T_h} \vec n\cdot(\vec F_2-\vec F_1)\,d\mu 
        = \lim_{d \kohti 0} \int_{B_{h,d}} \rho\,dV = 0,
\]
missä $\mu$ on tason $T$ pinta-alamitta ja $\vec n$ on $T$:n yksikkönormaalivektori, joka
osoittaa osajoukon $V_2$ suuntaan (ks.\ kuviot). Kun tässä tuloksessa annetaan edelleen 
$h$:n lähestyä $0$:aa ja huomioidaan kenttien $\vec F_1$ ja $\vec F_2$ jatkuvuus pisteessä
$(x,y,z)$, niin seuraa, että on oltava $\vec n\cdot(\vec F_1-\vec F_2)(x,y,z)=0$. Tämä on
voimassa jokaisessa $T$:n pisteessä, joten on päätelty, että tasolla $T$ on voimassa
jatkuvuusehto 
\begin{equation} \label{lähdey-2}
\vec n\cdot\vec F_1(x,y,z) = \vec n\cdot\vec F_2(x,y,z), \quad (x,y,z) \in T.
\end{equation}
Sekä tämä että \eqref{lähdey-1} ovat siis ehdon \eqref{kenttä ja lähde} seurauksia 
(tehtyjen säännöllisyysoletusten puitteissa).

Toisaalta jos lähdetään samoista säännöllisyysoletuksista ja oletetaan lisäksi \eqref{lähdey-1}
ja \eqref{lähdey-2}, niin näistä yhdessä seuraa \eqref{kenttä ja lähde}. Nimittäin jos 
$V$ on perusalue ja $A_i = V \cap V_i \neq \emptyset$, $i=1,2$, niin oletuksen \eqref{lähdey-1}
ja Gaussin lauseen perusteella voidaan päätellä
\begin{align*}
\text{\eqref{lähdey-1}} &\qimpl \sum_{i=1}^2 \int_{A_i} \nabla\cdot\vec F_i\,dV 
                                     = \sum_{i=1}^2 \int_{A_i} \rho_i\,dV = \int_V \rho\,dV \\
                        &\qimpl \sum_{i=1}^2 \int_{\partial A_i} \vec F_i\cdot d\vec a 
                                     = \int_V \rho\,dV.
\end{align*}
Koska tässä on
\[ 
\partial A_1 \cup \partial A_2 = \partial V \cup (\partial A_1 \cap \partial A_2), \quad
\partial A_1 \cap \partial A_2 \subset T, 
\]
niin oletuksen \eqref{lähdey-2} perusteella päätellään edelleen
\[
\sum_{i=1}^2 \int_{\partial A_i} \vec F_i\cdot d\vec a 
     = \int_{\partial V} \vec F\cdot d\vec a 
           + \int_{\partial A_1 \cap \partial A_2} \vec n\cdot(\vec F_1-\vec F_2)\,d\mu
     = \int_{\partial V} \vec F\cdot d\vec a.
\]
Siis ehto \eqref{kenttä ja lähde} on voimassa. Näin on päätelty, että tehtyjen
säännöllisyysoletusten puitteissa pätee
\[ \boxed{
\quad \text{Ehto \eqref{kenttä ja lähde}} \qekv 
      \begin{cases} 
      \ \nabla\cdot\vec F = \rho \quad          &\text{joukossa}\ V_1 \cup V_2, \quad\ykehys \\
      \ \vec n\cdot(\vec F_1-\vec F_2)= 0 \quad &\text{rajapinnalla}\ T. \quad\akehys
      \end{cases} }
\]
Johtopäätös on sama myös, jos rajapinta $\partial V_1 \cap \partial V_2$ on osa kaarevaa, 
riittävän säännöllistä pintaa $S$. Koskien fysikaalisen vektorikentän jatkuvuusehtoja
materiaalirajapinnalla on siis päätelty, että jos vektorikentällä on materiaalirajapinnan
läheisyydessä paloittain jatkuva lähde, niin \pain{kentän} \pain{normaalikom}p\pain{onentti}
\pain{on} j\pain{atkuva} materiaalirajapinnalla. Sen sijaan kentän tangentiaalikomponentin 
\pain{ei} tarvitse olla jatkuva.

Samaan tapaan kuin ehdossa \eqref{kenttä ja lähde} voidaan myös vektorikentän ja sen 
pyörrekentän välinen yhteys esittää integraalimuotoisena määritelmänä. Jos $\vec F$ on 
jatkuvasti derivoituva $\R^3$:ssa, niin $\vec F$:n pyörrekenttä on 
$\vec\omega = \nabla\times\vec F$ (vrt.\ Luku \ref{divergenssi ja roottori}). Tällöin pätee
yleistetyn Gaussin lauseen perusteella
\begin{equation} \label{kenttä ja pyörre}
\int_V \vec\omega\,dV = \int_{\partial V} d\vec a\times\vec F \quad
                        (V\subset\R^3\,\ \text{perusalue}).
\end{equation}
Jos $\vec F$ ei ole koko $\R^3$:ssa jatkuvasti derivoituva, tai $\vec\omega$ ei ole jatkuva,
niin katsotaan ehto \eqref{kenttä ja pyörre} pyörrekentän $\vec\omega$ määritelmäksi. Jos nyt
$\vec F$  toteuttaa samat ehdot kuin edellä ja oletetaan, että määritelmän
\eqref{kenttä ja pyörre} mukainen pyörrekenttä on muotoa 
\[
\vec\omega(x,y,z) = \begin{cases} \,\vec\omega_1(x,y,z), \quad \text{kun}\ (x,y,z) \in V_1, \\
                                  \,\vec\omega_2(x,y,z), \quad \text{kun}\ (x,y,z) \in V_2,
                    \end{cases}
\]
missä $\,\vec\omega_1$ ja $\,\vec\omega_2$ ovat molemmat koko $\R^3$:ssa määriteltyjä ja 
jatkuvia, niin samalla tavoin kuin edellä päätellään, että pätee
\[ \boxed{
\quad \text{Ehto \eqref{kenttä ja pyörre}} \qekv 
\begin{cases} 
\ \nabla\times\vec F = \vec\omega \quad         &\text{joukossa}\ V_1 \cup V_2, \quad\ykehys \\
\ \vec n\times(\vec F_1-\vec F_2)= \vec 0 \quad &\text{rajapinnalla}\ T. \quad\akehys
\end{cases} }
\]
Tässä tapauksessa \pain{kentän} \pain{tan}g\pain{entiaalikom}p\pain{onentti} \pain{on} 
j\pain{atkuva} materiaalirajapinnalla, sillä tangentiaalikomponentti on (vrt.\ 
Luku \ref{ristitulo})
\[ 
\vec F_t = \vec F - (\vec n\cdot\vec F)\,\vec n = - \vec n\times(\vec n\times\vec F). 
\]
Kentän normaalikomponentin ei tarvitse olla jatkuva.

\begin{Exa} Staattinen sähkökenttä $\vec E$ ja sähkövuon tiheys $\vec D$ toteuttavat Maxwellin
yhtälöt (vrt.\ Luku \ref{divergenssi ja roottori})
\[ 
\nabla\times\vec E = \vec 0, \quad \nabla\cdot\vec D = \rho, 
\]
missä $\,\rho\,$ on varaustiheys. Oletetaan, että edellä $V_1$ ja $V_2$ edustavat kahta eri 
materiaalia, joissa vallitsevat materiaalilait ovat
\[ 
\vec D = \epsilon_i\vec E \quad V_i\,\text{:ssä}, \quad i=1,2, 
\]
missä $\epsilon_i$ (= materiaalin $i$ sähköinen permittiivisyys) oletetaan $V_i$:ssa vakioksi.
Jos oletetaan $\rho$ paloittain jatkuvaksi kuten edellä, niin jatkuvuusehdot 
materiaalirajapinnalla ovat
\[
\begin{cases} 
\ \vec n\times(\vec E_1-\vec E_2) = \vec 0, \\ \ \vec n\cdot(\vec D_1-\vec D_2) = 0 
\end{cases}
\qekv \begin{cases} 
      \ \vec n\times\vec E_1 = \vec n\times\vec E_2, \\ 
      \ \epsilon_1\,\vec n\cdot\vec E_1 = \epsilon_2\,\vec n\cdot\vec E_2.
      \end{cases}
\]
Sähkökentän tangentiaalikomponentti on siis materiaalirajapinnalla jatkuva. Normaalikomponentti
sen sijaan on epäjatkuva, ellei ole joko $\epsilon_1 = \epsilon_2$ tai $\vec n\cdot\vec E_i=0$.
\loppu
\end{Exa}
\jatko \begin{Exa} (jatko) Esimerkissä olkoon 
\[ 
V_1 = \{(x,y,z) \in \R^3 \mid x+y+z<0\}, \quad V_2 = \{(x,y,z) \in \R^3 \mid x+y+z>0\}, 
\]
ja $\,\epsilon_1/\epsilon_2=2$. Laske sähkökenttä $\vec E_2(0,0,0)$, kun tiedetään, että 
$\vec E_1(0,0,0)$ $=$ $E(\vec i+2\vec j-\vec k)$.
\end{Exa}
\ratk Materiaalirajapinta on taso $T:\ x+y+z=0$. Tämän materiaalia $2$ kohti osoittava 
yksikkönormaalivektori on $\vec n = (\vec i+\vec j+\vec k)/\sqrt{3}$, joten
\[
\vec n\cdot\vec E_1(0,0,0) = \frac{2E}{\sqrt{3}}, \quad \vec n\times\vec E_1(0,0,0) 
                           = \frac{E}{\sqrt{3}}(-3\vec i+2\vec j+\vec k).
\]
Jatkuvuusehtojen perusteella on
\[ 
\vec n\cdot\vec E_2(0,0,0) = \frac{\epsilon_1}{\epsilon_2}\,\vec n\cdot\vec E_1(0,0,0), \quad
\vec n\times\vec E_2(0,0,0)=\vec n\times\vec E_1(0,0,0), 
\]
joten
\begin{align*}
\vec E_2(0,0,0) &=[\vec n\cdot\vec E_2(0,0,0)]\,\vec n
                            -\vec n\times[\vec n\times\vec E_2(0,0,0)] \\[3mm]
                &= 2\,[\vec n\cdot\vec E_1(0,0,0)]\,\vec n 
                            - \vec n\times[\vec n\times\vec E_1(0,0,0)] \\[2mm]
                &= \frac{4E}{3}(\vec i+\vec j+\vec k) + \frac{E}{3}\,(\vec i+4\vec j-5\vec k) \\
                &= \underline{\underline{\frac{E}{3}\,(5\vec i + 8\vec j - \vec k\,)}}. \loppu
\end{align*}

\pagebreak
\Harj
\begin{enumerate}

\item
Laske vektorikentän $\vec F$ vuo origokeskisen, $R$-säteisen pallopinnan
läpi: \vspace{1mm}\newline
a) \ $\vec F=x^2y^4z\vec k \qquad $
b) \ $\vec F=x\vec i-2y\vec j+4z\vec k \qquad$
c) \ $\vec F=ye^z\vec i+x^2e^z\vec j+xy\vec k$ \newline
d) \ $\vec F=(x^2+y^2)\vec i+(y^2-z^2)\vec j+z\vec k \qquad$
e) \ $\vec F=x^2\vec i+3yz^2\vec j+(3y^2z+x^2)\vec k$

\item
Laske vektorikentän $\vec F=x^2\vec i+y^2\vec j+z^2\vec k$ vuo annetun alueen $V\subset\R^3$
reunapinnan $\partial V$ läpi: \vspace{1mm}\newline
a) \ $V:\ (x-2)^2+y^2+(z-3)^2 \le 9 \quad\,\ $
b) \ $V:\ x^2+y^2+4(z-1)^2 \le 4$ \newline
c) \ $V:\,\ x,y,z \ge 0\,\ja\,x+y+z \le 3 \qquad$
d) \ $V:\ x^2+y^2 \le 2y\,\ja\,0 \le z \le 4$

\item
Tetraedrin muotoista aluetta $V\subset\R^3$ rajoittavat tasot $z=0$, $x=2y$, $x=-y$ ja
$y+z=a$ ($a>0$). Laske vektorikentän
\[
\vec F=(3xyz+e^{yz})\vec i+(x^2+y^2z)\vec j+(y^2-2yz^2)\vec k
\]
vuo $V$:n reunapinnan $\partial V$ läpi sisältä ulospäin.

\item
Olkoon $S:\,x^2+y^2+z^2=1$, $B \subset S$ ja
\[
V= \{(x,y,z)\in\R^3 \mid (x,y,z)=t(u,v,w),\ t\in[0,R]\,\ja\,(u,v,w) \in B\}.
\]
Laske $V$:n tilavuus $\mu(V)$ \ a) pallokoordinaattien avulla, \ b) soveltamalla Gaussin 
lausetta vektorikenttään $\vec F=x\vec i+y\vec j+z\vec k$.

\item
Pyörivässä virtauskentässä massavuon tiheys pisteessä $\,(x,y,z) \vastaa \vec r\,$ on 
$\vec F=Q\,\vec r\times(2\vec i-\vec j+3\vec k)$, missä $Q=24$ kg/m$^3$/s ja $\vec r$:n
yksikkö = m. Olkoon $V$ kuutio, jonka yksi kärki on origossa, origosta lähtevät särmät ovat 
positiivisilla koordinaattiakseleilla ja sivun pituus on $a=2$ m. Laske massavuo (yksikkö kg/s)
$V$:n kunkin sivutahkon läpi laskettuna positiivisena kuution sisältä ulospäin.

\item 
Vektorikenttä $\vec F$ ja sen lähde $\rho$ ovat a) pallokoordinaatistossa, 
b) lieriökoordinaatistossa muotoa $\vec F= F(r)\vec e_r$, $\rho = \rho(r)$. Laske $F(r)$ 
lähteen $\rho(r)$ avulla käyttämällä Gaussin kaavaa sopivasti valitussa alueessa $V$.

\item
Olkoot $u$ ja $v$ säännöllisiä funktioita perusalueessa $A\subset\R^2$ tai $V\subset\R^3$.
Soveltamalla Gaussin lausetta vektorikenttään $\vec F=u\nabla v-v\nabla u$ näytä, että
\begin{align*}
&\int_A \left(u\Delta v-v\Delta u\right)\,dxdy
                 = \int_{\partial V} \left(u\pder{v}{n}-v\pder{u}{n}\right)ds, \\
&\int_V \left(u\Delta v-v\Delta u\right)\,dxdydz 
                 = \int_{\partial V} \left(u\pder{v}{n}-v\pder{u}{n}\right)dS,
\end{align*}
missä $\partial/\partial n$ tarkoittaa suunnattua derivaattaa $\partial A$:n tai $\partial V$:n
normaalin suuntaan.

\item \index{zzb@\nim!Autojen säilyminen}
(Autojen säilyminen) Yksiulotteisessa liikennevirrassa on $\rho(t,x)$ autotiheys tiellä
(yksikkö autoa/km) ja $v(t,x)$ nopeus (yksikkö km/h). \ a) Määrittele autovuo tiellä ja johda
autojen säilymislaki. \ b) Pisteessä $x=0$ tie 1 kapenee tieksi 2. Tiellä 1 ($x<0$)
liikennevirta on tasainen: $\rho=26$ autoa/km ja $v=120$ km/h. Myös tiellä 2 liikennevirta on
tasainen ($\rho$ ja $v$ vakioita) siten, että tien maksimikapasiteetti $3000$ autoa/h ei ylity.
Voiko tällaisessa liikennetilanteessa piste $x=0$ olla ruuhkaton? Jos ei, niin monellako
autolla vähintään ruuhka kasvaa tunnissa?

\item 
Lämmönjohtumisen perusyhtälöt ovat
\[
\nabla\cdot\vec J = \rho, \quad \vec J = - \lambda \nabla u,
\]
missä $\vec J$ on lämpövirran tiheys, $\rho$ lämpölähteen tiheys, $u$ lämpötila ja 
$\lambda$ materiaalin lämmönjohtavuus. Olkoon $\lambda = \lambda_i =$ vakio $V_i$:ssa, 
$i = 1,2$, missä $\lambda_2 = 3 \lambda_1$ ja
\[
V_{1, 2} = \{\,(x, y, z) \in \R^3 \mid x+2y+2z \lessgtr 0 \,\}.
\]
Määritä lämpövirtavektorin $\vec J$ raja-arvo $\vec J_2 (0, 0, 0)$ materiaalin 2 ($V_2$) 
puolelta origoa lähestyttäessä, kun tiedetään, että ko. raja-arvo materiaalin 1 puolelta on 
$\vec J_1(0, 0, 0) = J(-\vec i+\vec j+3\vec k)$ ($J$ vakio). Lähde $\rho$ on kummassakin 
materiaalissa vakio ja lämpötila $u$ on materiaalirajapinnalla jatkuva.

\item (*)
Olkoon $A\subset\R^2$ tai $V\subset\R^3$ perusalue. Näytä, että jos $u_1$ ja $u_2$ ovat 
reuna-arvotehtävän
\[
\begin{cases} 
\,\Delta u=\rho &\text{$A$:ssa ($V$:ssä)}, \\ 
\,u=0 &\text{reunalla $\partial A$ (reunalla $\partial V$)}
\end{cases}
\] 
ratkaisuja ja lisäksi riittävän säännöllisiä, niin funktiolle $u=u_1-u_2$ pätee
\[
\int_A \abs{\nabla u}^2\,dxdy=0 \quad \left(\int_V \abs{\nabla u}^2\,dxdydz=0\right).
\]
Päättele, että reuna-arvotehtävän ratkaisu --- sikäli kuin olemassa ja riittävän säännöllinen
--- on yksikäsitteinen.

\end{enumerate} %Gaussin lauseen sovelluksia
\section{Stokesin lause} \label{stokesin lause}
\alku
\index{Stokesin lause|vahv}

Tarkastellaan aluksi tason perusaluetta $A$ (ks.\ Luku \ref{gaussin lause}) ja $A$:ssa
määriteltyä vektorikenttää $\vec F(x,y)=F_1(x,y)\vec i + F_2(x,y)\vec j$. Kun tulkitaan $\vec F$
avaruuden vektorikentäksi, niin $\vec F$:n roottori on
(vrt.\ Luku \ref{divergenssi ja roottori})
\[
\nabla\times\vec F
    =\left(\frac{\partial F_2}{\partial x}- \frac{\partial F_1}{\partial y}\right)\vec k
    =(\text{rot}\,\vec F\,)\,\vec k,
\]
joten Greenin tasokaavojen (ks.\ Luku \ref{gaussin lause}) perusteella
\begin{align*}
\int_A \text{rot}\,\vec F\,dxdy &= \int_{\partial A} (-n_yF_1+n_xF_2)\,ds \\
&= \int_{\partial A} \vec t\cdot\vec F\,ds,
\end{align*}
missä $\vec t$ on reunan tangenttivektori suunnistettuna kuvan mukaisesti, eli $\vec t$ osoittaa
oikealle reunan ulkonormaalin suunnalta katsottuna. Tätä sanotaan reunan 
\index{positiivinen suunnistus!b@reunan}%
\kor{positiiviseksi suunnistukseksi}.
\begin{figure}[H]
\begin{center}
\import{kuvat/}{kuvapot-11.pstex_t}
\end{center}
\end{figure}
\index{kiertointegraali} \index{polkuintegraali!c@kiertointegraali}%
Integraalia yli $\partial A$:n, em. tavalla laskettuna, sanotaan \kor{kiertointegraaliksi}
ja merkitään symbolilla $\oint_{\partial A}$. Huomioiden, että
(vrt.\ Luku \ref{polkuintegraalit})
\[
\vec t\,ds=d\vec r
\] 
nähdään, että kiertointegraali on yhdistelmä polkuintegraaleja (itse asiassa työintegraaleja),
joissa kukin reunan $\partial A$ erillinen osa kierretään suunnistuksen määräämällä tavalla.
Tulos mainituin merkinnöin on siis
\begin{equation} \label{Stokesin tasokaava}
\int_A \text{rot}\,\vec F\,dxdy=\oint_{\partial A} \vec F\cdot d\vec r.
\end{equation}

Olkoon seuraavaksi $\vec F$ avaruuden vektorikenttä ja $A$ perusalue avaruustasolla $T$.
Valitaan karteesinen koordinaatisto siten, että $T$ on $xy$-taso ja kirjoitetaan tässä 
koordinaatistossa $\vec F=F_1\vec i+F_2\vec j+F_3\vec k$ ja 
$\text{rot}\vec F=\partial_x F_2-\partial_y F_1$. Tällöin nähdään, että integraalikaava 
\eqref{Stokesin tasokaava} säilyttää pätevyytensä, koska kaavassa oikealla puolella
on $d\vec r\cdot\vec k=0$, jolloin $\vec F \cdot d\vec r = (F_1\vec i+F_2\vec j) \cdot d\vec r$.
Koska kaavassa vasemmalla puolella voidaan myös kirjoittaa 
$\,\text{rot}\vec F=(\nabla\times\vec F)\cdot\vec k$, niin voidaan vetää yleisempi johtopäätös:
Jos $\vec F$ on jatkuvasti derivoituva avaruuden vektorikenttä ja $A$ on perusalue
avaruustasolla, jonka yksikkönormaalivektori on $\vec n$, niin pätee integraalikaava
\begin{equation} \label{Stokesin avaruustasokaava}
\int_A (\nabla\times\vec F)\cdot\vec n\,dS = \oint_{\partial A} \vec F\cdot d\vec r.
\end{equation}
Tässä on normaalin $\vec n$ suunta ja reunan $\partial A$ suunnistus sidottava toisiinsa siten,
että kun reunaviivaa katsellaan normaalin $\vec n$ osoittamalta puolelta, niin tilanne on kuvan
mukainen.
\begin{figure}[H]
\begin{center}
\import{kuvat/}{kuvapot-12.pstex_t}
\end{center}
\end{figure}
Kaava \eqref{Stokesin avaruustasokaava} on erikoistapaus vieläkin yleisemmästä tuloksesta,
joka tunnetaan \linebreak \kor{Stokesin\footnote[2]{\vahv{Sir George Gabriel Stokes} 
(1819-1903) oli englantilainen fyysikko--matemaatikko. \index{Stokes, G. G.|av}} lauseen}
nimellä. Stokesin lauseen mukaan kaava \eqref{Stokesin avaruustasokaava} on tietyin edellytyksin
yleistettävissä koskemaan myös avaruuden kaarevia pintoja. Tämä yleisempi tulos on
integraalikaava (Stokesin kaava)
\begin{equation} \label{Stokesin avaruuskaava}
\boxed{\quad \int_A (\nabla\times\vec F)\cdot d\vec a
                              =\oint_{\partial A} \vec F\cdot d\vec r. \quad}
\end{equation}
Tässä $\vec F$ on avaruuden vektorikenttä, $A$ on (riittävän säännöllisen muotoinen) alue
avaruuden (riittävän säännöllisellä) pinnalla $S$ ja $d\vec a=\vec n\,dS$, missä $dS$
viittaa $S$:n pinta-alamittaan ja $\vec n=$ pinnan yksikkönormaalivektori.

Kaavassa \eqref{Stokesin avaruuskaava} edellytetään, että pinnan normaali $\vec n$ ja
reunaviivan $\partial A$ suunnistus on sidottu toisiinsa edellä kuvatulla tavalla.
Tällaisen suunnistuksen järjestäminen ei ole ongelma siinä tapauksessa, että $A$ on
avaruustasolla. Avaruuden kaarevilla pinnoilla voi kuitenkin ongelmia tulla, ja yleisessä
Stokesin lauseessa pintaa $A$ onkin rajoitettava geometrisella ehdolla: Pinnan on oltava
\index{suunnistuva pinta}%
oletetulla tavalla \kor{suunnistuva}. Tarkastellaan asiaa esimerkkien valossa.

\begin{multicols}{2} \raggedcolumns
\begin{Exa} Puolipallo on suunnistuva (vrt.\ kuvio), joten Stokesin kaava
\eqref{Stokesin avaruuskaava} on sille pätevä. Kaavan voi myös todistaa pallokoordinaatteja
käyttäen: Olkoon pallo $R$-säteinen ja
\[
\vec F=F(r,\theta,\varphi)\vec e_r+F_\theta(r,\theta,\varphi)\vec e_\theta
                                  +F_\varphi(r,\theta,\varphi)\vec e_\varphi.
\]
Tällöin (vrt.\ Luku \ref{divergenssi ja roottori})
\[
\nabla\times\vec F=\frac{1}{r\sin\theta}[\partial_\theta(\sin\theta\,F_\varphi)
                  -\partial_\varphi F_\theta]\vec e_r
                  +[\ldots]\vec e_\theta+ [\ldots]\vec e_\varphi,
\]
\begin{figure}[H]
\begin{center}
\import{kuvat/}{kuvapot-13.pstex_t}
\end{center}
\end{figure}
\end{Exa}
\end{multicols}
joten kun valitaan $\vec n=\vec e_r$ (toinen vaihtoehto olisi $\vec n=-\vec e_r$), niin
\begin{align*}
\int_A (\nabla\times\vec F)\cdot d\vec a
&=\int_0^{\pi/2}\int_0^{2\pi} \frac{1}{R\sin\theta}[\partial_\theta(\sin\theta F_\varphi)
                            -\partial_\varphi F_\theta]\,R^2\sin\theta\,d\theta d\varphi \\
&= R\int_0^{2\pi}
   \left(\int_0^{\pi/2} \partial_\theta (\sin\theta F_\varphi)\,d\theta\right)d\varphi 
    -R \int_0^{\pi/2}\left(\int_0^{2\pi} \partial_\varphi F_\theta\,d\varphi\right)d\theta \\
&= R\int_0^{2\pi}\left[\sijoitus{\theta=0}
                 {\theta=\pi/2}\bigl[\sin\theta\,F_\varphi(\theta,\varphi)\bigr]\right]d\varphi
   -R\int_0^{\pi/2}\left[\sijoitus{\varphi=0}
                   {\varphi=2\pi} F_\theta(\theta,\varphi)\right]d\theta \\
&= R\int_0^{2\pi} F_\varphi(\tfrac{\pi}{2},\varphi)\,d\varphi.
\end{align*}
Reunaviivalla $\partial A$ on $\,\theta=\frac{\pi}{2}\,$ ja
$\,\vec F\cdot d\vec r=\vec F\cdot (R\,d\varphi\,\vec e_\varphi)=RF_\varphi\,d\varphi$,
joten todetaan kaava \eqref{Stokesin avaruuskaava} päteväksi puolipallolle. \loppu

%\pagebreak
\begin{Exa} Toisessa esimerkkitapauksessa tarkastellaan pintaa, joka saadaan poistamalla
suorakulmaisen särmiön ulkopinnasta kaksi vierekkäistä sivutahkoa, vrt.\ kuvio.
\begin{figure}[H]
\begin{center}
\import{kuvat/}{kuvapot-14.pstex_t}
\end{center}
\end{figure}
Tässä tapauksessa pinta koostuu tasopinnoista $A_i$, joille kullekin pätee Stokesin kaava
\eqref{Stokesin avaruuskaava}, eli
\[
\int_{A_i} (\nabla\times\vec F)\cdot d\vec a=\oint_{\partial A_i} \vec F\cdot d\vec r.
\]
Kun normaalin suunta jokaisella osalla valitaan osoittamaan särmiön ulkopuolelle, niin nähdään,
että
\begin{align*}
\int_A (\nabla\times\vec F)\cdot d\vec a 
&= \sum_{i} \int_{A_i} (\nabla\times\vec F)\cdot d\vec a \\
&= \sum_i \oint_{\partial A_i} \vec F\cdot d\vec r \\
&= \int_{\partial A} \vec F\cdot d\vec r,
\end{align*}
\begin{multicols}{2} \raggedcolumns
\parbox{5cm}{sillä viivaintegraalit osien yhteisten särmien $\partial A_i\cap\partial A_j$ yli
kumoutuvat summauksessa, vrt.\ kuvio. Siis kaava \eqref{Stokesin avaruuskaava} on
pätevä tässäkin tapauksessa. Samaan tulokseen tultaisiin myös poistamalla särmiöstä mitkä
sivutahkot tahansa (1--5 kpl). \loppu}
\begin{figure}[H]
\begin{center}
\import{kuvat/}{kuvapot-15.pstex_t}
\end{center}
\end{figure}
\end{multicols}
\end{Exa}
\begin{Exa} Kolmantena esimerkkinä tarkastellaan pintaa, joka saadaan lieriövaipasta
leikkaamalla vaippa poikki, kiertämällä toista päätä $180\aste$, ja liittämällä päät jälleen
yhteen.
\begin{figure}[H]
\begin{center}
\import{kuvat/}{kuvapot-16.pstex_t}
\end{center}
\end{figure}
%\epsfig{file=kuvat/sylvaippa.eps} $\hookrightarrow$ \epsfig{file=kuvat/moebius.eps}
\index{Mzz@Möbiuksen nauha}%
Tuloksena oleva pinta on nk. \kor{Möbiuksen nauha}\footnote[2]{\hist{August Möbius} (1790--1868)
oli saksalainen matemaatikko. \index{Mzz@Möbius, A.|av}}. Möbiuksen nauhalle Stokesin kaava
\eqref{Stokesin avaruuskaava} \pain{ei} ole pätevä. Sen sijaan kaava on kyllä voimassa, jos
liitossauma jätetään avoimeksi ja viivaintegraalit sauman kummallakin puolella huomioidaan
kaavassa \eqref{Stokesin avaruuskaava}. Möbiuksen nauhan tapauksessa nämä saumaintegraalit
lasketaan samansuuntaisina, jolloin ne eivät saumaa yhteen liitettäessä kumoudu, toisin kuin
lieriöpinnan tapauksessa, vrt.\ kuvio.
\begin{figure}[H]
\begin{center}
\import{kuvat/}{kuvapot-17.pstex_t}
\end{center}
\end{figure}
Möbiuksen nauha on siis erimerkki ei--suunnistuvasta pinnasta, jolle Stokesin kaava 
\eqref{Stokesin avaruuskaava} ei päde. Möbiuksen nauha on itse asiassa
\index{yksipuolinen pinta}%
\kor{yksipuolinen} pinta, jolla normaalin $\vec n$ suuntaa ei voida valita ristiriidattomasti
niin, että se muuttuisi jatkuvasti pintaa pitkin kuljettaessa.
(Jos pinta leikataan auki, niin liimauskohdassa normaali on epäjatkuva.) \loppu
\end{Exa}

Stokesin lause voidaan yksinkertaisen pinnan tapauksessa todistaa pinnan parametrisaatiota
käyttäen, mutta todistus on tällöin melko tekninen ja vaatii suhteellisen voimakkaita
säännöllisyysoletuksia. Läpinäkyvämpi ja yleispätevämpi todistus saadaan, kun pintaa
approksimoidaan kolmion muotoisilla tasopinnoilla, samalla tavoin kuin pinta-alamitan
määrittelyssä (vrt. Luku \ref{pintaintegraalit}). Koska Stokesin tasokaava pätee jokaiselle
tasokolmiopinnalle, seuraa summaamalla, että Stokesin kaava \eqref{Stokesin avaruuskaava}
pätee myös kolmioista muodostetulle pinnan $A$ approksimaatiolle --- edellyttäen, että
summauksessa viivaintegraalit yli vierekkäisten kolmioiden yhteisten sivujen kumoutuvat.
Suunnistuvuusehto on juuri tässä. Sikäli kuin pinta on suunnistuva, saadaan kolmioverkkoa
tihentämällä raja-arvotulos \eqref{Stokesin avaruuskaava} edellyttäen, että vektorikenttä
$\vec F$, pinta $A$ (pinnan parametrisaatio), ja reunaviiva $\partial A$ ovat riittävän
säännöllisiä. Vektorikentän osalta riittää, että se on jatkuvasti derivoituva
(todellisuudessa riittää hieman vähempikin), eivätkä pintaa $A$ koskevat olettamuksetkaan ole
käytännön kannalta kovin rajoittavia, vrt.\ esimerkit edellä.

Stokesin lauseella, kuten Gaussin lauseellakin, on useampiulotteiset vastineensa. Nämä kuuluvat
matematiikan alaan nimeltä \kor{differentiaaligeometria}. Tällaisiin laajempiin yhteyksiin
sijoittuu luontevimmin myös Stokesin lauseen tarkempi todistus, joka tässä sivuutetaan.
\begin{Exa}
Kappale liikkuu voimakentässä
\[
\vec F=\frac{F}{a}\,(y\,\vec i-2z\,\vec j+3x\,\vec k\,)
\]
pitkin ympyrärataa, joka on pallopinnan $x^2+y^2+z^2=a^2$ ja tason $x+y+z=0$ leikkausviiva. 
Määritä Stokesin lauseen avulla voimakentän tekemä työ yhden kierroksen aikana, kun
kiertosuunta on pisteestä $(10,10,10)$ katsottuna myötäpäivään.
\end{Exa}
\ratk Kappaleen liikerata on $\partial A$, missä $A$ on origokeskinen, $a$-säteinen kiekko
tasolla $x+y+z=0$. Stokesin lauseen mukaan
\[
W=\oint_{\partial A} \vec F\cdot d\vec r=\int_A (\nabla\times\vec F)\cdot\vec n\,dS.
\]
Tässä on
\[
\nabla\times\vec F
=\frac{F}{a} \left|\begin{array}{ccc} 
 \vec i & \vec j & \vec k \\ \partial_x & \partial_y & \partial_z \\ y & -2z & 3x
 \end{array}\right|
=\frac{F}{a}(2\vec i-3\vec j -\vec k).
\]
Oletettua kiertosuuntaa vastaa $\,\vec n=-\frac{1}{\sqrt{3}}(\vec i+\vec j+\vec k)$, joten
$(\nabla\times\vec F)\cdot\vec n = \frac{2F}{\sqrt{3}\,a}$. Siis
\[
W = \int_A \frac{2F}{\sqrt{3}\,a}\,dS = \frac{2F}{\sqrt{3}\,a}\mu(A)
  = \frac{2F}{\sqrt{3}\,a}\,\pi a^2 = \underline{\underline{(2\pi/\sqrt{3})\,Fa}}. \loppu
\]

\subsection{Yleistetty Stokesin lause}
\index{Stokesin lause!a@yleistetty|vahv}

Kuten Gaussin lause, voidaan myös Stokesin lause yleistää niin, että pistetulon tilalla voi olla
yleisempi vektoritulo, eli joko ristitulo tai skalaarin ja vektorin kertolasku. Yleistetty
Stokesin kaava on
\[
\boxed{\quad \oint_{\partial A} d\vec r\ast\vec F=\int_A (d\vec a\times\nabla)\ast\vec F. \quad}
\]
Tässä on jälleen asetettava $\vec F\hookrightarrow f$, kun $\ast=$ `tyhjä'. Yleistetty kaava
pätee samanlaisin edellytyksin kuin peruskaava \eqref{Stokesin avaruuskaava}, ja sen voi myös 
perustella samaan tapaan kuin edellä, eli lähtemällä kaavan tasomuodosta.

\Harj
\begin{enumerate}

\item 
Olkoon $A=\{(x,y)\in \R^2 \mid x^2+y^2 \le 1\,\ja\,x \ge 0\}$. Laske seuraavat kiertointegraalit
yli $\partial A$:n (vastapäivään kiertäen) sekä suoraan viivaintegraaleina että muuntamalla ne
ensin tasointegraaleiksi (yleistettyä) Stokesin kaavaa käyttäen. \vspace{1mm}\newline
a) \ $\oint_{\partial A} \left[\,y^2\,dx+(x+y^2)\,dy\,\right] \qquad$
b) \ $\oint_{\partial A} (x+y^2)\,d\vec r$ \newline
c) \ $\oint_{\partial A} (xy^2\vec i+\abs{y}\vec j\,) \cdot d\vec r \qquad\qquad$
d) \ $\oint_{\partial_A} (y^3\vec i-x\vec j\,) \times d\vec r$

\item 
Laske $\oint_{\partial A} \vec F\cdot d\vec r$ \ a) suoraan viivaintegraalina, \  
b) Stokesin lauseen avulla, kun $\vec F=x\vec k$ ja $A \subset S$ vastaa 
parametrien arvoja $(u,v)\in [0,1]\times[0,3]$ pinnalla $\,S:\ x=8u^2,\ y=v^2,\ z=4uv$.

\item
Näytä Stokesin lauseen avulla, että jos $S$ on pallon $K:\,x^2+y^2+z^2=R^2$ ja tason 
$T:\,x+y+z=0$ leikkauskäyrä, niin $\oint_S (y\,dx+z\,dy+x\,dz)=\pm\sqrt{3}\pi R^2$.

\item
Laske pintaintegraali $\int_A \nabla\times(\vec k\times\vec r\,) \cdot d\vec a$, kun $S$, 
$A \subset S$ ja $S$:n yksikkönormaalivektorin $\vec n=n_x\vec i+n_y\vec j+n_z\vec k$
suunta on annettu seuraavilla ehdoilla. \vspace{1mm}\newline
a) \ $S:\,z=0, \quad A:\,x^2+y^2 \le 1, \quad n_z \ge 0$ \newline
b) \ $S:\,x^2+y^2+z^2=1, \quad A:\,z \ge 0, \quad n_z \ge 0$ \newline
c) \ $S:\,x^2+y^2+z^2=1, \quad A:\,z \le 0, \quad n_z \le 0$

\item
Halutaan laskea kiertointegraali $\oint_S \vec F \cdot d\vec r$ yli suljetun käyrän
$S:\,\vec r=\cos t\,\vec i+\sin t\,\vec j+\sin 2t\,\vec k,\ t\in[0,2\pi]$, kun
$\vec F=(e^x-y^3)\vec i+(e^y+x^3)\vec j+e^z\vec k$. Laske integraali Stokesin kaavan avulla
huomioimalla, että $S$ on pinnalla $z=2xy$.

\item
Olkoon $S$ lieriön $K:(x-1)^2+4y^2=16$ ja tason $T:2x+y+z=3$ \linebreak leikkauskäyrä ja
$\vec F=[\sin(x^2)+y^2+z^2]\vec i+(2xy+z)\vec j+(xz+2yz)\vec k$. Laske \linebreak
$\oint_S \vec F \cdot d\vec r$, kun $S$:n suunnistus on vastapäivään kaukaa positiiviselta
$z$-akselilta katsoen.

\item
Valitse pinnalle $A:\,z=9-x^2+y^2,\ x^2+y^2 \le 9$ suunnistus ja laske 
$\int_A \nabla\times\vec F \cdot d\vec a$, kun $\vec F=-y\vec i+x^2\vec j+z\vec k$.

\item
Laske Stokesin kaavan avulla pintaintegraali
\[
\int_{A} \nabla\times[(x-z^2)\vec i+(x^3+z)\vec j+xy\vec k] \cdot d\vec a,
\]
missä $A$ on kartiopinnan $\,S:\,x^2+y^2=(z-1)^2$ tasojen $z=0$ ja $z=1$ väliin jäävä osa.

\item 
Sähkövirran tiheys $\vec J$ ja magneettikentän voimakkuus $\vec H$ ovat lieriökoordinaatistossa
muotoa $\vec J= J(r)\vec e_z$, $\vec H = H(r)\vec e_\varphi$. Määrää $H(r)$ funktion $J(r)$ 
avulla käyttämällä Maxwellin yhtälöä $\nabla\times\vec H = \vec J$ ja Stokesin kaavaa 
sopivasti valitulla pinnalla $A$.

\item
Näytä yleistetty Stokesin kaava oikeaksi siinä tapauksessa, että $\vec F$ on avaruuden
vektorikenttä ja $A \subset T$, missä $T$ on avaruustaso.

\item
Olkoon $S$ yksinkertainen suljettu käyrä avaruustasolla $T$, jonka yksikkönormaalivektori on
$\vec n=a\vec i+b\vec j+c\vec k$. Näytä, että $S$:n sisään jäävän pinnan $A$ ala on
\[
\mu(A)=\frac{1}{2}\left|\oint_S \bigl[(bz-cy)dx+(cx-az)dy+(ay-bx)dz\bigr]\right|.
\]

\item
Määritellään pinta $A$ parametrisaatiolla
\[
A:\quad \begin{cases}
        \,x=\cos 2u(a+bv\sin nu), \\ \,y=\sin 2u(a+bv\sin nu), \\ \,z=bv\cos nu,
        \end{cases} \quad
(u,v) \in \bigl[-\tfrac{\pi}{2},\tfrac{\pi}{2}\,\bigr)\times
          \bigl[-\tfrac{\pi}{2},\tfrac{\pi}{2}\,\bigr],
\]
missä $0<b<a$ ja $n\in\N$. Millainen on $A$:n reuna $\partial A\,$? Millä $n$:n arvoilla $A$ on
suunnistuva pinta?

\item (*)
Laske kiertointegraali $\oint_S \vec F \cdot d\vec r$, kun
$\,\vec F=ye^x\vec i+(x^2+e^x)\vec j+e^z\vec k\,$ ja $S$ on suljettu parametrinen käyrä
\begin{align*}
\vec r \,=\ &(2\cos t-3\sin t)\vec i+(2+3\cos t+6\sin t)\vec j+(1+6\cos t-2\sin t)\vec k, \\
            &\,t\in[0,2\pi].
\end{align*}

\end{enumerate} %Stokesin lause
\section{Pyörteetön vektorikenttä} \label{pyörteetön vektorikenttä}
\alku
\index{pyzz@pyörteetön vektorikenttä|vahv}
\index{vektorikenttä!b@pyörteetön|vahv}

Jos $\vec F$ on fysikaalinen virtaus-, sähkö-, magneetti- ym. kenttä, sanotaan kenttää
$\vec \omega=\nabla\times\vec F$ kentän pyörrekentäksi (vrt.\ Luku 
\ref{divergenssi ja roottori}). Stokesin lause ilmaisee yhteyden kentän $\vec F$ ja pyörrekentän
$\vec\omega$ välillä:
\[
\oint_{\partial A} \vec F\cdot d\vec r=\int_A \vec\omega\cdot d\vec a.
\]
Tällä yhteydellä on paljon käyttöä esimerkiksi sähkömagneetikassa, jossa pyörrekentät ovat 
näkyvässä roolissa jo perusyhtälöissä (Maxwellin yhtälöissä, vrt.\ Luku 
\ref{divergenssi ja roottori}). Jos $\vec F$ on pyörteetön kenttä, eli jos $\vec\omega=\vec 0$,
antaa Stokesin lause tuloksen
\begin{equation} \label{Kpot-1}
\oint_{\partial A} \vec F\cdot d\vec r=0.
\end{equation}
Luvun \ref{polkuintegraalit} termein tämä voidaan lukea: Pyörteettömän vektorikentän tekemä työ
suljettua polkua pitkin = 0. Sanotaan, että kenttä $\vec F$ on tällöin
\index{konservatiivinen vektorikenttä} \index{vektorikenttä!c@konservatiivinen}%
\kor{konservatiivinen} (energian säilyttävä).

Jos $\vec F$ on gradienttikenttä, ts. $\vec F=-\nabla u$, niin $\vec F$ on pyörteetön. Tulos 
\eqref{Kpot-1} on tällöin voimassa --- ja tiedettiin ilman Stokesin lausettakin, sillä 
gradienttikentän tekemä työ = 0, jos polun alku- ja loppupisteet ovat samat, vrt.\ Luku 
\ref{polkuintegraalit}. --- Mutta entä jos kysytäänkin toisin päin: Millainen on riittävän 
säännöllinen, mutta muuten mahdollisimman yleinen vektorikenttä $\vec F$, joka on joko (a)
konservatiivinen tai (b) pyörteetön? Seuraavassa keskitytään vastaamaan näihin kysymyksiin. 
Tarkastellaan aluksi tason vektorikenttiä.

\index{yhtenzy@yhtenäinen (joukko)}%
Olkoon $A\subset\R^2$ \pain{avoin} joukko. Sanotaan, että tällainen joukko on \kor{yhtenäinen}
(engl.\ connected), jos mielivaltaiset kaksi $A$:n pistettä ovat yhdistettävissä jatkuvalla, 
suoristuvalla parametrisella käyrällä, jonka päätepisteet ovat mainitut pisteet ja joka on 
kokonaisuudessaan $A$:ssa. Jatkossa sanotaan avointa ja yhtenäistä joukkoa
\index{alue}%
\kor{alueeksi} (engl.\ domain)\footnote[2]{Alue määritellään hieman yleisemmin joukkona
$B = A \cup S$, missä $A$ on avoin ja yhtenäinen ja $S \subset \partial A$. Jos
$S = \emptyset$, niin $B=A$ on \kor{avoin alue}. Jos $S=\partial A$, niin $B=\overline{A}$ on
\kor{suljettu alue}. \index{avoin alue|av} \index{suljettu alue|av}}. 
\begin{Lause} \label{Lpot-1} \index{gradienttikenttä|emph}
Jos tason vektorikenttä $\vec F$ on jatkuva ja konservatiivinen alueessa $A \subset \R^2$, niin
$\vec F$ on gradienttikenttä, ts.\ $\vec F=-\nabla u$, missä $u$ on $A$:ssa jatkuvasti
derivoituva.
\end{Lause}
\tod Olkoon $\vec r_0\vastaa (x_0,y_0)\in A$ kiinteä ja $\vec r\vastaa (x,y)\in A$ muuttuva 
$A$:n piste. Olkoon edelleen $p$ yksinkertainen, suoristuva parametrinen käyrä (polku) 
$\vec r=\vec r\,(t)$, $t\in [a,b]$, jolle $\vec r\,(a)=\vec r_0$ ja
$\vec r_b=\vec r\vastaa (x,y)$. Määritellään
\[
u(p,x,y)=-\int_p \vec F\cdot d\vec r.
\]
Jos $u$ ei riipu polun $p$ valinnasta, ts. $u(p,x,y)=u(x,y)$, niin jatkamalla polkua pisteestä
$\vec r$ seuraa jatkuvuusoletuksesta, että
\[
u(\vec r+\Delta\vec r\,)=-\vec F(\vec r\,)\cdot\Delta\vec r+o(|\Delta\vec r\,|).
\]
Tämän mukaan $u$ on differentioituva ja $\nabla u=-\vec F$. Riittää siis osoittaa, että
$u(p,x,y)$ riippuu vain polun päätepisteistä. Olkoon $p_1$ ja $p_2$ kaksi polkua joilla on sama
alkupiste $(x_0,y_0)$ ja sama loppupiste $(x,y)$. Jos vastaavat geometriset käyrät ovat $S_1$
ja $S_2$, niin
\[
S_1\cup S_2=\bigcup_i S_i,\quad S_i\subset A,
\]
missä $S_i$:t ovat suljettuja käyriä (vrt.\ kuvio).
\vspace{3mm}
\begin{figure}[H]
\begin{center}
\import{kuvat/}{kuvapot-18.pstex_t}
\end{center}
\end{figure}
Tällöin 
\[
\int_{p_1} \vec F\cdot d\vec r-\int_{p_2} \vec F\cdot d\vec r
                         = \sum_i \oint_{S_i} \vec F\cdot d\vec r.
\]
Koska $\vec F$ on konservatiivinen, on $\oint_{S_i} \vec F\cdot d\vec r=0$, joten
\[
\int_{p_1} \vec F\cdot d\vec r=\int_{p_2} \vec F\cdot d\vec r.
\]
Siis $u(p,x,y)=u(x,y)$ on polun valinnasta riippumaton. \loppu

\index{yhdesti yhtenäinen alue}%
Sanotaan, että alue $A\subset\R^2$ on \kor{yhdesti yhtenäinen} (engl.\ simply connected), jos
pätee $B \subset A$ aina kun $B$ on alue, jolle $\partial B\subset A$. Kun kirjoitetaan
$S=\partial B$, niin ehdon hieman havainnollisempi muotoilu on: $A\subset\R^2$ on yhdesti
yhtenäinen, jos jokainen suljettu käyrä $S \subset A$ voidaan kutistaa pisteeksi niin, että
käyrä pysyy kutistuessaan koko ajan $A$:ssa. Yhdesti yhteinäinen joukko on 'järvi ilman saaria'.
\begin{figure}[H]
\begin{center}
\import{kuvat/}{kuvapot-19.pstex_t}
\end{center}
\end{figure}
Seuraava Stokesin lauseen seurannaistulos on vektorikenttien teorian keskeisimpiä. Tuloksella
on paljon käyttöä fysiikassa.
\begin{Lause} \label{Lpot-2} \index{gradienttikenttä|emph}
Jos tason vektorikenttä $\vec F$ on jatkuvasti derivoituva ja pyörteetön (rot\,$\vec F=0$)
yhdesti yhtenäisessä alueessa $A$, niin $\vec F$ on gradienttikenttä.
\end{Lause}
\tod Kentän potentiaali määritetään polkuintegraalina kuten edellä, jolloin riittää jälleen
osoittaa, että integraalin arvo riippuu vain polun päätepisteistä. Tämän todistamiseksi
valitaan jälleen kaksi polkua, joilla on samat päätepisteet, jolloin (vrt.\ todistus edellä)
\[
\int_{p_1} \vec F\cdot d\vec r-\int_{p_2} \vec F\cdot d\vec r
                               = \sum_i \oint_{S_i} \vec F\cdot d\vec r.
\]
Nytkin on oikea puoli = 0, sillä oletetun yhdesti yhtenäisyyden perusteella käyrien $S_i$ 
sisäänsä sulkemat alueet $A_i$ sisältyvät joukkoon $A$, jolloin Stokesin tasokaavan
(edellinen luvun kaava \eqref{Stokesin tasokaava}) ja $\vec F$:n pyörteettömyyden
perusteella
\[
\oint_{S_i} \vec F\cdot d\vec r=\int_{A_i} \text{rot}\,\vec F\,dxdy=0. \loppu
\]
\vspace{0.2mm}
\begin{multicols}{2} \raggedcolumns
Lauseiden \ref{Lpot-1}--\ref{Lpot-2} todistuksien perusteella gradienttikentän potentiaali
saadaan lasketuksi työintegraalina:
\[
u(\vec r\,)=-\int_{p:\,\vec r_0\,\kohti\,\vec r} \vec F\cdot d\vec r.
\]
\setlength{\unitlength}{1cm}
\begin{center}
\begin{picture}(4,1.5)(0,0.5)
\put(-0.1,-0.1){$\bullet$}
\put(3.9,0.9){$\bullet$}
\spline(0,0)(1,1)(2,1)(3,0)(4,1)
\put(1,0.86){\vector(3,2){0.1}}
\put(0.2,-0.1){$\vec r_0$} \put(3.9,1.2){$\vec r$}
\put(1.2,0.5){$p$}
\end{picture}
\end{center}
\end{multicols}
Laskukaavassa voidaan polku $p$ voidaan valita vapaasti, kunhan se pysyy alueessa $A$
(jossa $\vec F$ on määritelty). Jos $A$:n geometria ei aseta rajoituksia, voidaan polku valita
esimerkiksi seuraamaan koordinaattiakselien suuntaisia suoria, jolloin potentiaalin
laskukaavaksi tulee
\begin{align*}
\vec F(x,y) &= F_1(x,y)\vec i + F_2(x,y)\vec j\,: \\
     u(x,y) &= -\int_{x_0}^x F_1(t,y_0)\,dt-\int_{y_0}^y F_2(x,t)\,dt.
\end{align*}
\begin{Exa}
Onko vektorikenttä
\[
\vec F(x,y)=(x^3-y^3+\cos x)\vec i-(3xy^2+e^{-y})\vec j
\]
gradienttikenttä?
\end{Exa}
\ratk Koska
\[
\text{rot}\, \vec F=-\partial_x(3xy^2+e^{-y})-\partial_y(x^3-y^3+\cos x)=0,
\]
niin $\vec F$ on gradienttikenttä $\R^2$:ssa (Lause \ref{Lpot-2}). Valinnalla $(x_0,y_0)=(0,0)$ 
potentiaaliksi saadaan
\begin{align*}
u(x,y) &= -\int_0^x (t^3+\cos t)\,dt+\int_0^y (3xt^2+e^{-t})\,dt \\
&= -\frac{1}{4}x^4-\sin x+xy^3-e^{-y}+2.
\end{align*}
Yleinen potentiaalifunktio on
\[
u(x,y)=-\frac{1}{4}x^4+xy^3-\sin x-e^{-y}+C\quad (C=\text{vakio}). \loppu
\]
\begin{Exa}
Polaarikoordinaatistossa on määritelty vektorikenttä $\vec F=\frac{1}{r}\,\vec e_\varphi$.
Tutki, onko $\vec F$  gradienttikenttä alueessa $A$, kun \vspace{1mm}\newline
a) $A=\{\,(x,y)\in\R^2 \; | \; x>0 \; \ja \; y>0\,\}$, \newline
b) $A=\{\,(x,y)\in\R^2 \; | \; (x,y)\neq(0,0)\,\}$.
\end{Exa}
\ratk Polaarikoordinaatistossa on (vrt.\ Luku \ref{divergenssi ja roottori})
\[
\vec F=F_r\vec e_r+F_\varphi \vec e_\varphi\,: \quad 
      \text{rot}\,\vec F=\frac{1}{r}\,\partial_r(rF_\varphi)-\frac{1}{r}\,\partial_\varphi F_r.
\]
Tässä on $F_r=0$ ja $F_\varphi=1/r$, joten $\vec F$ on pyörteetön jokaisessa alueessa, joka ei
sisällä origoa.

a) Tässä $A$ on yhdesti yhtenäinen, joten $\vec F$ on gradienttikenttä. Valitsemalla $r_0>0$
voidaan potentiaali $u(x,y)=v(r,\varphi)$ laskea polkuintegraalina
\begin{align*}
v(r,\varphi)
&= -\int_{r_0}^r F_r(t,0)\,dt-\int_0^\varphi F_\varphi(r,t)\,r\,dt \,=\, -\varphi \\ 
&\impl\quad u(x,y) \,=\, -\underline{\underline{\Arcsin\left(\frac{y}{\sqrt{x^2+y^2}}\right)}}.
\end{align*}

b) Tässä $A$ ei ole yhdesti yhtenäinen, joten tutkitaan erikseen, onko $\vec F$ 
konservatiivinen. Valitaan polku $p:\vec r=\vec r(\varphi)$, $0\leq\varphi\leq 2\pi$ siten,
että $\vec r(\varphi)$ kiertää $r_0$-säteisen ympyräviivan:
\[
\vec r(\varphi)=r_0\,\vec e_r,\quad r_0>0,\ 0\leq\varphi\leq 2\pi.
\]
Tällöin polun alku- ja loppupisteet ovat samat, $\vec r(\varphi)\vastaa (x,y)\in A$ jokaisella
$\varphi$ ja
\[
\int_p \vec F\cdot d\vec r=\int_0^{2\pi} \frac{1}{r_0}\cdot r_0\,d\varphi=2\pi\neq 0,
\]
Siis $\vec F$ ei ole $A$:ssa konservatiivinen eikä näin muodoin myöskään gradienttikenttä
$A$:ssa. \loppu

\subsection{Avaruuden pyörteetön vektorikenttä}

Vertaamalla Lauseen \ref{Lpot-1} todistusta Stokesin lauseen kolmiulotteiseen versioon 
nähdään, että todistus (ja siis myös Lauseen \ref{Lpot-1} väittämä) on sellaisenaan pätevä
myös avaruuden vektorikentälle. Sikäli kuin alueen geometria ei aseta rajoituksia, voidaan
kolmiulotteisen, konservatiivisen vektorikentän 
$\vec F = F_1(x,y,z)\vec i + F_2(x,y,z)\vec j + F_3(x,y,z)\vec k$ potentiaali laskea esimerkiksi
polkuintegraalina
\[
u(x,y,z) = -\int_{x_0}^x F_1(t,y_0,z_0)\,dt -\int_{y_0}^y F_2(x,t,z_0)\,dt
                                            -\int_{z_0}^z F_3(x,y,t)\,dt.
\]
Myös Lause \ref{Lpot-2} yleistyy kolmeen ulottuvuuteen, kunhan yhdesti yhtenäisyys
määritellään sopivasti. Tarkastellaan yksinkertaisia, suljettuja, suoristuvia käyriä
$S\subset A$, joille on löydettävissä Stokesin lauseen ehdot täyttävä pinta $B\subset\R^3$ 
siten, että $\partial B=S$. Jos jokaiselle tällaiselle käyrälle on pinta $B$ vielä määrättävissä
siten, että $B\subset A$, niin aluetta $A$ sanotaan yhdesti yhtenäiseksi. Esimerkiksi
\[
A=\{\,(x,y,z)\in\R^3 \; | \; a^2<x^2+y^2+z^2<R^2\,\} \quad (0 \le a < R)
\]
on yhdesti yhtenäinen (!). Sen sijaan
\[
A=\{\,(x,y,z)\in\R^3 \; | \; x^2+y^2+z^2<R^2 \; \ja \; x^2+y^2>a^2\,\} \quad (0 \le a < R)
\]
ei ole yhdesti yhtenäinen, sillä jos
\[
S=\{\,(x,y,z)\in\R^3 \; | \; x^2+y^2=b^2 \; \ja \; z=0\,\},
\]
niin $S\subset A$, kun $a<b<R$, mutta jos $B\subset\R^3$ on pinta, niin ehdot $S=\partial B$ ja
$B\subset A$ eivät voi yhtä aikaa toteutua.
\begin{Exa}
Pallokoordinaatistossa on määritelty vektorikenttä
\[
\vec F(r,\theta,\varphi)
            =\frac{1}{r^3}(2\cos\theta\,\vec e_r+\sin\theta\,\vec e_\theta),\quad r\neq 0.
\]
Määrää kentän potentiaali, jos sellainen on.
\end{Exa}
\ratk Tässä on $\vec F=F_r\vec e_r+F_\theta\vec e_\theta+F_\varphi\vec e_\varphi$, missä
\[
F_r=2r^{-3}\cos\theta,\quad F_\theta=r^{-3}\sin\theta,\quad F_\varphi=0,
\]
joten (ks.\ Luku \ref{div ja rot käyräviivaisissa})
\[
\nabla\times\vec F = 0\,\vec e_r + 0\,\vec e_\theta
  + \frac{1}{r}[\partial_r(rF_\theta)-\partial_\theta F_r]\vec e_\varphi = \vec 0\quad (r\neq 0).
\]
Koska origon poissulkeminen ei aiheuta yhtenäisyysongelmia kolmessa dimensiossa, niin $\vec F$
on gradienttikenttä. Potentiaali voidaan laskea ottamalla lähtöpisteeksi esimerkisi
$\vec r_0=(r_0,0,0)$, $r_0>0$, ja etenemällä koordinaattiviivojen suuntaisia polkuja:
\begin{align*}
u(r,\theta,\varphi) &= -\int_{r_0}^r F_r(t,0,0)\,dt-\int_0^\theta F_\theta(r,t,0)r\,dt
                                 -\int_0^\varphi F_\varphi(r,\theta,t)r\sin\theta\,dt \\
                    &= -\int_{r_0}^r 2t^{-3}\,dt-\int_0^\theta r^{-2}\sin t\,dt \\
                    &= \sijoitus{r_0}{r}t^{-2}+r^{-2}\sijoitus{0}{\theta} \cos t \\
                    &= r^{-2}\cos\theta+\text{vakio} \; (=-r_0^{-2}).
\end{align*}
Potentiaaliksi kelpaa siis $u(r,\theta,\varphi)=\underline{\underline{r^{-2}\cos\theta}}$. 
\loppu

\Harj
\begin{enumerate}

\item
Totea vektorikenttä $\vec F(x,y)=(y+2x)\vec i+x\vec j$ pyörteettömäksi. Laske tätä havaintoa
hyödyksi käyttäen polkuintegraali $\int_p \vec F \cdot d\vec r$, missä $p$ kulkee pisteestä
$(1,0)$ origon kautta pisteeseen $(1,2)$ pitkin ympyrän kaarta.

\item
Määritä funktio $f$ siten, että $f(0)=1$ ja tason vektorikenttä 
\[
\vec F(x,y)=f(y)[2xy\vec i+x^2(y+1)\vec j\,]
\] 
on pyörteetön. Mikä tällöin on kentän potentiaali?

\item
Osoita seuraavat vektorikentät koko avaruudessa pyörteettömiksi ja määritä kenttien 
potentiaalit.
\vspace{1mm}\newline
a) \ $z\vec i+z\cos y\vec j+(x+\sin y)\vec k \qquad$
b) \ $e^x\sin y\vec j+e^x\cos y\vec j+z\vec k$ \newline
c) \ $2xyz\vec i+x^2z\vec j+x^2y\vec k \qquad$
d) \ $-2xe^{-y}\vec i+(x^2e^{-y}+\sin z)\vec j+y\cos z\vec k$ \newline
e) \ $(x^2+y^2+z^2)(x\vec i+y\vec j+z\vec k)=r^2\vec r$

\item 
Vektorikentästä $\vec F$ tiedetään, että se on koko avaruudessa pyörteetön ja muotoa 
$\vec F(x,y,z)=xy^2\vec i+(\cos z+x^2y)\vec j+yf(z)\vec k$. Määrää kentän 
skalaaripotentiaali ja tämän avulla polkuintegraali $\int_p \vec F\cdot d\vec r$, missä
$p$:n alkupiste on $(0,0,0)$ ja loppupiste $(2,-1,3)$. 

\item (*) \index{zzb@\nim!Hzy@Häiriö kentässä}
(Häiriö kentässä) Tason vektorikenttään $\vec U_0=u_0\vec i$ ($u_0=$ vakio) tuodaan este
\[
A=\{(x,y)\in\R^2 \mid x^2+y^2 \le R^2\},
\] 
jolloin kenttä häiriintyy muotoon $\vec U=u(x,y)\vec i + v(x,y)\vec j$. Häiriintyneestä
kentästä $\vec U$ tiedetään, että se on sekä lähteetön että pyörteetön esteen ulkopuolella,
ts.\ $\text{div}\,\vec U = \text{rot}\,\vec U=0$ muualla kuin $A$:ssa. Lisäksi tiedetään, että
$\vec U \cdot \vec n = 0\,$ reunalla $\partial A$ (eli kun $x^2+y^2=R^2$, $\vec n=\,$
$\partial A$:n normaali) ja että $\vec U(x,y)\kohti\vec U_0$, kun $x^2+y^2\kohti\infty$.
Ongelmana on löytää nämä ehdot täyttävä kenttä $\vec U$. Koska $A$:n ulkopuolinen alue ei
ole yhdesti yhtenäinen, ei $\vec U$ välttämättä ole gradienttikenttä (vaikka onkin 
pyörteetön).

a) Näytä, että problemalla on ratkaisu, jonka napakoordinaattimuoto on
\[
\vec U=-\nabla f(r)\cos\varphi + g(r)\vec e_\varphi
\] 
eli määritä funktiot $f(r),g(r)$ mahdollisimman yleisessä muodossa niin, että annetut ehdot
toteutuvat. (Muita ratkaisuja ei ongelmalla ole). \newline
b) Totea, että em.\ ratkaisusta tulee yksikäsitteinen, kun asetetaan lisäehto muotoa
$v(R,0)=v_0$ (tässä karteesiset koordinaatit).

\end{enumerate} %Pyörteetön vektorikenttä

\printindex

\backmatter

%\include{takakansi}

\end{document}
