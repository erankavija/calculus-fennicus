\documentclass[12pt,a4paper,finnish]{book}
\usepackage[finnish]{babel}
\usepackage[utf8x]{inputenc}
\usepackage[T1]{fontenc}
\usepackage{amsmath,amssymb}
\usepackage{theorem}
\usepackage{graphicx}
\usepackage{apropos}
\usepackage{epsfig}
\usepackage{multicol}
\usepackage{amscd}
\usepackage{curves}
\usepackage{epic}
\usepackage{eepic}
\usepackage{float}
\usepackage{import}
\usepackage{rotating}
\usepackage{makeidx}

% Riittävä sidontamarginaali. Huom! Textwidth määriteltävä täällä,
% jottei sotke geometry-paketin toimintaa.
\usepackage[hcentering, textwidth=35em,bindingoffset=23mm]{geometry}

\setlength{\parindent}{0mm}
\setlength{\parskip}{\medskipamount}

\usepackage{jpheader}

% Hyperlinkit
\usepackage[colorlinks=true,linkcolor=blue,breaklinks=true]{hyperref}
\setcounter{secnumdepth}{1}
\setcounter{tocdepth}{2}

\renewcommand{\thechapter}{\Roman{chapter}}
\renewcommand{\thefootnote}{\fnsymbol{footnote}}
\renewcommand{\theequation}{\arabic{equation}}

\theoremheaderfont{\rmfamily\scshape}

%\includeonly{etukansi,esipuhe}
%\includeonly{01-0,I-1,I-2,I-3,I-4,I-5,I-6,I-7,I-8,I-9,I-10,I-11,I-12,
%II-0,II-1,II-2,II-3,II-3,II-4,II-5,II-6,II-7,II-8,
%III-0,III-1,III-2,III-3,III-4,
%IV-0,IV-1,IV-2,IV-3,IV-4,IV-5,
%V-0,V-1,V-2,V-3,V-4,V-5,V-6,V-7,V-8,V-9,
%exp-0,exp-1,exp-2,exp-3,exp-4,
%dif-0,dif-1,dif-2,dif-3,dif-4,dif-5,dif-6,
%int-0,int-1,int-2,int-3,int-4,int-5,int-6,int-7,int-8,int-9,
%dy-0,dy-1,dy-2,dy-3,dy-4,dy-5,dy-6,dy-7,dy-8,
%m-0,m-1,m-2,m-3,m-4,m-5,m-6,m-7,m-8,
%udif-0,udif-1,udif-2,udif-3,udif-4,udif-5,udif-6,udif-7,udif-8,udif-9,
%eig-0,eig-1,eig-2,eig-3,eig-4,
%uint-0,uint-1,uint-2,uint-3,uint-4,uint-5,uint-6,uint-7,
%Gauss-0,Gauss-1,Gauss-2,Gauss-3,Gauss-4,Gauss-5}

\makeindex

\begin{document}

\frontmatter

\include{etukansi}
\include{esipuhe}

\tableofcontents

\mainmatter

%Esimerkkien numeroinnin makroja
\newcommand{\alku}{\setcounter{Exa}{0} \setcounter{equation}{0}}
\newcommand{\jatko}{\addtocounter{Exa}{-1}}
\newcommand{\seur}{\addtocounter{Exa}{1}}

%Hakemison merkintöjä
\index{alkeisfunktio!a@nimetty|see{funktio C}}
\index{avaruuskäyrä|see{käyrä}}
\index{differentiaaliyhtälö!p@lineaarinen|see{lineaarinen DY}}
\index{divergoiminen|see{hajaantuminen}}
\index{funktio A!k@operaattori|see{differentiaaliop.}}
\index{homogeeninen DY|see{lineaarinen DY}}
\index{kasvava|see{monotoninen}}
\index{konvergoiminen|see{suppeneminen}}
\index{kosini ($\cos$)|see{funktio C}}
\index{kosekantti ($\csc$)|see{funktio C}}
\index{kotangentti ($\cot$)|see{funktio C}}
\index{kuvaus|see{funktio A}}
\index{kzyyrzy@käyräintegraali|see{viivaintegraali}}
\index{lineaarikombinaatio|see{l.-yhdistely}}
\index{mittasuhde|see{muuntosuhde}}
\index{nabla|see{differentiaalioperaattori}}
\index{operaattori!b@muu|see{differentiaalioperaattori}}
\index{ortonormeerattu kanta|see{kanta}}
\index{sekantti ($\sec$)|see{funktio C}}
\index{sijoitus|see{muuttujan vaihto}}
\index{sini ($\sin$)|see{funktio C}}
\index{siszy@sisätulo|see{skalaaritulo}}
\index{suurennussuhde|see{muuntosuhde}}
\index{tangentti ($\tan$)|see{funktio C}}
\index{tasokäyrä|see{käyrä}}
\index{tieintegraali|see{polkuintegraali}}
\index{tzy@täydellinen DY|see{lineaarinen DY}}
\index{vzy@vähenevä|see{monotoninen}}

\include{01-luvut-ja-lukujonot/I-0}  % Luvut ja lukujonot
\include{01-luvut-ja-lukujonot/I-1}  % Rationaaliluvut
\include{01-luvut-ja-lukujonot/I-2}  % Kunta
\include{01-luvut-ja-lukujonot/I-3}  % Logiikka ja joukko-oppi
\include{01-luvut-ja-lukujonot/I-4}  % Jonon käsite
\include{01-luvut-ja-lukujonot/I-5}  % Äärettömät desimaaliluvut
\include{01-luvut-ja-lukujonot/I-6}  % Lukujonon raja-arvo
\include{01-luvut-ja-lukujonot/I-7}  % Suppenevat lukujonot
\include{01-luvut-ja-lukujonot/I-8}  % Monotoniset ja rajoitetut lukujonot
\include{01-luvut-ja-lukujonot/I-9}  % Reaaliluvut
\include{01-luvut-ja-lukujonot/I-10} % Cauchyn jonot
\include{01-luvut-ja-lukujonot/I-11} % Reaalilukujen ominaisuuksia
\include{01-luvut-ja-lukujonot/I-12} % Klassinen sarjaoppi

\include{02-vektorit-ja-analyyttinen-geometria/II-0} % Vektorit ja analyyttinen geometria
\include{02-vektorit-ja-analyyttinen-geometria/II-1} % Euklidinen taso
\include{02-vektorit-ja-analyyttinen-geometria/II-2} % Tason vektorit
\include{02-vektorit-ja-analyyttinen-geometria/II-3} % Skalaaritulo
\include{02-vektorit-ja-analyyttinen-geometria/II-4} % *Abstrakti skalaaritulo
\include{02-vektorit-ja-analyyttinen-geometria/II-5} % Trigonometriset funktiot
\include{02-vektorit-ja-analyyttinen-geometria/II-6} % Avaruuden vektorit
\include{02-vektorit-ja-analyyttinen-geometria/II-7} % Suorat, tasot ja pinnat
\include{02-vektorit-ja-analyyttinen-geometria/II-8} % Käyräviivaiset koordinaatistot

\include{03-kompleksiluvut/III-0} % Kompleksiluvut
\include{03-kompleksiluvut/III-1} % Osoitinkunta
\include{03-kompleksiluvut/III-2} % Kompleksiluvut
\include{03-kompleksiluvut/III-3} % Algebran peruslause
\include{03-kompleksiluvut/III-4} % *Kompleksikertoiminen vektoriavaruus

\include{04-reaalimuuttujien-funktiot/IV-0} % Reaalimuuttujien funktiot
\include{04-reaalimuuttujien-funktiot/IV-1} % Yhden muuttujan funktiot
\include{04-reaalimuuttujien-funktiot/IV-2} % Käänteisfunktio
\include{04-reaalimuuttujien-funktiot/IV-3} % Kahden ja kolmen muuttujan funktiot
\include{04-reaalimuuttujien-funktiot/IV-4} % Parametriset käyrät ja pinnat
\include{04-reaalimuuttujien-funktiot/IV-5} % *Funktioavaruus

\include{05-jatkuvuus-ja-derivoituvuus/V-0} % Jatkuvuuvs ja derivoituvuus
\include{05-jatkuvuus-ja-derivoituvuus/V-1} % Jatkuvuuden käsite
\include{05-jatkuvuus-ja-derivoituvuus/V-2} % Funktion raja-arvo
\include{05-jatkuvuus-ja-derivoituvuus/V-3} % Derivaatta
\include{05-jatkuvuus-ja-derivoituvuus/V-4} % Trigonometristen funktioiden derivointi
\include{05-jatkuvuus-ja-derivoituvuus/V-5} % Ääriarvot. Sileys
\include{05-jatkuvuus-ja-derivoituvuus/V-6} % Differentiaalilaskun väliarvolause
\include{05-jatkuvuus-ja-derivoituvuus/V-7} % Kiintopisteiteraatio
\include{05-jatkuvuus-ja-derivoituvuus/V-8} % Analyyttiset kompleksifunktiot
\include{05-jatkuvuus-ja-derivoituvuus/V-9} % *Jatkuvuuden logiikka

\include{06-eksponenttifunktio/exp-0} %Eksponenttifunktio
\include{06-eksponenttifunktio/exp-1} %Yleinen eksponenttifunktio $E(x)$
\include{06-eksponenttifunktio/exp-2} %Funktiot $e^x$ ja $\ln x$
\include{06-eksponenttifunktio/exp-3} %Kompleksinen eksponenttifunktio
\include{06-eksponenttifunktio/exp-4} %Eksponenttifunktion sovellusesimerkkejä

\include{07-yhden-muuttujan-differentiaalilaskenta/dif-0} %Yhden muuttujan differentiaalilaskenta
\include{07-yhden-muuttujan-differentiaalilaskenta/dif-1} %Differentiaali ja muutosnopeus
\include{07-yhden-muuttujan-differentiaalilaskenta/dif-2} %Käyrän tangentti ja normaali
\include{07-yhden-muuttujan-differentiaalilaskenta/dif-3} %Käyrän kaarevuus
\include{07-yhden-muuttujan-differentiaalilaskenta/dif-4} %Taylorin polynomit ja Taylorin lause
\include{07-yhden-muuttujan-differentiaalilaskenta/dif-5} %Taylorin polynomien sovelluksia
\include{07-yhden-muuttujan-differentiaalilaskenta/dif-6} %Interpolaatiopolynomit

\include{08-integraali/int-0} %Integraali
\include{08-integraali/int-1} %Integraalifunktio
\include{08-integraali/int-2} %Osittaisintegrointi ja sijoitus
\include{08-integraali/int-3} %Osamurtokehitelmät. Sarjamenetelmä
\include{08-integraali/int-4} %Integraalifunktion numeerinen laskeminen. Määrätty integraali
\include{08-integraali/int-5} %Riemannin integraali
\include{08-integraali/int-6} %Analyysin peruslause
\include{08-integraali/int-7} %Riemannin integraalin laajennukset
\include{08-integraali/int-8} %Pinta-ala ja kaarenpituus
\include{08-integraali/int-9} %Numeerinen integrointi

\include{09-differentiaaliyhtalot/dy-0} %Differentiaaliyhtälöt
\include{09-differentiaaliyhtalot/dy-1} %Differentiaaliyhtälöiden peruskäsitteet
\include{09-differentiaaliyhtalot/dy-2} %Separoituva differentiaaliyhtälö
\include{09-differentiaaliyhtalot/dy-3} %Palautuvat toisen kertaluvun DY:t
\include{09-differentiaaliyhtalot/dy-4} %Ensimmäisen kertaluvun lineaarinen DY
\include{09-differentiaaliyhtalot/dy-5} %Lineaariset, vakiokertoimiset DY:t. Eulerin differentiaaliyhtälö
\include{09-differentiaaliyhtalot/dy-6} %*Yleinen toisen kertaluvun lineaarinen DY
\include{09-differentiaaliyhtalot/dy-7} %Diferentiaaliyhtälöiden numeeriset ratkaisumenetelmät
\include{09-differentiaaliyhtalot/dy-8} %*Picardin-Lindelöfin lause

\include{10-matriisit/m-0} %Matriisit
\include{10-matriisit/m-1} %Matriisialgebra
\include{10-matriisit/m-2} %Neliömatriisit. Käänteismatriisi
\include{10-matriisit/m-3} %Gaussin algoritmi
\include{10-matriisit/m-4} %Tuettu Gaussin algoritmi. Singulaariset yhtälöryhmät
\include{10-matriisit/m-5} %Determinantti
\include{10-matriisit/m-6} %Lineaarikuvaukset
\include{10-matriisit/m-7} %Affiinikuvaukset. Geometriset kuvaukset
\include{10-matriisit/m-8} %*Lineaaristen yhtälöryhmien perinteisiä sovellusesimerkkejä

\include{11-usean-muuttujan-differentiaalilaskenta/udif-0} %Usean muuttujan differentiaalilaskenta
\include{11-usean-muuttujan-differentiaalilaskenta/udif-1} %Usean muuttujan jatkuvuus ja raja-arvot
\include{11-usean-muuttujan-differentiaalilaskenta/udif-2} %Osittaisderivaatat
\include{11-usean-muuttujan-differentiaalilaskenta/udif-3} %Gradientti
\include{11-usean-muuttujan-differentiaalilaskenta/udif-4} %Divergenssi ja roottori
\include{11-usean-muuttujan-differentiaalilaskenta/udif-5} %Operaattorit grad, div, rot ja Laplace käyräviivaisissa koordinaatistoissa
\include{11-usean-muuttujan-differentiaalilaskenta/udif-6} %Epälineaariset yhtälöryhmät: Jacobin matriisi ja Newtonin menetelmä
\include{11-usean-muuttujan-differentiaalilaskenta/udif-7} %Kontraktiokuvuauslause. Käänteisfunktiolause. Implisiittifunktiolause
\include{11-usean-muuttujan-differentiaalilaskenta/udif-8} %Usean muuttujan ääriarvotehtävät
\include{11-usean-muuttujan-differentiaalilaskenta/udif-9} %Usean muuttujan Taylorin polynomit

\include{12-matriisin-ominaisarvot-ja-neliomuodot/eig-0} %Matriisin ominaisarvot ja neliömuodot
\include{12-matriisin-ominaisarvot-ja-neliomuodot/eig-1} %Matriisin ominaisarvot
\include{12-matriisin-ominaisarvot-ja-neliomuodot/eig-2} %Neliömuodon diagonalisointi
\include{12-matriisin-ominaisarvot-ja-neliomuodot/eig-3} %Pinnan kaarevuus
\include{12-matriisin-ominaisarvot-ja-neliomuodot/eig-4} %*Tensorit, vektorit ja skalaarit

\include{13-usean-muuttujan-integraalilaskenta/uint-0} %Usean muuttujan integraalilaskenta
\include{13-usean-muuttujan-integraalilaskenta/uint-1} %Pinta-alamitta ja tasointegraalit
\include{13-usean-muuttujan-integraalilaskenta/uint-2} %Tasointegraalien laskutekniikka
\include{13-usean-muuttujan-integraalilaskenta/uint-3} %Avaruusintegraalit
\include{13-usean-muuttujan-integraalilaskenta/uint-4} %Taso-ja avaruusintegraalien muuntaminen
\include{13-usean-muuttujan-integraalilaskenta/uint-5} %Integraalien sovellukset: Tiheys ja kokonaismäärä
\include{13-usean-muuttujan-integraalilaskenta/uint-6} %Viivaintegaalit
\include{13-usean-muuttujan-integraalilaskenta/uint-7} %Pintaintegraalit

\include{14-gaussin-ja-stokesin-lauseet/Gauss-0} %Gaussin ja Stokesin lauseet
\include{14-gaussin-ja-stokesin-lauseet/Gauss-1} %Vektorikentät ja polkuintegraalit
\include{14-gaussin-ja-stokesin-lauseet/Gauss-2} %Gaussin lause
\include{14-gaussin-ja-stokesin-lauseet/Gauss-3} %Gaussin lauseen sovelluksia
\include{14-gaussin-ja-stokesin-lauseet/Gauss-4} %Stokesin lause
\include{14-gaussin-ja-stokesin-lauseet/Gauss-5} %Pyörteetön vektorikenttä

\printindex

\backmatter

%\include{takakansi}

\end{document}
