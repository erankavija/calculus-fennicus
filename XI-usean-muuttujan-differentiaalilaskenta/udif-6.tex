\section[Epälineaariset yhtälöryhmät: Jacobin matriisi \\ ja Newtonin menetelmä]
{Epälineaariset yhtälöryhmät: Jacobin \\ matriisi ja Newtonin menetelmä}
\label{jacobiaani}
\sectionmark{Jacobin matriisi}
\alku

Tarkastellaan yleistä $m$ yhtälön ja $n$ tuntemattoman yhtälöryhmää muotoa
\[
\left\{ \begin{aligned}
&f_1(x_1,\ldots,x_n)\ =0 \\
&\quad\vdots \\
&f_m(x_1,\ldots,x_n)=0
\end{aligned} \right.
\]
missä funktiot $f_i\,,\ i=1 \ldots m$, ovat $n$ reaalimuuttujan reaaliarvoisia funktioita, eli 
funktioita tyyppiä $f_i\,:\ \DF_{f_i} \kohti \R,\ \DF_{f_i}\subset\R^n$. Yhtälöryhmä on 
\index{epzy@epälineaarinen yhtälöryhmä} \index{yhtzy@yhtälöryhmä!b@epälineaarinen}
\kor{epälineaarinen}, jos se ei ole lineaarinen, eli jos ainakin yksi funktioista $f_i$ on 
ei-affiininen kuvaus (vrt.\ Luku \ref{affiinikuvaukset}). Kun määritellään
\[
\mf = (f_1,\ldots,f_m), \quad \DF_\mf = \DF_{f_1}\cap\DF_{f_2}\cap\cdots\cap\DF_{f_m}\,,
\]
niin ym.\ yhtälöryhmän voi kirjoittaa kätevästi vektorimuodossa\,:
\[
\mf(\mx)=\mv{0}.
\]
Tässä siis $\mf$ on funktio (kuvaus) tyyppiä $\,\mf:\ \DF_\mf\kohti\R^m,\ \DF_\mf\subset\R^n$.
\index{funktio A!j@$n$ muuttujan vektoriarvoinen} \index{vektoric@vektoriarvoinen funktio}
\index{vektorimuuttujan funktio}%
Tämä on \kor{vektorimuuttujan vektoriarvoinen} funktio, jonka voi tulkita aiemmin Luvuissa 
\ref{lineaarikuvaukset}--\ref{affiinikuvaukset} määriteltyjen lineaari- ja affiinikuvausten
\index{vektorikenttä}%
yleistykseksi. Myös termiä \kor{vektorikenttä} (vrt.\ Luku \ref{divergenssi ja roottori}) voi
käyttää. Yhtälöryhmiä matriisialgebran keinoin käsiteltäessä tulkitaan $\mf$ tavallisesti
pystyvektoriksi. Kuten nähdään jatkossa, matriisialgebraa tarvitaan muun muassa, kun halutaan
määrittää algoritmisin keinoin ko.\ yhtälöryhmän \kor{ratkaisut}, eli joukko
\[
X=\{\mx\in\DF_\mf \mid \mf(\mx)=\mv{0}\}.
\]
Yhtälöryhmiä käytännössä (numeerisesti) ratkaistaessa on tavallisimmin $m=n$. 
\begin{Exa}
Epälineaaristen yhtälöryhmien
\[ 
\text{a)} \ \left\{ \begin{aligned} 
                    &3x+y+z+9=0 \\ &7x+y+3z+27=0 \\ &x^2+y^2+z^2-81=0 
                    \end{aligned} \right.  \qquad
\text{b)} \ \left\{ \begin{aligned} &x+y+z-16=0 \\ &x^2+y^2+z^2-81=0 \end{aligned} \right.
\]
vektorimuotoisia esitystapoja ovat
\begin{align*}
\text{a)}\ \mf(x,y,z)\,&=\,(\,3x+y+z+9,\ 7x+y+3x+27,\ x^2+y^2+z^2-81\,) \\
                       &=\,(0,0,0), \\
\text{b)}\ \mf(x,y,z)\,&=\,(x+y+z-16,\ x^2+y^2+z^2-81\,) \\
                       &=\,(0,0),
\end{align*}
tai pystyvektoreiden avulla
\begin{align*}
&\text{a)}\ \mf(x,y,z) = \begin{bmatrix} 
                         3x+y+z+9 \\ 7x+y+3z+27 \\ x^2+y^2+z^2-81 
                         \end{bmatrix} 
                       = \begin{bmatrix} 0\\0\\0 \end{bmatrix}, \\[3mm]
&\text{b)}\ \mf(x,y,z) = \begin{bmatrix} 
                         x+y+z-16 \\ x^2+y^2+z^2-81 
                         \end{bmatrix}\,=\,\begin{bmatrix} 0\\0 \end{bmatrix}.
\end{align*}
Algebrallisin (tai b-kohdassa geometrisin) keinoin voidaan todeta ratkaisuiksi
\[
\text{a)}\ X = \{\,(0,0,-9),\,(6,-6,3)\,\}, \quad \text{b)}\ X=\emptyset. \loppu
\]
\end{Exa}

\index{differentioituvuus}%
Vektoriarvoista funktiota $\mf$ sanotaan \kor{differentioituvaksi} pisteessä $\mx\in\DF_\mf$,
jos jokainen komponenttifunktio $f_i$ on ko.\ pisteessä differentioituva (Määritelmän 
\ref{differentioituvuus} mielessä). Jos $\mf$ on differentioituva, niin yhdistämällä 
komponenttifunktioiden differentiaalikehitelmät saadaan tulos
\[
\mf(\mx+\Delta\mx)=\mf(\mx)+\mJ\,\mf(\mx)\Delta\mx+\mr(\mx),
\]
missä $\mJ$ on $\mf$:n \kor{Jacobin}\footnote[2]{Saksalainen matemaatikko \vahv{K.G.J. Jacobi}  
eli vuosina 1804-1851. \index{Jacobi, K. G. J.|av}} \kor{matriisi} (engl. Jacobian)
\index{Jacobin matriisi} \index{differentiaalioperaattori!i@Jacobin matriisi}%
\[
\boxed{\quad \mJ\,\mf(\mx)=\begin{bmatrix} 
                           \dfrac{\ykehys\partial f_1}{\partial x_1}(\mx) & \ldots & 
                           \dfrac{\partial f_1}{\partial x_n}(\mx) \\ \vdots \\ 
                           \dfrac{\partial f_m}{\akehys\partial x_1}(\mx) & \ldots & 
                           \dfrac{\partial f_m}{\partial x_n}(\mx)
                           \end{bmatrix}\quad \text{(Jacobin matriisi)} \quad}
\]
ja jäännösfunktiolle $\mr$ pätee arvio
\[
\abs{\mr(\mx)} = o(\abs{\Delta\mx}), \quad \text{kun}\,\ \abs{\Delta x} \kohti 0.
\]
Tapauksessa $m=1$ Jacobin matriisi on sama kuin transponoitu gradientti, eli
$\mJ f = (\Nabla f)^T$. Tapauksessa $n=1$ Jacobin matriisi puolestaan palautuu
vektoriarvoisen yhden muuttujan funktion (tavalliseksi) derivaataksi.
Tapauksissa $m=2,3$ tämä on ennestään tuttu parametrisen käyrän
derivaattana (Luku \ref{derivaatta geometriassa}). Jacobin matriisi on siis jälleen
derivaattakäsitteen yleistys kahdellakin eri tavalla. Määritelmästä nähdään yleisemmin, että
Jacobin matriisi on esitettävissä joko gradienttien $\nabla f_i$ tai osittaisderivaattojen
$\partial_j\mf$ avulla muodoissa
\[
\mJ\mf \,=\, [\nabla f_1 \ldots \nabla f_m]^T \,=\, [\partial_1\mf \ldots \partial_n\mf].
\]
\begin{Exa} \label{yryhmä-esim}
Kun yhtälöryhmä
\[ \begin{cases}
    \,x^3-2xy^5-x-2 = 0 \\
   -x^3y+2xy^2+y^5+3 = 0
   \end{cases} \]
kirjoitetaan muotoon $\mf(x,y)=\mo$, $\ \mf=[f_1,f_2]^T$, niin Jacobin matriisi on
\[
\mJ\,\mf(x,y) = \begin{bmatrix}
\partial_x f_1 & \partial_y f_1 \\
\partial_x f_2 & \partial_y f_2
\end{bmatrix} \\
= \begin{bmatrix}
3x^2-2y^5-1 & -10xy^4 \\ -3x^2+2y^2 & -x^3+4xy+5y^4
\end{bmatrix}.
\]
Yhtälöryhmän eräs ratkaisu on $(x,y)=(2,1)$. Tässä pisteessä on
\[
\mJ\,\mf(2,1)=\begin{rmatrix} 9 & -20 \\ -10 & 5 \end{rmatrix}. \loppu
\]
\end{Exa}

Jos $\mf$ on differentioituva pisteessä $\ma$, niin approksimaatiota
\[
\mf(\mx) \approx \mf(\ma) + \mJ\mf(\ma)(\mx-\ma)
\]
\index{linearisaatio (funktion)}% \index{funktion approksimointi!ab@linearisaatiolla}%
sanotaan aiempaan tapaan (vrt.\ Luku \ref{derivaatta}) $\mf$:n \kor{linearisaatioksi} pisteessä
$\ma$. (Oikeammin kyse on 'affinisaatiosta'.) Koska approksimaation virhe on suuruusluokkaa
$\ord{\abs{\mx-\ma}}$ (differentioituvuuden määritelmän mukaisesti), niin johtopäätös on\,:
\[
\boxed{\quad\kehys \text{Differentioituva kuvaus}\ 
                 \approx\ \text{affiinikuvaus p\pain{aikallisesti}}. \quad}
\]
\jatko \begin{Exa} (jatko) Pisteen $(-1,1)\,\vastaa\,\ma=[-1,1]^T$ ympäristössä on likimain
\[
\mf(x,y)\,\approx\,\mf(-1,1)+\mJ(-1,1)(\mx-\ma)\,
     =\,\begin{rmatrix} 0\\3 \end{rmatrix}+\begin{rmatrix} 0&10\\-1&2 \end{rmatrix} 
                                           \begin{rmatrix} x+1\\y-1 \end{rmatrix}. \loppu
\]
\end{Exa}

\subsection{Newtonin menetelmä yhtälöryhmille}
\index{Newtonin menetelmä!a@yhtälöryhmälle|vahv}

Tarkastellaan yhtälöryhmää $\mf(\mx)=\mo$, missä $\mf$ on tyyppiä $\mf:\ \R^n\kohti\R^n$.
Newtonin menetelmässä tehdään yhtälöryhmässä linearisoiva approksimaatio pisteessä $\mx_k$,
eli ratkaistaan
\[
\mf(\mx_k)+\mJ(\mx_k)(\mx-\mx_k)=\mo.
\]
Kun ratkaisua merkitään $\mx=\mx_{k+1}$, on tuloksena iteraatiokaava
\[
\mJ(\mx_k)(\mx_{k+1}-\mx_k)=-\mf(\mx_k)
\]
eli
\[
\boxed{\kehys\quad \mx_{k+1}=\mx_k-\inv{\mJ(\mx_k)}\,\mf(\mx_k) \quad \text{(Newton)}. \quad}
\]
Perusajatus tässä on täsmälleen sama kuin yhden yhtälön ($n=1$) tapauksessa (vrt.\ Luku 
\ref{kiintopisteiteraatio}): kyseessä on
\index{kiintopisteiteraatio}%
\kor{kiintopisteiteraatio} muotoa
\[
\mx_{k+1}=\mF(\mx_k), \quad k=0,1,\ldots,
\]
missä $\mF(\mx)=\mx-\mJ(\mx)^{-1}\mf(\mx)$. 

Newtonin menetelmä suppenee myös yhtälöryhmien ratkaisussa samantapaisin oletuksin kuin yhden 
yhtälön tapauksessa. --- Kerrattakoon Luvusta \ref{usean muuttujan jatkuvuus}, että
vektorijonon $\seq{\mx_k}$ suppeneminen kohti vektoria $\mc\in\R^n$ tarkoittaa:
\[
\mx_k \kohti \mc \qekv \abs{\mx_k-\mc} \kohti 0 \quad (k\kohti\infty),
\]
missä $\abs{\cdot}$ on $\R^n$:n euklidinen normi. Kuten tapauksessa $n=1$, Newtonin menetelmän
\index{kvadraattinen!a@suppeneminen}%
suppeneminen on myös (riittävin oletuksin) \kor{kvadraattista} seuraavassa
konvergenssilauseessa lähemmin määriteltävällä tavalla. Lause on todistuksineen varsin
suoraviivainen yleistys Lauseesta \ref{Newtonin konvergenssi}, mutta todistuksen yksityiskohdat
ovat teknisempiä. (Todistus nojaa usean muuttujan Taylorin lauseeseen, ks.\ Luku
\ref{usean muuttujan taylorin polynomit} jäljempänä, vrt.\ myös
Harj.teht.\,\ref{taylorin lause}:\ref{H-dif-4: Newtonin konvergenssi}.) Sivuutetaan todistus.
\begin{Lause} \label{Newtonin konvergenssi -Rn}
Olkoon $\mf:\DF_\mf\kohti\R^n$, $\DF_\mf\subset\R^n$, $\mf=(f_i)$, ja olkoon $\mf(\mc)=\mo$, 
$\mc\in\R^n$. Olkoot edelleen osittaisderivaatat $\partial^\alpha  f_i(\mx)$ olemassa ja 
jatkuvia pisteen $\mc$ ympäristössä, kun $i=1\ldots n$ ja $\abs{\alpha}\leq 2$, ja olkoon 
Jacobin matriisi $\mJ\,\mf(\mc)$ säännöllinen. Tällöin Newtonin iteraatio 
$\mx_{k+1}=\mx_k-\inv{\mJ(\mx_k)}\,\mf(\mx_k)$ suppenee kohti $\mc$:tä, mikäli $\abs{\mx_0-\mc}$
on riittävän pieni, ja suppeneminen on kvadraattista:
\[
\abs{\mx_{k+1}-\mc}\leq C\abs{\mx_k-\mc}^2\quad (C=\text{vakio}).
\]
\end{Lause}
\begin{Exa} \label{Newton-2d: ex1} Etsi Newtonin menetelmällä ratkaisu yhtälöryhmälle 
\[ \begin{cases}
    \,x^3-2xy^5-x=2.04 \\
    -x^3y+2xy^2+y^5=-0.05
\end{cases} \]
pisteen $(-1,1)$ läheltä. 
\end{Exa}
\ratk Tässä on
\[
\mf(x,y)= \begin{bmatrix} x^3-2xy^5-x-2.04 \\ -x^3y+2xy^2+y^5+0.05 \end{bmatrix}
\]
ja Jacobin matriisi $\mJ(x,y)$ on sama kuin Esimerkissä \ref{yryhmä-esim}. Iteraatiokaavan
\[
[x_{k+1},\,y_{k+1}]^T = [x_k,\,y_k]^T -\mJ(x_k,y_k)^{-1}\mf(x_k,y_k)
\]
mukaan laskien saadaan
\begin{align*}
                       (x_0,y_0)\ &=\ (-1,\,1) \\
                       (x_1,y_1)\ &=\ (-0.942,1.004) \\
                       (x_2,y_2)\ &=\ (-0.918940,1.006156) \\
                       (x_3,y_3)\ &=\ (-0.914247,1.006622) \\
                       (x_4,y_4)\ &=\ (-0.914214,1.006633) \\
                       (x_5,y_5)\ &=\ (-0.914216,1.006633) \\
                                  &\vdots \quad \loppu
\end{align*}

Esimerkissä Newtonin iteraatio suppeni nopeasti, koska alkuarvaus oli lähellä ratkaisua. Ellei
hyvää alkuarvausta ole käytettävissä, voidaan ensin koettaa paikallistaa ratkaisut jollakin 
yksinkertaisemmalla menetelmällä. Esimerkiksi jos yhtälöryhmä ei ole kooltaan suuri, voidaan 
kokeilemalla tutkia, missä funktio
\[
F(\mx)=\sum_{i=1}^n [f_i(\mx)]^2
\]
saa pieniä arvoja. Vieläkin suoraviivaisempi on 'yrityksen ja erehdyksen' menetelmä: Iteroidaan
\index{suppenemisallas}%
vaihtelevilla alkuarvauksilla, kunnes sattumalta osutaan nk.\ \kor{suppenemisaltaaseen}, eli 
riittävän lähelle jotakin ratkisua. Mikäli iteraatio ei suppene esimerkiksi kymmenen kierroksen
jälkeen, vaihdetaan alkuarvausta.
\begin{Exa}
Ratkaise yhtälöryhmä
\[
\begin{cases}
\,x^2+y^2= 1 \\ \,e^x+\sin y=1
\end{cases}
\]
\end{Exa}
\ratk Tässä on
\[
\mf(x,y) = \begin{bmatrix} x^2+y^2-1 \\ e^x+\sin y-1 \end{bmatrix}, \qquad
\mJ\,\mf(x,y) = \begin{bmatrix} 2x & 2y \\ e^x & \cos y \end{bmatrix},
\]
joten Newtonin iteraatio saa muodon
\[
\begin{bmatrix} x_{k+1} \\ y_{k+1} \end{bmatrix} =
\begin{bmatrix} x_k \\ y_k \end{bmatrix} - 
\inv{\begin{bmatrix} 2x_k & 2y_k \\ e^{x_k} & \cos y^k \end{bmatrix}}
\begin{bmatrix} x_k^2+y_k^2-1 \\ e^{x_k}+\sin y_k-1\end{bmatrix}.
\]
Ratkaisuja on kaksi:
\begin{align*}
A:\quad &(x,y)=(\phantom{-}0.5538..,\, -0.8327..\,) \\
B:\quad &(x,y)=(-0.8089..,\phantom{-}\,0.5879..\,)
\end{align*}
Seuraavassa tuloksia alkuarvauksista riippuen. Merkintä * tarkoittaa epäonnistumista 
(singulaarinen Jacobin matriisi tai iteraation hajaantuminen).
\[
\begin{array}{lll}
(\phantom{-}1,\phantom{-}0)\kohti A\qquad & 
(\phantom{-}0\phantom{.0},\phantom{-}0)\kohti * \qquad & 
(-\phantom{1}5,\phantom{-}0\phantom{.0})\kohti B \\ 
(\phantom{-}1,\phantom{-}1)\kohti B\qquad & 
(\phantom{-}0.1,\phantom{-}0)\kohti A \qquad & 
(-10,\phantom{-}0\phantom{.0})\kohti B \\
(-1,\phantom{-}1)\kohti B\qquad & 
(-0.1,\phantom{-}0)\kohti * \qquad & 
(\phantom{-1}3,\phantom{-}3\phantom{.0})\kohti B \\
(-1,-1)\kohti B\qquad & 
(\phantom{-}5\phantom{.0},\phantom{-}0)\kohti * \qquad & 
(\phantom{-1}3,-3\phantom{.0})\kohti A \\
(\phantom{-}0,-1)\kohti A\qquad & 
(\phantom{-}0\phantom{.0},\phantom{-}5)\kohti B \qquad & 
(\phantom{-1}3,-1.5)\kohti A \\
(\phantom{-}1,-1)\kohti A\qquad & 
(\phantom{-}0\phantom{.0},-5)\kohti A \qquad & 
(\phantom{-1}3,-1.2)\kohti B \\
\end{array} \loppu
\]

\subsection{Suuret yhtälöryhmät. Jatkamismenettely}
\index{jatkamismenettely (algoritmi)|vahv}

Newtonin menetelmää käytetään yleisesti myös hyvin suurien epälineaaristen yhtälöryhmien
ratkaisuun. Tällaiset ongelmat syntyvät usein osana laajempaa laskentaprosessia, jolloin
on myös tavallista, että hyvä alkuarvaus on tarjolla laskennan aikaisemmista vaiheista.
Näin on  esimerkiksi ratkaistaessa differentiaaliyhtälösysteemeitä numeerisilla
askelmenetelmillä (ks.\ Luku \ref{DYn numeeriset menetelmät}). 

Jos halutaan vain ratkaista yksittäinen suurikokoinen yhtälöryhmä ilman ennakkotietoa
ratkaisun (ratkaisujen) sijainnista, voi toimivan alkuarvauksen löytäminen Newtonin
iteraatiolle olla hyvin vaikeaa. Tällaisessa tilanteessa varsin yleisesti käytetty on nk.
\kor{jatkamismenettely} (engl.\ continuation method), jossa ratkaistava ongelma 
parametrisoidaan muotoon
\[
P(t),\quad 0\leq t<1.
\]
Oletetaan, että $P(1)$ on ongelma, joka varsinaisesti halutaan ratkaista, ja että $P(0)$ on 
helppo ratkaista. Tällöin jatkamismenettelyn ideana on ratkaista peräkkäin
\[
P(t_0),P(t_1),\ldots,P(t_N),\quad t_N=1,
\]
missä (esimerkiksi) $t_k=k\Delta t$. Jos $\Delta t$ on pieni ja parametrisointi suoritettu 
hyvin, on probleeman $P(t_{k-1})$ ratkaisu hyvä alkuarvaus Newtonin iteraatiolle, kun 
ratkaistaan probleemaa $P(t_k)$. Näin voidaan askelittain edetä lopputilaan
$t_N=1$.\footnote[2]{Jatkamismenettelyyn perustuvan algoritmin suunnittelussa on yleensä
apua probleeman fysikaalisesta taustasta. Esimerkiksi rakenteiden mekaniikan ongelmassa, jossa
halutaan simuloida rakenteen suuria muodonmuutoksia kuormituksen alaisena, on luontevaa
menetelllä kuten koejärjestelyssä: Parametrisoidaan kuormituksen voimakkuus, eli lähdetään
kuormittamattomasta tilasta ja lisätään kuormitusta pienin askelin, kunnes päädytään haluttuun
lopputilaan. Parametrin $t$ voi tällöin kuvitella koejärjestelyyn liittyväksi 
aikamuuttujaksi. Aikamuuttujan lisääminen (aikaparametrisointi) on luonnollista
yleisemminkin silloin, kun fysiikasta peräisin olevaan ongelmaan etsitään ajasta riippumatonta
(tasapaino)ratkaisua.}
Ellei parempaa ideaa ole käytettävissä, voi yleisen yhtälöryhmän parametrisoida vaikkapa
muunnoksella muotoa
\[
\mf(\mx)=\mo\ \ext\ \\ \mg(\mx)+t\,[\,\mf(\mx)-\mg(\mx)\,]=\mo,
\]
missä $\mg(\mx)=\mo$ on jokin helposti ratkeava (esim.\ lineaarinen) yhtälöryhmä.
\begin{Exa} Yhtälöryhmälle %[ratk: (x,y)=(5,4)]
\[
\left\{ \begin{aligned}
        &x^3-2y^3+x+2y=10 \\ &x^3+y^3-2xy^2-3x-y=10
        \end{aligned} \right.
\]
antaa Newtonin algoritmi alkuarvauksilla $(x_0,y_0)=(1,1)$ ja $(x_0,y_0)=(-1,-1)$ ratkaisun
$(x,y)=(-2.25872,-2.42834)$. Kokeillaan, löytyykö jokin muu ratkaisu jatkamismenettelyllä:
Parametrisoidaan yhtälöryhmä muotoon
\[
\mf(t,\mx) = \begin{bmatrix}
             t(x^3-2y^3)+x+2y \\ t(x^3+2y^3-2xy^2)-3x-y
             \end{bmatrix}
           = \begin{bmatrix} 10\\10 \end{bmatrix}
\]
ja ratkaistaan tämä parametrin arvoilla $t_k=0.2k,\ k=0 \ldots 5$. Kun $k=0$, on ratkaisu
$(x,y)=(-6,8)$. Tästä eteenpäin ($k=1 \ldots 5$) käytetään edellisellä $k$:n arvolla saatua
ratkaisua alkuarvauksena Newtonin iteraatiolle. Tulos:
\begin{center}
\begin{tabular}{ll}
$t$   & $\text{Ratkaisu}$  \\ \hline \\
$0.2$ & $(10.5223,8.55714)$ \\
$0.4$ & $(7.60605,6.14808)$ \\
$0.6$ & $(6.30833,5.07754)$ \\
$0.8$ & $(5.53177,4.43769)$ \\
$1.0$ & $(5.00000,4.00000)$
\end{tabular}
\end{center}
Ratkaisu $(x,y)=(5,4)$ löytyy myös alkuarvauksia vaihtelemalla. Esim.\ $(x_0,y_0)=(3,3)$ tai
jopa $(x_0,y_0)=(0,0)$ johtavat tähän. \loppu
\end{Exa}

\subsection{Yksinkertaistettu Newtonin iteraatio}
\index{Newtonin menetelmä!a@yhtälöryhmälle|vahv}
 
Epälineearista yhtälöryhmää Newtonin menetelmällä ratkaistaessa on jokaisella 
iteraatiokierroksella suoritettava seuraavat laskuoperaatiot:
\begin{itemize}
\item[-] Funktioevaluaatio: $\ \mx_k\ \map\ \mf(\mx_k)=\my_k$.
\item[-] Jacobin matriisin \kor{päivitys} (engl.\ updating) 
         $\ \mx_k\ \map\ \mJ\mf(\mx_k)=\mA_k$. \index{pzy@päivitys (Jacobin matriisin)}
\item[-] Lineaarisen yhtälöryhmän ratkaisu: 
         $\ \my_k\map\mpu_k=\mA_k^{-1}\my_k\ (\mA_k\mpu_k=\my_k)$.
\item[-] Vähennyslasku: $\ \mx_k\,,\mpu_k\ \map\ \mx_k-\mpu_k=\mx_{k+1}$.
\end{itemize}
Sekä Jacobin matriisin päivittäminen että etenkin lineaarisen yhtälöryhmän ratkaiseminen ovat
suhteellisen raskaita operaatioita, jos yhtälöryhmän koko on suuri. Hyvin suuria yhtälöryhmiä
ratkaistaessa saatetaankin päivittämisessä hieman 'laiskotella', ts.\ päivitystä ei suoriteta
jokaisella iteraatiokierroksella. Tämä helpottaa myös lineaaristen yhtälöryhmien (suoraa) 
ratkaisemista, koska yhtälöryhmien kerroinmatriisi pysyy samana päivityksien välissä. Riittää
silloin laskea Jacobin matriisin $LU$-hajotelma jokaisen päivityksen jälkeen ja pitää se tallessa
seuraavaan päivitykseen asti (vrt.\ Luku \ref{Gaussin algoritmi}). Riippuen siitä, kuinka usein
päivitys suoritetaan, saadaan Newtonin menetelmälle erilaisia variaatioita, joista voidaan 
valita ongelmakohtaisesti tehokkain.

Em.\ variaatioista yksinkertaisimmassa lasketaan Jacobin matriisi vain kerran,
alkuarvauspisteessä. Kun merkitään $\mA=\mJ(\mx_0)$, niin iteraatiokaavaksi tulee
\[
\mx_{k+1}=\mx_k-\mA^{-1}\mf(\mx_k), \quad k=0,1\ldots
\]
%tai algoritmisemmin kirjoitettuna
%\begin{align*}
%&\mx_k\map\mb_k=\mf(\mx_k), \\
%&\mb_k\map\mpu_k\,:\ \mA\mpu_k=\mb_k, \\ 
%&\mx_{k+1}=\mx_k-\mpu_k\,, \quad k=0,1,\ldots
%\end{align*}
Tämä Newtonin menetelmän äärimmäinen yksinkertaistus ei ole aidon Newtonin menetelmän veroinen
käytännössä, sillä se suppenee (sikäli kuin suppenee) vain lineaarisesti 
(vrt.\ Harj.teht.\,\ref{kiintopisteiteraatio}:\,\ref{H-V-7: yksinkertaistettu Newton},
kun $n=1$). --- Sen sijaan teoreettisten tarkastelujen 'ajattelumenetelmäksi' tämä menetelmä
sopii hyvin, ks.\ seuraava luku. 
\begin{Exa} Esimerkin \ref{Newton-2d: ex1} tilanteessa on
\[
\mA=\mJ\mf(-1,1)= \begin{rmatrix} 0&10\\-1&2 \end{rmatrix}, \quad
\mA^{-1}=\frac{1}{10}\begin{rmatrix} 2&-10\\1&0 \end{rmatrix},
\]
joten ym.\ tavalla yksinkertaistetuksi Newtonin iteraatiokaavaksi tulee
\[
\begin{bmatrix} x_{k+1}\\y_{k+1} \end{bmatrix} 
   = \begin{bmatrix} x_k\\y_k \end{bmatrix}
         -\frac{1}{10}\begin{rmatrix} 2&-10\\1&0 \end{rmatrix}
         \left[\begin{array}{l} 
               x_k^3-2x_ky_k^5-x_k-2.04\\-x_k^3y_k+2x_ky_k^2+y_k^5-0.05 
               \end{array}\right].
\]
Iteraatio suppenee, mutta selvästi aitoa Newtonin iteraatiota hitaammin
(vrt.\ Esimerkki \ref{Newton-2d: ex1}):
\begin{align*}
\qquad (x_0,y_0)\ &=\ (-1,\,1) \\
       (x_1,y_1)\ &=\ (-0.942,1.004) \\
       (x_2,y_2)\ &=\ (-0.929318,1.005191) \\
       (x_3,y_3)\ &=\ (-0.923142,1.005780) \\
       (x_4,y_4)\ &=\ (-0.919696,1.006109) \\
       (x_5,y_5)\ &=\ (-0.917651,1.006304) \\
%      (x_6,y_6)\ &=\ (-0.916395,1.006424) \\
%      (x_7,y_7)\ &=\ (-0.915608,1.006500) \\
                  &\ \vdots \loppu
\end{align*}
%Iteraatio suppenee, mutta selvästi aitoa Newtonin iteraatiota hitaammin. \loppu
\end{Exa}

\Harj
\begin{enumerate}

\item
Laske seuraavissa tapauksissa $\mf$:n Jakobin matriisi ensimmäisessä annetussa pisteessä,
approksimoi $\mf$ affiinikuvauksella ja laske tämän avulla $\mf$:n arvo likimäärin toisessa
annetussa pisteessä. Vertaa tarkkaan arvoon. \vspace{1mm}\newline
a) \   $\mf(x,y)=(x^3+y^3,x^3-y^3),\,\ (1,1),\,\ (0.9,1.1)$ \newline
b) \   $\mf(x,y)=(e^x\sin\pi y,e^{-x}\cos\pi y),\,\ (0,1/3),\,\ (-0.04,0.30)$ \newline
c) \,\ $\mf(x,y,z)=(x^2y,x^2z,y^2-z^2),\,\ (1,3,3)\,\ (0.99,,3.02,2.97)$ \newline
d) \   $\mf(x,y,z)=(x^2+yz,y^2-x\ln z),\,\ (2,2,1),\,\ (1.98,2.01,1.03)$ \newline
e) \,\ $\mf(\mx)=(x_1x_2,\,x_1^2-x_3^2,\,x_2^2+x_3x_4,\,x_2x_4+x_4^2),\,\ (1,2,-2,-1),$ \newline 
       \phantom{e \,\ }  $(1.01,2.02,-2.02,-1.01)$
 
\item 
Yhtälöryhmällä
\[
[\mf(x,y,z)]^T \,=\, [\,x^3 - x^2 y + x z^3,\,xyz + z^2,\,x^3 + y^3 + z^3\,] \,=\, \mb^T
\]
on muuan helppo ratkaisu, kun $\mb^T=[1,2,3]$. Approksimoimalla $\mf$ affiinikuvauksella ko.\ 
ratkaisupisteen lähellä laske yhtälöryhmän ratkaisu likimäärin, kun $\mb=[1.04,1.98,3.02]^T$.

\item
Laske koordinaattimuunnoksien $(r,\varphi,z)\map(x,y,z)$ (lieriök. $\ext$ karteesinen) ja 
$(r,\theta,\varphi)\map(x,y,z)$ (pallok.\ $\ext$ karteesinen) Jacobin matriisit 
$\mJ(r,\varphi,z)$ ja $\mJ(r,\theta,\varphi)$. Missä pisteissä $\mJ$ on singulaarinen matriisi?

\item
Olkoon $\mf$ tyyppiä $\mf: \R^n\kohti\R^p$ ja $\mg$ tyyppiä $\mg: \R^p\kohti\R^m$. Johda
yhdistetyn funktion $\mF(\mx)=(\mg\circ\mf)(\mx)=\mg[\mf(\mx)]$ Jacobin matriisin laskusääntö
$\mJ\mF(\mx)=\mJ\mg[\mf(\mx)]\mJ\mf(\mx)$. \kor{Vihje}: Ketjusääntö!

\item
Etsi seuraaville yhtälöryhmille ratkaisu Newtonin menetelmällä lähtien alkuarvauksesta
$(x_0,y_0)=(1,1)$.
\[
\text{a)}\,\ \begin{cases} \,x^4+y^4=2xy^5\\ \,x+x^2+y^4=4 \end{cases} \quad
\text{b)}\,\ \begin{cases} \,x^3+3x=y^4+4y \\ \,\cos\pi x+xye^{-y}=0 \end{cases}
\]

\item
Etsi seuraavien yhtälöryhmien ratkaisut neljän merkitsevän numeron tarkkuudella käyttäen
kaksiulotteista Newtonin menetelmää. Laske a)- ja b)-kohdissa ratkaisu myös vaihtoehtoisella
tavalla, jossa eliminoidaan ensin toinen tuntemattomista ja käytetään 1-ulotteista Newtonin
menetelmää. \vspace{1mm}\newline
a) \ $y-e^x=0,\,\ x-\sin y=0$ \newline
b) \ $x^2+y^2=1,\,\ y=e^x\ $ (kaksi ratkaisua) \newline 
c) \ $y-\sin x=0,\,\ x^2+(y+1)^2=2\ $ (kaksi ratkaisua) \newline
d) \ $x^2-xy+2y^2=10,\,\ x^3y^2=2\ $ (neljä ratkaisua) \newline
e) \ $\sin x+\sin y=1,\,\ x^3=y^2<30\ $ (neljä ratkaisua)

\item 
Selvitä (kaksiulotteista) Newtonin menetelmää käyttäen, missä pisteissä Cartesiuksen lehden
$x^3 + y^3 = 3xy$ tangentin ja jonkin koordinaatiakselin välinen kulma on $45\aste$.

\item
Määritä likimäärin seuraavien yhtälöryhmien ratkaisut eliminoimalla ensin yksi tuntemattomista
ja käyttämällä sen jälkeen Newtonin menetelmää.
\[
\text{a)}\ \ \begin{cases} x^2+y^2+z^2=1\\z=xy\\6xz=1 \end{cases} \
\text{b)}\ \ \begin{cases} x^2+y^2+z^2=1\\y=\sin z\\z+z^2+z^3=x+y \end{cases} \
\text{c)}\ \ \begin{cases} x^2+y^4=1\\z=x^3y\\e^x=2y-z \end{cases}
\]

\item
Yhtälöryhmällä
\[
x+yze^x \,=\, x^3+y^3z \,=\, y+e^{xyz} \,=\, 2.1
\]
on ratkaisu pisteen $(0,1,2)$ lähellä. Iteroi tästä alkuarvauksesta kolme kertaa Newtonin
menetelmän yksinkertaistetulla versiolla, jossa Jacobin matriisia ei päivitetä. Vertaa oikean 
Newtonin iteraation antamiin tuloksiin.

\item (*)
Etsi jatkamismenettelyä ja Newtonin menetelmää käyttäen jokin ratkaisu yhtälöryhmälle
\[
\left\{ \begin{aligned}
        &3x^3-y^3+z^3+x-2z+40=0 \\ 
        &4x^3-3y^3-z^3-2x-3y-40=0 \\ 
        &x^3+y^3+z^3-5x-5y+5z-40=0
        \end{aligned} \right.
\] 
% Eräs atkaisu on (5,6,-6).)

\item (*)
Määritä suora, joka sivuaa kahdessa eri pisteessä käyrää
\[
K:\ x^2y^2-2xy^3+y^4-4x^2y+6xy^2-2y^3+4x^2-3y^2-12x+y=20.
\]
% Käyrän yhtälö: 4x+3y+24=(x-y-1)^2(y-2)^2. S: 4x+3y+24=0. Sivuamispisteet (-3,-4) ja (-7.5,2).

\item (*)
(Avaruuspysäköinti) Tähtien välisen avaruden pisteessä $P=(x,y,z)$ (pituusyksikkö = valovuosi)
on avaruusasema. Avaruusasemaan vaikuttaa lähimpien tähtien vetovoima $\vec G=-k\nabla u$,
missä $k$ on vakio ja
\[
u(x,y,z) = \sum_{i=1}^n \frac{m_i}{|P-P_i|}\,,
\]
missä $m_i$ on tähden $T_i$ massa ja $P_i$ sijainti. Määritä avaruuspysäköintiä varten kaikki
pisteet $P$, joissa $\vec G=\vec 0$, kun $m_i=iM,\ i=1 \ldots n\,$ ($M=$ vakio)
ja \vspace{1mm}\newline
a) \ $n=3,\,\ P_1=(0,0,0),\ P_2=(1,0,0),\ P_3=(0,1,0)$, \newline
b) \ $n=4,\,\ P_1=(0,0,0),\ P_2=(1,0,0),\ P_3=(0,1,0),\ P_4=(0,0,1)$. 
 
\end{enumerate}