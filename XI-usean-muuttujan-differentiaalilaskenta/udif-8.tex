\section{Usean muuttujan ääriarvotehtävät} \label{usean muuttujan ääriarvotehtävät}
\alku
\index{zyzy@ääriarvotehtävä|vahv}

Aloitetaan määritelmästä.
\begin{Def} \label{usean muuttujan ääriarvot}
\index{paikallinen maksimi, minimi, ääriarvo|emph}
\index{maksimi (funktion)!a@paikallinen|emph}
\index{minimi (funktion)!a@paikallinen|emph}
\index{zyzy@ääriarvo (paikallinen)|emph}
\index{laakapiste|emph}
Funktiolla $f:\DF_f\kohti\R$, $D_f\subset\R^n$, on pisteesä $\mc\in\DF_f$
\kor{paikallinen maksimi}, jos jollakin $\delta>0$ pätee
\[
0<\abs{\mx-\mc}<\delta \ \impl \ x\in\DF_f \text{ ja } f(\mx)<f(\mc),
\]
ja \kor{paikallinen minimi}, jos jollakin $\delta>0$ pätee
\[
0<\abs{\mx-\mc}<\delta \ \impl \ x\in\DF_f \text{ ja } f(\mx)>f(\mc).
\]
Jos $f$:llä on pisteessä $\mc$ paikallinen maksimi tai minimi, niin $f$:llä on 
\kor{paikallinen ääriarvo} pisteessä $\mc$. Jos $\mc\in\DF_f$ ei ole funktion $f$ paikallinen
ääriarvokohta, mutta jollakin $\delta>0$ pätee
\begin{align*}
\text{joko}\,:\quad &0<\abs{\mx-\mc}<\delta\ \impl \ x\in\DF_f \text{ ja } f(\mx) \le f(\mc),\\
\text{tai}\,: \quad &0<\abs{\mx-\mc}<\delta\ \impl \ x\in\DF_f \text{ ja } f(\mx) \ge f(\mc),
\end{align*}
niin sanotaan, että $\mc$ on $f$:n \kor{laakapiste}.
\end{Def}
\begin{Exa} \label{udif-8.1} Funktiolla $f(x,y)=ax^2+by^2$ on origossa paikallinen
ääriarvokohta, jos $ab>0$. Kyseessä on paikallinen minimi, jos $a>0$ ja $b>0$, ja paikallinen
maksimi, jos $a<0$ ja $b<0$. Jos on $ab=0$, niin origo on laakapiste. Tapauksessa $ab<0$ ei
origo ole ääriarvokohta eikä laakapiste. \loppu
\end{Exa}
Funktion paikallista ääriarvoa sanotaan myös
\index{suhteellinen ääriarvo}%
\kor{suhteelliseksi}, jolloin funktion maksimia/minimiä koko tarkastelun kohteena olevassa
joukossa (esim.\ koko määrittelyjoukossa) sanotaan
\index{maksimi (funktion)!b@absoluuttinen} \index{minimi (funktion)!b@absoluuttinen}
\index{absoluuttinen maksimi, minimi} \index{globaali ääriarvo}%
\kor{absoluuttiseksi}. Myös termiä \kor{globaali} ääriarvo käytetään, etenkin jos kyse on 
funktion koko määrittelyjoukosta.
\jatko \begin{Exa} (jatko) Jos joukko $A \subset \R^2$ sisältää origon, niin tapauksessa $ab>0$
on $f(0,0)=0$ funktion absoluuttinen minimi/maksimi $A$:ssa. \loppu
\end{Exa}  

Gradientin avulla voidaan helposti paikallistaa reaaliarvoisen funktion paikalliset minimit ja 
maksimit, jos funktio on differentioituva. Nimittäin Luvun \ref{gradientti} tarkastelujen 
perusteella seuraava tulos on varsin ilmeinen:
\begin{Prop} \label{ääriarvopropositio-Rn}
Jos $f:\DF_f\kohti\R$, $\DF_f\subset\R^n$, on differentioituva pisteessä $\mc\in\DF_f$ ja $\mc$
on$f$:n paikallinen ääriarvo- tai laakapiste, niin $\Nabla f(\mc)=\mo$.
\end{Prop}
\index{kriittinen piste}%
Gradientin nollakohtia sanotaan yleisemmin funktion \kor{kriittisiksi pisteiksi}.
\begin{Exa} \label{saari uudelleen}
Etsi funktion $f(x,y)=x^2+2y^2+2xy+4x+14y$ kriittiset pisteet. 
(Vrt. Luku \ref{kahden ja kolmen muuttujan funktiot}, Esimerkki \ref{saari}.)
\end{Exa} 
\ratk 
\[
\nabla f(x,y)=\vec 0 \qekv \begin{cases} 
                           \,\partial_x f(x,y)= 2x+2y+4=0, \\ 
                           \,\partial_y f(x,y)= 2x+4y+14=0 
                           \end{cases}\ \ekv \quad \begin{cases} \,x=3, \\ \,y=-5. \end{cases}
\]
Ainoa kriittinen piste on siis $(3,-5)$. \loppu

Esimerkissä kriittinen piste on funktion absoluuttinen minimi, kuten selvitettiin Luvussa 
\ref{kahden ja kolmen muuttujan funktiot} pelkin algebran keinoin. --- Kriitisen pisteen
ei yleisemmin tarvitse olla paikallinen ääriarvokohta tai laakapiste. Esimerkiksi funktion
\[
f(x,y)=x^2-y^2
\]
\index{satulapiste}%
kriittinen (origo) on nk.\ \kor{satulapiste} (engl. saddle point) --- termi ei kaivanne
lähempää määrittelyä. Funktion kriittisten pisteiden luokittelu ääriarvokohdiksi,
laakapisteiksi tai satulapisteiksi on ongelma, jonka (algebrallisella) ratkaisulla on
yleisempääkin mielenkiintoa. Asiaan palataan hieman myöhemmin. Tässä yhteydessä rajoitutaan
ääriarvotehtäviin, joissa kriittisten pisteiden luokittelulla ei ole keskeistä merkitystä.

\subsection{Rajoitettu optimointi} 
\index{rajoitettu optimointi|vahv}

Sovelluksissa esiintyvät usean muuttujan ääriarvotehtävät voidaan yleensä muotoilla 
\kor{rajoitetun optimoinnin} (engl.\ constrained optimization) ongelmina. Tyypillisessä 
rajoitetun optimoinnin ongelmassa halutaan etsiä reaaliarvoisen funktion
$f(\mx),\ \DF_f\subset\R^n$ minimi tai maksimi joukossa $A$, jonka määritelmä asetetaan
muodossa
\[ 
\mx \in A \qekv g_i(\mx) \ge 0, \quad i=1 \ldots m. 
\]
Sovelluksissa kyse on yleensä jonkin käytännön kannalta tärkeän (kuten taloudellisen hyödyn tai
tappion) maksimointi tai minimointi, eli
\index{optimointi} \index{rajoitusehto (optimoinnin)}%
\kor{optimointi}, tehtävän asettelussa väistämättömien \kor{rajoitusehtojen} voimassa ollessa.
--- Optimointialgoritmien suunnittelun (tai käytön)
kannalta on huomion arvoista, että epäyhtälörajoitteisiin voi sisältyä myös yhtälörajoitteita
muodossa
\[ 
g(\mx)= 0 \qekv \begin{cases} \ \ g(\mx) &\ge 0, \\ -g(\mx) &\ge 0. \end{cases} 
\]
\index{lineaarinen rajoite}%
Rajoitteita sanotaan \kor{lineaarisiksi}, jos funktiot $g_i$ ovat ensimmäisen asteen polynomeja
(affiinikuvauksia). Jos myös $f$ on tätä tyyppiä, on kyse
\index{lineaarinen optimointi!lineaarinen ohjelmointi}%
\kor{lineaarisen ohjelmoinnin} ongelmasta, vrt.\ luku \ref{affiinikuvaukset}.

Tarkastellaan esimerkkinä kahden muuttujan tilannetta, jossa optimointiongelman asettelu on
\[
f(x,y)=\min/\max\,! \quad \text{ehdoilla} \quad g_i(x,y) \ge 0, \quad i=1 \ldots m.
\]
Oletetaan jatkossa, että sekä $f$ että rajoitefunktiot $g_i$ ovat määriteltyjä ja 
differentioituvia $\R^2$:ssa. Ongelman voi purkaa kahteen osaan: Ratkaistaan ensin 
\kor{rajoittamaton} ongelma käymällä läpi $f$:n kaikki kriittiset pisteet. Kriittisistä 
\index{kzyypzy@käypä (kriittinen piste)}%
pisteistä \kor{käypiä} ovat ne, joissa asetetut rajoitusehdot ovat voimassa. Oletetaan, että
näitä on äärellinen määrä, jolloin voidaan suorittaa valinta: Valitaan pisteistä se, jossa $f$
saa pienimmän/suurimman arvonsa. 

Optimointitehtävää ei ole vielä ratkaistu, vaan hankalampi vaihe on vasta edessä: On määrättävä
erikseen funktion minimi/maksimi tarkasteltavan joukon $A$
\index{reuna (joukon, alueen)}%
\kor{reunalla} $\partial A$, joka määritellään
\[ 
(x,y) \in \partial A \qekv g_i(x,y)=0 \quad \text{jollakin}\ i. 
\]
Kuvassa on koko optimointiongelma kuvattuna yhden rajoitusehdon ($m=1$) tapauksessa.
\begin{figure}[H]
\setlength{\unitlength}{1cm}
\begin{center}
\begin{picture}(11,7)(0,-2)
\closecurve(1,0.4,6,1.8,5,4,0.3,3.5)
\put(1.12,2.48){$\bullet$} \put(3.12,1.45){$\bullet$} \put(6,1.8){$\bullet$} 
\put(3.15,-0.84){$\bullet$}
\put(0,-2){$\vec\nabla f=\vec 0$} \put(0.2,-1.5){\vector(1,4){1}} 
\put(0.2,-1.5){\vector(1,1){3}} 
\put(0.2,-1.5){\vector(4,1){3}}
\put(2,2.5){$g(x,y)>0$} \put(6.5,4.4){$g(x,y)=0$}
\put(7.8,1.9){\vector(-1,0){1.6}} 
\put(8,1.7){\parbox{3cm}{$f=\min/\max\, ! \,$, kun $g(x,y)=0$}}
\put(3.5,-0.85){(ei käypä)}
\curve(5,4,5.8,4.5,6,4.5,6.3,4.5)
\end{picture}
\end{center}
\end{figure}

\index{sidottu ääriarvotehtävä} \index{zyzy@ääriarvotehtävä!a@sidottu ääriarvotehtävä}%
Optimointiongelmaa yhden tai useamman yhtälörajoitteen alaisena sanotaan \kor{sidotuksi} 
ääriarvotehtäväksi. Kahden muuttujan tapauksessa ongelma ratkeaa luontevasti, jos oletetaan,
että jokainen rajoitusehto määrittelee parametrisen käyrän muotoa
\[ 
g_i(x,y)=0 \ \ekv \ x=x_i(t), \ y=y_i(t),\ t \in \R, \quad i = 1 \ldots m, 
\]
ja oletetaan vielä, että reuna $\partial A$ voidaan pilkkoa käyränkaariksi
\[ 
\partial A_i = \{(x,y) \in \R^2 \mid x = x_i(t),\ y = y_i(t),\ t \in [a_i,b_i]\}, 
\]
jotka yhdessä peittävät $\partial A$:n ja joiden yhteisiä pisteitä ovat enintään päätepisteet
$(x_i(a_i),y_i(a_i))$ ja $(x_i(b_i),y_i(b_i))$. Optimointiongelman jäljellä oleva osa on näillä
oletuksilla palautettu yhden muuttujan rajoitetuiksi optimointiongelmiksi:
\[ 
F_i(t) = f\bigl(x_i(t),y_i(t)\bigr) = \text{min/max\ !} \quad \text{kun} \quad 
                                      a_i \le t \le b_i\,, \quad i = 1 \ldots m.
\]
Nämä ongelmat ratkeavat normaaliin tapaan, eli etsimällä funktioiden $F_i$ kriittiset pisteet
(= derivaatan nollakohdat) avoimilta väleiltä $(a_i,b_i)$ ja tutkimalla erikseen päätepisteet
$a_i,b_i$. Sikäli kuin funktioilla $F_i$ on käypiä kriittisiä pisteitä äärellinen määrä, on 
löydetty äärellinen pistejoukko, josta optimoitavan funktion ääriarvot välttämättä löytyvät.
\begin{Exa} \label{esim: rajoitettu optimointi} Ratkaise rajoitetut optimointitehtävät
\begin{align*} 
&f(x,y) = 6x^2-4xy+9y^2+16x-22y = \text{min\ !} \quad \text{ja} \quad f(x,y) = \text{max\ !} \\
&\text{kun} \quad x \ge 0,\quad  x^2+(y-1)^2 \le 1.
\end{align*}
\end{Exa}
\ratk Kriittisiä pisteitä on yksi:
\[ 
\nabla f(x,y) = \vec 0 \qekv \begin{cases} 
                             \ 12x-4y+16=0, \\ -4x+18y-22=0 \end{cases} 
                       \qekv \begin{cases} \,x=-1, \\ \,y=1. \end{cases} 
\]
Tämä ei ole käypä, joten siirrytään tutkimaan funktiota tarkasteltavan joukon $A$ reunalla 
$\partial A$:
\[ 
(x,y) \in \partial A \qekv x=0 \quad \text{tai} \quad x^2+(y-1)^2=1. 
\]
Tämä jakautuu kahteen osaan:
\begin{align*}
&\partial A_1\,: \quad x=0\,\ \ja\,\ x^2 + (y-1)^2 \le 1 \qekv x=0\,\ \ja\,\ y \in [0,2]. \\
&\partial A_2\,: \quad x^2 + (y-1)^2 = 1\,\ \ja\,\ x \ge 0.
\end{align*}
Reunan osa $\partial A_1$ on jana, jolla
\[ 
F_1(y) = f(0,y) = 9y^2-22y. 
\]
Funktion $F_1$ kriittinen piste $y=11/9$ on käypä, joten merkitään muistiin $f$:n arvo tässä 
pisteessä sekä janan $\partial A_1$ päätepisteissä:
\[
f(0,11/9) = -121/9 = -13.4444.., \quad f(0,0)=0, \quad f(0,2) = -8. 
\]
Siirrytään reunan osalle $\partial A_2$. Tämä on puoliympyrän kaari, joka parametrisoituu 
muodossa
\[ 
x = \sin t, \quad y=1-\cos t, \quad t \in [0,\pi]. 
\]
Tutkittava funktio on näin ollen
\begin{align*}
 F_2(t) &= f(\sin t,1-\cos t) \\
        &= 6\sin^2t+4\sin t\cos t+9\cos^2 t+12\sin t+4\cos t-13, \quad t \in [0,\pi].
\end{align*}
Funktion $F_2$ kriittiset pisteet löytyvät ratkaisemalla yhtälö
\[ 
F_2'(t)= \ldots = -3\sin 2t + 4\cos 2t + 12\cos t - 4\sin t = 0. 
\]
Numeerisin keinoin saadaan ainoaksi käyväksi ratkaisuksi
\[ 
t = 0.927295 .., \quad\ (x(t),y(t)) = (0.800000..,0.400000..) = (4/5,2/5). 
\]
Tässä pisteessä on $F_2(0.927295..) = 8$. Koska reunan osien $\partial A_1$ ja $\partial A_2$
päätepisteet ovat samat, niin on päätelty, että $f$:n pienin ja suurin arvo löytyvät joukosta
$\{-121/9,\,0,\,-8,\,8\}$. Optimointitehtävien ratkaisut ovat siis
\begin{align*}
&f_{\text{min}} = -121/9 = -13.4444.. \quad 
                               \text{pisteessä}\ (x,y) = (0,11/9) = (0,\,1.1111..), \\
&f_{\text{max}} = 8 \quad \text{pisteessä}\ (x,y) = (4/5,\,2/5) = (0.8,\,0.4). \loppu
\end{align*}

\subsection{Lagrangen kertojien menetelmä}
\index{Lagrangen!c@kertojien menetelmä|vahv}

Edellä esitettyä ratkaisutapaa voi periaatteessa soveltaa useammankin muuttujan rajoitettuun
optimointitehtävään. Ongelmaksi kuitenkin muodostuvat sidotut ääriarvotehtävät, joita joudutaan
ratkaisemaan optimoitaessa kohdefunktiota tarkasteltavan joukon reunalla. Useamman kuin kahden
muuttujan tapauksessa tällaisten tehtävien ratkaiseminen parametrisoimalla, eli vapaiden
muuttujien lukumäärää vähentämällä, on usein hankalaa. Kolmen tai useamman muuttujan sidotuissa 
ääriarvotehtävissä käytetäänkin yleensä toista, paljon elegantimpaa ja yleispätevämpää
menetelmää, jota keksijänsä \hist{J. L. Lagrangen} (1736-1813) mukaan sanotaan 
\kor{Lagrangen kertojien} (engl.\ Lagrange multipliers) \kor{menetelmäksi}. 

Tarkastellaan aluksi jälleen kahden muuttujan tilannetta. Olkoon etsittävä funktion $f(x,y)$ 
paikalliset ääriarvokohdat käyrällä $S$, jonka määrittelee sidosehto $g(x,y)=0$. Olkoon 
$P=(x_0,y_0) \in S$ eräs tällainen ääriarvokohta ja olkoon $\vec t\,$ $S$:n tangenttivektori
pisteessä $P$. Tällöin on oltava
\begin{multicols}{2} \raggedcolumns
\[
\vec t\cdot\nabla f(x_0,y_0)=0.
\]
\begin{figure}[H]
\setlength{\unitlength}{1cm}
\begin{center}
\begin{picture}(4,2)
\curve(0,0,1,0.5,3.5,1.2)
\put(0.9,0.4){$\bullet$}
\put(1,0.5){\vector(3,1){2}}
\put(2.8,1.3){$\vec t$}
\put(0.9,0){$P$}
\put(3.7,1.15){$g(x,y)=0$}
\end{picture}
\end{center}
\end{figure}
\end{multicols}
Toisaalta koska käyrä $g(x,y)=0$ funktion $g$ tasa-arvokäyrä, niin pätee myös 
(vrt.\ Luku \ref{gradientti})
\[
\vec t\cdot\nabla g(x_0,y_0)=0.
\]
Näin ollen, sikäli kuin $\nabla g(x_0,y_0)\neq\vec 0$ (mikä jatkossa oletetaan --- pisteet, 
joissa $g(x_0,y_0)=0$ ja $\nabla g(x_0,y_0)=\vec 0$, on tutkittava erikseen), on oltava jollakin
$\lambda\in\R$
\[
\nabla f(x_0,y_0)+\lambda\nabla g(x_0,y_0)=\vec 0.
\]
Kysyttyjä paikallisia ääriarvopisteitä voidaan siis hakea ratkaisemalla yhtälöryhmä
\[
\begin{cases}
\,f_x(x,y)+\lambda g_x(x,y)=0, \\
\,f_y(x,y)+\lambda g_y(x,y)=0, \\
\,g(x,y)=0.
\end{cases}
\]
Tässä on kolme tuntematonta ja kolme yhtälöä, joten ainakin muodollinen puoli on kunnossa.

Em.\ yhtälöryhmään päädytään helpommin muistettavalla tavalla, kun määritellään funktio
\[
F(x,y,\lambda)=f(x,y)+\lambda g(x,y).
\]
Tässä on siis rajoitusehdossa esiintyvä funktio $g$ tuotu mukaan kertoimella $\lambda$ ---
\kor{Lagrangen kertojalla} --- painotettuna. Ym.\ yhtälöryhmään päädytään nyt yksinkertaisesti,
kun haetaan $F$:n kriittiset pisteet ehdoista
\[
\begin{cases}
\,\partial_x F(x,y,\lambda)=0, \\
\,\partial_y F(x,y,\lambda)=0, \\
\,\partial_\lambda F(x,y,\lambda)=0.
\end{cases}
\]
Em.\ geometrinen perustelu toimii myös kolmen muuttujan tapauksessa. Nimittäin jos 
$P=(x_0,y_0,z_0)$ on funktion $f(x,y,z)$ paikallinen ääriarvopiste pinnalla $S:\,g(x,y,z)=0$,
niin jokaiselle pinnan tangenttivektorille $\vec t$ pisteessä $P$ on oltava voimassa
\[
\vec t\cdot\nabla f(x_0,y_0,z_0)=0.
\]
Tällöin $\nabla f(x_0,y_0,z_0)$ on pinnan $S$ normaali, joten jollakin $\lambda\in\R$
\[
\nabla f(x_0,y_0,z_0)+\lambda\nabla g(x_0,y_0,z_0)=\vec 0
\]
(olettaen, että $\nabla g(x_0,y_0,z_0)\neq\vec 0\,$). Ääriarvopisteet löytyvät tällöin funktion
\[
F(x,y,z,\lambda)=f(x,y,z)+\lambda g(x,y,z)
\]
kriittisten pisteiden joukosta.
\begin{Exa}
Hae funktion $f(x,y,z)=xy+yz-xz$ ääriarvot yksikköpallon pinnalla.
\end{Exa}
\ratk Rajoitusehto on $x^2+y^2+z^2=1$, joten haetaan funktion
\[
F(x,y,z,\lambda)=xy+yz-xz+\lambda(x^2+y^2+z^2-1)
\]
kriittisiä pisteitä:
\[
\begin{cases}
\,\partial_x F =y-z+2\lambda x = 0, \\
\,\partial_y F =x+z+2\lambda y = 0, \\
\,\partial_z F =y-x+2\lambda z = 0, \\
\,\partial_\lambda F =x^2+y^2+z^2-1 = 0.
\end{cases}
\]
Kolme ensimmäistä yhtälöä on matriisimuodossa
\[
\begin{bmatrix} 2\lambda & 1 & -1 \\ 1 & 2\lambda & 1 \\ -1 & 1 & 2\lambda \end{bmatrix}
\begin{bmatrix} x \\ y \\ z \end{bmatrix}=\mo.
\]
Tälle haetaan ei-triviaalia ratkaisua, joten kerroinmatriisin on oltava singulaarinen. 
Kerroinmatriisin determinantti on
\[
\det(\mA)=8\lambda^3-6\lambda-2=2(\lambda-1)(2\lambda+1)^2,
\]
joten on oltava joko $\lambda=1$ tai $\lambda=-1/2$. Kun $\lambda=1$, tulee yhtälöryhmäksi
\[
\begin{rmatrix} 2 & 1 & -1 \\ 1 & 2 & 1 \\ -1 & 1 & 2 \end{rmatrix}
\begin{bmatrix} x \\ y \\ z \end{bmatrix}=\mo \ \underset{\text{(Gauss)}}{\ekv} \ 
\begin{rmatrix} 2 & 1 & -1 \\ 0 & \frac{3}{2} & \frac{3}{2} \\ 0 & 0 & 0 \end{rmatrix}
\begin{bmatrix} x \\ y \\ z \end{bmatrix}=\mo.
\]
Tämän ratkaisu yhdessä rajoitusehdon $x^2+y^2+z^2=1$ kanssa on
\[
(x,y,z)=t(1,-1,1),\quad t=\pm 1/\sqrt{3}\,,
\]
ja funktion $f$ arvo näissä pisteissä on
\[
f\left(\pm\frac{1}{\sqrt{3}},\mp\frac{1}{\sqrt{3}},\pm\frac{1}{\sqrt{3}}\right)
=\underline{\underline{-1}}.
\]
Jos $\lambda=-1/2$, niin yhtälöryhmäksi tulee
\[
\begin{rmatrix} -1 & 1 & -1 \\ 1 & -1 & 1 \\ -1 & 1 & -1 \end{rmatrix}
\begin{bmatrix} x \\ y \\ z \end{bmatrix}=\mo \ \ekv \
\begin{rmatrix} -1 & 1 & -1 \\ 0 & 0 & 0 \\ 0 & 0 & 0 \end{rmatrix}
\begin{bmatrix} x \\ y \\ z \end{bmatrix}=\mo.
\]
Rajoitusehdon toteuttavia ratkaisuja ovat $\,(x,y,z)=(t-s,t,s)$, missä $t,s\in\R$
toteuttavat
\[
(t-s)^2+t^2+s^2=1 \ \ekv \ 2(t^2+s^2-ts)=1.
\]
Funktion $f$ arvoksi näissä pisteissä tulee
\begin{align*}
f(x,y,z) &= (t-s)t+ts-(t-s)s \\
&= t^2+s^2-ts=\underline{\underline{1/2}}.
\end{align*}
Siis
\begin{alignat*}{2}
f_{\min} &= -1  \quad & &\text{pisteissä } \ \pm\frac{1}{\sqrt{3}}(1,-1,1), \\
f_{\max} &= 1/2 \quad & 
         &\text{käyrällä}\ S:\,(x,y,z)=(t-s,t,s),\ \ (t,s)\in\R^2,\ t^2+s^2-ts=1/2.
\end{alignat*}
Maksimipisteet muodostavat yksikköpallon isoympyrän tasolla $T: x-y+z=0$. \loppu

Lagrangen kertojien menetelmä toimii, poikkeustilanteita lukuunottamatta, myös useamman 
rajoitusehdon tapauksessa. Jos on haettava funktion $f(x_1,\ldots,x_n)$ ääriarvot 
rajoitusehdoilla
\[
g_i(x_1,\ldots,x_n)=0,\quad i=1\ldots m<n,
\]
niin nämä löydetään funktion
\[
F(\mx,\boldsymbol{\lambda})=f(\mx)+\sum_{i=1}^m \lambda_i g_i(\mx)
\]
kriittisten pisteiden joukosta. Esimerkiksi tapauksessa $n=3$, $m=2$ on tutkittava funktiota
\[
F(x,y,z,\lambda,\mu)=f(x,y,z)+\lambda g_1(x,y,z)+\mu g_2(x,y,z).
\]
Menetelmän idean voi tässäkin tapauksessa perustella geometrisesti: Rajoitusehtojen voi olettaa
määrittelevän avaruuskäyrän $K$, joka on pintojen
\[
S_1: \ g_1(x,y,z)=0,\quad S_2: \ g_2(x,y,z)=0
\]
leikkaus. Jos pintojen normaalit käyrän pisteessä $(x_0,y_0,z_0)$ ovat
\[
\vec n_i=\nabla g_i(x_0,y_0,z_0),\quad i=1,2,
\]
niin käyrän $K$ tangentti $\vec t$ on näitä kumpaakin vastaan kohtisuora. Ääriarvoehto on samaa
muotoa kuin yhden rajoitusehdon tapauksessa, eli on oltava
\[
\vec t\cdot\nabla f(x_0,y_0,z_0)=0.
\]
Koska tässä $\vec t$ siis on kohtisuorassa vektoreita $\vec n_1$ ja $\vec n_2$ vastaan, niin
sikäli kuin $\vec n_1$ ja $\vec n_2$ ovat lineaarisesti riippumattomat, on oltava
$\nabla f(x_0,y_0,z_0)=\lambda\vec n_1+\mu\vec n_2$ jollakin $(\lambda,\mu)\in\R^2$ eli
\[
\nabla f(x_0,y_0,z_0)=\lambda\nabla g_1(x_0,y_0,z_0)+\mu\nabla g_2(x_0,y_0,z_0).
\]
\begin{Exa}
Minimoi $\,f(x,y,z)=x+2y-z\,$ ehdoilla $x^2+y^2+z^2=12$ ja $x+y+z=1$. 
\end{Exa}
\ratk Etsitään funktion 
\[ 
F(x,y,z,\lambda,\mu)=x+2y-z+\lambda(x^2+y^2+z^2-12)+\mu(x+y+z-1)
\]
kriittiset pisteet:
\[
\begin{cases}
 \,2\lambda x+\mu = -1 \\ 
 \,2\lambda y+\mu = -2 \\ 
 \,2\lambda z+\mu = 1 \\
 \,z^2+y^2+z^2 = 12 \\ 
 \,x+y+z = 1
\end{cases} 
\impl \ \ \begin{cases} \,2\lambda(x+y+z)+3\mu=-2 \\ \,x+y+z=1 \end{cases}
\impl \ \ 2\lambda=-3\mu-2.
\]
Yhtälöryhmän kolmesta ensimmäisestä ja neljännestä yhtälöstä seuraa myös
\[ 
4\lambda^2(x^2+y^2+z^2)=(\mu+1)^2+(\mu+2)^2+(\mu-1)^2=48\lambda^2 . 
\]
Tässä on edellisen tuloksen perusteella $48\lambda^2 = 12(3\mu+2)^2$, joten seuraa
\begin{align*}
105\mu^2+140\mu+42=0\,\ 
                 &\impl\quad\ (\mu,\lambda)=\begin{cases}
                                         (-0.877485, \ \phantom{-}0.316228), \\
                                         (-0.455848, \ -0.316228)
                                         \end{cases} \\
                 &\impl\,\ (x,y,z)=\begin{cases}
                                   (-0.19371, \ -1.77485, \ \phantom{-}2.96856), \\
                                   (\phantom{-}0.86038, \ \phantom{-}2.44152, \ -2.30190).
                                   \end{cases}
\end{align*}
Näistä ensimmäinen piste antaa minimiarvon $f_{\min}=\underline{\underline{-6.71197}}$.
\loppu

Lagrangen kertojien menetelmää käytettäessä ei syntyvän epälineaarisen yhtälöryhmän
ratkaiseminen luonnollisesti aina onnistu käsinlaskulla. Ratkaisut etsitään silloin
numeerisin keinoin, tavallisimmin Newtonin menetelmällä. Numeeriseen ratkaisemiseen
yhdistettynä Lagrangen kertojien menetelmä näyttääkin itse asiassa parhaat puolensa, 
sillä ratkaisualgoritmista tulee tällä tavoin hyvin suoraviivainen ja helposti 
(konevoimin) toteutettava. 
\begin{Exa} Esimerkissä \ref{esim: rajoitettu optimointi} ratkaistiin sidosehdon 
parametrisoinnilla ääriarvotehtävä
\begin{align*}
&f(x,y)=6x^2-4xy+9y^2+16x-22y = \text{max !} \\
&\text{ehdolla} \quad g(x,y)=x^2+(y-1)^2-1=0.
\end{align*}
Lagrangen kertojaa käytettäessä etsitään funktion
\[
F(x,y,\lambda) = 6x^2-4xy+9y^2+16x-22y + \lambda[x^2+(y-1)^2-1]
\]
kriittiset pisteet eli ratkaistaan yhtälöryhmä
\[
\begin{cases}
\,6x-2y+x\lambda-8=0, \\ -2x+9y+(y-1)\lambda-11=0, \\ \,x^2+(y-1)^2-1=0.
\end{cases}
\]
Kun ratkaisussa käytetään Newtonin menetelmää, saadaan iteraatiokaavaksi
\[
\begin{bmatrix} x_{k+1} \\ y_{k+1} \\ \lambda_{k+1} \end{bmatrix}
= \begin{bmatrix} x_k \\ y_k \\ \lambda_k \end{bmatrix}
- \begin{bmatrix} 
  \lambda_k+6 & -2 & x_k \\ -2 & \lambda_k+9 & y_k-1 \\ 2x_k & 2y_k-2 & 0
  \end{bmatrix}^{-1}
  \begin{bmatrix}
  6x_k-2y_k+x_k\lambda_k-8 \\ -2x_k+9y_k+(y_k-1)\lambda_k-11 \\ x_k^2+(y_k-1)^2-1
  \end{bmatrix}.
\]
Alkuarvauksella $(x_0,y_0,\lambda_0)=(1,0.5,0)$ saadaan
\begin{center}
\begin{tabular}{llll}
$k$ & $x_k$      & $y_k$      & $\lambda_k$  \\ \hline \\
$0$ & $1.000000$ & $0.500000$ & $\phantom{-1}0.0000$ \\
$1$ & $1.000000$ & $0.750000$ & $-12.5000$ \\
$2$ & $0.815287$ & $0.136146$ & $-14.9283$ \\
$3$ & $0.818006$ & $0.376563$ & $-14.8590$ \\
$4$ & $0.800355$ & $0.399765$ & $-14.9999$ \\
$5$ & $0.800000$ & $0.400000$ & $-15.0000$
\end{tabular}
\end{center}
\end{Exa}
Suppeneminen oli hieman onnekasta, sillä alkuarvaus osoittautui Lagrangen kertojan osalta
kehnoksi. Parempi alkuarvaus olisi voitu laskea esim.\ seuraavasti (yleisempi menettely: ks.\
Harj.teht.\,\ref{H-udif-8: lambdan alkuarvaus}):
\begin{align*}
&\vec a=\nabla f(x_0,y_0)=26\vec i-17\vec j, \quad \vec b=\nabla g(x_0,y_0)=2\vec i-\vec j, \\
&\vec a+\lambda\vec b \approx \vec 0\,\ \impl\,\ 
\lambda \approx -\frac{\vec a\cdot\vec b}{|\vec b\,|^2} = -13.8 = \lambda_0. \loppu
\end{align*}

\Harj
\begin{enumerate}

\item
Määritä funktion suurin ja pienin arvo annetuilla rajoitusehdoilla. Käytä
sidosehdoissa (mikäli niitä on) parametrisointia. \vspace{1mm}\newline
a) \ $x-x^2+y^2, \quad 0 \le x \le 2,\,\ 0 \le y \le 1$ \newline
b) \ $xy-2x, \quad -1 \le x \le 1,\,\ 0 \le y \le 1$ \newline
c) \ $xy-x^3y^2, \quad 0 \le x,y \le 1$ \newline
d) \ $xy(a-2x-y), \quad \abs{x} \le a,\,\ \abs{y} \le a$ \newline
e) \ $xy(1-x-2y), \quad x,y \ge 0,\,\ x+y \le 1$ \newline 
f) \ $x^2-2y^2-xy-x, \quad x,y \ge -1,\,\ x+y \le 1$ \newline
g) \ $xy-x^2, \quad x^2+y^2 \le 1$ \newline
h) \ $3+x-x^2-y^2, \quad 2x^2+y^2 \le 1$ \newline
i) \ $y^2-2x^2, \quad x^2-y \le 1,\,\ x+y \le 1$ \newline 
j) \ $xy+x-y, \quad x^2+y^2 \le 1,\,\ x+y+1 \ge 0,\,\ x-y-1 \le 0$ \newline
k) \ $x^3+3y^3, \quad 0 \le xy \le 1,\,\ x^2+y^2 \le 4$ \newline
l) \ $\sin x+\sin y-\sin(x+y), \quad 0 \le x,y \le \pi$ \newline
m) \ $\sin x\cos y, \quad x,y \ge 0,\,\ x+y \le 2\pi$ \newline
n) \ $\sin x\sin y\sin(x+y), \quad x,y \ge 0,\,\ x+y \le \pi$ \newline
o) \ $(x-3y)e^y, \quad \abs{x} \le 1,\,\ \abs{y} \le 1$ \newline
p) \ $xe^{-x^2-y^2}, \quad x^2+y^2 \le R^2$ \newline
q) \ $(x+2y)e^{-x^2-y^2}, \quad (x,y)\in\R^2$ \newline
r) \ $xy^2e^{-xy}, \quad x,y \ge 0$ \newline
s) \ $(2x-y)(1+x^2+y^2)^ {-1}, \quad (x,y)\in\R^2$ \newline
t) \ $(1+2x+3y)(1+x^2+y^2)^{-1}, \quad (x,y)\in\R^2$ \newline
u) \ $xy^2+yz^2, \quad x^2+y^2+z^2 \le 1$ \newline
v) \ $xy+yz, \quad x^2+y^2+z^2 \le 1$ \newline
x) \ $xy^2z^3, \quad x^2+y^2+z^2 \le 1$ \newline
y) \ $xyze^{-x^2-2y^2-3z^2}, \quad (x,y,z)\in\R^3$ \newline
z) \ $(x+y+z)e^{-x^2-2y^2-3z^2}, \quad (x,y,z)\in\R^3$ \newline
å) \ $(x+2y+3z+4u)e^{-x^2y^2z^2u^2}, \quad (x,y,z,u)\in\R^4$

\item
Halutaan määrätä funktion $f(x,y,z)=xy+z^2$ minimi- ja maksimiarvo pallopinnalla
$S:\,x^2+y^2+z^2=1$. Ratkaise tehtävä \ a) parametrisoimalla $S$ pallonpintakoordinaateilla, \
b) Lagrangen kertojien avulla. \ c) Mitkä ovat $f$:n minimi- ja maksimiarvot ehdolla
$x^2+y^2+z^2 \le 1$\,?

\item
Määritä Lagrangen kertojien avulla funktion
\[
\text{a)}\ \ f(x,y,z)=x^2+y^2+z^2 \qquad \text{b)}\ \ f(x,y,z)=xy
\]
pienin ja suurin arvo pallon $S_1:\,x^2+y^2+z^2+4x+4y=0$ ja lieriön $S_2:\, x^2+y^2+2x+2y=6$ 
leikkauskäyrällä.

\item
Ratkaise seuraavat sidotut ääriarvotehtävät Lagrangen kertojien avulla. \vspace{1mm}\newline
a) \ $x^2+y^2=$ min!/max! käyrällä $S:\,x^2+8xy+7y^2=45$ \newline
b) \ $x^2+y^2=$ min!/max! käyrällä $S:\,x^4+x^2y^2+5y^4=95$ \newline
c) \ $x^3y^5=$ max! ehdolla $x+y=8$ \newline
d) \ $x+2y-3x=$ min!/max! pallopinnalla $x^2+y^2+z^2=1$ \newline
e) \ $x^2+y^2+z^2=$ min! ehdolla $xyz^2=2$ \newline
f) \ $xyz=$ min!/max! pallopinnalla $x^2+y^2+z^2=12$ \newline
g) \ $x=$ min!/max! käyrällä $S:\,z=x+y\,\ja\,x^2+2y^2+2z^2=8$ \newline
h) \ $x^2+y^2+z^2=$ min!/max! käyrällä $S:\,z^2=x^2+y^2\,\ja\,x-2z=3$ \newline
i) \ $z=$ min!/max! käyrällä $S:\,x^2+y^2=8\,\ja\,x+y+z=1$ \newline
j) \ $x^3+y^3+z^3=$ min!/max! käyrällä $S:\,x+y+z=3\,\ja\,x^2+y^2+z^2=27$ \newline
k) \ $\ma^T\mx=$ min!/max! ehdolla $\abs{\mx}=1\,\ (\ma,\mx\in\R^n)$ \newline
l) \ $x_1^{2m} + \ldots + x_n^{2m}=$ min! ehdolla $x_1 + \ldots + x_n=a \neq 0\,\ (m\in\N)$

\item
Ratkaise Lagrangen kertojaa käyttäen: \vspace{1mm}\newline
a) \ \ Mikä on origon lyhin etäisyys käyrästä $S:\,x^2+8xy+7y^2=45\,$? \newline
b) \   Mikä pinnan $S:\,xy+2yz+3xz=0$ piste $Q$ on lähinnä pistettä 
       $\phantom{\text{b) \ \ }} P=(-2,2,-2)\,$?

\item
Tetraedrin yksi kärki on origossa ja muut kolme kärkeä ovat origosta suuntiin $\vec i$,
$\vec i-\vec j+\vec k$ ja $-\vec i+\vec j+\vec k$. Lisäksi origon etäisyys tetraedrin
vastakkaisesta sivutahkosta $=3$. Mikä on tetraedrin pienin tilavuus näillä ehdoilla?

\item
a) Näytä, että suorakulmaisista särmöistä, joilla on sama sivutahkojen yhteenlaskettu
pinta-ala, kuutio on tilavuudeltaan suurin. \vspace{1mm}\newline
b) Kanneton suorakulmaisen särmiön muotoinen laatikko valmistetaan kahdesta eri materiaalista
siten, että pohjamateriaali on (pintayksikköä kohti) viisi kertaa kalliimpaa kuin laatikon
seinämien materiaali. Miten laatikon mitat on valittava, kun halutaan minimoida 
materiaalikustannukset ehdolla, että laatikon tilavuus $=a^3$\,?

\item
a) Ellipsoidin muotoinen jalokivi
\[
A: \quad \frac{x^2}{a^2}+\frac{y^2}{b^2}+\frac{z^2}{c^2}\,\le\,1
\]
halutaan hioa suorakulmaisen särmiön muotoiseksi. Miten särmiön mitat on valittava, jotta
kiveä menisi hukkaan mahdollisimman vähän? \vspace{1mm}\newline
b) Näytä, että jos $\alpha,\beta$ ja $\gamma$ ovat kolmion kulmat, niin
\[
\sin\frac{\alpha}{2}\sin\frac{\beta}{2}\sin\frac{\gamma}{2}\,\le\,\frac{1}{8}\,.
\]

\item
Laske (tarvittaessa numeerisin keinoin) funktion $f(x,y)=(x-y^2)e^{xy}$ pienin ja suurin
arvo joukossa \ a) $A:\,x^2+y^2 \le 4$, \ b) $A:\,x^4+y^4 \le 16$.

\item (*) \index{aritmeettinen keskiarvo} \index{geometrinen keskiarvo}
          \index{keskiarvo!b@aritmeettinen, geometrinen}
Todista väittämä: Lukujen $a_k \ge 0,\ k=1 \ldots n\,$ \kor{geometrinen keskiarvo} on enintään
yhtä suuri kuin \kor{aritmeettinen keskiarvo}, ts.\ 
\[
\sqrt[n]{a_1 \cdots a_n}\,\le\,\frac{1}{n}(a_1+ \ldots + a_n).
\]
\kor{Vihje}: Aseta sopiva sidosehto!

\item (*)
Etsi funktion $f(x,y,z)=(x+2y)e^{xz}+xyz+y^3$ ääriarvokohdat pallopinnalla
$S:\,(x-3)^2+(y+3)^2+(z+4)^2=16$. Menetelmä: Lagrangen kertojat ja Newton!

\item (*) \label{H-udif-8: lambdan alkuarvaus}
Yleisessä Lagrangen kertojien menetelmässä etsitään funktion
$F(\mx,\boldsymbol{\lambda})=f(\mx)+\sum_{i=1}^m \lambda_i g_i(\mx),\ \mx\in\R^n,\ m<n$
kriittistä pistettä $(\mc,\boldsymbol{\mu})$. Oletetaan, että eräässä $\mc$:n ympäristössä
$U_\delta(\mc)$ vektorit $\Nabla g_i(\mx),\ i=1 \ldots m\,$ ovat lineaarisesti riippumattomat
ja että on käytettävissä alkuarvaus $\mx_0 \in U_\delta(\mc)$. Näytä, että jos alkuarvaus
$\boldsymbol{\lambda}_0$ määrätään $\mx_0$:n avulla laskemalla
\[
|\Nabla f(\mx_0)-\sum_{i=1}^m\lambda_i\Nabla g_i(\mx_0)|^2 = \text{min!} 
                \ \ \impl\ \ (\lambda_i)=\boldsymbol{\lambda}_0,
\]
($|\cdot| = \R^n$:n euklidinen normi), niin pätee:
$\mx_0=\mc\ \impl\ \boldsymbol{\lambda}_0=\boldsymbol{\mu}$. Millaiseen 
$\boldsymbol{\lambda}_0$:n laskukaavaan tämä menettely johtaa tapauksessa $m=1$\,? Sovella
menettelyä sidottuun ääriarvotehtävään
\begin{align*}
&f(x,y,z)=x^2+y^2+z^2 = \text{min!} \\
&\text{ehdoilla} \quad x^3y+xy^3+2=0, \quad x^3y+y^3z+xz^3+1=0,
\end{align*}
kun alkuarvaus on $(x_0,y_0,z_0)=(1,-1,1)$. Määritä $F$:n kriittinen piste likimäärin yhdellä
Newtonin iteraation askeleella.

\end{enumerate}
