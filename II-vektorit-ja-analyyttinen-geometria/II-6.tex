\section{Avaruuden vektorit. Ristitulo} \label{ristitulo}
\sectionmark{Avaruuden vektorit}
\alku
\index{vektoria@vektori (geometrinen)!b@avaruuden \Ekolme|vahv}


Avaruusgeometrisissa tarkasteluissa klassisena lähtökohtana on taaskin
\index{euklidinen!ab@pisteavaruus \Ekolme} \index{pisteavaruus}%
\kor{euklidinen pisteavaruus}, tällä kertaa nimeltään $\Ekolme$, 'E kolme'. Seuraavat
avaruudelliset käsitteet oletetaan jatkossa (myös myöhemmissä luvuissa) tunnetuiksi:
\begin{itemize}
\item piste, (avaruus)jana, \kor{avaruuskolmio}
\item \kor{avaruussuora}, \kor{avaruustaso}, avaruuden puolisuora = \kor{avaruussuunta}
\item \kor{tetraedri}, \kor{suuntaissärmiö}, \kor{monitahokas}
\item \kor{suorakulmainen särmiö}
\item suorakulmaisen särmiön, suuntaissärmiön ja tetraedrin \kor{tilavuus}
\end{itemize}
\index{laskuoperaatiot!cc@avaruusvektoreiden|(}
\kor{Avaruusvektorin} määrittelyn lähtökohtana on euklidisen avaruusgeometrian perusoletus
(tai perusaksioomien seuraus), että geometria jokaisella $\Ekolme$:n avaruustasolla on sama kuin
$\Ekaksi$:ssa. Tällöin voidaan ensinnäkin mitata avaruusjanan pituus (avaruustasolla, joka 
sisältää janan). Kun pituuteen liitetään avaruussuunta, tulee määritellyksi avaruusvektori 
suuntajanana. Vektori on siis jälleen yhdistetty tieto pituudesta (= vektorin itseisarvo) ja
suunnasta.

Vektorin kertominen skalaarilla (reaaliluvulla) määritellään kuten tasossa, ts.\ skaalataan
vektorin pituus ja joko säilytetään suunta tai vaihdetaan se vastakkaiseksi, riippuen kertojan
etumerkistä. Vektorien yhteenlaskun määrittelemiseksi olkoon $O$ $\Ekolme$:n referenssipiste = 
\index{origo}%
\kor{origo} ja $\vec a=\Vect{OA}$ ja $\vec b=\Vect{OB}$ kaksi avaruuden vektoria. Tällöin on
olemassa avaruustaso $T$, joka kulkee pisteiden $O,A,B$ kautta (täsmälleen yksi, jos pisteet 
eivät ole samalla avaruussuoralla). Kun siirrytään tasolle $T$, voidaan $\vec a+\vec b$ 
määritellä $T$:n kolmiodiagrammilla, eli avaruuskolmion avulla. Myös avaruusvektoreiden
\index{skalaaritulo!b@avaruusvektoreiden}%
\kor{skalaaritulo} määritellään kuten tasossa:
\[
\vec a \cdot \vec b = \abs{\vec a} \abs{\vec b} \cos \kulma(\vec a, \vec b).
\]
Tässä geometrinen konstruktio $\kulma(\vec a, \vec b)\map\cos\kulma(\vec a,\vec b)$ toimii
tasolla $T$ oletuksen mukaan kuten $\Ekaksi$:ssa, vrt.\ Luku \ref{skalaaritulo}. Skalaaritulolle
ovat myös voimassa samat lait kuin tasossa. Näistä kuitenkin osittelulaki
\[
\vec a \cdot (\vec b + \vec c\,) = \vec a \cdot \vec b + \vec a \cdot \vec c
\]
kaipaa lisäperusteluja, sillä kolmea avaruuden vektoria ei yleisesti voi sijoittaa samaan
tasoon.
\begin{figure}[H]
\begin{center}
\import{kuvat/}{kuvaII-10.pstex_t}
\end{center}
\end{figure}
Kuviossa $C$ ei ole yleisesti samassa tasossa kuin pisteet $O,A,B$. Suorat $l$ ja $l'$ kuitenkin
ovat yhdensuuntaiset (suuntavektori = $\vec a$), jolloin ne voidaan leikata kohtisuorasti
pisteiden $B$ ja $C$ kautta kulkevilla avaruustasoilla (yhdensuuntaiset katkoviivat kuvassa).
Leikkauspisteistä ja pisteistä $O,A,B$ muodostettavat avaruuskolmiot $OB'B$ ja $OC'C$
(ks.\ kuvio) ovat tällöin suorakulmaiset. Kun nyt skalaaritulon tasogeometrista määritelmää 
sovelletaan tasoilla, joihin ko.\ kolmiot sisältyvät, ja huomioidaan, että avaruuden 
suorakulmiossa $BB'C'C''$ sivujanat $BC''$ ja $B'C'$ ovat yhtä pitkät, niin seuraa 
(vrt.\ Luku \ref{skalaaritulo})
\begin{align*}
\vec a \cdot \vec b + \vec a \cdot \vec c 
       &= \abs{\vec a}\left(\abs{\overrightarrow{OB'}} + \abs{\overrightarrow{BC''}}\right) \\
       &= \abs{\vec a}\left(\abs{\overrightarrow{OB'}} + \abs{\overrightarrow{B'C'}}\right)
          =\abs{\vec a}\abs{\overrightarrow{OC'}} = \vec a \cdot (\vec b + \vec c).
\end{align*}
Siis osittelulaki on pätevä.
\index{laskuoperaatiot!cc@avaruusvektoreiden|)}

Yhteenlaskun, skalaarilla kertomisen ja skalaaritulon tultua määritellyksi joukossa 
$V=\{\text{avaruuden vektorit}\}$ on $(V,\R)$ jälleen vektoriavaruus ja skalaaritulolla 
(sisätulolla) $\vec u,\vec v \map \vec u\cdot\vec v$ varustettuna sisätuloavaruus, 
vrt.\ Luvut \ref{skalaaritulo}--\ref{abstrakti skalaaritulo}. 
Jos $\vec a$, $\vec b$ ja $\vec c$ ovat kolme
avaruuden vektoria, jotka poikkeavat $\vec 0$:sta eivätkä ole saman avaruustason suuntaiset,
niin voidaan päätellä avaruusgeometrisesti (vrt.\ vastaava tasogeometrinen päättely Luvussa
\ref{tasonvektorit}), että jokainen $\vec v \in V$ voidaan esittää yksikäsitteisesti muodossa
$\vec v=x\vec a+y\vec b+z\vec c$, missä $x,y,z\in\R$. Mainitut ehdot vektoreille 
$\vec a,\vec b,\vec c$ voidaan pelkistää ehdoksi
\index{lineaarinen riippumattomuus}%
\[
\boxed{\kehys\quad x\vec a+y\vec b+z\vec c=\vec 0 \qimpl x=y=z=0. \quad}
\]
Sanotaan tällöin, että vektorit $\vec a,\vec b,\vec c$ ovat \kor{lineaarisesti riippumattomat},
ja että vektorisysteemi $\{\vec a,\vec b,\vec c\}$ on $V$:n
\index{kanta}%
\kor{kanta}. Siis avaruusvektoreista muodostuva vektoriavaruus on $3$-ulotteinen: dim $V=3$.

Kun jokainen avaruuden vektori $\vec v \in V$ lausutaan annettujen kantavektoreiden 
lineaariyhdistelynä muodossa $\vec v = x\vec a+y\vec b+z\vec c$, niin syntyy kääntäen 
yksikäsitteinen vastaavuus $V \leftrightarrow \Rkolme$, missä $\Rkolme$ ('R kolme') on 
reaalikukukolmikkojen joukko:
\[
\Rkolme\ =\ \{ (x,y,z) \ | \ x \in \R, \ y \in \R, \ z \in \R \}\ =\ \R \times \R \times \R.
\]
\index{vektorib@vektori (algebrallinen)!b@$\R^3$:n}
\index{laskuoperaatiot!cd@vektoriavaruuden $(\R^3,\R)$}%
Myös $\Rkolme$ on vektoriavaruus, jossa vektorien laskuoperaatiot määritellään 
(vrt.\ $\Rkaksi$:n operaatiot)
\begin{align*}
(x_1,y_1,z_1)+(x_2,y_2,z_2)\ &=\ (x_1+x_2,\,y_1+y_2,\,z_1+z_2), \\
             \lambda(x,y,z)\ &=\ (\lambda x,\lambda y,\lambda z) \quad (\lambda\in\R).
\end{align*}

Avaruusvektoreiden muodostaman vektoriavaruuden
\index{kanta!a@ortonormeerattu}%
\kor{ortonormeerattu kanta} on kolmen vektorin 
systeemi $\{\vec i, \vec j, \vec k\}$, joka toteuttaa
\[
\abs{\vec i} = \abs{\vec j} = \abs{\vec k} = 1, \quad 
\vec i \cdot \vec j = \vec j \cdot \vec k = \vec i \cdot \vec k = 0.
\]
Lisäksi oletetaan yleensä, että $\{\vec i, \vec j, \vec k\}$ on nk.
\index{oikeakätinen (vektori)systeemi}%
\kor{oikeakätinen 
systeemi}.\footnote[2]{Oikeakätinen vektorisysteemi muuttuu \kor{vasenkätiseksi}
(ja vastaavasti vasenkätinen oikeakätiseksi), jos yhden vektorin suunta vaihdetaan
vastakkaiseksi. \index{vasenkätinen vektorisysteemi|av}} Tällä tarkoitetaan oikeaan käteen
liittyvää (funktio)vastaavuutta
\[
(\,\text{peukalo},\,\text{etusormi},\,\text{keskisormi}\,)\ \vast\ (\vec i,\vec j,\vec k)
\]
tai vaihtoehtoisesti
\[
(\,\text{peukalo},\,\text{etusormi},\,\text{keskisormi}\,)\ \vast\ (\vec k,\vec i,\vec j).
\] 
\begin{figure}[H]
\begin{center}
\import{kuvat/}{kuvaII-11.pstex_t}
\end{center}
\end{figure}
Em.\ oletuksin sanotaan $\Ekolme$:n koordinaatistoa $\{O,\vec i,\vec j,\vec k\}$
\index{koordinaatisto!b@karteesinen}% 
\kor{karteesiseksi}. Suoria ja tasoja, joilla koordinaateista kaksi (suora) tai yksi (taso)
\index{koordinaattiakseli} \index{koordinaattiakseli!b@--taso}%
saa arvon $0$, sanotaan \kor{koordinaattiakseleiksi} ja \kor{koordinaattitasoiksi}.
Nämä nimetään ko.\ suorilla tai tasoilla muuttuvien koordinaattien mukaan, esim.\
$x$-akseli, $xy$-taso.

Avaruuden vektoreiden skalaarituloa vastaa $\Rkolme$:n
\index{euklidinen!c@skalaaritulo} \index{skalaaritulo!c@$\R^n$:n euklidinen}%
\kor{euklidinen skalaaritulo}
(vrt.\ Määritelmä \ref{R2:n euklidinen skalaaritulo ja normi})
\[
\mpu_1 = (x_1,y_1,z_1), \ \mpu_2 = (x_2,y_2,z_2)\,: \quad
                                    \mpu_1 \cdot \mpu_2 = x_1x_2 + y_1y_2 + z_1z_2.
\]
Tälle ovat voimassa skalaaritulon aksioomat (ks.\ Luku \ref{abstrakti skalaaritulo}), joten
pätee myös Cauchyn--Schwarzin epäyhtälö (Lause \ref{schwarzR})
\[
\abs{\mpu_1\cdot\mpu_2} \le \abs{\mpu_1}\abs{\mpu_2},
\]
eli
\[
\abs{x_1x_2 + y_1y_2 + z_1z_2} \leq (x_1^2+y_1^2+z_1^2)^{1/2}(x_2^2+y_2^2+z_2^2)^{1/2},
\]
missä
\[
\abs{\mpu}=(x^2+y^2+z^2)^{1/2}, \quad \mpu=(x,y,z)\in\Rkolme
\]
on $\Rkolme$:n
\index{normi!euklidinen} \index{euklidinen!b@normi}%
\kor{euklidinen normi}.
\begin{Exa} Avaruuskolmion kärjet ovat $A=(-1,1,2)$, $B=(4,-2,3)$ ja $C=(-3,3,3)$. Laske kolmion
kulmien mitat asteina (yhden desimaalin tarkkuus).
\end{Exa}
\ratk Jos $\vec a=\Vect{AB}$ ja $\vec b=\Vect{AC}$, niin skalaaritulon määritelmän mukaan
\[
\cos\kulma BAC = \frac{\vec a\cdot\vec b}{\abs{\vec a}\abs{\vec b}}\,.
\]
Tässä on $\vec a=5\vec i-3\vec j+\vec k$ ja $\vec b=-2\vec i+2\vec j+\vec k$, joten saadaan
\[
\cos\kulma BAC\ 
  =\ \frac{5 \cdot (-2) + (-3) \cdot 2 +1 \cdot 1}{(5^2+3^2+1^2)^{1/2}(2^2+2^2+1^2)^{1/2}}\
  =\ -\sqrt{\frac{5}{7}}\,.
\]
Vastaavalla tavalla laskien saadaan
\[
\cos\kulma ABC\ =\ \sqrt{\frac{250}{259}}\,, \qquad
\cos\kulma ACB\ =\ \sqrt{\frac{32}{37}}\,.
\]
Laskimen avulla saadaan vastaukseksi: 
\[
\kulma BAC \approx 147.7\aste, \quad \kulma ABC \approx 10.7\aste, \quad
\kulma ACB \approx 21.6\aste. \loppu
\]

\subsection{Vektorien ristitulo}
\index{laskuoperaatiot!cc@avaruusvektoreiden|vahv}

Avaruuden vektoreille on määritelty skalaaritulon ohella toinen kertolaskun luonteinen
operaatio, jota sanotaan \kor{ristituloksi} tai \kor{vektorituloksi} (engl. cross product,
vector product). Jos $V=\{\text{avaruuden vektorit}\}$, niin ristitulo on kuvaus (funktio)
tyyppiä  $V \times V \rightarrow V$, ts. tulos on vektori (tästä nimitys vektoritulo).
Ristitulo merkitään $\vec a \times \vec b$, luetaan '$a$ risti $b$'.
\begin{Def} (\vahv{Ristitulo}) \label{ristitulon määritelmä}
\index{ristitulo (vektoritulo)|emph} \index{vektoritulo (ristitulo)|emph}
Avaruusvektorien $\,\vec a, \vec b\,$ \kor{ristitulo} eli \kor{vektoritulo} on vektori 
$\vec a \times \vec b$, joka toteuttaa
\begin{enumerate}
\item $\abs{\vec a \times \vec b} = \abs{\vec a}\abs{\vec b}\sin{\kulma(\vec a, \vec b)}$
\item $\vec a \times \vec b \ \perp \ \vec a \,\ \ja \,\ \vec a \times \vec b \ \perp \ \vec b$
\item $\{\vec a,\ \vec b,\ a \times \vec b\} \ \text{on oikeakätinen systeemi}$.
\end{enumerate}
\end{Def}
Säännössä (1) esiintyvä kulman sini tulkitaan ei-negatiiviseksi, ts.\
\[
\sin\kulma(\vec a,\vec b) = \sin\alpha,
\]
missä $\alpha$ on \pain{sisäkulman} mittaluku ($0\le\alpha\le\pi$). Sääntö (2) jättää
ristitulon suunnalle kaksi vaihtoehtoa, joista valinta suoritetaan säännön (3) ilmaisemalla
oikean käden säännöllä
\[
(\,\text{peukalo},\,\text{etusormi},\,\text{keskisormi}\,) 
                       \quad \map \quad (\vec a,\ \vec b,\ \vec a \times \vec b).
\]
\begin{figure}[H]
\begin{center}
\import{kuvat/}{kuvaII-14.pstex_t}
\end{center}
\end{figure}
Määritelmästä \ref{ristitulon määritelmä} seuraa vektoritulolle 'vino' vaihdantalaki
\begin{equation} \label{cross1}
\boxed{\kehys\quad (\vec b \times \vec a) = - (\vec a \times \vec b). \quad}
\end{equation}
Erityisesti on
\[
\vec a \times \vec a = \vec 0
\]
ja yleisemmin
\[
\vec a \times \vec b = \vec 0 \qekv \vec a \parallel \vec b \ \tai \ \vec a = \vec 0 \ 
                                                              \tai \ \vec b = \vec 0.
\]
(Tässä $\vec a \parallel \vec b$ tarkoittaa: $\vec a \uparrow \uparrow \vec b$ tai 
$\vec a \uparrow \downarrow \vec b$.) Ristitulosta (jollei muuten) voidaan siis päätellä, ovatko
kaksi nollavektorista poikkeavaa vektoria yhdensuuntaiset.

Vektoritulo ei ole liitännäinen:
\[
(\vec a \times \vec b) \times \vec c\ \neq\ \vec a \times (\vec b \times \vec c).
\]
Esim.\ jos $\vec a=\vec i$, $\vec b=\vec i$ ja $\vec c=\vec j$, on vasen puoli $=\vec 0$,
mutta oikea puoli $=-\vec j$.

Jos $\lambda \in \R$ ja $\lambda \geq 0$, niin ristitulon määritelmästä seuraa
\[
(\lambda\vec a) \times \vec b \,=\, \vec a \times (\lambda\vec b) 
                              \,=\, \lambda(\vec a \times \vec b).
\] 
Suuntaussäännöstä seuraa myös että jos $\vec a$:n tai $\vec b$:n suunta vaihdetaan, niin vaihtuu
myös $(\vec a \times \vec b)$:n suunta, ts.
\[
(-\vec a) \times \vec b \,=\, \vec a \times (-\vec b) \,=\, - (\vec a \times \vec b).
\]
Yhdistämällä nämä tulokset todetaan, että ristitulon ja skalaarilla kertomisen voi yhdistää 
normaaleilla osittelulaeilla:
\begin{equation} \label{cross2}
\boxed{\kehys\quad (\lambda \vec a) \times \vec b \,=\, \vec a \times (\lambda \vec b) 
                     \,=\, \lambda (\vec a \times \vec b), \quad \lambda \in \R. \quad}
\end{equation}
Ristitulon ja vektorien yhteenlaskun välillä toimivat myös normaalit osittelulait:
\begin{equation} \label{cross3} \boxed{ \begin{aligned}
\ykehys\quad \vec a \times (\vec b + \vec c) 
                           &= \vec a \times \vec b + \vec a \times \vec c, \\
\akehys (\vec a + \vec b) \times \vec c      
                           &= \vec a \times \vec c + \vec b \times \vec c. \quad
\end{aligned} } \end{equation}
Nämä lait eivät kuitenkaan ole määritelmän perusteella ilmeisiä, vaan tarvitaan hieman 
geometrista erittelyä: Olkoon $T$ taso, jonka normaalivektori $=\vec a$. Tällöin 
$\vec a \times \vec b$ voidaan ymmärtää kolmivaiheisena geometrisena operaationa:
\begin{itemize}
\item[1.] Projisioidaan vektori $\vec b$ tasolle $T$ (\kor{ortogonaaliprojektio}).
          Tulos $\vec b_T$. \index{ortogonaaliprojektio}
\item[2.] Suoritetaan vektorin $\vec b_T$ \kor{kierto} tasolla $T$ suunnasta $\vec a$ katsottuna
          vastapäivään kulman $\pi/2$ verran (vektorin pituus säilyy).
          \index{kierto!a@geom.\ kuvaus}
\item[3.] Skaalataan vaiheen 2 tulos kertomalla luvulla $\abs{\vec a}$.
\end{itemize}
Vaiheita 1--2 on havainnollistettu kahdella kuvalla alla. Ensimmäisessä kuvassa tarkkailija on
suunnassa $\vec a \times \vec b$ pisteestä $P$, toisessa suunnassa $\vec a$. Kuviin on merkitty
myös toisen kuvan tarkkailusuunta.
\begin{figure}[H]
\setlength{\unitlength}{1cm}
\begin{center}
\begin{picture}(12,8.2)(0,1)
\path(0,3)(5,3)
\put(3.9,2.9){$\bullet$}
\put(4,3){\arc{1}{-2.35}{-1.59}}
\put(4,3){\vector(-1,1){3}}
\put(4,3){\vector(0,1){4}}
\put(4,3){\vector(-1,0){3}}
\put(10,3){\vector(-1,1){2}}
\put(10,3){\vector(1,1){2}}
%\put(10,3){\arc{1.5}{-2.3561}{-0.7854}}
\put(10.5,3.5){\line(-1,1){0.15}} \put(10.07,3.71){\line(5,-1){0.2}}
\put(9.5,3.5){\line(1,1){0.15}} \put(9.93,3.71){\line(-5,-1){0.2}}
\put(9.5,3.5){\line(0,1){0.12}} \put(9.5,3.5){\line(1,0){0.12}} 
\put(4.2,6.7){$\vec a$}
\put(1.2,3.2){$\vec b_T$}
\put(1.2,6){$\vec b$}
\put(3.6,3.6){$\alpha$}
\put(4.1,2.5){$P$}
\put(5.2,2.9){$T$}
\dashline{0.2}(1,6)(1,3) \put(1,3){\line(1,2){0.1}} \put(1,3){\line(-1,2){0.1}}
\put(9.9,2.9){$\bullet$}
\put(10.1,2.5){$P$}
\put(11.8,4.3){$\vec b_T$}
\put(8.4,4.8){$\vec a\times\vec b$}
%\put(10,3){\arc{1}{-2.35}{-0.8}}
%\put(9.76,3.41){\vector(-3,-2){0.1}}
\path(10.15,3.15)(10,3.3)(9.85,3.15)
\put(3.87,7.5){$\Downarrow$}
\put(2.7,8.2){\pain{Kuvakulma (b)}}
\put(3.87,-2.5){\begin{turn}{45}
\put(3.87,7.5){$\Downarrow$}
\put(2.7,8.2){\pain{Kuvakulma (a)}}
\end{turn}}
\put(0,1){\parbox{5cm}
         {\begin{center}(a) Tarkkailija suunnassa $\vec a\times\vec b$\end{center}}}
\put(6,1.2){\parbox{8cm}{\begin{center}(b) Tarkkailija suunnassa $\vec a$\end{center}}}
\end{picture}
\end{center}
\end{figure}
Ensimmäisessä kuvassa näkyy projektio-operaatio $\vec b \map \vec b_T$. Kierto-operaation
(vaihe 2) tulos osoittaa katsojan suuntaan. Toisessa kuvassa ei näy vektoria $\vec a$, ja 
$\vec b$:stä nähdään vain sen projektio $\vec b_T$. Kierto-operaatio näkyy tässä kuvassa.

Olkoon nyt $\,\vec b, \vec c\,$ avaruusvektoreita. Tällöin yhteenlaskudiagrammi
$\,\vec b + \vec c = \vec d\,$ nähdään alla olevan kuvion mukaisena em.\ 
tarkkailutilanteessa (b)\,:
\begin{figure}[H]
\setlength{\unitlength}{1cm}
\begin{center}
\begin{picture}(6,3)(0,0.5)
\put(0,0){\vector(4,1){6}} 
\put(0,0){\vector(1,1){3}}
\put(3,3){\vector(2,-1){3}}
\put(2.1,2.6){$\vec b_T$}
\put(5.5,1.9){$\vec c_T$}
\put(5.5,0.8){$\vec d_T$}
\end{picture}
\end{center}
\end{figure}
Kuvassa nähdään vain oikean yhteenlaskudiagrammin (kolmio avaruudessa) projektio tasolle $T$. 
Kuva kuitenkin kertoo, että projisoitu diagrammi on tason vektoreiden yhteenlaskudiagrammi, ts.
$\,\vec d_T = \vec b_T + \vec c_T\,$ eli
\[
\text{\kor{projektio}}(\vec b + \vec c) 
                    = \text{\kor{projektio}}(\vec b) + \text{\kor{projektio}}(\vec c).
\]
Toisaalta kierto vaiheessa 2 merkitsee vain koko projisoidun yhteenlaskudiagrammin kiertoa,
jolloin
\[
\text{\kor{kierto}}(\vec b_T + \vec c_T) 
                 = \text{\kor{kierto}}(\vec b_T) + \text{\kor{kierto}}(\vec c_T).
\]
Yhdistämällä nämä kaksi tulosta seuraa:
\begin{align*}
\text{\kor{kierto}}\,[\text{\kor{projektio}}(\vec b + \vec c\,)]
&= \text{\kor{kierto}}\,[\text{\kor{projektio}}(\vec b\,) + \text{\kor{projektio}}(\vec c\,)] \\
&= \text{\kor{kierto}}\,[\text{\kor{projektio}}(\vec b\,)] 
                                   + \text{\kor{kierto}}\,[\text{\kor{projektio}}(\vec c\,)].
\end{align*}
Kun tässä molemmat puolet kerrotaan luvulla $\abs{\vec a}$ ja todetaan, että
\[
\abs{\vec a}\, \text{\kor{kierto}}\,[\text{\kor{projektio}}(\Vect{\rule{0mm}{1.5mm}\ldots})] 
                                     \,=\, \vec a \times (\Vect{\rule{0mm}{1.5mm}\ldots}),
\]
niin on todistettu osittelulain \eqref{cross3} edellinen osa. Jälkimmäinen osa seuraa tästä
sekä vaihdantalaista \eqref{cross1}\,:
\[
(\vec a + \vec b) \times \vec c \,=\, -\vec c \times(\vec a + \vec b)
                                \,=\, -\vec c \times \vec a - \vec c \times \vec b
                                \,=\, \vec a \times \vec c + \vec b \times \vec c.
\]

\subsection{Ristitulon determinanttikaava}

Ristitulo lasketaan käytännössä ortonormeeratussa kannassa $\{\vec i,\vec j,\vec k\}$. Koska
\begin{align*}
&\vec i \times \vec i = \vec j \times \vec j = \vec k \times \vec k = \vec 0, \\
&\vec i \times \vec j = \vec k, \quad \vec j \times \vec k 
                      = \vec i, \quad \vec k \times \vec i = \vec j, \\
&\vec j \times \vec i = -\vec k, \quad \vec k \times \vec j 
                      = -\vec i, \quad \vec i \times \vec k = -\vec j,
\end{align*}
niin sääntöjen \eqref{cross2}--\eqref{cross3} perusteella
\begin{align*}
\vec a \times \vec b &= (x_1\vec i + y_1\vec j + z_1\vec k) 
                                            \times (x_2\vec i + y_2\vec j + z_2\vec k) \\
                     &= (y_1z_2-y_2z_1)\vec i - (x_1z_2-x_2z_1)\vec j + (x_1y_2-x_2y_1)\vec k.
\end{align*}
Ilmaistaan tämä taulukkomuotoisella muistisäännöllä
\[
\vec a \times \vec b = \left| \begin{array}{ccc}
\vec i & \vec j & \vec k \\
x_1 & y_1 & z_1 \\
x_2 & y_2 & z_2 
\end{array} \right|.
\]
Tässä siis toinen ja kolmas vaakarivi muodostuvat $\vec a$:n ja $\vec b$:n koordinaateista.
Kyse on nk.\ 
\index{determinantti|(}%
\kor{kolmirivisestä determinantista}, joka purkautuu ensin säännöllä
\[
\vec a \times \vec b = \left|\begin{array}{cc} 
y_1 & z_1 \\
y_2 & z_2
\end{array} \right| \vec i -
\left|\begin{array}{cc} 
x_1 & z_1 \\
x_2 & z_2
\end{array} \right| \vec j +
\left|\begin{array}{cc} 
x_1 & y_1 \\
x_2 & y_2
\end{array} \right| \vec k,
\]
missä uudet taulukko-oliot ovat \index{determinantti|)}%
\kor{kaksirivisiä determinantteja}.\footnote[2]{Determinanttioppi on käsinlaskussa
ennen paljon käytetty, oma erikoinen matematiikan lajinsa. Tähän yhteyteen lainataan 
determinanttiopista vain ristitulon käsittelyssä tarvittavat muistisäännöt.} Nämä on saatu 
alkuperäisestä kolmirivisestä poistamalla ne rivit ja sarakkeet, joiden yhtymäkohdassa on 
kertoimena oleva $\vec i$, $\vec j$ tai $\vec k$ --- siis ensimmäinen rivi sekä sarake, 
johon ko.\ vektori sisältyy --- huom.\ myös merkin vaihto kaavan keskimmäisessä termissä!
Kaksiriviset determinantit puretaan lopulta säännöllä
\[
\left|\begin{array}{cc} 
a_1 & b_1 \\
a_2 & b_2
\end{array} \right| = a_1b_2 - a_2b_1.
\]
\begin{Exa} Jos $\vec a = 3\vec i -6\vec j + 9\vec k$, $\vec b = 2\vec i + 5\vec j - 3\vec k$, 
$\vec c = -\vec i + 2\vec j - 3\vec k$, niin
\begin{align*}
\vec a \times \vec b &= \left| \begin{array}{rrr} \vec i & \vec j & \vec k \\ 3 & -6 & 9 \\ 
                                                          2 & 5 & -3 \end{array} \right| \\
                     &= [(-6)\cdot(-3)-5\cdot 9]\vec i - [3\cdot(-3)-2\cdot 9]\vec j 
                                                       + [3\cdot 5 -2\cdot(-6)]\vec k \\
                     &= -27\vec i -27\vec j + 27\vec k, \\
\vec a \times \vec c &= \left|\begin{array}{rrr} \vec i & \vec j & \vec k \\ 3&-6&9 \\
                                                -1&2&-3 \end{array}\right| = \vec 0. 
\end{align*}
Jälkimmäinen tulos merkitsee, että vektorit $\vec a$ ja $\vec c$ ovat yhdensuuntaiset (itse 
asiassa vastakkaissuuntaiset). \loppu
\end{Exa}

\subsection{Skalaarikolmitulo $\vec a\times\vec b\cdot\vec c$}
\index{skalaarikolmitulo|vahv}
\index{laskuoperaatiot!cc@avaruusvektoreiden|vahv}

Kolmen avaruusvektorin \kor{skalaarikolmitulo} määritellään
\[
\vec a \times \vec b \cdot \vec c \,=\, (\vec a \times \vec b\,) \cdot \vec c\,.
\]
Tässä on $\vec a \times \vec b$ laskettava ensin, joten sulkeita ei tarvitse merkitä. 
Karteesisessa koordinaatistossa komponenttimuodossa laskettuna on 
\[
\vec a \times \vec b \cdot \vec c = \vec c \cdot (\vec a \times \vec b\,) = 
\left| \begin{array}{ccc}
x_3 & y_3 & z_3 \\
x_1 & y_1 & z_1 \\
x_2 & y_2 & z_2 
\end{array} \right|,
\]
missä $\,\vec a \leftrightarrow (x_1,y_1,z_1)$, $\,\vec b \leftrightarrow (x_2,y_2,z_2)$ ja 
$\,\vec c \leftrightarrow (x_3,y_3,z_3)$. Determinantin määritelmästä seuraa, että rivit voidaan
'kierrättää' siten, että ensimmäinen rivi joutuu pohjimmaiseksi determinantin arvon muuttumatta:
\[
\left| \begin{array}{ccc}
x_3 & y_3 & z_3 \\
x_1 & y_1 & z_1 \\
x_2 & y_2 & z_2 
\end{array} \right| =
\left| \begin{array}{ccc}
x_1 & y_1 & z_1 \\
x_2 & y_2 & z_2 \\
x_3 & y_3 & z_3 
\end{array} \right|.
\]
Näin ollen pätee 'pisteen ja ristin vaihtosääntö'
\[
\boxed{\kehys\quad \vec a \times \vec b \cdot \vec c = \vec a \cdot \vec b \times \vec c. \quad}
\]

\subsection{Kolmion pinta-ala. Tetraedrin tilavuus}
\index{pinta-ala!a@avaruuskolmion|vahv}
\index{tilavuus!a@tetraedrin|vahv}

Ristitulolla ja skalaarikolmitulolla on fysikaalisten sovellutusten ohella myös hauskoja
geometrisia käyttökohteita. Esimerkiksi avaruuskolmion (myös tasokolmion) p\pain{inta-ala}
voidaan määrätä kätevästi ristitulon avulla, jos tunnetaan kolmion kärkipisteet karteesisessa
koordinaatistossa. Esimerkki valaiskoon asiaa.
\begin{Exa} \label{avaruuskolmion ala}
Avaruuskolmion kärjet ovat pisteissä $A=(1,0,0)$, $B=(1,1,-1)$ ja $C=(0,2,2)$. Mikä on kolmion
pinta-ala?
\end{Exa}
\ratk Tunnetun kolmion pinta-alakaavan ja \newline
vektoritulon määritelmän perusteella:
\begin{multicols}{2} \raggedcolumns
\begin{align*}
\text{ala} &= \frac{1}{2} \cdot \text{kanta} \cdot \text{korkeus} \\
           &= \frac{1}{2}\abs{\Vect{AB}}\abs{\Vect{AC}}\sin\kulma BAC \\
           &= \frac{1}{2}\abs{\Vect{AB}\times\Vect{AC}}.
\end{align*}
\begin{figure}[H]
\setlength{\unitlength}{1cm}
\begin{center}
\begin{picture}(4.5,2.5)(0,0.5)
\put(0,1){\vector(4,-1){4}}
\put(0,1){\vector(3,2){3}}
\put(3,3){\vector(1,-3){1}}
\put(-0.2,0.5){$A$} \put(4.1,-0.2){$B$} \put(3.2,3){$C$}
\end{picture}
\end{center}
\end{figure}
\end{multicols}
Tässä on $\,\Vect{AB}=\vec j - \vec k, \ \Vect{AC}=-\vec i +2\vec j + 2 \vec k$,
joten
\begin{align*}
\Vect{AB}\times\Vect{AC} = 
\left| \begin{array}{ccc}
\vec i & \vec j & \vec k \\
0 & 1 & -1 \\
-1 & 2 & 2 
\end{array} \right|
&\,=\, \left|\begin{array}{rr} 
1 & -1 \\
2 & 2
\end{array} \right|\,\vec i\ -\
\left|\begin{array}{rr} 
0 & -1 \\
-1 & 2
\end{array} \right|\,\vec j\ +\
\left|\begin{array}{rr} 
0 & 1 \\
-1 & 2
\end{array} \right|\,\vec k \\
&\,=\, (1 \cdot 2 + 1 \cdot 2)\,\vec i - (0 \cdot 2 - 1 \cdot 1)\,\vec j 
                                   +(0 \cdot 2 + 1 \cdot 1)\,\vec k \\
&\,=\, 4\vec i + \vec j + \vec k.
\end{align*}
Siis pinta-ala $=\tfrac{1}{2}\sqrt{4^2 + 1^2 + 1^2}= \frac{3}{2}\sqrt{2}$. \loppu
%Sivutuotteena saatiin pisteitten $A,B,C$ kautta kulkevan tason normaalivektori 
%$\vec n = 4\vec i + \vec j + \vec k$.

Kolmion kolmiulotteinen vastine on tetraedri. Tämän \pain{tilavuus} saadaan kätevästi
lasketuksi skalaarikolmitulon avulla, kun tunnetaan kärkipisteet karteesisessa 
koordinaatistossa. Esimerkki valaiskoon jälleen asiaa.
\begin{Exa}
Tetraedrin $K$ kärkinä ovat pisteet $O=(0,0,0)$, $A=(1,0,0)$, $B=(1,1,-1)$ ja $C=(0,2,2)$.
Määritä tetraedrin tilavuus $V$.
\end{Exa}
\ratk Tetraedrin tilavuuden tunnettu laskukaava on
\[
V = \frac{1}{3}\,\cdot\, \text{pohjan ala}\,\cdot\,\text{korkeus}.
\]
Jos tässä pohjaksi tulkitaan kolmio $ABC$, niin
\[
V = \frac{1}{6}\abs{\Vect{AO}\cdot\Vect{AB}\times\Vect{AC}},
\]
sillä
\[
\text{pohjan ala} = \frac{1}{2}\abs{\Vect{AB} \times \Vect{AC}}, 
\quad \text{korkeus} = \abs{\Vect{AO'}} = \abs{\Vect{AO}}\abs{\cos\kulma OAO'},
\]
missä $\Vect{AO'}$ on vektorin $\Vect{AO}$ kohtisuora projektio vektorin 
$\Vect{AB} \times \Vect{AC}$ suuntaan, ks.\ kuvio.
\begin{figure}[H]
\setlength{\unitlength}{1cm}
\begin{center}
\begin{picture}(8,4)(-0.1,-3)
\put(0,0){\line(1,-1){3}} \put(0,0){\line(2,-1){7}} \put(0,0){\line(4,-1){8}}
\dashline{0.2}(3,-3)(8,-2)
\path(3,-3)(7,-3.5)(8,-2)
\put(3,-3){\vector(0,1){4}}
\dashline{0.2}(0,0)(3,0)
\put(-0.1,0.2){$O$} \put(2.9,-3.5){$A$} \put(7.9,-1.8){$B$} \put(6.9,-4){$C$}
\put(3.2,-0.1){$O'$}
\put(3.2,0.8){$-\Vect{AB}\times\Vect{AC}$}
\path(2.8,0)(2.8,-0.2)(3,-0.2) \path(3,-2.6)(3.4,-2.5)(3.4,-2.9) 
\path(3,-2.8)(3.2,-2.825)(3.2,-3)
\end{picture}
\end{center}
\end{figure}
Koska
\[
\Vect{AO}\cdot\Vect{AB}\times\Vect{AC} =
\left| \begin{array}{rrr}
-1 & 0 &  0 \\
 0 & 1 & -1 \\
-1 & 2 &  2 
\end{array} \right|
=
(-1) \cdot \left| \begin{array}{rr}
1 & -1 \\
2 & 2
\end{array} \right| = -4,
\]
niin $\,V=\dfrac{1}{6}\cdot\abs{-4}=\dfrac{2}{3}$ \loppu

\subsection{Suunnikkaan ala. Suuntaissärmiön tilavuus}
\index{pinta-ala!b@avaruussuunnikkaan|vahv}
\index{tilavuus!b@suuntaissärmiön|vahv}

Suunnikas on tason nelikulmio, jonka vastakkaiset sivut ovat yhdensuuntaiset. 
Avaruudessa suunnikkaan kolmiulotteinen vastine on suuntaissärmiö eli $6$-tahokas, jonka
sivut ovat suunnikkaita (avaruustasoilla). Tämän erikoistapaus on suorakulmainen särmiö
(sivut suorakulmioita). Jos janat $OA$, $OB$ ja $OC$ ovat suuntaissärmiön kolme särmää,
niin sanotaan, että vektorit
\[
\vec a=\Vect{OA}, \quad \vec b=\Vect{OB}, \quad \vec c=\Vect{OC}
\]
\index{virittää (suunnikas, särmiö)}%
\kor{virittävät} ko.\ särmiön. Vastaavasti vektorit $\,\vec a,\vec b$, samoin vektorit
$\,\vec a,\vec c\,$ ja $\,\vec b,\vec c$, virittävät suunnikkaan (avaruustasolla).
Suuntaissärmiön tilavuuden laskukaava on: $\,V=$ pohjan ala $\cdot$ korkeus. Tässä pohja on
suunnikas, jonka pinta-ala saadaan ristitulon avulla (ala = kanta $\cdot$ korkeus), joten 
särmiön tilavuus saadaan skalaarikolmitulon avulla kuten tetraedrin tapauksessa. Laskukaavat,
kun virittäjävektorit tunnetaan:
\[
\boxed{ \begin{aligned}
\rule{0mm}{5mm}\quad &\text{Sunnikkaan pinta-ala} \qquad =\ \abs{\vec a \times \vec b}. \\
\akehys\quad &\text{Suuntaissärmiön tilavuus}\   
                   =\ \abs{\vec a \times \vec b \cdot \vec c\,}. \quad
        \end{aligned} }
\]

\subsection{Vektorikolmitulo $\vec a \times (\vec b \times \vec c\,)$}
\index{vektorikolmitulo|vahv}
\index{laskuoperaatiot!cc@avaruusvektoreiden|vahv}

Jos $T$ on taso, jonka suuntavektoreina ovat $\vec b, \vec c$ (ol. $\vec b \nparallel \vec c$),
niin vektori $\,\vec b \times \vec c\,$ on tason normaalivektori. Näin ollen 
$\vec a  \times(\vec b \times \vec c\,)$ on jälleen tason suuntainen, ts.
\[
\vec a  \times(\vec b \times \vec c\,) = \lambda \vec b + \mu \vec c
\]
joillakin $\lambda, \mu \in \R$. Toisaalta tämä vektori on myös $\vec a$:ta vasaan kohtisuora,
eli
\[
\vec a \cdot (\lambda \vec b + \mu \vec c\,) 
                     = \lambda (\vec a \cdot \vec b\,) + \mu(\vec a \cdot \vec c\,) = 0.
\]
Tämä toteutuu täsmälleen, kun $\lambda = \gamma\,\vec a \cdot \vec c$ ja 
$\mu = - \gamma\,\vec a \cdot \vec b$ jollakin $\gamma \in \R$, joten on päätelty, että
\[
\vec a \times (\vec b \times \vec c\,) 
             = \gamma\,[\,(\vec a \cdot \vec c\,)\vec b - (\vec a \cdot \vec b\,)\vec c\,\,].
\]
Tapauksessa $\,\vec a = \vec b = \vec i$, $\vec c = \vec j\,$ tämä toteutuu, kun $\gamma=1$,
joten päätellään, että vektorikolmitulolle pätee purkukaava\footnote[2]{Purkusääntöä tarkemmin
perusteltaessa olisi myös näytettävä, että  $\gamma$ ei riipu vektoreista $\vec a$, $\vec b$,
$\vec c$. Perustelut sivuutetaan tässä, sillä purkusääntö on näytettävissä oikeaksi myös
suoraan vektorien komponenttimuodosta.}
\[
\boxed{\kehys\quad \vec a \times (\vec b \times \vec c\,)
                    =(\vec a \cdot \vec c\,)\vec b - (\vec a \cdot \vec b\,)\vec c. \quad}
\]

\Harj
\begin{enumerate}

\item
Vektorin $\vec u$ koordinaatit kannassa $\{\vec i,\vec j,\vec k\}$ ovat $(x,y,z)$. Mitkä
ovat $\vec u\,$:n koordinaatit kannassa $\{\vec i+\vec j,\,\vec j+\vec k,\,\vec i+\vec k\}\,$?

\item
Laske seuraavien avaruusvektorien muodostamat kulmat $\kulma(\vec a,\vec b)$ asteina: \newline
a) \ $\vec a=\vec i,\ \vec b=\vec i+\vec j+\vec k \quad$
b) \ $\vec a=\vec i+\vec j,\ \vec b=\vec j+\vec k$

\item
a) Avaruusvektoreista $\vec a,\vec b,\vec c$ tiedetään, että $\abs{\vec a}=\abs{\vec b}=1$, 
$\abs{\vec c}=2$, $\vec a\perp\vec b$ ja $\kulma(\vec a,\vec c)=\kulma(\vec b,\vec c)=60\aste$.
Mikä on vektorin $\vec u=\vec a+\vec b+\vec c$ pituus\,? \vspace{1mm}\newline
b) Määritä yksikkövektorit, jotka muodostavat yhtä suuret kulmat vektoreiden $\vec k$, 
$\vec j+\vec k$ ja $\vec i+\vec j+\vec k$ kanssa. \vspace{1mm}\newline
c) Janasta $OP$ ($O=$ origo) tiedetään, että janan pituus on $5$ ja että jana muodostaa 
positiivisen $x$-akselin kanssa kulman $\alpha=32\aste$ ja positiivisen $y$-akselin kanssa
kulman $\beta=73\aste$. Laske $P$:n koordinaatit.

\item
a) Säännöllisen tetraedrin kärjestä lähtevät särmävektorit ovat $\vec a$, $\vec b$ ja $\vec c$.
Laske $\cos\kulma(\vec a,\vec b+\vec c)$ ja
$\cos\kulma(\vec a+\vec b+2\vec c,\,2\vec a-\vec b)$. \vspace{1mm}\newline
b) Pöydällä oleva A4-kokoinen paperiarkki on suorakulmio $ABCD$, jonka sivun pituuksien suhde
on $|AB|/|BC|=\sqrt{2}$. Arkki taitetaan pitkin lävistäjää $AC$ siten, että kolmio-osa
$ABC$ jää pöydälle ja osa $ACD$ kääntyy kolmioksi $ACD'$ pöydän pintaa vastaan vastaan
kohtisuoralle tasolle. Laske $\cos\kulma D'AB$.

\item
a) Määritä kaikki avaruusvektorit $\vec u$, jotka täyttävät ehdot (1) $\abs{\vec u}=1$, \newline
(2) $\vec u=\lambda(\vec i+\vec j)+\mu(\vec j+\vec k)$ jollakin $(\lambda,\mu)\in\Rkaksi$
ja (3) $\vec u \perp \vec i-2\vec j-\vec k$. \vspace{1mm}\newline
b) Jaa vektori $\vec u=\vec i+2\vec j-3\vec k$ kolmeen komponenttiin, joista yksi on vektorin
$\vec a=2\vec i+\vec j+\vec k$ suuntainen, toinen vektorin $\vec b=2\vec i+3\vec j-3\vec k$
suuntainen ja kolmas kohtisuorassa vektoreita $\vec a$ ja $\vec b$ vastaan. Mikä on luvun
$\sin\kulma(\vec u,\vec v\,)$ pienin arvo, kun $\vec v=\lambda\vec a + \mu\vec b$, 
$(\lambda,\mu)\in\Rkaksi$ $(\vec v\neq\vec 0\,)$\,? \vspace{1mm}\newline
c) Muodosta ortonormeerattu, oikeakätinen avaruusvektorien kanta $\{\vec a,\vec b,\vec c\,\}$ 
siten, että $\vec a$ on vektorin $\vec i+\vec j+\vec k$ suuntainen ja $\vec b$ on $xy$-tason 
suuntainen. Määräytyykö kanta yksikäsitteisesti näistä ehdoista?

\item \label{H-I-7:oktaedri}
Sijoita säännöllinen oktaedri (= kahdeksansivuinen monitahokas, jonka sivutahkot ovat 
tasasivuisia kolmioita) mukavaan asentoon koordinatistoon siten, että yksi kärki on origossa.
Laske tästä kärjestä alkavien särmien vektoriesitykset ja näiden avulla sivutahkojen 
normaalivektorit. Laske edelleen näiden avulla vierekkäisten sivutahkojen välinen 
\index{diedrikulma}%
\kor{diedrikulma}, eli kulma, jonka sivutahkot muodostavat yhteisen särmän suunnasta
katsottuna.

\item
Osoita: \ $\vec a+\vec b+\vec c=\vec 0\ \impl\ 
                       \vec a\times\vec b=\vec b\times\vec c=\vec c\times\vec a\,$.

\item
a) Pisteet $(-1,-2,4)$, $(5,-1,0)$, $(2,-3,6)$ ja $(1,-1,1)$ ovat sekä tetraedrin että
suuntiassärmiön kärkiä. Laske kummankin tilavuus ja pinnan ala. \vspace{1mm}\newline
 b) Tason kolmion kaksi kärkeä ovat pisteissä $(1,-2)$ ja $(3,3)$. Millaisessa $\Ekaksi$:n
pistejoukossa on kolmannen kärjen täsmälleen oltava, jotta kolmion pinta-ala $=10$\,? Kuva!

\item
Päättele, että avaruusvektorit $\vec a,\,\vec b,\,\vec c$ ovat lineaarisesti riippuvat
täsmälleen kun vektoreiden virittämän suuntaissärmiön tilavuus $=0$. Tutki tällä kriteerillä
ovatko seuraavat vektorisysteemit lineaarisesti riippumattomat: \vspace{1mm}\newline
a) \ $\{\vec i-\vec j+\vec k,\,3\vec i+2\vec j+\vec k,\,\vec i+\vec j-5\vec k\} \quad$
b) \ $\{\vec i-2\vec k,\,2\vec i-\vec j+3\vec k,\,5\vec i-2\vec j+4\vec k\}$

\item \label{H-II-8: vektorihajotelma}
Olkoon $\vec n$ avaruuden yksikkövektori. Halutaan esittää avaruusvektori $\vec F$ muodossa
$\vec F= \vec F_1 + \vec F_2$, missä $\vec F_1 || \vec n$ ja $\vec F_2 \perp \vec n$.
Näytä, että $\vec F_1 = (\vec F \cdot \vec n)\vec n$ ja 
$\vec F_2 = -\vec n\times(\vec n\times\vec F)$. 

\item
Todista: \newline
a) \ $\abs{\vec a\times\vec b}^2 = \abs{\vec a}^2\abs{\vec b}^2
                                   -(\vec a\cdot\vec b\,)^2$. \newline
b) \ $\abs{\vec a\times(\vec a\times \vec b\,)}^2
     =\abs{\vec a}^2\bigl[\abs{\vec a}^2\abs{\vec b}^2-(\vec a\cdot\vec b\,)^2\bigr]$. \newline
c) \ $(\vec a\times\vec b\,)\times(\vec b\times\vec c\,)\cdot(\vec c\times\vec a\,)\,
        =\,(\vec a\times\vec b\cdot\vec c\,)^2$.
                                              
\item
a) Vektorit $\vec a,\,\vec b,\,\vec c$ ovat lineaarisesti riippumattomat. Sievennä
lauseke \newline
$[(\vec a\times\vec b\,)\times(\vec b\times\vec c\,)]\times(\vec c\times\vec a\,)$. Millä 
ehdolla lauseke on $=\vec 0$\,? \vspace{1mm}\newline
b) Määritä vektorin $\,(\vec a\times(\vec a\times(\vec a\times(\vec a\times
(\vec a\times(\vec a\times\vec b\,))))))\,$ pituus, kun $\abs{\vec a}=3$, $\abs{\vec b}=1$ ja
$\vec a\cdot\vec b=-2$. \vspace{1mm}\newline
c) Määritä vektorit $\vec u$, jotka toteuttavat $\,\vec u\cdot\vec j=0\,$ ja
$\,\vec u\times(\vec k\times\vec u)=\vec k$.

\item (*) a) Näytä, että tetraedrin keskijanat, eli kärjen ja vastakkaisen sivun keskiön 
yhdysjanat, leikkaavat toisensa samassa pisteessä (= tetraedrin keskiö). Missä suhteessa 
keskijanat jakautuvat leikkauspisteessä? \newline
c) Anna kaksi esimerkkiä tetraedrista, jonka kaikilla neljällä korkeusjanalla on yhteinen piste.

\item (*)
Pisteet $A=(1,1,4)$, $B=(1,-1,3)$, $C=(-1,-1,2)$ ja $D=(0,-2,2)$ ovat eräällä avaruustasolla
$T$. Tutki, millainen ko.\ tason kuvio syntyy, kun pisteet yhdistetään mainitussa
järjestyksessä janoilla suljetuksi murtoviivaksi. Onko kyseessä nelikulmio? Valitse haluamasi 
kuvakulma (koordinaatisto) tasossa $T$ ja piirrä kuva!

\item (*)
Pisteet $(-3,-1,4)$, $(0,-1,-2)$, $(2,5,1)$, $(3,2,7)$ ja $(5,1,-2)$ ovat $5$-tahokkaan $K$
kärkipisteet. Laske $K$:n tilavuus ja pinnan ala.
%Kaikki paitsi kolmas piste ovat tasolla $2x-5y+z=3$ eli kyseessä on pyramidi.

\end{enumerate}