\section{Funktion käsite} \label{funktio}
\alku

\kor{Funktio} l. \kor{kuvaus} (engl. function, map, mapping) on matematiikassa kolmikko muotoa
\[
\{\text{joukko} \ A, \ \text{sääntö} \ f, \ \text{joukko} \ B\}.
\]
Tällä tarkoitetaan, että on jokin sääntö $f$, jonka mukaan määräytyy y\pain{ksikäsitteinen}
$y \in B$ j\pain{okaisella} $x \in A$. Yhteys $x$:n $y$:n välillä merkitään $y=f(x)$ ja
lausutaan '$y$ on $f$ $x$'. Voidaan myös käyttää \pain{liittämisnuolta} ja merkitä 
$x \map f(x) (=y)$. Sanotaan, että $A$ on $f$:n \kor{määrittelyjoukko} (tai lähtöjoukko,
engl. domain) ja $B$ \kor{maalijoukko} ja merkitään $f:A \kohti B$. Funktion määrittelyjoukon
alkioihin viitataan yleisnimellä \kor{muuttuja} (engl.\ variable). Jos $y=f(x)$, niin sanotaan,
että $y$ on $f$:n \kor{arvo} (engl.\ value) \kor{$x$:ssä} (tai muuttujan arvolla $x$);
lyhyemmin lausutaan '$f$ $x$:ssä' tai usein myös '$f$ pisteessä $x$'. Maalijoukon osajoukko
\[
\{y \in B \ | \ y=f(x) \ \text{jollakin} \ x \in A\}
\]
on nimeltään $f$:n \kor{arvojoukko} tai $A$:n \kor{kuva} $f$:ssä (engl.\ range, image). Tätä 
merkitään lyhyesti $f(A)$.
\begin{figure}[H]
\begin{center}
\import{kuvat/}{kuvaII-4.pstex_t}
\end{center}
\end{figure}
\begin{Exa}
Jos $x$ ja $y$ ovat reaalilukuja ja kirjoitetaan $f(x,y)=x^2y-y^3$, tarkoitetaan (ellei 
määrittelyjoukosta toisin sovita) funktiota
\[
\{ \Rkaksi, \ f(x,y)=x^2y-y^3, \ \R \},
\]
joka siis liittää mihin tahansa reaalilukupariin $(x,y)$ laskusäännön ilmaiseman reaaliluvun.
Funktion arvojen määräämisessä tarvitaan tässä tapauksessa ainoastaan reaalilukujen 
kerto- ja vähennyslaskuoperaatioita. Esim.
\begin{align*}
&f(3,-2)=3^2 \cdot (-2) - (-2)^3 = -10 \\
&f(2,\sqrt{2})=2^2\cdot\sqrt{2}-(\sqrt{2})^3=0 \\
&f(\pi,e)=\pi^2 e - e^3=6.742829374372954.. \loppu
\end{align*}
\end{Exa}
Hyvin yleisesti ja abstraktisti ajatellen voidaan funktiota pitää 'säännön' sijasta enemmänkin
'luettelona', joka liittää kuhunkin alkioon $x \in A$ yksikäsitteisen ($x$:stä riippuvan)
alkion $y \in B$. Näin ajatellen funktiosta tulee puhtaasti joukko-opillinen käsite: Funktion
määrittelee joukon $A \times B = \{(x,y) \mid x \in A\, \ja\, y \in B\}$ (= $A$:n ja $B$:n
karteesinen tulo, vrt.\ alaviite Luvussa \ref{tasonvektorit}) mikä tahansa osajoukko
$F \subset A \times B$, joka toteuttaa ehdon
\[
x \in A \ \impl \ (x,y) \in F\,\ \text{täsmälleen yhdellä}\,\ y \in B.
\]
Nimittäin tällaista joukkoa vastaa $A$:ssa määritelty funktio, kun tulkitaan
\[
y=f(x)\ \ekv\ (x,y) \in F.
\]
Funktion käsitteen kannalta ei siis ole lainkaan merkitystä sillä, määräytyvätkö funktion arvot
jonkin säännön mukaan laskemalla vai pelkästään 'luettelosta katsomalla'. 
\begin{Exa} Rationaalilukujen jono $\{a_1, a_2, \ldots\}$ tarkoittaa 'luetteloa' $1 \map a_1$, 
$2 \map a_2$, \ldots, joten tämä on tulkittavissa kuvaukseksi tyyppiä $f: \ \N \kohti \Q$, eli 
siis myös joukoksi
\[
F= \{ (1,a_1), (2,a_2),\ldots \} \subset \N \times \Q \loppu
\]
\end{Exa}
Kaikki tähän asti esiintyneet \pain{laskuo}p\pain{eraatiot} ovat itse asiassa funktioita. 
Funktion voi tällöin mieltää 'laskukoneeksi', joka antaa laskuoperaation lopputuloksen
annetuilla lähtötiedoilla.
\begin{Exa}
Reaalilukujen yhteenlasku ja kertolasku ovat funktioita tyyppiä $f:\ \Rkaksi \map \R$ 
laskusäännöillä
\[
\text{yhteenlasku:} \quad f(x,y)=x+y, \qquad \text{kertolasku:} \quad f(x,y)=xy.
\]
Jakolasku on funktio $f: \{(x,y) \in \Rkaksi \ | \ y \neq 0 \} \kohti \R$ säännöllä
$f(x,y)=x/y$. \loppu
\end{Exa}
\begin{Exa} \label{skalaaritulo funktiona} Edellisessä luvussa liitettiin euklidisen tason
kulmaan $\kulma(\vec a,\vec b)$ geometrisin keinoin määräytyvä reaaliluku, jota merkittiin 
$\cos\kulma(\vec a,\vec b)$. Koska ko.\ luku on yksikäsitteinen, niin kyseessä on funktio 
$\,\cos:\ A \kohti \R$, missä $A=$ kaikkien kulmien joukko. Funktion arvojoukko on 
$[-1,1]\subset\R$. Määrittelyjoukossa voidaan kulmat haluttaessa tulkita tason vektoripareina
$(\vec a,\vec b)$, missä $\vec a,\vec b \neq \vec 0$, tai yksikkövektorien pareina 
($\abs{\vec a}=\abs{\vec b}=1$).  \loppu
\end{Exa}

\subsection{Käänteisfunktio}

Aiemmin (esim.\ Luvussa \ref{tasonvektorit}) on käytetty merkintää $A \vast B$, kun joukkojen
$A$ ja $B$ alkiot vastaavat toisiaan 'yksi yhteen'. Tällöin jos '$x$:\,ää vastaa $y$' merkitään
$x \map y$, niin nähdään, että vastaavuusmerkintä $x \vast y$ tarkoittaa samaa kuin
$x \map y\ \ja\ y \map x$. Vastaavuudessa on siis kyse funktioista $f_1:\,A \kohti B$ ja 
$f_2:\,B \kohti A$, joilla on ominaisuus
\[
f_1(x) = y\,\ \ekv\,\ f_2(y)=x \quad (x \in A,\ y \in B).
\]
Tämä tarkoittaa, että $f_1$ ja $f_2$ ovat toistensa \kor{käänteisfunktioita}. Yleisesti jos
$f:\ A \kohti B$ on funktio, niin $f$:n käänteisfunktio on olemassa, eli $f$ on \kor{kääntyvä} 
(engl.\ invertible), jos pätee
\[
x_1 \neq x_2\ \impl\ f(x_1) \neq f(x_2) \quad \forall x_1,x_2 \in A.
\] 
Sanotaan tällöin, että $f$ on 1-1 eli 'yksi yhteen' (engl. one to one), tai että $f$ on
\kor{kääntäen yksikäsitteinen} funktiona $f:\ A \kohti f(A)$. Käänteisfunktion yleinen symboli 
on $f^{-1}$, luetaan '$f$ miinus $1$'. 
\begin{figure}[H]
\begin{center}
\import{kuvat/}{kuvaII-5.pstex_t}
\end{center}
\end{figure}
\begin{figure}[H]
\begin{center}
\import{kuvat/}{kuvaII-6.pstex_t}
\end{center}
\end{figure}
Jos $f:\,A \kohti B$ on 1-1, niin käänteisfunktion $\inv{f}$ määrittelyjoukko $= f$:n
arvojoukko $f(A)$ ja laskusääntö (liittämissääntö) siis 
\[ 
x=\inv{f}(y)\,\ \ekv\,\ y=f(x)\ \ (y \in f(A)).
\] 
\begin{Exa} Olkoon $A=\{1,2,3\}\subset\N$ ja $B=\{4,5,6\}\subset\N$ ja määritellään funktiot 
$f:\ A \kohti B$ ja $g:\ A \kohti B$ asettamalla
\begin{align*}
&f(1)=6,   \quad f(2)=4,     \quad f(3)=5 \\
&g(1)=4,\, \quad g(2)=5,\, \quad g(3)=4
\end{align*}
Tällöin $f$ on kääntyvä ja
\[
\inv{f}(4)=2, \quad \inv{f}(5)=3, \quad \inv{f}(6)=1.
\]
Funktio $g$ ei ole kääntyvä, koska $g(1)=g(3)$. Joukko-opillisesti ilmaistuna on
\begin{align*}
f\       &=\ \{(1,6),(2,4),(3,5)\}\ \subset\ A \times B \\
g\       &=\ \{(1,4),(2,5),(3,4)\}\ \subset\ A \times B \\
\inv{f}\ &=\ \{(6,1),(4,2),(5,3)\}\ =\ \{(4,2),(5,3),(6,1)\}\ \subset\ B \times A \loppu
\end{align*}  
\end{Exa}

\subsection{Reaalifunktio}

'Funktioiden äiti' on \kor{reaalifunktio} eli reaalimuuttujan reaaliarvoinen funktio muotoa
$f: A \kohti \R$, missä $A \subset \R$. Myös termiä 'yhden muuttujan funktio' käytetään usein 
tässä rajatussa merkityksessä. Reaalifunktio ilmoitetaan tavallisimmin pelkkänä laskusääntönä,
sikäli kuin sääntö on yksinkertaisesti ilmaistavissa. Tällöin oletetaan (ellei toisin sovita),
että määrittelyjoukko on suurin $\R$:n osa\-joukko, jossa laskusääntöä voi soveltaa.
Yksittäistä laskusäännön käyttöä, eli laskuoperaatiota $x \map f(x)$ sanotaan 
\kor{funktioevaluaatioksi}. Usein tämä onnistuu vain numeerisesti (eli likimäärin), 
jolloin käytännössä pystytään vain laskemaan äärellinen määrä termejä lukujonosta, jonka 
raja-arvo $=f(x)$.
\begin{Exa} Reaalifunktion $f(x)=x^2/(x^2-2)$ määrittelyjoukko on \newline
$A\ =\ \{x\in\R \mid x^2 \neq 2\}\ 
    =\ (-\infty,-\sqrt{2}) \cup (-\sqrt{2},\sqrt{2}) \cup (\sqrt{2},\infty)$ \loppu
\end{Exa}
\begin{Exa} Potenssisarja $\sum_k a_k x^k$ voidaan tulkita reaalifunktioksi, jonka arvo
$x$:ssä = sarjan summa ja määrittelyjoukko = sarjan suppenemisväli. Esimerkiksi funktion
\[
f(x)=\sum_{k=1}^\infty \frac{x^k}{k}
\]
määrittelyjoukko on väli $[-1,1)$, vrt.\ Luku \ref{potenssisarja}. Ellei ole $x=0$, on
funktioevaluaatio $x \map f(x)$ tässä tapauksessa suoritettava numeerisesti, esim.\ 
laskemalla sarjan osasummia. \loppu
\end{Exa}

Reaalimuuttujan funktio voidaan kätevästi havainnollistaa \kor{kuvaajan} (engl.\ graph) avulla. 
Funktion $f: A \kohti \R$ kuvaaja joukossa $B \subset A\ (A\subset\R)$ on euklidisen avaruuden 
$\Ekaksi$ pistejoukko
\[ 
S = \{P = (x,y) \mid y = f(x)\,\ja\, x \in B\},
\]
missä $P$:n koordinaatit $(x,y)$ viittaavat karteesiseen koordinaatistoon (ellei toisin sovita).
Myös reaalilukujono voidaan tällä tavoin kuvata $\Ekaksi$:n pistejonona (kuten jo aiemmin 
tehtiin Luvuissa \ref{jonon raja-arvo} ja \ref{Cauchyn jonot}). Lukujono tulkitaan tällöin 
funktioksi $f:\ \N \kohti \R$.
\begin{figure}[H]
\setlength{\unitlength}{1cm}
\begin{center}
\begin{picture}(12.5,8)(0,-2)
\multiput(0,0)(7,0){2}{
\put(0,0){\vector(1,0){6}} \put(1,-1){\vector(0,1){6}}
\put(1.9,-0.5){$1$} \put(2.9,-0.5){$2$} \put(3.9,-0.5){$3$} \put(4.9,-0.5){$4$} 
\put(5.8,-0.5){$x$} \put(1.2,4.8){$y$}
\multiput(2,0)(1,0){4}{\line(0,-1){0.1}}}
\put(3.5,1.5){$S:\ y=f(x)$}
\put(9,0.5){$\scriptstyle{\bullet}$} \put(10,1.33){$\scriptstyle{\bullet}$}
\put(11,2.25){$\scriptstyle{\bullet}$} \put(12,3.2){$\scriptstyle{\bullet}$}
\put(2.5,-2){$f(x)=\dfrac{x^2}{x+1}$} \put(9.5,-2){$a_n=\dfrac{n^2}{n+1}$}
\put(1,0){
\curve( 
0, 0,
0.5, 0.17,
1, 0.5,
1.5, 0.9,
2, 1.33,
3, 2.25,
4, 3.2)}
\end{picture}
\end{center}
\end{figure}

\Harj
\begin{enumerate}

\item
Psykiatrin vastaanottoa voi kuvata matemaattisesti kolmikkona \{psykiatri,
potilas, diagnoosi\}. Pohdi seuraavissa tapauksissa (kussakin erikseen), millaisilla oletuksilla
kyseessä on funktio: \newline
a) \ psykiatri : potilaat $\kohti$ diagnoosit \newline
b) \ potilas  :  psykiatrit $\kohti$ diagnoosit \newline
c) \ psykiatri : diagnoosit $\kohti$ potilaat \newline
d) \ potilas : diagnoosit $\kohti$ psykiatrit

\item
Funktiot $f$ ja $g$ määritellään laskusäännöillä $f(x,y)=x^5+2x^4y-2x^2y^3$ ja
$g(x,y,z)=(x+y+z)/(1+x^2+y^2+z^2)$, missä $x,y,z\in\R$. Laske $f(2,-3)$, $f(\sqrt{2},\sqrt{2})$,
$g(0,0,0)$ ja $g(1,-2,3)$. 

\item 
Mitkä ovat seuraavien funktioiden määrittelyjoukot ($x,y\in\R$)? 
\[
\text{a)}\,\ f(x)=x^{0} \quad\ \text{b)}\,\ f(x)=\frac{x^2}{x} \quad\ 
\text{c)}\,\ f(x,y)=\frac{x+y}{x^2-y^2}
\]
Voiko nämä funktiot ilmaista määrittelyjoukossaan jollakin yksinkertaisemmalla laskusäännöllä?

\item
Mitä yhteistä ja mitä eroa on reaalifunktioilla \newline
$\displaystyle{f(x)=\sum_{k=1}^\infty \left(\frac{x}{2}\right)^k\,\ \text{ja} \quad
               g(x)=\frac{x}{2-x}\ }$?

\item
Mitkä seuraavista joukon $\Rkaksi=\R\times\R$ osajoukoista $F$ ovat funktioita? Määrittele myös 
(joukko-opillisesti) käänteisfunktiot, jos olemassa. \newline
a) \ $F=\{(1,2),(2,1),(3,3),(\pi,e),(e,\pi)\}$ \newline
b) \ $F=\{(1,2),(2,3),(3,3),(3,2),(2,1)\}$ \newline
c) \ $F=\{(n,n^2) \mid n\in\Z\}$ \newline
d) \ $F=\{(x,x^3) \mid x\in\Q\}$

\item
Tulkitse funktioina (määrittely- ja arvojoukko!) avaruusvektorien laskuoperaatiot: yhteenlasku,
skalaarilla kertominen, pistetulo ja ristitulo. Millainen funktio on skalaarikolmitulo?
 
\item
a) Onko olemassa funktio $f$, jonka määrittelyjoukko $=\R$ ja $\forall x\in\R$ pätee
$f(x)=\sum_{k=0}^\infty x^k$\,? Jos vastaus on myönteinen, niin määrittele $f$! \newline 
b) Yhden reaalimuuttujan sisältävän predikaatin (Luku \ref{logiikka}) voi tulkita funktioksi
ja jopa reaalifunktioksi. Miten?

\item
Rationaalilukujono $\{a_n,\ n=1,2,\ldots\}$ voidaan tulkita funktioksi tyyppiä $f:\ \N\kohti\Q$.
Tutki, onko ko.\ funktio kääntyvä seuraavissa tapauksissa: \newline
a) \ $a_n=(-1)^n,\ \ $ b) \ $a_n=(n+1)/(n+2),\ \ $ c) \ $a_n=n^3 2^{-n},\ \ $
d) \ $a_n=n^2 2^{-n}$

\item (*) 
Funktioista puhuttaessa lausutaan $f(x)$ usein '$f$ pisteessä $x$'. Olkoon nyt $x\in \R$ 
annettu 'piste' ja määritellään 'pistefunktio' $x$ seuraavasti:
\[
x(f)=f(x)\text{ }\forall f\in A,
\]
missä $A=\{$reaalifunktiot, jotka on määritelty pisteessä $x\}$. Onko tällainen 
funktion määrittely todella mahdollinen ja jos on, miten $x(f)$ pitäisi lausua?

\end{enumerate}