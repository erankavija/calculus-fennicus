\section{Käyräviivaiset koordinaatistot} \label{koordinaatistot}
\alku
Euklidinen taso ja avaruus ovat matemaattisina käsitteinä aineettomia tai \kor{isotrooppisia} 
siinä mielessä, että mikään piste tai suunta ei niissä ole erikoisasemassa. Erityisesti voidaan
karteesisen koordinaatiston origo, ja myös koordinaattiakselien suunnat 
ortogonaalisuusehtojen puitteissa, valita mielivaltaisesti, ilman että avaruuden geometriset 
lait siitä muuttuisivat. Sovellettaessa geometrista ajattelua fysikaaliseen maailmaan törmätään
kuitenkin siihen realiteettiin, että maailmassa yleensä on materiaa, joka ei jakaudu tasaisesti.
Voi esimerkiksi esiintyä eri materiaalien välisiä rajapintoja. Fysikaalinen ongelma voi olla
ei-isotrooppinen myös siinä vaikuttavien ulkoisten voimien (kuten gravitaation) vuoksi. 
Tällaisissakin tilanteissa ongelmassa voi kuitenkin olla \kor{symmetrioita}, millä tarkoitetaan,
että vaikka koko avaruus ei ole isotrooppinen, niin fysikaalisen ongelman geometria kuitenkin on
johonkin suuntaan tai joihinkin suuntiin liikuttaessa sama. Esimerkiksi 'pannukakkumaailmassa',
jossa puoliavaruus
\[
\mathit{E}_-^{\mathit{3}} = \{P=(x,y,z) \in \Ekolme \ | \ z \leq 0 \}
\]
on homogeenisen materian täyttämä ja muu osa avaruudesta on tyhjää (tai homogeenista toista 
ainetta), geometria säilyy samana liikuttaessa tasoilla, joilla $z=$ vakio. Toisin sanoen: 
geometria on koordinaateista $x,y$ riippumaton. Tässä on kyseessä \kor{tasosymmetrinen} tilanne.
Hieman realistisemmassa (vaikka edelleen maakeskisessä) 'pallomaailmassa', jossa materian 
täyttämää palloa ympäröi homogeeninen avaruus, geometria on puolestaan \kor{pallosymmetrinen}.
Lopulta 'tekniikan maailmassa', jossa avaruuden homogeenisuuden rikkoo yksinäinen, 
poikkipinnaltaan pyöreä kappale (maitohinkki, putki, koaksiaalikaapeli ym.) tilanne on 
\kor{lieriösymmetrinen}.
\begin{figure}[H]
\begin{center}
\import{kuvat/}{kuvaII-19.pstex_t}
\end{center}
\end{figure}
Lieriösymmetrian vastine $\Ekaksi$:ssa on \kor{ympyräsymmetria} (pyörähdyssymmetria).
\begin{figure}[H]
\begin{center}
\import{kuvat/}{kuvaII-20.pstex_t}
\end{center}
\end{figure}
Em.\ tasosymmetrisessä tilanteessa on karteesinen koordinaatisto probleeman 
geometriaan sopiva, kunhan yksi kantavektoreista (yleensä $\vec k$) valitaan symmetriatason
normaaliksi. Muissa esimerkkitilanteissa sen sijaan tulee kyseeseen siirtyminen sellaiseen 
\kor{käyräviivaiseen} (engl. curvilinear) koordinaatistoon, jossa jokin koordinaateista on vakio 
materiaalirajapinnalla. Koordinaatistoa sanotaan käyräviivaiseksi, jos \kor{koordinaattiviivat},
eli $\Ekaksi$:n tai $\Ekolme$:n pistejoukot, joilla vain yksi koordinaateista on muuttuva, eivät
kaikki ole suoria. Seuraavassa esitellään fysiikan kolme tavallisinta käyräviivaista 
koordinaatistoa, jotka liittyvät em. esimerkkitilanteisiin. Nämä ovat:
\begin{itemize}
\item  \kor{napa}- eli \kor{polaarikoordinaatisto} ($\Ekaksi$),
\item  \kor{lieriö}- eli \kor{sylinterikoordinaatisto} ($\Ekolme$),
\item  \kor{pallokoordinaatisto} ($\Ekolme$).
\end{itemize}
Tällaisia koordinaatistoja sovellettaessa (niinkuin yleensäkin matematiikkaa sovellettaessa) 
tehdään yleensä käytännön ongelmatilannetta koskevia yksinkertaistavia olettamuksia eli 
idealisaatioita -- joskus voimakkaita.

\subsection{Napakoordinaatisto}

Napakoordinaatistossa Euklidisen tason pisteen $P$ koordinaatit ilmoitetaan origon ($O$) ja 
$P$:n välisellä etäisyydellä $r$ ja nk. \kor{napakulmalla} $\varphi$, joka mittaa vektorin 
$\overrightarrow{OP}$ ja positiivisen $x$-akselin välistä kulmaa $x$-akselista lähtien 
vastapäivään. Kaikki tason pisteet saadaan, kun napakulma valitaan esim.\ väliltä $[0,2\pi)$ tai
$(-\pi,\pi]$. Koska origossa napakulma $\varphi$ ei ole määritelty, niin sovitaan, että
$O=(0,\varphi)\ \forall\varphi$.
\begin{figure}[H]
\begin{center}
\import{kuvat/}{kuvaII-21.pstex_t}
\end{center}
\end{figure}
Muunnoskaavat napakoordinaatistosta karteesiseen ovat
\[
\boxed{\kehys\quad x=r\cos{\varphi}, \quad y=r\sin{\varphi} \quad}
\]
ja päinvastaiseen suuntaan esimerkiksi
\[
\boxed{\kehys\quad r=\sqrt{x^2 + y^2}\,, \quad \sin{\varphi}=y/\sqrt{x^2 + y^2} \quad}
\]
Tässä on $\varphi$:n määräämiseksi osattava 'kääntää' trigonometrinen sini. Kun huomioidaan myös
$x$:n ja $y$:n etumerkit, niin kulma $\varphi$ määräytyy yksikäsitteisesti esim.\ välillä 
$[0,2\pi)$, kun $r>0$. 
\begin{Exa}
Millainen pistejoukko on
\[
S=\{P=(r,\varphi) \in \Ekaksi \mid r=2\cos\varphi, \ 
                   \varphi\in[-\tfrac{\pi}{2}\,,\tfrac{\pi}{2}]\}\,?
\]
\end{Exa}
\ratk Jos $r > 0$, niin
\begin{align*}
r=2\cos{\varphi} \ &\ekv \ r^2=2r\cos{\varphi} \\
&\ekv x^2+y^2=2x \\
&\ekv (x-1)^2+y^2=1.
\end{align*}
Myös origo on käyrässä mukana $(r=0, \ \varphi=\frac{\pi}{2})$, joten kyseessä on ympyrä, jonka
säde $=1$ ja keskipiste $=(1,0)$. \loppu
\begin{figure}[H]
\begin{center}
\import{kuvat/}{kuvaII-22.pstex_t}
\end{center}
\end{figure}

\begin{multicols}{2} \raggedcolumns
Napakoordinaatistossa koordinaattiviivat, joilla $r=\text{vakio}$, ovat origokeskisiä ympyröitä
ja viivat, joilla $\varphi=\text{vakio}$, ovat origosta lähteviä puolisuoria. Koska nämä 
leikkaavat toisensa kohtisuorasti (ks.\ kuvio), niin voidaan sanoa, että polaarikoordinaatisto
on \kor{suorakulmainen}, vaikkakaan ei suoraviivainen koordinaatisto.
\begin{figure}[H]
\setlength{\unitlength}{1cm}
\begin{center}
\begin{picture}(6,4.5)(-1,-0.5)
\put(-1,0){\line(1,0){5}}
\put(0,-1){\line(0,1){5}}
\put(0,0){\arc{6}{-1.7}{0.15}}
\put(0,0){\line(2,3){2.5}}
\dashline{0.2}(0.3,3.4)(3.3,1.4)
\put(1.59,2.43){$\scriptstyle{\bullet}$}
\put(1.56,2.8){$P$}
\put(3.5,0.5){$r=\text{vakio}$}
\path(2.93,0.6)(3.3,0.6)
\put(3,3){$\varphi=\text{vakio}$}
\path(2.08,3.1)(2.8,3.1)
\end{picture}
\end{center}
\end{figure}
\end{multicols}
Kun napakoordinaatistossa tarkastellaan vektoreita, ilmaistaan nämä yleensä tarkkailupisteen
$P$ mukana 'pyörivässä' kannassa $\{\vec e_r,\vec e_\varphi\}$, missä kantavektorit ovat 
koordinaattiviivojen yksikkötangenttivektorit $P$:ssä. Suunnat valitaan siten, että
vektorit osoittavat ko.\ koordinaattiviivalla muuttuvan koordinaatin kasvun suuntaan:
\begin{figure}[H]
\setlength{\unitlength}{1cm}
\begin{center}
\begin{picture}(8,4)(0,-0.4)
\put(0,0){\vector(1,0){8}} \put(7.8,-0.4){$x$} \put(-0.07,-0.07){$\scriptstyle{\bullet}$}
\put(0,0){\line(3,1){6}} \put(5.93,1.93){$\scriptstyle{\bullet}$}
\put(6,2){\vector(-1,3){0.33}} \put(6,2){\vector(3,1){1}}
%\put(6,2){\arc{0.6}{-1.9}{-0.35}} \put(6.05,2.1){$\scriptstyle{\bullet}$}
\path(6.15,2.05)(6.1,2.2)(5.95,2.15) 
\put (-0.1,-0.5){$O$} \put(5.95,1.6){$P$}
\put(0,0){\arc{2}{-0.33}{0}} \put(0.95,0.28){\vector(-1,4){0.01}} \put(1.2,0.15){$\varphi$}
\put(5.8,3){$\vec e_\varphi$} \put(7,2){$\vec e_r$}
\end{picture}
\end{center}
\end{figure}
Jos pisteen $P$ napakoordinaatit ovat $(r,\varphi)$, niin kantavektorit ovat (vrt.\ kuva)
\[
\boxed{\kehys\quad \vec e_r=\cos\varphi\vec i+\sin\varphi\vec j, \quad
                   \vec e_\varphi=-\sin\varphi\vec i+\cos\varphi\vec j \quad}
\]
Jos $P$ ajatellaan kiinnitetyksi, voidaan mikä tahansa tason vektori esittää kannassa 
$\{\vec e_r, \vec e_\varphi\}$, eli $\{P,\vec e_r,\vec e_\varphi\}$ on tason koordinaatisto. 
Toisaalta jos $P$:tä muutetaan siten, että koordinaatti $\varphi$ muuttuu, niin muuttuvat myös 
$\vec e_r$ ja $\vec e_\varphi$. Koordinaatisto $\{P, \vec e_r, \vec e_\varphi\}$ poikkeaa siis 
tässä mielessä vastaavasta suoraviivaisesta koordinaatistosta $\{P, \vec i, \vec j\}$, jossa 
kantavektorit $\vec i,\vec j$ pysyvät vakioina referenssipisteen $P$ muuttuessa. 

Tyypillinen fysikaalinen sovellutustilanne pisteen $P$ mukana 'pyörivälle' koordinaatistolle 
$\{P,\vec e_r,\vec e_\varphi\}$ on esimerkiksi sellainen, missä $P$ on liikkuvan kappaleen 
kiinteä piste (esim.\ painopiste) ja tarkastellaan vektoreita, joiden fysikaalisena 
vaikutuspisteenä on $P$ ja jotka ovat luontevasti esitettävissä kantavektorien 
$\vec e_r,\vec e_\varphi$ avulla. Tällaisia vektoreita ovat esim.\ kappaleen painopisteeseen 
vaikuttava keskeisvoima, kun voimakeskuksena on $O$, tai pyörivän kappaleen nopeus ja kiihtyvyys
kappaleeseen sidotussa pisteessä $P$, kun $O$ on pyörimiskeskus.
\begin{Exa}
Avaruusalukseen, joka on pisteeesä $P:(x=3, y=4)$ ja jonka massa $m=1000$, vaikuttaa planeetan 
vetovoima $\vec G=-90\vec e_r$ ja suunnatun rakettimoottorin työntövoima 
$\vec F=500(-\vec i + \vec j)$. Avaruusalus on liikkeessä suuntaan $\vec e_\varphi$. Mihin 
suuntaan liikerata kaartuu, ja mikä on aluksen kiihtyvyys liikeradan suuntaan\,?
\end{Exa}
\ratk Pisteessä $P$ on $\vec e_r = \frac{1}{5}(3\vec i + 4\vec j)$, 
$\,\vec e_\varphi = \frac{1}{5}(-4\vec i + 3\vec j)$, joten
\[ \begin{cases}
(\vec F + \vec G) \cdot \vec e_r = \frac{500}{5}[(-1) \cdot 3 + 1 \cdot 4] - 90 =10 \\
(\vec F + \vec G) \cdot \vec e_\varphi = \vec F \cdot \vec e_\varphi = 700
\end{cases} 
\impl\ \vec F + \vec G = 10 \vec e_r + 700 \vec e_\varphi.
\]
Koska $(\vec F + \vec G) \cdot \vec e_r >0$, niin rata kaartuu oikealle (poispäin planeetasta).
Ratakiihtyvyys määräytyy Newtonin liikelaista:
\[
\vec a \cdot \vec e_\varphi = \frac{1}{m}(\vec F + \vec G) \cdot \vec e_\varphi = 0.7 \loppu 
\]

\subsection{Lieriökoordinaatisto}

\begin{multicols}{2} \raggedcolumns
Lieriökoordinaatisto on napakoordinaatiston vastine kolmessa dimensiossa. Karteesisista 
koordinaateista $x,y,z$ jätetään $z$ ennalleen ja muunnetaan $(x,y) \map (r,\varphi)$ samalla 
tavoin kuin napakoordinaatistoon siirryttäessä. Lieriökoordinaatiston 'pyöriviä'
kantavektoreita merkitään $\vec e_r, \vec e_\varphi, \vec e_z$. Tässä $\vec e_r,\vec e_\varphi$
ovat samat kuin polaarikoordinaatistossa ja $\,\vec e_z = \vec k\,$ kuten karteesisessa 
koordinaatistossa. Kanta $\{\vec e_r,\vec e_\varphi,\vec e_z\}$ on ortonormeerattu oikeakätinen
systeemi.
\begin{figure}[H]
\setlength{\unitlength}{1cm}
\begin{center}
\begin{picture}(5,6)(-2,-2)
\put(0,0){\vector(1,0){3}} \put(2.8,-0.4){$y$}
\put(0,0){\vector(0,1){4}} \put(0.2,3.8){$z$}
\put(0,0){\vector(-3,-2){2}} \put(-2.3,-1.7){$x$}
\path(0,0)(1.5,-1)(1.5,2)
\dashline{0.1}(1.5,2)(0,3)
\put(1.43,1.93){$\scriptstyle{\bullet}$} 
\put(1.5,2){\vector(4,-3){0.7}} \put(2,1){$\vec e_r$}
\put(1.5,2){\vector(0,1){1}} \put(1.6,3.1){$\vec e_z$}
\put(1.5,2){\vector(2,1){0.8}} \put(2.4,2.3){$\vec e_\varphi$}
\put(1.67,2){\begin{turn}{-90} 
   $\overbrace{\hspace{3cm}}^{\displaystyle{\text{\begin{turn}{90}$z$\end{turn}}}}$ \end{turn}}
\put(-0.4,-0.1){\begin{turn}{-33.7} 
   $\underbrace{\hspace{1.8cm}}_{\displaystyle{\text{\begin{turn}{33.7}$r$\end{turn}}}}$
\end{turn}}
\put(0,0){\arc{0.8}{0.588}{2.56}} 
\put(0.33,-0.23){\vector(1,1){0.01}} \put(-0.15,-0.7){$\varphi$}
%\put(1.5,-1){\arc{0.6}{3.75}{4.7}} \put(1.35,-0.9){$\scriptscriptstyle{\bullet}$}
\path(1.35,-.9)(1.35,-.75)(1.5,-.85)
\end{picture}
\end{center}
\end{figure}
\end{multicols}
\begin{Exa}
\kor{Ruuviviiva} on avaruuden $\Ekolme$ pistejoukko
\[
S=\{P=(r,\varphi,z) \ | \ r=R, \ z=a\varphi\} \quad (a \neq 0) \loppu
\]
\end{Exa}
Lieriökoordinaatiston koordinaattiviivat ovat $\Ekolme$:n pistejoukkoja, joissa kaksi 
koordinaateista $r,\varphi,z$ saa vakioarvon. Näille saadaan seuraavat geometriset luonnehdinnat
(vrt.\ kuvio edellä):
\begin{align*}
&S_r\       (\varphi,z\ \text{vakioita}): \quad \text{$z$-akselilta lähtevä, 
                                                      $xy$-tason suuntainen puolisuora} \\
&S_\varphi\ (r,z \text{vakioita})\,\ :  \quad \text{ympyrä $xy$-tason suuntaisella tasolla} \\
&S_z\       (r,\varphi\ \text{vakioita})\, : \quad \text{$z$-akselin suuntainen suora}
\end{align*}
\kor{Koordinaattipinnoiksi} sanotaan sellaisia $\Ekolme$:n pistejoukkoja, joilla yksi
koordinaateista saa vakioarvon. Lieriökoordinaatistossa näiden geometriset luonnehdinnat ovat
\begin{align*}
&r=\ \text{vakio}:       \quad\, \text{lieriö, akselina $z$-akseli} \\
&\varphi=\ \text{vakio}: \quad   \text{$z$-akseliin rajoittuva puolitaso} \\
&z=\ \text{vakio}:       \quad   \text{$xy$-tason suuntainen taso}
\end{align*}

\subsection{Pallokoordinaatisto}

Pisteen $P=(x,y,z)$ \kor{pallokoordinaatit} ovat (ks.\ kuvio)
\begin{itemize}
\item[-] $P\,$:n etäisyys origosta: $\ r=\sqrt{x^2+y^2+z^2}$
\item[-] paikkavektorin $\vec r = \overrightarrow{OP}$ ja positiivisen $z$-akselin 
         (vektorin $\vec k$) muodostama kulma $\theta$, $\ 0 \leq \theta \leq \pi$
\item[-] kulma $\varphi\,,\ 0 \le \varphi < 2\pi$, jonka muodostavat positiivinen $x$-akseli
         (vektori $\vec i$) ja vektorin $\vec r = \overrightarrow{OP}$ ortogonaaliprojektio 
         $xy$-tasolle
\end{itemize}
Koordinaatteja $\theta, \varphi$ sanotaan \kor{pallonpintakoordinaateiksi}. 
\begin{figure}[H]
\setlength{\unitlength}{1cm}
\begin{center}
\begin{picture}(8,6)(-2,-2)
\put(0,0){\vector(1,0){3}} \put(2.8,-0.4){$y$}
\put(0,0){\vector(0,1){4}} \put(0.2,3.8){$z$}
\put(0,0){\vector(-3,-2){2}} \put(-2.3,-1.7){$x$}
\dashline{0.1}(0,0)(1.5,-1)
\dashline{0.1}(1.5,-1)(1.5,2)
\dashline{0.1}(1.5,2)(0,3)
\put(1.43,1.93){$\scriptstyle{\bullet}$} 
\put(1.5,2){\vector(1,-1){0.65}} \put(2.2,1){$\vec e_\theta$}
\put(1.5,2){\vector(3,4){0.67}} \put(1.6,3.1){$\vec e_r$}
\put(1.5,2){\vector(2,1){0.8}} \put(2.4,2.3){$\vec e_\varphi$}
\put(-1.7,3){\begin{turn}{-90} 
$\underbrace{\hspace{3cm}}_{\displaystyle{\text{\begin{turn}{90}
                                               $r\cos\theta$\end{turn}}}}$ \end{turn}}
\put(-0.7,0){\begin{turn}{-33.7} 
$\underbrace{\hspace{1.8cm}}_{\displaystyle{\text{\begin{turn}{33.7}$r\sin\theta$\end{turn}}}}$
\end{turn}}
\put(0,0){\arc{0.8}{0.588}{2.56}} 
\put(0.33,-0.23){\vector(1,1){0.01}} \put(-0.15,-0.7){$\varphi$}
\path(0,0)(1.5,2)
\put(1.7,2){\begin{rotate}{-127} 
$\overbrace{\hspace{2.5cm}}^{\displaystyle{\text{\begin{turn}{127}$r$\end{turn}}}}$ \end{rotate}}
\put(0,0){\arc{0.8}{-1.57}{-0.93}}
\put(0.1,0.5){$\theta$}
%\put(0,3){\arc{0.6}{0.6}{1.57}} \put(0.03,2.75){$\scriptscriptstyle{\bullet}$}
%\put(1.5,-1){\arc{0.6}{3.75}{4.7}} \put(1.35,-0.9){$\scriptscriptstyle{\bullet}$}
\path(0,2.85)(0.15,2.75)(0.15,2.9)
\path(1.35,-.9)(1.35,-.75)(1.5,-.85)
\put(1.4,2.2){$P$}
\end{picture}
\end{center}
\end{figure}
Muunnoskaavat ovat
\[
\boxed{\quad
\begin{array}{ll}
\ykehys x &= \ r \sin{\theta} \cos{\varphi} \quad \\
        y &= \ r \sin{\theta} \sin{\varphi} \\
        z &= \ r \cos{\theta} \akehys
\end{array}
}
\]

Pallokoordinaatiston koordinaattiviivat ovat seuraavaa tyyppiä:
\begin{align*}
&S_r\ (\theta,\varphi\ \text{vakioita}): \quad \text{origosta lähtevä puolisuora} \\
&S_\theta\ (r,\varphi\ \text{vakioita}):  \quad \text{origokeskinen, $z$-akseliin rajoittuva
                                                                              puoliympyrä} \\
&S_\varphi\ (r,\theta\ \text{vakioita}): \quad \text{ympyrä $xy$-tason suuntaisella tasolla}
\end{align*}
Koordinaattipinnat voidaan luokitella seuraavasti:
\begin{align*}
&r=\ \text{vakio}:       \quad\, \text{origokeskinen pallo} \\
&\theta=\ \text{vakio}:  \quad   \text{puolikartio, kärkenä origo} \\
&\varphi=\ \text{vakio}: \quad   \text{$z$-akseliin rajoittuva puolitaso}
\end{align*}

Myös pallokoordinaatistossa koordinaattiviivat leikkaavat toisensa kohtisuorasti, ts.\
koordinaattiviivojen tangentit (ks.\ edellinen luku) ovat leikkauspisteessä parittain
kohtisuorat. Koordinaattiviivojen yksikkötangenttivektoreista (tangenttien suuntavektoreista)
muodostettua avaruuden vektorien kantaa merkitään $\{\vec e_r, \vec e_\theta, \vec e_\varphi\}$
ja määritellään (vrt. kuvio edellä)
\[
\boxed{
\begin{array}{ll}
\ykehys\quad \vec e_r &= \ \sin{\theta} \cos{\varphi} \vec i + \sin{\theta}\sin{\varphi} \vec j 
                                                             + \cos{\theta} \vec k \quad \\[4pt]
\quad   \vec e_\theta &= \ \cos{\theta} \cos{\varphi} \vec i + \cos{\theta}\sin{\varphi} \vec j 
                                                             - \sin{\theta} \vec k \\[4pt]
\quad  \vec e_\varphi &= \ -\sin{\varphi} \vec i + \cos{\varphi} \vec j \akehys
\end{array}
}
\]
Tässä $\vec e_\varphi$ on sama kuin polaari- ja sylinterikoordinaatistoissa ja $\vec e_r$:n 
lauseke seuraa välittömästi edellä esitetyistä koordinaattien muunnoskaavoista:
\begin{align*}
\vec e_r\,=\,\frac{1}{\abs{\vec r\,}}\vec r\,
               &=\,\frac{1}{r}(x\vec i+y\vec j+z\vec k) \\
               &=\,\sin{\theta} \cos{\varphi} \vec i + \sin{\theta}\sin{\varphi} \vec j 
                                                                  + \cos{\theta} \vec k.
\end{align*}
Tämän jälkeen $\vec e_\theta$ on laskettavissa tiedosta, että 
$\{\vec e_r,\vec e_\theta,\vec e_\varphi\}$ on ortonormeerattu, oikeakätinen systeemi: 
\[
\vec e_\theta=\vec e_\varphi\times\vec e_r.
\]
\begin{Exa} Pisteen $P$ karteesiset koordinaatit ovat $(x,y,z)=(-2,-3,-6)$. Määritä $P$:n
pallokoordinaatit sekä vektorin $\vec v=\vec i-\vec j$ koordinaatit $(v_r,v_\theta,v_\varphi)$
pallokoordinaatiston kannassa $\{\vec e_r,\vec e_\theta,\vec e_\varphi\}$ ko.\ pisteessä.
\end{Exa}
\ratk Jos merkitään $\vec r=\overrightarrow{OP}=-2\vec i-3\vec j-6\vec k$, niin
$r=\abs{\vec r\,}=7$ ja
\[
\vec e_r = \frac{1}{r}\,\vec r =\frac{1}{7}(-2\vec i-3 \vec j-6\vec k).
\]
Tällöin
\begin{align*}
\cos\theta\,&= \vec e_r\cdot\vec k = -\frac{6}{7}\,, \quad \sin\theta=\frac{\sqrt{13}}{7}\,, \\
\cos\varphi &= \frac{\vec i\cdot(-2\vec i-3\vec j)}{\sqrt{2^2+3^2}}=-\frac{2}{\sqrt{13}}\,,
                                                 \quad \sin\varphi =-\frac{3}{\sqrt{13}}\,,
\end{align*}
joten $P$:n pallokoordinaatit ovat
\[
(r,\theta,\varphi) \approx (7,\,149.0\aste,\,236.3\aste).
\]
Kantavektoreiden $\vec e_\theta,\vec e_\varphi$ tarkat arvot ovat
\begin{align*}
\vec e_\theta  &=\,-\frac{6}{7}\left(-\frac{2}{\sqrt{13}}\right)\vec i
                  -\frac{6}{7}\left(-\frac{3}{\sqrt{13}}\right)\vec j
                  -\frac{\sqrt{13}}{7}\vec k, \\
               &=\,\frac{1}{7\sqrt{13}}\,(12\vec i+18\vec j-13\vec k), \\
\vec e_\varphi &=\,\frac{1}{\sqrt{13}}\,(3\vec i-2\vec j),
\end{align*}
joten kysytyt $\vec v$:n koordinaatit ovat
\begin{align*}
&v_r       = \vec v\cdot\vec e_r       =  \frac{1}{7}          \approx  0.143, \\
&v_\theta  = \vec v\cdot\vec e_\theta  = -\frac{6}{7\sqrt{13}} \approx -0.238, \\
&v_\varphi = \vec v\cdot\vec e_\varphi =  \frac{5}{\sqrt{13}}  \approx  1.387.
\end{align*}
Tarkistus:
\[
v_r^2+v_\theta^2+v_\varphi^2\,=\,\frac{1274}{13 \cdot 49}\,
                              =\,2\,=\,\abs{\vec v\,}^2 \quad\text{OK!} \loppu
\]

%\begin{multicols}{2} \raggedcolumns
\begin{Exa}
Lentokone lähtee päiväntasaajan pisteestä $A:$ $90^\circ$ läntistä pituutta ja lentää pitkin 
isoympyrää, joka kulkeee pisteen  $B:$ $45^\circ$ itäistä pituutta, $60^\circ$ pohjoista 
leveyttä, kautta. Määritä koneen lentosuunta pisteessä $B$ sekä kannassa 
$\{\vec i, \vec j, \vec k\}$ että suhteessa ilmansuuntiin pisteessä $B$.
\end{Exa}
\begin{figure}[H]
\begin{center}
\import{kuvat/}{kuvaII-23.pstex_t}
\end{center}
\end{figure}
%\end{multicols}
\ratk Oletetaan pallon säteeksi $R=1$, jolloin (ks.\ kuva)
\begin{align*}
\Vect{OA} &= - \vec j, \\
\Vect{OB} &= \sin{30^\circ}\cos{45^\circ} \vec i 
+ \sin{30^\circ}\sin{45^\circ} \vec j + \cos{30^\circ} \vec k \\
&= \frac{1}{2\sqrt{2}} \vec i + \frac{1}{2\sqrt{2}} \vec j +
\frac{\sqrt{3}}{2} \vec k.
\end{align*}
Koneen lentorata on tasossa, joka kulkee pisteiden $O, A, B$ kautta, joten tason
normaalivektori on
\[
\vec n = \Vect{OA} \times \Vect{OB} 
       = -\frac{\sqrt{3}}{2}\vec i + \frac{1}{2\sqrt{2}}\vec k.
\]
Koska lentorata pisteessä $B$ on myös kohtisuorassa vektoria $\Vect{OB}$ vastaan, on
lentosuunnan oltava sama kuin vektorilla
\[
\vec n\times\Vect{OB} 
   = \left|\begin{array}{ccc}
     \vec i & \vec j & \vec k \\
     -\tfrac{\sqrt{3}}{2} & 0 & \tfrac{1}{2\sqrt{2}} \\
     \tfrac{1}{2\sqrt{2}} & \tfrac{1}{2\sqrt{2}} & \tfrac{\sqrt{3}}{2}
     \end{array} \right|
   = -\frac{1}{8}\vec i + \frac{7}{8}\vec j - \frac{1}{4}\sqrt{\frac{3}{2}}\vec k.
\]
Tämän vektorin pituus on $\sqrt{56}/8$, joten lentosuunnan osoittava yksikkövektori on
\[
\underline{\underline{\vec v\,^0 = \frac{1}{\sqrt{56}}(-\vec i + 7\vec j -\sqrt{6}\vec k)}}.
\]
Itään osoittava yksikkövektori pisteessä $B$ on
\[
\vec e_\varphi \,=\, - \sin{45^\circ} \vec i + \cos{45^\circ} \vec j 
               \,=\, \frac{1}{\sqrt{2}}(-\vec i + \vec j),
\]
ja etelään osoittava yksikkövektori on
\begin{align*}
\vec e_\theta &= \cos{30^\circ} \cos{45^\circ} \vec i +
\cos{30^\circ} \sin{45^\circ} \vec j - \sin{30^\circ}\vec k \\
&= \frac{1}{2}\sqrt{\frac{3}{2}}\,(\vec i + \vec j) - \frac{1}{2} \vec k.
\end{align*}
Koska
\begin{align*}
\vec v\,^0 \cdot \vec e_\varphi &= \frac{2}{\sqrt{7}}\ \approx\, \cos{41^\circ}, \\
\vec v\,^0 \cdot \vec e_\theta &= \sqrt{\frac{3}{7}}\, \approx\, \cos{49^\circ},
\end{align*}
on lentosuunta $B$:ssä kaakosta hieman itään päin. \loppu
\begin{figure}[H]
\begin{center}
\import{kuvat/}{kuvaII-24.pstex_t}
\end{center}
\end{figure}

\Harj
\begin{enumerate}

\item
Seuraavassa on annettu piste $P$ joko karteesisessa, lieriö- tai pallokoordinaatistossa.
Laske $P$:n koordinaatit muissa kahdessa koordinaatistossa (tarkasti, jos mahdollista,
muuten likiarvoina). \vspace{1mm}\newline 
$P=(x,y,z): \quad$ 
a)\, $(1,1,1) \qquad\,$ 
b)\, $(2,3,-1) \qquad$ 
c)\, $(-1,-\sqrt{3},-5)$ \newline
$P=(r,\varphi,z): \quad$
d)\, $(\sqrt{2},\frac{\pi}{2},1) \quad\,$ 
e)\, $(1,\frac{\pi}{3},-1) \qquad$
f)\, $(6\sqrt{2},\frac{7\pi}{4},3)$ \newline
$P=(r,\theta,\varphi): \quad $
g)\, $(1,\frac{\pi}{2},\pi) \qquad$
h)\, $(\sqrt{3},\frac{3\pi}{4},\frac{3\pi}{4}) \quad$
i)\, $(3,\frac{2\pi}{3},\frac{5\pi}{6})$

\item
a) Tutki, millainen on polaarikoordinaattien avulla ilmaistu tasokäyrä
\[
S=\{P=(r,\varphi) \mid r=\cos\varphi+\sin\varphi,\ 
                       \varphi\in[-\tfrac{\pi}{4},\tfrac{3\pi}{4}]\}.
\]
b) Tason ympyrän $S$ keskipiste on $(x_0,y_0)$ ja säde $=R$. Näytä, että
$P=(x,y) \in S$ täsmälleen kun jollakin $\varphi\in[0,2\pi)$ pätee
\[ 
\begin{cases} \,x=x_0+R\cos\varphi \\ \,y=y_0+R\sin\varphi \end{cases}
\]
Miten tämä liittyy napakoordinaatistoon, ja mikä on vastaava väittämä koskien
avaruuden pallopintaa $\,S:\ (x-x_0)^2+(y-y_0)^2+(z-z_0)^2=R^2$\,?

\item
Pisteiden $P_1$ ja $P_2$ pallokoordinaatteja $(r_1,\theta_1,\varphi_1)$ ja 
$(r_2,\theta_2,\varphi_2)$ verrataan joukon 
$[0,\infty)\times[0,\pi]\times[0,2\pi)\subset\Rkolme$ alkioina. Luettele tapaukset,
joissa vertailu antaa tuloksen $P_1=P_2$, vaikka koordinaatit eivät ole samat.

\item
Pisteessä $P$, jonka karteesiset koordinaatit tunnetaan, vaikuttaa voima $\vec F$.
Muunna $\vec F$ annettuun kantaan ko.\ pisteessä. \vspace{1mm}\newline
a)\, $P=(-2,3,6), \quad \vec F=\vec i-\vec j+\vec k, \quad 
                        \{\vec e_r,\vec e_\varphi,\vec e_z\}$ \newline
b)\, $P=(-2,3,6), \quad \vec F=\vec i+\vec j+\vec k, \quad 
                        \{\vec e_r,\vec e_\theta,\vec e_\varphi\}$ \newline
c)\, $P=(6,-6,-3), \quad \vec F=-9\vec e_r\ \text{(pallok.)}, \quad 
                        \{\vec i,\vec j,\vec k\}$ \newline
d)\, $P=(-6,-6,3), \quad \vec F=-9\vec e_r\ \text{(lieriök.)}, \quad 
                        \{\vec e_r,\vec e_\theta,\vec e_\varphi\}$ \newline
e)\, $P=(-1,\sqrt{3},-2\sqrt{3}), \quad \vec F=\vec e_\theta-\vec e_\varphi\
                         \text{(pallok.)}, \quad \{\vec i,\vec j,\vec k\}$

\item
Karteesisen koordinaatiston origo on maapallon keskipisteessä ja vektori $\vec k$ osoittaa
pohjoisnavalle päin. Maapallon pinnalla eteläisen pallonpuoliskon pisteessä, jonka
pallonpintakoordinaatit ovat $\theta=135\aste,\ \varphi=120\aste$, laiva on matkalla
luoteeseen ja lentokone koilliseen. Määritä laivan ja lentokoneen liikesuunnat
yksikkövektoreina a) pallokoordinaatistossa vektorien $\{\vec e_\theta,\vec e_\varphi\}$ 
avulla, b) karteesisessa koordinaatistossa vektorien $\{\vec i,\vec j,\vec k\}$ avulla.

\item
Helsingistä $(60^{\circ}N,25^{\circ}E)$ lennetään lyhintä tietä Tokioon 
 $(36^{\circ}N,140^{\circ}E)$. Miten pitkä on matka ja mihin 
ilmansuuntaan on Helsingistä lähdettävä? Maapallon säde on $6370$ km.

\item (*)
Pallokoordinaatin $\theta $ avulla ilmaistu puolikartio $\theta =30^{\circ} $
leikataan tasolla $T: y+z-4=0$. Ilmaise syntyvä leikkauskäyrä  $S\subset E^3$ ensin 
pallokoordinaattien $(r,\theta ,\varphi)$ avulla. Muunna sitten käyrä tason $T$ 
napakoordinaatistoon, jossa origo on pisteessä $(0,0,4)$ ja napakulmaa mitataan sunnasta
$\vec i$, ja edelleen $T$:n vastaavaan karteesiseen koordinaatistoon. Hahmottele käyrä 
viimeksi mainitussa koordinaatistossa.

\end{enumerate}
