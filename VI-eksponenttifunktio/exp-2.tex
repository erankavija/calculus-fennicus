\section{Funktiot $e^x$ ja $\ln x$} \label{exp(x) ja ln(x)}
\alku
\index{funktio C!f@$\exp$ ($e^x$, $e^z$)|vahv}

Edellisessä luvussa eksponenttifunktio $E(x)$ konstruoitiin jatkamalla tunnettu funktio
$\,b^x,\ x\in\Q$. Tällä tavoin tuli varmistetuksi, että eksponenttifunktioita on olemassa,
että luku $b=E(1)$ määrää $E(x)$:n yksikäsitteisesti, ja että eksponenttifunktion arvot
voidaan laskea numeerisesti suoraan määritelmästä käsin. Mitä \pain{ei} tullut varmistetuksi
on esimerkiksi, onko $E(x)$ mahdollisesti myös derivoituva, eikä vain jatkuva. Yleisemminkin
jäi avoimeksi, kuinka säännöllisestä funktiosta on ylipäänsä kyse. Seuraavassa etsitään
vastausta näihin kysymyksiin lähestymällä eksponenttifunktiota toisesta suunnasta.

Lähtökohdaksi otetaan eksponenttifunktion aksiooman E3 (ks.\ edellinen luku) vahvistaminen
muotoon \index{eksponenttifunktio!a@aksioomat}
\begin{itemize}
\item[(E3')]  $E$ on derivoituva pisteessä $x=0$.
\end{itemize}
Koska tiedetään, että $E(0)=1$ (Lause \ref{eksponenttifunktion ominaisuudet} (b)), niin
aksiooman E3' mukaan on olemassa luku $a\in\R$ siten, että
\[
E'(0) \,=\, \lim_{x \kohti 0} \frac{E(x)-1}{x} \,=\, a.
\]
Kun yhdistetään tämä tulos ja aksiooma E4, niin seuraa
\[
\lim_{\Delta x \kohti 0} \frac{E(x+\Delta x)-E(x)}{\Delta x}
 \,=\, \lim_{\Delta x \kohti 0} E(x)\,\frac{E(\Delta x)-1}{\Delta x} \,=\, aE(x), \quad x\in\R.
\]
Siis $E$ on derivoituva jokaisessa pisteessä $x\in\R$ ja
\index{derivoimissäännöt!g@eksponenttifunktio}%
\[
E'(x)=aE(x), \quad x\in\R.
\]
On päätelty, että jos aksioomat E1,\,E2,\,E3',\,E4 toteuttava funktio $E(x)$ on
olemassa, niin tämä on derivoituva $\R$:ssä ja jollakin $a\in\R$ ratkaisu probleemalle
\[ 
\text{(P)} \quad \begin{cases} \,y'=ay, \quad x\in\R, \\ \,y(0)=1. \end{cases}
\]
Tässä on kyse alkuarvotehtävästä, jossa on ratkaistava differentiaaliyhtälö
$y'=ay$ alkuehdolla $y(0)=1$ (vrt.\ Luvussa \ref{väliarvolause 2} tarkasteltu
vastaava probleema differentiaaliyhtälölle $y'=f(x)$).
\index{differentiaaliyhtälö!a@eksponenttifunktion}%
\begin{Lause} \label{exp-dy} Probleemalla (P) on ratkaisu $y(x)=E(x)$, missä
\[ 
E(x) = \sum_{k=0}^\infty \frac{(ax)^k}{k!} = 1 + ax + \frac{(ax)^2}{2} + \ldots 
\]
Tämä ratkaisu on yksikäsitteinen, ja $E(x)$ on eksponenttifunktio.
\end{Lause}
\tod Potenssisarjan derivoimissäännöstä (Lause \ref{potenssisarja on derivoituva}) nähdään
välittömästi, että väitetty $y(x)$ on probleeman (P) ratkaisu. Nähdään myös, että funktiolla
$E(x)$ on eksponenttifunktiolta vaadittavat ominaisuudet (E1)--(E3), sillä $E(0)=1$ ja $E$
on määritelty koko $\R$:ssä suppenevan potenssisarjan summana, siis derivoituvana ja jopa
sileänä funktiona.

Näytetään seuraavaksi, että $E(x)$ toteuttaa myös eksponenttifunktion aksiooman (E4). Tätä
silmällä pitäen tarkastellaan ensin funktiota
\[
u(x) = E(x)E(-x), \quad x\in\R.
\]
Koska $E'=aE$, niin tulon derivoimissääntöä käyttäen todetaan, että
\begin{equation} \label{exp-apuy 1}
u'(x)=0, \quad x\in\R.
\end{equation}
Koska $u(0)=1$ ja siis $u'=0$, niin seuraa, että $u(x)=1\ \forall x\in\R$
(Lause \ref{Integraalilaskun peruslause}). Siis $E(x)E(-x)=1\ \forall x\in\R$, ja näin ollen 
$E(x) \neq 0\ \forall x\in\R$ ja
\begin{equation} \label{exp-apuy 2}
E(-x) = [E(x)]^{-1}, \quad x\in\R.
\end{equation}
Olkoon seuraavaksi $t\in\R$ kiinnitetty ja tarkastellaan funktiota
\[
u(x) = E(t+x)E(-t)E(-x), \quad x\in\R.
\]
Derivoimalla muuttujan $x$ suhteen ja käyttämällä differentiaaliyhtälöä $E'=aE$ todetaan, että 
jälleen pätee \eqref{exp-apuy 1}. Koska tuloksen \eqref{exp-apuy 2} mukaan on $u(0)=1$, niin 
päätellään kuten edellä ja käyttämällä uudelleen tulosta \eqref{exp-apuy 2}, että
\[
u(x)=E(t+x)E(-t)E(-x)=1\,\ \ekv\,\ E(t+x)= E(t)E(x), \quad x\in\R.
\]
Tämä pätee jokaisella $t\in\R$, joten aksiooma E4 on voimassa funktiolle $E(x)$. Siis 
$E(x)$ on eksponenttifunktio (jokaisella $a\in\R$).

Enää on todistamatta probleeman (P) ratkaisun yksikäsitteisyys. Tätä silmällä pitäen
olkoon $y(x)$ (P):n (mikä tahansa) ratkaisu ja tarkastellaan funktiota
\[
u(x)=y(x)E(-x), \quad x\in\R.
\]
Derivoimalla ja käyttämällä differentiaaliyhtälöitä $y'=ay$ ja $E'=aE$ todetaan, että jälleen 
pätee \eqref{exp-apuy 1}. Koska $u(0)=1$, niin seuraa
\[
u(x)=y(x)E(-x)=1\,\ \ekv\,\ y(x)=E(x), \quad x\in\R. \loppu
\]

Sanotaan jatkossa \kor{peruseksponenttifunktioksi}, symboli $\,\exp$, probleeman (P) ratkaisua,
kun $a=1$. Määritelmä potenssisarjana on siis
\[
\boxed{\quad \exp(x) = \sum_{k=0}^\infty \frac{x^{\ygehys k}}{\agehys k!}\,. \quad}
\]
Jos merkitään $e=\exp(1)$, niin edellisen luvun merkinnöin voidaan kirjoittaa
\[
\boxed{\quad\kehys \exp(x) = e^x. \quad}
\]
Tässä $e$ on todellakin Neperin luku, sillä pätee (ks.\ Propositiot \ref{Neperin jonot} ja
\ref{e:n sarja})
\[
\exp(1) = \sum_{k=0}^\infty \frac{1}{k!} = \lim_{n\kohti\infty}\left(1+\frac{1}{n}\right)^n
                                         = e = 2.71828182845905..
\]
Tässä jälkimmäinen raja-arvotulos on erikoistapaus yleisemmästä tuloksesta
(ks.\ Harj.teht.\,\ref{H-exp-2: eksoottisia raja-arvoja})
\[
\boxed{\Akehys\quad \lim_n \Bigl(1+\frac{\ygehys x}{n}\Bigr)^n = e^x, \quad x\in\R. \quad }
\]

Peruseksponenttifunktion $e^x$ määritelmästä seuraa derivoimissääntö
\[
\boxed{\kehys\quad \dif e^x = e^x. \quad}
\]
Monet funktion $e^x$ ominaisuuksista voidaan helposti johtaa tästä säännöstä tai 
potenssisarjaesityksestä. Esimerkiksi potenssisarjaesityksestä seuraa 
\[
e^x > x^m/m!\ \ \forall m \in \N, \quad \text{kun}\ x>0, 
\]
mistä on edelleen helposti pääteltävissä raja-arvotulokset
\begin{equation} \label{exp-raja-arvot}
\boxed{\quad \lim_{x\kohti\infty} x^{\ygehys -\alpha}e^x = \infty 
  \qekv \lim_{x\kohti\infty}x^\alpha e^{-x} = 0 \quad \forall \alpha \in \R_{\agehys}. \quad}
\end{equation}
Näiden mukaan funktio $e^x$ kasvaa nopeammin, ja funktio $e^{-x}$ vähenee nopeammin, kuin mikään
potenssifunktio, kun $x\kohti\infty$. Potenssisarjan perusteella voidaan myös $e^x$:n 
numeroarvoja laskea helposti, sillä pienillä $\abs{x}$:n arvoilla (esim.\ kun $\abs{x} \le 1$) 
sarja suppenee nopeasti ja suuremmilla taas voidaan käyttää kaavaa $e^x = (e^{x/n})^n$. 
Derivoimissäännöstä (myös potenssisarjasta) nähdään, että funktio $e^x$ on $\R$:ssä 
mielivaltaisen monta kertaa derivoituva (sileä) funktio. 
\begin{Exa} Jatka funktio $f(x)=(e^x-1-x)/x^2$ pisteeseen $x=0$.
\end{Exa}
\ratk $e^x$:n potenssisarjasta nähdään, että
\[
f(x)= \frac{1}{2!} + \frac{x}{3!} + \frac{x^2}{4!} + \ldots\ 
                           =\ \sum_{k=0}^\infty \frac{x^k}{(k+2)!}\,, \quad x \neq 0.
\]
Kun asetetaan $f(0)=1/2$, niin $\forall x\in\R$ pätee siis
\[
f(x)\,=\,\sum_{k=0}^\infty \frac{x^k}{(k+2)!} 
    \,=\,\begin{cases}
         \,\dfrac{e^x-1-x}{x^2}\,, &\text{kun}\ x \neq 0, \\ \tfrac{1}{2}\,, &\text{kun}\ x=0.
         \end{cases} 
\]
Tässä potenssisarjan suppenemissäde on $\rho=\infty$, joten $f$ ei ole vain jatkuva, vaan
$\R$:ssä (myös pisteessä $x=0$\,!) mielivaltaisen monta kertaa derivoituva. \loppu

Edellisessä luvussa todettiin, että yleinen eksponenttifunktio on muotoa $E(x)=b^x$, missä 
$b=E(1) \in \R_+$. Laskuissa usein kätevämpi on peruseksponenttifunktiosta johdettu 
vaihtoehtoinen esitystapa
\[
\boxed{\kehys\quad E(x)=e^{ax}, \quad E(1)=b=e^a. \quad}
\]
Luku $a\in\R$ määräytyy yksikäsitteisesti yhtälöstä $e^a=b$, koska $e^x:\ \R\kohti\R_+$ on 
bijektio. Jokainen eksponenttifunktio on siis muotoa $E(x)=e^{ax}$ jollakin $a\in\R$. 

Todettakoon vielä lähtökohtana olleesta probleemasta (P), että sen ratkaisuina saadaan siis 
kaikki eksponenttifunktiot $y(x)=e^{ax}$, kun $a$:n arvoa vaihdellaan. Jos ratkaisuksi
halutaan tietty eksponenttifunktio $E(x)$, niin $a$ määräytyy joko em.\ tavalla luvun
$b=E(1)$ kautta tai ehdosta $E'(0)=a$. Jos $a$ on kiinnitetty, niin differentiaaliyhtälön
$y'=ay$ y\pain{leinen} \pain{ratkaisu} (=kaikkien ratkaisujen joukko) on
(Harj.teht.\,\ref{H-exp-2: exp-dy})
\[
y(x)=Ce^{ax}, \quad C\in\R.
\]
Alkuehdon $y(0)=y_0$ toteuttava ratkaisu on tämän mukaan yksikäsitteinen ja saadaan
valitsemalla $C=y_0$.

\subsection{Logaritmifunktio}
\index{logaritmifunktio|vahv}
\index{funktio C!g@$\log$, $\ln$|vahv}

Koska eksponenttifunktio $E:\R\kohti\R_+$ on bijektio kun $E(1)=b\neq 1$, on sillä 
käänteisfunktio, jota sanotaan \kor{logaritmifunktioksi}. Käänteisfunktiota merkitään
symbolilla $\log_b\,$:
\[
y=b^x \ \ekv \ x=\log_b y.
\]
\index{luonnollinen logaritmi}%
Peruseksponenttifunktion $e^x$ käänteisfunktiota sanotaan \kor{luonnolliseksi logaritmiksi}, 
symbolina $\ln$ (toisinaan $\log$):
\[
\boxed{\kehys\quad y=e^x \ \ekv \ x=\ln y. \quad}
\]
Luonnollinen logaritmi, samoin kuin muutkin logaritmifunktiot, on koko määrittelyjoukossaan 
($=\R_+$) jatkuva ja aidosti kasvava, ja $\ln x \kohti \infty$ kun $x\kohti\infty$ ja 
$\ln x \kohti -\infty$ kun $x\kohti 0^+$. Sijoituksilla $x = \pm \ln t\ \ekv\ t = e^{\pm x}$ 
voidaan raja-arvotuloksista \eqref{exp-raja-arvot} päätellä, että rajoilla $x\kohti\infty$ ja
$x\kohti 0^+$ funktio $\abs{\ln x}$ kasvaa hitaammin kuin mikään potenssifunktio:
\begin{equation} \label{ln-raja-arvot}
\boxed{\quad \lim_{\akehys x\kohti\infty}x^{\ykehys -\alpha}\ln x\ =
             \lim_{x\kohti 0^+}x^\alpha \ln x = 0 \quad \forall \alpha\in\R_+. \quad}
\end{equation}
\begin{figure}[H]
\setlength{\unitlength}{1cm}
\begin{center}
\begin{picture}(8,4)(-1,-1)
\put(-1,0){\vector(1,0){8}} \put(6.8,-0.4){$x$}
\put(0,-2){\vector(0,1){5}} \put(0.2,2.8){$y$}
\put(1,0){\line(0,-1){0.1}}
\put(0,1){\line(-1,0){0.1}}
\put(0.93,-0.4){$\scriptstyle{1}$}
\put(-0.4,0.9){$\scriptstyle{1}$}
\put(5,1.3){$\scriptstyle{y=\ln x}$}
\curve(
    0.2500,   -1.3863,
    0.7500,   -0.2877,
    1.2500,    0.2231,
    1.7500,    0.5596,
    2.2500,    0.8109,
    2.7500,    1.0116,
    3.2500,    1.1787,
    3.7500,    1.3218,
    4.2500,    1.4469,
    4.7500,    1.5581,
    5.2500,    1.6582,
    5.7500,    1.7492,
    6.2500,    1.8326,
    6.7500,    1.9095)
\end{picture}
%\caption{$y=\ln x$}
\end{center}
\end{figure}
Eksponentti- ja logaritmifunktioilla laskettaessa käytetään yleisimmin 
peruseksponenttifunktiota $e^x$ ja sen käänteisfunktiota $\ln x$. Useimmin tarvittavat, suoraan
määritelmästä ja eksponenttifunktion ominaisuuksista seuraavat kaavat ovat
(Harj.teht.\,\ref{H-exp-2: kaavat})\,:
\begin{alignat}{3}
e^{\ln x} &= x,               &\quad &x\in\R_+.            \label{exp-kaava 1}\\
\ln(xy)   &= \ln x + \ln y,   &\quad &x,y\in\R_+.         \label{exp-kaava 2}\\
\ln (1/x) &= -\ln x,          &\quad &x\in\R_+.           \label{exp-kaava 3}\\
\ln x^y   &= y\ln x,          &\quad &x\in\R_+, \ y\in\R. \label{exp-kaava 4}
\end{alignat}
Kaavat \eqref{exp-kaava 1} ja \eqref{exp-kaava 4} yhdistämällä saadaan laskukaava
\begin{equation} \label{exp-kaava 5}
\boxed{\kehys\rule{0mm}{4.5mm}\kehys\quad x^y=e^{y\ln x}, \quad x\in\R_+\,,\ y\in\R. \quad}
\end{equation}
Kaavasta \eqref{exp-kaava 1}, eksponenttifunktion derivoimissäännöstä ja yhdistetyn funktion 
derivoimissäännöstä saadaan johdetuksi logaritmifunktion $\ln x$ derivoimissääntö:
\index{derivoimissäännöt!h@logaritmifunktio}% 
\[ 
1=\dif x = \dif e^{\ln x} = e^{\ln x} \dif \ln x 
         = x \dif \ln x \qimpl \dif \ln x = 1/x, \quad x>0. 
\]
Tämän perusteella on $\dif \ln(-x) = (-1)/(-x) = 1/x,\ x<0$, joten saadaan yleisempi sääntö
\begin{equation} \label{exp-kaava 6}
\boxed{\kehys\quad \dif \ln \abs{x} = \dfrac{1}{x}\,, \quad x \neq 0. \quad} 
\end{equation}
\begin{Exa} Kaavan \eqref{exp-kaava 5} ja säännön \eqref{exp-kaava 6} perusteella
\[ 
\dif x^x = \dif e^{x\ln x} = e^{x\ln x}\dif(x\ln x) = (\ln x + 1)\,x^x, \quad x>0.
\]
Tuloksesta nähdään, että $f(x)=x^x$ on aidosti vähenevä välillä $(0,1/e]$ ja aidosti kasvava
välillä $[1/e,\infty)$. Siis $f$:n absoluuttinen minimiarvo on $f_{min}=f(1/e)=e^{-1/e}$.
Rajalla $x \kohti 0^+$ on $f(0^+)=1$ (Harj.teht.\,\ref{H-exp-2: raja-arvoja}e). \loppu
\end{Exa}
\begin{multicols}{2} \raggedcolumns
\begin{Exa} Matemaatikon mökki on kahden tien risteyksessä. Selvitä mökin sijainti, kun 
tiedetään, että kummallakin tiellä on oheinen viitta. 
\vspace{1cm}
\[ 
\boxed{ \rule[-2cm]{0mm}{2.5cm} } \boxed{ \quad x^y = y^x \quad } 
\]
\end{Exa}
\end{multicols}
\ratk Tie \,1\, on ilmeisesti puolisuora $y=x$, $x>0$. Tiellä \,2\, taas ovat esim.\ pisteet 
$(2,4)$ ja $(4,2)$. Teiden leikkauspisteen selvittämiseksi käytetään ensin kaavoja 
\eqref{exp-kaava 1} ja \eqref{exp-kaava 4} (ol.\ $x,y\in\R_+$):
\[
x^y=y^x\ \ekv\ e^{y\ln x} = e^{x\ln y}\ \ekv\ y\ln x = x\ln y\ 
                                        \ekv\ \ln y/y = \ln x/x.
\]
Säännöstä \eqref{exp-kaava 6} ja Luvun \ref{derivaatta} derivoimissäännöistä seuraa
\[ 
\dif \ln x/x = x^{-2}(1 - \ln x), \quad x>0, 
\]
joten funktio $f(x)=\ln x/x$ on aidosti kasvava välillä $(0,e]$ ja aidosti vähenevä välillä 
$[e,\infty)$. Näin ollen, jos $x\in\R_+$ ja $x>1$, $x\neq e$, niin yhtälöllä $f(y)=f(x)$ on 
kaksi ratkaisua $y$, joista toinen on $y=x$ (ks.\ kuvio). Kun $x=e$ tai $x \le 1$, on ainoa 
ratkaisu $y=x$. Päätellään, että mökki on pisteessä $(e,e)$. \loppu
\begin{multicols}{2} \raggedcolumns
\begin{figure}[H]
\setlength{\unitlength}{1cm}
\begin{center}
\begin{picture}(5,5)(0,-2)
\put(0,0){\vector(1,0){5}} \put(4.8,-0.4){$x$}
\put(0,-2){\vector(0,1){5}} \put(0.2,2.8){$y$}
\put(1,0){\line(0,-1){0.1}}
\put(0,2){\line(-1,0){0.1}}
\put(1.0,-0.3){$\scriptstyle{1}$}
\put(-0.4,1.9){$\scriptstyle{2}$}
\put(1.0,-1.0){$\scriptstyle{y=2e\,\ln x/x}$}
\dashline{0.1}(2.7183,0)(2.7183,2)
\dashline{0.1}(4.5,1.8171)(1.87,1.8171) \dashline{0.1}(4.5,1.8171)(4.5,0)
\put(2.64,-0.2){$\scriptstyle{e}$} \put(4.42,-0.2){$\scriptstyle{x}$}
%\curve(
%    0.5000,   -1.5367,
%    1.0000,         0,
%    1.5000,    1.4696,
%    2.0000,    1.8842,
%    2.5000,    1.9926,
%    3.0000,    1.9909,
%    3.5000,    1.9459,
%    4.0000,    1.8842,
%    4.5000,    1.8171,
%    5.0000,    1.7500)
\curve(
     0.7500,   -2.0853,
     1.0000,         0,
     1.2500,    0.9705,
     1.5000,    1.4696,
     1.7500,    1.7385,
     2.0000,    1.8842,
     2.2500,    1.9594,
     2.5000,    1.9926,
     2.7500,    1.9999,
     3.0000,    1.9909,
     3.2500,    1.9716,
     3.5000,    1.9459,
     3.7500,    1.9162,
     4.0000,    1.8842,
     4.2500,    1.8509,
     4.5000,    1.8171,
     4.7500,    1.7834,
     5.0000,    1.7500)
\end{picture}
%\caption{$y=(e/x)\ln x$}
\end{center}
\end{figure}
\begin{figure}[H]
\setlength{\unitlength}{1cm}
\begin{center}
\begin{picture}(5,5)(0,0)
\put(0,0){\vector(1,0){5}} \put(4.8,-0.4){$x$}
\put(0,0){\vector(0,1){5}} \put(0.2,4.8){$y$}
\put(1,0){\line(0,-1){0.1}}
\put(0,1){\line(-1,0){0.1}}
\put(0.93,-0.4){$\scriptstyle{1}$}
\put(-0.4,0.9){$\scriptstyle{1}$}
\put(0,0){\line(1,1){4}}
\put(1.4,1){$\scriptstyle{\text{tie 1}}$}
\put(4.7,2){$\scriptstyle{\text{tie 2}}$}
\curve(
   2.7183,             2.7183,
   3.2183,             2.3356,
   3.7183,             2.0981,
   4.2183,             1.9360,
   4.7183,             1.8182)
\curve(
   2.7183,             2.7183,
   2.3356,             3.2183,
   2.0981,             3.7183,
   1.9360,             4.2183,
   1.8182,             4.7183)
\end{picture}
%\caption{$y^x=x^y$}
\end{center}
\end{figure}
\end{multicols}
\begin{Exa}
$f(x)=\ln\left|\dfrac{1-\cos x}{\sin x}\right|, \quad f'(x)=\,$?
\end{Exa}
\ratk Säännön \eqref{exp-kaava 6} ja Lukujen \ref{derivaatta}--\ref{kaarenpituus}
derivoimissääntöjen perusteella
\begin{align*}
\dif f(x)&=\frac{\sin x}{1-\cos x}\cdot \left[1-\frac{(1-\cos x)\cos x}{\sin^2 x}\right] \\
         &=\frac{\sin x}{1-\cos x}\cdot\frac{\sin^2 x+\cos^2 x-\cos x}{\sin^2 x} \\
         &=\frac{1}{\sin x}\,. \loppu
\end{align*}

Esimerkin tulos, ja muunnoksella $x\hookrightarrow\pi/2-x$ saatava vastaava tulos, on syytä 
panna korvan taakse:
\[ \boxed{
\begin{aligned}
\ykehys\quad \dif\ln\left|\dfrac{1-\cos x}{\sin x}\right|\ 
                 &=\ \dfrac{1}{\sin x}\,, \quad\quad x\in\R,\ \ \sin x\neq 0, \\[1mm]
             \dif\ln\left|\dfrac{1-\sin x}{\cos x}\right|\ 
                 &=\ -\dfrac{1}{\cos x}\,, \quad x\in\R,\ \ \cos x\neq 0. \quad\akehys
\end{aligned} } \]
Tässä on itse asiassa
$\,\D \dfrac{1-\cos x}{\sin x} = \dfrac{\sin x}{1+\cos x}=\tan\dfrac{x}{2}\,$
(ks.\ Harj.teht.\,\ref{trigonometriset funktiot}:\ref{H-II-5: trigtuloksia}b).

\Harj
\begin{enumerate}

\item \label{H-exp-2: exp-dy}
Näytä, että differentiaaliyhtälön $y'=ay$ yleinen ratkaisu $\R$:ssä on $y(x)=Ce^{ax}$.\,
\kor{Vihje}: Tutki funktiota $u(x)=e^{-ax}y(x)$.

\item \label{H-exp-2: kaavat}
a)--d) Perustele laskukaavat (5)--(8).

\item \label{H-exp-2: raja-arvoja}
Määritä tai näytä oikeaksi raja-arvot
\begin{align*}
&\text{a)}\ \lim_{x \kohti 0} \frac{e^{2x}-1-2x}{x^2} \qquad\qquad
 \text{b)}\ \lim_{x \kohti 0} \frac{1-6x+18x^2-e^{-6x}}{x^3} \\
&\text{c)}\ \lim_{x \kohti 0} \frac{3e^x-e^{-x}-2e^{2x}}{x^2} \qquad\
 \text{d)}\ \lim_{x\kohti\infty} x\left(e^{4/x}-e^{1/x}\right) \\[2mm]
&\text{e)}\ \lim_{x \kohti 0^+} x^x=1 \qquad 
 \text{f)}\ \lim_{x \kohti 0^+} x^{x^x}=0 \qquad
 \text{g)}\ \lim_{x \kohti 0^+} x^{\sqrt{x}}=1
\end{align*}

\item
Derivoi seuraavat funktiot: \newline
a) \ $2^x \quad$ 
b) \ $e^{\sqrt{x}} \quad$ 
c) \ $\pi^{1/x} \quad$ 
d) \ $\ln(\ln x)\quad$
e) \ $x^{2x} \quad$ 
f) \ $x^{x^x}$

\item
Määritä seuraavien funktioiden paikalliset ääriarvokohdat ja absoluuttiset maksimi- ja 
minimarvot, sikäli kuin olemassa. Hahmottele myös kuvaajat. \newline
a) \ $x\ln x,\quad$ b) \ $\sqrt[10]{\abs{x}}\ln\abs{x},\quad$ c) \ $\abs{x}e^{1/x^2},\quad$
d) \ $x^{-\pi}\ln x,\quad$ e) \ $x^{1/x}$. 

\item
a) \ Millä $a$:n arvoilla funktio $f(x)=2\ln x +x^2-ax+1$ on aidosti kasvava välillä
$(0,\infty)\,$? \newline
b) \ Näytä, että jos $0<a<b$, niin 
$\,\displaystyle{1-\frac{a}{b}\,<\,\ln \frac{b}{a}\,<\,\frac{b}{a}-1}$. \newline
c) \ Todista laskukaavat $\ \log_a b\cdot\log_b a=1\ $ ja $\ \log_b a\cdot\log_c b=\log_c a$.

\item
Määritä funktion $f(t)=e^{-x}\sin x$ paikalliset ääriarvokohdat sekä absoluuttiset maksimi-
ja minimiarvot välillä $[0,\infty)$ sekä hahmottele $f$:n kuvaaja.

\item
Päättele, että kiintopisteiteraatio $x_{n+1}=e^{-x_n},\, n=0,1,\dots$
suppenee jokaisella $x_0\in\R$. Määritä kiintopiste iteroimalla ensin
kolme kertaa alkuarvauksesta $x_0=0$ ja kiihdyttämällä sen jälkeen Newtonin menetelmällä.

\item
Näytä, että differentiaaliyhtälön $y''-y'=0$ yleinen ($\R$:ssä kahdesti derivoituva) ratkaisu on
$y(x)=C_1 e^x +C_2\,,\ C_1\,,C_2\in\R$.

\item
a) Näytä, että funktiolle
\[
f(x) = \begin{cases}
       e^{-\frac{1}{x^2}}, &\text{kun}\ x \neq 0 \\ 0, &\text{kun}\ x=0
       \end{cases}
\]
pätee $f^{(k)}(0)=0\ \forall k\in\N$. Hahmottele funktion kuvaaja. \vspace{1mm}\newline
b) Halutaan määritellä funktio $f$ ehdoilla \vspace{2mm}\newline
(i) \   $\,\ f(x)>0\,$ kun $\,x\in(0,1)$ \newline
(ii)\   $\,\ f(x)=0\,$ kun $\,x \le 0\,$ tai $\,x \ge 1$ \newline
(iii) \ $f$ on koko $\R$:ssä sileä, eli mielivaltaisen monta kertaa derivoituva \vspace{2mm}
\newline
Näytä, että eräs vaatimukset täyttävä funktio on
\[
f(x)=\begin{cases} 
     e^{-\frac{1}{x(1-x)}}, &\text{kun}\ x\in(0,1) \\ 0, &\text{muulloin}
     \end{cases}
\]
Hahmottele tämän kuvaaja. \vspace{1mm}\newline
c) Näytä, että funktio
\[
f(x)=\begin{cases} 
     0,             &\text{kun}\ x/\pi\in\Z \\ e^{-\cot^2 x}, &\text{muulloin}
     \end{cases}
\]
on koko $\R$:ssä mielivaltaisen monta kertaa derivoituva (sileä) funktio. Hahmottele $f$:n
kuvaaja välillä $[0,\pi]$.

\item (*)
Millä $k$:n arvoilla käyrällä $y=\ln|x|$ ja suoralla $y=k(x-1)+1$ on täsmälleen yksi yhteinen
piste?

\item (*)
Millä $a$:n arvoilla yhtälöllä $\,e^x=x+ax^2\,$ on reaalisia ratkaisuja a) ei yhtään, 
b) vain yksi, c) täsmälleen kaksi, d) kolme tai enemmän? Määritä ratkaisut numeerisin keinoin
tapauksissa $a=1$ ja $a=2$.

\item (*)
a) Näytä, että
\[
1+x\,\le\,e^x\,\le\,1+x+(e-2)\,x^2 \quad \forall\,x\in[0,1].
\]
b) Olkoon $a \ge 1$ ja $n\in\N,\ n\ge\ln a$. Näytä, että
\[
1+\frac{\ln a}{n}\,\le\,\sqrt[n]{a}\,\le\,1+\frac{\ln a}{n}+(e-2)\left(\frac{\ln a}{n}\right)^2.
\]
Kuinka suuri on approksimaation $\,\sqrt[n]{a} \approx 1+\ln a/n\,$ virhe todellisuudessa, kun
$a=\pi$ ja $n=100\,$?

\item (*)
Näytä, että differentiaaliyhtälön $y'=2xy$ yleinen ratkaisu $\R$:ssä on
$y(x)=Ce^{x^2},\ C\in\R$.

\item (*) \index{Hermiten!a@polynomi}
\kor{Hermiten polynomi} $H_n(x)$ määritellään derivoimiskaavalla
\[
D^ne^{-x^2}=H_n(x)e^{-x^2}, \quad n\in\N\cup\{0\}.
\]
Näytä, että pätee palautuskaava
\[
H_{n+1}(x)=-2xH_n(x)-2nH_{n-1}(x), \quad n\in\N.
\]
 
\item (*) \label{H-exp-2: eksoottisia raja-arvoja}
Tarkastellaan funktioita
\[
f(x)=\left(1+\frac{1}{x}\right)^x,\,\ x\in(0,\infty), \qquad
g(x)=\left(1-\frac{1}{x}\right)^x,\,\ x\in[1,\infty).
\]
a) Näytä, että $f$ ja $g$ ovat määrittelyväleillään aidosti kasvavia. \newline
b) Näytä, että
\begin{align*}
&\lim_{x\kohti\infty}\left(1+\frac{1}{x}\right)^x
    =\lim_{n\kohti\infty}\left(1+\frac{1}{n}\right)^n=e, \\
&\lim_{x\kohti\infty}\left(1-\frac{1}{x}\right)^x
    =\lim_{n\kohti\infty}\left(1-\frac{1}{n}\right)^n=\frac{1}{e}\,.
\end{align*}
c) Näytä, että 
$\,\displaystyle{\lim_{n\kohti\infty}\left(1+\frac{x}{n}\right)^n=e^x\ \forall x\in\R}$.

\item (*) \index{zzb@\nim!Matemaatikon mökki}
(Matemaatikon mökki) Matemaatikon mökki on koordinaatistossa, jonka origo on Helsingissä,
positiivinen $x$-akseli osoittaa itään ja pituusyksikkö $=100$ km. Mökki on erään tien
varressa kohdassa, jossa tie on itä-länsi-suuntainen. Määritä mökin sijainti, kun tiedetään,
että tien yhtälö on
\[
(2x)^y=y^{3x} \quad (x,y>0)
\]
ja lisäksi tiedetään, että tien päätepiste a) on Helsinki, b) ei ole Helsinki.

\item (*) \label{H-exp-2: kosini ja sini} \index{zzb@\nim!Pieniä ihmeitä}
(Pieniä ihmeitä) Olkoon $A\in\R$. Halutaan löytää $\R$:ssä derivoituvat funktiot $u$ ja $v$,
jotka ratkaisevat alkuarvotehtävän
\[
\text{(P)} \quad \begin{cases} 
                 \,u'=-v,\,\ v'=u, \quad x\in\R, \\ \,u(0)=A,\,\ v(0)=0. 
                 \end{cases}
\]
a) Totea: Eräs ratkaisu on $u(x)=A\cos x,\ v(x)=A\sin x$. \newline
b) Todista: Jos $u$ ja $v$ ovat mikä tahansa (P):n ratkaisu, niin
\[
[u(x)]^2+[v(x)]^2\,=\,A^2\ \ \forall x\in\R.
\]
c) Näytä: Jos $A=0$, niin (P):n ainoa ratkaisu on $u(x)=v(x)=0$. \newline
d) Päättele: a-kohdan ratkaisu on ainoa (P):n ratkaisu. \newline
d) Näytä: (P):n ratkaisu on myös
\[
u(x)=A\sum_{k=0}^\infty (-1)^k\frac{x^{2k}}{(2k)!}\,, \qquad
v(x)=A\sum_{k=0}^\infty (-1)^k\frac{x^{2k+1}}{(2k+1)!}\,.
\]
e) Päättele: Jokaisella $x\in\R$ pätee
\[
\cos x\,=\,\sum_{k=0}^\infty (-1)^k\frac{x^{2k}}{(2k)!}\,, \qquad
\sin x\,=\,\sum_{k=0}^\infty (-1)^k\frac{x^{2k+1}}{(2k+1)!}\,.
\]

\end{enumerate}