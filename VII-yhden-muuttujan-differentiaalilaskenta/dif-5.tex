\section{Taylorin polynomien sovelluksia} \label{taylorin polynomien sovelluksia}
\alku
\index{Taylorin polynomi|vahv}

Tässä luvussa tarkastellaan eräitä tavallisia Taylorin polynomien ja Taylorin lauseen 
käyttötapoja funktiotutkimuksessa ja funktioiden approksimoimisessa. Taylorin polynomeja 
laskettaessa tai jäännöstermiä arvioitaessa voidaan laskuja usein huomattavasti lyhentää ja 
selkiyttää käyttämällä nk.\ \kor{suuruusluokkamerkintöjä}. Esimakua tällaisista merkinnöistä 
on saatu jo edellisen luvun Esimerkeissä \ref{nopea Taylor 1} ja \ref{nopea Taylor 2}, joissa
käytettiin lyhennysmerkintää $[x^m]$. Tämän tilalla on tavallisempaa käyttää merkintää
$\ordoO{\abs{x}^m}$, joka luetaan 'suuruusluokkaa $x^m$' tai vain 'oo $x^m$'. Toinen,
merkitykseltään hiukan voimakkaampi suuruusluokkamerkintä on $o(\abs{x}^m)$, joka luetaan
'pikku oo $x^m$'. Näissä merkinnöissä on $\abs{x}^m$ nk.\ vertailufunktio. Tämän tilalla voi
olla yleisempi, ei-negatiivisia arvoja saava funktio, esim.\ 
$\abs{x-x_0}^\alpha,\ \alpha\in\Q$.
\begin{Def} \label{iso oo ja pikku oo} \vahv{(Suuruusluokkamerkinnät $\mathcal{O}$ ja $o$)}
\index{suuruusluokkamerkinnät|emph} 
Jos $f$ ja $g$ on määritelty välillä $[x_0-a,x_0+a]$ ($a>0$) ja 
$g(x) \ge 0\ \forall\,x\in[x_0-a,x_0+a]$, niin käytetään merkintää
\[
f(x)=\mathcal{O}(g(x)), \quad x\in[x_0-a,x_0+a],
\]
jos on olemassa vakio $C\in\R_+$, siten että pätee
\[
\abs{f(x)}\leq Cg(x) \quad\forall\ x\in[x_0-a,x_0+a].
\]
Jos $g(x)>0\,\ \forall\,x \neq x_0$ ja pätee $\,\lim_{x\kohti x_0} (f/g)(x)=0$, niin käytetään
merkintää
\[
f(x)=o(g(x)), \quad \text {kun}\ x\kohti x_0.
\]
\end{Def}

Jos merkinnällä $\ordoO{..}$ halutaan ainoastaan kertoa, että arvio on pätevä jossakin $x_0$:n 
ympäristössä, niin tämä voidaan ilmaista kirjoittamalla, kuten vastaavassa merkinnässä $o(..)$,
\[
f(x)=\ordoO{g(x)}, \quad \text{kun}\ x\kohti x_0.
\]
Näissä merkinnöissä voi raja-arvon $x_0$ tilalla olla $\pm\infty$, jolloin kyse on $f$:n 
arvioimisesta suurilla $\abs{x}$:n arvoilla. Vertailufunktio on tällöin tyypillisesti 
$g(x)=\abs{x}^\alpha$ jollakin $\alpha\in\Q$.

Laskuissa suuruusluokkamerkintöjä on kätevä käsitellä kuten funktioita, jolloin niitä voi 
yhdistellä funktioiden tavoin. Esimerkiksi $\,f(x)+o(x^2)$ tarkoittaa funktiota $\,f(x)+g(x)$,
missä $\,g(x)=o(x^2)$, ja $\,\ordoO{x^2}+\ordoO{y^2}$ tarkoittaa samaa kuin $\ordoO{x^2+y^2}$.
Määritelmästä \ref{iso oo ja pikku oo} on helposti todennettavissa seuraavat
suuruusluoka-algebran säännöt (Harj.teht.\,\ref{H-VIII-5: suuruusluokka-algebra}). Säännöissä
oletetaan, että $x,y \ge 0$.
\[ \boxed{ \begin{aligned} 
\ykehys\quad \Ord{x}+\Ord{y} &= \Ord{\max\{x,y\}}, \quad \\
\Ord{x}\Ord{y}               &= \Ord{xy}, \quad \\
\ord{x}\Ord{y}               &= \ord{xy}. \quad
           \end{aligned} } \]
Sovelluksissa kahta jälkimmäistä sääntöä käytetään tavallisimmin muodossa
\[
\Ord{x^\alpha}\Ord{x^\beta}=\Ord{x^{\alpha+\beta}}, \quad
\ord{x^\alpha}\Ord{x^\beta}=\ord{x^{\alpha+\beta}}, \quad x \ge 0,\ \alpha,\beta\in\R.
\]
\begin{Exa} Muuttujan vaihdon $x^2=t$ ja Taylorin lauseen avulla päätellään, että pisteen
$x=0$ lähellä (esim.\ välillä $[-1,1]$) on
\[
\sqrt{x^2+1} = \sqrt{1+t} = 1+\frac{1}{2}\,t + \Ord{t^2} = 1+\frac{1}{2}\,x^2+\Ord{x^4}.
\]
Suurilla $x$:n arvoilla ($x\kohti\infty$) saadaan muuttujan vaihdolla $x^{-2}=t$ vastaavasti
arvio
\[
\sqrt{x^2+1} = x\sqrt{1+x^{-2}} = x\left(1+\frac{1}{2x^2}+\Ord{x^{-4}}\right)
                                = x+\frac{1}{2x}+\Ord{x^{-3}}. \loppu
\]
\end{Exa}
\begin{Exa} Näytä, että pienillä $|x|$:n arvoilla on
\[
f(x) \,=\, \frac{1-x}{1+2x-x^2+\Ord{|x|^3}} \,=\,1-3x+7x^2+\Ord{|x|^3}.
\]
\ratk Koska pienillä $|t|$:n arvoilla on $1/(1+t)=1-t+t^2+\Ord{|t|^3}$, niin
suuruusluokka-algebra antaa
(vrt.\ edellisen luvun Esimerkit \ref{nopea Taylor 1}--\ref{nopea Taylor 2})
\begin{align*}
f(x)\,&=\, (1-x)\left\{1-\left[2x-x^2+\Ord{|x|^3}\right]
                        +\left[2x-x^2+\Ord{|x|^3}\right]^2+\Ord{|x|^3}\right\} \\
    \,&=\, (1-x)[1-(2x-x^2)+4x^2]+\Ord{|x|^3} \,=\, 1-3x+7x^2+\Ord{|x|^3}. \loppu
\end{align*}
\end{Exa}

\subsection{Funktion approksimointi}
%\index{funktion approksimointi!b@Taylorin polynomilla|vahv}

Taylorin polynomien avulla voidaan differentiaaliin perustuva approksimaatio
\[
f(x+\Delta x)\approx f(x)+f'(x)\Delta x
\]
(vrt. Luku \ref{differentiaali}) sekä yleistää että tarkentaa: Jos $f$ on $n+1$ kertaa 
jatkuvasti derivoituva pisteen $x$ ympäristössä, niin
\[
\boxed{\begin{aligned}
\quad f(x+\Delta x)=f(x)
         &+f'(x)\Delta x+\frac{1}{2!}f''(x)(\Delta x)^2 \\
         &+ \cdots + \frac{1}{n!}f^{(n)}(x)(\Delta x)^n+\Ord{\abs{\Delta x}^{n+1}}. \quad
\end{aligned}}
\]
Hyvä arvio approksimaation virheelle, tai ainakin sen suuruusluokalle, on yleensä
\[
\text{virhe}\ =\ \text{tarkka}-\text{approksimaatio}\ 
                     \approx\ \frac{1}{(n+1)!}f^{(n+1)}(x)(\Delta x)^{n+1}.
\]
\begin{Exa} $\sqrt[10]{1000}\ \approx\ ?$ \end{Exa}
\ratk
\[
\sqrt[10]{1000}\ =\ \sqrt[10]{1024-24}\ =\ 2\sqrt[10]{1-\tfrac{3}{128}}\,.
\]
Kun merkitään $f(x)=\sqrt[10]{1-x}$, niin
\begin{alignat*}{2}
f'(x)&=-\tfrac{1}{10}(1-x)^{-9/10},\quad &f''(x)&=-\tfrac{9}{100}(1-x)^{-19/10}, \\
f'''(x)&=-\tfrac{171}{1000}(1-x)^{-29/10},\quad &f^{(4)}(x)&=-\tfrac{4959}{10000}(1-x)^{-39/10}
\end{alignat*}
\begin{align*}
\impl \ &\sqrt[10]{1000} =2f(\tfrac{3}{128}) \\
&\approx 2\left[1+\left(-\tfrac{1}{10}\right)\cdot\tfrac{3}{128}
                 +\tfrac{1}{2}\cdot\left(-\tfrac{9}{100}\right)\left(\tfrac{3}{128}\right)^2 +
       \tfrac{1}{6}\cdot\left(-\tfrac{171}{1000}\right)\left(\tfrac{3}{128}\right)^3\right] \\
&\approx 2\,(\,1-0.00234275-0.0000247192-0.000000366926\,) \\[2mm]
&= 1.995262327671.
\end{align*}
Virhe (tarkka arvo $-$ likiarvo) on suuruusluokkaa
\[
+\frac{1}{4!}f^{(4)}(0)\cdot (\Delta x)^4 
             = -\tfrac{1653}{80000}\cdot\left(\tfrac{3}{128}\right)^4\approx -6\cdot 10^{-9}.
\]
Oikea 12-desimaalinen arvo on
\[
\sqrt[10]{1000}\approx 1.995262314969. \loppu
\]

\begin{Exa}\ Arvioi $\,\sin 31^\circ\,$ lähtien arvosta $\,\sin 30^\circ = \tfrac{1}{2}$.
\end{Exa}
\ratk Kun $f(x)=\sin x$,\ $x=\tfrac{\pi}{6}$,\ $\Delta x=\tfrac{\pi}{180}$, on
\begin{align*}
f(x+\Delta x) \,&\approx\, \sin x + \cos x\,\Delta x-\tfrac{1}{2}\sin x\,(\Delta x)^2 \\
              \,&=\, \tfrac{1}{2}+\tfrac{\sqrt{3}}{2}\cdot\tfrac{\pi}{180}
                  -\tfrac{1}{2}\cdot\tfrac{1}{2}\cdot\left(\tfrac{\pi}{180}\right)^2\ 
                 \approx\ 0.51503882.
\end{align*}
Virhe (tarkka arvo $-$ likiarvo) on luokkaa
\[
-\frac{1}{3!}\cos x\,(\Delta x)^3\approx -8\cdot 10^{-7}.
\]
Oikea 8-desimaalinen arvo on: $\,\sin 31^\circ\approx 0.51503807$. \loppu

\subsection{Paikalliset ääriarvot}

Lauseen \ref{ääriarvolause} mukaan derivoituvan funktion paikallisessa ääriarvokohdassa on
myös derivaatan nollakohta. Jos funktio on derivaatan nollakohdan ympäristössä riittävän 
säännöllinen, niin Taylorin polynomien avulla voidaan selvittää, onko kyseessä todella 
ääriarvokohta ja jos, niin minkälaatuinen.
\begin{Lause} \label{Taylorin ääriarvolause}
\index{paikallinen maksimi, minimi, ääriarvo|emph}
\index{maksimi (funktion)!a@paikallinen|emph} 
\index{minimi (funktion)!a@paikallinen|emph}
\index{zyzy@ääriarvo (paikallinen)|emph}
\index{kriittinen piste!a@luokittelu|emph}
Jos $f$ on pisteen $c$ ympäristössä $n$ kertaa jatkuvasti derivoituva, $n\geq 2$, ja pätee
$\,f^{(k)}(c)=0,\quad k=1\ldots n-1\,$ ja $\,f^{(n)}(c)\neq 0$, niin
\begin{itemize}
\item[a)] jos $n$ on pariton, niin $f$:llä ei ole $c$:ssä paikallista ääriarvoa,
\item[b)] jos $n$ on parillinen, niin $f$:llä on $c$:ssä
\begin{itemize}
\item[-] paikallinen minimi, jos $f^{(n)}(c)>0$,
\item[-] paikallinen maksimi, jos $f^{(n)}(c)<0$.
\end{itemize}
\end{itemize}
\end{Lause}
\tod Lauseen \ref{Taylorin approksimaatiolause} ja oletuksien mukaan
\[
f(x)=f(c)+\frac{1}{n!}f^{(n)}(c)(x-c)^n+o(\abs{x-c}^n), \quad \text{kun}\ x \kohti c.
\]
Koska $f^{(n)}(c)\neq 0$, tämä voidaan kirjoittaa myös muotoon
\[
f(x)=f(c)+\frac{1}{n!}f^{(n)}(c)\,[\,1+o(1)\,]\,(x-c)^n, \quad \text{kun}\ x\kohti c,
\]
jolloin nähdään, että riittävän pienellä $\delta$ ($\delta>0$) on oltava
\[
f(x)=f(c)+\frac{1}{n!}f^{(n)}(c)k(x)(x-c)^n,\quad x\in [c-\delta,c+\delta],
\]
missä esimerkiksi
\[
\frac{1}{2}\leq k(x)\leq \frac{3}{2}\,.
\]
Näin ollen jos $n$ on parillinen ja $f^{n)}(c)>0$, on $f(x)>f(c)$ kun $x\in[c-\delta,c)$ tai
$x\in(c,c+\delta]$, jolloin $c$ on paikallinen minimikohta
(Määritelmä \ref{paikallinen ääriarvo}). Muissa tapauksissa on päättely vastaava
(vrt.\ kuvio). \loppu
\begin{figure}[H]
\setlength{\unitlength}{1cm}
\begin{center}
\begin{picture}(11,4)(0,-2)
\multiput(0,0)(3,0){4}{\vector(1,0){2}}
\multiput(1.8,-0.5)(3,0){4}{$x$}
\multiput(0.9,-0.4)(3,0){4}{$c$}
\multiput(1,0)(3,0){4}{\dashline{0.1}(0,0)(0,1)}
\put(0,-2){\parbox{2cm}{\small $n$ pariton, $f^{(n)}(c)>0$}}
\put(3,-2){\parbox{2cm}{\small $n$ pariton, $f^{(n)}(c)<0$}}
\put(6,-2){\parbox{2cm}{\small $n$ parillinen $f^{(n)}(c)>0$}}
\put(9,-2){\parbox{2cm}{\small $n$ parillinen, $f^{(n)}(c)<0$}}
\curve(
  0,  0.5,
0.2,0.744,
0.4,0.892,
0.6,0.968,
0.8,0.996,
  1,    1,
1.2,1.004,
1.4,1.032,
1.6,1.108,
1.8,1.256,
  2,  1.5)
\curve(
  3,  1.5,
3.2,1.256,
3.4,1.108,
3.6,1.032,
3.8,1.004,
  4,    1,
4.2,0.996,
4.4,0.968,
4.6,0.892,
4.8,0.744,
  5,  0.5)
\curve(
  6,   1.5,
6.2,1.2048,
6.4,1.0648,
6.6,1.0128,
6.8,1.0008,
  7,     1,
7.2,1.0008,
7.4,1.0128,
7.6,1.0648,
7.8,1.2048,
  8,   1.5)
\curve(
  9,   0.5,
9.2,0.7952,
9.4,0.9352,
9.6,0.9872,
9.8,0.9992,
  10,     1,
10.2,0.9992,
10.4,0.9872,
10.6,0.9352,
10.8,0.7952,
  11,   0.5)
\end{picture}
\end{center}
\end{figure}
\begin{Exa}
Tutki mahdollisen ääriarvokohdan laatu seuraavissa tapauksissa: \vspace{2mm}\newline
a) \   $f(x)=\sin x + \cos x, \quad c=\pi/4$ \newline
b) \   $f(x)=e^{-x}+\sin x + \sqrt{1-x^2}, \quad c=0$ \newline
c) \,\ $f(x)=1-\cos x-\sqrt{1+x^2}, \quad c=0$
\begin{align*} 
\text{\ratk} \quad &\text{a)} \quad f'(\pi/4)=0,\quad f''(\pi/4)=-\sqrt{2} \qquad
                                    \qimpl \text{paik.\ maksimi.} \hspace{1cm} \\[1mm]
                   &\text{b)} \quad f'(0)=f''(0)=0,\quad f'''(0)=-2 \quad\,
                                    \qimpl \text{ei ääriarvoa.} \\
                   &\text{c)} \quad f^{(k)}(0)=0,\ k=1 \ldots 3,\ f^{(4)}(0)=5 \,
                                    \qimpl \text{paik.\ minimi.} \loppu
\end{align*}
\end{Exa}

\subsection{Funktio $\tfrac{f(x)}{g(x)}=\tfrac{0}{0}\,$, kun $x=a$}
%\index{funktion approksimointi!b@Taylorin polynomilla|vahv}

Jos funktiot $f$ ja $g$ ovat pisteen $a$ ympäristössä säännöllisiä ja $f(a)=g(a)=0$, niin
Taylorin lauseen avulla voidaan tutkia, millainen funktio $F(x)=f(x)/g(x)$ on pisteen $a$
ympäristössä. Ensinnäkin jos halutaan ratkaista raja-arvokysymys
\[
\lim_{x\kohti a} \frac{f(x)}{g(x)}=\,?
\]
niin Taylorin lauseeseen perustuva menettely on seuraava: Oletetaan, että $f$ ja $g$ ovat $n$
kertaa jatkuvasti derivoituvia pisteen $a$ ympäristössä ja että pätee
\[
\frac{f(a)}{g(a)}=\frac{0}{0}\,,\quad \frac{f'(a)}{g'(a)}=\frac{0}{0}\,,\ \ldots,\ 
                                      \frac{f^{(n-1)}(a)}{g^{(n-1)}(a)}=\frac{0}{0}\,, \quad 
                                      \frac{f^{(n)}(a)}{g^{(n)}(a)}=\frac{A}{B}\,,
\]
missä $A \neq 0$ tai $B \neq 0$. Tällöin Lauseen \ref{Taylorin approksimaatiolause} mukaan
\[
\frac{f(x)}{g(x)} = \frac{\frac{A}{n!}\,(x-a)^n + \ord{\abs{x-a}^n)}}{\frac{B}{n!}\,(x-a)^n 
                                                + \ord{\abs{x-a}^n)}}
                  = \frac{A+\ord{1}}{B+\ord{1}}, \quad \text{kun}\ x \kohti a.
\]
Päätellään, että jos $A \neq 0$ ja $B=0$, niin $\abs{f(x)/g(x)}\kohti\infty$, kun
$x \kohti a$, jolloin raja-arvoa ei ole (reaalilukuna --- voi olla $\lim=\pm\infty$). Muussa
tapauksessa, eli jos $B \neq 0$, on
\[
\lim_{x \kohti a} \frac{f(x)}{g(x)} = \frac{A}{B} = \frac{f^{(n)}(a)}{g^{(n)}(a)}\,,
\]
ja tällöin on yleisemminkin
\[
\lim_{x \kohti a} \frac{f(x)}{g(x)} 
               = \lim_{x \kohti a} \frac{f'(x)}{g'(x)} 
               =\ \ldots\ = \lim_{x \kohti a} \frac{f^{(n)}(x)}{g^{(n)}(x)}\,.
\]
Tämä tulos on siis voimassa (tehtyjen säännöllisyysoletusten puitteissa), kunhan ketjun 
\pain{viimeinen} raja-arvo on olemassa. --- Tulos on jo ennestään tuttu l'Hospitalin
sääntönä (Lause \ref{Hospital}).
\begin{Exa} Määritä raja-arvo 
$\ \displaystyle{\lim_{x \kohti 0} \frac{\sin 2x-2x}{\sinh x-x}\,}$. 
\end{Exa}
\ratk \,Funktiot $f(x)=\sin 2x-2x$ ja $g(x)=\sinh x-x$ ovat sileitä $\R$:ssä. Derivoimalla 
todetaan
\[
\frac{f'(x)}{g'(x)}=\frac{2\cos 2x-2}{\cosh x-1}\,, \quad 
\frac{f''(x)}{g''(x)}=\frac{-4\sin 2x}{\sinh x}\,, \quad
\frac{f'''(x)}{g'''(x)}=\frac{-8\cos 2x}{\cosh x}\,,\ \ldots
\]
Havaitaan, että $f^{(k)}(0)=g^{(k)}(0)=0$, kun $k=0,1,2$, ja $f'''(0)=-8,\ g'''(0)=1$. Siis
\[
\lim_{x \kohti 0} \frac{\sin 2x-2x}{\sinh x-x} = \frac{-8}{1} = -8. \loppu
\]

Taylorin lauseen perusteella saadaan siis raja-arvolle $\lim_{x \kohti a} f(x)/g(x)$ sama
laskukaava kuin l'Hospitalin säännöllä. Funktiosta $f(x)/g(x)$ saadaan kuitenkin Taylorin
polynomien avulla selville paljon muutakin. Seuraavassa esimerkki vaativammasta tehtävän
asettelusta.
\begin{Exa} Määritä polynomit $p$ ja $q$ siten, että väleillä $[-1,0)$ ja $(0,1]$ pätee
\[
F(x)=\frac{\sin 2x-2x}{\sinh x-x} \,=\, p(x)+\Ord{x^4},\quad\ 
G(x)=\frac{\sin x}{2x^2}+\frac{\cos x -1}{x^3}=q(x)+\Ord{\abs{x}^5}.
\]
\begin{align*}
\text{\ratk} \quad  \sin 2x -2x\
         &=\ \Bigl(2x-\frac{1}{6}(2x)^3+\frac{1}{120}(2x)^5+\Ord{\abs{x}^7}\Bigr)-2x 
                                                                           \hspace{2cm} \\
         &=\ -\frac{4}{3}x^3+\frac{4}{15}x^5+\Ord{\abs{x}^7}, \\
\sinh x -x \,\ 
         &=\ \Bigl(x+\frac{1}{6}x^3+\frac{1}{120}x^5+\Ord{\abs{x}^7}\Bigr)-x \\
         &=\ \frac{1}{6}x^3+\frac{1}{120}x^5+\Ord{\abs{x}^7} \\
\qimpl \frac{\sin 2x-2x}{\sinh x-x}\ 
         &=\ \frac{-8+\frac{8}{5}x^2+\ordoO{x^4}}{1+\frac{1}{20}x^2+\Ord{x^4}} \\[2mm]
         &=\ -8+2x^2+\Ord{x^4} \,=\, p(x)+\Ord{x^4}. \\[5mm]
\frac{\sin x}{2x^2}+\frac{\cos x -1}{x^3}\ 
         &=\ \frac{1}{2x^2}\Bigl(x-\frac{x^3}{6}+\frac{x^5}{120}+\Ord{\abs{x}^7}\Bigr) \\
         & \quad\ + \frac{1}{x^3}\Bigl(-\frac{1}{2}x^2 +\frac{1}{24}x^4-\frac{1}{720}x^6 + 
                                                      \Ord{x^8}\Bigr) \\
         &=\ -\frac{1}{24}x+\frac{1}{360}x^3+\Ord{\abs{x^5}} \,=\, q(x)+\Ord{\abs{x}^5}.
\end{align*}

\vspace{2mm}

Tuloksista voi päätellä esimerkiksi raja-arvot
\[
\lim_{x\kohti 0}\left(\frac{\sin 2x-2x}{x^2\sinh x-x^3}+\frac{8}{x^2}\right) = 2, \qquad 
\lim_{x\kohti 0} \left(\frac{\sin x}{2x^3}+\frac{\cos x -1}{x^4}\right)=-\frac{1}{24}\,.
\]
Nähdään myös, että jos asetetaan (jatkamisperiaatteella) $F(0)=-8$ ja $G(0)=0$, niin $F$:llä
on pisteessä $x=0$ paikallinen minimi ja $G$ on origon ympäristössä aidosti vähenevä. \loppu
\end{Exa}

\subsection{Differenssiapproksimaatiot}
\index{differenssiapproksimaatio|vahv}

Taylorin lauseella on usein keskeinen rooli, kun halutaan johtaa virhearvioita numeerisille
(likimääräisille) laskentamenetelmille, jotka perustuvat oletettuun funktion säännöllisyyteen.
Tarkastellaan esimerkkinä derivaattojen numeerisessa laskemisessa käytettäviä nk.\
\kor{differenssiapproksimaatioita}. Näissä ei funktiota oleteta tunnetuksi lausekkeena vaan
riittää tuntea funktion arvo äärellisen monessa (tyypillisesti vain muutamassa) pisteessä
tarkasteltavan pisteen lähellä. Seuraavat kolme ovat differenssiapproksimaatioista
tavallisimmat (ol.\ $h>0$):
\begin{align}
f'(a)  &\,\approx\, \frac{f(a+h)-f(a)}{h}\,, \label{dif1} \tag{\text{a}} \\
f'(a)  &\,\approx\, \frac{f(a+h)-f(a-h)}{2h}\,, \label{dif2} \tag{\text{b}} \\
f''(a) &\,\approx\, \frac{f(a+h)-2f(a)+f(a-h)}{h^2}\,. \label{dif3} \tag{\text{c}}
\end{align}
Näistä (a) perustuu suoraan derivaatan määritelmään. Approksimaatioita (b) ja (c) sanotaan 
\index{keskeisdifferenssiapproksimaatio}%
\kor{keskeis}differenssiapproksimaatioiksi, syystä että funktio evaluoidaan näissä $a$:n
suhteen symmetrisessä pisteistössä.

Em.\ approksimaatioille voidaan johtaa virhearvio Taylorin lauseesta olettaen, että $f$ on
riittävän säännöllinen. Tarkastellaan esimerkkinä approksimaatiota (b), muut jätetään
harjoitustehtäviksi (Harj.teht.\,\ref{H-dif-5: virhearvio (a)},\ref{H-dif-5: virhearvio (c)}).
\begin{Prop} \label{keskeisdifferenssin tarkkuus} Jos $f$ on kolmesti jatkuvasti derivoituva
välilä $[a-h,a+h]$, niin differenssiapproksimaatiolle (b) pätee virhearvio
\[
\left|\,f'(a)-\frac{f(a+h)-f(a-h)}{2h}\,\right| \,\le\,\frac{1}{6}M\,h^2, \quad
                                          M=\max_{x\in[a-h,a+h]}|f'''(x)|.
\]
\end{Prop}
\tod Oletuksien perusteella on joillakin $\,\xi_1\in(a,a+h)\,$ ja $\,\xi_2\in(a-h,a)$ 
\begin{align*}
f(a+h) &= f(a)+f'(a)h+\frac{1}{2}f''(a)\,h^2+\frac{1}{6}f'''(\xi_1)\,h^3, \\
f(a-h) &= f(a)-f'(a)h+\frac{1}{2}f''(a)\,h^2-\frac{1}{6}f'''(\xi_2)\,h^3
\end{align*}
(Lause \ref{Taylor}: välin $[a,b]$ tilalla väli $[a-h,a+h]$, $x_0=a$, $x=a \pm h$, $n=2$).
Näistä seuraa vähennyslaskulla
\[
f'(a)-\frac{f(a+h)-f(a-h)}{2h} = -\frac{1}{12}\,[f'''(\xi_1)+f'''(\xi_2)]\,h^2.
\]
Käyttämällä oikealla arviota $|f'''(\xi_1)+f'''(\xi_2)| \le |f'''(\xi_1)|+|f'''(\xi_2)| \le 2M$
seuraa väite. \loppu

Proposition \ref{keskeisdifferenssin tarkkuus} perusteella approksimaatio (b) on tarkka
toisen asteen polynomeille (joille $M=0$). Valitsemalla $f(x)=\frac{1}{6}M(x-a)^3$ nähdään
myös, että virhearvio on tehdyin oletuksin tarkin mahdollinen.

Yleisesti sanotaan, että differenssiapproksimaation (tarkkuuden)
\index{kertaluku!b@tarkkuuden}%
\kor{kertaluku} on $r$, jos
sen virhe on $\mathcal{O}(h^r)$ mutta ei $o(h^r)$ yleiselle, riittävän säännölliselle
funktiolle. Approksimaation (b) kertaluku on siis $r=2$, eli kyseessä on
\kor{toisen kertaluvun} approksimaatio. Approksimaation (c) kertaluku on samoin $r=2$
(Harj.teht.\,\ref{H-dif-5: virhearvio (c)}), sen sijaan (a) on ensimmäisen kertaluvun ($r=1$)
approksimaatio (Harj.teht.\,\ref{H-dif-5: virhearvio (a)}).

\Harj
\begin{enumerate}

\item \label{H-VIII-5: suuruusluokka-algebra}
Perustele suuruusluokka-algebran säännöt (ol.\ $x,y \ge 0$)
\[
\Ord{x}+\Ord{y} = \Ord{\max\{x,y\}}, \quad 
\Ord{x}\Ord{y} = \Ord{xy}, \quad
\ord{x}\Ord{y} = \ord{xy}.
\]

\item
Mitä funktion $f$ ominaisuuksia tarkoitetaan seuraavilla merkinnöillä? \newline
a) \ $f(x)=f(a)+\ord{1}, \quad \text{kun}\,\ x \kohti a$. \newline
b) \ $f(x_1)-f(x_2)=\Ord{\abs{x_1-x_2}}, \quad \text{kun}\,\ x_1,x_2\in[a,b]$. \newline
c) \ $f(x)=f(a)+k(x-a)+o\,(\abs{x-a}), \quad \text{kun}\,\ x \kohti a$.

\item
Anna suuruusluokka-arviot seuraavien approksimaatioiden virheille: \vspace{1mm}\newline
a) \ $\sqrt{x^4+2} \approx \sqrt{2}, \quad \text{kun}\ x \kohti 0$ \vspace{2mm}\newline
b) \ $\sqrt{x^4+3x} \approx \sqrt{3x}, \quad \text{kun}\ x \kohti 0^+$ \vspace{1mm}\newline
c) \ $\sqrt{x^4+4x} \approx x^2, \quad \text{kun}\ x\kohti\pm\infty$ \vspace{1mm}\newline
d) \ $\sqrt{x^4+4x} \approx x^2+2x^{-1} \quad \text{kun}\ x\kohti\infty$ \vspace{0.5mm}\newline
e) \ $x/\sin x \approx 1, \quad \text{kun}\ x \kohti 0$ \vspace{1mm}\newline
f) \ $\ln(1+e^x) \approx x, \quad \text{kun}\ x\kohti\infty$

\item
Arvioi approksimaation $f(x) \approx T_2(x,0)$ virheen itseisarvo välillä $[-1/2,\,1/2]$
seuraaville funktioille: \vspace{1mm}\newline
a) \, $\sqrt{1+x}\quad\,\ $ 
b) \, $\sqrt[10]{1+x}\quad\,\ $ 
c) \, $\tan x\quad\,\ $ 
d) \, $\ln(1+x)\quad\,\ $ 
e) \, $e^{\sin x}$   

\item
Arvioi seuraavien funktioapproksimaatioiden virhe:
\begin{align*}
&\text{a)}\ \ \sqrt[3]{8+x} \approx 2+\frac{x}{12}-\frac{x^2}{288}\,, \quad 
                                         \text{kun}\ \abs{x} \le 1 \\
&\text{b)}\ \ \sin x \approx x-\frac{1}{6}x^3, \quad 
                                         \text{kun}\ x\vastaa\alpha\in[0\aste,10\aste] \\
&\text{c)}\ \ \frac{1}{x^4}\left[\frac{1}{\sqrt{1+x^2}}-\sqrt{1-x^2}\right] 
            \approx \frac{1}{2}\,, \quad \text{kun}\ \abs{x} \le 0.15\,\ \text{ja}\,\ x \neq 0
\end{align*}

\item
Jos $T_n(x,0)$ on funktion $f(x)=\cos x$ Taylorin polynomi, niin millä $a$:n ja $n$:n arvoilla
voidaan taata, että \newline
a) \ $\abs{\cos x-T_2(x,0)} \le 10^{-4}$ välillä $[-a,a]$, \newline
b) \ $\abs{\cos x-T_n(x,0)} \le 10^{-4}$ välillä $[-\pi/2,\,\pi/2]$\,?

\item
a) Laske luvulle $1/\sqrt[4]{e}$ rationaalinen likiarvo approksimoimalla funktiota $e^x$ toisen
asteen Taylorin polynomilla. Arvioi approksimaation virhe Taylorin lauseen avulla ja vertaa
virheen tarkkaan arvoon. \vspace{1mm}\newline
b) Laske luvulle $a=\sqrt[12]{4000}$ rationaalinen likiarvo kirjoittamalla $4000=4096(1-x)$ ja
approksimoimalla funktiota $f(x)=\sqrt[12]{1-x}$ toisen asteen Taylorin polynomilla. Arvioi
approksimaation virhe Taylorin lauseen avulla ja vertaa virheen tarkkaan arvoon.

\item
Suhteellisuusteorian mukaan vauhdilla $v$ liikkuvan partikkelin liike-energia on
\[
E_k = \frac{mc^2}{\sqrt{1-\frac{v^2}{c^2}}}-mc^2,
\]
missä $m$ on partikkelin massa ja $c \approx 3 \cdot 10^8$ m/s on valon nopeus. Millaisilla 
vauhdin $v$ arvoilla approksimaation $E_k \approx \frac{1}{2} mv^2$ suhteellinen virhe on
enintään $0.01\%$\,? 

\item 
Seuraavilla funktioilla piste $x=0$ on mahdollinen paikallinen minimi- tai maksimikohta.
Tutki asia derivoimalla!
\[
\text{a)}\ (1-x)e^x \quad\
\text{b)}\ (2-x^2)\cos x \quad\
\text{c)}\ (x-x^4)\sin x \quad\
\text{d)}\ \sqrt{1-x^2}-1/\sqrt{1+x^2}
\]

\item
Millä $a$:n ja $b$:n arvoilla funktiolla
\[
\text{a)}\ e^x+ax+bx^2 \quad\ 
\text{b)}\ x\sin x+ax^2+bx^4 \quad\ 
\text{c)}\ \sin x+\cos x+ax+bx^2
\]
on paikallinen minimi pisteessä $x=0$\,?

\item
Yhtälö $\,y(\cos y-\sin y)=2\sin x+\cos x+ax+b\,$ määritelee pisteen $x=0$ ympäristössä
funktion $y(x)$. Määritä vakiot $a$ ja $b$, kun tiedetään, että $y(x)$ saavuttaa pisteessä
$x=0$ paikallisen ääriarvon $y(0)=0$. Onko kyseessä maksimi vai minimi?

\item
Määritä seuraavat raja-arvot:
\begin{align*}
&\text{a)}\ \ \lim_{x \kohti 0}\frac{\sqrt[7]{1+x}-1}{x} \qquad
 \text{b)}\ \ \lim_{x \kohti 0}\frac{3x}{\tan 4x} \qquad
 \text{c)}\ \ \lim_{x \kohti 0}\frac{ 2\cos x-2+x^2 }{x^4 } \\
&\text{d)}\ \ \lim_{x \kohti 0}\frac{x-\sin x}{x-\tan x} \qquad
 \text{e)}\ \ \lim_{x \kohti 0}\frac{10^x-e^x}{x} \qquad
 \text{f)}\ \ \lim_{x \kohti 0}\frac{1-\cos ax}{1-\cos bx} \\
&\text{g)}\ \ \lim_{x\kohti\pi/2}\frac{\cos 3x}{\pi-2x} \qquad
 \text{h)}\ \ \lim_{t\kohti\pi/2}\frac{\ln\sin t}{\cos t} \qquad
 \text{i)}\ \ \lim_{x \kohti 0^+}\frac{\sin^2 x}{\tan x-x} \\
&\text{j)}\ \ \lim_{x \kohti 1}\frac{\ln(ex)-1}{\sin\pi x} \qquad
 \text{k)}\ \ \lim_{x \kohti -\infty} x\sin\frac{1}{x} \qquad
 \text{l)}\ \ \lim_{x\kohti\infty}x^2\ln(\cos\frac{\pi}{x}) \\[1mm]
&\text{m)}\ \ \lim_{x\kohti\infty} \left[\sqrt{x^2+154x}-\ln(5+e^x)\right] \qquad
 \text{n)}\ \ \lim_{x\kohti\infty}\left(\sqrt[6]{x^6+42x^5+77x}-x\right) \qquad
\end{align*}

\item
Määritä seuraaville funktioille $f$ mahdollisimman alhaista astetta oleva polynomi $p$
siten, että annetulla $m$ ja jollakin $\delta>0$ on $f(x)=p(x)+\Ord{|x|^m}$, kun
$0<|x|<\delta$. Tutki myös pisteen $x=0$ laatu mahdollisena $f$:n paikallisena
ääriarvokohtana, kun asetetaan $f(0)=\lim_{x \kohti 0} f(x)$.
\begin{align*}
&\text{a)}\,\ \frac{1-\cos x}{x^2}\,,\,\ m=7 \qquad
 \text{b)}\,\ \frac{\sin x-x}{x^3}\,,\,\ m=8 \qquad
 \text{c)}\,\ \frac{x^3}{\sin x-x}\,,\,\ m=3 \\
&\text{d)}\,\ \frac{x}{e^x-1}\,,\,\ m=3 \qquad
 \text{e)}\,\ \frac{\ln(1+x)}{x}\,,\,\ m=5 \qquad
 \text{f)}\,\ \frac{\sqrt{1+x}-1}{\sqrt[3]{1-x}-1}\,,\,\ m=3
\end{align*}

\item \label{H-dif-5: virhearvio (a)}
Näytä, että jos $f$ on jatkuva välillä $[a,a+h]$ ja
kahdesti derivoituva välillä $(a,a+h)$, niin jollakin $\xi\in(a,a+h)$ pätee
\[
f'(a)-\frac{f(a+h)-f(a)}{h} \,=\, -\frac{1}{2}\,f''(\xi)\,h.
\]

\item (*) \label{H-dif-5: virhearvio (c)}
Näytä Taylorin lauseen avulla, että jos $f$ on neljä
kertaa jatkuvasti derivoituva välillä $[a-h,a+h]$, niin pätee
\[
\left|\,f''(a)-\frac{f(a+h)-2f(a)+f(a-h)}{h^2}\,\right| 
               \,\le\,\frac{1}{12}\left(\max_{x\in[a-h,a+h]}|f^{(4)}(x)|\right) h^2.
\]
Miten arvio toteutuu funktiolle $f(x)=(x-a)^4$\,?

\item (*)
a) Näytä, että eräillä (millä?) $A$:n ja $B$:n arvoilla pätee
\[
\Arccos x\,=\,A\sqrt{1-x}+B(1-x)^{3/2}+\mathcal{O}\left((1-x)^{5/2}\right),
\quad \text{kun}\ x \kohti 1^-.
\]
b) Näytä, että jos sykloidin yhtälöt $x=R(t-\sin t),\ y=R(1-\cos t)$ eliminoidaan muotoon
$y=y(x)$ (ratkaisemalla $t=t(x)$ ensimmäisestä yhtälöstä), niin origon ympäristössä pätee
\[
y(x) = 3\sqrt[3]{\frac{Rx^2}{6}}+\mathcal{O}\left(\sqrt[3]{\frac{x^4}{R}}\,\right).
\]

\item (*)
Olkoon $c$ funktion $f$ yksinkertainen nollakohta ja olkoon $f$ kolme kertaa jatkuvasti
derivoituva $c$:n ympäristössä. Halutaan määrätä $c$ Newtonin menetelmästä muunnellulla
iteraatiolla $\,x_{n+1}=F(x_n)$, missä
\[
F(x)=x-\frac{f(x)}{f'(x)}+[f(x)]^2g(x).
\]
Miten $g(x)$ on valittava, jotta iteraation suppeneminen kohti $c$:tä on vähintään
kuutiollista? Millä ehdolla suppeneminen on täsmälleen kuutiollista? Sovella menetelmää 
funktioon $f(x)=x^2-a,\ a>0$
(vrt.\ Harj.teht.\,\ref{kiintopisteiteraatio}:\ref{H-V-7: kuutiollisia iteraatioita}b).

\end{enumerate}
