\section{Interpolaatiopolynomit} \label{interpolaatiopolynomit}
\alku
\index{interpolaatiopolynomi|vahv}

Tavallisessa \kor{polynomi-interpolaatiossa} on lähtöajatuksena, että funktiosta $f$ tunnetaan
vain äärellinen määrä pistearvoja:
\[
f(x_i)=f_i,\quad i=0 \ldots n.
\]
Näiden tietojen perusteella halutaan esittää $f$ likimäärin polynomina muuallakin kuin
pisteissä $x_i$, esim.\ jollakin välillä. Interpolaatio on kyseessä erityisesti silloin, kun
$f(x)$ halutaan laskea 'välipisteissä' $x \in [\min\{x_i\},\max\{x_i\}]$, muussa tapauksessa
(eli kun $x < \min\{x_i\}$ tai $x > \max\{x_i\}$) sanotaan, että kyseessä on
\kor{ekstrapolaatio}. Annetut tiedot voivat olla esim.\ mitattua 'dataa'. Toinen yleinen
sovellustilanne on sellainen, jossa funktio $f$ tunnetaan epäsuorasti, esim.\
differentiaaliyhtälön ratkaisuna, ja halutaan laskea arvot $f_i$ valituissa pisteissä $x_i$.
Tällöin polynomi-interpolaatioista on hyötyä itse laskenta-algoritmin suunnittelussa.

Funktion polynomiapproksimaatiossa on yleensä perusoletuksena (tai ainakin toivomuksena), että
funktio on riittävän säännöllinen, jolloin Taylorin lauseen perusteella tiedetään, että 
funktiota voi (ainakin lyhyellä välillä) approksimoida hyvin polynomilla, nimittäin Taylorin
polynomilla. Koska itse Taylorin polynomia ei ym.\ tiedoista voi suoraan määrätä, on luontevaa
valita approksimoivaksi polynomiksi $p$ sellainen, joka sopii annettuihin tietoihin, eli
toteuttaa
\begin{equation} \label{interpolaatioehdot}
p(x_i)=f(x_i),\quad i=0 \ldots n.
\end{equation}
Koska tässä on ehtoja $n+1$ kpl, on edelleen luonnollista valita $p$:n asteluvuksi $n$, jolloin
$p$:ssä on vapaita kertoimia myöskin $n+1$ kpl. Näin määriteltyä polynomia $p$ sanotaan $f$:n 
\index{interpolaatiopolynomi!a@Lagrangen} \index{Lagrangen!b@interpolaatio(polynomi)}%
\kor{Lagrangen interpolaatiopolynomiksi} pisteissä $x_i$. Pisteitä $x_i$ sanotaan tässä 
yhteydessä \kor{interpolaatiopisteiksi} ja ehtoja \eqref{interpolaatioehdot} 
\kor{interpolaatioehdoiksi}. Nämä  ehdot todella määrittelevät yksikäsitteisen polynomin
astetta $n$:
\begin{Prop} \label{interpolaatiopolynomin yksikäsitteisyys}
Lagrangen interpolaatiopolynomi on yksikäsitteinen.
\end{Prop}
\tod Ehdot \eqref{interpolaatioehdot} toteuttavan interpolaatiopolynomin olemassaolo seuraa 
jäljempänä esitettävästä laskukaavasta \eqref{Lagrangen interpolaatiokaava}, joten riittää
näyttää yksikäsitteisyys. Olkoot siis $p_1$ ja $p_2$ molemmat ehdot \eqref{interpolaatioehdot}
täyttäviä polynomeja astetta $n$. Tällöin $q(x)=p_1(x)-p_2(x)$ on polynomi enintään astetta $n$
ja pätee
\[
q(x_i)=0, \quad i=0 \ldots n,
\]
eli $q$:lla on $n+1$ reaalista nollakohtaa. Algebran peruslauseen mukaan on tällöin oltava
$q=0$. Siis $p_1=p_2$, eli ehdot \eqref{interpolaatioehdot} täyttävä polynomi (sikäli kuin
olemassa) on yksikäsitteinen. \loppu

Lagrangen interpolaatiopolynomin approksimaatiovirheelle pätee seuraava tulos, joka muistuttaa 
Taylorin polynomien virhekaavaa (Lause \ref{Taylor}). Jäljempänä nähdäänkin
(Lause \ref{usean pisteen Taylor}), että nämä kaksi tulosta ovat erikoistapauksia yleisemmästä
interpolaatiopolynomien virhekaavasta. Jatkossa sanotaan äärellisen pisteistön $X$ 
\index{virittää (väli)}%
\kor{virittämäksi} väliksi suljettua väliä $[a,b]$, missä $a = \min \{x \mid x \in X\}$ ja 
$b = \max \{x \mid x \in X\}$.
\begin{Lause} \label{Lagrangen interpolaatiovirhe}
Olkoon $f\,$ jatkuva välillä $[a,b]$ ja $n+1$ kertaa derivoituva välillä $(a,b)$, missä $[a,b]$
on pisteen $x$ ja erillisten pisteiden $x_0,\ldots,x_n$ virittämä väli, $n\in\N$. Tällöin jos
$p$ on $f$:n Lagrangen interpolaatiopolynomi pisteissä $x_0,\ldots,x_n$, niin jollakin
$\xi \in (a,b)$ pätee virhekaava
\[
f(x)-p(x)=\frac{1}{(n+1)!}\,f^{(n+1)}(\xi) \prod_{i=0}^n (x-x_i).
\]
\end{Lause}
\tod perustuu samaan ideaan kuin Taylorin lauseen todistus Luvussa \ref{taylorin lause}.
Ensinnäkin jos $x\in\{x_0,\ldots,x_n\}$, on väittämä tosi jokaisella $\xi \in (a,b)$, joten
voidaan olettaa, että $x\notin\{x_0,\ldots,x_n\}$. Merkitään
\[
w(t)=\prod_{i=0}^n (t-x_i)
\]
ja tutkitaan funktiota
\[
g(t)=f(t)-p(t)- Hw(t),\quad H=[f(x)-p(x)]/w(x).
\]
Funktion $g$ nollakohtia ovat interpolointipisteet $x_0,\ldots,x_n$ ja lisäksi piste $x$. Koska
nollakohtia välillä $[a,b]$ on siis ainakin $n+2$ kpl, ja koska näiden välissä on aina 
derivaatan nollakohta (Lause \ref{Rollen lause}), on $g'(t)$:llä ainakin $n+1$ nollakohtaa
avoimella välillä $(a,b)$. Tällöin $g''$:lla on ainakin $n$ nollakohtaa tällä välillä, ja
lopulta $g^{(n+1)}$:lla ainakin yksi nollakohta $\xi\in (a,b)$. Mutta tällöin
\begin{align*}
0 = g^{(n+1)}(\xi) &= f^{(n+1)}(\xi)-p^{(n+1)}(\xi)-H w^{(n+1)}(\xi) \\
                   &= f^{(n+1)}(\xi)-H (n+1)! \\[2mm]
      \impl\quad H &= \frac{f(x)-p(x)}{w(x)}=\frac{1}{(n+1)!}\,f^{(n+1)}(\xi). \loppu
\end{align*}

Interpolaatioperiaatteista on astelukuun $n=1$ perustuva
\index{lineaarinen interpolaatio}%
\kor{lineaarinen interpolaatio} 
yksinkertaisuutensa vuoksi hyvin yleisesti käytetty. Esimerkiksi käyrän $y=f(x)$ approksimointi
pisteiden $(x_i,f(x_i))$ kautta kulkevalla murtoviivalla (vaikkapa kuvan piirtämiseksi tai
kaarenpituuden arvioimiseksi) tarkoittaa $f$:n \kor{paloittain lineaarista}
interpolaatiota.
\begin{figure}[H]
\begin{center}
\import{kuvat/}{kuvaipol-1.pstex_t}
\end{center}
\end{figure}
Jos $f$ on kahdesti jatkuvasti derivoituva interpolointivälillä $[x_0,x_1]$, niin lineaarisen
interpolaation virhe ko.\ välillä on Lauseen \ref{Lagrangen interpolaatiovirhe} mukaan enintään
\begin{align*}
\max_{x\in [x_0,x_1]} \abs{f(x)-p(x)} 
         &\le \max_{x\in[x_0,x_1]} \left|\frac{1}{2}(x-x_0)(x-x_1)\right|
              \max_{\xi\in[x_0,x_1]}\abs{f''(\xi)} \\
         &= \frac{1}{8}\,(x_1-x_0)^2\max_{x\in [x_0,x_1]} \abs{f''(x)}.
\end{align*}
Jos interpolointi tapahtuu paloittain välillä $[a,b]$ ja interpolointipisteiden väli on
enintään $h$, niin ko.\ approksimaation maksimivirhe on siis enintään
\[
\max_{x\in [a,b]} \abs{f(x)-\tilde{f}(x)}\leq\frac{1}{8}\,h^2\max_{x\in [a,b]} \abs{f''(x)}.
\]
Tämä arvio ei ole parannettavissa, sillä jos $f''(x)=$ vakio (eli $f$ on toisen asteen
polynomi) ja interpolointipisteet ovat tasaväliset, niin virhearvion yläraja totetuu
peräkkäisten interpolointipisteiden puolivälissä.

\index{kvadraattinen!b@interpolaatio}%
Jos \kor{kvadraattisessa} (= toisen asteen) interpolaatiossa pisteet $x_0,x_1,x_2$ ovat 
tasavälein ja väli $=h$, niin
\[
\max_{x\in [x_0,x_2]} \abs{(x-x_0)(x-x_1)(x-x_2)}
     = \max_{x\in [-h,h]} \abs{x}(h^2-x^2)=\frac{2}{3\sqrt{3}}\,h^3,
\]
joten kvadraattinen interpolaatiovirhe on tasavälisten interpolointipisteiden virittämällä 
välillä enintään
\[
\max_{x\in [x_0,x_2]} \abs{f(x)-p(x)}
   \le \frac{1}{9\sqrt{3}}\,h^3\max_{x\in [x_0,x_2]} \abs{f'''(x)} \quad (h=x_1-x_0=x_2-x_1).
\]
\begin{Exa}
Välillä $(0,\infty)$ määritelty funktio $E$ toteuttaa ehdot
\[
E'(x) = -\frac{e^{-x}}{x}\,, \quad x>0, \qquad \lim_{x\kohti\infty} E(x)=0.\footnote[2]{Funktio
on nimeltään \kor{eksponentiaali-integraalifunktio}.
\index{eksponentiaali-integraalifunktio|av}}
\]
Halutaan määrittää $E$:n arvot likimäärin välillä $[1,2]$ laskemalla ensin $E$ pisteissä 
$x_i=1+(i-1)h,\ i=0\ldots n,\ h=1/n$ (oletetetaan, että tämä onnistuu hyvin tarkasti) ja 
käyttämällä interpolaatiota muissa pisteissä. Kuinka suuri on $n$:n oltava, jos käytetään \ 
a) lineaarista, \ b) kvadraattista interpolaatiota ja halutaan, että interpolaatiovirhe on 
enintään $5\cdot 10^{-9}$\,?
\end{Exa} 
\ratk Koska $E'(x) = -e^{-x}/x,\ x>0$, niin
\begin{align*}
E''(x)  &= e^{-x}(\tfrac{1}{x}+\tfrac{1}{x^2}) 
              \qimpl \abs{E''(x)}\leq 2/e,\quad x\in [1,2], \\
E'''(x) &= -e^{-x}(\tfrac{1}{x}+\tfrac{2}{x^2}+\tfrac{2}{x^3}) 
              \qimpl \abs{E'''(x)}\leq 5/e,\quad x\in [1,2].
\end{align*}
Näin ollen riittää valita $N=n+1$ siten, että pätee
\begin{align*}
&\text{a)} \quad \frac{1}{8}\cdot\frac{2}{e}\cdot\left(\frac{1}{N-1}\right)^2
                \le 5\cdot 10^{-9} \qekv \underline{\underline{N \ge 4290}}, \\
&\text{b)}\quad \frac{1}{9\sqrt{3}}\cdot\frac{5}{e}\cdot\left(\frac{1}{N-1}\right)^3
                \le 5\cdot 10^{-9} \qekv \underline{\underline{N \ge 288}}. \loppu
\end{align*}

\subsection{Lagrangen kantapolynomit}
\index{interpolaatiopolynomi!a@Lagrangen|vahv}
\index{Lagrangen!b@interpolaatio(polynomi)|vahv}

\index{polynomisovitus}%
Korkeampiasteisissa polynomi-interpolaatioissa on usein kyse \kor{polynomisovituksesta} 
pisteisiin $(x_i,f(x_i))$, jolloin arvot $f(x_i)$ voivat olla esimerkiksi mitattuja. 
Kehittyneissä numeerisen ja symbolisen analyysin ohjelmistoissa on tähän tehtävään omat 
komentonsa (esim.\ Mathematica: \verb|Fit|, Matlab: \verb|Polyfit|). Käsin laskettaessa, tai
etenkin haluttaessa tutkia interpolaatiopolynomin ominaisuuksia teoreettiselta kannalta,
voidaan käyttää esitysmuotoa
\begin{equation} \label{Lagrangen interpolaatiokaava}
p(x)=\sum_{i=0}^n f(x_i)L_i(x),
\end{equation}
\index{kantapolynomi (Lagrangen)}%
missä nk.\ (Lagrangen) \kor{kantapolynomit} $L_i(x)$ (astetta $n$) määräytyvät 
interpolaatioehdoista
\begin{equation} \label{Lagrangen kantaehdot}
L_i(x_j)=\begin{cases} 
         1, &\text{kun}\  j=i, \\ 0, &\text{kun}\ j\neq i \quad (j \in \{0, \ldots, n\}). 
         \end{cases}
\end{equation}
Helposti on tarkistettavissa, että nämä ehdot toteuttava polynomi on (vrt.\ kuvio)
\[
L_i(x) \,=\, \frac{(x-x_0)\,\cdots\,(x-x_{i-1})(x-x_{i+1})\,\cdots\,(x-x_n)}
               {\,(x_i-x_0)\,\cdots\,(x_i-x_{i-1})(x_i-x_{i+1})\,\cdots\,(x_i-x_n)}
       \,=\, \frac{\prod_{j \neq i} (x-x_j)}{\prod_{j \neq i} (x_i-x_j)}\,.
\]
\begin{figure}[H]
\begin{center}
\import{kuvat/}{kuvaipol-2.pstex_t}
\end{center}
\end{figure}
Ehdoista \eqref{Lagrangen kantaehdot} nähdään myös välittömästi, että polynomi 
\eqref{Lagrangen interpolaatiokaava} täyttää asetetut interpolaatioehdot 
\eqref{interpolaatioehdot}. Lagrangen interpolaatio-ongelman ratkeavuus on siis näin tullut 
todetuksi.
\begin{Exa} \label{interpolaatioesimerkki}
Säännöllisestä funktiosta tiedetään mittaustuloksina
\[
f_1=f(0.1)=1.4491,\quad f_2=f(0.2)=1.4832,\quad f_3=f(0.3)=1.5166.
\]
Arvioi $f(0)$.
\end{Exa}
\ratk Sovitetaan mittaustuloksiin toisen asteen Lagrangen interpolaatiopolynomi $p$ ja
arvioidaan $f(0)\approx p(0)$. Tässä on $\{x_0,x_1,x_2\}=\{0.1,0.2,0.3\}$, joten
\begin{align*}
p(0) &= 1.4491\cdot\frac{(0-0.2)\cdot (0-0.3)}{(0.1-0.2)\cdot (0.1-0.3)} \\
&+ 1.4832\cdot\frac{(0-0.1)\cdot (0-0.3)}{(0.2-0.1)\cdot (0.2-0.3)} \\
&+ 1.5166\cdot\frac{(0-0.1)\cdot (0-0.2)}{(0.3-0.1)\cdot (0.3-0.2)} \\
&= 3\cdot 1.4491-3\cdot 1.4832+1\cdot 1.5166=\underline{\underline{1.4143}}.
\end{align*}
Tässä 'mittaustulokset' on itse asiassa saatu funktiosta $f(x)=\sqrt{2+x}$, jolle $f(0)=1.4142$.
\loppu

Kantapolynomeihin perustuva esitysmuoto \eqref{Lagrangen interpolaatiokaava} on erityisen
kätevä silloin, kun halutaan arvioida funktion \pain{evaluoinnin} \pain{virheiden} 
(esim.\ mittausvirheiden) vaikutus interpolointitulokseen. Nimittäin jos virheellisten arvojen
$\tilde{f}_i$ (oikea arvo $=f_i$) perusteella lasketaan interpolaatiopolynomi $\tilde{p}$, niin
kaavan \eqref{Lagrangen interpolaatiokaava} mukaan
\[ 
p(x)-\tilde{p}(x) = \sum_{i=0}^n (f_i-\tilde{f}_i) L_i(x). 
\]
Jos erityisesti tiedetään, että $\abs{f_i-\tilde{f}_i} \le \delta,\ i=0,\ldots n$, niin
\[ 
\abs{p(x)-\tilde{p}(x)} \le \delta \sum_{i=0}^n \abs{L_i(x)} = \delta K(x). 
\]
Tässä määritelty virheen vahvistuskerroin $K(x)=\sum_{i=0}^n \abs{L_i(x)}$ siis kertoo, kuinka
paljon evaluointivirheet voivat pahimmillaan vahvistua pisteessä $x$.
\jatko \begin{Exa} (jatko) Esimerkissä on $L_1(0)=3$, $L_2(0)=-3$ ja $L_3(0)=1$. Siis
$K(0)=3+3+1=7$, eli virheiden vaikutus pisteessä $x=0$ on pahimmassa tapauksessa $7$-kertainen
yksittäisiin evaluointivirheisiin verrattuna. Jos oletetaan, että virheet ovat enintään
$10^{-4}$ itseisarvoltaan, niin pahin vaihtoehto toteutuu, kun
$f_1-\tilde{f}_1=-(f_2-\tilde{f}_2)=f_3-\tilde{f}_3=\pm 10^{-4}$. \loppu
%\[ 
%\abs{p(0) - \tilde{p}(0)} \le 10^{-4} \cdot (3+3+1) = 7 \cdot 10^{-4}. 
%\]
\end{Exa}

\subsection{Ekstrapolaatio}
\index{ekstrapolaatio|vahv}

Esimerkissä \ref{interpolaatioesimerkki} funktiota approksimoitiin interpolaatiopisteiden 
virittämän välin ulkopuolella, jolloin sanotaan että kyse on \kor{ekstrapolaatiosta}. 
Ekstrapolaatio on vanhastaan hyvin suosittu ja melko yleispätevä tapa parantaa numeeristen 
laskujen tarkkuutta. Ekstrapolaatiota voidaan käyttää aina, kun laskettavan suureen voidaan 
otaksua riippuva säännöllisellä (eli sileällä) tavalla jostakin laskentaan liittyvästä 
parametrista. Olkoon esimerkiksi laskettava suure reaaliluku $a$, joka määräytyy raja-arvona
\[
a=\lim_{h\kohti 0^+} f(h),
\]
missä jokainen $f(h)$, $h>0$, on laskettavissa, mutta laskenta tulee yhä työläämmäksi $h$:n 
pienetessä. Jos nyt voidaan olettaa, että funktio $f(x)$ on säännöllinen jollakin välillä 
$[0,b]$, $b>0$, voidaan numeerisen algoritmin antamia approksimaatioita
\[
a\approx a_n=f(x_n),\quad n=1,2,\ldots\quad (x_n\kohti 0^+)
\]
parantaa ekstrapolaatiolla. Näin syntyy nk.\ \kor{ekstrapolaatiotaulukko}, jossa laskettuihin 
tuloksiin sovitetaan yhä korkeampiasteisia polynomeja $p(x)$, ja arvioidaan kunkin polynomin
avulla $a\approx p(0)$\,:

\begin{center}
\begin{tabular}{lllll}
 & $\text{aste}=0$ & $\text{aste}=1$ & $\text{aste}=2$ & $\text{aste}=3$ \\ \hline \\
$x_1$ & $f(x_1)$ \\
$x_2$ & $f(x_2)$ & $p^{(1,2)}(0)$ \\
$x_3$ & $f(x_3)$ & $p^{(2,3)}(0)$ & $p^{(1,3)}(0)$ \\
$x_4$ & $f(x_4)$ & $p^{(3,4)}(0)$ & $p^{(2,4)}(0)$ & $p^{(1,4)}(0)$
\end{tabular}
\end{center}
Tässä $p^{(i,j)}(x)$ tarkoittaa pisteisiin $\,x_i\ldots x_j\,$ sovitettua interpolaatiopolynomia
astetta $j-i$ ($p^{(i,i)}(x)=f(x_i)=$ vakio). Osoittautuu, että taulukon sarakkeet määräytyvät
palautuvasti edellisen sarakkeen avulla. Nimittäin
\begin{equation} \label{Nevillen kaava}
\boxed{\ p^{(i,j)}(x)=\frac{(x_j-x)p^{(i,j-1)}(x)+(x-x_i)p^{(i+1,j)}(x)}{x_j-x_i}\,. \quad}
\end{equation}
Tämä palautuskaava (perustelu induktiolla: Harj.teht.\,\ref{H-dif-6: Nevillen kaava}) helpottaa
taulukon muodostamista huomattavasti.\footnote[2]{Kaavaan \eqref{Nevillen kaava} perustuvaa
ekstrapolaatiotaulukkoa sanotaan \kor{Nevillen kaavioksi}. Ennen tietokoneiden aikaa
tällaisilla (käsinlaskua helpottavilla) algoritmisilla keksinnöillä oli huomattava käytännön
merkitys. \index{Nevillen kaavio|av}} 
Seuraavassa esimerkki kaavaan \eqref{Nevillen kaava} perustuvasta 'laskemisen taiteesta'.
\begin{Exa} \label{Neville} \index{Stirlingin kaava}
\kor{Stirlingin kaavan} mukaan $\phi(n)=n!/(\sqrt{2\pi n}\,e^{-n}n^n) \approx 1$ suurilla $n$:n
arvoilla. Ekstrapoloi $\phi(100)$ arvoista $\phi(n)$, $n=5\ldots 9$, kun tiedetään lisäksi,
että $\phi(n)=f(1/n)$, missä $f(x)$ on sileä funktio välillä $[0,1]$.
\end{Exa}
\ratk Muodostetaan ekstrapolaatiotaulukko

\begin{center}
\begin{tabular}{llllll}
$x_i$  & $f(x_i)$&$\text{aste}=1$&$\text{aste}=2$&$\text{aste}=3$&\text{aste}=4 \\ \hline \\
$1/5$  & $1.0167..$ \\
$1/6$  & $1.0139..$ & $1.00076..$ \\
$1/7$  & $1.0119..$ & $1.00078..$ & $1.000823..$ \\
$1/8$  & $1.0104..$ & $1.00079..$ & $1.000827..$ & $1.000833565..$ \\
$1/9$  & $1.0092..$ & $1.00080..$ & $1.000830..$ & $1.000833632..$ & $1.000833708..$ \\
\end{tabular}
\end{center}
Oletettavasti neljännen asteen interpolaatio antaa tarkimman tuloksen, joten
$\phi(100)\approx\underline{\underline{1.00083371}}$. (Oikea arvo on on $1.000833677..\,$)
\loppu

Esimerkissä olennaista oli lisätieto, joka mahdollisti oikean muuttujan valinnan ($x_i=1/n_i$)
ekstrapoloinnissa.

\subsection{*Yleistetty polynomi-interpolaatio}
\index{interpolaatiopolynomi!b@yleistetty|vahv}

Jos $p$ on polynomi astetta $n$ ja toteuttaa ehdot
\begin{equation} \label{yleiset interpolaatioehdot} 
\begin{aligned}
p^{(k)}(x_i)&= f^{(k)}(x_i),\quad k=0\ldots\nu_i-1,\ \ i=1\ldots m, \\ 
           &\qquad\ \text{missä}\quad \nu_i \in \N \quad \text{ja} \quad \sum_{i=1}^m \nu_i=n+1,
\end{aligned} \end{equation}
niin sanotaan, että $p$ on funktion $f$ \kor{yleistetty interpolaatiopolynomi} pisteissä
$x_i$. Tässä ja jatkossa oletetaan, että ehdoissa \eqref{yleiset interpolaatioehdot}
esiintyvät $f$:n derivaatat (jos $\nu_i \ge 2$) ovat olemassa pisteissä $x_i$. Ehtojen
\eqref{yleiset interpolaatioehdot} mukaisesti interpolaatiopisteiden lukumäärä $m$ voi
yleistetyssä polynomi-interpolaatiossa astetta $n$ olla mikä tahansa välillä $1 \le m \le n+1$.
Tapauksessa $m=n+1$ on oltava $\nu_i=1\ \forall i$, jolloin kyseessä on Lagrangen
interpolaatio. Toisessa ääripäässä ($m=1$) on taas oltava $\nu_1=n+1$, jolloin $p(x)=f$:n
Taylorin polynomi $T_n(x,x_1)$.
\begin{Prop} Jos $p$ on polynomi astetta $n$, niin $p$ määräytyy ehdoista 
\eqref{yleiset interpolaatioehdot} yksikäsitteisesti. 
\end{Prop}
\tod Interpolaatio-ongelman \eqref{yleiset interpolaatioehdot} ratkeavuus voidaan todeta
samaan tapaan kuin Lagrangen interpolaation tapauksessa, ks.\
Harj.teht.\,\ref{H-dif-6: yleinen interpolaatio-ongelma}. Yksikäsitteisyyden toteamiseksi
riittää osoittaa, että jos $f^{(k)}(x_i)=0\ \forall i,k$, niin on oltava $p=0$
(vrt.\ Proposition \ref{interpolaatiopolynomin yksikäsitteisyys} todistus). Tässä tapauksessa
on $x_i$  polynomin $p$ $\,\nu_i$-kertainen nollakohta interpolaatioehtojen
\eqref{yleiset interpolaatioehdot} mukaan, joten on oltava
\[ 
p(x) = q(x)\,\prod_{i=1}^m (x-x_i)^{\nu_i} = q(x)\,w(x), 
\]
missä $q$ on polynomi. Mutta $p$ on astetta $n$, ja ehtojen \eqref{yleiset interpolaatioehdot}
perusteella $w$ on astetta $n+1$, joten ainoa mahdollisuus on $q=0$, jolloin myös $p=0$.
\loppu

Seuraava yleinen polynomiapproksimaatiotulos, jonka täydellistä todistusta ei esitetä,
sisältää erikoistapauksina sekä Lauseen \ref{Lagrangen interpolaatiovirhe} että Taylorin
lauseen \ref{Taylor}.
\begin{Lause} \label{usean pisteen Taylor} \vahv{(Usean pisteen Taylorin lause)} 
\index{Taylorin lause!b@usean pisteen|emph}
Olkoon $f$ jatkuva välillä $[a,b]$ ja $n+1$ kertaa derivoituva välillä $(a,b)$, missä $[a,b]$
on pisteen $x$ ja erillisten pisteiden $x_0,\ldots,x_m$ virittämä väli. Tällöin jos $p$ on
ehdoilla \eqref{yleiset interpolaatioehdot} määritelty yleistetty interpolaatiopolynomi
astetta $n$, niin jollakin $\xi \in (a,b)$ pätee virhekaava
\[
f(x)-p(x) = \frac{1}{(n+1)!}\,f^{(n+1)}(\xi)\,\prod_{i=1}^m (x-x_i)^{\nu_i}.
\]
\end{Lause}
\tod (idea) Jos $x \in \{x_1,\ldots, x_m\}$, on väite tosi $\forall\xi$. Olkoon siis 
$x \neq x_i\ \forall i$, ja otetaan tarkastelun kohteeksi funktio 
\[
g(t)=f(t)-p(t)- Hw(t), \quad H=[f(x)-p(x)]/w(x), \quad w(x)=\prod_{i=1}^m (x-x_i)^{\nu_i}.
\]
Koska $w(t)=t^{n+1}+(\text{polynomi astetta}\ n)$, niin $w^{(n+1)}(t)=(n+1)!\ \forall t$, joten
virhekaava väittää, että $g^{(n+1)}(\xi)=0$ jollakin $\xi \in (a,b)$. Aiemmin tämä on osoitettu
tapauksissa $m=1,\ \nu_1=n+1$  (Taylorin lause) ja $m=n+1,\ \nu_i=1$
(Lause \ref{Lagrangen interpolaatiovirhe}). Yleisemmässäkin tapauksessa on päättely vastaava.
Esimerkiksi olkoon 
\[ 
\nu_i=2,\ i=1\ldots m, \quad m \ge 2, \quad n=2m-1. 
\]
Tällöin koska $g(x_i)=0,\ i=1\ldots m$, ja $g(x)=0$, on pisteiden $x$ ja $x_i,\ i=1\ldots m$,
välisillä avoimilla väleillä kullakin $g'$:n nollakohta (yhteensä $m$ kpl). Toisaalta on myös 
$g'(x_i)=0,\ i=1\ldots m$, joten $g'$:lla on välillä $[a,b]$ ainakin $2m$ nollakohtaa. Tästä 
seuraa (vrt.\ Lauseen \ref{Lagrangen interpolaatiovirhe} todistus), että ainakin yhdessä
pisteessä $\xi \in (a,b)$ on $g^{(2m)}(\xi)=g^{(n+1)}(\xi)=0$, eli väite on tosi oletetussa
tapauksessa. (Yleinen tapaus sivuutetaan.) \loppu

Edellä tarkastellussa tapauksessa, jossa $\nu_i=2,\ i=1\ldots m$, sanotaan polynomia $p$ 
funktion $f$
\index{interpolaatiopolynomi!c@Hermiten} \index{Hermiten!b@interpolaatio(polynomi)}%
\kor{Hermiten} interpolaatiopolynomiksi pisteissä $x_i$.
\setcounter{Exa}{0}
\begin{Exa} (jatko) Montako tasavälistä jakopistettä $x_i$ tarvitaan, jos pisteiden välissä
käytetään kolmannen asteen Hermiten interpolaatiota?
\end{Exa}
\ratk Lauseen \ref{usean pisteen Taylor} mukaan interpolaatiovirhe välillä $[x_i,x_{i+1}]$ on 
enintään
\begin{align*}
\abs{E(x)-p(x)} &\leq \max_{x\in [x_i,x_{i+1}]} \frac{1}{4!} (x-x_i)^2(x-x_{i+1})^2
                      \max_{\xi\in[x_i,x_{i+1}]}\abs{E^{(4)}(\xi)} \\
                &=    \frac{1}{384}\,h^4\max_{x\in[x_i,x_{i+1}]}\abs{E^{(4)}(x)}.
\end{align*}
Välillä $[1,2]$ on $\abs{E^{(4)}(x)} \le 16/e$, joten vaadittuun tarkkuuteen riittää:
\[
\frac{1}{384}\cdot\frac{16}{e}\cdot\left(\frac{1}{N-1}\right)^4\leq 5\cdot 10^{-9} 
           \qekv \underline{\underline{N\geq 43}}. \loppu
\]

\Harj
\begin{enumerate}

\item
Funktiosta $f(x)$ tiedetään: $f(-0.1)=1.70 \pm 0.05$, $f(0.2)=1.80 \pm 0.03$ ja
$-1 \le f''(x) \le 0$ välillä $[-0.1,\,0.2]$. Määritä mahdollisimman ahdas väli, jolla $f(0)$
varmasti sijaitsee.

\item
Funktion $e^x$ arvot on laskettu viiden merkitsevän numeron tarkkuudella (normaalipyöristys)
välin $[0,2]$ pisteissä $x_i=i/100,\ i=0 \ldots 200$. Arvioi, kuinka suuri on näistä
arvoista lasketun a) lineaarisen,\ b) kvadraattisen interpolaation virhe enintään välillä 
$[0,2]$. Arvioi erikseen pyöristysvirheiden vaikutus.

\item
Eräästä välillä $(0,\infty)$ määritellystä, säännöllisestä funktiosta $F$ tiedetään, että
$F$ saavuttaa absoluuttisen minimiarvonsa välillä $[1,2]$. Lisäksi tiedetään, että
$F(0.5)=\sqrt{\pi}$, $F(1)=1$, $F(1.5)=\frac{1}{2}\sqrt{\pi}$ ja $F(2)=1$. Laske $F$:n
minimikohta ja -arvo likimäärin käyttäen a) kvadraattista interpolaatiota pisteissä
$\{0.5,1,1.5\}$, \ b) kvadraaattista interpolaatiota pisteissä $\{1,1.5,2\}$, \ c) kolmannen
asteen interpolaatiota kaikissa neljässä pisteessä. (Oletukset täyttää $\Gamma$-funktio, ks.\
Harj.teht.\,\ref{integraalin laajennuksia}:\ref{H-int-7: Gamma} ja Propositio \ref{Gamma(1/2)}.)

\item
Parametrista käyrää $\vec r=\vec u(t)=x(t)\vec i+y(t)\vec j,\ t\in[a,b]$, approksimoidaan
pisteiden $(x(t_i),y(t_i)),\ i=0 \ldots n$ kautta kulkevalla murtoviivalla
$\vec r = \vec v(t)=\hat{x}(t)\vec i+\hat{y}(t)\vec j\ (a=t_0 < t_1 < \cdots < t_n = b)$. 
Jos $x(t)$ ja $y(t)$ ovat välillä $[a,b]$ kahdesti jatkuvasti derivoituvia, 
$\abs{x''(t)} \le M$, $\abs{y''(t)} \le M$ ja $t_i-t_{i-1} \le h$, niin kuinka suuri on
enintään $\delta_h=\max_{t\in[a,b]}\abs{\vec u(t)-\vec v(t)}$\,? Vertaa arviota todellisuuteen 
tapauksessa $x(t)=R\cos t,\ y(t)=R\sin t,\ t\in[0,\pi]$.

\item %\index{zzb@\nim!Rukkaus}
Heilurilla varustettu seinäkello jätättää vuorokaudessa $5$ min $24$ s. Kelloa
rukataan kiertämällä heilurin päässä olevaa ruuvia kiinni $5$ täyttä kierrosta (jolloin
heilurin varsi hieman lyhenee). Rukkauksen jälkeen havaitaan kellon edistävän $3$ min $36$ s
vuorokaudessa. \ a) Miten kelloa kannattaa seuraavaksi rukata? \ b) Olkoon kellon suhteellinen
edistämä vuorokaudessa $f(x)$, missä $x=$ ruuvin kiertymä (kierroksina) oikeasta säätöasennosta
kiinni päin. Arvioi, montako sekuntia kello edistää tai jätättää vuorokaudessa a-kohdan
rukkauksen jälkeen, jos oletetaan, että $f''(0)=+2 \cdot 10^{-5}$.

\item
Ekstrapoloi luvuista $7! \ldots 10!$ raja-arvo $\,a=\lim_{n\kohti\infty} n!e^nn^{-n-\frac{1}{2}}$
ja vertaa tarkkaan arvoon $a=\sqrt{2\pi}$. (Vrt.\ Esimerkki \ref{Neville}.)

\item 
Todista Lause \ref{usean pisteen Taylor} tapauksessa $m=n=3,\ \nu_1=\nu_3=1,\ \nu_2=2$.

\item
Funktioiden $\sin x$ ja $\cos x$ arvot halutaan määrätä välillä $[0,\pi/4]$ siten, että
funktioiden arvot lasketaan ensin riittävän tarkasti ko.\ välin tasavälisessä pisteistössä ja
sen jälkeen käytetään kolmannen asteen Hermiten interpolaatiota pisteiden välillä. Montako
interpolointipistettä tarvitaan, jos virheen sallitaan olevan enintään $5 \cdot 10^{-9}$\,?

\pagebreak

\item \label{H-dif-6: interpolaatiot ja differessikaavat}
Halutaan laskea numeerisesti funktion $f$ derivaatta $f^{(k)}(a)$ approksimaatiolla
$f^{(k)}(a) \approx p^{(k)}(a)$, missä $p(x)$ on $f$:n interpolaatiopolynomi. Näytä, että
seuraavissa tapauksissa vaihtoehtoiset interpolaatiot johtavat samaan
differenssiapproksimaatioon --- millaiseen? (Vrt.\ edellinen luku.) \vspace{1mm}\newline
a) $k=1$: Lineaarinen interpolaatio pisteissä $a \pm h$ tai kvadraattinen interpolaatio
pisteissä $a$ ja $a \pm h$. \newline
b) $k=2$: Kvadraattinen interpolaatio pisteissä $a$ ja $a \pm h$ tai yleistetty kolmannen
asteen interpolaatio samoissa pisteissä lisäehdolla $p'(a)=f'(a)$. 

\item (*) \label{H-dif-6: Nevillen kaava}
Todista palautuskaava \eqref{Nevillen kaava} induktiolla.

\item (*) \index{zzb@\nim!Rajatieto}
(Rajatieto) Lukujonosta $\seq{a_n}$, missä $a_n = \sum_{k=1}^n k^{-5/4},\ n=1,2, \ldots\,$
tiedetään, että on olemassa (tuntemattomat) $a,b,c\in\R$ siten, että suurilla $n$:n arvoilla
pätee
\[
a_n=a+n^{-1/4}(b+c n^{-1})+\Ord{n^{-9/4}}. 
\]
Ekstrapoloi raja-arvo $\lim_n a_n=a$ tämän tiedon perusteella mahdollisimman tarkasti tuloksista
$a_{100}=3.331779$, $a_{200}=3.532117$, $a_{400}=3.700964$. (Tarkka arvo kuudella
desimaalilla: $a=4.595112$.)

\item (*)  \label{H-dif-6: yleinen interpolaatio-ongelma}
Näytä, että jos interpolaatio-ongelmassa \eqref{yleiset interpolaatioehdot} on
$f^{(k)}(x_i)=1$ kun $i=j$ ja $k=l$ $(1 \le j \le m,\ 0 \le l \le \nu_j-1)$ ja
$f^{(k)}(x_i)=0$ muulloin, niin ongelmalla on ratkaisu
\[
p(x) = L_{j,l}(x) = q(x) \prod_{\substack{i=1 \\ i \neq j}}^m (x-x_i)^{\nu_i},
\]
missä $q$ on polynomi muotoa $q(x)=\sum_{r=l}^{\nu_j-1} c_r (x-x_j)^r$. Päättele edelleen, että
interpolaatio-ongelman \eqref{yleiset interpolaatioehdot} ratkaisu yleisessä tapauksessa on
\[
p(x) = \sum_{i=1}^m \sum_{k=0}^{\nu_i-1} f^{(k)}(x_i) L_{i,k}(x).
\] 

\end{enumerate}