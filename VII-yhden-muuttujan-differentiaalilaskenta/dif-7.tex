\section{Interpolaatiopolynomit} \label{interpolaatiopolynomit}
\alku

Tavallisessa \kor{polynomi-interpolaatiossa} on lähtöajatuksena, että funktiosta $f$ tunnetaan
vain äärellinen määrä pistearvoja:
\[
f(x_i)=f_i,\quad i=0,\ldots,n.
\]
Näiden tietojen perusteella halutaan esittää $f$ likimäärin polynomina muuallakin kuin
pisteissä $x_i$, esim.\ jollakin välillä. 'Interpolaatio' on kyseessä erityisesti silloin, kun
$f(x)$ halutaan laskea 'välipisteissä' $x \in [\min\{x_i\},\max\{x_i\}]$, muussa tapauksessa
(eli kun $x < \min\{x_i\}$ tai $x > \max\{x_i\}$) sanotaan, että kyseessä on
\kor{ekstrapolaatio}. Annetut tiedot voivat olla esim.\ mitattua 'dataa'. Toinen yleinen
sovellustilanne on sellainen, jossa funktio $f$ tunnetaan epäsuorasti, esim.\
differentiaaliyhtälön ratkaisuna, ja halutaan laskea arvot $f_i$ valituissa pisteissä $x_i$.
Tällöin polynomi-interpolaatioista on hyötyä itse laskenta-algoritmin suunnittelussa.

Funktion polynomiapproksimaatiossa on yleensä perusoletuksena (tai ainakin toivomuksena), että
funktio on riittävän säännöllinen, jolloin Taylorin lauseen perusteella tiedetään, että 
funktiota voi (ainakin lyhyellä välillä) approksimoida hyvin polynomilla, nimittäin Taylorin
polynomilla. Koska itse Taylorin polynomia ei ym.\ tiedoista voi suoraan määrätä, on luontevaa
valita approksimoivaksi polynomiksi $p$ sellainen, joka mahdollisimman hyvin sopii annettuihin 
tietoihin, eli toteuttaa
\begin{equation} \label{interpolaatioehdot}
p(x_i)=f(x_i),\quad i=0 \ldots n.
\end{equation}
Koska tässä on ehtoja $n+1$ kpl, on edelleen luonnollista valita $p$:n asteluvuksi $n$, jolloin
$p$:ssä on vapaita kertoimia myöskin $n+1$ kpl. Näin määriteltyä polynomia $p$ sanotaan $f$:n 
\kor{Lagrangen interpolaatiopolynomiksi} pisteissä $x_i$. Pisteitä $x_i$ sanotaan tässä 
yhteydessä \kor{interpolaatiopisteiksi} ja ehtoja \eqref{interpolaatioehdot} 
\kor{interpolaatioehdoiksi}. Nämä  ehdot todella määrittelevät yksikäsitteisen polynomin
astetta $n$:
\begin{Prop} \label{interpolaatiopolynomin yksikäsitteisyys}
Lagrangen interpolaatiopolynomi on yksikäsitteinen.
\end{Prop}
\tod Ehdot \eqref{interpolaatioehdot} toteuttavan interpolaatiopolynomin olemassaolo seuraa 
jäljempänä esitettävästä laskukaavasta \eqref{Lagrangen interpolaatiokaava}, joten riittää
näyttää yksikäsitteisyys. Olkoot siis $p_1$ ja $p_2$ molemmat ehdot \eqref{interpolaatioehdot}
täyttäviä polynomeja astetta $n$. Tällöin $q(x)=p_1(x)-p_2(x)$ on polynomi enintään astetta $n$
ja pätee
\[
q(x_i)=0, \quad i=0,\ldots,n,
\]
eli $q$:lla on $n+1$ reaalista nollakohtaa. Algebran peruslauseen mukaan on tällöin oltava
$q=0$. Siis $p_1=p_2$, eli ehdot \eqref{interpolaatioehdot} täyttävä polynomi on 
yksikäsitteinen. \loppu

Lagrangen interpolaatiopolynomin approksimaatiovirheelle pätee seuraava tulos, joka muistuttaa 
Taylorin polynomien virhekaavaa (Lause \ref{Taylor}). Jäljempänä nähdäänkin, että nämä kaksi
tulosta ovat erikoistapauksia yleisemmästä interpolaatiopolynomien virhekaavasta 
(Lause \ref{usean pisteen Taylor}). Jatkossa sanotaan äärellisen pisteistön $X$ 
\kor{virittämäksi} väliksi suljettua väliä $[a,b]$, missä $a = \min \{x \mid x \in X\}$ ja 
$b = \max \{x \mid x \in X\}$.
\begin{Lause} \label{Lagrangen interpolaatiovirhe}
Olkoon $f\,$ $n+1$ kertaa derivoituva avoimella välillä $A\supset[a,b]$, missä $[a,b]$ on
pisteen $x$ ja erillisten pisteiden $x_0,\ldots,x_n$ virittämä väli. Tällöin jos $p$ on $f$:n
Lagrangen interpolaatiopolynomi pisteissä $x_0,\ldots,x_n$, niin jollakin $\xi \in (a,b)$
pätee virhekaava
\[
f(x)-p(x)=\frac{1}{(n+1)!}\,f^{(n+1)}(\xi) \prod_{i=0}^n (x-x_i).
\]
\end{Lause}
\tod perustuu samaan ideaan kuin Taylorin lauseen 'lyhyen kaavan' mukainen todistus Luvussa 
\ref{taylorin lause}. Ensinnäkin jos $x\in\{x_0,\ldots,x_n\}$, on väittämä tosi jokaisella 
$\xi \in (a,b)$, joten voidaan olettaa, että $x\notin\{x_0,\ldots,x_n\}$. Merkitään
\[
w(t)=\prod_{i=0}^n (t-x_i)
\]
ja tutkitaan funktiota
\[
g(t)=f(t)-p(t)- Hw(t),\quad H=[f(x)-p(x)]/w(x).
\]
Funktion $g$ nollakohtia ovat interpolointipisteet $x_0,\ldots,x_n$ ja lisäksi piste $x$. Koska
nollakohtia välillä $[a,b]$ on siis ainakin $n+2$ kpl, ja koska näiden välissä on aina 
derivaatan nollakohta, on $g'(t)$:llä ainakin $n+1$ nollakohtaa avoimella välillä $(a,b)$. 
Tällöin $g''$:lla on ainakin $n$ nollakohtaa tällä välillä, ja lopulta $g^{(n+1)}$:lla ainakin
yksi nollakohta $\xi\in (a,b)$. Mutta tällöin
\begin{align*}
0 = g^{(n+1)}(\xi) &= f^{(n+1)}(\xi)-p^{(n+1)}(\xi)-H w^{(n+1)}(\xi) \\
                   &= f^{(n+1)}(\xi)-H (n+1)! \\[2mm]
      \impl\quad H &= \frac{f(x)-p(x)}{w(x)}=\frac{1}{(n+1)!}\,f^{(n+1)}(\xi) \loppu
\end{align*}

Interpolaatioperiaatteista on astelukuun $n=1$ perustuva \kor{lineaarinen interpolaatio} 
yksinkertaisuutensa vuoksi hyvin yleisesti käytetty. Esimerkiksi käyrän $y=f(x)$ approksimointi
pisteiden $(x_i,f(x_i))$ kautta kulkevalla murtoviivalla (vaikkapa kaarenpituuden 
määräämistarkoituksessa) tarkoittaa $f$:n \kor{paloittain lineaarista} interpolaatiota.
\begin{figure}[H]
\begin{center}
\import{kuvat/}{kuvaipol-1.pstex_t}
\end{center}
\end{figure}
Lineaarisen interpolaation virhe interpolointivälillä $[x_0,x_1]$ on Lauseen 
\ref{Lagrangen interpolaatiovirhe} mukaan enintään
\begin{align*}
\max_{x\in [x_0,x_1]} \abs{f(x)-p(x)} 
         &= \max_{x\in [x_0,x_1]} \abs{\frac{1}{2}(x-x_0)(x-x_1)f''(\xi)} \\
         &\le \frac{1}{8}\,(x_1-x_0)^2\max_{x\in [x_0,x_1]} \abs{f''(x)}.
\end{align*}
Jos interpolointi tapahtuu paloittain välillä $[a,b]$ ja interpolointipisteiden väli on
enintään $h$, niin ko.\ approksimaation maksimivirhe on siis enintään
\[
\max_{x\in [a,b]} \abs{f(x)-\tilde{f}(x)}\leq\frac{1}{8}\,h^2\max_{x\in [a,b]} \abs{f''(x)}.
\]
Tämä arvio ei ole parannettavissa, sillä jos $f$ on toisen asteen polynomi (jolloin $f''(x)=$ 
vakio) ja interpolointipisteet ovat tasaväliset, niin Lauseen
\ref{Lagrangen interpolaatiovirhe} virhekaavan mukaan virhearvion yläraja totetuu
peräkkäisten interpolointipisteiden puolivälissä.

Jos \kor{kvadraattisessa} (= toisen asteen) interpolaatiossa pisteet $x_0,x_1,x_2$ ovat 
tasavälein ja väli $=h$, niin
\[
\max_{x\in [x_0,x_2]} \abs{(x-x_0)(x-x_1)(x-x_2)}
     = \max_{x\in [-h,h]} \abs{x}(h^2-x^2)=\frac{2}{3\sqrt{3}}\,h^3,
\]
joten kvadraattinen interpolaatiovirhe on tasavälisten interpolointipisteiden virittämällä 
välillä enintään
\[
\max_{x\in [x_0,x_2]} \abs{f(x)-p(x)}
   \le \frac{1}{9\sqrt{3}}\,h^3\max_{x\in [x_0,x_2]} \abs{f'''(x)}, \quad (h=x_1-x_0=x_2-x_1).
\]
\begin{Exa}
Välillä $(0,\infty)$ määritelty funktio $F$ toteuttaa ehdot
\[
F'(x) = -\frac{e^{-x}}{x}\,, \quad x>0, \qquad \lim_{x\kohti\infty} F(x)=0.\footnote[2]{Funktio
on nimeltään \kor{eksponentiaali-integraalifunktio}.}
\]
Halutaan määrittää $F$:n arvot likimäärin välillä $[1,2]$ laskemalla ensin $F$ pisteissä 
$x_i=1+(i-1)h,\ i=0\ldots n,\ h=1/n$ (oletetetaan, että tämä onnistuu hyvin tarkasti) ja 
käyttämällä interpolaatiota muissa pisteissä. Kuinka suuri on $n$:n oltava, jos käytetään \ 
a) lineaarista, \ b) kvadraattista interpolaatiota ja halutaan, että interpolaatiovirhe on 
enintään $5\cdot 10^{-9}$\,?
\end{Exa} 
\ratk Koska $F'(x) = -e^{-x}/x,\ x>0$, niin
\begin{align*}
F''(x)  &= e^{-x}(\tfrac{1}{x}+\tfrac{1}{x^2}) 
              \qimpl \abs{F''(x)}\leq 2/e,\quad x\in [1,2], \\
F'''(x) &= -e^{-x}(\tfrac{1}{x}+\tfrac{2}{x^2}+\tfrac{2}{x^3}) 
              \qimpl \abs{F'''(x)}\leq 5/e,\quad x\in [1,2].
\end{align*}
Näin ollen riittää valita $N=n+1$ siten, että pätee
\begin{align*}
&\text{a)} \quad \frac{1}{8}\cdot\frac{2}{e}\cdot\left(\frac{1}{N-1}\right)^2
                \le 5\cdot 10^{-9} \qimpl \underline{\underline{N\geq 11075}} \\
&\text{b)}\quad \frac{1}{9\sqrt{3}}\cdot\frac{5}{e}\cdot\left(\frac{1}{N-1}\right)^3
                \le 5\cdot 10^{-9} \qimpl \underline{\underline{N\geq 288}} \loppu
\end{align*}

\subsection{Lagrangen kantapolynomit}

Korkeampiasteisissa polynomi-interpolaatioissa on usein kyse \kor{polynomisovituksesta} 
pisteisiin $(x_i,f_(x_i))$, jolloin arvot $f(x_i)$ voivat olla esimerkiksi mitattuja. 
Kehittyneissä numeerisen ja symbolisen analyysin ohjelmistoissa on tähän tehtävään omat 
komentonsa (esim.\ Mathematica: \verb|Fit|, Matlab: \verb|Polyfit|). Käsin laskettaessa, tai
haluttaessa tutkia interpolaatiopolynomin ominaisuuksia teoreettiselta kannalta, voidaan
käyttää esitysmuotoa
\begin{equation} \label{Lagrangen interpolaatiokaava}
p(x)=\sum_{i=0}^n f(x_i)L_i(x),
\end{equation}
missä nk.\ (Lagrangen) \kor{kantapolynomit} $L_i(x)$ (astetta $n$) määräytyvät 
interpolaatioehdoista
\begin{equation} \label{Lagrangen kantaehdot}
L_i(x_j)=\begin{cases} 
         1, &\text{kun}\  j=i \\ 0, &\text{kun}\ j\neq i \quad (j \in \{0, \ldots, n\}) 
         \end{cases}
\end{equation}
Helposti on tarkistettavissa, että nämä ehdot toteuttava polynomi on (vrt.\ kuvio)
\[
L_i(x) \,=\, \frac{(x-x_0)\,\cdots\,(x-x_{i-1})(x-x_{i+1})\,\cdots\,(x-x_n)}
               {\,(x_i-x_0)\,\cdots\,(x_i-x_{i-1})(x_i-x_{i+1})\,\cdots\,(x_i-x_n)}
       \,=\, \frac{\prod_{j \neq i} (x-x_j)}{\prod_{j \neq i} (x_i-x_j)}\,.
\]
\begin{figure}[H]
\begin{center}
\import{kuvat/}{kuvaipol-2.pstex_t}
\end{center}
\end{figure}
Ehdoista \eqref{Lagrangen kantaehdot} nähdään myös välittömästi, että polynomi 
\eqref{Lagrangen interpolaatiokaava} täyttää asetetut interpolaatioehdot 
\eqref{interpolaatioehdot}. Lagrangen interpolaatio-ongelman ratkeavuus on siis näin tullut 
todetuksi.
\begin{Exa} \label{interpolaatioesimerkki}
Säännöllisestä funktiosta tiedetään mittaustuloksina
\[
f(0.3)=1.5166,\quad f(0.2)=1.4832,\quad f(0.1)=1.4491.
\]
Arvioi $f(0)$.
\end{Exa}
\ratk Sovitetaan mittaustuloksiin toisen asteen Lagrangen interpolaatiopolynomi ja arvioidaan 
$f(0)\approx p(0)$. Tässä on $\{x_0,x_1,x_2\}=\{0.1,0.2,0.3\}$, joten
\begin{align*}
p(0) &= 1.4491\cdot\frac{(0-0.2)\cdot (0-0.3)}{(0.1-0.2)\cdot (0.1-0.3)} \\
&+ 1.4832\cdot\frac{(0-0.1)\cdot (0-0.3)}{(0.2-0.1)\cdot (0.2-0.3)} \\
&+ 1.5166\cdot\frac{(0-0.1)\cdot (0-0.2)}{(0.3-0.1)\cdot (0.3-0.2)} \\
&= 3\cdot 1.4491-3\cdot 1.4832+1\cdot 1.5166=\underline{\underline{1.4143}}.
\end{align*}
Tässä 'mittaustulokset' on itse asiassa saatu funktiosta $f(x)=\sqrt{2+x}$, jolle $f(0)=1.4142$.
\loppu

Kantapolynomeihin perustuva esitysmuoto \eqref{Lagrangen interpolaatiokaava} on erityisen
kätevä silloin, kun halutaan arvioida \pain{mittausvirheiden} interpolointitulokseen. 
Nimittäin jos virheellisten mittaustulosten $f(x_i)=\tilde{f}_i$ (oikea tulos $=f_i$)
perusteella lasketaan interpolaatiopolynomi $\tilde{p}$, niin kaavan
\eqref{Lagrangen interpolaatiokaava} mukaan
\[ 
p(x)-\tilde{p}(x) = \sum_{i=0}^n (f_i-\tilde{f}_i) L_i(x). 
\]
Jos erityisesti tiedetään, että $\abs{f_i-\tilde{f}_i} \le \delta,\ i=0,\ldots n$, niin
\[ 
\abs{p(x)-\tilde{p}(x)} \le \delta \sum_{i=0}^n \abs{L_i(x)} = \delta K(x). 
\]
Tässä määritelty virheen vahvistuskerroin $K(x)$ siis kertoo, kuinka paljon (itseis\-arvoltaan 
enintään tietyn suuruiset) mittausvirheet voivat pahimmillaan vahvistua pisteessä $x$.
\jatko \begin{Exa} (jatko) Jos oletetaan, että annettujen mittaustulosten virheet ovat enintään
$10^{-4}$ itseisarvoltaan, niin virheiden vaikutus pisteessä $x=0$ on enintään
\[ 
\abs{p(0) - \tilde{p}(0)} \le 10^{-4} \cdot (3+3+1) = 7 \cdot 10^{-4}. 
\]
Virheen vaikutus pisteessä $x=0$ voi siis pahimmassa tapauksessa olla $7$-kertainen
mittausvirheisiin verrattuna. \loppu
\end{Exa}

\subsection{Ekstrapolaatio}

Esimerkissä \ref{interpolaatioesimerkki} funktiota approksimoitiin interpolaatiopisteiden 
virittämän välin ulkopuolella, jolloin sanotaan että kyse on \kor{ekstrapolaatiosta}. 
Ekstrapolaatio on vanhastaan hyvin suosittu ja melko yleispätevä tapa parantaa numeeristen 
laskujen tarkkuutta. Ekstrapolaatiota voidaan käyttää aina, kun laskettavan suureen voidaan 
otaksua riippuva säännöllisellä (eli sileällä) tavalla jostakin laskentaan liittyvästä 
parametrista. Olkoon esimerkiksi laskettava suure reaaliluku $a$, joka määräytyy raja-arvona
\[
a=\lim_{h\kohti 0^+} f(h),
\]
missä jokainen $f(h)$, $h>0$, on laskettavissa, mutta laskenta tulee yhä työläämmäksi $h$:n 
pienetessä. Jos nyt voidaan olettaa, että funktio $f(x)$ on säännöllinen jollakin välillä 
$[0,a]$, $a>0$, voidaan numeerisen algoritmin antamia approksimaatioita
\[
a\approx a_n=f(x_n),\quad n=1,2,\ldots\quad (x_n\kohti 0^+)
\]
parantaa ekstrapolaatiolla. Näin syntyy nk.\ \kor{ekstrapolaatiotaulukko}, jossa laskettuihin 
tuloksiin sovitetaan yhä korkeampiasteisia polynomeja $p(x)$, ja arvioidaan kunkin polynomin
avulla $a\approx p(0)$:

\begin{center}
\begin{tabular}{lllll}
 & $\text{aste}=0$ & $\text{aste}=1$ & $\text{aste}=2$ & $\text{aste}=3$ \\ \hline \\
$x_1$ & $f(x_1)$ \\
$x_2$ & $f(x_2)$ & $p^{(1,2)}(0)$ \\
$x_3$ & $f(x_3)$ & $p^{(2,3)}(0)$ & $p^{(1,3)}(0)$ \\
$x_4$ & $f(x_4)$ & $p^{(3,4)}(0)$ & $p^{(2,4)}(0)$ & $p^{(1,4)}(0)$
\end{tabular}
\end{center}
Tässä $p^{(i,j)}(x)$ tarkoittaa pisteisiin $x_i\ldots x_j$ sovitettua interpolaatiopolynomia. 
Osoittautuu, että taulukon sarakkeet määräytyvät palautuvasti edellisen sarakkeen avulla. 
Nimittäin
\begin{equation} \label{Nevillen kaava}
\boxed{\ p^{(i,j)}(x)=\frac{(x_j-x)p^{(i,j-1)}(x)+(x-x_i)p^{(i+1,j)}(x)}{x_j-x_i} \quad}
\end{equation}
Tämä palautuskaava (jonka voi perustella induktiolla --- 
Harj.teht.\,\ref{H-VIII-7: Nevillen kaava}) helpottaa taulukon muodostamista 
huomattavasti.\footnote[2]{Kaavaan \eqref{Nevillen kaava} perustuvaa ekstrapolaatiotaulukkoa 
sanotaan \kor{Nevillen kaavioksi}. Ennen tietokoneiden aikaa tällaisilla algoritmisilla 
keksinnöillä oli huomattava käytännön merkitys, koska ne helpottivat käsinlaskua.} 
Seuraavassa esimerkki kaavaan \eqref{Nevillen kaava} perustuvasta 'laskemisen taiteesta'.
\begin{Exa} \label{Neville}
Määritellään $\phi(n)=n!/(\sqrt{2\pi n}\,e^{-n}n^n)$. Ekstrapoloi $\phi(100)$ arvoista 
$\phi(n)$, $n=5\ldots 10$, kun tiedetään, että $\phi(n)=f(1/n)$, missä $f$ on sileä funktio 
välillä $[0,1]$.
\end{Exa}
\ratk Muodostetaan ekstrapolaatiotaulukko

\begin{center}
\begin{tabular}{lllll}
$x_i$ & $f(x_i)$ & $\text{aste}=1$ & $\text{aste}=2$ & $\text{aste}=3$ \\ \hline \\
$1/5$ & $1.0167\ldots$ \\
$1/6$ & $1.0139\ldots$ & $1.00076\ldots$ \\
$1/7$ & $1.0119\ldots$ & $1.00078\ldots$ & $1.000823\ldots$ \\
$1/8$ & $1.0104\ldots$ & $1.00079\ldots$ & $1.000827\ldots$ & $1.000833565\ldots$ \\
$1/9$ & $1.0092\ldots$ & $1.00080\ldots$ & $1.000830\ldots$ & $1.000833632\ldots$ \\
$1/10$ & $1.0083\ldots$ & $1.00081\ldots$ & $1.000831\ldots$ & $1.000833658\ldots$
\end{tabular}
\end{center}

\begin{center}
\begin{tabular}{lllll}
& $\text{aste}=3$ & $\text{aste}=4$ & $\text{aste}=5$ \\ \hline \\
& $1.000833565\ldots$ \\
& $1.000833632\ldots$ & $1.000833708\ldots$ \\
& $1.000833658\ldots$ & $1.000833693\ldots$ & $1.000833679\ldots$
\end{tabular}
\end{center}
Siis $\phi(100)\approx\underline{\underline{1.000833679}}$. (Oikea arvo on on $1.0008336778..$)
\loppu

Esimerkissä olennaista oli tieto, että voidaan kirjoitaa $\phi(n)=f(1/n)$, missä $f$ on origon 
lähellä säännöllinen reaalifunktio. Ekstrapoloitaessa on siis osattava valita oikea muuttuja.

\subsection{*Yleistetty polynomi-interpolaatio}

Jos $p$ on polynomi astetta $n$ ja toteuttaa ehdot
\begin{equation} \label{yleiset interpolaatioehdot} 
\begin{aligned}
p^{(k)}(x_i) &= f^{(k)}(x_i),\quad k=0\ldots\nu_i-1,\ \ i=1\ldots m, \\ 
             &\qquad\ \text{missä}\quad \nu_i \in \N \quad \ja \quad \sum_{i=1}^m \nu_i=n+1,
\end{aligned} \end{equation}
niin sanotaan, että $p$ on funktion $f$ \kor{yleistetty interpolaatiopolynomi} pisteissä
$x_i$. Interpolaatiopisteiden lukumäärä $m$ voi siis yleistetyssä polynomi-interpolaatiossa
astetta $n$ olla mikä tahansa välillä $1 \le m \le n+1$. Tapauksessa $m=n+1$ on ehtojen
\eqref{yleiset interpolaatioehdot} mukaan oltava $\nu_i=1\ \forall i$, jolloin kyseessä on 
tavallinen Lagrangen interpolaatio. Toisessa ääripäässä ($m=1$) on taas oltava $\nu_1=n+1$, 
jolloin $p =f$:n Taylorin polynomi $T_n(x,x_1)$.
\begin{Prop} Jos $p$ on polynomi astetta $n$, niin $p$ määräytyy ehdoista 
\eqref{yleiset interpolaatioehdot} yksikäsitteisesti. 
\end{Prop}
\tod Interpolaatio-ongelman \eqref{yleiset interpolaatioehdot} ratkeavuus voidaan todeta
samaan tapaan kuin Lagrangen interpolaation tapauksessa, ks.\
Harj.teht.\,\ref{H-VIII-7: yleinen interpolaatio-ongelma}. Yksikäsitteisyyden toteamiseksi
riittää osoittaa, että jos $f^{(k)}(x_i)=0\ \forall i,k$, niin on oltava $p=0$
(vrt.\ Proposition \ref{interpolaatiopolynomin yksikäsitteisyys} todistus). Tässä tapauksessa
on $x_i$  polynomin $p$ $\,\nu_i$-kertainen juuri interpolaatioehtojen
\eqref{yleiset interpolaatioehdot} mukaan, joten on oltava
\[ 
p(x) = q(x)\,\prod_{i=1}^m (x-x_i)^{\nu_i} = q(x)\,w(x), 
\]
missä $q$ on polynomi. Mutta $p$ on astetta $n$, ja ehtojen \eqref{yleiset interpolaatioehdot}
perusteella $w$ on astetta $n+1$, joten ainoa mahdollisuus on $q=0$, jolloin myös $p=0$.
\loppu

Seuraava yleinen polynomiapproksimaatiotulos, jonka täydellistä todistusta ei esitetä,
sisältää erikoistapauksina sekä Lauseen \ref{Lagrangen interpolaatiovirhe} että Taylorin
lauseen \ref{Taylor}.
\begin{Lause} \label{usean pisteen Taylor} \vahv{(Usean pisteen Taylorin lause)} 
Olkoon $f\,$ $n+1$ kertaa derivoituva avoimella välillä $A\supset[a,b]$, missä $[a,b]$ on
pisteen $x$ ja erillisten pisteiden $x_0,\ldots,x_m$ virittämä väli. Tällöin jos $p$ on
ehdoilla \eqref{yleiset interpolaatioehdot} määritelty yleistetty interpolaatiopolynomi
astetta $n$, niin jollakin $\xi \in (a,b)$ pätee virhekaava
\[
f(x)-p(x) = \frac{1}{(n+1)!}\,f^{(n+1)}(\xi)\,\prod_{i=1}^m (x-x_i)^{\nu_i} 
          = \frac{1}{(n+1)!}\,f^{(n+1)}(\xi)\,w(x).
\]
\end{Lause}
\tod (idea) Jos $x \in \{x_1,\ldots, x_m\}$, on väite tosi $\forall\xi$. Olkoon siis 
$x \neq x_i\ \forall i$, ja otetaan tarkastelun kohteeksi funktio 
\[
g(t)=f(t)-p(t)- Hw(t),\quad H=[f(x)-p(x)]/w(x).
\]
Koska $w^{(n+1)}(t)=(n+1)!\ \forall t$, niin virhekaava väittää, että $g^{(n+1)}(\xi)=0$ 
jollakin $\xi \in (a,b)$. Aiemmin tämä on osoitettu tapauksissa $m=1,\ \nu_1=n+1$ 
(Taylorin lause) ja $m=n+1,\ \nu_i=1$ (Lause \ref{Lagrangen interpolaatiovirhe}). 
Yleisemmässäkin tapauksessa on päättely vastaava. Esimerkiksi olkoon 
\[ 
\nu_i=2,\ i=1\ldots m, \quad m \ge 2, \quad n=2m-1. 
\]
Tällöin koska $g(x_i)=0,\ i=1\ldots m$, ja $g(x)=0$, on pisteiden $x$ ja $x_i,\ i=1\ldots m$,
välisillä avoimilla väleillä kullakin $g'$:n nollakohta (yhteensä $m$ kpl). Toisaalta on myös 
$g'(x_i)=0,\ i=1\ldots m$, joten $g'$:lla on välillä $[a,b]$ ainakin $2m$ nollakohtaa. Tästä 
seuraa, että ainakin yhdessä pisteessä $\xi \in (a,b)$ on $g^{(2m)}(\xi)=g^{(n+1)}(\xi)=0$,
eli väite on tosi oletetussa tapauksessa. (Yleinen tapaus sivuutetaan). \loppu

Edellä tarkastellussa tapauksessa, jossa $\nu_i=2,\ i=1\ldots m$, sanotaan polynomia $p$ 
funktion $f$ \kor{Hermiten interpolaatiopolynomiksi} pisteissä $x_i$.
\setcounter{Exa}{0}
\begin{Exa} (jatko) Montako tasavälistä jakopistettä $x_i$ tarvitaan, jos pisteiden välissä
käytetään kolmannen asteen Hermiten interpolaatiota?
\end{Exa}
\ratk Lauseen \ref{usean pisteen Taylor} mukaan interpolaatiovirhe välillä $[x_i,x_{i+1}]$ on 
enintään
\begin{align*}
\abs{F(x)-p(x)} &\leq \max_{x\in [x_i,x_{i+1}]} \frac{1}{4!} (x-x_i)^2(x-x_{i+1})^2 M \\
                &=    \max_{x\in [0,h]} \frac{1}{4!} x^2(x-h)^2 M = \frac{1}{384}\,Mh^4,
\end{align*}
jos $\abs{F^{(4)}(\xi)}\leq M$. Välillä $[1,2]$ on $\abs{F^{(4)}(x)}\leq 16/e$, joten
vaadittuun tarkkuuteen riittää:
\[
\frac{1}{384}\cdot\frac{16}{e}\cdot\left(\frac{1}{N-1}\right)^4\leq 5\cdot 10^{-9} 
           \qimpl \underline{\underline{N\geq 51}} \loppu
\]

\Harj
\begin{enumerate}

\item
Funktiosta $f(x)$ tiedetään: $f(-0.1)=1.70 \pm 0.05$, $f(0.2)=1.80 \pm 0.03$ ja
$-1 \le f''(x) \le 0$ välillä $[-0.1,\,0.2]$. Määritä mahdollisimman ahdas väli, jolla $f(0)$
varmasti sijaitsee.

\item
Funktion $e^x$ arvot on laskettu viiden merkitsevän numeron tarkkuudella (normaalipyöristys)
välin $[0,2]$ pisteissä $x_i=i/100,\ i=0 \ldots 200$. Arvioi, kuinka suuri on näistä
arvoista lasketun a) lineaarisen,\ b) kvadraattisen interpolaation virhe enintään välillä 
$[0,2]$. Arvioi erikseen pyöristysvirheiden vaikutus.

\item
Eräästä välillä $(0,\infty)$ määritellystä, säännöllisestä funktiosta $F$ tiedetään, että
$F$ saavuttaa absoluuttisen minimiarvonsa välillä $[1,2]$. Tiedetään lisäksi, että
$F(0.5)=\sqrt{\pi}$, $F(1)=1$, $F(1.5)=\sqrt{\pi}/2$ ja $F(2)=1$. Laske $F$:n minimikohdalle
ja -arvolle likiarvo käyttäen a) kvadraattista interpolaatiota pisteissä $\{0.5,1,1.5\}$, \
b) kvadraaattista interpolaatiota pisteissä $\{1,1.5,2\}$, \ c) kolmannen asteen 
interpolaatiota kaikissa neljässä pisteessä.

\item
Parametrista käyrää $\vec r=\vec u(t)=x(t)\vec i+y(t)\vec j,\ t\in[a,b]$, approksimoidaan
pisteiden $(x(t_i),y(t_i)),\ i=0 \ldots n$ kautta kulkevalla murtoviivalla
$\vec r = \vec v(t)=\hat{x}(t)\vec i+\hat{y}(t)\vec j\ (a=t_0 < t_1 < \cdots < t_n = b)$. 
Jos $x(t)$ ja $y(t)$ ovat välillä $[a,b]$ kahdesti jatkuvasti derivoituvia, 
$\abs{x''(t)} \le M$, $\abs{y''(t)} \le M$ ja $t_i-t_{i-1} \le h$, niin kuinka suuri on
enintään $\delta_h=\max_{t\in[a,b]}\abs{\vec u(t)-\vec v(t)}$\,? Vertaa arviota todellisuuteen 
tapauksessa $x(t)=R\cos t,\ y(t)=R\sin t,\ t\in[0,\pi]$.

\item
Heilurilla varustettu seinäkello jätättää vuorokaudessa $5$ min $24$ s. Kelloa rukataan 
kiertämällä heilurin päässä olevaa ruuvia kiinni $5$ täyttä kierrosta (jolloin heilurin varsi
hieman lyhenee). Rukkauksen jälkeen havaitaan kellon edistävän $3$ min $36$ s vuorokaudessa.
\ a) Miten kelloa on seuraavaksi rukattava? \ b) Arvioi, kuinka tarkaksi kellon käynti tuli
a-kohdan rukkauksella, jos kellon suhteellinen edistämä on likimäärin
$f(x)=(1-\frac{x}{400})^{-1/2}-1$, missä $x=$ ruuvin kiertymä (kierroksina) oikeasta 
säätöasennosta mitattuna.

\item
Ekstrapoloi luvuista $7! \ldots 10!$ raja-arvo $\,a=\lim_{n\kohti\infty} n!e^nn^{-n-\frac{1}{2}}$
ja vertaa tarkkaan arvoon $a=1/\sqrt{2\pi}$ (vrt.\ Esimerkki \ref{Neville}).

\item
Funktioiden $\sin x$ ja $\cos x$ arvot halutaan määrätä välillä $[0,\pi/4]$ siten, että
funktioiden arvot lasketaan ensin riittävän tarkasti ko.\ välin tasavälisessä pisteistössä ja
sen jälkeen käytetään kolmannen asteen Hermiten interpolaatiota pisteiden välillä. Montako
interpolointipistettä tarvitaan, jos virheen sallitaan olevan enintään $5 \cdot 10^{-9}$\,?

\item 
Todista Lause \ref{usean pisteen Taylor} tapauksessa $m=n=3,\ \nu_1=\nu_3=1,\ \nu_2=2$.

\item (*) \label{H-VIII-7: Nevillen kaava}
Todista palautuskaava \eqref{Nevillen kaava} induktiolla.

\item (*)
Olkoot $x_0, \ldots, x_n$ erillisiä pisteitä ja olkoot luvut $y_i\in\R,\ i= 0 \ldots n$ 
annettuja. Halutaan löytää funktio muotoa
\[
F(x)=\sum_{j=0}^n c_j e^{jx}, \quad c_j\in\R
\]
siten, että pätee $F(x_i)=y_i,\ i=0 \ldots n$. Näytä, että ongelmalla on yksikäsitteinen
ratkaisu.

\item (*)
Lukujono $\seq{a_n}$, missä $a_n = \sum_{k=1}^n k^{-5/4},\ n=1,2, \ldots\,$, suppenee, mutta
niin hitaasti, että raja-arvoa ei voi määrätä suoraan summeeraamalla. Voidaan kuitenkin
otaksua, että suurilla $n$:n arvoilla pätee
\[
a_n=a+n^{-1/4}(c_0+c_1n^{-1}+c_2n^{-2}+ \ldots)\,. 
\]
Määritä tämän oletuksen perusteella raja-arvo $a$ likimäärin ekstrapoloimalla tuloksista
$a_{100}=3.3317786$, $a_{200}=3.5321167$, $a_{400}=3.7009640$, $a_{800}=3.8431087$. Vertaa
tarkkaan arvoon $a=4.5951118$.

\item (*)  \label{H-VIII-7: yleinen interpolaatio-ongelma}
Näytä, että jos interpolaatio-ongelmassa \eqref{yleiset interpolaatioehdot} on
$f^{(k)}(x_i)=1$ kun $i=j$ ja $k=l$ $(1 \le j \le m,\ 0 \le l \le \nu_j-1)$ ja
$f^{(k)}(x_i)=0$ muulloin, niin ongelmalla on ratkaisu
\[
p(x) = L_{j,l}(x) = q(x) \prod_{\substack{i=1 \\ i \neq j}}^m (x-x_i)^{\nu_i},
\]
missä $q$ on polynomi muotoa $q(x)=\sum_{r=l}^{\nu_j-1} c_r (x-x_j)^r$. Päättele edelleen, että
interpolaatio-ongelman \eqref{yleiset interpolaatioehdot} ratkaisu yleisessä tapauksessa on
\[
p(x) = \sum_{i=1}^m \sum_{k=0}^{\nu_i-1} f^{(k)}(x_i) L_{i,k}(x).
\] 

\end{enumerate}