\section{Funktion raja-arvo} \label{funktion raja-arvo} \alku
 
Jatkuvuuteen liittyy läheisesti käsite \kor{funktion raja-arvo}. Raja-arvo kertoo funktion 
arvojen käyttäytymisestä lähestyttäessä jotakin pistettä $a\in\R$ funktion määrittelyjoukosta 
käsin. Pisteen $a$ ei tarvitse olla määrittelyjoukossa, riittää että sitä voidaan lähestyä 
mielivaltaisen lähelle. Tyypillinen sovellustilanne onkin juuri tällainen. Toisaalta, jos
piste on määrittelyjoukossa, ei raja-arvo silti riipu funktion arvosta tässä pisteessä.

Funktion raja-arvoja on kahdentyyppisiä, varsinaisia eli raja-arvoja ilman lisämääreitä, ja 
\index{toispuolinen raja-arvo}%
\kor{toispuolisia} raja-arvoja (engl. one-sided limit). Määritelmät ovat seuraavat.
\begin{Def} \vahv{(Funktion raja-arvo)}\ \label{funktion raja-arvon määritelmä}
\index{funktion raja-arvo|emph} \index{raja-arvo!b@reaalifunktion|emph}
\index{vasemmanpuoleinen raja-arvo|emph} \index{oikeanpuoleinen raja-arvo|emph}
Funktiolla $f:\DF_f\kohti\R$, $\DF_f\subset\R$, on \kor{pisteessä} $a$ (tai $a$:ssa) 
\kor{raja-arvo} $A\in\R$, jos jokaiselle reaalilukujonolle $\seq{x_n}$ pätee
\[
\bigl(\,x_n\in\DF_f\ \ja\ x_n\neq a\,\bigr)\,\forall n\,\ \ja\,\ x_n\kohti a\,\ 
                          \impl\,\ f(x_n)\kohti A,
\]
ja oletus on voimassa jollekin jonolle $\{x_n\}$. Funktiolla on pisteessä $a$ 
\kor{vasemmanpuoleinen raja-arvo} $A_-\in\R$, jos jokaiselle reaalilukujonolle $\seq{x_n}$
pätee
\[
\bigl(\,x_n\in\DF_f\ \ja\ x_n< a\,\bigr)\,\forall n\,\ \ja\,\ x_n\kohti a\,\ 
                          \impl\,\ f(x_n)\kohti A_-,
\]
ja oletus on voimassa jollekin jonolle $\{x_n\}$. Funktiolla on pisteessä $a$
\kor{oikeanpuoleinen raja-arvo} $A_+\in\R$, jos jokaiselle reaalilukujonolle $\seq{x_n}$
pätee
\[
\bigl(\,x_n\in\DF_f\ \ja\ x_n> a\,\bigr)\,\forall n\,\ \ja\,\ x_n\kohti a\,\
                          \impl\,\ f(x_n)\kohti A_+,
\]
ja oletus on voimassa jollekin jonolle $\{x_n\}$.
\end{Def}
Raja-arvon määritelmän lisäehto 'oletus voimassa jollekin jonolle $\seq{x_n}$' tarkoittaa,
että lähestyminen ($x_n \kohti a$) oletetulla tavalla on mahdollista, ts.\ että $a$ \pain{ei}
\pain{ole} \pain{eristett}y p\pain{iste} joukossa $\DF_f \cup \{a\}$
(ks.\ Harj.teht.\,\ref{jatkuvuuden käsite}:\ref{H-V-1: eristetty piste}). Toispuolisissa
raja-arvoissa lisäehto tarkoittaa vastaavasti, että $a$ ei ole eristetty piste joukossa
$[\DF_f\cap(-\infty,a)]\cup\{a\}$ (vasemmanpuoleinen raja-arvo) tai joukossa
$[\DF_f\cap(a,\infty)]\cup\{a\}$ (oikeanpuoleinen raja-arvo). Eristetyssä pisteessä
raja-arvoa ei siis määritellä, olipa piste määrittelyjoukossa $\DF_f$ tai ei.

Määritelmässä esiintyville raja-arvoille käytetään merkintöjä
\[
A = \lim_{x\kohti a} f(x), \quad A_\pm = \lim_{x\kohti a^\pm} f(x),
\]
tai merkitään (vrt.\ lukujonon raja-arvomerkinnät)
\[
f(x) \kohti A, \quad \text{kun}\ x \kohti a, \qquad 
f(x) \kohti A_\pm, \quad \text{kun}\ x \kohti a^\pm.
\]
Toispuolisille raja-arvoille kätevä merkintätapa on myös
\[
A_+ = f(a^+), \quad A_- = f(a^-).
\]
\begin{Exa} \label{helppo raja-arvo} Määritä $\,\lim_{x \kohti 1}f(x)$, kun 
$f(x)=(x-1)/(x^2+x-2)$.
\end{Exa}
\ratk Määrittelyjoukko on $\DF_f=\{x\in\R \mid x \neq 1\,\ja\,x \neq -2\}$ ja
$f(x)=1/(x+2)\ \forall x\in\DF_f$. Näin ollen jos $x_n\in\DF_f\ \forall n$
($\,\impl\, x_n \neq 1\ \forall n$) ja $x_n \kohti 1$, niin $f(x_n)=1/(x_n+2) \kohti 1/3$.
Siis $\,\lim_{x \kohti 1}f(x)=1/3$ (Määritelmä \ref{funktion raja-arvon määritelmä}). \loppu

Kuten lukujonon, myös funktion raja-arvomerkinnöissä, voi olla $A$:n tai $A_\pm$:n
tilalla $\infty/-\infty$, jolloin tarkoitetaan Määritelmän 
\ref{funktion raja-arvon määritelmä} mukaisesti, että $f(x)$ kasvaa/vähenee rajatta,
kun $x \kohti a$ tai $x \kohti a^\pm$. 
\jatko \begin{Exa} (jatko) Esimerkin funktiolle pätee
\[
\lim_{x \kohti\,-2^-} f(x) = -\infty, \quad \lim_{x \kohti\,-2^+} f(x) = \infty. \loppu
\]
\end{Exa}

Sikäli kuin raja-arvo $\lim_{x \kohti a} f(x)$ on olemassa, yhtyvät myös toispuoliset 
raja-arvot (sikäli kuin määriteltävissä) tähän. Toisaalta on mahdollista, että molemmat 
toispuoliset raja-arvot ovat olemassa pisteessä $a$ mutta eri suuret, jolloin raja-arvoa
pisteessä $a$ ei ole. Mitä tulee jatkuvuuden ja raja-arvon väliseen yhteyteen, nähdään
Määritelmistä \ref{funktion jatkuvuus} ja \ref{funktion raja-arvon määritelmä}, että sikäli
kuin raja-arvo $\lim_{x \kohti a} f(x)$ on määriteltävissä (eli $a$ ei ole eristetty piste
joukossa $\DF_f \cup \{a\}$), pätee
\[ \boxed{ \begin{aligned}
\ykehys\quad f \text{ jatkuva pisteessä } a\in\DF_f \ \ekv \ 
       \lim_{x\kohti a} f(x) = f(a) \quad \\ (\text{$a$ ei eristetty piste}). \quad\akehys
\end{aligned} } \]
Jos $f$:llä on toispuolinen raja-arvo $f(a^+)$ tai $f(a^-)$ ja pätee joko $f(a)=f(a+)$ tai 
\index{oikealta jatkuva} \index{vasemmalta jatkuva}
\index{jatkuvuus (yhden muuttujan)!ea@oikealta, vasemmalta}%
$f(a)=f(a^-)$, niin sanotaan vastaavasti, että $f$ on \kor{oikealta jatkuva} tai 
\kor{vasemmalta jatkuva} pisteessä $a$. Jos $a\in\DF_f$ ja pistettä $a$ voidaan lähestyä 
molemmista suunnista $\DF_f$:stä käsin, niin on ilmeistä, että $f$ on jatkuva $a$:ssa
täsmälleen kun $f$ on sekä vasemmalta että oikealta jatkuva $a$:ssa. Edelleen nähdään, että
$f$ on jatkuva suljetulla välillä $[a,b]$ (Määritelmä \ref{jatkuvuus välillä}) täsmälleen kun
$f$ on jatkuva avoimella välillä $(a,b)$ ja lisäksi oikealta jatkuva $a$:ssa ja vasemmalta
jatkuva $b$:ssä.

Kuten jatkuvuus, myös raja-arvo on määriteltävissä vaihtoehtoisella 
$(\eps,\delta)$-kritee\-rillä, vrt.\ Määritelmä \ref{vaihtoehtoinen jatkuvuus}. Vaihtoehtoinen
määritelmä muotoillaan tässä lauseena raja-arvolle $\lim_{x\kohti a} f(x)= A\in\R$. Todistus
sivuutetaan, sillä se on hyvin samanlainen kuin Lauseen \ref{jatkuvuuskriteerien yhtäpitävyys}
todistus edellisessä luvussa. (Toispuolisille raja-arvoille on muotoiltavissa vastaava tulos,
samoin tapauksille $\lim_{x\kohti a} f(x) = \pm\infty$.)
\begin{*Lause} \vahv{(Raja-arvon $(\eps,\delta)$-kriteeri)} \label{approksimaatiolause}
\index{funktion raja-arvo|emph} \index{raja-arvo!b@reaalifunktion|emph}
Funktiolla $f:\DF_f\kohti\R$, $\DF_f\subset\R$, on raja-arvo $\lim_{x\kohti a} f(x)=A\in\R$
täsmälleen kun $\DF_f\cap[(a-\delta,a)\cup(a,a+\delta)]\neq\emptyset$ $\forall \delta>0$ ja
jokaisella $\eps>0$ on olemassa $\delta>0$ siten, että jokaisella $x\in\R$ pätee
\[
x\in\DF_f \ \ja \ 0<\abs{x-a}<\delta \ \impl \ \abs{f(x)-A}<\eps.
\]
\end{*Lause}
Lukujonoista tiedetään, että suppeneva lukujono on rajoitettu 
(Lause \ref{suppeneva jono on rajoitettu}). Funktiolle, jolla on raja-arvo, pätee Lauseen
\ref{approksimaatiolause} perusteella vastaava tulos:
\begin{Lause} \label{raja-arvo ja rajoitettu funktio} \index{rajoitettu!c@funktio|emph} 
Jos $\lim_{x \kohti a} f(x) = A\in\R$, niin $f$ on \kor{rajoitettu} jossakin pisteen
ympäristössä, ts.\ $\exists \delta>0$ ja $C\in\R_+$ siten, että
\[ 
|f(x)| \le C \quad \forall\ x\in(a-\delta,a+\delta)\cap\DF_f. 
\]
\end{Lause}
\tod Valitaan $\eps=1$ ja vastaava $\delta>0$ niin, että Lauseen \ref{approksimaatiolause}
ehto on voimassa. Tällöin seuraa kolmioepäyhtälöstä, että väittämä on tosi, kun
$C=\abs{A}+1$. \loppu

\subsection{Funktion approksimointi raja-arvolla}
%\index{funktion approksimointi!a@raja-arvolla|vahv}

Jos funktiosta tunnetaan raja-arvo $A = \lim_{x\kohti a} f(x)$, niin Lauseen 
\ref{approksimaatiolause} mukaan voidaan raja-arvopisteen lähellä käyttää approksimaatiota 
$f(x) \approx A$. Lause ei tosin anna (eikä tehdyin oletuksin voikaan antaa) mitään 
kvantitatiivista tietoa approksimaation tarkkuudesta, koska $\delta$:n riippuvuutta
$\eps$:sta ei tunneta. Tyypillisissä esimerkkitapauksissa funktiosta $f$ kuitenkin yleensä
tiedetään lauseessa oletettua enemmän, jolloin approksimaatiolle on ehkä mahdollista johtaa 
kvantitatiivinen virhearvio tämän lisätiedon perusteella. Likimääräisessä funktioevaluaatiossa
raja-arvotieto voi auttaa etenkin silloin, kun funktion laskukaava suoraan käytettynä on
altis numeerisille pyöristysvirheille raja-arvopisteen lähellä.
\begin{Exa} \label{raja-arvolla approksimointi} Funktio $f(x) = (1-\cos x)/x^2$ on määritelty,
kun $x \neq 0$. Myöhemmin (Luku \ref{kaarenpituus}) osoitetaan, että 
$\lim_{x \kohti 0} f(x) = 1/2$. Tähän tulokseen perustuva approksimaatio
\[
f(0.000000003) \approx \lim_{x \kohti 0} f(x) = 0.5
\]
on huomattavasti turvallisempi kuin $f$:n laskukaavan suora käyttö, sillä lasku\-operaatiossa
$x \map 1-\cos x$ tapahtuu huomattava merkitsevien numeroiden kato, kun $\abs{x}$ on pieni 
(vrt.\ Luku \ref{desimaaliluvut}). Paitsi turvallinen, raja-arvoon perustuva
\mbox{approksimaatio} on tässä tapauksessa myös hyvin tarkka: virhe on alle $10^{-18}$. \loppu
\end{Exa}

\subsection{Raja-arvojen yhdistely}
\index{funktion raja-arvo!a@raja-arvojen yhdistely|vahv}

Raja-arvojen laskemista helpottavat seuraavat lauseet, jotka ovat Lauseiden 
\ref{jatkuvuuden yhdistelysäännöt} ja \ref{yhdistetyn funktion jatkuvuus} vastineita. 
Todistukset ovat Määritelmän \ref{funktion raja-arvon määritelmä} perusteella suoraviivaisia
(Harj.teht.\,\ref{H-V-2: todistuksia}; ks.\ myös 
Harj.teht.\,\ref{jatkuvuuden käsite}:\ref{H-V-1: eristetty piste}). Lauseet pätevät ilmeisin
muutoksin myös toispuolisille raja-arvoille.
\begin{Lause} \label{funktion raja-arvojen yhdistelysäännöt}
Jos $\lim_{x\kohti a} f(x) = A\in\R$, niin
\begin{align*}
\lim_{x\kohti a} (\lambda f)(x) &= \lambda A \quad \forall \lambda\in\R. \\
\intertext{Jos lisäksi $\lim_{x\kohti a} g(x)=B\in\R$ ja $a$ ei ole joukon
$(\DF_f\cap\DF_g)\cup\{a\}$ eristetty piste, niin}
\lim_{x\kohti a} (f+g)(x)       &=A+B, \\  
\lim_{x\kohti a} (fg)(x)        &=AB.
\intertext{Jos lisäksi $B\neq 0$ ja $a$ ei ole joukon
$\DF_{f/g}\cup\{a\}=\{x\in\DF_f\cap\DF_g \mid g(x) \neq 0\}\cup\{a\}$ eristetty piste, niin}
\lim_{x\kohti a} (f/g)(x)       &=A/B.
\end{align*}
\end{Lause}
\begin{Lause} \label{yhdistetyn funktion raja-arvo} Jos $\lim_{x\kohti a} g(x) = A\in\R$, $f$
on jatkuva pisteessä $x=A$ ja $a$ ei ole joukon $\DF_{f \circ g}\cup\{a\}$ eristetty piste,
niin $\lim_{x\kohti a} (f\circ g)(x) = f(A)$. 
\end{Lause}
\jatko \begin{Exa} (jatko) Koska $g(x) = \sqrt{\abs{x}}$ on jatkuva $\R$:ssä 
(Propositio \ref{potenssifunktion jatkuvuus}), niin esimerkin raja-arvotuloksesta ja Lauseesta
\ref{yhdistetyn funktion raja-arvo} seuraa
\[
\lim_{x \kohti 0} \frac{\sqrt{1-\cos x}}{\abs{x}} 
      = \lim_{x \kohti 0} \sqrt{\frac{1-\cos x}{x^2}} = \frac{1}{\sqrt{2}}\,. \loppu 
\]
\end{Exa}
\begin{Exa} Funktioille $f_0(x)=1$ ja $f_1(x)=x$ ovat voimassa ilmeiset raja-arvotulokset
\[
\lim_{x \kohti a} f_0(x) = 1, \quad \lim_{x \kohti a} f_1(x) = a, \quad a \in \R.
\]
Näiden ja Lauseen \ref{funktion raja-arvojen yhdistelysäännöt} perusteella seuraa
\[
\lim_{x \kohti a} \frac{(x+1)^2}{x+2} = \frac{(a+1)^2}{a+2}\,, \quad \text{kun}\ a \neq -2.
\]
Tulos on selvä myös Lauseen \ref{jatkuvuuden yhdistelysäännöt} perusteella, sillä tämän mukaan
rationaalifunktio $f$ on jatkuva koko määrittelyjoukossaan, jolloin 
$\lim_{x\kohti a} f(x) = f(a)$, kun $a\in\DF_f$. \loppu
\end{Exa}

\subsection{Raja-arvon laskeminen sijoituksella}
\index{funktion raja-arvo!b@laskeminen sijoituksella|vahv}

Raja-arvon $\,\lim_{x \kohti a} f(x)$ laskemista on usein mahdollista helpottaa tekemällä
\kor{sijoitus} eli \kor{muuttujan vaihto} $\,x=u(t)$. Muuttujaa vaihdettaessa tukeudutaan
seuraavaan lauseeseen, joka on helposti muotoiltavissa myös toispuolista raja-arvoa
koskevaksi (Harj.teht.\,\ref{H-V-2: todistuksia}c).
\begin{Lause} \label{raja-arvo sijoituksella} Olkoon $u:\ U\kohti[a-\delta,a+\delta]$ 
($\delta>0$) jatkuva bijektio, missä $U$ on suljettu väli. Tällöin jos $u(\alpha)=a$, niin
pätee
\[
\lim_{x \kohti a} f(x) = \lim_{t\kohti\alpha} f(u(t)) = \lim_{t\kohti\alpha} F(t)=A,
\]
sikäli kuin raja-arvo oikealla on olemassa ($A\in\R$ tai $A=\pm\infty$).
\end{Lause}
\tod Lauseen \ref{käänteisfunktion jatkuvuus} perusteella myös käänteisfunktio
$u^{-1}:\ [a-\delta,a+\delta] \kohti U$ on jatkuva bijektio. Olkoon nyt $x_n\in\DF_f$,
$x_n \neq a\ \forall n$ ja $x_n \kohti a$. Tällöin jostakin indeksistä eteenpäin on 
$\abs{x_n-a}<\delta$, jolloin voidaan kirjoittaa $\,x_n=u(t_n),\ t_n \in U$. Koska
$u:\ U \kohti [a-\delta,a+\delta]$ on injektio, niin $x_n \neq a\ \impl\ t_n\neq\alpha$.
Koska $u^{-1}$ on jatkuva pisteessä $a$, niin 
$x_n \kohti a\ \impl\ t_n=u^{-1}(x_n) \kohti u^{-1}(a)=\alpha$. Tällöin oletuksen
$\lim_{t\kohti\alpha} F(t)=A$ perusteella pätee $f(x_n)=f(u(t_n))=F(t_n) \kohti A$. On näytetty,
että $\,\lim_{x \kohti a} f(x)=A$ (Määritelmä \ref{funktion raja-arvon määritelmä},
kun $A\in\R$; päättely toimii myös, kun $A=\pm\infty$). \loppu

\begin{Exa} \label{raja-arvo muuttujan vaihdolla} Määritä raja-arvo
$\D \,A=\lim_{x \kohti 81} \frac{\sqrt{x}-9}{\sqrt[4]{x}-3}\,$.
\end{Exa}
\ratk Tässä sopiva sijoitus on $\,\sqrt[4]{x}=t$, jolloin on $\,x=t^4=u(t)$, ja arvoa
$x=81\,(=a)$ vastaa $t=3\,(=\alpha)$. Koska
\[
F(t)=f(t^4)=\frac{t^2-9}{t-3}=t+3=G(t), \quad \text{kun}\ t \neq 3,
\]
niin funktion $G$ jatkuvuuden perusteella on $\lim_{t \kohti 3} F(t)=G(3)=6$. Lauseen
\ref{raja-arvo sijoituksella} oletukset ovat voimassa ($\delta \le 3$), joten kysytty
raja-arvo on $A=6$. \loppu
\begin{Exa} Sijoituksella $x=2t$ saadaan (vrt.\ Luku \ref{trigonometriset funktiot})
\[
\lim_{x \kohti 0^+} \frac{\sin\frac{x}{2}}{\sqrt{1-\cos x}}
  = \lim_{t \kohti 0^+} \frac{\sin t}{\sqrt{1-\cos 2t}}
  = \lim_{t \kohti 0^+} \frac{\sin t}{\sqrt{2\sin^2 t}}
  = \lim_{t \kohti 0^+} \frac{1}{\sqrt{2}} 
  = \frac{1}{\sqrt{2}}\,.
\]
Tässä loppusievennys perustui päättelyyn
$\,t\in(0,\pi]\,\impl\,\sqrt{\sin^2 t}=\sin t$. \loppu
\end{Exa}

Esimerkeissä suoritettiin raja-arvolaskuille hyvin tyypillinen nelivaiheinen lasku\-operaatio
\[
\lim_{x \kohti a} f(x) = \lim_{t\kohti\alpha} f(u(t)) = \lim_{t\kohti\alpha} F(t)
                       = \lim_{t\kohti\alpha} G(t) = G(\alpha).
\]
Tässä
\begin{enumerate}
\item Sijoitetaan $x=u(t)$ [\,tai $v(x)=t\,\impl\,x=u(t)$\,] ja lasketaan $\alpha=u^{-1}(a)$.
\item Sievennetään $f(u(t))$ lausekkeeksi $F(t)$.
\item Pelkistetään $F(t)$ lausekkeeksi $G(t)$, kun $t\neq\alpha$.
\item Lasketaan raja-arvo vedoten $G$:n j\pain{atkuvuuteen} pisteessä $t=\alpha$.
\end{enumerate}
Sikäli kuin muuttujaa ei vaihdeta, lasku supistuu vaiheiksi 3--4 ($F=f,\,t=x$), kuten
Esimerkissä \ref{helppo raja-arvo} edellä.

\subsection{Funktion jatkaminen}
\index{jatkaminen (funktion)|vahv}

Jos $a\in\R$ ei ole funktion $f$ määrittelyjoukossa, mutta on olemassa (aito) raja-arvo
$\lim_{x\kohti a} f(x)=A\in\R$, niin on luonnollista sisällyttää $a$ määrittelyjoukkoon
asettamalla $f(a)=A$. Näin menetellen $f$:stä tulee Määritelmän \ref{funktion jatkuvuus}
mukaisesti jatkuva pisteessä $a$. Funktion määrittelyjoukon laajentamista tällä tavoin
kutsutaan \kor{funktion jatkamiseksi}.
\begin{Exa} Esimerkissä \ref{helppo raja-arvo} todettiin, että funktiolla
$f(x)=(x-1)/(x^2+x-2)$ on raja-arvo $\lim_{x \kohti 1}f(x)=1/3$. Kun piste $x=1$ sisällytetään
$f$:n määrittelyjoukkoon asettamalla $f(1)=1/3$, niin $f$ tulee jatketuksi funktioksi
$f(x)=1/(x+2)$. Enempää ei määrittelyjoukkoa voi tässä laajentaa jatkamalla, koska
$f$:llä ei ole reaalista (tai yleisempääkään) raja-arvoa, kun $x \kohti -2$. \loppu
\end{Exa}
\begin{Exa} Esimerkin \ref{raja-arvolla approksimointi} raja-arvotiedon
(ja Esimerkin \ref{jatkuvuuden käsite}:\ref{trig yhdistely} tiedon) perusteella funktio
\[ 
f(x) = \begin{cases} \,\dfrac{1-\cos x}{x^2}\,, &\text{kun}\ x \neq 0, \\[2mm]
                     \,\dfrac{1}{2}\,,          &\text{kun}\ x=0
       \end{cases}
\]
on jatkuva $\R$:ssä. Funktion arvo $0$:ssa on määrätty jatkamalla. \loppu
\end{Exa}             

\subsection{Paloittainen jatkuvuus}
\index{jatkuvuus (yhden muuttujan)!f@paloittainen|vahv} \index{paloittainen!b@jatkuvuus|vahv}

Funktion toispuoliset raja-arvot tulevat käyttöön erityisesti sellaisissa sovellustilanteissa,
joissa (usein fysikaalista perua oleva) funktio on jatkuva muualla paitsi erillisissä 
\index{hyppyepäjatkuvuus}%
pisteissä, joissa sillä on yksinkertainen nk.\ \kor{hyppyepäjatkuvuus}
(engl.\ jump discontinuity). Asetetaan tällaisia käytännön tarpeita silmällä pitäen
\begin{Def}
Funktio $f:\DF_f\kohti\R$, $\DF_f\subset\R$, on välillä $[a,b]$ \kor{paloittain jatkuva} 
(engl.\ piecewise continuous), jos $\exists$ pisteet $c_k$, $k=0\ldots n$, $n\in\N$ siten,
että
\[
a=c_0<c_1<\ldots<c_n=b
\]
ja pätee
\begin{itemize}
\item[(i)] $(c_{k-1},c_k)\subset\DF_f$ ja $f$ on jatkuva välillä
$(c_{k-1},c_k), \quad k=1 \ldots n$,
\item[(ii)] $\exists$ toispuoliset raja-arvot
\begin{align*}
\lim_{x\kohti c_k^+} f(x) &= f(c_k^+), \quad k=0 \ldots n-1, \\ 
\lim_{x\kohti c_k^-} f(x) &= f(c_k^-), \quad k=1 \ldots n.
\end{align*}
\end{itemize}
\end{Def}
Huomattakoon, että raja-arvot $f(c_k^\pm)$ eivät raja-arvon määritelmän mukaisesti riipu $f$:n
mahdollisista arvoista pisteissä $c_k$. Jos molemmat toispuoliset raja-arvot ovat olemassa,
mutta eivät yhdy, on kyseessä hyppyepäjatkuvuus (hyppy= $f(c_k^+)-f(c_k^-)$). Jos yhtyvät, on
funktio pisteessä $c_k$ joko jatkuva tai määriteltävissä jatkuvaksi raja-arvon avulla
(jatkamismenettely).
\begin{Exa} \label{sahafunktio} Olkoon $[x]=$ suurin kokonaisluku, jolle pätee $[x] \le x$.
Funktio $f(x)=x-[x]$, eli
\[
f(x)=x-k,\quad \text{kun } x\in[k,k+1), \quad k\in\Z,
\]
on jokaisella välillä $[a,b]\subset\R$ paloittain jatkuva. Pisteissä $k\in\Z$ funktio
voitaisiin määritellä miten tahansa (tai jättää määrittelemättä) ilman, että sillä 
vaikutettaisiin toispuolisiin raja-arvoihin $f(k^+)=0$ ja $f(k^-)=1$. \loppu
\begin{figure}[H]
\setlength{\unitlength}{1cm}
\begin{center}
\begin{picture}(14,3)(-1,-1)
\put(-1,0){\vector(1,0){14}} \put(12.8,-0.4){$x$}
\put(0,-1){\vector(0,1){3}} \put(0.2,1.8){$y$}
\multiput(0,0)(1,0){12}{\drawline(0,0)(0,-0.1)} \put(0.9,-0.5){$1$} \put(1.9,-0.5){$2$}
\drawline(-0.1,1)(0,1) \put(-0.4,0.9){$1$}
\multiput(0,0)(1,0){12}{\drawline(0,0)(1,1)}
\end{picture}
\end{center}
\end{figure}
\end{Exa}
\index{zza@\sov!Kiihdytys}%
\begin{Exa} \vahv{Kiihdytys}. Auto on paikallaan moottorin käydessä. Ajan hetkellä $t=0$
polkaistaan kaasupoljin pohjaan. Kiihtyvyys $f(t)=$ ?
\end{Exa}
\ratk Idealisoidun matemaattisen mallin mukaan on
\[
f(t)=
\begin{cases}
\,0,              &\text{kun}\ t<0, \\
\,a=\text{vakio}, &\text{kun}\ t\ge 0.
\end{cases}
\]
Todellisuudessa hyppyepäjatkuvuutta ei hetkellä $t=0$ esiinny. Matemaattinen malli (jos hyvä)
riippuukin esimerkissä olennaisesti siitä, millaisessa \pain{aikaskaalassa} tapahtumia
tarkastellaan. Eri skaaloissa funktio $f$ voi näyttää hyvin erilaiselta. \loppu
\begin{figure}[H]
\setlength{\unitlength}{1cm}
\begin{center}
\begin{picture}(14,4)(0,0)
\multiput(0,0)(8,0){2}{
\put(0,1){\vector(1,0){6}} \put(5.8,0.6){$t$}
\put(1,0){\vector(0,1){4}} \put(1.2,3.8){$y$}
}
\multiput(2,1)(1,0){4}{\drawline(0,0)(0,-0.1)} \put(1.7,0.5){$1$ s}
\drawline(0.9,3)(1,3) \put(0.6,2.9){$a$}
\multiput(10,1)(1,0){4}{\drawline(0,0)(0,-0.1)} \put(9.5,0.5){$1$ ms}
\drawline(8.9,3)(9,3) \put(8.6,2.9){$a$}
\dashline{0,2}(9,2.97)(14,2.97)
\spline(1,1)(1.1,3)(1.3,2.9)(1.4,3)(2,3)(3,3)(4,3)(5,3)
\spline(9,1)(9.5,2)(9.7,1.5)(10,2.3)(10.3,2.2)(10.6,2.5)(11,2.7)(11.5,2.8)(12,2.9)(13,2.95)
\put(3,3.3){$y=f(t)$} \put(11,3.3){$y=f(t)$}
\end{picture}
\end{center}
\end{figure}

\subsection{Raja-arvot $\displaystyle{\lim_{x\kohti\pm\infty}f(x)}$}
\index{funktion raja-arvo|vahv} \index{raja-arvo!b@reaalifunktion|vahv}

Raja-arvo $\lim_{x\kohti\infty} f(x)$ määritellään samoin kuin edellä, eli sijoittamalla
yleiseen määritelmään $a=\infty$. Tällöin $x_n\kohti\infty$ tarkoittaa, että $\{x_n\}$ kasvaa
rajatta. Raja-arvo $\lim_{x\kohti -\infty} f(x)$ määritellään vastaavasti. Lause 
\ref{funktion raja-arvojen yhdistelysäännöt} on pätevä myös raja-arvoille 
$\lim_{x\kohti\pm\infty} f(x)$, samoin Lause \ref{yhdistetyn funktion raja-arvo}, kun 
määrittelyjoukkoa $\DF_{f\circ g}$ koskeva oletus muutetaan joko ehdoksi
$\DF_{f\circ g}\cap(M,\infty)\neq\emptyset\ \forall M\in\R$ (jos $x \kohti \infty$) tai ehdoksi 
$\DF_{f\circ g}\cap(-\infty,M)\neq\emptyset\ \forall M\in\R$ (jos $x \kohti- \infty$).

Raja-arvoja $\lim_{x\kohti\infty} f(x)$ määrättäessä usein kätevä on sijoitus $x=t^{-1}$, eli
siirtyminen tarkastelemaan funktiota $F(t)=f(1/t)$. Koska 
$x\kohti\pm\infty\ \ekv\ t\kohti 0^\pm$, niin raja-arvojen määritelmistä nähdään, että pätee
(vrt.\ Lause \ref{raja-arvo sijoituksella})
\[
\lim_{x\kohti\pm\infty} f(x) = \lim_{t\kohti 0^\pm} F(t) = F(0^\pm), \quad F(t)=f(1/t).
\]
\begin{Exa} (jatko) Muuttujan vaihdolla $x=t^{-1}$ päätellään Propostitioon 
\ref{potenssifunktion jatkuvuus} ja Lauseisiin \ref{jatkuvuuden yhdistelysäännöt} ja 
\ref{yhdistetyn funktion jatkuvuus} vedoten:
\[
\lim_{x\kohti\infty} \frac{\sqrt[3]{x}+\sqrt[4]{x}}{\sqrt[3]{8x+3}} 
                 = \lim_{t\kohti 0^+} \frac{1+\sqrt[12]{t}}{\sqrt[3]{8+3t}}
                 = \frac{1+0}{\sqrt[3]{8+0}} = \frac{1}{2}\,. \loppu
\]
\end{Exa}
Lukujonojen teoriasta tiedetään, että monotoninen ja rajoitettu lukujono suppenee.
Raja-arvoja $\,\lim_{x\kohti\infty}f(x)\,$ koskeva vastaava väittämä on
\begin{Lause} \label{monotonisen funktion raja-arvo} Jos funktio $f$ on monotoninen ja
rajoitettu välillä $[a,\infty)$, niin on olemassa raja-arvo $\lim_{x\kohti\infty}f(x)\in\R$.
\end{Lause}
\tod Olkoon $f$ esim.\ kasvava välillä $[a,\infty)$. Tällöin jos merkitään $y_k=f(k)$,
$k\in\N$ ja $k \ge a$, niin $\seq{y_k}$ on kasvava ja rajoitettu lukujono, joten
$y_k \kohti y\in\R$. Olkoon nyt $\seq{x_n}$ mikä tahansa lukujono, jolle pätee
$x_n\ge a\ \forall n$ ja $x_n\kohti\infty$. Tällöin jos $x_n\in[k,k+1)$, $k\in\N$, niin
$y_k \le f(x_n) \le y_{k+1}$, koska $f$ on kasvava. Tässä $k\kohti\infty$ kun $n\kohti\infty$
(koska $x_n\kohti\infty$), joten päätellään (Propositio \ref{jonotuloksia}[V2]), että
$\lim_nf(x_n)=\lim_ky_k=y$. Koska tämä pätee jokaiselle mainitut ehdot täyttävälle jonolle
$\seq{x_n}$, niin $\,\lim_{x\kohti\infty}f(x)=y$. Jos $f$ on vähenevä välillä $[a,\infty)$, niin
päättely on vastaava. \loppu

\subsection{Asymptootit}
\index{asymptootti|vahv}

Sanotaan, että funktio $g(x)$ on funktion $f(x)$ \kor{asymptootti}, jos 
\[
\lim_{x\kohti \infty} [f(x)-g(x)]=0, \ \text{ tai }\ \lim_{x\kohti -\infty} [f(x)-g(x)]=0.
\]
Asymptootin ideana on approksimoida funktiota jollakin (mieluiten) yksinkertaisella
funktiolla $g(x)$, kun joko $x$ on hyvin suuri (merkintä $x \gg 1$) tai $-x$ on hyvin suuri 
(merkintä $x \ll -1$)\,:
\[
f(x)\approx g(x), \quad \text{kun}\ x\gg 1\ 
                        \text{tai}\ x\ll -1.\footnote[2]{Asymptootin perinteisempi ja
rajoitetumpi geometrinen määritelmä on tasokäyrään liittyen \pain{suora}, jota 'käyrä lähestyy
äärettömyydessä'. Esimerkiksi käyrän $S:\,y=x^2/(x+1)$ asymptootteja ovat tämän tulkinnan
mukaan suorat $y=x-1$ ja $x=-1$. Jälkimmäinen on nk.\ pystysuora asymptootti, jolla ei ole
funktiovastinetta.}
\]
Kyse on Lauseesta \ref{approksimaatiolause}, joka on yleistettävissä myös raja-arvoja 
$\lim_{x\kohti\pm\infty} f(x)$ koskevaksi: Esimerkiksi jos $\DF_f\cap\DF_g\supset[a,\infty)$ ja
$\lim_{x\kohti\infty} [f(x)-g(x)]=0$, niin jokaisella $\eps>0$ on olemassa $M\in[a,\infty)$
siten, että $\,\abs{f(x)-g(x)}<\eps\ \forall x>M$.
\begin{Exa} Funktion $f(x)=\sqrt{x^2+4x}\,$ eräs asymptootti, kun $|x| \gg 1$, on $g(x)=|x+2|$,
sillä rajoilla $x\kohti\pm\infty\,$ on
\begin{align*}
f(x)-g(x) \,=\, \sqrt{x^2+4x}-|x+2| 
          &\,=\, \frac{x^2+4x-(x+2)^2}{\sqrt{x^2+4x}+|x+2|} \\
          &\,=\, -\frac{4}{\sqrt{x^2+4x}+|x+2|}\ \kohti\ 0. \loppu
\end{align*}
\end{Exa}

\Harj
\begin{enumerate}

\item \label{H-V-2: todistuksia}
a) Todista Lause \ref{funktion raja-arvojen yhdistelysäännöt}. \
b) Todista Lause \ref{yhdistetyn funktion raja-arvo}. \
c) Muotoile ja todista Lauseen \ref{raja-arvo sijoituksella} vastine koskien toispuolista
raja-arvoa $\lim_{x \kohti a^+} f(x)$.

\item
Funktiosta $f$ tiedetään, että $[-1,1]\subset\DF_f$, $f(0)=0$ ja 
$\,\sqrt{2-x^2} \le f(x) \le \sqrt{2+9\abs{x}}$, kun $0 < \abs{x} \le 1$. Näytä, että
$\lim_{x \kohti 0}f(x)=\sqrt{2}$.

\item
a) Funktiosta $f$ tiedetään, että $\,\lim_{x \kohti 0^+}f(x)=A$. Näytä, että jos $f$ on 
parillinen, niin $\,\lim_{x \kohti 0^-}f(x)=A$, ja jos pariton, niin
$\,\lim_{x \kohti 0^-}f(x)=-A$. \newline
b) Funktiosta $f$ tiedetään, että $\,\lim_{x \kohti 0^+}f(x)=A$ ja
$\,\lim_{x \kohti 0^-}f(x)=B$. Laske raja-arvot $\lim_{x \kohti 0^+} f(x^2-x)$ ja
$\lim_{x \kohti 0^-} f(x^2+x^3)$. 

\item
Määritä seuraavat raja-arvot, joko reaalilukuna tai muodossa $\pm\infty$, tai totea 
vaihtoehtoisesti, ettei raja-arvoa ole. Vaihda tarvittaessa muuttujaa.
\begingroup
\allowdisplaybreaks

\begin{align*}
&\text{a)}\ \lim_{x \kohti 4} (x^2-4x+1) \qquad
 \text{b)}\ \lim_{x \kohti 3} \frac{x+3}{x+6} \qquad
 \text{c)}\ \lim_{x\kohti\pi} \frac{(x+\pi)^2}{\pi x} \\[1mm]
&\text{d)}\ \lim_{x \kohti -2} \frac{x^2+2x}{x^2-4} \qquad
 \text{e)}\ \lim_{x \kohti 2} \left(\frac{1}{x-2}-\frac{4}{x^2-4}\right) \qquad
 \text{f)}\ \lim_{x \kohti \frac{\pi}{2}} \frac{\sin(2x+\pi)}{\cot x} \\
&\text{g)}\ \lim_{t \kohti 0} \frac{(t+1)^2-(t-1)^2}{t} \qquad
 \text{h)}\ \lim_{s \kohti 0} \frac{s^2+3s}{(s+2)^2-(s-2)^2} \\
&\text{i)}\ \lim_{x \kohti -3} \abs{x-3} \qquad
 \text{j)}\ \lim_{x \kohti 2} \frac{\abs{x^2-4x+3}}{x^2+2x-3} \qquad
 \text{k)}\ \lim_{x \kohti 1} \frac{\abs{x^2-4x+3}}{x^2+2x-3} \\
&\text{l)}\ \lim_{x \kohti 0} \frac{\sqrt{1+2x+3x^2}-\sqrt{1-x}}{x} \qquad
 \text{m)}\ \lim_{y \kohti 1} \frac{y-4\sqrt{y}+3}{y^2-1} \\
&\text{n)}\ \lim_{x \kohti 0} \frac{\sqrt{2-x}-\sqrt{2+x}}{x\sqrt{x}} \qquad
 \text{o)}\ \lim_{x \kohti 2} \frac{\sqrt{4-4x+x^2}}{x-2} \qquad 
 \text{p)}\ \lim_{t \kohti 8} \frac{t^{2/3}-4}{t^{1/3}-2} \\
&\text{q)}\ \lim_{x \kohti 3^+} \frac{\abs{x-3}}{3-x} \qquad
 \text{r)}\ \lim_{x \kohti 2^-} \frac{\sqrt{4-4x+x^2}}{x-2} \qquad
 \text{s)}\ \lim_{x \kohti \pi^+} \frac{\sqrt{1+\cos x}}{\cos\frac{x}{2}} \\
&\text{t)}\ \lim_{x \kohti -0.4^+} \frac{2x+5}{5x+2} \qquad
 \text{u)}\ \lim_{x \kohti 1^+} \frac{x}{\sqrt{x^2-1}} \qquad
 \text{v)}\ \lim_{x \kohti 1^+} \frac{\sqrt{x^2-x}}{x-x^2} \\
&\text{x)}\ \lim_{x\kohti\infty} \frac{1+x+x^4}{2+30x+200x^3} \qquad
 \text{y)}\ \lim_{x\kohti\infty} \left(\frac{x^2}{x+1}-\frac{x^2}{x-1}\right) \\
&\text{z)}\ \lim_{x \kohti -\infty} \frac{2x-1}{\sqrt{3x^2+x+1}} \qquad
 \text{å)}\ \lim_{x\kohti\infty} \frac{x\sqrt{x+1}\,(1-\sqrt{2x+3})}{7-6x+4x^2} \\[1mm]
&\text{ä)}\ \lim_{x\kohti\infty} \left(\sqrt{x^2+9x}-\sqrt{x^2-5x}\right) \qquad
 \text{ö)}\ \lim_{x \kohti -\infty} \left(\sqrt{x^2-4x}-\sqrt{x^2+10x}\right)
\end{align*}%
\endgroup

\item \label{H-V-2: väittämiä}
Todista Määritelmän \ref{funktion raja-arvon määritelmä} ja lukujonojen teorian
avulla: \vspace{1mm}\newline
a) Jos $f$ on monotoninen ja rajoitettu välillä $(a,b)\subset\DF_f$, niin on olemassa
raja-arvot $f(a^+)$ ja $f(b^-)$. \newline
b) Jos $f(x) \le g(x)\ \forall x\in(a,b)\subset\DF_f\cap\DF_g$, niin $f(a^+) \le g(a^+)$ ja
$f(b^-) \le g(b^-)$ sikäli kuin raja-arvot ovat olemassa reaalilukuina.

\item
Laajenna funktion $\,\D f(x)=\frac{x^2-1}{\sqrt{x+3}-2}\,$
määrittelyjoukko väliksi $[-3,\infty)$ jatkamalla. Mikä on jatketun funktion sievennetty
laskusääntö?

\item
Olkoon $f$ määritelty välillä $\,[1,\infty)\,$ ja olkoon $\,y_n=f(n)$, $n\in\N$. Näytä, että
pätee $\,\lim_{x\kohti\infty} f(x)=A\ \impl\ \lim_ny_n=A$. Näytä vastaesimerkillä, että
käänteinen implikaatio ei ole tosi.

\item
Määritä seuraaville funktioille $f(x)$ tai $y(x)$ asymptootti $g(x)$, joka on annettua
muotoa ($a,b\in\R$) ja mahdollisimman tarkka (ellei yksikäsitteinen). Tarkastele erikseen
tapaukset $x\kohti\infty$ ja $x \kohti -\infty$. \vspace{2mm}\newline
a) $\D \,f(x)=\frac{\abs{x+3}}{2x-1}\,, \quad g(x)=a$ \newline
b) $\D \,f(x)=\frac{(x+3)^2}{\abs{3x+1}}\,, \quad g(x)=ax+b$ \vspace{1mm}\newline
c) $\D \,f(x)=\sqrt{2x^2+3x+\cos x}, \quad g(x)=ax+b$ \vspace{3mm}\newline
d) $\D \,f(x)=\frac{1}{\sqrt{3x^2+4x}+\sqrt{x^2+x+1}}\,, \quad g(x)=\frac{a}{x}$ 
                                                         \vspace{1mm}\newline
e) $\D \,f(x)=\frac{\abs{3x+2}}{2x+3}\,, \quad g(x)=a+\frac{b}{x}$ \vspace{2mm}\newline
f) $\D \,9(x-1)^2-16(y+2)^2=25,\ y(x) \ge -2, \quad g(x)=ax+b$ \vspace{3mm}\newline
g) $\D \,(2x^2+1)y+x\,\cos y=(2x+1)^3, \quad g(x)=ax+b$

\item
Funktiolla $f(x)=\sqrt{x^4+8x^3+35x^2+78x+98}$ on asymptootteina toisen asteen polynomeja. 
Koeta löytää mahdollisimman tarkka tällainen asymptootti $p(x)$ ja arvioi approksimaation 
$f(x)\approx p(x)$ virhe, kun $|x|\ge 100$.

\item (*)
Olkoon $a>0$ ja $m,n\in\N$. Laske raja-arvo
$\displaystyle{\,\lim_{x \kohti a} \frac{\sqrt[m]{x}-\sqrt[m]{a}}{\sqrt[n]{x}-\sqrt[n]{a}}\,}$.

\item (*) Reaalifunktiosta $f$ tiedetään, että $f(x)=x+1\ \forall x\in\R,\ x \not\in X$, missä
$X\subset\R$ on äärellinen joukko. Joukkoa $X$ ei tunneta, eikä pisteistä $x \in X$ tiedetä
muuta kuin että $f$ on näissä pisteissä joko määritelty jollakin tuntemattomilla tavalla
(reaaliarvoisena) tai jätetty määrittelemättä. Näytä, että $\,\lim_{x \kohti 0} f(x)=1$.

\end{enumerate}