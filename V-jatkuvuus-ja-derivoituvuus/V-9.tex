\section{*Jatkuvuuden logiikka} \label{jatkuvuuden logiikka}
\alku

Tässä luvussa todistetaan Luvussa \ref{jatkuvuuden käsite} esitetyt kaksi jatkuvien
funktioiden päälausetta, Weierstrassin lause (Lause \ref{Weierstrassin peruslause}) ja
käänteisfunktion jatkuvuutta koskeva Lause \ref{käänteisfunktion jatkuvuus}. Todistukset
perustuvat Luvussa \ref{Cauchyn jonot} esitettyyn osajonojen teoriaan, ja ne ovat melko
vaativia. Erityisesti rajoitettuja reaalilukujonoja koskeva Bolzanon--Weierstrassin lause
(Lause \ref{B-W}) on todistuksissa ahkerassa käytössä. Weierstrassin lauseen todistuksen
jatkoksi näytetään, että sama todistustekniikka, yhdistettynä eräisiin polynomeja koskeviin
teknisempiin väittämiin, johtaa jopa Algebran peruslauseen todistukseen. Luvun lopussa
määritellään vielä käsite \kor{tasainen jatkuvuus}, jota voi pitää Lipschitz-jatkuvuuden
minimalistisena vastineena. Tasaisen jatkuvuuteen liittyen todistetaan eräs reaalianalyysin
hämmästyttävimmistä lauseista.

Koska tarkoituksena on todistaa Lauseet \ref{Weierstrassin peruslause} ja 
\ref{käänteisfunktion jatkuvuus} hieman yleisemmässä muodossa, määritellään aluksi
suljettua väliä yleisempi \kor{kompaktin} joukon käsite. Asiayhteyden vuoksi esitellään
samalla muitakin $\R$:n nk.\ \kor{topologisia} peruskäsitteitä.
 
\subsection{Avoimet, suljetut ja kompaktit joukot}

Avointa väliä
\[ 
U_{\delta}(x) = (x-\delta,x+\delta) \quad (\delta>0)
\] 
\index{ympzy@($\delta$-)ympäristö}%
sanotaan pisteen $x \in \R$ (avoimeksi) \kor{ympäristöksi} (tai tarkemmin 
$\delta$-ympäristöksi, engl.\ $\delta$-neighbourhood; vrt.\ vastaava kompleksitason käsite
Määritelmässä \ref{analyyttinen funktio}).
\begin{Def} \label{avoimet ym. joukot}
\index{avoin joukko|emph} \index{suljettu joukko|emph} \index{rajoitettu!b@joukko|emph}
\index{kompakti joukko|emph}
Joukko $A\subset\R$ on
\begin{itemize}
\item[-] \kor{avoin}, jos $\forall x \in A$ pätee
         $\,(x-\delta,x+\delta)\subset A\,\ \text{jollakin}\ \delta>0$,
\item[-] \kor{suljettu}, jos kaikille reaalilukujonoille $\seq{x_n}$ pätee: \newline
         $\,x_n \in A\ \forall n\ \ja\ x_n\kohti x\in\R\ \impl\ x\in A$,
\item[-] \kor{rajoitettu}, jos $\exists C\in\R_+$ siten, että 
         $\,\abs{x} \le C\ \forall x \in A$,
\item[-] \kor{kompakti}, jos $A$ on suljettu ja rajoitettu.
\end{itemize}
\end{Def}
\begin{Exa} Avoin väli $(a,b)$ on avoin joukko, sillä jos $x\in(a,b)$, nii
$U_\delta(x)\subset(a,b)$, kun $0<\delta\le\min\{x-a,b-x\}$. Suljettu väli on vastaavasti
suljettu joukko, sillä jos $x_n\in[a,b]\ \forall n$ ja $x_n \kohti x\in\R$, niin
$a \le x \le b$ (Lause \ref{jonotuloksia} [V1]) eli $x\in[a,b]$. Koska suljettu väli on myös
rajoitettu ($\abs{x} \le C = \max\{\abs{a},\abs{b}\}\ \forall x\in[a,b]$), niin suljettu
väli on kompakti joukko. \loppu
\end{Exa}
\begin{Exa}
\[
\begin{array}{ll}
\left[0,1\right] \cup [2,3]   &\text{on kompakti (ei avoin)}, \\[1mm]
(0,1) \cup (2,3)              &\text{on avoin ja rajoitettu (ei suljettu)}, \\[1mm]
\R                            &\text{on avoin ja suljettu (ei rajoitettu)}, \\[1mm]
[0,1)                         &\text{on rajoitettu (ei avoin eikä suljettu)}, \\[1mm]
\left[0,\infty\right)         &\text{on suljettu (ei avoin eikä rajoitettu)}, \\[1mm]
(0,\infty)                    &\text{on avoin (ei suljettu eikä rajoitettu)}, \\[1mm]
\Q                            &\text{ei avoin, ei suljettu eikä rajoitettu}, \\[1mm]
\emptyset                     &\text{(tyhjä) on avoin ja kompakti}. \qquad\quad\loppu
\end{array}
\]
\end{Exa}

\begin{Def}
\index{komplementti (joukon)|emph} \index{sulkeuma (joukon)|emph}
\index{reuna (joukon, alueen)|emph} \index{siszy@sisäpiste|emph}
Joukon $A\subset\R$
\begin{itemize}
\item[-] \kor{komplementti} on $\ \complement(A)=\{x\in\R \mid x\notin A\}$.
\item[-] \kor{sulkeuma} (engl. closure) on 
\[
\overline{A}=\{x\in\R \mid \exists\ \text{jono}\ \{x_n\}\ \text{siten, että}\ 
                                     x_n\in A\ \forall n\,\ \ja\,\ x_n \kohti x\}.
\]
\item[-] \kor{reuna} (engl. boundary) on 
         $\ \partial A=\overline{A}\cap \overline{\complement(A)}$.
\item[-] \kor{sisäpisteiden joukko} on $\ A_0 = \complement(\,\overline{\complement(A)}\,)$.
\end{itemize}
\end{Def}
Joukon sulkeuma on nimensä mukaisesti suljettu joukko, ja toinen määritelmä onkin:
$\overline{A}$ on pienin suljettu joukko, joka sisältää $A$:n. Sisäpisteiden joukko $A_0$ on
avoin (ks.\ Lause \ref{avoin vs suljettu} jäljempänä); tämän vaihtoehtoisia määritelmiä ovat:
$A_0$ on suurin $A$:n avoin osajoukko, tai
\[ 
A_0 = \{x \in A \mid x \not\in \partial A\}. 
\]
Pätee myös $\,\overline{A} = A_0 \cup \partial A,\ A_0 \cap \partial A = \emptyset$, samoin
pätee
\[ 
A\ \text{suljettu}\ \ekv\ A = \overline{A}\ \ekv\ \partial A \subset A.
\]
\begin{Exa}
Joukkojen
\[
A=(0,1],\quad B=(-1,0)\cup (0,1),\quad C=\Q
\]
sulkeumat, reunat ja sisäpisteiden joukot ovat
\[
\begin{array}{rclrclrcl}
\overline{A} &=& [0,1],\   & \overline{B} &=& [-1,1],\      & \overline{C} &=& \R, \\
  \partial A &=& \{0,1\},\ & \ \partial B &=&  \{-1,0,1\},\ & \ \partial C &=& \R, \\
         A_0 &=& (0,1),    & B_0          &=& B,\           & C_0          &=& \emptyset. 
                                                                           \quad\loppu  
\end{array}
\]
\end{Exa}
Haluttaessa selvittää, onko annettu joukko avoin, suljettu tai kompakti, helpottuu tehtävä
usein huomattavasti seuraavia loogisia väittämiä hyödyntämällä. Väittämistä ensimmäisen voi
tulkita myös suljetun tai avoimen joukon määritelmäksi, jos vain toinen käsitteistä on
määritelty erikseen. 
\begin{Lause} \label{avoin vs suljettu} Pätee
\[ 
A\,\text{ avoin }\ \ekv\ \complement(A)\,\text{ suljettu}, \qquad 
A\,\text{ suljettu }\ \ekv\ \complement(A)\,\text{ avoin}. 
\]
\end{Lause}
\begin{Lause} \label{unionit ja leikkaukset} Pätee
\begin{align*}
&A,B\,\text{ avoimia/suljettuja/kompakteja } \\
& \ \impl\ A\cup B\,\text{ ja } A\cap B\text{ avoimia/suljettuja/kompakteja}.
\end{align*}
\end{Lause}
\tod Väittämiin sisältyy yhteensä kymmenen implikaatioväittämää, joiden todistukset ovat
kaikki melko lyhyitä. Todistetaan esimerkkinä ainoastaan väittämä
$\,A$ avoin $\impl\ \complement (A)$ suljettu\, eli: Jos $A$ on avoin, niin jokaiselle
reaalilukujonolle $\seq{x_n}$ pätee
\[ 
x_n \in \complement(A)\ \forall n\,\ \ja\,\ x_n \kohti x \qimpl x \in \complement (A). 
\]
Tehdään vastaoletus: $x\not\in\complement (A)\ \ekv\ x \in A$. Tällöin 
$(x-\delta,x+\delta) \subset A$ jollakin $\delta>0$, koska $A$ oli avoin. Tällöin koska
$x_n\in\complement(A)\ \forall n$, on oltava
\[ 
\abs{x_n-x} \ge \delta\ \ \forall n \qimpl x_n \not\kohti x.
\]
Tämä on looginen ristiriita, koska oletettiin, että $x_n \kohti x$. \loppu
\begin{Exa} Joukko $A=(-\infty,a)\cup(b,\infty)$ on Lauseen \ref{unionit ja leikkaukset}
(tai suoraan Määritelmän \ref{avoimet ym. joukot}) mukaan avoin. Lauseen
\ref{avoin vs suljettu} mukaisesti komplementti
\[
\complement (A) = \begin{cases} 
                  \,[a,b], &\text{kun}\ a<b, \\ 
                  \,\{a\}, &\text{kun}\ a=b, \\
                  \,\emptyset, &\text{kun}\ a>b
                  \end{cases}
\]
on suljettu. \loppu
\end{Exa}
\begin{Exa} Yhden alkion sisältävä joukko $A=\{a\}$ on Määritelmän \ref{avoimet ym. joukot}
mukaan kompakti, joten Lauseen \ref{unionit ja leikkaukset} mukaan samoin on jokainen kahden,
kolmen, jne.\ alkion joukko. Siis jokainen äärellinen joukko on kompakti. \loppu
\end{Exa}
\begin{Exa} Rationaalifunktion $f(x)=p(x)/q(x)$ ($p$ ja $q$ polynomeja) määrittelyjoukko on
$\DF_f = \complement (A)$, missä $A = \{q\text{:n nollakohdat}\}$. Äärellisenä joukkona $A$
on suljettu (jopa kompakti), joten Lauseen \ref{avoin vs suljettu} mukaan $\DF_f$ on avoin.
\loppu \end{Exa} 

\subsection{Weierstrassin lause}
\index{Weierstrassin lause|vahv}

Funktion $f$ jatkuvuus suljetulla välillä $[a,b]\subset\DF_f$ tarkoitti Määritelmän
\ref{jatkuvuus välillä} mukaisesti jatkuvuutta 'sisältä päin' ko.\ välillä. Asetetaan
vastaavalla tavalla yleisempi määritelmä.
\begin{Def} \label{jatkuvuus kompaktissa joukossa}
\index{jatkuvuus (yhden muuttujan)!g@kompaktissa joukossa|emph}
Reaaliunktio $f$ on \kor{jatkuva kompaktissa joukossa} $K\subset\DF_f$, jos jokaiselle
reaalilukujonolle $\seq{x_n}$ pätee
\[
x_n \in K\ \forall n\,\ \ja\,\ x_n \kohti x\,\ 
                               \impl\,\ f(x_n) \kohti f(x).\footnote[2]{Huomautettakoon,
että määritelmän ehto toteutuu mille tahansa reaalifunktiolle $f$, jos $K\subset\DF_f$ on
äärellinen (ja siis kompakti) joukko. Vrt.\ alkuperäinen jatkuvuuden määritelmä
(Määritelmä \ref{funktion jatkuvuus}), jonka mukaan funktio on jatkuva määrittelyjoukkonsa
eristetyissä pisteissä.}
\]
\end{Def}
\begin{*Lause} \label{kompaktissa joukossa jatkuva funktio on rajoitettu}
Jos $f$ on jatkuva kompaktissa joukossa $K\subset\DF_f$, niin $f(K)$ on rajoitettu joukko.
\end{*Lause}
\tod Käytetään epäsuoraa todistustapaa, eli tehdään vastaoletus: $f(K)$ ei ole rajoitettu.
Tällöin on olemassa jono $\seq{x_n}$ siten, että
\[
x_n \in K\ \forall n\,\ \ja\,\ \abs{f(x_n)}\kohti\infty.
\]
Koska $A$ on (kompaktina joukkona) rajoitettu, on jono $\seq{x_n}$ rajoitettu. Tällöin
jonolla on Bolzanon--Weierstrassin lauseen (Lauseen \ref{B-W}) mukaan suppeneva osa\-jono. Kun
indeksoidaan tämä osajono jonoksi $\seq{x_k}$, niin pätee siis $x_k \kohti x\in\R$. Koska $K$
on (kompaktina joukkona) suljettu, on oltava $x \in K$. On siis löydetty jono $\seq{x_k}$,
jolle pätee $x_k \in K\ \forall k$ ja $x_k \kohti x \in K$. Tällöin jatkuvuusoletuksen ja
Määritelmän \ref{jatkuvuus kompaktissa joukossa} mukaan on oltava $f(x_k) \kohti f(x)$.
Toisaalta koska alkuperäiselle jonolle $\seq{f(x_n)}$ pätee $\abs{f(x_n)}\kohti\infty$, niin
myös $\abs{f(x_k)}\kohti\infty$. Oletuksista ja vastaoletuksesta seurasi siis, että on
olemassa jono $\seq{x_k}$, jolle pätee sekä $f(x_k) \kohti f(x)\in\R$ että
$\abs{f(x_k)}\kohti\infty$. Tämä on looginen ristiriita, joten lause on todistettu. \loppu
 
Seuraava tulos sisältää erikoistapauksena Weierstrassin lauseen
\ref{Weierstrassin peruslause}.
\begin{*Lause} \label{weierstrass} Jos reaalifunktio $f$ on jatkuva kompaktissa joukossa
$K\subset\DF_f$, niin $f$ saavuttaa $K$:ssa pienimmän ja suurimman arvonsa. 
\end{*Lause}
\tod Koska $f(K)$ on Lauseen \ref{kompaktissa joukossa jatkuva funktio on rajoitettu} mukaan
rajoitettu, niin Lauseen \ref{supremum-lause} mukaan tällä joukolla on supremum:
$\,\sup f(K) = y\in\R$. Supremumin määritelmän (ks.\ Luku \ref{reaalilukujen ominaisuuksia})
mukaisesti on olemassa jono $\seq{y_n}$, jolle pätee $\,y_n \in f(K)\ \forall n$ ja
$y_n \kohti y$. Koska $y_n \in f(K)$, on $f(x_n) = y_n$ jollakin $x_n \in K$. Koska $K$ on
(kompaktina joukkona) rajoitettu, on jono $\seq{x_n}$ on rajoitettu, joten sillä on Lauseen
\ref{B-W} mukaan suppeneva osajono. Indeksoidaan tämä uudelleen jonoksi $\seq{x_k}$, jolloin
pätee $x_k \kohti x\in\R$. Koska $K$ on (kompaktina joukkona) suljettu, niin on oltava
$x \in K$. Tällöin oletuksen ja Määritelmän \ref{jatkuvuus kompaktissa joukossa} mukaan
$f(x_k) \kohti f(x)$. Mutta jono $\seq{f(x_k)}$ on alkuperäisen jonon $\seq{f(x_n)}$ osajono,
joten pätee myös $f(x_k) \kohti y$ (Lause \ref{suppenevat osajonot}). Siis $f(x)=y$ 
(koska lukujonon raja-arvo on yksikäsitteinen). Koska $y$ on $f(K)$:n yläraja, on 
$f(t) \le y\,\ \forall t \in K$ --- siis $f$ saavuttaa joukossa $K$ suurimman arvonsa
pisteessä $x$. Lause on näin todistettu maksimiarvon osalta. Minimiarvon osalta lause
todistetaan joko vastaavalla päättelyllä tai soveltamlla jo todistettua väittämää funktioon
$-f$. \loppu
\begin{Exa} Funktio 
\[ f(x)= \begin{cases} \,1/x,\,\ &\text{kun}\ x>0, \\ \,0, &\text{kun}\ x=0 
         \end{cases} \]on jatkuva (kompaktissa) joukossa $K=[a,b]$, jos $0<a<b$. Tällöin $f$:n maksimimiarvo $K$:ssa
$= f(a)$ ja minimiarvo $= f(b)$. Kompaktilla välillä $[0,1]\subset\DF_f$ ei $f$ ole 
Määritelmän \ref{jatkuvuus kompaktissa joukossa} mukaisesti jatkuva, eikä $f$ myöskään saavuta
maksimiarvoaan tällä välillä (minimiarvo $=0$). Välin $A=(0,1]$ (rajoitettu, ei suljettu)
jokaisessa pisteessä $f$ on jatkuva, mutta $f$ ei saavuta $A$:ssa maksimiarvoaan
(minimiarvo $=1$). Välillä A=(1,2) (rajoitettu, ei suljettu) $f$ ei saavuta kumpaakaan
ääriarvoaan. Välillä $A=[1,\infty)$ (suljettu, ei rajoitettu) $f$ on jatkuva jokaisessa
pisteessä, mutta $f$ ei saavuta $A$:ssa minimiarvoaan (maksimiarvo $=1$). \loppu
\end{Exa}

\subsection{Algebran peruslause}
\index{Algebran peruslause|vahv}

Käsitteet avoin, suljettu ja kompakti joukko määritellään kompleksitasossa aivan samalla
tavoin kuin $\R$:ssä. Erona on ainoastaan, että avoimen joukon määrittelyssä tarvittava
ympäristö on $\C$:ssä kiekon muotoinen: $\,U_\delta(c)=\{z\in\C \mid \abs{z-c}<\delta\}$
(Määritelmä \ref{analyyttinen funktio}). Lauseet \ref{avoin vs suljettu} ja
\ref{unionit ja leikkaukset} pätevät myös kompleksitasossa.
\begin{Exa} Kiekko $K=\{z\in\C \mid \abs{z} \le R\}$ on suljettu, ts.\ jokaiselle
kompleksilukujonolle $\seq{z_n}$ pätee: 
$z_n \in K\,\forall n\,\ja\,z_n \kohti z\ \impl\ z \in K$. $K$ on myös rajoitettu
($\abs{z} \le C=R\ \forall z \in K$), joten $K$ on kompakti joukko. Komplementti
$\complement(K)=\{z\in\C \mid \abs{z}>R\}$ on avoin joukko. \loppu
\end{Exa}
Kompleksifunktion jatkuvuus määriteltiin edellisessä luvussa vastaavalla tavalla kuin
reaalifunktioille, ja myös jatkuvuus kompaktissa joukossa (Määritelmä 
\ref{jatkuvuus kompaktissa joukossa}) yleistyy vastaavasti koskemaan kompleksifunktioita.
Weierstrassin lauseen vastine kompleksifunktioille on seuraava väittämä, jonka todistus
noudattaa hyvin tarkoin Lauseen \ref{weierstrass} todistuksen logiikkaa.
(Todistus sivuutetaan.)
\begin{*Lause} \label{weierstrass kompleksifunktioille}
Jos $f$ on kompaktissa joukossa $K\subset\C$ jatkuva kompleksiarvoinen funktio, niin
$\abs{f}$ saavuttaa $K$:ssa pienimmän ja suurimman arvonsa.
\end{*Lause}
Algebran peruslauseen (Lause \ref{algebran peruslause}) todistus voidaan perustaa tähän
tulokseen sekä Luvun \ref{III-3} tarkasteluihin: Olkoon $p(z)$ polynomi astetta $n \ge 1$
(ei vakio). Tällöin $\abs{p(z)}$ saavuttaa Lauseen \ref{weierstrass kompleksifunktioille}
mukaan minimiarvonsa kompaktissa joukossa
\[
K=\{z\in\C \ | \ \abs{z}\leq R\}
\]
jokaisella $R \in \R_+$. Koska $\abs{p(z)} \sim \abs{z}^n$, kun $\abs{z}\kohti\infty$ 
(ks.\ Propositio \ref{polynomin kasvu}), niin päätellään, että $R$:n ollessa riittävän suuri
on $\abs{p}$:n minimikohta $K$:ssa samalla $\abs{p}$:n absoluuttinen minimikohta $\C$:ssä. Siis 
$\abs{p}$ saavuttaa jossakin pisteessä $c \in \C$ absoluuttisen minimiarvonsa. Tällöin on
seuraavan väittämän mukaan oltava $p(c)=0$, jolloin Algebran peruslause on todistettu.
\begin{Lause} \label{polynomitulos} Jos kompleksimuuttujan polynomi $p$ ei ole vakio, niin 
$p(c)=0$ jokaisessa pisteessä $c \in \C$, jossa $\abs{p}$:llä on paikallinen minimi. 
\end{Lause}
\tod Ks. Harj.teht.\,\ref{III-3}:\ref{H-III-3: avaintulos}. \loppu

\subsection{Käänteisfunktion jatkuvuus}
\index{jatkuvuus (yhden muuttujan)!e@käänteisfunktion|vahv}

Seuraava käänteisfunktion jatkuvuuden takaava lause on yleistys Lauseesta 
\ref{käänteisfunktion jatkuvuus}. Todistuksessa on Bolzanon--Weierstrassin lause jälleen
keskeisessä roolissa.
\begin{*Lause} \label{R:n käänteisfunktiolause} Jos $f$ on jatkuva kompaktissa joukossa
$A\subset\DF_f$ ja $f:\,A \kohti B$ on bijektio, niin $B$ on kompakti ja
$\inv{f}:\,B \kohti A$ on samoin jatkuva kompaktissa joukossa $B$.
\end{*Lause}
\tod  Näytetään ensin, että $B$ on kompakti joukko. Lauseen
\ref{kompaktissa joukossa jatkuva funktio on rajoitettu} mukaan $B$ on rajoitettu, joten 
riittää osoittaa, että $B$ on suljettu. Oletetaan siis, että $y_n \in B\ \forall n$ ja 
että $y_n \kohti y\in\R$. Koska $y_n \in B=f(A)$, niin jokaisella $n$ on $y_n=f(x_n)$ jollakin
$x_n \in A$. Koska $A$ on (kompaktina joukkona) rajoitettu, niin jono $\seq{x_n}$ on
rajoitettu, jolloin tällä jonolla on osajono $\seq{x_k}$, joka suppenee: $x_k \kohti x\in\R$
(Lause \ref{B-W}). Koska $A$ on (kompaktina joukkona) suljettu, niin $x \in A$. Tällöin koska
$f$ on $x$:ssä jatkuva Määritelmän \ref{jatkuvuus kompaktissa joukossa} mukaisesti, niin
 $f(x_k) \kohti f(x)$. Mutta $\seq{f(x_k)}=\seq{y_k}$ on jonon $\seq{y_n}$ osajono, joten
$f(x_k)=y_k \kohti y$. Lukujonon raja-arvon yksikäsitteisyyden perusteella on silloin
$f(x)=y$, joten $y \in f(A)=B$. On näytetty, että jokaiselle lukujonolle $\seq{y_n}$ pätee: 
$y_n \in B\ \forall n\,\ \ja\,\ y_n \kohti y\ \impl\ y \in B$. Siis $B$ on suljettu.

Käänteisfunktion $\inv{f}$ väitetyn jatkuvuuden osoittamiseksi on näytettävä: Jos
$y_n \in B,\ n=1,2,\ldots\,$ ja $f(x_n)=y_n\,,\ x_n \in A$, niin pätee
\[
y_n\kohti y \in B \qimpl x_n\kohti x\in A \ \ja \ x=f^{-1}(y).
\]
Koska $x_n \in A$ ja $A$ on kompakti, niin jono $\seq{x_n}$ on rajoitettu, joten jonolla on
suppeneva osajono $\seq{x_k}:\ x_k \kohti x \in \R$. Koska $A$ on kompakti, niin $x \in A$,
jolloin $f$:n jatkuvuuden nojalla $f(x_k) \kohti f(x)$. Mutta jono $\seq{f(x_k)}$ on jonon 
$\seq{f(x_n)} = \seq{y_n}$ osajono, joten oletuksen $y_n \kohti y$ mukaan on oltava myös 
$f(x_k) \kohti y$. Siis $f(x)=y\ \ekv\ x=f^{-1}(y)$. On päätelty, että jonolla $\seq{x_n}$
on ainakin osajono $\seq{x_k}$, jolle pätee $x_k \kohti x \in A$ ja $x = f^{-1}(y)$. Näytetään
nyt, että myös alkuperäiselle jonolle pätee $x_n \kohti x$, jolloin lause on todistettu. 
Tehdään vastaoletus: $x_n \not\kohti x$. Tällöin Lauseen \ref{negaatioperiaate}
mukaan on olemassa toinen osajono $\{x_l\}$ ja $\eps>0$ siten, että pätee
\[
\abs{x_l-x}\geq\eps\quad\forall l\in\N.
\]
Tälläkin osajonolla on kuitenkin suppeneva osajono $\{x_\nu\}$, jolle siis pätee 
$x_\nu\kohti x'\neq x$ kun $\nu\kohti\infty$. Tällöin on jälleen $x'\in A$, koska $A$ on
suljettu, joten $f$:n jatkuvuuden perusteella $f(x_\nu)\kohti f(x')$. Toisaalta koska jono
$\seq{f(x_\nu)}$ on edelleen alkuperäisen jonon $\seq{f(x_n)} = \seq{y_n}$ osajono, niin
oletuksesta $y_n \kohti y$ seuraa, että myös $f(x_\nu) \kohti y$. Siis pätee sekä 
$f(x_\nu) \kohti f(x')$ että $f(x_\nu) \kohti y$, jolloin on oltava $f(x')=y=f(x)$.
Oletuksista ja vastaoletuksesta on näin johdettu päätelmä: On olemassa $x \in A$ ja 
$x' \in A$ siten, että $x \neq x'$ ja $f(x)=f(x')$. Mutta (toistaiseksi käyttämättömän)
oletuksen mukaan $f$ on injektio, joten  $x \neq x'\ \impl\ f(x) \neq f(x')$. Siis
 $f(x)=f(x')$ ja $f(x) \neq f(x')$ --- looginen ristiriita, joka osoittaa tehdyn
vastaoletuksen vääräksi. Siis $x_n \kohti x=f^{-1}(y)$. \loppu
\begin{Exa}
Määritellään funktio $f$ joukossa
\begin{multicols}{2} \raggedcolumns
\[
D_f=A=[-1,0]\cup (1,2]
\]
seuraavasti:
\[
f(x)=\begin{cases}
x &,\text{ kun } x\in [-1,0] \\
x-1 \ &, \text{ kun } x\in (1,2]
\end{cases}
\]
\begin{figure}[H]
\setlength{\unitlength}{1cm}
\begin{center}
\begin{picture}(4,4)(-2,-2)
\put(-2,0){\vector(1,0){4}} \put(1.8,-0.4){$x$}
\put(0,-2){\vector(0,1){4}} \put(0.2,1.8){$y$}
\drawline(-1,-1)(0,0)
\drawline(1,0)(2,1)
\put(-0.1,-0.1){$\bullet$}
\end{picture}
%\caption{$y=f(x)$}
\end{center}
\end{figure}
\end{multicols}
Tällöin $f$ on koko määrittelyjoukossaan jatkuva ja $f:A \kohti [-1,1]$ on bijektio, joten myös 
$\inv{f}:[-1,1] \kohti A$ on bijektio (vrt. Luku \ref{käänteisfunktio}). Mutta \inv{f} ei ole 
jatkuva pisteessä $x=0$. \loppu
\end{Exa}
Esimerkissä on käänteisfunktion jatkuvuuden kannalta ongelmana, että määrittelyjoukon osaväli 
$(1,2]$ ei ole suljettu. Ongelmaa ei voi poistaa ottamalla $x=1$ mukaan määrittelyjoukkoon,
sillä jos asetetaan $f(1)=0$, niin $f(0)=f(1)$, jolloin $f$ ei ole injektio, ja jos asetetaan 
$f(1)=c \neq 0$, niin $f$ ei ole (oikealta) jatkuva pisteessä $x=1$.

\subsection{Tasainen jatkuvuus}
\index{tasainen jatkuvuus|vahv} \index{jatkuvuus (yhden muuttujan)!h@tasainen|vahv}

Tarkastellaan vielä uutta jatkuvuuden käsitettä, jota voi pitää Luvussa 
\ref{väliarvolause 2} määritellyn Lipschitz-jatkuvuuden minimalistisena vastineena 
(vrt.\ Määritelmä \ref{funktion l-jatkuvuus} ja Lause \ref{tasaisen jatkuvuuden käsite} alla).
\begin{Def} Reaalifunktio $f$ on joukossa $A\subset\DF_f$ \kor{tasaisesti jatkuva} 
(engl. uniformly continuous), jos kaikille reaalilukujonoille $\seq{x_n}$ ja $\seq{t_n}$ pätee
\[
x_n,t_n\in A \ \ja \ x_n-t_n\kohti 0 \ \impl \ f(x_n)-f(t_n)\kohti 0.
\]
\end{Def}
Jos $f$ on tasaisesti jatkuva välillä $(c-\delta,c+\delta)$, $\delta>0$, niin valitsemalla 
$y_n=c$ ym.\ määritelmässä nähdään, että $f$ on jatkuva pisteessä $c$. Tasainen jatkuvuus 
on siis tässä mielessä vahvempi ominaisuus kuin Määritelmän \ref{funktion jatkuvuus} mukainen
\index{pisteittäinen jatkuvuus}%
jatkuvuus, jota myös tavataan sanoa \kor{pisteittäiseksi} (engl. pointwise).
\begin{Exa}
$f(x)=1/x$ on koko määrittelyjoukossaan (pisteittäin) jatkuva, mutta ei tasaisesti jatkuva: 
Esim.\ jos $x_n=1/n$ ja $t_n=2/n$, niin $x_n-t_n \kohti 0$, mutta 
$f(x_n)-f(t_n) = n/2 \not\kohti 0$.
\end{Exa}
Tasaisen jatkuvuuden merkitys näkyy selvemmin seuraavasta tuloksesta, jota voi myös pitää 
tasaisen jatkuvuuden vaihtoehtoisena määritelmänä, vrt. Lause \ref{approksimaatiolause}.
Todistus (joka sivuutetaan) on idealtaan sama kuin Lauseen
\ref{jatkuvuuskriteerien yhtäpitävyys} todistus.
\begin{*Lause} \label{tasaisen jatkuvuuden käsite}
Funktio $f:\DF_f\kohti\R$, $\DF_f\subset\R$, on joukossa $A\subset\DF_f$ tasaisesti jatkuva 
täsmälleen kun jokaisella $\eps>0$ on olemassa $\delta>0$ siten, että $\forall x_1,x_2\in\R$
pätee
\[
x_1,x_2\in A \ \ja \ \abs{x_1-x_2}<\delta \ \impl \ \abs{f(x_1)-f(x_2)}<\eps.
\]
\end{*Lause}
\begin{Exa} Suljetulla välillä $A=[a,b]$ Lipschitz-jatkuva funktio (Määritelmä
\ref{funktion l-jatkuvuus}) on ko.\ välillä myös tasaisesti jatkuva, sillä Lauseen
\ref{tasaisen jatkuvuuden käsite} ehto toteutuu jokaisella $\eps>0$, kun valitaan
$\delta=\eps/L$, missä $L=$ Lipschitz-vakio. \loppu
\end{Exa}
Seuraava lause --- joka jälleen nojaa Bolzanon--Weierstrassin lauseeseen --- kuuluu
tavanomaisen reaalianalyysin hämmästyttävimpiin tuloksiin.
\begin{*Lause} \label{kompaktissa joukossa jatkuva on tasaisesti jatkuva}
Jos reaalifunktio $f$ on jatkuva kompaktissa joukossa $K\subset\DF_f$, niin $f$ on
$K$:ssa tasaisesti jatkuva.
\end{*Lause}
\tod Tehdään vastaoletus: $f$ ei ole $K$:ssa tasaisesti jatkuva. Tällöin on olemassa jonot 
$\{x_n\}$ ja $\{t_n\}$ siten, että
\[
x_n,t_n\in K\ \forall n\,\ \ja\,\ x_n-t_n\kohti 0\,\ \ja\,\ f(x_n)-f(t_n) \not\kohti 0.
\]
Koska $f(x_n)-f(t_n) \not\kohti 0$, niin jonolla $\{f(x_n)-f(t_n)\}$ on Lauseen 
\ref{negaatioperiaate} mukaan osajono $\{f(x_k)-f(t_k)\}$ siten, että jollakin $\eps>0$ pätee
\[
\abs{f(x_k)-f(t_k)} \ge \eps \quad\forall k\in\N.
\]
Mutta jonot $\seq{x_k}$ ja $\seq{t_k}$ ovat rajoitettuja (koska $x_k,t_k \in K$ ja $K$ on
rajoitettu), joten ensinnäkin jonolla $\seq{x_k}$ on suppeneva osajono $\seq{x_l}$
(Lause \ref{B-W}). Kun nyt siirrytään tarkastelemaan jonon $\seq{t_k}$ vastaavaa osajonoa
$\seq{t_l}$, niin tällä on edelleen suppeneva osajono $\seq{t_\nu}$. Mutta tällöin
vastaava jonon $\seq{x_l}$ osajono $\seq{x_\nu}$ suppenee myös (koska on suppenevan jonon
osajono). Näin on löydetty lukuparien $(x_k,t_k)$ muodostamasta jonosta osajono
$\seq{(x_\nu,t_\nu)}$, jossa pätee:  
\[
x_\nu \kohti x\in \R\,\ \ja\,\ t_\nu \kohti t\in\R.
\]
Tässä on oltava $x,t \in K$, koska $K$ on suljettu, ja on myös oltava $x=t$, koska 
$x_n-t_n \kohti 0\ \impl\ x_\nu-t_\nu \kohti 0$. Koska siis $x_\nu \kohti x$ ja 
$t_\nu \kohti x$, $x \in K$, niin $f$:n jatkuvuuden perusteella
\[
f(x_\nu)-f(t_\nu) \kohti f(x)-f(x) = 0.
\]
On siis päätelty, että $f(x_\nu)-f(t_\nu) \kohti 0$, ja toisaalta 
$\abs{f(x_\nu)-f(t_\nu)}\ge\eps>0\ \forall \nu$. Tämä on  looginen ristiriita, joten
vastaoletus on väärä ja lause siis todistettu. \loppu

\Harj
\begin{enumerate}

\item
Mitä ominaisuuksista (avoin, suljettu, rajoitettu, kompakti) on seuraavilla 
$\R$:n osajoukoilla:
\newline
a) \ $A=(0,1) \cup \{\pi\}$ \newline
b) \ $A=[-1000,\,10)$ \newline
c) \ $A=[-1000,\,1000000000000000000000000]$ \newline
d) \ $A=(-0.00000000002,\,0.00000000000000007)$ \newline
e) \ $A=\N$ \newline
f) \ $A=\{\frac{1}{n} \mid n\in\N\}$ \newline
g) \ $A=(-\infty,\,3]$ \newline
h) \ $A=\{\frac{1}{n} \mid n\in\N\}\cup[-1,0]$

\item
Näytä, että jos $A\subset\R$, niin pätee
\[
A\ \text{kompakti} \qimpl \sup A=\max A\,\ \ja\ \inf A=\min A.
\]
Näytä vastaesimerkillä, että implikaatio ei päde kääntäen.

\item
Todista seuraavat Lauseiden \ref{avoin vs suljettu} ja \ref{unionit ja leikkaukset}
osaväittämät: \vspace{1mm}\newline
a) \ $A$ suljettu $\impl$ $\complement(A)$ avoin \newline
b) \ $A$ ja $B$ avoimia $\impl$ $A \cap B$ avoin \newline
c) \ $A$ ja $B$ suljettuja $\impl$ $A \cup B$ suljettu

\item
Määrittele seuraavien joukkojen reuna $\partial A\,$: \vspace{1mm}\newline
a) \ $A=\{1,3,7\} \quad$ 
b) \ $A=\{\frac{1}{n} \mid n\in\N\} \quad$ 
c) \ $A=(0,1)\cap\Q$

\item
Näytä, että jos $f$ on $\R$:ssä jatkuva funktio ja $B\subset\R$ on suljettu joukko, niin
myös $A=\{x\in\R \mid f(x) \in B\}$ on suljettu. Päättele, että erityisesti $f$:n
nollakohtien joukko on suljettu.

\item
Näytä, että yhtälö
\[
8-63y^5+90y^7-35y^9=x
\]
määrittelee välillä $[0,8]$ funktion $y=f(x)$ ja että $f:\ [0,8] \kohti [0,a]$ on jatkuva
bijektio eräällä --- millä? --- $a$:n arvolla.

\item \index{Hzz@Hölder-jatkuvuus}
Funktio $f$ on välillä $[a,b]$ \kor{Hölder--jatkuva indeksillä} $\alpha$,
$\alpha\in(0,1]\cap\Q$, jos jollakin $C\in\R_+$ pätee
\[
|f(x_1)-f(x_2)| \,\le\, C|x_1-x_2|^\alpha, \quad x_1,x_2\in[a,b].
\]
a) Näytä, että välillä $[a,b]$ Hölder--jatkuva funktio
on ko.\ välillä tasaisesti jatkuva. \ b) Näytä, että funktio $f(x)=\sqrt[n]{x},\ n\in\N$ on
välillä $[0,1]$ Hölder--jatkuva täsmälleen indekseillä $0 < \alpha \le 1/n$.

\item (*)
Olkoon $f$ jatkuva välillä $[0,1]$ ja muodostetaan lukujono $\seq{y_n}$ seuraavasti:
\[
y_n = \max_i\{f(i/2^n),\ i=0 \ldots 2^n\}, \quad n=0,1,\ldots
\]
Näytä, että $\seq{y_n}$ on monotonisesti kasvava lukujono ja että
\[
\max_{x\in[0,1]}f(x)=\lim_n y_n\,.
\]
Löytyykö myös $f$:n maksimikohta tällä tavoin (algoritmisesti)\,?

\end{enumerate}