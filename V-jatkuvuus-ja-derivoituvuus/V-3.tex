\section{Derivaatta} \label{derivaatta}
\alku

Tarkastellaan funktion $f$ approksimoimista pisteen $a\in\DF_f$ lähellä perustuen
erilaisiin olettamuksiin funktion ennustettavuudesta. Ensinnäkin jos $f$ jatkuva
pisteessä $a$, niin Määritelmän \ref{vaihtoehtoinen jatkuvuus} mukaan $f$ on $a$:n lähellä
likimain vakio: $f(x) \approx f(a)$ kun $x \approx a$. Tämän approksimaation virhe on
$a$:n lähellä tyypillisesti muotoa $f(x)-f(a) \approx L(x-a)$, missä $L$ on $f$:stä ja $a$:sta
riippuva vakio. Erikoistapauksia (kuten $f(x)=$ vakio tai $f(x)=x^2,\ a=0$) lukuunottamatta
mainittu virhearvio on yleisesti paras mahdollinen.
\begin{Exa} Jos $f(x)=x^3$ ja $a \neq 0$, niin
\[
f(x)-f(a) \,=\, x^3-a^3 \,=\, (x-a)(x^2+ax+a^2) \,=\, (x-a)g(x).
\]
Tässä $g(x)=x^2+ax+a^2$ on (polynomina) jatkuva, joten pisteen $x=a$ lähellä on
$f(x)-f(a) \approx L(x-a)$, missä $L=g(a)=3a^2$. \loppu
\end{Exa} 
Jos jatkuvalle funktiolle halutaan pisteen $a$ lähellä olennaisesti tarkempi approksimaatio
kuin $f(x) \approx f(a)$, on approksimaation oltava jotakin muuta tyyppiä kuin 
$f(x) \approx A =$ vakio, sillä näistä vaihtoehdoista valinta $A=f(a)$ on paras. Koska 
vakio = polynomi astetta $n=0$, niin luonteva seuraava yritys on käyttää approksimaatiossa
polynomia $p$ astetta $n=1$. Kun ennakkoehdoksi asetetaan $p(a)=f(a)$, niin approksimaatio
saa muodon
\[
f(x) \approx f(a) + k(x-a).
\]
Tässä kerroin $k \in \R$ määrätään (jos mahdollista) niin, että virheelle
$g(x)=f(x)-f(a)-k(x-a)$ saadaan $a$:n lähellä olennaisesti parempi arvio kuin
$\abs{g(x)} \le L\abs{x-a}$. Asetetaan minimiehdoksi
\[
\lim_{x \kohti a}\,\frac{g(x)}{x-a} = 0 
                  \,\ \ekv\,\ \lim_{x \kohti a} \left[\frac{f(x)-f(a)}{x-a}-k\right] = 0
                  \,\ \ekv\,\ \lim_{x \kohti a} \frac{f(x)-f(a)}{x-a} = k.
\]
Riippuen funktiosta $f$ kerroin $k$ siis joko määräytyy tästä ehdosta yksikäsitteisesti
(raja-arvo oikealla olemassa  reaalilukuna) tai sitten kerrointa ei voi määrätä, jolloin
ym.\ approksimaatioyritys katsotaan epäonnistuneeksi.
\jatko \begin{Exa} (jatko) Esimerkin funktiolle pätee
\begin{align*}
f(x) &\,=\, f(a)+(x-a)(x^2+ax+a^2) \\
     &\,=\, f(a)+3a^2(x-a)+(x-a)(x^2+ax-2a^2) \\
     &\,=\, f(a)+3a^2(x-a)+(x-a)^2(x+2a).
\end{align*}
Siis jos valitaan $k=3a^2$, niin $g(x)/(x-a)=(x-a)(x+2a) \kohti 0$ kun $x \kohti a$, joten
approksimaatio onnistui. \loppu
\end{Exa}
\begin{Def} \vahv{(Derivaatta)} \label{derivaatan määritelmä}
\index{derivaatta|emph} \index{derivoituvuus|emph} \index{linearisaatio (funktion)|emph}
%\index{funktion approksimointi!ab@linearisaatiolla|emph}
Funktio $f:\DF_f\kohti\R$, $\DF_f\subset\R$, on pisteessä $a\in\DF_f$ \kor{derivoituva} 
(engl.\ differentiable), jos $(a-\delta,a+\delta)\subset\DF_f$ jollakin $\delta>0$ ja on
olemassa raja-arvo
\[
\lim_{x\kohti a} \frac{f(x)-f(a)}{x-a} = k \in \R.
\]
Lukua $k$ kutsutaan $f$:n \kor{derivaataksi} (engl. derivative) pisteessä $a$, ja merkitään 
$k=f'(a)$. Jos $f$ on pisteessä $a$ derivoituva, niin polynomia 
\[ 
p(x) = f(a) + f'(a)(x-a) 
\]
sanotaan $f$:n \kor{linearisoivaksi approksimaatioksi} eli \kor{linearisaatioksi}
pisteessä $a$.
\end{Def}
Linearisoivan approksimaation geometrinen vastine on pisteen $P=(a,f(a))$ kautta kulkeva suora,
jonka
\index{kulmakerroin} \index{tangentti (käyrän)}%
\kor{kulmakerroin} on $k=f'(a)\ (=p'(a))$. Sanotaan, että ko.\ suora on käyrän
$S: y=f(x)$ \kor{tangentti} pisteessä $P$. Tämän mukaisesti siis 'derivaatta on tangentin
kulmakerroin'.
\begin{figure}[H]
\setlength{\unitlength}{1cm}
\begin{picture}(13,7)(-4,-2)
\put(0,-1.5){\vector(0,1){5.5}} \put(0.2,3.7){$y$}
\put(-2,0){\vector(1,0){10}} \put(7.7,-0.4){$x$}
\curve(0,0,0.5,0.0625,1,0.25,1.5,0.5625,2,1,2.5,1.5625,3,2.25,3.5,3.0625,4,4)
\put(0,-1){\line(1,1){4}}
\put(1.9,-0.4){$a$} \dashline{0.1}(2,0)(2,1)
\put(4.2,4){$S: y=f(x)$} \put(4,2.6){$y=p(x)$}
\put(1.93,0.93){$\scriptstyle{\bullet}$} \put(2.2,0.8){$P$}
\end{picture}
\end{figure}
\jatko \begin{Exa} (jatko) Esimerkin tuloksen perusteella funktio $\,f(x)=x^3\,$ on 
derivoituva jokaisessa pisteessä $a\in\R$ ja $f'(a)=3a^2$. Käyrän $S: y=x^3$ tangentin
yhtälö pisteessä $x=a$ on siis $y=p(x)=a^3+3a^2(x-a)$. \loppu \end{Exa}
\begin{Exa} \label{ei-derivoituva f} Funktio $f(x)=\abs{x-a}$ ei ole derivoituva pisteessä
$a$, sillä
\[
\frac{f(x)-f(a)}{x-a} \,=\, \begin{cases} 
                            \ \ 1, \quad &\text{kun}\ x>a, \\
                               -1, \quad &\text{kun}\ x<a. 
                            \end{cases} \loppu
\]
\end{Exa}
Jos $f$ on derivoituva pisteessä $a$, niin Määritelmän \ref{derivaatan määritelmä} ja
Lauseen \ref{funktion raja-arvojen yhdistelysäännöt} perusteelle pätee
\[
\lim_{x \kohti a} f(x) \,=\, \lim_{x \kohti a}\left[f(a)+(x-a)\,\frac{f(x)-f(a)}{x-a}\right]
                      \,=\, f(a) + 0 \cdot f'(a) = f(a).
\]
Siis derivoituvuus on jatkuvuutta vahvempi ominaisuus:
\[
\boxed{\kehys\quad f \text{ derivoituva pisteessä } a \ 
                              \impl \ f \text{ jatkuva pisteessä } a. \quad}
\]

\subsection{Derivaatta operaattorina}

Kun derivaatan määritelmässä kirjoitetaan $x-a=\Delta x$ ja asetetaan $a$:n tilalle $x$, niin
määritelmä saa muodon
\[
f'(x) = \lim_{\Delta x\kohti 0} \frac{f(x+\Delta x)-f(x)}{\Delta x}\,.
\]
Derivaatan muita merkintöjä ovat
\[
\boxed{\quad f'(x)=\frac{\ykehys df}{\akehys dx}=\frac{d}{dx}f=\dif f(x). \quad}
\]
Kahdessa viimeksi kirjoitetussa merkinnässä tulkitaan derivaatta kohteestaan erillisenä
\kor{operaattorina} eli 'funktion funktiona':
\[
\dif:f \map f', \quad \dif = \frac{d}{dx}\,.
\]
Operaattoreiksi sanotaan yleisemmin sellaisia funktioita, joiden sekä määrittely- että 
maalijoukkona ovat (esim.\ reaalimuuttujan) funktiot. Derivoinnin suorittavaa
'funktion funktiota' $\dif$ sanotaan
\index{differentiaalioperaattori!a@derivoinnin}%
\kor{differentiaalioperaattoriksi}.

Differentiaalioperaattori voi määritelmänsä mukaisesti toimia vain sellaisissa pisteissä,
joissa $f$ on derivoituva, ts. $f'$:n määrittelyjoukossa
\[
\DF_{f'}=\{x\in\DF_f \ | \ f \text{ derivoituva pisteessä } x\}.
\]
Useat tavalliset funktiot ovat derivoituvia 'melkein kaikkialla' niin, että $\DF_{f'}$
saadaan $\DF_f$:stä poistamalla enintään äärellinen tai numeroituva määrä pisteitä. 
\begin{Exa} Funktio $f(x)=x-[x]$, missä $[x]=$ suurin kokonaisluku, jolle pätee $[x] \le x$,
on derivoituva muualla paitsi pisteissä $k \in \Z$, eli $D_{f'}=\{x\in\R \mid x\not\in\Z\}$. 
Tässä joukossa on $f'(x)=1$. (Vrt.\ Esimerkki \ref{sahafunktio} edellisessä luvussa.) \loppu
\end{Exa}

Jos $f$:n derivaatta $f'$ on edelleen derivoituva pisteessä $x$, voidaan määritellä $f$:n 
\kor{toinen derivaatta} pisteessä $x$:
\[
f''(x)=\lim_{\Delta x\kohti 0} \frac{f'(x+\Delta x)-f'(x)}{\Delta x} = \dif f'(x)
      = \dif(\dif f(x)) = \dif^2 f(x).
\]
Pisteissä, joissa $f''$ on edelleen on derivoituva, voidaan määritellä kolmas derivaatta
$f'''(x)$ jne. Yleisesti funktion \kor{$n$:s derivaatta} pisteessä $x$ on
\[
f^{(n)}=\frac{d^n}{dx^n}f(x)=\dif^n f(x).
\]
\index{kertaluku!ba@derivaatan}%
Sanotaan, että $n$ on derivaatan $f^{(n)}$ \kor{kertaluku}, ja sovitaan, että $f^{(0)}=f$.

\subsection{Derivoimissäännöt}
\index{derivoimissäännöt!a@summa, tulo, osamäärä|vahv}

Yleisissä \kor{derivoimissäännöissä} esitetään laskukaavat funktioiden algebrallisten
johdannaisten 
\[
f,g,\lambda\ \map\ \lambda f,\ f+g,\ fg,\ f/g,\ f \circ g,\ f^{-1}
\]
derivaattojen laskemiseksi $f'$:n ja $g'$:n avulla. Derivoimissäännöistä yksinkertaisimmat
ovat
\begin{align*}
\dif(\lambda f) &= \lambda f', \quad \lambda\in\R, \\
\dif(f+g)       &= f'+g',
\end{align*}
jotka voidaan yhdistää säännöksi
\begin{equation} \label{D1}
\boxed{\kehys\quad \dif(\alpha f + \beta g)=\alpha \dif f + \beta \dif g, \quad 
                   \alpha,\beta\in\R. \quad}
\end{equation}
\index{lineaarisuus!a@derivoinnin}%
Sääntö \eqref{D1} merkitsee, että $\dif$ on \kor{lineaarinen} operaattori. Sääntö on pätevä
jokaisessa pisteessä, jossa $f$ ja $g$ ovat molemmat derivoituvia, kuten voidaan helposti
todeta raja-arvojen yhdistelysääntöjen (Lause \ref{funktion raja-arvojen yhdistelysäännöt})
perusteella. Samoin ehdoin pätee tulon derivoimissääntö
\begin{equation} \label{D2}
\boxed{\kehys\quad \dif(fg)=f'g+fg'. \quad}
\end{equation}
Tämän perustelemiseksi kirjoitetaan
\begin{multline*}
f(x+\Delta x)g(x+\Delta x)-f(x)g(x) \\
              =[f(x+\Delta x)-f(x)]g(x+\Delta x)+f(x)[g(x+\Delta x)-g(x)],
\end{multline*}
jaetaan puolittain $\Delta x$:llä, ja sovelletaan mainittuja raja-arvojen 
yhdistelysääntöjä. Huomioidaan myös, että $g$ on (derivoituvana) jatkuva $x$:ssä. 
\begin{Exa} \label{potderiv 1} Lähtemällä ilmeisestä derivoimissäännöstä $\dif x = 1$ ja 
soveltamalla sääntöä \eqref{D2} funktiojonoon
$f_1(x) = x,\ f_n(x)=xf_{n-1}(x),\ n = 2,3,\ldots\,$ nähdään induktiolla oikeaksi sääntö 
\[
\dif x^m=mx^{m-1}, \quad m\in\N\cup\{0\}.
\]
Toinen tapa johtaa tämä tulos on lähteä suoraan derivaatan määritelmästä ja käyttää
binomikaavaan perustuvaa hajotelmaa:
\begin{align*}
\frac{(x+\Delta x)^m-x^m}{\Delta x}\,=\ 
                      &mx^{m-1}+\binom{m}{2}x^{m-2}\Delta x + \ldots +(\Delta x)^{m-1} \\
              \kohti\ &mx^{m-1}, \quad \text{kun}\ \Delta x \kohti 0. \loppu
\end{align*}
\end{Exa}
Kun esimerkin tuloksen ohella huomioidaan myös sääntö \eqref{D1}, nähdään että polynomin
derivaatta on toinen polynomi (astetta alempi).
\begin{Exa}
\begin{align*}
f(x)       &=x^5-4x^4+6x^3+x^2-2x+3 \\
f'(x)      &=5x^4-16x^3+18x^2+2x-2 \\
f''(x)     &=20x^3-48x^2+36x+2 \\
f'''(x)    &=60x^2-96x+36 \\
f^{(4)}(x) &=120x-96 \\
f^{(5)}(x) &=120 \\
f^{(n)}(x) &=0, \quad n\geq 6 \loppu
\end{align*}
\end{Exa}

Jos $f$ ja $g$ ovat $n$ kertaa derivoituvia pisteessä $x$, niin soveltamalla säännön
\eqref{D2} oikealla puolella uudelleen samaa sääntöä yhdessä säännön \eqref{D1} kanssa
saadaan johdetuksi binomikaavaa (Propositio \ref{binomikaava}) muistuttava
\kor{Leibnizin sääntö} \index{Leibnizin sääntö}
\[
\boxed{\quad \dif^n(fg) = \sum_{k=0}^n \binom{\ykehys n}{\akehys k}\,f^{(k)}g^{(n-k)}. \quad}
\]

Jos $\,h(x)=(f/g)(x)$, niin tulon derivoimissääntö \eqref{D2} sovellettuna identiteettiin
$h(x)g(x)=f(x)\,$ antaa tuloksen $h'g+hg'=f'$. Pisteissä, joissa $g(x) \neq 0$, voidaan tästä
ratkaista $h$:n eli osamäärän derivoimissäännöksi
\begin{equation} \label{D3}
\boxed{\quad \dif\left(\frac{f}{g}\right)
           =\frac{\ykehys f'}{\akehys g}-\frac{fg'}{g^2}=\frac{f'g-fg'}{g^2}\,. \quad}
\end{equation}
Säännöstä \eqref{D3} sekä polynomin derivoimissäännöstä nähdään, että jokainen
rationaalifunktio $f=p/q$ ($p$ ja $q$ polynomeja) on koko määrittelyjoukossaan, eli joukossa
$\{x\in\R \mid q(x) \neq 0\}$, derivoituva mielivaltaisen monta kertaa ja että derivaatat ovat
ovat samassa joukossa määriteltyjä rationaalifunktioita.
\begin{Exa} \label{ratfunktion derivaatta} Säännön \eqref{D3} ensimmäisen laskukaavan mukaan
\[
\frac{d}{dx}\left(\frac{x}{1-x^2}\right) = \frac{1}{1-x^2}+\frac{2x^2}{(1-x^2)^2}
                                         = \frac{1+x^2}{(1-x^2)^2}\,. \loppu
\]
\end{Exa}
\begin{Exa} \label{potder 2} Soveltamalla sääntöä \eqref{D3} tapaukseen $f(x)=1$,
$g(x)=x^m,\ m\in\N$ nähdään, että Esimerkin \ref{potderiv 1} sääntö $\dif x^m = mx^{m-1}$ on
pätevä $\forall m\in\Z$. \loppu
\end{Exa}
\begin{Lause} (\vahv{Yhdistetyn funktion derivaatta}) \label{yhdistetyn funktion derivaatta}
\index{derivoimissäännöt!b@yhdistetty funktio|emph}% 
Jos $g$ on derivoituva pisteessä $x$ ja $f$ on derivoituva pisteessä $g(x)$, niin $f \circ g$
on derivoituva pisteessä $x$ ja
\[
\boxed{\kehys\quad \dif f(g(x))=f'(g(x))\,g'(x). \quad}
\]
\end{Lause}
\tod Merkitään $g(x+\Delta x)-g(x)=\Delta g$. Tällöin koska $f$ on derivoituva
pisteessä $g(x)$, niin pätee (vrt.\ derivaatan määritelmä edellä)
\[
f(g(x+\Delta x)) = f(g(x)+\Delta g) = f(g(x))+f'(g(x))\Delta g + h(\Delta g)\Delta g,
\]
missä $h(\Delta g) \kohti 0$, kun $\Delta g \kohti 0$. Koska $g$ on derivoituva pisteessä
$x$ ja koska $\,\Delta x \kohti 0\ \impl\ \Delta g \kohti 0$, niin seuraa
\begin{align*}
\dif (f\circ g)(x) &= \lim_{\Delta x \kohti 0}\,\frac{f(g(x+\Delta x))-f(g(x))}{\Delta x} \\
                   &= \lim_{\Delta x \kohti 0} \left[f'(g(x))\frac{\Delta g}{\Delta x} 
                                     + h(\Delta g)\frac{\Delta g}{\Delta x}\right] \\[2mm]
                   &= f'(g(x))g'(x) + 0 \cdot g'(x) = f'(g(x))g'(x). \loppu
\end{align*}

Soveltamalla Lauseen \ref{yhdistetyn funktion derivaatta} sääntöä ja Esimerkin \ref{potder 2}
tulosta funktioon $g\circ f$, missä $g(x)=x^m$, saadaan derivoimissääntö
\begin{equation} \label{D4}
\boxed{\kehys\quad \dif[f(x)]^m=m[f(x)]^{m-1}f'(x),\quad m\in\Z. \quad}
\end{equation}
\begin{Exa} Jatkamalla Esimerkistä \ref{ratfunktion derivaatta} säännöillä \eqref{D2},
\eqref{D4} ja \eqref{D1} saadaan
\begin{align*}
\frac{d^2}{dx^2}\left(\frac{x}{1-x^2}\right)=\frac{d}{dx}\left(\frac{1+x^2}{(1-x^2)^2}\right) 
                   &=\frac{2x}{(1-x^2)^2}+(1+x^2)\,\frac{(-2)(-2x)}{(1-x^2)^3} \\
                   &=\frac{6x+2x^3}{(1-x^2)^3}\,. \loppu
\end{align*}
\end{Exa}
Jos $f(x)=x^{m/n}$, $m\in\Z$, $n\in\N$, $x>0$, niin $\,[f(x)]^n=x^m$. Derivoimalla tässä
puolittain säännöllä \eqref{D4} saadaan
\[
n[f(x)]^{n-1}f'(x) = mx^{m-1} \qimpl f'(x) = (m/n)\,x^{m/n-1}.
\]
Näin on johdettu yleisen potenssifunktion derivoimissääntö
\[ 
\boxed{\kehys\quad \dif x^\alpha = \alpha x^{\alpha-1}, \quad x\in\R_+\,, \ \alpha\in\Q. \quad}
\]

\begin{Exa} Derivoi $\D f(x) = \sqrt[3]{1+\sqrt{x}}$. \end{Exa}
\ratk Potenssifunktion ja yhdistetyn funktion derivoimissääntöjen perusteella
\begin{align*}
f'(x) = \frac{1}{3}(1+\sqrt{x})^{-2/3}\cdot\frac{1}{2}\,x^{-1/2}
      &=\,\frac{(1+\sqrt{x})^{1/3}}{6\sqrt{x}\,(1+\sqrt{x})} \\
      &=\,\frac{\sqrt[3]{1+\sqrt{x}}}{6(x+\sqrt{x})}\,, \quad x>0. \loppu
\end{align*}

Yhdistetyn funktion derivoimissäännöstä (Lause \ref{yhdistetyn funktion derivaatta}) voi
johtaa myös yleisen käänteisfunktion derivoimissäännön: Oletetaan, että $f$ on derivoituva
pisteessä $x$ ja että käänteisfunktio $g=\inv{f}$ on derivoituva pisteessä $f(x)$. Tällöin
identiteetistä $\,g(f(x))=x\,$ seuraa derivoimalla puolittain, että $\ g'(f(x))f'(x)=1$.
Siis tehdyin oletuksin ja lisäoletuksella $f'(x) \neq 0$ pätee
\begin{equation} \label{D5}
\boxed{\quad \dif\inv{f}(y)=\frac{1}{f'(x)},\quad y=f(x). \quad}
\end{equation}
Käänteisfunktion derivoituvuuden perustelee tarkemmin seuraava lause.
\begin{Lause} (\vahv{Käänteisfunktion derivaatta}) \label{käänteisfunktion derivoituvuus}
\index{derivoimissäännöt!c@käänteisfunktio|emph}%
Olkoon $f: [a-\delta,a+\delta] \kohti U$ jatkuva bijektio jollakin $\delta>0$ ja olkoon
$f$ derivoituvua pisteessä $a$. Tällöin $\inv{f}: U \kohti [a-\delta,a+\delta]$ on
derivoituva pisteessä $b=f(a)$ täsmälleen kun $f'(a) \neq 0$, jolloin 
$\,\dif \inv{f}(b)=1/f'(a)$.
\end{Lause}
\tod Koska $f$ on jatkuva ja $1-1$ välillä $[a-\delta,a+\delta]$ ja $f$:n arvojoukko
ko.\ valillä $=U$, niin Lauseen \ref{ensimmäinen väliarvolause} mukaan pätee
$[b-\eps,b+\eps] \subset U$, missä
\[
\eps \,=\, \min\{\,\abs{f(a-\delta)-f(a)},\,\abs{f(a+\delta)-f(a)}\,\} \,>\, 0.
\]
Tällöin jos $f'(a) \neq 0$, niin tekemällä välillä $[b-\eps,b+\eps]$ muuttujan
vaihto $y=f(x)$ seuraa lauseiden \ref{raja-arvo sijoituksella} ja
\ref{funktion raja-arvojen yhdistelysäännöt} perusteella
\[
\lim_{y \kohti b} \frac{\invf(y)-\invf(b)}{y-b} 
           \,=\,\lim_{x \kohti a} \frac{x-a}{f(x)-f(a)}
           \,=\,\left[\lim_{x \kohti a} \frac{f(x)-f(a)}{x-a}\right]^{-1} 
           \,=\, \frac{1}{f'(a)}\,,
\]
joten $\inv{f}$ on derivoituva $b$:ssä ja $\dif\inv{f}(b)=1/f'(a)$ (Määritelmä
\ref{derivaatan määritelmä}). Jos $f'(a)=0$, seuraa Lauseesta \ref{raja-arvo sijoituksella}
vastaavasti, että
\[
\lim_{y \kohti b} \left|\frac{\invf(y)-\invf(b)}{y-b}\right|
            \,=\, \lim_{x \kohti a} \left|\frac{x-a}{f(x)-f(a)}\right|
            \,=\ \infty,
\]
joten tässä tapauksessa $\invf$ ei ole derivoituva $b$:ssä. \loppu
\begin{Exa} \label{algebrallinen käänteisfunktio: derivaatta} Funktiolle $f(x)=x^5+3x$ 
Lauseen \ref{käänteisfunktion derivoituvuus} oletukset ovat voimassa jokaisella $a\in\R$
(myös jokaisella $\delta>0$, ks.\ Esimerkki \ref{algebrallinen käänteisfunktio} Luvussa
\ref{käänteisfunktio}), joten sääntö \eqref{D5} on soveltuva: 
\[
\dif\inv{f}(y)=\frac{1}{5x^4+3}, \quad y=x^5+3x.
\]

Esimerkiksi $\dif\inv{f}(0)=1/3$, koska $y=0 \impl x=0$. Jos halutaan laskea 
$\dif\inv{f}(1)$, on ensin ratkaistava (numeerisesti) yhtälö $x^5+3x=1$. Jos ratkaisu
merkitään $x=a$, niin $\dif\inv{f}(1)=1/(5a^4+3)$. \loppu 
\end{Exa}

\subsection{Implisiittinen derivointi}
\index{implisiittinen derivointi|vahv} \index{derivoimissäännöt!d@implisiittifunktio|vahv}

Jos funktio $y(x)$ on määritelty implisiittisesti muodossa (vrt.\ Luku \ref{käänteisfunktio})
\[ 
F(x,y) = 0 \qekv y = y(x), 
\]
niin derivaatta $y'(x)$ on mahdollista laskea derivoimalla puolittain yhtälö
\[ 
F(x,y(x)) = 0.
\]
Tällaista epäsuoraa menettelyä sanotaan \kor{implisiittiseksi derivoinniksi}. Toistaiseksi
tunnetuilla derivointisäännöillä implisiittinen derivointi onnistuu esim.\ silloin kun
funktio F on (kahden muuttujan) polynomi tai rationaalifunktio.
\jatko \begin{Exa} (jatko) Jos esimerkissä merkitään $\inv{f}(x)=y(x)$, niin $y(x)$ voidaan
tulkita implisiittifunktioksi, joka määräytyy yhtälöstä $\,y^5+3y=x$, eli
\[
[y(x)]^5+3y(x)=x, \quad x\in\R.
\]
Kun tässä derivoidaan puolittain soveltaen yhdistetyn funktion derivoimissääntöä ja sääntöä
\eqref{D1}, niin seuraa
\[ 
\bigl[5[y(x)]^4+3\bigr]\,y'(x)=1\ \impl\ \ y'(x)=\frac{1}{5y^4+3} \quad (y=y(x)). \loppu
\]
\end{Exa}
\begin{Exa} Yhtälö 
\[ 
x^2 + y^2 = 1 
\]
määrittelee välillä $[-1,1]$ kaksihaaraisen implisiittifunktion, jonka kumpikin haara on
derivoituva välillä $(-1,1)$. Jos $x\in(-1,1)$, niin $y'(x)$ lasketaan helposti
implisiittisellä dervoinnilla:
\begin{align*}
 \dif[x^2 + (y(x))^2] &= 2x + 2y(x)y'(x) = \dif\,1 = 0
                      &\impl \quad y'(x) = -\,\dfrac{x}{y}\,,\ \ y=y(x).
\end{align*}
Tulos on pätevä kummallakin implisiittifunktion haaralla
\[ 
y(x) = \pm \sqrt{1-x^2}\,, 
\]
kuten myös suoralla derivoinnilla voi todeta:
\[
y(x) = \pm\sqrt{1-x^2}\,\ \impl\,\ y'(x) = \pm\left(\frac{-x}{\sqrt{1-x^2}}\right) 
                                         = -\,\frac{x}{y(x)}\,, \quad x \in (-1,1). \loppu
\]
\end{Exa}

\subsection{Potenssisarjan derivointi}

Tähänastisten tarkastelujen perusteella voidaan pitää pääsääntönä, että tavanomaiset
(algebran keinoin lausekkeina määritellyt) funktiot ovat derivoituvia määrittelyjoukossaan,
lukuunottamatta mahdollisia erillisiä pisteitä, joita voi olla äärellinen tai joskus
numeroituva määrä. Näytetään luvun lopuksi, että pääsäännöstä eivät tee poikkeusta myöskään
funktiot, jotka on määritelty potenssisarjojen summina. Tällaisten funktioiden myötä
derivoituvien funktioiden joukko laajenee itse asiassa merkittävästi, kuten myöhemmin tullaan
näkemään.

Olkoon reaalifunktio $f$ määritelty muodossa
\[
f(x) = \sum_{k=0}^\infty a_k x^k.
\]
Oletetaan, että tässä potenssisarjan suppenemissäde on joko $\rho\in\R_+$ tai $\rho=\infty$.
Tällöin suurin avoin väli, joka sisältyy $f$:n määrittelyjoukkoon, on $(-\rho,\rho)$
(vrt.\ Luku \ref{potenssisarja}). Jatkossa näytetään, että $f$ on derivoituva koko tällä
välillä ja että derivaatta voidaan laskea yksinkertaisesti derivoimalla sarja termeittäin,
ts.\ pätee
\[
f'(x) = \sum_{k=1}^\infty k a_k x^{k-1}, \quad x \in (-\rho,\rho).
\]
Derivoinnin tuloksena saadun sarjan suppenemissäde on myös $\rho$ (Lause
\ref{potenssisarjan skaalaus}), joten $f'$ on edelleen derivoituva välillä $(-\rho,\rho)$,
ja derivoinnin tulos on siis
\[
f''(x) = \sum_{k=2}^\infty k(k-1)\,a_k x^{k-2}, \quad x \in (-\rho,\rho).
\]
Myös tämän sarjan suppenemissäde $=\rho$ (Lause \ref{potenssisarjan skaalaus}),
joten $f''$ on välillä $(-\rho,\rho)$ edelleen derivoituva, jne. Päätellään, että
potenssisarjan summana määritelty funktio on avoimella suppenemisvälillään itse asiassa
mielivaltaisen monta kertaa derivoituva funktio (!). Päätelmä perustui siis potenssisarjojen
suppenemisteoriaan ja seuraavaan väittämään, joka on todistettavissa suoraan derivaatan
määritelmästä lähtien (eli aiempiin derivoimissääntöihin vetoamatta).
\begin{Lause} \label{potenssisarja on derivoituva} \vahv{(Potenssisarjan derivaatta)}\,
\index{potenssisarja!b@derivointi|emph} \index{derivoimissäännöt!e@potenssisarja|emph}%
Jos potenssisarjan $\sum_{k=0}^\infty a_k x^k$ suppenemissäde on $\rho>0$, niin sarjan summana 
määritelty funktio $f(x)$ on derivoituva välillä $(-\rho,\rho)$ ja ko.\ välillä pätee 
$f'(x)=\sum_{k=1}^\infty k a_k x^{k-1}$. 
\end{Lause}
\tod Olkoon $x\in (-\rho,\rho)$ ja valitaan $\Delta x \neq 0$ siten, että
\[
\abs{x}+\abs{\Delta x} \le \rho_0<\rho.
\]
Binomikaavan mukaan
\begin{align*}
(x+\Delta x)^k\ &=\ \sum_{l=0}^k \binom{k}{l} (\Delta x)^l x^{k-l} \\
                &=\ x^k + kx^{k-1}\Delta x + \sum_{l=2}^k \binom{k}{l} (\Delta x)^l x^{k-l}.
\end{align*}
Kirjoitetaan tässä viimeinen termi summausindeksin vaihdolla muotoon
\begin{align*}
\sum_{l=2}^k \binom{k}{l} (\Delta x)^l x^{k-l}\ 
              &= \sum_{i=0}^{k-2} \binom{k}{i+2} (\Delta x)^{i+2} x^{k-2-i} \\
              &=\ (\Delta x)^2 \sum_{i=0}^{k-2} \binom{k}{i+2} (\Delta x)^i x^{k-2-i},
\end{align*}
ja edelleen
\[
\sum_{i=0}^{k-2} \binom{k}{i+2} (\Delta x)^i\,x^{k-2-i} 
                = \sum_{i=0}^{k-2} c_i\,\binom{k-2}{i} (\Delta x)^i\,x^{k-2-i},
\]
missä
\[
c_i = \binom{k}{i+2} \binom{k-2}{i}^{-1} 
    = \frac{k!}{(k-i-2)!\,(i+2)!} \cdot \frac{(k-i-2)!\,i!}{(k-2)!} 
    = \frac{k(k-1)}{(i+1)(i+2)}\,.
\]
Koska
\[
c_i \le \frac{1}{2}\,k(k-1) < \frac{1}{2}\,k^2, \quad i = 0 \ldots k-2,
\]
ja oli $\abs{x}+\abs{\Delta x} \le \rho_0$, niin saadaan jokaisella $k \ge 2$ arvio
\begin{align*}
\left|\frac{(x+\Delta x)^k - x^k}{\Delta x} - kx^{k-1}\right|\ 
 &=\ \abs{\Delta x} \left|\sum_{i=0}^{k-2} c_i\,\binom{k-2}{i} (\Delta x)^i\,x^{k-2-i}\right| \\
 &\le\ \abs{\Delta x} \sum_{i=0}^{k-2} c_i\,\binom{k-2}{i} \abs{\Delta x}^i\,\abs{x}^{k-2-i} \\
 &\le\ \frac{1}{2}k^2\abs{\Delta x} 
               \sum_{i=0}^{k-2} \binom{k-2}{i} \abs{\Delta x}^i\,\abs{x}^{k-2-i} \\
 &= \frac{1}{2}k^2\abs{\Delta x}(\abs{x}+\abs{\Delta x})^{k-2} \\
 &\le \frac{1}{2}k^2\abs{\Delta x}\rho_0^{k-2}.
\end{align*}
Näin ollen
\begin{align*}
\left| \frac{f(x+\Delta x)-f(x)}{\Delta x} - \sum_{k=1}^\infty ka_kx^{k-1} \right|\ 
       &=\ \left|\sum_{k=2}^\infty a_k
                 \left[\frac{(x+\Delta x)^k - x^k}{\Delta x} - kx^{k-1}\right]\right| \\
       &\le\ \sum_{k=2}^\infty \abs{a_k}
             \left|\frac{(x+\Delta x)^k - x^k}{\Delta x} - kx^{k-1}\right| \\
       &\le\ \abs{\Delta x}\sum_{k=2}^\infty \frac{1}{2}k^2\abs{a_k}\rho_0^{k-2}.
\end{align*}
Tässä oikealla oleva sarja suppenee Lauseen \ref{potenssisarjan skaalaus} perusteella, koska
$\rho_0<\rho$, joten
\[
\left|\frac{f(x+\Delta x)-f(x)}{\Delta x} - \sum_{k=1}^\infty ka_kx^{k-1}\right| 
       \le\ C\,\abs{\Delta x}, \quad\ C = \sum_{k=2}^\infty \frac{1}{2}k^2\abs{a_k}\rho_0^{k-2}.
\]
Saatu arvio on pätevä, kun $x \in (-\rho,\rho)$ ja $\abs{\Delta x} < \delta$, missä esim.\ 
\[ 
\delta = \frac{1}{2}\,(\rho-\abs{x}) > 0. 
\]
Siis $f$ on jokaisessa pisteessä $x \in (-\rho,\rho)$ derivoituva ja
\[
f'(x) = \lim_{\Delta x \kohti 0}\,\frac{f(x+\Delta x)-f(x)}{\Delta x}\ 
      =\ \sum_{k=1}^\infty ka_kx^{k-1}. \loppu
\]
\begin{Exa} Laske $\,\sum_{k=1}^\infty k q^k$, kun $\abs{q}<1$. \end{Exa}
\ratk Derivoidaan funktio
\[ 
f(x) = \sum_{k=0}^\infty x^k = \frac{1}{1-x}\,, \quad x \in (-1,1) 
\]
toisaalta Lauseen \ref{potenssisarja on derivoituva} perusteella ja toisaalta 
rationaalifunktiona:
\[ 
f'(x) \,=\, \sum_{k=1}^\infty kx^{k-1} \,=\, \frac{1}{(1-x)^2}\,, \quad x\in(-1,1). 
\]
Valitaan $x=q$ ja kerrotaan puolittain $q$:lla:
\[
q\sum_{k=1}^\infty kq^{k-1} \,=\, \sum_{k=1}^\infty kq^k \,=\, \frac{q}{(1-q)^2}\,. \loppu
\]

\Harj
\begin{enumerate}

\item
Laske suoraan määritelmästä (derivoimissääntöjä käyttämättä) linearisoiva approksimaatio
seuraaville funktioille pisteessä $a=1$. Piirrä myös kuva! 
\[
\text{a)}\,\ f(x)=\frac{x}{1+2x} \qquad 
\text{b)}\,\ f(x)=\frac{x}{2+x^2} \qquad
\text{c)}\,\ f(x)=\sqrt{10+6x}
\]

\item
Laske seuraavien funktioiden derivaatat pisteessä $a\in\DF_f$ suoraan derivaatan
määritelmästä:
\[
\text{a)}\,\ f(x)=x^3+3x^2 \qquad 
\text{b)}\,\ f(x)=\frac{1}{x} \qquad
\text{c)}\,\ f(x)=\frac{1}{\sqrt{x}}
\]

\item
Olkoon $f$ derivoituva pisteessä $a$. Määritä seuraavat raja-arvot $f(a)$:n ja $f'(a)$:n
avulla:
\begin{align*}
&\text{a)}\ \lim_{h \kohti 0} \frac{f(a+h^2)-f(a-h)}{h} \qquad
 \text{b)}\ \lim_{x \kohti a} \frac{xf(a)-af(x)}{x-a} \\
&\text{c)}\ \lim_{t \kohti 0^+} \frac{f(a+\alpha t)-f(a+\beta t)}{t}\,,\,\ \alpha,\beta\in\R
\end{align*}

\item \label{H-V-4: trig-esim}
Lähtien derivaatan määritelmästä määritä $f'(0)$ tai näytä, että $f$ ei ole derivoituva
$0$:ssa:
\begin{align*}
&\text{a)}\,\ f(x)=\begin{cases}
                   \,x(1+\sqrt{\abs{x}}\cos\dfrac{1}{x}, &\text{kun}\ x \neq 0, \\[2mm] 
                   \,0,                                  &\text{kun}\ x=0
                   \end{cases} \\[2mm]
&\text{b)}\,\ f(x)=\begin{cases}
                   \,x\sin^2\dfrac{1}{x}\,, &\text{kun}\ x \neq 0, \\[2mm] 
                   \,0,                     &\text{kun}\ x=0
                   \end{cases} \\[2mm]
&\text{c)}\,\ f(x)=\begin{cases}
                   \,1+x\cos\dfrac{1}{x}\sqrt[100]{x\left|\sin\dfrac{1}{x}\right|},
                                            &\text{kun}\ x>0, \\[2mm]
                   \,1+x^2,                 &\text{kun}\ x \le 0
                   \end{cases}
\end{align*}

\item
Palauta osamäärän derivoimissääntö tulon derivoimissääntöön johtamalla ensin funktion
$1/f$ derivoimissääntö suoraan derivaatan määritelmästä.

\item
Laske kohdissa a)--c) $f^{(n)}(x)$ kun $n=1,2$ ja kohdissa d)--f) derivaatan
$f^{(n)}(x),\ n\in\N$ yleinen lauseke:
\begin{align*}
&\text{a)}\,\ f(x)=\frac{x}{\sqrt{a^2-x^2}} \qquad 
 \text{b)}\,\ f(x)=\frac{1}{\sqrt{x}+1} \qquad
 \text{c)}\,\ f(x)=\sqrt{x+\sqrt{x}} \\
&\text{d)}\,\ f(x)=\frac{1}{1+x} \qquad\quad\ \ 
 \text{e)}\,\ f(x)=\sqrt{x+1} \qquad\
 \text{f)}\,\ f(x)=\frac{1-x}{1+x}
\end{align*}

\item
Tiedetään, että $f(2)=1$, $f'(2)=3$ ja $f''(2)=-1$. Laske
\[
\text{a)}\ \left[\frac{d}{dx}\left(\frac{x^2}{f(x)}\right)\right]_{x=2} \qquad
\text{b)}\ \left[\frac{d}{dx}\left(\frac{f(x)}{x^2}\right)\right]_{x=2} \qquad
\text{c)}\ \left[\frac{d^2}{dx^2}\left(\frac{f(x)}{x^2}\right)\right]_{x=2}
\]

\item
Funktio $f$ määritellän funktion $g(x)=x^2-3x+4$ kaksihaaraisena käänteisfunktiona. Laske
$f'(2)$ molemmilla haaroilla \ a) ratkaisemalla ensin yhtälö $g(x)=2$ ja käyttämällä
käänteisfunktion derivoimissääntöä, \ b) ratkaisemalla $\,g(y)=x\,\impl\,y=f(x)$,
derivoimalla ja sijoittamalla $x=2$. 

\item
Seuraavat yhtälöt määrittelevät implisiittifunktion $y(x)$. Laske implisiittisellä
derivoinnilla $y'(x)$ annetussa pisteessä $(x,y)=(x,y(x))$\,: \newline
a) \ $2x^2+3y^2=5,\ (x,y)=(1,1)$ \newline
b) \ $2x^2+3y^2=5,\ (x,y)=(1,-1)$ \newline
c) \ $x^2y^3-x^3y^2=12,\ (x,y)=(-1,2)$ \newline
d) \ $(x-1)(x+2y+1)=y^2,\ (x,y)=(2,-1)$

\item
Funktio $g$ on funktion $f$ käänteisfunktio eli $y=f(x)\,\ekv\, x=g(y)$. Johda
derivoimissääntö
\[
g''(y)=-\frac{f''(x)}{[f'(x)]^3}\,, \quad y=f(x)
\]
ja laske säännöllä $g''(1)$, kun $f(x)=x^5+2x+1$.

\item
Laske implisiittisesti derivoimalla $y''$ $x$:n ja $y$:n avulla:
\[
\text{a)}\,\ xy=x+y \qquad \text{b)}\,\ x^2+y^2=1 \qquad \text{c)}\,\ x^3-y^2+y^3=x
\]

\item \label{H-V-3: cos ja sin potenssisarjoina}
a) Näytä, että $\,xy''=y$, kun määritellään
\[
y(x)=\sum_{k=1}^\infty \frac{x^k}{k!(k-1)!}\,, \quad x\in\R.
\]
b) Näytä, että $\,u'=-v$ ja $\,v'=u$, kun määritellään
\[
u(x)=\sum_{k=0}^\infty (-1)^k \frac{x^{2k}}{(2k)!}\,, \quad
v(x)=\sum_{k=0}^\infty (-1)^k \frac{x^{2k+1}}{(2k+1)!}\,, \quad x\in\R.
\]
 
\item
Seuraavat funktiot ovat rationaalifunktioita välillä $(-1,1)$. Laske funktioiden lausekkeet
potenssisarjaa $\sum_{k=0}^\infty x^k$ derivoimalla.
\begin{align*}
&\text{a)}\ f(x)=\sum_{k=2}^\infty kx^k \qquad 
 \text{b)}\ f(x)=\sum_{k=3}^\infty (-1)^k kx^k \qquad
 \text{c)}\ f(x)=\sum_{k=1}^\infty k^2 x^k \\
&\text{d)}\ f(x)=\sum_{k=0}^\infty (k+2)^2 x^k \quad\ \
 \text{e)}\ f(x)=\sum_{k=1}^\infty k^3 x^k \quad\ \
 \text{f)}\ f(x)=\sum_{k=0}^\infty (k+1)^3 x^k 
\end{align*}

\item (*)
Todista Leibnizin sääntö.

\item (*) 
Olkoon $a\in\R$ reaalikertoimisen polynomin $p$ nollakohta ja olkoon $m\in\N,\ m\ge 2$.
Näytä, että $a$ on $m$-kertainen nollakohta täsmälleen kun $p^{(k)}(a)=0,\ k=1 \ldots m-1$
ja $p^{(m)}(a) \neq 0$.

\item (*)
a) Näytä, että jos potenssisarja $\sum_k a_k x^k$ suppenee välillä $(-\rho,\rho)$ ja
funktio $g$ on derivoituva välillä $(a,b)\subset(-\rho,\rho)$, niin funktion
\[
f(x) = \sum_{k=0}^\infty x^k g(x), \quad x\in(a,b)
\]
derivaatta välillä $(a,b)$ on laskettavissa derivoimalla sarja
termeittäin.\vspace{1mm}\newline
b) Määritä $f'(x)$ ja $f''(x)$ välillä $(0,\infty)$, kun
\[
f(x) = \sum_{k=0}^\infty \frac{1}{k!}\,x^{k+\tfrac{1}{2}}, \quad x \ge 0.
\]

\end{enumerate}