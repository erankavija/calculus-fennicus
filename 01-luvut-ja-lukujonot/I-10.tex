\section{Cauchyn jonot} \label{Cauchyn jonot}
\alku

Tässä luvussa 'lukujono' tarkoittaa reaalilukujen jonoa. Tarkastelun kohteena on lukujonojen
(myös rationaalilukujonojen) teorian edelleen avoin kysymys, joka kuuluu:
Täsmälleen millaisilla, jonoa itseään koskevilla ehdoilla lukujono $\seq{a_n}$ suppenee, ts.\
raja-arvo $\lim_na_n=a$ on olemassa Määritelmän \ref{jonon raja} mukaisesti reaalilukuna?
Toistaiseksi tunnetut suppenemisen ehdot Lauseissa \ref{monotoninen ja rajoitettu jono} ja
\ref{monotoninen ja rajoitettu jono - yleistys} ovat riittäviä, eivät välttämättömiä.
Etsittäessä suppenemiskysymyksen täydellista ratkaisua, ts.\ sekä välttämättömiä että riittäviä
ehtoja suppnemiselle, ratkaisun avaimeksi osoittautuu
\index{Cauchyn!a@jono|emph} \index{lukujono!f@Cauchyn jono|emph}
\begin{Def} \label{Cauchyn jono} (\vahv{Cauchyn}\footnote[2]{Ranskalainen matemaatikko 
\hist{Augustin Louis Cauchy} (1789-1857) kuuluu matemaatikoiden tähtikaartiin kautta
aikojen. Cauchy täsmensi merkittävästi matematiikan käsitteistöä ja loi pohjaa uusille
tutkimussuunnille. Hänen laaja tuotantonsa ulottui myös fysiikkaan, mm.\ kiinteän aineen
kimmoteoriaan. \index{Cauchy, A. L.|av}} 
\vahv{jono}) Lukujono $\{a_n\}$ on \kor{Cauchy} eli \kor{Cauchyn jono}, jos jokaisella
$\eps > 0$ on olemassa $N\in\N$ siten, että pätee
\[
\abs{a_n - a_m}\ < \eps \quad \text{kun}\ \,n,m > N.
\]
\end{Def}

\begin{Exa} Näytä, että sarjan $\,\sum_{k=0}^\infty (-1)^k/(k+1)^2\,$ osasummien jono
\[
\seq{s_n} = \left\{1,\,1-\frac{1}{2^2},\,1-\frac{1}{2^2}+\frac{1}{3^2},\ldots\right\}
          = \left\{1,\,\frac{3}{4},\,\frac{7}{9},\ldots\right\}
\]
on Cauchyn jono. \end{Exa}
\ratk Jos $\,0 \le m < n$, niin kolmioepäyhtälöä soveltaen voidaan arvioida \newline
(ks.\ Harj.teht.\,\ref{monotoniset jonot}:\ref{H-I-8: kaksi sarjaa})
\[
\abs{s_n-s_m} \,=\, \left|\sum_{k=m+1}^n \frac{(-1)^k}{(k+1)^2}\right|
              \,\le\, \sum_{k=m+1}^n \frac{1}{(k+1)^2} \,<\, \frac{1}{m+1}\,.
\]
Näin ollen jos $\eps>0$ ja valitaan $N\in\N$ siten, että $1/(N+1)<\eps$ (mahdollista
jokaisella $\eps>0$), niin
\[
\abs{s_n-s_m} \,<\, \max\left\{\frac{1}{n+1},\,\frac{1}{m+1}\right\} \,<\, \eps,
                    \quad \text{kun}\ n,m >N.
\]
Siis $\seq{s_n}$ on Cauchy. \loppu
\begin{Exa} Jos $a \neq 0$, niin lukujono $\seq{a_n}=\seq{(-1)^n a}$ ei ole Cauchy.
Nimittäin jos $0 < \eps \le 2\abs{a}$, niin Määritelmän \ref{Cauchyn jono} ehto ei
toteudu millään $N\in\N$, koska $\abs{a_n - a_m}=2\abs{a} \ge \eps\,$ aina kun $n-m$
on pariton. \loppu \end{Exa}
 
Cauchyn jonoilla on samankaltaisia ominaisuuksia kuin suppenevilla jonoilla. Näytetään
ensinnäkin:
\begin{Lause} \label{suppeneva jono on Cauchy} Suppeneva jono on Cauchy. \end{Lause}
\tod Jos $\lim_na_n=a\in\R$ Määritelmän \ref{jonon raja} mukaisesti, niin jokaisella $\eps>0$
on olemassa $N\in\N$ siten, että pätee $\abs{a_n - a} < \eps/2\,$ kun $n>N$. Tällöin on
kolmioepäyhtälön nojalla
\begin{align*}
\abs{a_n - a_m}\ &=\ \abs{(a_n - a) + (a - a_m)} \\
                 &\le\ \abs{a_n-a}+\abs{a_m-a}\ <\ \frac{\eps}{2} + \frac{\eps}{2} = \eps, 
                                                   \quad \text{kun}\ n,m > N. \loppu 
\end{align*}%
Suppenevia lukujonoja koskevilla Lauseilla \ref{suppeneva jono on rajoitettu} ja 
\ref{raja-arvojen yhdistelysäännöt} on seuraavat vastineet Cauchyn jonoille:
\begin{Lause} \label{Cauchyn jono on rajoitettu} Cauchyn jono on rajoitettu.
\end{Lause}
\begin{Lause} \label{Cauchyn jonojen yhdistelysäännöt} 
\index{Cauchyn!b@jonojen yhdistely|emph} (\vahv{Cauchyn jonojen yhdistelysäännöt})
Jos lukujonot $\{a_n\}$ ja $\{b_n\}$ ovat Cauchyn jonoja ja $\lambda\in\R$, niin myös lukujonot
$\{a_n + b_n\}$, $\{\lambda a_n\}$ ja $\{a_n b_n\}$ ovat Cauchyn jonoja. Jos lisäksi 
$\abs{b_n} \ge \delta > 0\ \ \forall n$, niin myös $\{a_n/b_n\}$ on Cauchyn jono.
\end{Lause}
Näiden lauseiden todistukset (jotka sivuutetaan, ks.\ Harj.teht.\,\ref{H-I-9: todistuksia})
ovat hyvin samanlaisia kuin mainittujen vastinlauseiden. Esimerkiksi Lauseen 
\ref{Cauchyn jono on rajoitettu} väittämään päädytään, kun vertailukohdaksi otetaan raja-arvon
sijasta jonon termi $a_m$, missä indeksi $m$ valitaan siten, että pätee 
$\abs{a_n - a_m} < 1$ kun $n>m$, vrt.\ Määritelmä \ref{Cauchyn jono} ja Lauseen 
\ref{suppeneva jono on rajoitettu} todistus.

Päätulos jatkossa on
\begin{*Lause} \label{Cauchyn kriteeri} \index{Cauchyn!c@suppenemiskriteeri|emph}
\index{suppeneminen!a@lukujonon|emph} (\vahv{Cauchyn suppenemiskriteeri})
Reaalilukujono $\seq{a_n}$ suppenee kohti reaalilukua täsmälleen kun $\seq{a_n}$ on Cauchy. 
\end{*Lause}
\jatko\jatko \begin{Exa} (jatko) Lauseen \ref{Cauchyn kriteeri} perusteella esimerkin
sarja suppenee. Osasummia riitttävän pitkälle laskemalla selviää, että sarjan summa katkaistuna
$10$ merkitsevään numeroon on
\[
\sum_{k=0}^\infty \frac{(-1)^k}{(k+1)^2} \,=\, 0.8224670334\ldots \loppu
\]
\end{Exa} \seur
Koska jo tiedetään, että jokainen suppeneva jono on Cauchy (Lause 
\ref{suppeneva jono on Cauchy}), niin Lauseen \ref{Cauchyn kriteeri} todistamiseksi riittää
näyttää todeksi implikaatio
\begin{equation} \label{Cauchyn jonon suppenevuus}
\seq{a_n}\ \text{on Cauchy}\ \ \impl\ \ \seq{a_n}\ \text{suppenee}. \tag{$\star$}
\end{equation}
Todistus on useampivaiheinen. Ensin on tutkittava lukujonon raja-arvon määritelmää
tarkemmin \kor{osajonon} käsitteen pohjalta. Seuraavassa nämä tarkastelut ja niihin perustuva
väittämän \eqref{Cauchyn jonon suppenevuus} todistus erotetaan omaksi 
osaluvukseen\footnote[2]{Merkintä (*) luvun tai osaluvun otsikossa kertoo 
(tässä ja jatkossa), että kyse on näkökulmaa laajentavasta --- usein myös muuta tekstiä 
haastavammasta --- tekstin osasta.}.

\subsection{*Osajonot}

\begin{Def} \index{osajono|emph} \index{lukujono!g@osajono|emph} 
Lukujono $\{b_k\}_{k=1}^{\infty}$ on jonon $\{a_n\}_{n=m}^{\infty}$ \kor{osajono}
(engl.\ subsequence), jos on olemassa indeksit $\ m \le n_1 < n_2 < \ldots\ $ siten, että
\[
b_k = a_{n_k}, \quad k = 1,2, \ldots
\]  \end{Def}

\begin{Exa} Jos $\x = \{x_n\}_{n=p}^{\infty}$ on jaksollinen desimaaliluku, niin jollakin
$m \in \N$ ja riittävän suurella $k \in \N$ osajono
\[
\{x_k, x_{k+m}, x_{k+2m}, \ldots \}\ =\ \{x_{k+(l-1)m}\}_{l=1}^{\infty}
\]
on geometrinen sarja, vrt.\ Luku \ref{jonon raja-arvo}. Tässä siis osajonon indeksinä on $l$ ja
$n_l = k + (l-1)m$. \loppu 
\end{Exa}
\begin{Exa} Lukujono $\{\,a_n = (-1)^n,\ \ n=0,1,2, \ldots\,\}$ ei suppene, mutta sillä on
(kohti rationaalilukua) suppenevia osajonoja, esim.\ $\{\,a_0, a_2, a_4, \ldots\,\}$. \loppu 
\end{Exa}
\begin{Exa} Rajatta kasvavan lukujonon $\seq{n} = \{1,2,3,\ldots\}$ jokainen osajonokin on
rajatta kasvava. \loppu
\end{Exa}
Osajonon käsitettä valaisee hieman seuraava 'lämmittelylause'.
\begin{Lause} \label{suppenevat osajonot} $\lim_na_n=a\in\R$ täsmälleen kun $\lim_k b_k = a\,$
jokaiselle jonon $\{a_n\}$ osajonolle $\{b_k\}$.
\end{Lause}
\tod Väittämän osa \ \fbox{$\impl$} \ on ilmeinen, sillä Määritelmän \ref{jonon raja} 
perusteella osajono suppenee vähintään yhtä nopeasti kuin itse jono. Tästä huolimatta osa 
\fbox{$\limp$} on vieläkin ilmeisempi, sillä osajonoksi kelpaa myös itse jono $\{a_n\}$. \loppu

Seuraava tulos, jossa osajonon käsite on keskeinen, on avain jatkon kannalta. Tuloksessa 
tulkitaan proposition $a_n \kohti a$ negaatio ($a_n \not\kohti a$) Määritelmään \ref{jonon raja}
perustuen. --- Huomattakoon, että kyse ei ole negaatiosta 'ei suppene' (kohti mitään lukua),
koska $a_n \not\kohti a$ on tosi myös, kun $a_n \kohti b \neq a$.
\begin{*Lause} \label{negaatioperiaate} Jos $\seq{a_n}$ on lukujono ja $a\in\R$, niin
$\,a_n \not\kohti a\,$ täsmälleen kun $\exists\,\eps > 0\,$ ja osajono 
$\,\{\,a_{n_1},a_{n_2}, \ldots\,\}\,$ siten, että 
$\,\abs{a_{n_k}-a} \ge \eps\ \ \forall k \in \N$.
\end{*Lause}
\tod Lähdetään raja-arvon määritelmästä (Määritelmä \ref{jonon raja})
\begin{align*}
a_n \kohti a \quad                 &\ekv \quad \forall \eps > 0\ \,\text{pätee}\ [\,\ldots\,]
\intertext{ja ryhdytään negaation purkuun: Ensinnäkin (vrt.\ Luku \ref{logiikka})} 
a_n \not\kohti a \quad             &\ekv \quad \text{jollakin}\ \eps > 0\ \,
                                               \text{ei päde}\ [\,\ldots\,]
\intertext{Tässä on $\,\ [\,\ldots\,]\,=\,[\ \exists N\ \in \N\ $ siten, että 
$\ \{\,\ldots\,\}\ ]$, \ joten}
\text{ei päde}\ [\,\ldots\,] \quad &\ekv \quad \not\exists N \in \N\ \ 
                                               \text{siten, että}\ \{\,\ldots\,\} \\[3mm]
                                   &\ekv \quad ei\,\{\,\ldots\,\}\ \ \forall N \in \N.
\intertext{Tässä on $\,\ \{\,\ldots\,\}\,=\,\{\,\abs{a_n-a} < \eps\ \ \forall n>N\,\}$, \ 
joten}
ei\,\{\,\ldots\,\} \quad           &\ekv \quad \abs{a_n-a} \ge \eps\ \ \text{jollakin}\ n>N.
\end{align*}
Siis on päätelty:
\[
a_n \not\kohti a \quad \ekv \quad \text{jollakin}\ \eps>0\ 
    \text{pätee:}\quad \forall\ N \in \N\ (\,\abs{a_n-a} \ge \eps\ \text{jollakin}\ n>N\,)\,.
\]
Kyseisellä $\eps$ voidaan nyt  menetellä seuraavasti:
\begin{itemize}
\item[1.] Valitaan $\,N=1\ \,\,$   ja $\,n_1>N\,$ siten, että $\,\abs{a_{n_1}-a} \ge \eps$.
\item[2.] Valitaan $\,N=n_1\,$     ja $\,n_2>N\,$ siten, että $\,\abs{a_{n_2}-a} \ge \eps$.
\item[$\ \vdots$]
\end{itemize}
Konstruktiota voidaan jatkaa loputtomasti, joten saadaan osajono
$\{\,a_{n_1},a_{n_2}, \ldots\,\}$ (ks.\ kuva alla). Tällä on vaadittu ominaisuus, joten on
todistettu väittämän osa \ \fbox{$\impl$}\,. Osa \ \fbox{$\limp$} \ on ilmeinen raja-arvon
määritelmän perusteella. \loppu

\begin{figure}[H]
\import{kuvat/}{kuvaI-1.pstex_t}
\end{figure}

Kun Lausetta \ref{negaatioperiaate} sovelletaan Cauchyn jonoon, saadaan johdetuksi 
mielenkiintoinen tulos.
\begin{*Lause} \label{Cauchyn jonon vaihtoehdot} Jos lukujono $\{a_n\}$ on Cauchyn jono, niin 
jokaisella $x\in\R$ on voimassa täsmälleen yksi seuraavista vaihtoehdoista:
\begin{itemize}
\item[1.] $\,a_n \kohti x$.
\item[2.] $\exists\,\eps>0$ ja $N \in \N$ siten, että $\ a_n\ < x - \eps\ $ aina kun $\ n>N$.
\item[3.] $\exists\,\eps>0$ ja $N \in \N$ siten, että $\ a_n\ > x + \eps\ $ aina kun $\ n>N$.
\end{itemize} 
\end{*Lause}
\tod Ensinnäkin on joko $a_n \kohti x$ (vaihtoehto 1) tai $a_n \not\kohti x$.
Jälkimmäisessä vaihtoehdossa on Lauseen \ref{negaatioperiaate} mukaan löydettävissä luku 
$\eps>0$ ja osajono $\{\,b_k = a_{n_k},\ k = 1, 2, \ldots\,\}$ siten, että 
$\,\abs{b_k - x} \ge 2 \eps\ \,\forall k$. (Tässä on luvun $\eps$ tilalle kirjoitettu $2 \eps$
mukavuussyistä.) Koska $\{a_n\}$ on Cauchyn jono, on myös olemassa lukua $\eps$ vastaava
indeksi $N$ siten, että pätee $\abs{a_n - a_m} < \eps$ kun $n,m > N$. Valitaan nyt osajonossa
$\{b_k\}$ indeksi $k$ siten, että $n_k > N$, ja kirjoitetaan
\[
a_n\ =\ b_k + (a_n - b_k).
\]
Tässä on $b_k = a_m,\ m = n_k > N$, joten $\abs{a_n - b_k} < \eps$, kun $n > N$. Myös oli
$\abs{b_k - x} \ge 2 \eps$, joten on oltava joko $b_k \ge x + 2 \eps$ tai $b_k \le x - 2 \eps$.
Jos  $b_k \ge x + 2 \eps$ (kuten kuvassa alla), niin arvioidaan
\[
a_n\ \ge x + 2 \eps - \abs{a_n - a_k}\ >\ x + 2 \eps - \eps\ =\ x + \eps, \quad \text{kun}\ n>N.
\]
Jos  $b_k \le x - 2 \eps$, niin arvioidaan
\[
a_n\ \le x - 2 \eps + \abs{a_n - a_k}\ <\ x - 2 \eps + \eps\ =\ x -\eps, \quad \text{kun}\ n>N.
\]
Lause on näin todistettu. \loppu
\vspace{5mm}
\begin{figure}[H]
\import{kuvat/}{kuvaI-2.pstex_t}
\end{figure}

Lauseen \ref{Cauchyn jonon vaihtoehdot} mukaan Cauchyn jonon on 'valittava puolensa' annetun
luvun --- esimerkiksi rationaaliluvun --- suhteen, ellei ko.\ luku ole jonon raja-arvo.
Käyttäen tätä tulosta 'työhevosena' voidaan nyt todistaa väittämä
\eqref{Cauchyn jonon suppenevuus} ja siis myös Lause \ref{Cauchyn kriteeri}. Todistus on
muuten olennaisesti sama kuin Lauseen \ref{monotoninen ja rajoitettu jono} todistus, paitsi
että monotonisuuden sijasta vedotaan Lauseeseen \ref{Cauchyn jonon vaihtoehdot}. 
\begin{*Lause} \label{Cauchyn jono suppenee} Cauchyn jono suppenee. \end{*Lause}
\tod Olkoon $\seq{a_n}$ Cauchyn jono. Jos $a_n \kohti x\in\Q$, niin lauseen väittämä on tosi
, joten voidaan olettaa, että $\seq{a_n}$ ei suppene kohti rationaalilukua. Tällöin on
ensinnäkin jokaisen kokonaisluvun $x \in \Z$ kohdalla voimassa Lauseen
\ref{Cauchyn jonon vaihtoehdot} vaihtoehdoista joko 2 tai 3. Koska Cauchyn jono on rajoitettu
(Lause \ref{Cauchyn jono on rajoitettu}), niin silloin on löydettävissä yksikäsitteinen
$x_0 \in \Z$ siten, että $a_n > x_0 + \eps_1$ kun $n > N_1$ ja $a_n < x_0 + 1 - \eps_2$ kun
$n > N_2$ (tässä $\eps_1,\eps_2 > 0$ ja $N_1,N_2 \in \N$), jolloin 
\[
x_0\ <\ a_n\ <\ x_0 + 1, \quad \text{kun}\ n>N = \max \{N_1,N_2\}.
\]
Oletetaan jatkossa, että $x_0 \ge 0$ (muussa tapauksessa tutkitaan jonoa $\seq{-a_n}$).
Jatkamalla päättelyä vastaavasti on löydettävissä yksikäsitteiset
$d_k \in \{0,\ldots,9\}$, $ k = 1,2,\ldots$ ja vastaavat äärelliset desimaaliluvut 
$x_k = x_0.d_1 \ldots d_k$ ja indeksit $N_k \in \N$ siten, että pätee
\[
x_k\ <\ a_n\ <\ x_k + 10^{-k}, \quad \text{kun}\ n>N_k, \quad k=0,1,2,\ldots
\]
Tämän mukaan on $\lim_n (a_n-x_n)=0$ eli $\lim_n a_n =\x=\seq{x_n}\in\R$ Määritelmien
\ref{jonon raja - desim} ja \ref{reaaliluvut desimaalilukuina} mukaisesti. Lauseen
\ref{suppeneminen kohti reaalilukua} perusteella $\lim_na_n=\x$ myös Määritelmän
\ref{jonon raja} mukaisesti. \loppu

\subsection{Reaaliluvut Cauchyn jonoina}
\index{reaalilux@reaalilukujen!a@laskuoperaatiot|vahv}
\index{laskuoperaatiot!b@reaalilukujen|vahv}
\index{reaalilux@reaalilukujen!b@järjestysrelaatio|vahv}
\index{jzy@järjestysrelaatio!c@$\R$:n|vahv}

Määritelmä \ref{jonon raja - desim} tulkitsee äärettömän desimaaliluvun 'suppenevan itseensä'
lukujonona. Tällainen näkökulma voidaan ottaa yleisemminkin rationaalilukujen Cauchyn jonoihin,
jolloin saadaan seuraava vaihtoehtoinen määritelmä reaaliluvuille (vrt.\ Määritelmät
\ref{reaaliluvut desimaalilukuina} ja \ref{samastus DD}).
\begin{Def} \label{reaaliluvut Cauchyn jonoina} \index{reaaliluvut!c@Cauchyn jonoina|emph}
(\vahv{Reaaliluvut Cauchyn jonoina}) Reaaliluvut ovat rationaalilukujen Cauchyn jonoja. Kaksi
reaalilukua $a=\seq{a_n}$ ja $b=\seq{b_n}$ ovat samat täsmälleen kun $\lim_n\,(a_n - b_n) = 0$. 
\end{Def}
Määritelmän mukaisesti reaaliluvut ovat rationaalilukujen Cauchyn jonojen muodostamia 
\pain{ekvivalenssiluokkia} (tai samastusluokkia, vrt.\ Luku \ref{logiikka}), joiden sisällä 
eri jonot samastetaan määritelmän kriteerillä. Jokaiseen samastusluokkaan kuuluu äärettömän
monta erilaista jonoa; esim.\ jonot $\seq{a_n}$ ja $\seq{a_n + n^{-k}}$ kuuluvat samaan
luokkaaan jokaisella $k\in\N$. Ellei kyseessä ole äärellinen desimaaliluku, on samaan 
samastusluokkaan kuuluvista Cauchyn jonoista täsmälleen yksi Määritelmän
\ref{reaaliluvut desimaalilukuina} mukainen reaaliluvun 'edustaja', ts.\ ääretön desimaaliluku.

Määritelmään \ref{reaaliluvut Cauchyn jonoina} perustuen on laskuoperaatioiden määrittely 
reaaliluvuille varsin yksinkertaista: Jos $a=\seq{a_n}$ ja $b=\seq{b_n}$ ovat reaalilukuja 
(eli $\seq{a_n}$ ja $\seq{b_n}$ ovat rationaalilukujen Cauchyn jonoja), niin myös jonot
$\seq{a_n + b_n}$ ja $\seq{a_n b_n}$ ovat Cauchyn jonoja 
(Lause \ref{Cauchyn jonojen yhdistelysäännöt}) ja edustavat sellaisenaan lukuja $a+b$ ja $ab$.
Vastaavasti jos $b \neq 0$ (eli $b_n \not\kohti 0$), niin jostakin indeksistä $n=m$ alkaen on
$\abs{b_n} \ge \delta>0$ (Lause \ref{Cauchyn jonon vaihtoehdot}), jolloin $\seq{a_n/b_n}$ on
ko.\ indeksistä eteenpäin Cauchy (Lause \ref{Cauchyn jonojen yhdistelysäännöt}) ja edustaa
lukua $a/b$.

Myös järjestysrelaation määrittely käy Cauchyn jonojen avulla helposti, sillä jos
$a=\seq{a_n}$, niin vaihtoehdot $a=0$, $a>0$ ja $a<0$ määräytyvät asettamalla $x=0$ Lauseessa
\ref{Cauchyn jonon vaihtoehdot}\,: $\,a=0$ vaihtoehdossa 1, $a<0$ vaihtoehdossa 2, ja $a>0$
vaihtoehdossa 3. Yleisemmin lukujen $a$ ja $b$ suuruusjärjestys ratkaistaan vertaamalla lukua
$a-b=\seq{a_n-b_n}$ lukuun $0$ mainitulla tavalla.

Jos Määritelmä \ref{reaaliluvut Cauchyn jonoina} tulkitaan laskennallisesti, niin yksittäistä
Cauchyn jonoa voi pitää algoritmina, joka tuottaa reaaliluvulle jonon rationaalisia 
approksimaatioita. Algoritmi on yleisesti sitä parempi, mitä nopeammin jono suppenee 
(tai 'suippenee') suhteessa tarvittavien (rationaalisten) laskuoperaatioiden määrään. Cauchyn
jonoille voidaan siis asettaa myös laatukriteereitä!
\begin{Exa} Eräs rationaalilukujen Cauchyn jonojen ekvivalenssiluokka on $\sqrt{2}$. Tähän
luokkaan kuuluva ääretön desimaaliluku, eli lukujono $\{1,1.4,1.414,\ldots\}$ vastaa
algoritmina edellisen luvun Esimerkin \ref{neliöjuuri 2} kymmenjakoalgoritmia. Toinen samaan
ekvivalenssiluokkaan kuuluva lukujono on Luvun \ref{monotoniset jonot} Esimerkissä
\ref{sqrt 2 algoritmina} tarkasteltu. Tämä on algoritmina kymmenjakoa selvästi tehokkaampi.
\loppu \end{Exa}

\subsection{*Bolzanon--Weierstrassin lause}

Asiayhteyden vuoksi esitettäköön vielä osajonon käsitteeseen liittyvä lause, joka kuuluu
modernin matemaattisen analyysin peruskiviin.

\begin{*Lause} \label{B-W} \index{Bolzanon--Weierstrassin lause|emph}
(\vahv{Bolzano-Weierstrass}\footnote[2]{Bolzanon-Weierstrassin lauseella on monia muotoja.
Perusidean esitti ensimmäisenä t\v{s}ekkiläinen matemaatikko, filosofi ja pappi
\hist{Bernard Bolzano} (1781-1848) v.\ 1817, mutta tulos jäi vähälle huomiolle. Saksalainen
\hist{Karl Weierstrass} (1815-1897), jota on pidetty 'modernin matemaattisen analyysin isänä',
todisti lauseen myöhemmin tuntematta Bolzanon työtä. \index{Bolzano, B.|av} 
\index{Weierstrass, K.|av}}) \ Jokaisella rajoitetulla reaalilukujonolla on suppeneva osajono. 
\end{*Lause}
\tod Todistus on perusidealtaan puolituskonstruktio, jossa looginen päättely nojaa
kaksipaikkaiseen predikaattiin
\[
Q(x,y): \quad x \le a_n \le y \quad 
\text{äärettömän monella indeksin $n$ arvolla.}
\]
Koska $\{a_n\}$ on rajoitettu jono, on löydettävissä luvut $a$ ja $b=a+L$ siten, että $Q(a,b)$
on tosi. Valitaan jokin indeksi $n_1$ siten, että $a \le a_n \le b$ kun $n=n_1$, ja tutkitaan
seuraavaksi, onko $Q(a,(a+b)/2)$ tosi. Jos on, asetetaan $b$:n uudeksi arvoksi $(a+b)/2$. Jos
ei, on proposition $Q((a+b)/2,b)$ oltava tosi (koska $Q(a,b)$ oli tosi). Tässä tapauksessa
asetetaan $a$:n uudeksi arvoksi $(a+b)/2$. Toisen luvuista $a,b$ tultua näin uudelleen
määritellyksi tiedetään, että $Q(a,b)$ on jälleen tosi. Etsitään nyt indeksi $n_2 > n_1$ siten,
että $a \le a_n \le b$ kun $n=n_2$ (mahdollista, koska $Q(a,b)$ oli tosi). Etenemällä samalla
tavoin saadaan konstruoiduksi alkuperäisen jonon osajono
$\{\,b_k = a_{n_k},\ k = 1,2,\ldots\,\}$, jolla on konstruktion perusteella ominaisuus
$\abs{b_k - b_l}\ \le\ 2^{-N} L \quad \text{kun}\ k,l > N$. Näin ollen jos jokaisella $\eps>0$
valitaan $N\in\N$ siten, että $\,2^{-N}L<\eps$, niin 
$\abs{b_k-b_l}<\eps$ kun $\,k,l>N$. Määritelmän \ref{Cauchyn jono} mukaan $\{b_k\}$ on Cauchyn
jono ja näin ollen suppee (Lause \ref{Cauchyn jono suppenee}). \loppu

\subsection{*Epäkonstruktiivinen päättely}
\index{epzy@epäkonstruktiivinen päättely|vahv}

Lauseen \ref{B-W} todistukseen kätkeytyy filosofinen ongelma,
joka esiintyy matemaattisessa ajattelussa yleisemminkin: Jos predikaatissa $Q(x,y)$
ajatellaan $x$ ja $y$ kiinnitetyiksi (sidotuiksi), niin proposition $P = Q(x,y)$ totuusarvon
selville saamiseksi on käytävä kirjaimellisesti läpi j\pain{okainen} jonon $\seq{a_n}$ termi,
ts.\ on suoritettava \pain{äärettömän} \pain{monta} lukujen vartailua. Tämä ei tietenkään ole 
käytännössä mahdollista. Matemaattisessa analyysissä katsotaan kuitenkin myös tällainen nk.\ 
\kor{epäkonstruktiivinen} päättely luvalliseksi. (Viime kädessä 'luvan' antavat joukko-opin 
perusaksioomat.) Epäkonstruktiiviselle todistukselle on tunnus\-omaista, että todistusta ei voi
seurata 'laskemalla', ellei käytettävissä ole jotakin lisätietoa, ts.\ ellei tehdä 
lisäoletuksia. Esimerkiksi aiemmin esitetty Lauseen \ref{monotoninen ja rajoitettu jono}
todistus on myös epäkonstruktiivinen, sillä tämäkin sisältää kuvitelman kaikkien jonon termien
vertaamisesta kiinteään lukuun. Epäkonstruktiivisia ovat itse asiassa myös Lauseiden 
\ref{negaatioperiaate} ja \ref{Cauchyn jonon vaihtoehdot} todistukset, sillä 
'olemassa olevaksi' väitetty luku $\eps$ konstruoidaan todistuksissa vain loogisena 
välttämättömyytenä. Mitään laskennallisesti toimivaa ideaa luvun määrittämiseksi ei anneta ---
eikä tehtyjen yleisten oletusten puitteissa olisi mahdollistakaan antaa.

Kaikissa mainituissa todistuksissa filosofiset ongelmat poistuvat (tai ainakin vähenevät), jos
oletetaan tarkasteltava lukujono sellaiseksi, että todistuksen sisältämät loogiset askeleet
voidaan toteuttaa (yksi kerrallaan) \pain{äärellisellä} \pain{määrällä} (viime kädessä)
\pain{rationaalisia} laskuoperaatioita tai vertailuja. Tällaista lukujonoa voi kutsua
vaikkapa \kor{ennustettava}ksi. Esimerkiksi Lukujen \ref{jono} --- \ref{reaaliluvut}
esimerkkijonot, tai yleisemmin kaikki 'ennustettavasti määritellyt' yksittäiset lukujonot, ovat
tällaisia. Lisäoletuksen ollessa voimassa voidaan todistuksissa esitettyjä ajatuskonstruktioita
seurata myös laskennallisesti. Lauseen \ref{negaatioperiaate} tapauksessa tämä tapahtuu esim.\ 
hakemalla käypää $\eps$:n arvoa jonosta $\,\{\,\eps_n = 10^{-n},\ n = 1,2, \ldots\,\}$.
Lisäoletuksen perusteella voidaan millä tahansa kiinteällä $n \in \N$ ratkaista äärellisellä
laskutyöllä, onko $\eps = \eps_n$ käypä valinta. Ellei ole, asetetaan $n \leftarrow n+1$, ja 
jatketaan laskemista. Ennustettavuusoletuksen perusteella laskun tiedetään päättyvän johonkin
indeksiin, vaikkei etukäteen tiedetä, mihin. 


\Harj
\begin{enumerate}

\item \label{H-I-9: todistuksia}
a) Todista Lause \ref{Cauchyn jono on rajoitettu}. \newline
b) Todista Lauseen \ref{Cauchyn jonojen yhdistelysäännöt} väittämät koskien lukujonoja
$\seq{a_n+b_n}$ ja $\seq{\lambda a_n}$. \newline
c) Todista Lauseen \ref{Cauchyn jonojen yhdistelysäännöt} väittämä koskien lukujonoa
$\seq{a_n b_n}$. \newline
d) Todista: Jos $\seq{a_n}$ on Cauchy ja $\abs{a_n} \ge \delta>0\ \forall n$, niin 
$\seq{a_n^{-1}}$ on Cauchy. \newline
e) Todista Lauseen \ref{Cauchyn jonojen yhdistelysäännöt} väittämä koskien lukujonoa
$\seq{a_n/b_n}$.

\item
Olkoon $\{a_k,\ k=0,1,\ldots\}$ lukujono ja $b_n=\sum_{k=0}^n 2^{-k}a_k,\ n=0,1,\ldots\,$
Näytä, että jos $\seq{a_k}$ on rajoitettu, niin $\seq{b_n}$ on Cauchy.

\item
Olkoon $\seq{a_n}$ rationaalilukujono. Näytä todeksi tai epätodeksi: \newline
a) \ $a_n^3 \kohti 4\,\ \impl\,\ \seq{a_n}$ on Cauchy, $\quad$
b) \ $a_n^4 \kohti 3\,\ \impl\,\ \seq{a_n}$ on Cauchy.

\item
Jos $\seq{b_k}$ on jonon $\seq{a_n,\ n=1,2, \ldots}$ suppeneva osajono, niin mitkä ovat
mahdolliset osajonon raja-arvot seuraavissa tapauksissa?
\[
\text{a)}\ \ a_n = \frac{(n+1)^n}{(-n)^n} \qquad 
\text{b)}\ \ a_n = \frac{n^3+(-1)^n(n-1)^3}{(n+2)^3+(-1)^n n^3}
\]

\item
a) Olkoon luvun $1/7\,$ $n$:s desimaali $=a_n$ ja luvun $1/17\,$ $n$:s desimaali $=b_n\,$.
Määritä lukujonon $\seq{a_n+b_n}$ suppenevien osajonojen mahdolliset raja-arvot. \
b) Olkoon $d_n =$ tunnetun jaksottoman desimaaliluvun $\pi=3.14159628..\,$ $n$:s desimaali.
Esitä kymmenen lukujonoa, joista vähintään kaksi on jonon $\seq{d_n}$ suppenevia osajonoja.
Perustele!

\item \label{H-I-9: suppenevat osajonot}
a) Näytä Lause \ref{suppenevat osajonot} päteväksi myös, kun $a=\infty$, \ b) Näytä, että
lukujono ei ole rajoitettu täsmälleen kun jonolla on osajono, joka on aidosti monotoninen ja
joko rajatta kasvava tai rajatta vähenevä.

\item (*)
Todista: \ Jos $\seq{a_n}$ ja $\seq{b_n}$ ovat rajoitettuja lukujonoja, niin on olemassa 
aidosti kasvava indeksijono $\seq{n_k}$ siten, että jonot $\seq{a_{n_k}\,,\ k=1,2, \ldots}$
ja $\seq{b_{n_k}\,,\ k=1,2, \ldots}$ ovat molemmat suppenevia.

\item (*) Käyttäen Cauchyn jonoihin perustuvaa reaaliluvun määritelmää näytä, että
$(\R,+,\cdot,<)$ on järjestetty kunta.

\item (*) \index{satunnaisluku(jono)}
\kor{Satunnaisluku}jono on rationaalilukujono $\{a_n,\ n=1,2,\ldots\}$, jolle pätee 
$0 \le a_n \le 1\ \forall n$ ja jolla on ominaisuus: Jos $0 \le x < y \le 1$ ja $N_n =$ niiden
indeksien $k\in\N$ lukumäärä, joille pätee $k \le n$ ja $x \le a_k \le y$ ($n\in\N$), niin
$\lim_n N_n/n = y-x$. Bolzanon-Weierstrassin lauseen mukaan jonolla $\seq{a_n}$ on suppeneva
osajono. Näytä, että pätee vahvempi tulos: Jos $0 \le x \le 1$, niin löytyy osajono, jonka
raja-arvo $=x$. 

\item (*)
Näytä, että jokaisella reaalilukujonolla on monotoninen osajono.

\end{enumerate}