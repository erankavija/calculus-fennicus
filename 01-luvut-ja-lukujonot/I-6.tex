\section{Lukujonon raja-arvo} \label{jonon raja-arvo}
\alku

Tässä ja seuraavissa luvuissa tarkastelun kohteena ovat j\pain{är}j\pain{estet}y\pain{n}
\pain{kunnan} lukujonot $\seq{a_n}$. Oletetaan siis, että $a_n\in\K\ \forall n$, missä 
$(\K,+,\cdot,<)$ on järjestetty kunta; tämä voi olla joko rationaalilukujen kunta tai 
jokin tämän kuntalaajennus, jossa myös järjestysrelaatio on määritelty. Esimerkiksi voi
olla $\K=\J$, vrt.\ Luku \ref{kunta}.

Lukujonon $\{a_n\}$ sanotaan
\index{lukujono!d@suppeneva}%
\kor{suppenevan} eli \kor{konvergoivan} (engl.\ converge)
kohti lukua $a$, jos '$n$:n kasvaessa $a_n$ tulee yhä lähemmäksi $a$:ta'. Täsmällisepi
määritelmä seuraa hieman tuonnempana; tässä vaiheessa riittäköön toteamus, että
suppeneminen tarkoittaa esimerkiksi seuraavaa:
\begin{align*}
\abs{a_n - a}\ &<\ 10^{-1} \quad \text{indeksistä $n=N_1$ alkaen}, \\
\abs{a_n - a}\ &<\ 10^{-2} \quad \text{indeksistä $n=N_2$ alkaen}, \\
               &\ \vdots \\
\abs{a_n - a}\ &<\ 10^{-k} \quad \text{indeksistä $n=N_k$ alkaen}, \\
               &\ \vdots
\end{align*}
Tässä jokainen $N_k$ \pain{on} \pain{äärellinen}, eli $N_k \in \N$ jokaisella $k \in \N$.
Suppenemista on havainnollistettu alla olevissa kuvissa graafisesti\footnote[2]{Lukujen 
graafisen (geometrisen) havainnollistamisen menetelmiä pidetään jälleen tuttuina. Asiaan 
palataan myöhemmin Luvussa II.}.

\begin{figure}[H]
\setlength{\unitlength}{1cm}
\begin{center}
\begin{picture}(12,5)(0,-1)
\path(0,1)(12,1)
\multiput(1,1)(10,0){2}{\line(0,1){0.15}}
\multiput(5.5,1)(1,0){2}{\line(0,1){0.15}}
\put(6,1){\line(0,1){0.15}}
\put(1,3){$\overbrace{\hspace{10cm}}^{\displaystyle{a_n, \ n>N_1}}$}
\put(5,1.5){$\overbrace{\hspace{1cm}}^{\displaystyle{a_n, \ n>N_2}}$}
\put(5.9,0.5){$a$}
\put(0.5,0.5){$a-10^{-1}$} \put(10.5,0.5){$a+10^{-1}$}
\put(4,-1){$a-10^{-2}$} \put(5,-0.5){\vector(1,3){0.5}}
\put(6.65,-1){$a+10^{-2}$} \put(7,-0.5){\vector(-1,3){0.5}}
\end{picture}
\end{center}
\end{figure}

Jos lukujono suppenee kohti $a$:ta, sanotaan lukua $a$ 

jonon \kor{raja-arvo}ksi (engl.\ limit). Myös termiä
\index{limes}%
\kor{limes} käytetään (lat.\ limes = raja). Merkintätapoja ovat
\[
\lim_{n \kohti \infty} a_n\ =\ a \qquad \text{tai} \qquad \lim_n a_n\ =\ a
\]
tai kuten jatkossa useammin:
\[
a_n \kohti a \quad (\text{kun}\ \ n \kohti \infty).
\]
Merkinnöissä '$\lim$' luetaan 'limes' ja symboli '\kohti' luetaan 'suppenee kohti',
'menee kohti' tai 'lähestyy'. 

\begin{figure}[H]
\setlength{\unitlength}{1cm}
\begin{center}
\begin{picture}(14,14)(-2,-0.5)
\put(0,0){\vector(1,0){12}} \put(11.8,-0.5){$n$}
\put(0,0){\vector(0,1){13.5}}
\put(0,2){\line(1,0){12}}
\Thicklines
\put(0,7){\line(1,0){12}}
\thinlines
\put(0,12){\line(1,0){12}}
\put(0,6.5){\line(1,0){12}}
\put(0,7.5){\line(1,0){12}}
\multiput(0.5,0)(0.5,0){23}{\line(0,1){0.1}}
\put(0.44,1){$\scriptstyle{\bullet}$}
\put(0.94,5){$\scriptstyle{\bullet}$}
\put(1.44,9){$\scriptstyle{\bullet}$}
\put(1.84,13){$\scriptstyle{\bullet}$}
\put(2.44,10){$\scriptstyle{\bullet}$}
\put(2.94,7){$\scriptstyle{\bullet}$}
\put(3.44,4){$\scriptstyle{\bullet}$}
\put(3.94,6){$\scriptstyle{\bullet}$}
\put(4.44,8){$\scriptstyle{\bullet}$}
\put(4.94,7){$\scriptstyle{\bullet}$}
\put(5.44,6){$\scriptstyle{\bullet}$}
\put(5.94,6.7){$\scriptstyle{\bullet}$}
\put(6.44,7.2){$\scriptstyle{\bullet}$}
\put(6.94,6.7){$\scriptstyle{\bullet}$}
\put(7.44,7.1){$\scriptstyle{\bullet}$}
\put(7.94,6.75){$\scriptstyle{\bullet}$}
\put(8.44,7.05){$\scriptstyle{\bullet}$}
\put(8.94,7.05){$\scriptstyle{\bullet}$}
\put(9.44,6.8){$\scriptstyle{\bullet}$}
\put(9.94,6.8){$\scriptstyle{\bullet}$}
\put(10.44,7){$\scriptstyle{\bullet}$}
\put(10.94,6.85){$\scriptstyle{\bullet}$}
\put(11.44,7){$\scriptstyle{\bullet}$}
\put(11.94,6.9){$\scriptstyle{\bullet}$}
\dashline{0.2}(2.5,12)(2.5,0)
\dashline{0.2}(6,7)(6,0)
\path(2.5,0)(2.5,-0.5) \put(2.5,-0.5){\vector(1,0){0.5}} \put(3.2,-0.6){$n>N_1$}
\path(6,0)(6,-0.5) \put(6,-0.5){\vector(1,0){0.5}} \put(6.7,-0.6){$n>N_2$}
\put(0.4,-0.5){$1$} \put(0.9,-0.5){$2$} \put(1.4,-0.5){$3$} 
\multiput(0,2)(0,10){2}{\line(-1,0){0.1}}
\multiput(0,6.5)(0,1){2}{\line(-1,0){0.1}}
\put(0,7){\line(-1,0){0.1}} \put(-0.5,6.9){$a$}
\put(-2,8.1){$a+10^{-2}$}
\put(-2,5.5){$a-10^{-2}$}
\put(-1.7,11.9){$a+10^{-1}$}
\put(-1.7,1.9){$a-10^{-1}$}
\put(-0.6,8){\vector(1,-1){0.5}}
\put(-0.6,6){\vector(1,1){0.5}}
\end{picture}
\end{center}
\end{figure}

Suppeneminen on lukujonolle varsin voimakas vaatimus, ja onkin helppo esittää esimerkkejä 
jonoista, jotka eivät suppene kohti mitään lukua. Tällainen on vaikkapa
\index{rajatta kasvava (lukujono)}%
\kor{rajatta kasvava} lukujono $\{a_n\} = \{n\}$. Tässä tapuksessa on tapana kirjoittaa
\[
a_n \kohti \infty\ \ \text{kun}\ \ n \kohti \infty \qquad 
                     \text{tai jopa:} \quad\lim_n a_n = \infty.
\]
Luvallisia luku- ja puhetapoja ovat '$a_n$ lähestyy ääretöntä' tai 'raja-arvo on ääretön'. 
Näiden sanontojen mukaisesti merkintä '$\lim_{n \kohti \infty}$' luetaan yleensä 
'limes $n$ lähestyy ääretöntä', tarkoittaen siis raja-arvoa, kun $n$ kasvaa rajatta.
Vastaavasti jos lukujono on
\index{rajatta vähenevä (lukujono)}%
\kor{rajatta vähenevä}, voidaan kirjoittaa 
$\,a_n \kohti - \infty\,$ tai $\,\lim_n a_n = - \infty$, ja lukea 'raja-arvo on miinus 
ääretön'. Näistä merkintä- ja puhetavoista huolimatta rajatta kasvava tai rajatta vähenevä
lukujono ei ole suppeneva.
\begin{Exa} \label{jonoja} Etsi mahdollinen raja-arvo rationaalilukujonolle
$\seq{a_n}_{n=1}^\infty$, kun \vspace{1mm}\newline
a) \ $a_n = n^2 \qquad\qquad\,$ 
b) \ $a_n = 100-n \qquad\qquad$ 
c) \ $a_n = (-1)^n$ \vspace{1mm}\newline
d) \ $a_n = (n+1)/n \quad$
e) \ $a_n = (10^{100}+ 2n)/(10^{100} + n)$
\end{Exa}
\ratk a) Kyseessä on rajatta kasvava jono: $a_n \kohti \infty$. Raja-arvoa ei siis ole.

b) Tämä jono on rajatta vähenevä: $a_n \kohti - \infty$. Raja-arvoa ei ole.

c) Jonon termit saavat vuorotellen arvoja $+1$ ja $-1$, mutta eivät 'lähesty' kumpaakaan näistä
tai mitään muutakaan lukua. Päätellään, että tälläkään jonolla ei ole raja-arvoa.

d) Tässä on ensimmäinen suppeneva jono: Kirjoittamalla
\[
a_n = 1+\frac{1}{n}
\]
nähdään, että $n$:n kasvaessa termit tulevat yhä lähemmäksi lukua $1$, joten päätellään, että 
$\lim_n a_n = 1$.

e) Tämä jono on myös suppeneva, mutta oikea raja-arvo paljastuu vasta hyvin suurilla indeksin
arvoilla: Jos esimerkiksi tutkitaan jonon termejä indeksin arvoilla 
$n = N, N+1,\ldots,N+10^{10}$, niin saadaan seuraavat tulokset:
\[
N \le n \le N + 10^{10} \quad \impl \quad \begin{cases}
                                          a_n \approx 1, \quad \ \ \text{kun}\ N = 1,        \\
                                          a_n \approx 1, \quad \ \ \text{kun}\ N = 10^{50},  \\
                                          a_n \approx 3/2,\ \      \text{kun}\ N = 10^{100}, \\
                                          a_n \approx 2, \quad \ \ \text{kun}\ N = 10^{120}.
                                          \end{cases}
\]
Vasta viimeinen testi kertoo totuuden: $\lim_n a_n = 2$. \loppu

Esimerkki \ref{jonoja}e) paljastaa, että lukujonon raja-arvoa ei välttämättä heti 'näe' tai 
löydä kokeilemallakaan. Tämäntyyppinen epävarmuus voidaan poistaa vain täsmällisemmällä 
raja-arvon määritelmällä.

\begin{Def} \label{jonon raja} (\vahv{Lukujonon raja-arvo})
\index{lukujonon raja-arvo|emph} \index{lukujono!d@suppeneva|emph}
\index{raja-arvo!a@lukujonon|emph} \index{suppeneminen!a@lukujonon|emph} Lukujono $\{a_n\}$
suppenee kohti raja-arvoa $a$, jos jokaisella $\eps > 0$\footnote[2]{Kreikkalainen kirjain
$\eps$ (luetaan  'epsilon') on matematiikassa hyvin vakiintunut pienen positiivisen luvun
symboli. Toinen usein käytetty symboli on myös kreikkalaiseen kirjaimistoon kuuluva $\delta$
(luetaan 'delta').} on olemassa $N\in\N$ siten, että pätee
\[
\abs{a_n - a}\ < \eps, \quad \text{kun}\ n > N.
\]
\end{Def}
Määritelmän mahdollisimman 'pakattu' muoto on 
\[
\lim_n a_n = a 
    \qekv \forall \eps > 0\ \exists N \in \N\ (\,\abs{a_n-a} < \eps\ \,\forall n > N\,).
\]
Jos $a$:n tilalla on $\infty$, on määritelmässä ehdon $\abs{a_n - a} < \eps$ tilalla oltava ehto
$a_n > M$, missä $M$ on mielivaltaisen suuren luvun symboli. Määritelmän tiivis muoto on tällöin
\[
\lim_n a_n = \infty \qekv \forall M\ \exists N \in \N\ (\,a_n > M\ \,\forall n > N\,).
\]

Määritelmässä \ref{jonon raja} $N$ riippuu yleensä (jokseenkin aina!) $\eps$:sta. Riippuvuuden
voimakkuudella ei ole väliä, kunhan jokaisella p\pain{ositiivisella} $\eps$ on löydettävissä
\pain{äärellinen} indeksi $N$ (eli luku $N\in\N$) siten, että asetettu ehto on voimassa. Ehto
puolestaan merkitsee, että indeksistä $N$ eteenpäin jonon \pain{kaikki} termit pysyvät enintään
$\eps$:n päässä $a$:sta. Kun ehtoa kiristetään antamalla $\eps$:lle yhä pienempiä mutta 
positiivisia arvoja, niin $N$ yleensä kasvaa, mutta pysyy aina äärellisenä. Vastaavasti jos
$\lim_n a_n = \infty$, niin indeksistä $N$ eteen päin jonon kaikki termit ovat lukua $M$
suurempia. Luvun $M$ kasvaessa kasvaa yleensä myös $N$, mutta pysyy äärellisenä jokaisella
$M$:n (äärellisellä) arvolla.

\begin{Lause} \label{raja-arvon yksikäsitteisyys} Jos lukujono $\{a_n\}$ on suppeneva, niin 
raja-arvo $a = \lim_n a_n$ on yksikäsitteinen.
\end{Lause}
\tod Jos $a_n \kohti a$ ja $b \neq a$, niin kolmioepäyhtälön (Lause \ref{kolmioepäyhtälö}) ja 
suppenemisen määritelmän perusteella jokaisella $\eps>0$ on olemassa $N\in\N$ siten, että
pätee
\begin{align*}
\abs{a_n - b}\ &=\ \abs{\,(a_n - a) + (a-b)\,} \\
               &\ge\ \abs{a-b} - \abs{a_n - a} \\
               &>\ \abs{a-b} - \eps, \quad \text{kun}\ n > N.
\end{align*}
Kun tässä valitaan $\eps = \tfrac{1}{2}\abs{a-b}$ (mahdollista, koska 
$a \neq b\ \impl\ \abs{a-b} > 0\,$), niin kyseisellä $\eps$:n arvolla pätee
\[
\abs{a_n - b} > \eps, \quad \text{kun}\ n > N,
\]
joten suppenemisen määritelmän mukaan $a_n \not\kohti b$. On näytetty toteen lauseen
väittämä muodossa
\[
a_n \kohti a\ \ \ja\ \ b \neq a \quad \impl \quad a_n \not\kohti b. \quad \loppu
\]
\jatko \begin{Exa}(jatko) c) $a_n = (-1)^n$. \ Jos valitaan mikä tahansa luku $a$, niin 
$\forall n \in \N$ pätee
\[
\max \{\abs{a_n - a},\ \abs{a_{n+1} - a}\}\ =\ \max \{\abs{1-a},\ \abs{1+a}\}\ \ge\ 1.
\]
Näin ollen jos valitaan $\eps < 1$, niin suppenevuden määritelmän ehto
\[
\abs{a_n - a}\ < \eps, \quad \text{kun}\ n > N
\]
ei ole voimassa millään indeksin $N$ arvolla. Siis Määritelmän \ref{jonon raja} mukaista 
raja-arvoa ei ole olemassa.

d) $a_n = (n+1)/n$. \ Kun $a=1$, pätee
\[
\abs{a_n - a} = \frac{1}{n} < \eps, \quad \text{kun}\ \ n > \frac{1}{\eps}\,.
\]
Näin ollen jos valitaan esimerkiksi $N=$ (lukua $1/\eps$ lähinnä suurempi kokonaisluku),
niin raja-arvon määritelmässä asetettu ehto toteutuu ko.\ $N$:n arvolla. Siis
$a_n \kohti 1$.

e) $a_n = (10^{100}+2n)/(10^{100}+n)$. \ Kun valitaan $a=2$, niin
\[
a_n - a = \dfrac{10^{100}+2n}{10^{100}+n} - 2 
        = - \dfrac{10^{100}}{10^{100}+n} \qimpl \abs{a_n - a} < \frac{10^{100}}{n}\,.
\]
Näin ollen
\[
\abs{a_n - a} < \eps, \quad \text{kun}\ \ n \ge 10^{100} \cdot {\eps}^{-1}.
\]
Siis jos valitaan $N$ esimerkiksi siten, että
\[
10^{100} \cdot {\eps}^{-1}\ \le\ N\ <\ 10^{100} \cdot {\eps}^{-1} + 1,
\]
niin suppenevuuden määritelmässä asetettu vaatimus $\abs{a_n - a} < \eps$ on voimassa, kun 
$n > N$. Vaikka $N$ on hyvin suuri (jo kun $\eps=1$), on $N$ kuitenkin äärellinen jokaisella 
$\eps > 0$, joten määritelmän mukaan $\,a_n \kohti 2$. \loppu 
\end{Exa}

Seuraavaa raja-arvotulosta tarvitaan käytännössä (ja teoriassakin) usein.
\begin{Prop} \label{potenssilimes} $\D \
\lim_{n \kohti \infty} q^n = \begin{cases}     
                                 0,      \quad\,\ \text{jos}\ -1 < q < 1,  \\
                                 \infty, \quad    \text{jos}\ \    q\,> 1.
                             \end{cases} $
\end{Prop}
\tod Jos $q=0$, niin $q^n = 0\ \forall n \in \N$, jolloin $q^n \kohti 0$ raja-arvon määritelmän
mukaan. Jos $0<q<1$, niin $q = 1/(1+x)$, missä $x>0$. Tällöin on Bernoullin epäyhtälön 
(Propositio \ref{Bernoulli}) perusteella
\[
q^n = \frac{1}{(1+x)^n} < \frac{1}{1+nx} < \frac{1}{nx}\,, \quad n \ge 2.
\]
Tässä on jokaisella $\eps>0$
\[
\frac{1}{nx} < \eps, \quad \text{kun}\ \ n > \frac{1}{x \eps}\,,
\]
joten valitsemalla $N\in\N$ siten, että $N \ge (x \eps)^{-1}$ (mahdollista aina kun $x>0$
ja $\eps > 0$), seuraa
\[
0 < q^n < \eps, \quad \text{kun}\ n > N.
\]
Raja-arvon määritelmän perusteella $q^n \kohti 0$. Jos lopulta $-1 < q < 0$, niin
$q^n = (-1)^n (-q)^n$, missä $0 < -q < 1$, joten valisemalla $N$ samalla tavoin kuin
edellä seuraa
\[
-\eps < q^n < \eps, \quad \text{kun}\ n > N.
\]
Jälleen $q^n \kohti 0$ raja-arvon määritelmän perusteella, joten ensimmäinen väittämä on 
todistettu. Tapausta $q>1$ koskeva toinen väittämä todistetaan vastaavalla tavalla,
kirjoittamalla ensin $q = 1+x$ ja käyttämällä Bernoullin epäyhtälöä. \loppu

Lukujonon suppenemisen määritelmästä todettakoon vielä, että siinä asetettua ehtoa 
'jokaisella $\eps>0$' on mahdollista lieventää määritelmän muuttumatta. Esimerkiksi riittää
valita $\eps$ jonosta $\{10^{-k},\ k=0,1,2,\ldots\,\}$, jolloin määritelmäksi tulee
(vrt.\ johdattelu luvun alussa): Jokaisella $k=0,1,2,\ldots$ on olemassa indeksi $N_k\in\N$
siten, että $\abs{a_n-a} < 10^{-k}$ kun $n>N_k$, eli:
\[
\lim_n a_n = a 
\qekv \forall k\in\N\cup\{0\}\ \exists N_k\in\N\ (\,\abs{a_n-a} < 10^{-k}\ \,\forall n > N_k\,).
\]
Tässä voi $10^{-k}$:n tilalla olla yhtä hyvin $q^k,\ 0<q<1$, tai vielä yleisemmin, $\eps$ on
mahdollista poimia mistä tahansa aidosti vähenevästä (ks.\ Määritelmä \ref{monotoninen jono})
lukujonosta $\seq{b_k}$, kunhan on ensin varmistettu Määritelmään \ref{jonon raja} vedoten, että
$\lim_k b_k=0$. Tulos on täsmällisemmin muotoiltuna seuraava. Todistus jätetään
harjoitustehtäväksi (Harj.teht.\,\ref{H-I-6: jonon raja}a).
\begin{Prop} \label{jonon raja 2} Olkoon $\{b_k,\ k=1,2,\ldots\}$ aidosti vähenevä lukujono,
jonka raja-arvo on $\lim_k b_k = 0$. Tällöin lukujonon $\seq{a_n}$ raja-arvo $=a$ täsmälleen
kun jokaisella $k\in\N$ on olemassa indeksi $N_k\in\N$ siten, että $\abs{a_n-a}<b_k$ kun $n>N_k$.
\end{Prop}

\subsection{Sarjan summa}
\index{sarjan summa|vahv}

Jos sarja $\,\sum_{n=m}^{\infty} a_n$ suppenee (osasummiensa lukujonona, vrt.\ Luku \ref{jono}),
niin raja-arvoa sanotaan \kor{sarjan summaksi}. Raja-arvo merkitään
\[
s\ =\ \lim_{n \kohti \infty}\,\sum_{k=m}^n a_k\ =\ \sum_{k=m}^{\infty} a_k.\footnote[2]{Jos 
sarja on suppeneva, niin asiayhteydestä on selvitettävä, tarkoittaako 
$\,\sum_{n=m}^{\infty} a_n\,$ itse sarjaa vai sen summaa. Esimerkiksi merkintä 
$\sum_{k=m}^\infty a_k = \sum_{k=m}^\infty b_k$ voi tarkoittaa joko, että sarjat ovat samat 
lukujonoina (jolloin ne ovat samat myös termeittäin: $a_k=b_k\ \forall k$), tai ainoastaan, että
sarjojen summat ovat samat.}
\]  
\begin{Exa} \label{geometrinen sarja} Tutki perusmuotoisen geometrisen sarjan suppenemista. \end{Exa}
\ratk Perusmuotoinen geometrinen sarja on lukujono
\[
\{s_n\}\ =\ \{\, \sum_{k=0}^n q^k\,\}\ 
         =\ \begin{cases}
            \{\,n+1,\ n = 0,1, \ldots\,\}, \qquad\qquad\qquad \ \ \text{jos}\ q=1, \\
            \{\,(q^{n+1}-1)/(q-1),\ n = 0,1, \ldots\,\}, \quad \text{jos}\ q \neq 1.
            \end{cases}
\]
Jos $q=1$, niin sarja ei suppene ($s_n \kohti \infty$). Jos $q = -1$, niin 
$\,\{s_n\} = \{\,1,0,1,0 \ldots\,\}$, joten sarja ei suppene tässäkään tapauksessa. Jos 
$\abs{q} > 1$, niin arvioidaan kolmioepäyhtälön avulla
\[
\abs{s_n}\ =\ \dfrac{1}{\abs{q-1}}\,\abs{q^{n+1}-1}\ 
           \ge\ \dfrac{1}{\abs{q-1}}\,\left(\abs{q}^{n+1} - 1\right).
\]
Proposition \ref{potenssilimes} avulla päätellään tästä, että $\abs{s_n} \kohti \infty$, joten
tässäkään tapauksessa sarja ei suppene. Jäljelle jääneessä tapauksessa $-1 < q < 1$ valitaan 
raja-arvokandidaatiksi $\ s = 1/(1-q)$, jolloin
\[
\abs{s_n - s}\ =\ \dfrac{\abs{q}^{n+1}}{\abs{1-q}}\,.
\]
Olkoon $\eps>0$, jolloin on myös $\abs{1-q}\,\eps>0$. Koska $\abs{q}^{n+1} \kohti 0$
(Propositio \ref{potenssilimes}) ja $\abs{1-q}\,\eps>0$, niin lukujonon raja-arvon määritelmän
perusteella on olemassa indeksi $N\in\N$ siten, että
\[
\abs{q}^{n+1} < \abs{1-q}\,\eps \qimpl \abs{s_n-s}<\eps, \quad \text{kun}\ n>N.
\]
Tässä $\eps>0$ oli mielivaltainen ja $N\in\N$, joten lukujonon raja-arvon määritelmän 
perusteella $\ \lim_n s_n = s$. Siis geometrinen sarja suppenee täsmälleen kun $\abs{q} < 1$,
ja raja-arvo on tällöin
\[
\lim_n s_n\ =\ \lim_n\,(\,\sum_{k=0}^n q^k\,)\ =\ \frac{1}{1-q}\,. \quad \loppu
\]

Sarjan summamerkintää käyttäen esimerkin tulos on
\[
\boxed{\quad \sum_{k=0}^{\infty} q^k\ =\ \frac{1}{1-q}\,, \quad |q|<1. \quad}
\]

\subsection{Rationaalilukujono, jonka raja-arvo ei ole rationaalinen}

Rationaalilukujonon raja-arvon ei tarvitse olla rationaaliluku, sillä Määritelmän
\ref{jonon raja} mukaisesti riittää, että raja-arvo on löydettävissä lukujoukosta
$\K\supset\Q$, jossa laskuoperaatiot ja järjestysrelaatio ovat määriteltyjä niin,
että $(\K,+,\cdot,<)$ on rationaalilukujen kunnan laajennus. Seuraavassa on esimerkki 
rationaalilukujonosta, joka suppenee, kunhan lukujoukkoon $\K\supset\Q$ sisältyy 
myös luku $a=\sqrt{2}$ (vrt.\ Luku \ref{kunta}, Esimerkki \ref{muuan kunta}).
\begin{Exa} \label{sqrt 2} Tarkastellaan palautuvaa rationaalilukujonoa
\[
a_0 = 2, \quad a_{n+1} = \frac{a_n}{2} + \frac{1}{a_n}\,, \quad n = 0,1,\ldots
\]
Palautuskaavasta seuraa, että $a_n>0\ \forall n$ (induktio!) ja että
\[
a_{n+1}^2 -2 = \frac{1}{4 a_n^2}\left(a_n^2 - 2\right)^2, \quad n = 0,1,\ldots
\]
Tästä nähdään (induktio!), että on oltava $a_n^2-2>0\ \forall n$, jolloin voidaan edelleen 
päätellä:
\[
a_{n+1}^2-2\,=\,\frac{a_n^2-2}{4 a_n^2}\,(a_n^2-2)\,
             <\,\frac{a_n^2}{4a_n^2}\,(a_n^2-2)
             =\,\frac{1}{4}\,(a_n^2-2), \quad n=0,1, \ldots
\]
Koska $a_0^2-2 = 2$, niin seuraa (induktio!)
\[
0\,<\,a_n^2-2\,<\,2\cdot 4^{-n}, \quad n=0,1, \ldots
\]
Propositioiden \ref{potenssilimes} ja \ref{jonon raja 2} perusteella seuraa tästä, että
$a_n^2 \kohti 2$.

On päätelty, että $a_n>0\ \forall n$ ja lisäksi, että lukujono $\seq{a_n^2}$ on suppeneva ja
$a_n^2 \kohti 2$. Näytetään nyt, että Määritelmän \ref{jonon raja} mukaisesti on oltava 
$\,\lim_n a_n=a=\sqrt{2}$. Tätä silmällä pitäen kirjoitetaan ensin
\[
a_n^2 - 2 = a_n^2 - a^2 = (a_n + a)(a_n - a).
\]
Koska oli $a_n>0$ ja $a_n^2>2=a^2$ jokaisella $n$, niin $a_n>a\ \forall n$ 
(ks.\ Luku \ref{kunta}). On myös $a>1$, joten seuraa
\begin{align*}
a_n^2-2\,&=\,(a_n+a)(a_n-a)\,>\,2a\,(a_n-a)\,>\,2(a_n-a) \\[1mm]
         &\impl\quad a_n-a\,<\,2^{-1}(a_n^2-2)\,<\,4^{-n}, \quad n=0,1, \ldots
\end{align*}
Siis $\,0<a_n-a<4^{-n}\ \forall n$, joten $\,a_n \kohti a\,$ (vrt.\ päättely edellä).
Lukujonolle $\seq{a_n}$ on näin löydetty ei-rationaalinen raja-arvo $a=\sqrt{2}$. \loppu
\end{Exa}

Esimerkin lukujonolla $\seq{a_n}$ ei voi olla rationaalista raja-arvoa, koska 
$\sqrt{2}\not\in\Q$ ja Lauseen \ref{raja-arvon yksikäsitteisyys} mukaan raja-arvo on
yksikäsitteinen. Rationaalilukujonojen joukossa on siis sellaisia, jotka 
'näyttävät suppenevan', mutta joilla ei kuitenkaan ole raja-arvoa rationaalilukujen
joukossa. Milloin rationaalilukujonolle on yleisemmin löydettävissä raja-arvo 
lukujoukkoa (rationaalilukujen kuntaa) laajentamalla ja milloin ei, on keskeinen kysymys
jatkossa.

\Harj
\begin{enumerate}

\item
Seuraavissa esimerkeissä lukujono $\seq{a_n}$ joko suppenee kohti lukua $a$ tai kasvaa
rajatta. Määritä pienin $N\in\N$ siten, että pätee joko $\abs{a_n-a}<1/100$ tai $a_n>100$,
kun $n>N$\,:
\begin{align*}
&\text{a)}\ \ a_n = \frac{4n-1}{2n+1} \qquad 
 \text{b)}\ \ a_n = \frac{4n+100}{2n+100} \qquad
 \text{c)}\ \ a_n = \frac{3n-10000}{2n+10000} \\
&\text{d)}\ \ a_n = \frac{n^2}{2n+1} \qquad 
 \text{e)}\ \ a_n = \frac{n^2}{3n+100} \qquad\,
 \text{f)}\ \ a_n = \frac{n^2}{4n+10000}
\end{align*}

\item
Olkoon
\[
a_n = \begin{cases} 1/k, &\text{kun}\ n=10^k,\ k\in\N, \\
                    0,   &\text{muulloin ($n\in\N$)},
      \end{cases} \quad
b_n = \begin{cases} 10^{-100}, &\text{kun}\ n=10^k,\ k\in\N, \\
                    0,         &\text{muulloin ($n\in\N$)}.
      \end{cases}
\]
Näytä, että \ a) $a_n \kohti 0$, \ b) \ $\seq{b_n}$ ei suppene kohti mitään lukua.

\item \label{H-I-6: lukujonopari}
Todista, että lukujonoille pätee: \vspace{1mm}\newline
a) \ $\lim_n a_n=a\,\ \impl\,\ \lim_n|a_n|=|a|$. \vspace{1mm}\newline
b) \ $\lim_n a_n=a\ \ja\ a_n \ge b_n \ge a\ \forall n\,\ \impl\,\ \lim_n b_n=a$.
\vspace{1mm}\newline
c) \ $a_n \kohti a\ \ja\ b_n \kohti b\,\ 
                    \ekv\,\ c_n=\max\{\abs{a_n-a},\abs{b_n-b}\} \kohti 0$. \vspace{1mm}\newline
d) \ $a_n \kohti a\ \ja\ b_n \kohti b\,\ 
                    \ekv\,\ (a_n-a)^2+(b_n-b)^2 \kohti 0$. \vspace{1mm}\newline
e) \ $a_n \kohti a\ \ja\ b_n \kohti b\ \ja\ c_n \kohti c\,\
                    \ekv\,\ (a_n-a)^2+(b_n-b)^2+(c_n-c)^2 \kohti 0$.

\item
Lukujonoista $\seq{a_n}$ ja $\seq{b_n}$ tiedetään, että $a_n \kohti a$ ja $b_n \kohti a$ ja
lukujonosta $\seq{c_n}$, että jokaisella $n$ on joko $c_n=a_n$ tai $c_n=b_n$ (ei tiedetä, kumpi).
Näytä, että $c_n \kohti a$.

\item
Olkoon $q\in\Q,\ q>0$. Laske $\displaystyle{\ \lim_n\,q^{-n}\sum_{k=0}^n 2^k}\,$ eri $q$:n 
arvoilla.

\item
Määritä pienimmät Proposition \ref{jonon raja 2} mukaiset luvut $N_k,\ k=1,2,3$, kun
$b_k=10^{-k}$ ja $a_n=0.9^n,\ n\in\N$.

\item (*) \label{H-I-6: jonon raja}
a) Todista Propositio \ref{jonon raja 2}. \ b) Muotoile ja todista Proposition
\ref{jonon raja 2} vastine koskemaan lukujonoa $\seq{a_n}$, joka kasvaa rajatta.

\item (*)
Todista: \ $a_n>0\ \forall n\ \ja\ \lim_n a_n = 0\ \impl\ \exists \max_n\{a_n\}$. Näytä myös 
vastaesimerkillä, että $\,\lim_n a_n = 0\ \not\impl\ \exists \max_n\{a_n\}$.

\item (*)
Olkoon $k\in\N$, $\Q_k = \{\,\text{äärelliset desimaaliluvut muotoa}\ x_0.d_1 \ldots d_k\,\}$,
ja $a_n\in\Q_k\ \forall n\in\N$. Näytä, että jos $\,\lim_n a_n=a$, niin $a\in\Q_k$ ja jollakin
$N\in\N$ pätee: $a_n=a$ kun $n>N$. 

\item (*)
Olkoon $\{\,a_k,\ k=1,2, \ldots\,\}$ rationaalilukujono ja $\{\,N_k,\ k=1,2, \ldots\,\}$ aidosti
kasvava indeksijono (luonnollisten lukujen jono) siten, että $N_1=1$. Määritellään jono
$\seq{b_n}$ asettamalla
\[
b_n=a_k, \quad \text{kun}\ N_k \le n < N_{k+1}, \quad k=1,2, \ldots
\]
Todista: \ $\lim_k a_k=a\ \impl\ \lim_n b_n = a$.

\item (*) \label{H-I-6: sqrt-kunta}
Tarkastellaan kunnassa $(\J,+,\cdot,<)$ (ks.\ Luku \ref{kunta}) palautuvasti määriteltyä
lukujonoa
\[
a_0 = x, \quad a_{n+1} = \frac{1}{2}\left(a_n + \frac{x}{a_n}\right), \quad n=0,1, \ldots
\]
missä $x\in\J,\ x>0$. Näytä: \\ \\
a) $\displaystyle{\quad 
      a_{n+1}^2-x = \frac{1}{4a_n^2}\left(a_n^2-x\right)^2, \quad n=0,1, \ldots}$ \\ \\
b) $\quad a_n\ge\sqrt{x}, \quad n=1,2, \ldots$ \\ \\ 
c) $\quad \lim_n a_n^2 = x.$ \\ \\
d) $\quad \lim_n a_n = \sqrt{x}.$

\end{enumerate}
