\section{Potenssisarja} \label{potenssisarja}
\alku

Sarjaa muotoa
\[
\sum_{k=0}^\infty a_k x^k\ =\ a_0 + a_1\,x + a_2\,x^2 + \ldots
\]
sanotaan \kor{potenssisarja}ksi (engl.\ power series). Luvut $a_k$ ovat nimeltään 
\kor{potenssisarjan kertoimet}, ja symbolia $x$, joka myös edustaa reaalilukua, sanotaan sarjan
\kor{muuttuja}ksi. Termi 'muuttuja' kertoo, että $x$:n lukuarvo voi vaihdella. Kyseessä ei siis
ole pelkästään lukujono, vaan pikemminkin joukko lukujonoja, missä kertoimet $a_k$ ajatellaan 
kiinnitetyiksi ja muuttujalle $x$ sallitaan erilaisia (reaali)arvoja.\footnote[1]{Tapauksessa
$x=0$ potenssisarja tulkitaan lukujonoksi $\{a_0,a_0,\ldots\}$, eli sarjamerkinnässä sovitaan,
että $0^0=1$.}

Kuten tullaan havaitsemaan, potenssisarja on matemaattisessa analyysissä hyvin keskeinen käsite
ja työkalu. Tässä luvussa tutkitaan toistaiseksi vain sarjan suppenemiskysymystä, jonka luonteva
muotoilu on: Millä $x$:n arvoilla sarja suppenee? Koska suppenevan sarjan tapauksessa summa 
yleensä riippuu $x$:n arvosta, ilmaistaan tämä kirjoittamalla sarjan summaksi $s(x)$, luetaan
'$s$ $x$'.
\begin{Exa} Perusmuotoinen geometrinen sarja $\sum_{k=0}^\infty x^k$ on esimerkki 
potenssisarjasta. Sarja suppenee täsmälleen kun $x \in (-1,1)$, ja sarjan summa on tällöin 
$s(x) = 1/(1-x)$. \loppu 
\end{Exa}
Esimerkin perusteella potenssisarjaa voi pitää geometrisen sarjan yleistyksenä. Osoittautuukin,
että suppenevuustarkasteluissa geometrinen sarja on hyvä vertailukohta. Tällaiseen vertailuun 
perustuu esimerkiksi seuraava, potenssisarjojen suppenemisteorian keskeisin tulos.
\begin{Lause} \vahv{(Potenssisarjan suppeneminen)} \label{suppenemissäde} Potenssisarjalle 
$\sum_{k=0}^\infty a_k x^k$ on voimassa jokin seuraavista vaihtoehdoista:
\begin{itemize}
\item[(a)] Sarja suppenee vain kun $x=0$.
\item[(b)] $\exists \rho\in\R_+$ siten, että sarja suppenee, myös itseisesti, kun 
           $x\in (-\rho,\rho)$ ja hajaantuu aina kun $\abs{x}>\rho$. Tällöin $\rho$ on sarjan
           \kor{suppenemissäde}.
\item[(c)] Sarja suppenee itseisesti $\forall x\in\R$.
\end{itemize}
\end{Lause}
\tod Riittää osoittaa, että (b) tai (c) toteutuu silloin kun (a) ei. Oletetaan siis, että sarja
suppenee jollakin $x_0 \in \R$, $x_0 \neq 0$. Tällöin on oltava (vrt.\ Korollaari 
\ref{sarjan suppenemiskorollaari} ja Lause \ref{suppeneva jono on rajoitettu}\,)
\[
a_nx_0^n\kohti 0 \ \impl \ \abs{a_nx_0^n}\leq C\quad \forall n \quad (C\in\R_+).
\]
Tästä seuraa, että jos $\abs{x}<\abs{x_0}$, niin
\[
\abs{a_nx^n}\leq C\abs{x/x_0}^n,
\]
jolloin majoranttiperiaatteen nojalla sarja suppenee itseisesti geometrisen sarjan 
($q=\abs{x/x_0}<1$) tavoin. Siis on todettu: Jos sarja suppenee kun $x=x_0\neq 0$, niin se 
suppenee itseisesti $\forall x\in (-\abs{x_0},\abs{x_0})$. Tästä seuraa edelleen, että jos sarja
hajaantuu kun $x=x_1$, niin se hajaantuu aina kun $\abs{x}>\abs{x_1}$, sillä muutoin se jo 
todetun perusteella olisi sekä suppeneva että hajaantuva, kun $x=x_1$. On siis päätelty: Jos (a)
ei ole voimassa, niin joko (c) sarja suppenee jokaisella $x \in \R$, jolloin se suppenee 
jokaisella $x \in \R$ myös itseisesti, tai (b) sarja suppenee kun $\abs{x} < \rho$ ja hajaantuu
kun $\abs{x} > \rho$, missä $\rho$ määritellään (vrt.\ Lause \ref{supremum-lause})
\[
\rho=\sup\,\{x \in \R \mid \text{sarja $\sum_{k=0}^\infty a_k x^k$ suppenee}\}.
\]
Lause on näin todistettu. \loppu

Lauseen vaihtoehdoissa (a), (c) voidaan sopia merkintätavoista
\[
\text{(a) } \rho=0,\quad \text{(c) } \rho=\infty.
\]
Sikäli kuin potenssisarjan suppenemissäde pystytään määräämään, ratkaisee 
Lause \ref{suppenemissäde} siis sarjan suppenemiskysymyksen täydellisesti, lukuun ottamatta 
muuttujan arvoja  $x = \rho$ ja $x = -\rho$, kun $\rho \in \R_+\,$. Nämä on selvitettävä 
tapauskohtaisesti, vrt.\ Esimerkki \ref{suppenemisvälejä} jäljempänä. Keskeisimmässä 
laskennallisessa ongelmassa, eli suppenemissäteen määräämisessä, auttaa usein seuraava tulos:
\begin{Lause} \vahv{(Suppenemissäteen laskukaava)} \label{suppenemissäteen laskukaava}
Jos potenssisarjassa $\sum_{k=0}^\infty a_k x^k$ on $a_k \neq 0\ \forall k \ge 1$ ja on 
olemassa raja-arvo
\[
\rho=\lim_{k\kohti\infty} \left|\frac{a_k}{a_{k+1}}\right|,
\]
missä $\rho\in\R_+$, $\rho=0$ tai $\rho=\infty$, niin sarjan suppenemissäde $=\rho$.
\end{Lause}
\tod Olkoon ensin $\abs{x} < \rho$. Valitaan $q$ siten, että $\abs{x} < q < \rho$ 
(esim.\ $q = (\abs{x}+\rho)/2$). Tällöin koska $\abs{a_k}/\abs{a_{k+1}} \kohti \rho$ ja 
$q < \rho$, niin jostakin indeksistä $k=m$ alkaen on $\abs{a_k}/\abs{a_{k+1}} \ge q$ 
(lukujonon suppenemisen määritelmässä valittu $\eps = \rho-q$). Tällöin voidaan päätellä
\[
\abs{a_{k+1}} \le q^{-1}\,\abs{a_k},\quad k = m,m+1, \ldots 
               \qimpl \abs{a_k} \le q^{m-k}\abs{a_m}, \quad k \ge m,
\]
jolloin
\[
\abs{a_k x^k} \le q^{m-k}\abs{a_m}\abs{x}^k = q^m\,\abs{a_m}\left(\frac{\abs{x}}{q}\right)^k 
                                            = C\left(\frac{\abs{x}}{q}\right)^k, \quad k \ge m.
\]
Tässä on $\abs{x}/q < 1$, joten päätellään majoranttiperiaatteen nojalla, että sarja 
$\sum_{k=0}^\infty a_k x^k$ suppenee itseisesti, indeksistä $k=m$ eteenpäin geometrisen sarjan
tavoin. Siis jos $\rho \in \R_+$, niin sarja suppenee itseisesti kun $\abs{x} < \rho$. Jos 
oletettu raja-arvo on $\rho = \infty$, niin ym.\ päättelyssä voidaan $\rho \in \R_+$ valita 
miten tahansa, joten tässä tapauksessa myös suppenemissäde $= \infty$.

Jos $\rho\in\R_+$ ja $\abs{x} > \rho$, niin nähdään samalla tavoin kuin yllä, että indeksin $m$
ollessa riittävän suuri pätee jokaisella $k \ge m$
\[
\abs{a_k x^k} \ge q^{m-k}\abs{a_m}\abs{x}^k = q^m\,\abs{a_m}\left(\frac{\abs{x}}{q}\right)^k 
                                            = C\left(\frac{\abs{x}}{q}\right)^k, \quad k \ge m.
\]
missä nyt $\rho < q < \abs{x}$. Näinollen $\abs{a_k x^k}\kohti \infty$ kun $k \kohti \infty$, 
jolloin sarja $\sum_k a_k x^k$ ei ole Cauchyn jono ja siis hajaantuu. Siis jos $\rho \in \R_+$,
niin sarja suppenee kun $\abs{x} < \rho$ ja hajaantuu kun $\abs{x} > \rho$, eli sarjan 
suppenemissäde $= \rho$. Jos $\rho = 0$, niin sarja todetaan samalla päättelyllä hajaantuvaksi
aina kun $x \neq 0$, eli tässä tapauksessa suppenemissäde $= 0$. Lause on näin todistettu. \loppu
\begin{Exa} Määritä potenssisarjan suppenemissäde, kun kertoimet ovat
\[
\text{a) } a_k=k!\quad \text{b) } a_k=(-1)^k (k+1)^{-1}\quad 
\text{c) } a_k=k^2\cdot 3^{-k}\quad \text{d) } a_k=1/k!\,.
\]
\end{Exa}
\ratk Lauseen \ref{suppenemissäteen laskukaava} perusteella
\begin{align*}
\text{a) } \rho &= \lim_k 1/(k+1) = 0, \qquad\quad\,\   
\text{b) } \rho = \lim_k\,(k+2)/(k+1) = 1, \\
\text{c) } \rho &= \lim_k\,3k^2/(k+1)^2 = 3, \qquad     
\text{d) } \rho = \lim_k\,(k+1) = \infty. \qquad\loppu
\end{align*}
Jos potenssisarja suppenee, kun $x \in A$, ja hajaantuu, kun $x \not\in A$, niin joukkoa 
$A\ (A\subset\R)$ sanotaan ko.\ sarjan \kor{suppenemisväli}ksi. Lauseen \ref{suppenemissäde}
mukaisesti $A$ on jokin vaihtoehdoista $(-\rho,\rho)$, $[-\rho,\rho)$, $(-\rho,\rho]$, 
$[-\rho,\rho]$, missä $\rho=$ suppenemissäde. 
\begin{Exa} \label{suppenemisvälejä} Määritä potenssisarjan  $\sum_{k=0}^\infty a_k x^k$ 
suppenemisväli, kun 
\[
\text{a)}\ a_k = 1, \quad \text{b)}\ a_k = 1/(k+1), \quad 
\text{c)}\ a_k = (-1)^k/(k+1), \quad \text{d)}\ a_k = 1/(k+1)^2.
\]
\end{Exa}
\ratk Lauseen \ref{suppenemissäteen laskukaava} laskukaava antaa suppenemissäteeksi $\rho = 1$
kaikissa tapauksissa, joten sarjat suppenevat, kun $\abs{x} < 1$, ja hajaantuvat, kun 
$\abs{x} > 1$. Tapauksissa $x = \pm 1$ esimerkkisarjat edustavat jo entuudestaan tuttuja 
geometrisen, harmonisen, alternoivan ja yliharmonisen sarjan tyyppejä. Tulokset yhdistämällä 
saadaan vastaukseksi
\[
\text{a)}\ (-1,1), \quad \text{b)}\ [-1,1), \quad \text{c)}\ (-1,1], 
\quad \text{d)}\ [-1,1]. \loppu
\]
\begin{Exa} Mikä on sarjan
\[
\sum_{k=0}^\infty (-1)^k (k^2+1) \dfrac{x^{2k}}{2^k}\ =\ 1 - x^2 + \dfrac{5}{4} x^4 - \ldots
\]
suppenemisväli? 
\end{Exa}
\ratk Lauseen \ref{suppenemissäteen laskukaava} laskukaava suppenemissäteelle ei sovellu 
suoraan, koska sarja on muotoa $\sum_{k=0}^\infty a_k x^k$, missä $a_k = 0$ parittomilla 
indeksin arvoilla. Tehtävä ratkeaa kuitenkin yksinkertaisesti vaihtamalla muuttujaksi $y=x^2$,
jolloin sarja saa muodon $\sum_{k=0}^\infty b_k y^k$. Lause \ref{suppenemissäteen laskukaava}
soveltuu tähän sarjaan, sillä
\[
\lim_k \left|\dfrac{b_k}{b_{k+1}}\right|\ 
               =\ \lim_k \dfrac{(k^2 +1)\,2^{-k}}{[(k+1)^2 + 1]\,2^{-k-1}}\ =\ 2.
\]
Siis sarja suppenee, kun $\abs{y} = x^2 < 2$, ja hajaantuu, kun $\abs{y} = x^2 > 2$. 
Suppenemissäde muuttujan $x$ suhteen on näinollen $\rho = \sqrt{2}$. Muuttujan arvoilla 
$x=\pm\sqrt{2}$ sarja nähdään hajaantuvaksi, joten suppenemisväli on $(-\sqrt{2},\sqrt{2})$.
\loppu
\begin{Exa} Millä arvoilla $x \in \R_+$ suppenee sarja 
\[ 
\sum_{k=0}^\infty \dfrac{x^{2k-1/2}}{k!}\ 
          =\ \dfrac{1}{\sqrt{x}} + x\sqrt{x} + \dfrac{x^3\sqrt{x}}{2} + \ldots \ \ ?
\]
\end{Exa}
\ratk Sikäli kuin sarja suppenee, niin raja-arvojen yhdistelysääntöjen 
(Lause \ref{raja-arvojen yhdistelysäännöt}) nojalla summa $s(x)$ on kirjoitettavissa
\[
s(x)\ =\ \dfrac{1}{\sqrt{x}} \sum_{k=0}^\infty \dfrac{(x^2)^k}{k!}\,.
\]
Päätellään, että sarja suppenee täsmälleen sellaisilla arvoilla $x\in\R_+$, joilla potenssisarja
$\sum_{k=0}^\infty x^{2k}/k!$ suppenee. Muuttujan vaihdolla $y=x^2$ ja Lauseen 
\ref{suppenemissäteen laskukaava} avulla todetaan ko.\ potenssisarjan suppenemissäteeksi 
$\rho = \infty$, joten kysytty sarja suppenee jokaisella $x \in \R_+$. \loppu
\begin{Exa} \label{potenssilimes ja potenssisarja} Päättele potenssisarjojen teorian avulla 
raja-arvotulos
\[ \abs{q} < 1 \qimpl \lim_k k^m q^k = 0\ \ \ \forall m \in \N. \] 
\end{Exa}
\ratk Koska (vrt.\ Lause \ref{raja-arvojen yhdistelysäännöt})
\[
\lim_k \dfrac{k^m}{(k+1)^m}\ =\ \lim_k \dfrac{1}{(1+1/k)^m}\ =\ 1,
\]
niin Lauseen \ref{suppenemissäteen laskukaava} mukaan potenssisarjan 
$\sum_{k=1}^\infty k^m x^k$ suppenemissäde on $\rho = 1$. Koska sarja siis suppenee kun $x=q$,
seuraa väite Korollaarista \ref{sarjan suppenemiskorollaari}. \loppu

Päätetään potenssisarjojen ensiesittely seuraavaan kauniiseen tulokseen, jolla tulee olemaan 
myöhempää käyttöä.
\begin{Lause} \label{potenssisarjan skaalaus} Jos $m \in \Z$, niin potenssisarjoilla 
$\sum_{k=1}^\infty a_k x^k$ ja $\sum_{k=1}^\infty k^m a_k x^k$ on sama suppenemissäde. 
\end{Lause}
%\tod Jos potenssisarjan $\sum_{k=0}^\infty a_k x^k$ kertoimilla on Lauseessa 
%\ref{suppenemissäteen laskukaava} oletettu raja-arvo-ominaisuus, niin jokaisella $m \in \Z$ 
%pätee (vrt.\ Esimerkki \ref{potenssilimes ja potenssisarja})
%\[
%\lim_k \left|\dfrac{k^m a_k}{(k+1)^m a_{k+1}}\right|\ 
%         =\ \lim_k \left(\dfrac{k^m}{(k+1)^m}\right) \lim_k \left|\dfrac{a_k}{a_{k+1}}\right|\
%         =\ 1 \cdot \rho\ =\ \rho,
%\]
%joten väite on tässä tapauksessa tosi Lauseen \ref{suppenemissäteen laskukaava} perusteella. 
%Yleisempi päättely on esim.\ seuraava: 
\tod Olkoon sarjan $\sum_{k=0}^\infty a_k x^k$ suppenemissäde
$ = \rho \neq 0$ ja olkoon $\abs{x} < \rho$. Valitaan $q$ siten, että $\abs{x} < q < \rho$ ja 
kirjoitetaan
\[
\abs{k^m a_k x^k}\ =\ k^m \left(\dfrac{\abs{x}}{q}\right)^k \abs{a_k}\,q^k.
\]
Koska $\abs{x}/q < 1$, niin tässä on (vrt.\ Esimerkki \ref{potenssilimes ja potenssisarja})
\[ 
k^m \left(\dfrac{\abs{x}}{q}\right)^k \kohti 0 \quad \text{kun}\ k \kohti \infty. 
\]
Näinollen $k^m (\abs{x}/q)^k \le C$ jollakin $C \in \R_+$ ja siis 
$\abs{k^m a_k x^k} \le C \abs{a_k}\,q^k \ \forall k \in \N$. Oletuksen perusteella sarja 
$\sum_{k=1}^\infty \abs{a_k}\,q^k$ on suppeneva (koska oli $0 < q < \rho$), joten 
majoranttiperiaatteen nojalla sarja $\sum_{k=1}^\infty k^m a_k x^k$ suppenee, kun 
$\abs{x} < \rho$. On siis päätelty, että jos sarjan $\sum_{k=1}^\infty k^m a_k x^k$ 
suppenemissäde $= \rho_1$, niin $\rho_1 \ge \rho$ jos $\rho \in \R_+$ ja $\rho_1 = \infty$ 
jos $\rho = \infty$. Päättely toimii myös toisinpäin: Kun kirjoitetaan 
$b_k = k^m a_k\ \ekv\ a_k = k^{-m} b_k$, niin seuraa samalla tavoin, että $\rho \ge \rho_1$ jos
$\rho_1 \in \R_+$ ja $\rho = \infty$ jos $\rho_1 = \infty$. Yhdistämällä päätelmät todetaan, 
että on oltava $\rho_1 = \rho$ kaikissa tapauksissa (myös kun $\rho=0$). \loppu

\Harj
\begin{enumerate}

\item
Määritä seuraavien potenssisarjojen suppenemissäteet:
\[
\text{a)}\ \  \sum_{k=0}^\infty \frac{k^2}{k+1}x^k \quad\ 
\text{b)}\ \ \sum_{k=0}^\infty (k-2)2^{k+1}x^k \quad\
\text{c)}\ \  \sum_{k=1}^\infty (-1)^k (\frac{x}{k})^k
\]

\item
Määritä seuraavien potenssisarjojen suppenemisvälit:
\[
\text{a)}\ \ \sum_{k=1}^\infty \frac{1}{\sqrt{k}} x^k \quad\ 
\text{b)}\ \ \sum_{k=1}^\infty \frac{(-1)^k}{\sqrt{k}} 3^k x^k \quad\
\text{c)} \ \ \sum_{k=1}^\infty \frac{\sqrt{k+1}}{k^2} 2^k x^k
\]

\item
Totea seuraavat sarjat joko suppeneviksi tai hajaantuviksi potenssisarjojen teoriaan vedoten:
\[
\text{a)}\ \ \sum_{k=1}^\infty (-1)^k k^{-10} \pi^k e^{-k} \quad\ 
\text{b)}\ \ \sum_{k=1}^\infty \frac{k!}{(3k)^k} \quad\
\text{c)}\ \ \sum_{k=1}^\infty \frac{(2k)^k}{k!}
\]

\item
Määritä seuraavien potenssisarjojen suppenemisvälit:
\[
\text{a)}\ \ \sum_{k=0}^\infty \frac{2^k}{k} x^{2k} \quad\ 
\text{b)}\ \ \sum_{k=1}^\infty \frac{1}{k^3 3^k} x^{3k} \quad \
\text{c)}\ \ \sum_{k=0}^\infty \frac{1}{k^{1/3} 3^k} x^{3k}
\]

\item
Potenssisarjassa $\sum_{k=0}^\infty a_k x^k$ on $a_k=1$, kun $k=m!\,$ ja $m\in\N$, muulloin on
$a_k=0$. \ a) Mikä on sarjan suppenemisväli? \ b) a-kohdan perusteella sarja suppenee, kun 
$x=1/10$. Laske summa $20$ merkitsevän numeron tarkkuudella.

\item
Potenssisarjan kertoimista tiedetään, että $k \le a_k \le k^2\ \forall k\in\N$. Mikä on sarjan
suppenemisväli?

\item
Olkoon potenssisarjojen $\sum_k a_k x^k$ ja $\sum_k b_k x^k$ suppenemissäteet $\rho_1$ ja
$\rho_2$. Mitä voidaan sanoa sarjan $\sum_k(a_k+b_k)\,x^k$ suppenemissäteestä, jos \newline
a) $\rho_1=\rho_2=\infty$, \ b) $\rho_1=0,\ \rho_2 \neq 0$, \ c) $\rho_1=\rho_2\in[0,\infty)$,
\ d) $a_k b_k \ge 0\ \forall k$, \ e) ei aseteta lisäehtoja?

\item
Määritä muuttujan vaihdolla seuraaville sarjoille joukko $A\subset\R$ siten, että sarja 
suppenee täsmälleen kun $x \in A$:
\[
\text{a)}\ \ \sum_{k=1}^\infty \frac{1}{k} (2x+1)^k \quad\ 
\text{b)}\ \ \sum_{k=1}^\infty \frac{(-1)^k}{k}(x^2-4x+3)^k
\]

\item
Oletetaan Lauseen \ref{suppenemissäteen laskukaava} laskukaava päteväksi potenssisarjalle 
$\sum_k a_k x^k$. Todista Lause \ref{potenssisarjan skaalaus} tämän avulla.

\item
Näytä, että sarjoilla $\sum_{k=0}^\infty a_k x^k$, $\sum_{k=0}^\infty (k+2)^2 a_k x^k$ ja 
$\sum_{k=1}^\infty \frac{k^2+1}{k} a_k x^k$ on kaikilla sama suppenemissäde. 

\item (*)
Esitä jokin rationaalikertoiminen potenssisarja, jonka suppenemisväli on 
$[-\sqrt{e},\sqrt{e}\,]$ ($e=$ Neperin luku).

% Tehtävä vanhasta luvusta I-12 (ei kuulu tähän lukuun)
\item (*) (Hidas sarja) \label{H-I-12: hidas sarja}
Halutaan arvioida yliharmonisen sarjan $\sum_{k=1}^\infty k^{-5/4}$ suppenemisnopeus ja löytää
toimiva algoritmi sarjan summan $s$ laskemiseksi halutulla tarkkuudella. \newline
a) \ Näytä Bernoullin epäyhtälön avulla, että jokaisella $k\in\N$ pätee
\[
a_k = \frac{(k+1)^{1/4}}{k^{1/4}}-1\ <\ \frac{1}{4k}\,, \quad 
b_k = 1-\frac{k^{1/4}}{(k+1)^{1/4}}\ >\ \frac{1}{4(k+1)}\,.
\]
b) \ Päättele a-kohdan tuloksesta, että
\[
\frac{1}{4(k+1)^{5/4}}\ <\ \frac{1}{k^{1/4}}-\frac{1}{(k+1)^{1/4}}\ 
                        <\ \frac{1}{4k^{5/4}}\,, \quad k\in\N.
\]
c) \ Päättele b-kohdan tuloksesta, että
\[
\frac{4}{(n+1)^{1/4}}\ <\ \sum_{k=n}^\infty \frac{1}{k^{5/4}}\ 
                       <\ \frac{4}{n^{1/4}}\,, \quad n\in\N.
\]
d) \ Halutaan laskea $s$ suoraan osasummiin perustuvalla approksimaatiolla $s \approx s_n$ 
siten, että virhe on enintään $10^{-4}$. Lähtien c-kohdan tuloksesta arvioi tarvittava 
laskenta-aika (aikayksikkö: vuosi!) koneella, joka summeeraa sarjaa $10^{10}$ termiä/s. \newline
e) \ c-kohdan tuloksesta voidaan johtaa myös approksimaatio 
\[
s\ =\ s_n + \sum_{k=n+1}^\infty \frac{1}{k^{5/4}}\ \approx\ s_n+\frac{4}{(n+1)^{1/4}}\ =\ s_n^*.
\]
Päättele virhearvio
\[
0\ <\ s_n^*-s\ <\ \frac{4}{(n+1)^{1/4}}-\frac{4}{(n+2)^{1/4}}\ <\ \frac{1}{(n+1)^{5/4}}\,.
\]
f) \ Arvioi, kauanko laskeminen d-kohdassa korkeintaan kestää, kun käytetään e-kohdan 
parannettua algoritmia. --- Entä jos vaaditaan, että virhe on enintään $10^{-15}$\,?

\end{enumerate}