\chapter{Luvut ja lukujonot}

''Jumala loi luonnolliset luvut --- kaikki muu on ihmistyötä'' oli saksalaisen matemaatikon 
\index{Kronecker L.}%
\hist{Leopold Kronecker}in (1823-1891) aforismi ja kannanotto koskien matematiikan alkuperää.
--- Epäilemättä lukumäärää tai järjestystä kuvaavien sanojen 'yksi', 'kaksi', jne eriytyminen
luonnollisessa kielessä on ollut eräs lähtökohta ja perusedellytys matematiikan syntymiselle.
Merkkejä lukujen ja suhteiden tajusta on säilynyt jopa 30 000 vuoden takaa, joten matematiikan
voi perustellusti katsoa olevan ihmisen kulttuurissa \kor{sui generis}, omaa lajiaan.
Kulttuuri-ilmiönä matematiikkaa voi jossain määrin ymmärtää rinnastamalla se muihin
kulttuurin lajeihin. Esimerkiksi jos tarkastellaan matematiikan historiaa ja kumulatiivista
rakennetta, tai käyttötapoja, voi nähdä monia yhtymäkohtia muihinkin mahtaviin ihmishengen
ilmentymiin, erityisesti luonnollisiin kieliin ja tekniikkaan.

Matematiikkaa voi kuitenkin syvemmin oppia ymmärtämään vain matematiikkaa opiskelemalla.
Opiskelu aloitetaan tässä luvussa 'alusta', eli (matemaattisen) \kor{luvun} käsitteestä. Luvun
punaisena lankana on \kor{reaalilukujen} konstruoiminen lähtien 'järjellisistä' eli
\kor{rationaalisista} luvuista. Matematiikan historiassa reaaliluvun käsite on täsmentynyt
lopullisesti vasta 1800-luvulla. Historiallisena ongelmana --- ja edelleen ongelmana
opetuksessa --- on käsitteeseen väistämättä (muodossa tai toisessa) sisältyvä 'loputtomuuden'
ajatus. Tässä tekstissä reaalilukuja lähestytään rationaalilukujen muodostamien
\kor{lukujonojen}, ja näiden erikoistapausten, \kor{äärettömien desimaalilukujen} kautta.
Reaaliluvun käsite tulee tällä tavoin liitetyksi 'oikeaan laskemiseen' ja myös tavanomaisiin
numeromerkintöihin. Päättymättömät lukujonot tulevat jatkossa käyttöön monessa muussakin
yhteydessä. Niihin perustuvat viime kädessä sekä monet laskentamenetelmät (algoritmit) että
reaalilukuihin perustuvan matemaattisen analyysin eli \kor{reaalianalyysin} käsitteet.