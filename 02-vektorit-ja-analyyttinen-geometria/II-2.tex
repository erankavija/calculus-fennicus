\section{Tason vektorit} \label{tasonvektorit}
\alku
\index{vektoria@vektori (geometrinen)!a@tason \Ekaksi|vahv}%

\kor{Vektori} on matemaattisena käsitteenä hieman kaksijakoinen. Se voi olla lähtökohtaisesti 
geometrinen olio, jolloin se sovelluksissa ajatellaan usein fysikaalisena, 'maailmassa 
vaikuttavana'. Tällaisia 'fysiikkavektoreita' ovat mm.
\begin{itemize}
\item  paikka, nopeus, kiihtyvyys, kulmanopeus
\item  voima, momentti
\item  X-kentän voimakkuus, X=gravitaatio, sähkö, magneetti...
\end{itemize}
Toisaalta vektori voidaan tulkita myös algebrallisesti, jolloin päädytään geometris-fysikaalista
vektoria yleisempään (myös abstraktimpaan) vektorikäsitteeseen. Jäljempänä nähdään ensimmäisiä
esimerkkejä myös tällaisista 'matematiikkavektoreista' (myöhemmin esimerkkejä tulee paljon 
lisää). Jatkossa lähtökohta vektorin käsitteeseen on kuitenkin aluksi geometrinen.

Euklidisessa tasossa \Ekaksi\, vektori määritellään geometrisesti \kor{suuntajanana}, ts. 
janana, jolla on suunta. Ajatus on tällöin, että jana sisältyy sen toisesta päätepisteestä 
lähtevään puolisuoraan. Puolisuoran ilmaisema suunta kuvataan janan päähän merkityllä
nuolenkärjellä:
\begin{figure}[H]
\setlength{\unitlength}{1cm}
\begin{center}
\begin{picture}(13,6)(0,-0.5)
%\Thicklines
\put(1,3){\line(2,1){5}} \put(0,0){\line(4,1){4}}
\put(9,3){\vector(2,1){3}} \put(12,1){\vector(-4,-1){3}}
\put(0.95,3.1){\line(1,-2){0.1}} \put(0.9,2.5){$A$}
\put(3.95,4.6){\line(1,-2){0.1}} \put(3.9,4.0){$B$}
\put(3.975,1.1){\line(1,-4){0.05}} \put(3.925,0.5){$A$}
\put(0.975,0.35){\line(1,-4){0.05}} \put(0.925,-0.25){$B$}
\put(8.9,2.5){$A$} \put(11.9,4){$B$}\put(11.925,0.5){$A$} \put(8.925,-0.25){$B$}
\put(6.5,4){$\hookrightarrow$}
\put(6.5,0){$\hookrightarrow$}
\end{picture}
\end{center}
\end{figure}
Vektorilla tarkoitetaan täsmällisemmin vain suuntajanan sisältävää \pain{osatietoa} 
\[
\{\text{(janan) pituus, suunta}\}.
\] 
Tiettyyn suuntajanaan liittyvä vektori merkitään $\overrightarrow{AB}$, yleisemmin 
käytetään vektorimerkintöjä $\vec a, \vec b, \vec v$ jne.

Huomattakoon, että vektori $\overrightarrow{AB}$ siis \pain{ei} 'tiedä', missä piste
$A$ sijaitsee euklidisessa tasossa. Näin ollen 'vektoria saa siirtää', kunhan 'ei kierrä
eikä venytä'. (Fysikaalinen vastine: Voiman vaikutus on sama vaikutuspisteestä
riippumatta.) 

Jos tunnetaan vektorit $\vec a=\overrightarrow{AB}$ ja $\vec b=\overrightarrow{BC}$,
niin tunnetaan myös $\vec c=\overrightarrow{AC}$\,:
\begin{figure}[H]
\setlength{\unitlength}{1cm}
\begin{center}
\begin{picture}(6,5)
%\Thicklines
\put(0,0){\vector(2,3){2}} \put(2,3){\vector(4,1){4}} \put(0,0){\vector(3,2){6}}
\put(-0.1,-0.1){$\bullet$} \put(1.9,2.9){$\bullet$} \put(5.9,3.9){$\bullet$}
\put(-0.2,0.3){$A$} \put(1.9,3.2){$B$} \put(5.9,4.2){$C$}
\put(0.9,1.8){$\vec a$} \put(4.8,3.8){$\vec b$} \put(4,2.8){$\vec c$}
\end{picture}
\end{center}
\end{figure}
\index{laskuoperaatiot!c@tason vektoreiden|(}%
Sanotaan, että $\vec c$ on vektoreiden $\vec a, \vec b$ \kor{summavektori} ja merkitään
\[
\vec c=\vec a+\vec b.
\]
Summa määrätään siis geometrisesti kolmiodiagrammilla. Näin määritellylle vektorien 
yhteenlaskulle pätevät tavanomaiset vaihdanta- ja liitäntälait:
\begin{itemize}
\item[(V1)] $\quad \vec a+\vec b = \vec b+\vec a$
\item[(V2)] $\quad (\vec a+\vec b\,)+\vec c = \vec a+(\vec b+\vec c\,)$
\end{itemize}
Tässä ei ole kyse aksioomista (vrt.\ vastaavat kunta-aksioomat, Luku \ref{kunta}) vaan
väittämistä, joiden todistus on geometrinen:
\begin{figure}[H]
\setlength{\unitlength}{1cm}
\begin{center}
\begin{picture}(14,5)
%\Thicklines
\put(0,0){\vector(2,3){2}} \put(2,3){\vector(4,1){4}}
\dashline{0.2}(0,0)(4,1) \dashline{0.2}(4,1)(6,4)
\put(3.9,0.975){\vector(4,1){0.1}} \put(5.9,3.85){\vector(2,3){0.1}}
\put(1.2,2.2){$\vec a$} \put(5.2,4){$\vec b$} \put(3.5,1.1){$\vec b$} \put(5.8,3.3){$\vec a$}

\put(6,0){\vector(2,3){2}} \put(8,3){\vector(4,1){4}}
\dashline{0.2}(6,0)(12,4) \put(11.9,3.933){\vector(3,2){0.1}} \put(12,4){\vector(1,-1){2}}
\put(6,0){\vector(4,1){8}}
\dashline{0.2}(8,3)(14,2) \put(13.9,2.0167){\vector(4,-1){0.1}}
\put(7.2,2.2){$\vec a$} \put(11.2,4){$\vec b$} \put(13.6,2.5){$\vec c$} 
\put(12,1.1){$\vec a+\vec b+\vec c$}
\put(12.9,2.3){$\scriptstyle{\vec b+\vec c}$} \put(11.5,3.3){$\scriptstyle{\vec a+\vec b}$}
\end{picture}
\end{center}
\end{figure}
Vektorin $\vec a = \overrightarrow{AB}$
\index{itseisarvo}%
\kor{itseisarvo} (arkisemmin 'pituus') on
\[
\abs{\vec a}=|\overrightarrow{AB}|=\abs{AB}=\text{janan }AB\text{ pituus}.
\]
Tälle pätee \kor{kolmioepäyhtälö} (vrt.\ Lause \ref{kolmioepäyhtälö})
\index{kolmioepäyhtälö!b@tason vektoreiden}%
\[
\abs{\vec a+\vec b} \le \abs{\vec a}+\abs{\vec b}.
\]
Jos $\lambda \in \R$, niin $\lambda \vec a$ tarkoittaa
\index{skaalaus (vektorin)}%
\kor{skaalattua} vektoria, jolle pätee:
\begin{itemize}
\item[(1)] $\quad |\lambda \vec a|=|\lambda||\vec a|$
\item[(2)] $\quad \lambda \vec a \uparrow\uparrow \vec a,
                                 \text{ jos } \lambda > 0, \text{ ja }
                  \lambda \vec a \uparrow\downarrow \vec a, \text{ jos } \lambda<0$
\end{itemize}
Tässä $\uparrow\uparrow$ tarkoittaa yhdensuuntaisuutta ja $\uparrow\downarrow$ 
vastakkaissuuntaisuutta. Vektorin skaalausta $\vec a \map \lambda \vec a \ (\lambda \in \R)$
sanotaan vektorialgebrassa yleisemmin \kor{skalaarilla kertomiseksi}. Skaalaajia eli
reaalilukuja kutsutaan siis tässä laskuoperaatiossa
\index{skalaari}%
\kor{skalaareiksi}. Skalaarilla kertomisen ja yhteenlaskun
määritelmistä on pääteltävissä (Harj.teht.\,\ref{H-II-1: perusteluja}a), että seuraavat
'luonnonmukaiset' säännöt ovat voimassa (vrt.\ kunta-aksioomat Luvussa \ref{kunta})\,:
\begin{itemize}
\item[(V3)] $\quad (\lambda + \mu)\vec a = \lambda \vec a + \mu \vec a$
\item[(V4)] $\quad \lambda(\mu \vec a\,) = (\lambda \mu)\vec a$
\item[(V5)] $\quad \lambda(\vec a + \vec b\,)= \lambda \vec a + \lambda \vec b$
\item[(V6)] $\quad 1 \vec a =  \vec a$
\end{itemize}
Operaation $\vec a \kohti \lambda \vec a$ määritelmän mukaisesti on $|0 \vec a|=0$.
Tällaista vektoria sanotaan
\index{nollavektori}%
\kor{nollavektoriksi} ja merkitään $\,0 \vec a = \vec 0$.
Nollavektorin suunta on epämääräinen  (ei määritelty). Nollavektori on vektorien
yhteenlaskun nolla-alkio, ts.\ pätee
\begin{itemize}
\item[(V7)] $\quad \vec a + \vec 0 = \vec a\ \forall \vec a$ 
\end{itemize}
Jokaisella vektorilla $\vec a$ on myös
\index{vastavektori}%
\kor{vastavektori} $-\vec a$, jolle pätee
\begin{itemize} 
\item[(V8)] $\quad \vec a + (-\vec a\,) = \vec 0$
\end{itemize}
Nimittäin kun säännössä (V3) valitaan $\lambda=1, \mu=-1$ ja käytetään nollavektorin
määritelmää ja sääntöä (V6), niin seuraa
\[
-\vec a = (-1)\,\vec a.
\]
Vastavektorin avulla tulee määritellyksi myös vektorien vähennyslasku:
\[
\vec a - \vec b = \vec a + (-\vec b\,).
\]
\begin{figure}[H]
\setlength{\unitlength}{1cm}
\begin{center}
\begin{picture}(6,2.3)(2,0.5)
%\Thicklines
\put(4,0){\vector(2,3){2}} \put(4,0){\vector(4,1){4}} \put(8,1){\vector(-1,1){2}}
\dashline{0.2}(4,0)(2,2) \put(2.1,1.9){\vector(-1,1){0.1}} 
\dashline{0.2}(6,3)(2,2) \put(2.1,2.025){\vector(-4,-1){0.1}}
\put(5.8,2.3){$\vec a$} \put(7.6,0.4){$\vec b$} \put(6.5,2.7){$\vec a-\vec b$} 
\put(2.7,1.4){$\vec a+(-\vec b\,)$}
\end{picture}
\end{center}
\end{figure}
Säännöt (V1)--(V8), yhdessä nollavektorin ja vastavektorin määritelmien kanssa, muodostavat
(tason vektoreiden)) \kor{vektorialgebran} laskulait.
\index{laskuoperaatiot!c@tason vektoreiden|)}
\begin{Exa} \label{kolmion keskipiste}
Laskulaeista (V1)--(V8) seuraava identiteetti
\[
\vec a + \frac{2}{3} \left( \frac{1}{2}\vec b - \vec a \right)\ 
             =\ \vec b + \frac{2}{3} \left( \frac{1}{2}\vec a - \vec b \right)\
             =\ \frac{2}{3} \left[ \vec b + \frac{1}{2}\left( \vec a - \vec b \right) \right]\
             =\ \frac{1}{3} \left( \vec a + \vec b \right)
\]
voidaan lukea: Kolmion keskijanat leikkaavat toisensa samassa pisteessä, joka jakaa keskijanat
suhteessa $2:1$ (vrt.\ kuvio). Leikkaupistettä sanotaan 
\index{keskizza@keskiö (kolmion)}%
kolmion \kor{keskiöksi} (keskipisteeksi). \loppu
\begin{figure}[H]
\setlength{\unitlength}{1cm}
\begin{center}
\begin{picture}(6.5,4.6)(0,0)
%\Thicklines
\put(0,0){\vector(1,2){2}} \put(0,0){\vector(1,0){6}}
\path(2,4)(6,0)(1,2)
\drawline(2,4)(3,0) \drawline(0,0)(4,2) \put(0,0){\vector(2,1){2.67}}
\put(1.4,3.5){$\vec a$} \put(5.5,-0.5){$\vec b$}
\put(0,-0.5){$A$} \put(1.9,4.2){$B$} \put(6.2,-0.1){$C$}
\put(1.2,1.2){$\scriptstyle{\frac{1}{3}(\vec a+\vec b\,)}$}
\end{picture}
\end{center}
\end{figure}
\end{Exa}

\subsection{Vektoriavaruus}
\index{vektoric@vektoriavaruus|vahv}%

Jos tason vektorien joukkoa merkitään symbolilla $V$, niin edellä olevilla määritelmillä on 
syntynyt algebra, joka merkittäköön $(V,\R)$. Laskusääntöjen (V1)--(V8) (yleisemmin:
aksioomien) ollessa voimassa sanotaan, että kyseessä on (lineaarinen) \kor{vektoriavaruus}.
Käsitteeseen siis sisältyvät
\index{laskuoperaatiot!ca@ yleisen vektoriavaruuden}
\begin{itemize}
\item  vektorien muodostama joukko $V$ (jossa operoidaan)
\item  vektorien yhteenlasku (+)
\item  nk. \kor{skalaarien} muodostama \kor{kertojakunta}, tässä = $\R$
\item  skalaarin ja vektorin kertolasku
\end{itemize}
\index{skalaari} \index{kertojakunta}%
Lyhennetty sanonta '$V$ on vektoriavaruus' tarkoittaa koko tätä algebraa, jolloin 
kertojakunta on juuri $\R$, tai muuten asiayhteydestä selvä. Sanonnat \kor{reaalikertoiminen}
vektoriavaruus tai yleisemmin \kor{$\K$-kertoiminen} vektoriavaruus kiinnittävät kertojakunnan
tarkemmin. Kertojakunnan ei siis tarvitse olla $\R$, vaan se voisi olla esim. $\K=\Q$ tai 
$\K=\G=\{\text{geometriset luvut}\}$. Lineaarinen vektoriavaruus on hyvin keskeinen käsite
matematiikassa, ja sillä on keskeinen rooli myös monissa matematiikan sovelluksissa.

\subsection{Kanta ja koordinaatisto}
\index{kanta|vahv} \index{koordinaatisto|vahv}%

Olkoon $\vec v \in V$ mielivaltainen tason vektori ja $\vec a, \vec b \in V$ kaksi vektoria, 
joille pätee
\[
\vec a, \vec b \neq \vec 0 ,\quad \vec a \neq \lambda \vec b \quad \forall \ \lambda \in \R.
\] 
Tällöin voidaan geometrisella konstruktiolla (ks.\ kuvio) löytää yksikäsitteiset $x,y \in \R$
siten, että
\[
\vec v = x \vec a + y \vec b.
\]
\begin{figure}[H]
\setlength{\unitlength}{1cm}
\begin{center}
\begin{picture}(6,4.5)(0,-0.5)
%\Thicklines
\put(0,0){\vector(1,1){4}} \put(0,0){\vector(1,0){2}} \put(0,0){\vector(3,1){6}}
\put(0,0){\vector(1,0){4}} \put(0,0){\vector(1,1){2}}
\dashline{0.2}(6,2)(2,2) \dashline{0.2}(6,2)(4,0)
\put(3.1,3.6){$\vec b$} \put(0.9,1.6){$y\vec b$} \put(1.5,-0.5){$\vec a$} 
\put(3.3,-0.5){$x\vec a$} \put(5.5,2.2){$\vec v$}
\end{picture}
\end{center}
\end{figure}
Sanotaan tällä perusteella, että $\vec v$ on $\vec a$:n ja $\vec b$:n
\index{lineaariyhdistely}%
\kor{lineaariyhdistely} (lineaarinen yhdistely, \kor{lineaarikombinaatio}) ja että
$x \vec a$ ja $y \vec b$ ovat $\vec v$:n
\index{komponentti (vektorin)}%
(vektori)\kor{komponentit} $\vec a$:n ja $\vec b$:n suuntaan. 

Kun siis kaksi em.\ ehdot täyttävää vektoria on valittu, on koko $V$ esitettävissä muodossa
\[
V=\{x \vec a + y \vec b \mid x,y \in \R\}.
\]
Sanotaan, että vektoripari $\{ \vec a, \vec b \}$ on $V$:n \kor{kanta}\footnote[2]{Kanta on
vektoreiden j\pain{är}j\pain{estett}y joukko, joten merkintä $(\vec a,\vec b)$ olisi
loogisempi. Erinäisten sekaannusten välttämiseksi käytetään tässä yhteydessä kuitenkin
yleisemmin aaltosulkeita.} ja että $x,y$ ovat vektorin $\vec v = x \vec a + y \vec b \in V$
\index{koordinaatti (vektorin)}%
\kor{koordinaatit} kannassa $\{\vec a,\vec b\,\}$. Jos koordinaatit esitetään lukuparina
$(x,y)$ (pari = kahden alkion järjestetty joukko), niin on synnytetty yhteys $V$:n ja
tällaisten lukuparien joukon välille. Merkitään jälkimmäistä joukkoa symbolilla $\Rkaksi$,
äännetään 'R kaksi':
\[
\Rkaksi = \{ (x,y) \mid x \in \R\,\ja\,y \in \R \}.\index{karteesinen tulo|av}%
\footnote[3]{Joukko $\R^2$ voidaan myös  merkitä $\R\times\R$, jolloin käytetään
joukko-opillista \kor{karteesisen tulon} merkintää
\[
A \times B = \{ (x,y) \mid x \in A\,\ja\,y \in B \}.
\] 
Tämä luetaan '$A$:n ja $B$:n karteesinen tulo', tai vain '$A$ risti $B$'. Joukon
$A \times B$ alkiot ovat siis pareja. Näiden välinen samastusrelaatio on
\[
(x_1,y_1) = (x_2,y_2) \qekv  x_1=x_2\ \wedge\ y_1=y_2.
\] }
\]

Em. sopimuksilla on itse asiassa luotu \kor{kääntäen yksikäsitteinen} (=molempiin suuntiin 
yksikäsitteinen)
\index{vastaavuus ($\ensuremath  {\leftrightarrow }$)}%
\kor{vastaavuus} $V$:n ja $\Rkaksi$:n välille. Merkitään tätä: 
\[
V \vast \Rkaksi.
\]
Vastaavuuteen liittyen nimetään $\Rkaksi$ kantaan $\{\vec a, \vec b\,\}$ liittyväksi $V$:n
\index{koordinaattiavaruus}% 
\kor{koordinaatti-avaruudeksi}. Nimitys jo viittaa siihen, että myös $\Rkaksi$ on tulkittavissa
\index{vektorib@vektori (algebrallinen)!a@$\R^2$:n}%
lineaariseksi vektoriavaruudeksi. Nimittäin kaikki vektoriavaruuden laskulait ovat voimassa,
kun yhteenlasku ja skalaarilla kertominen $\Rkaksi$:ssa määritellään
\index{laskuoperaatiot!cb@vektoriavaruuden $(\Rkaksi,\R)$}
\begin{align*}
(x_1,y_1)+(x_2,y_2)& = (x_1+x_2,y_1+y_2), \\
\lambda(x,y) &= (\lambda x, \lambda y).
\end{align*}
Nämä operaatiot myös vastaavat $V$:n laskutoimituksia, sillä
\begin{align*}
\vec v_1=x_1 \vec a + y_1 \vec b, \ \vec v_2= x_2 \vec a + y_2 \vec b \ 
                               &\impl \ \vec v_1 + \vec v_2 = (x_1+x_2)\vec a 
                                                                        + (y_1+y_2)\vec b, \\
\vec v = x \vec a +y \vec b \  &\impl \ \lambda \vec v = (\lambda x)\vec a +(\lambda y)\vec b.
\end{align*}

Vektoriavaruus $(\Rkaksi,\R)$, mainituin laskuoperaatioin, on esimerkki (lajissaan ensimmäinen)
algebrallisten vektoreiden muodostamasta vektoriavaruudesta. Tämä avaruus on
siis geometrisista tulkinnoista vapaa. Toisaalta tulkitsemalla $(\Rkaksi,\R)$ tason vektorien
koordinaattiavaruudeksi (valitussa kannassa) saadaan vektorien laskuoperaatiot muunnetuksi 
algebralliseen muotoon. Esimerkiksi yhteenlaskussa tällainen laskukaavio on
\[
\begin{array}{cccccc}
&\vec v_1   &  &\vec v_2   &     &\vec v_1+\vec v_2 \\
&\downarrow &  &\downarrow &     &\uparrow \\ 
&(x_1,y_1)  &  &(x_2,y_2)  &\map &(x_1+x_2,y_1+y_2)
\end{array}
\]
Kaavion etuna on, että itse laskuoperaatiossa ei tarvita mitään tietoa vektorien 'ulkonäöstä'.
Monia geometrian tuloksia voidaan tällä tavoin johtaa algebran keinoin (vrt.\ Esimerkki 
\ref{kolmion keskipiste} edellä).

Koordinaattiavaruus $\Rkaksi$ voidaan yhtä hyvin liittää myös siihen alkuperäiseen 
pisteavaruuteen $\Ekaksi$, josta koko vektoriajatus oli lähtöisin. Nimittäin jos vektoreita 
ajatellaan $\Ekaksi$:n suuntajanoina, 'joita voi siirrellä', niin voidaan yhtä hyvin ajatella,
että jokainen vektori vastaa yksikäsitteistä suuntajanaa, jonka lähtöpiste on kiinteä piste $O$.
Näin on synnytetty $\Ekaksi$:n ja $V$:n kääntäen yksikäsitteinen vastaavuus:
\[
P \in \Ekaksi \ \leftrightarrow \ \overrightarrow{OP} \in V.
\]
Kun nyt tulkitaan myös edellä valitut kantavektorit $\vec a, \vec b$ pisteestä $O$ alkaviksi
suuntajanoiksi, niin euklidiseen tasoon on luotu \kor{koordinaatisto}\footnote[2]{Ajatuksen 
koordinaatiston avulla tapahtuvasta geometrian aritmetisoimisesta toi matematiikkaan 
ranskalainen filosofi-matemaatikko \hist{Ren\'e Descartes} (1596-1650) tutkielmassaan 
''La g\'eom\'etrie'', joka ilmestyi laajemman filosofisen teoksen liitteenä vuonna 1637. 
Tutkielma merkitsi \kor{analyyttisen geometrian} alkua, ja enteili yleisemminkin geometrian 
'algebralisoitumista' --- trendiä, joka erityisesti tietokoneiden aikakaudella on entisestään
vahvistunut. \index{Descartes, R.|av}}, jota merkitään $\{O,\vec a,\vec b\}$. Pistettä $O$
sanotaan koordinaatiston 
\index{origo}%
\kor{origoksi}.

\begin{figure}[H]
\setlength{\unitlength}{1cm}
\begin{center}
\begin{picture}(6,4)(-2,-0.2)
%\Thicklines
\put(0,0){\vector(3,2){4}} \put(0,0){\vector(-2,3){2}}
\put(-0.1,-0.5){$O$} \put(-1.7,2.8){$\vec a$} \put(3.5,2.6){$\vec b$}
\end{picture}
\end{center}
\end{figure}
\index{koordinaatti (pisteen)}%
Jos $P \in \Ekaksi$ ja $\Vect{OP} = x \vec a + y \vec b$, niin sanotaan, että \kor{pisteen}
$P$ \kor{koordinaatit} valitussa koordinaatistossa $\{O,\vec a,\vec b\,\}$ ovat 
$(x,y)$. Koordinaatiston ollessa kiinitetty voidaan koordinaattiparia $(x,y)$ haluttaessa 
pitää $P$:n 'nimenä', jolloin on lupa kirjoittaa muitta mutkitta
\[
P=(x,y).
\]
Näin on luotu kääntäen yksikäsitteinen vastaavuus $\Ekaksi \ \leftrightarrow \ \Rkaksi$.
\begin{Exa} Olkoon $T$ tason (aito) kolmio, jonka kärkikipisteet ovat $0,A,B$. Merkitään 
$\vec a=\Vect{OA}$, $\vec b=\Vect{OB}$. Tällöin $T$:n kärkipisteet, $T$:n sivujen 
keskipisteet ja $T$:n keskiö koordinaattistossa $\{O,\vec a,\vec b\,\}$ ovat
(vrt.\ Esimerkki \ref{kolmion keskipiste})
\begin{align*}
&\text{kärkipisteet:} \quad O=(0,0), \quad A=(1,0), \quad B=(0,1), \\
&\text{sivujen keskipisteet:} \quad O'=(1/2,1/2), \quad A'=(0,1/2), \quad B'=(1/2,0), \\
&\text{keskiö:} \quad C=(1/3,1/3). \quad\loppu
\end{align*}
\end{Exa}
Koordinaatistoon tarvitaan siis ensinnäkin referenssipiste $O$. Tästä nk. origosta valitaan
kaksi suuntaa, jotka määrääväät vektoreiden $\vec a, \vec b$ suunnat. Vielä päätetään, että 
mitattaessa etäisyyttä $O$:sta on pituusyksikkö $\alpha$ mentäessä $\vec a$:n suuntaan ja 
$\beta$ mentäessä $\vec b$:n suuntaan. Tässä $\alpha,\beta \in \R$ ja $\alpha,\beta>0$. Kun nyt
valitaan $\vec a$ ja $\vec b$ siten, että $|\vec a|=\alpha$ ja $|\vec b|=\beta$, ovat $\vec a$
ja $\vec b$ yksikäsitteisesti määrätyt ja koordinaatisto siis valmis. Origon kautta kulkevia
suoria
\begin{align*}
S_1=\{ P \in \Ekaksi \ | \ \overrightarrow{OP} = \lambda \vec a, \ \lambda \in \R \}, \\
S_2=\{ P \in \Ekaksi \ | \ \overrightarrow{OP} = \mu \vec b, \ \mu \in \R \},
\end{align*}
\index{suuntavektori} \index{koordinaattiakseli}%
joiden \kor{suuntavektoreina} ovat $\vec a, \vec b$, sanotaan \kor{koordinaattiakseleiksi}.
\begin{figure}[H]
\setlength{\unitlength}{1cm}
\begin{center}
\begin{picture}(10,6)(-3,-2)
\Thicklines
\put(0,0){\vector(3,2){4}} \put(0,0){\vector(-1,1){2}}
\thinlines
\put(0.3,-0.1){$O=(0,0)$} \put(-2.1,1.4){$\vec a$} \put(3.5,2.7){$\vec b$}
\put(-3,-2){\line(3,2){8}} \put(2,-2){\line(-1,1){5}}
\put(-1.8,2){$(1,0)$} \put(4,2.2){$(0,1)$}
\put(2.2,-2){$S_1$} \put(-2.5,-2){$S_2$}
\put(-0.07,-0.07){\piste} \put(3.93,2.6){\piste} \put(-2.07,1.93){\piste}
\end{picture}
\end{center}
\end{figure}

\subsection{Lineaarinen riippumattomuus}
\index{lineaarinen riippumattomuus|vahv}

Koordinaatiston kantavektoreista $\vec a,\vec b$ edellä tehdyt oletukset
($\vec a,\vec b \neq \vec 0$ ja $\vec a,\vec b$ eivät saman- tai vastakkaissuuntaiset) 
voidaan asettaa lyhyemmin ehtona
\begin{equation} \label{lin riippumattomat}
x \vec a + y \vec b = \vec 0 \qimpl x=y=0. \tag{$\star$}
\end{equation}
Jos tason vektoreilla on ominaisuus \eqref{lin riippumattomat}, niin sanotaan, että 
$\{\vec a,\vec b\,\}$ on \kor{lineaarisesti} \kor{riippumaton} (vektori)\kor{systeemi}
(= joukko) tai että $\vec a$ ja $\vec b$ ovat lineaarisesti riippumattomat. Tason
vektoriavaruuden $V$ kannaksi kelpaa siis mikä tahansa lineaarisesti riippumaton
vektorisysteemi $\{\vec a,\vec b\,\}$.

Jos vektorit $\vec a,\vec b$ eivät ole lineaarisesti riippumattomat, niin ne ovat
\index{lineaarinen riippuvuus}%
\kor{lineaarisesti riippuvat}. Ehdon  \eqref{lin riippumattomat} mukaisesti näin on, jos
\[
x\vec a + y\vec b=\vec 0 \quad \text{jollakin}\ (x,y) \neq (0,0).
\]
Tämä ehto puolestaan toteutuu täsmälleen kun joko
(i) $\vec a = \vec 0 \,\impl\, x\vec a+0\vec b=\vec 0$ $\forall x\in\R$,
(ii) $\vec b = \vec 0 \,\impl\, 0\vec a+y\vec b=\vec 0\ \forall y\in\R$, tai
(iii) $\vec a$ ja $\vec b$ ovat yhdensuuntaiset (= saman- tai vastakkaissuuntaiset, ol.\
$\vec a \neq \vec 0$ ja $\vec b \neq \vec 0\,$), jolloin 
$\exists\lambda\in\R,\ \lambda \neq 0$ siten, että $\vec a + \lambda\vec b = \vec 0$. 
\begin{Exa} Tason vektoreista $\vec a, \vec b$ tiedetään, että $\vec a-\vec b$ ja 
$\vec a+2\vec b$ ovat lineaarisesti riippuvat. Voidaanko päätellä, että myös $\vec a$ ja 
$\vec b$ ovat lineaarisesti riippuvat\,? \end{Exa}
\ratk Annetun tiedon mukaan on jollakin $(x,y) \neq (0,0)$
\[
x(\vec a-\vec b)+y(\vec a+2\vec b\,)=\vec 0.
\]
Vektorialgebran säännöillä tämä saadaan muotoon
\[
\vec 0 = (x+y)\vec a+(-x+2y)\vec b = x'\vec a+y'\vec b.
\]
Koska
\[
\begin{cases} \,x' = 0 \\ \,y' = 0 \end{cases} \qekv \begin{cases} \,\ x+y = 0 \\ -x+2y = 0 
\end{cases} \qekv \begin{cases} \,x =0 \\ \,y = 0 \end{cases}
\]
ja tiedetään, että $(x,y) \neq (0,0)$, niin päätellään, että $(x',y') \neq (0,0)$. Koska siis
jollakin $(x',y') \neq (0,0)$ on $x'\vec a+y'\vec b=\vec 0$, niin vastaus on: Voidaan\,! \loppu

\subsection{Koordinaatiston vaihto}
\index{koordinaatisto!a@koordinaatiston vaihto|vahv}%

Kun tason geometrisia tehtäviä ratkotaan vektorialgebran keinoin, on tehtävään sopivan 
koordinaatiston valinta yleensä ratkaisemisen ensimmäinen askel (vrt.\ Esimerkki 
\ref{kolmion keskipiste} edellä). Jos valittu koordinaatisto osoittautuu ratkaisun kuluessa
epämukavaksi, voidaan suorittaa \kor{koordinaatiston vaihto}. Koordinaatiston vaihto koostuu
\index{origon siirto} \index{kanta!b@kannan vaihto}%
\kor{origon siirrosta} ja vektoriavaruuden \kor{kannan vaihdosta}, tai pelkästään jommasta
kummasta. Kuvassa pisteen $P$ koordinaatit ovat $(x,y)$ koordinaatistossa $\{O,\vec a,\vec b\,\}$
ja $(x',y')$ koordinaatistossa $\{O',\vec c,\vec d\,\}$.
\begin{figure}[H]
\setlength{\unitlength}{1cm}
\begin{center}
\begin{picture}(10,6)(-3,-2)
\Thicklines
\put(-2,-2){\vector(1,1){3}} \put(-2,-2){\vector(-1,1){2}}
\put(4.5,2){\vector(-1,-4){0.5}} \put(4.5,2){\vector(1,0){3}}
\thinlines
\put(-2,-2){\vector(1,2){2.5}} \put(4.5,2){\vector(-4,1){4}}
\put(-1.9,-2.3){$O$} \put(-4.1,-0.6){$\vec a$} \put(1,0.4){$\vec b$}
\put(7.3,2.2){$\vec c$} \put(4.3,0){$\vec d$} \put(4.5,2.2){$O'$}
\put(0.5,3.2){$P=(x,y)=(x',y')$}
\end{picture}
\end{center}
\end{figure}
Vektorien yhteenlaskudiagrammin mukaan on
\[
\Vect{OP}=\Vect{OO'}+\Vect{O'P}\,\ \ekv\,\ x\vec a+y\vec b=\Vect{OO'}+(x'\vec c+y'\vec d\,).
\]
Olkoon tässä $\Vect{OO'}=\alpha\,\vec a+\beta\,\vec b$, eli $O'=(\alpha,\beta)$
koordinaatistossa $\{O,\vec a,\vec b\,\}$, ja
\[
\vec c=\lambda_1\vec a+\mu_1\vec b, \quad \vec d=\lambda_2\vec a+\mu_2\vec d,
\]
eli vektoreiden $\vec c,\vec d\,$ koordinaatit kannassa $\{\vec a,\vec b\}$ ovat
$(\lambda_1,\mu_1)$ ja $(\lambda_2,\mu_2)$. Tällöin em.\ yhtälö sievenee vektorialgebran
säännöillä muotoon
\[
(x-\lambda_1 x'-\lambda_2 y'-\alpha)\,\vec a+(y-\mu_1 x'-\mu_2 y'-\beta)\,\vec b\,=\,\vec 0.
\]
Koska vektorit $\vec a$ ja $\vec b$ ovat lineaarisesti riippumattomat, niin seuraa
\[
\begin{cases} \,x-\lambda_1 x'-\lambda_2 y'-\alpha=0\\ \,y-\mu_1 x'-\mu_2 y'-\beta =0 \end{cases}
\ \ekv \quad
\begin{cases} \,\lambda_1 x'+\lambda_2 y'=x-\alpha \\ \,\mu_1 x'+\mu_2 y'=y-\beta \end{cases}
\]
Ratkaisemalla tästä $(x',y')$ koordinaattien $(x,y)$ avulla --- yhtälöryhmä ratkeaa 
aina kun $\vec c\,$ ja $\vec d$ ovat lineaarisesti riippumattomat --- saadaan selville 
\index{koordinaattimuunnos}%
\kor{koordinaattimuunnoksen} $(x,y) \ext (x',y')$ laskukaavat. Yhtälöryhmästä nähdään myös 
suoraan, miten käänteinen muunnos $(x',y') \ext (x,y)$ on laskettava. 
\begin{Exa} Olkoon
\[
\Vect{OO'}=\vec a - \vec b, \quad \vec c=2\vec a+\vec b, \quad \vec d=\vec a-2\vec b.
\]
Em.\ laskun kulku on tällöin
\begin{align*}
&x\vec a+y\vec b\ =\ (\vec a-\vec b\,)+x'(2\vec a+\vec b\,)+y'(\vec a-2\vec b\,) \\
&\qekv (x-2x'-y'-1)\,\vec a + (y-x'+2y'+1)\,\vec b = \vec 0 \\
&\qekv \begin{cases} \,2x'+y'=x-1 \\ \,x'-2y'=y+1 \end{cases}
\end{align*}
Ratkaisemalla saadaan koordinaattimuunnoksen laskukaavoiksi
\[
\begin{cases} \,x=2x'+y'+1\\ \,y=x'-2y'-1 \end{cases}\ \ekv \quad
\begin{cases} \,x'=\frac{1}{5}(2x+y-1) \\ \,y'=\frac{1}{5}(x-2y-3) \end{cases} \loppu
\]
\end{Exa}

\subsection{Dimensio. Aliavaruus}
\index{dimensio|vahv} \index{aliavaruus|vahv}

Koska tason vektoriavaruuden kannassa on oltava kaksi lineaarisesti riippumatonta vektoria, niin
sanotaan, että $V$ on $2$-\kor{ulotteinen} vektoriavaruus tai että $V$:n \kor{dimensio}
(ulotteisuus) on $2$. Merkitään
\[
\text{dim } V=2.
\]
Myös koodinaattiavaruus $\Rkaksi$ on vektoriavaruutena $2$-ulotteinen. Nimittäin jos merkitään
\[
\ma=(1,0), \quad \mb=(0,1),
\]
niin $\Rkaksi$:n vektorialgebran mukaan on $(x,y)=x\ma + y\mb$, eli jokainen
$\mv v = (x,y) \in \Rkaksi$ voidaan esittää yksikäsitteisesti $\ma$:n ja $\mb$:n
lineaariyhdistelynä. Siis $\{\ma,\mb\}$ on $\Rkaksi$:n kanta, ja koska kannassa on kaksi
vektoria, niin $\text{dim}(\Rkaksi)=2$.

Tason tai $\Rkaksi$:n vektoreista voidaan muodostaa myös $1$-\kor{ulotteisia} vektoriavaruuksia.
Tason vektoreiden tapauksessa nämä ovat kaikki ilmaistavissa jonkin vektorin
$\vec a \in V, \  \vec a \neq \vec 0$ avulla muodossa
\[
W=\{\vec v = x\vec a,\ x\in\R\}.
\]
Koska ilmeisesti pätee
\begin{align*}
\vec v_1, \vec v_2 \in W &\qimpl \vec v_1 + \vec v_2 \in W, \\
\vec v \in W             &\qimpl \lambda\vec v \in W \ \ \forall\,\lambda \in \R,
\end{align*}
on $W$ itsekin vektoriavaruus. Sen kantaan tarvitaan vain yksi vektori, esim $\vec a$, joten 
$\text{dim } W=1$. Koska myös $W \subset V$, sanotaan, että $W$ on $V$:n (aito) \kor{aliavaruus}
(engl.\ subspace) ja että $\vec a$
\index{virittää (aliavaruus)}%
\kor{virittää} (engl.\ span) $W$:n. Aliavaruuden $W$ geometrinen vastine on origon kautta
kulkeva suora $S\subset\Ekaksi$. Tämä on
\index{pisteavaruus}%
\kor{$1$-ulotteinen pisteavaruus}.
\begin{figure}[H]
\setlength{\unitlength}{1cm}
\begin{center}
\begin{picture}(10,3)
\put(0,0){\line(4,1){10}}
\put(3.9,0.9){$\bullet$} \put(3.9,0.5){$O$}
\Thicklines \put(4,1){\vector(4,1){4}} \thinlines
\put(7.9,1.5){$\vec a$}
\curve(1,0.25,1.3,0.1,1.6,0.1) \put(1.8,0){$S\leftrightarrow W$}
\end{picture}
\end{center}
\end{figure}
Vastaavuus
\[
P \in S \ \leftrightarrow \ \Vect{OP} = x \vec a \in W \ \leftrightarrow \ x \in \R
\]
synnyttää kääntäen yksikäsitteisen vastaavuuden $\R$:n ja pisteavaruuden $S$ välille. Lukujen 
geometrisointi lukusuoran pisteiksi (vrt.\ Luku \ref{geomluvut}) perustui juuri tähän 
vastaavuuteen.
    
\Harj
\begin{enumerate}

\item \label{H-II-1: perusteluja}
a) Perustele vektorialgebran säännöt (V4)--(V6) geometrisesti. \vspace{1mm}\newline
b) Vektoreille pätee kolmioepäyhtälö
$\abs{\vec a+\vec b} \le \abs{\vec a}+\abs{\vec b}$. Päättele geometriaan enempää
turvautumatta, että pätee myös $\,\abs{\vec a}-\abs{\vec b} \le \abs{\vec a+\vec b}$. 
%\vspace{1mm}\newline

\item
Tason vektoreista $\vec a,\vec b$ oletetaan, että $\vec a,\vec b\neq\vec 0$ ja että
$\vec a$ ja $\vec b$ eivät ole yhdensuuntaiset. Millä vektorin $\vec c$ arvoilla voidaan
vektoreita $\vec a+\vec b$, $\vec a+\vec c$ ja $\vec b+2\vec c\,$ 'siirtelemällä' muodostaa
kolmio?

\item
Todista Harjoitustehtävän \ref{geomluvut}:\ref{H-II-1: yhdenmuotoisuus}b väittämä
vektorialgebran avulla.

\item
Nelikulmiossa $ABCD$ on $\Vect{AB} = \vec{a},\ \Vect{AD} = \vec{b}$ ja
$\Vect{BC} = (1/2)(\vec{a} + \vec{b}\,)$. Laske vektorien $\vec{a}$ ja $\vec{b}$
avulla vektori $\Vect{AE}$, missä $E$ on nelikulmion lävistäjien leikkauspiste.

\item
Suunnikkaassa ABCD kärki $A$ yhdistetään sivun $CD$ keskipisteeseen $P$ ja kärki $B$ sivun
$AD$ keskipisteeseen $R$. Yhdysjanat leikatkoot pisteessä $X$. Lausu vektori $\Vect{AX}$
vektoreiden $\vec u=\Vect{AB}$ ja $\vec v=\Vect{AD}$ avulla.

\item
Kolmiossa $ABC$ merkitään $\vec a=\Vect{AB},\ \vec b=\Vect{AC}$. \ a) Päättele geometrisesti,
että vektori $\vec c=\abs{\vec b}\vec a+\abs{\vec a}\vec b$ puolittaa kulman $BAC$. \
b) Piste $D$ on janalla $BC$ ja jana $AD$ puolittaa kulman $BAC$. Todista vektorilaskulla
\kor{kulmanpuolittajalause}: Janojen $BD$ ja $DC$ pituuksien suhde
$=\abs{\vec a}/\abs{\vec b}$.

\item
Olkoot pisteet $M$ ja $N$ kolmioiden $ABC$ ja $DEF$ keskiöt. Näytä, että \newline
$\Vect{AD}+\Vect{BE}+\Vect{CF}=3\Vect{MN}$.

\item
Kolmion $ABC$ sivut $BC$, $CA$ ja $AB$ jakautuvat pisteissä $A^*$, $B^*$ ja $C^*$ suhteessa
$m:n$. Todista, että kolmioiden $ABC$ ja $A^*B^*C^*$ keskiöt yhtyvät.

\item
Olkoon $\{\vec a,\vec b\,\}$ tason vektoriavaruuden kanta ja olkoon $\vec u=2\vec a+3\vec b$,
$\vec v=-3\vec a+2\vec b$ ja $\vec w=-\vec a-2\vec b$. Määritä vektorin $2\vec u-\vec v+\vec w$
koordinaatit kannassa $\{\vec a,\vec b\,\}$. Piirrä kuva!

\item
Vektorit $\vec a,\vec b$ muodostavat tason vektoriavaruuden $V$ kannan. Näytä, että myös 
vektorit $\vec u=\vec a+\vec b$, $\vec v=\vec a-2\vec b$ muodostavat kannan. Laske 
vektorin $\vec w=\vec a-\vec b$ koor\-di\-naa\-tit tässä kannassa.

\item
Kolmiossa $ABC$ piste $O$ puolittaa sivun $AB$, piste $E_1$ jakaa sivun $BC$ suhteessa $1:2$
ja piste $E_2$ sivun $CA$ suhteessa $1:3$. Määritä kolmion kärkipisteiden koordinaatit siinä
koordinaatistossa, jonka origo on piste $O$ ja kantavektorit ovat $\Vect{OE}_1$ ja
$\Vect{OE}_2\,$.

\item 
Jos $\vec a,\vec b$ ovat lineaarisesti riippumattomat tason vektorit, niin millä $t$:n arvoilla
($t\in\R$) vektorit $\vec a-t\vec b$ ja $t\vec a-2\vec b$ ovat myös lineaarisesti
riippumattomat\,?

\item
Pisteen $P$ koordinaatit ovat $(x,y)$ koordinaatistossa $\{O,\vec a,\vec b\,\}$ ja $(x',y')$ 
koordinaatistossa $\{O',\vec c,\vec d\,\}$. Johda koordinaattimuunnoksen laskukaavat molempiin
suuntiin, kun \newline
a) \ $O'=O$, $\,\vec c=\vec a+\vec b\,$ ja $\,\vec d=\vec a-\vec b$. \newline
b) \ $\Vect{O'O}=\vec c+2\vec d$, $\,\vec c=2\vec a+3\vec b\,$ ja $\,\vec d=-\vec a+2\vec b$. 

\item
Olkoon $\vec a=\Vect{OA}$ ja $\vec b=\Vect{OB}$. Johda kordinaattimuunnoksen laskukaavat 
koordinaatistosta $\{O,\vec a,\vec b\,\}$ koordinaatistoon $\{O',\vec c,\vec d\,\}$,
kun $O'=$ kolmion $OAB$ keskiö ja $\vec c=\Vect{O'A},\ \Vect d=\Vect{O'B}$. Mitkä ovat kolmion
kärkipisteiden ja sivujen keskipisteiden koordinaatit jälkimmäisessä koordinaatistossa?

\end{enumerate}