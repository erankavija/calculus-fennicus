\section{Suorien, tasojen ja pintojen geometriaa} \label{suorat ja tasot}
\sectionmark{Suorat, tasot ja pinnat}
\alku

Kun geometrian tehtäviä ratkotaan algebran keinoin käyttäen hyväksi (yleensä karteesista) 
koordinaatistoa, puhutaan
\index{analyyttinen geometria}%
\kor{analyyttisestä geometriasta}. Vektorit, skalaari- ja 
ristituloineen, tarjoavat moniin tehtäviin kätevän apuneuvon.

Jos $O$ on Euklidisen avaruuden ($\Ekaksi$ tai $\Ekolme$) origo ja $P$ on ko.\ avaruuden 
piste, sanotaan vektoria $\overrightarrow{OP}$ tästedes pisteen $P$
\index{paikkavektori}%
\kor{paikkavektoriksi} ja merkitään
\[
\vec r = \overrightarrow{OP} \quad \text{(paikkavektori)}.
\]
Käytetään myös vastaavuusmerkintää
\[ 
\vec r \vastaa P \ \ekv \ \vec r = \overrightarrow{OP}.
\]
\subsection{Suora}
\index{suora|vahv}

Jos $P_0=(x_0,y_0) \vastaa \vec r_0 = x_0\vec i + y_0\vec j$ on euklidisen tason piste ja 
$\vec v=\alpha\vec i + \beta\vec j$ on tason vektori, $\vec v \neq \vec 0$, niin pistejoukko
\index{parametri(sointi)!a@suoran|(}%
\[
S=\{P \in \Ekaksi \ | \ P \vastaa \vec r = \vec r_0 + t\vec v \quad \text{jollakin} \ t \in \R\}
\]
on suora, jonka
\index{suuntavektori}%
\kor{suuntavektori} $=\vec v$ \ ja joka kulkee pisteen $P_0$ kautta.
\begin{figure}[H]
\setlength{\unitlength}{1cm}
\begin{center}
\begin{picture}(9.5,5.5)(-1,-0.5)
\put(0,0){\vector(0,1){2}} \put(0,0){\vector(2,1){8}}
\Thicklines
\put(0,2){\vector(4,1){3}}
\thinlines
\path(-1,1.75)(9,4.25)
\put(-0.1,1.9){$\bullet$} \put(-0.1,-0.1){$\bullet$} \put(7.9,3.9){$\bullet$}
\put(-0.1,-0.5){$O$} \put(0.2,1.6){$\vec r_0$} \put(-0.1,2.2){$P_0$} \put(2.7,2.9){$\vec v$} 
\put(8.2,3.5){$P$}
\put(7.5,3.2){$\vec r$}
\end{picture}
\end{center}
\end{figure}
Sanotaan, että yhtälö
\begin{equation} \label{suora par}
\vec r = \vec r_0 + t\vec v
\end{equation}
on suoran $S$ \kor{parametrimuotoinen yhtälö} tai \kor{parametrisointi} (parametrisaatio) ja
että $t$ on \kor{parametri}. Yleisemmin näillä termeillä tarkoitetaan, että 
\index{parametri(sointi)!a@suoran|)}%
tarkasteltavan pistejoukon (tässä suoran) jokainen piste saadaan annetusta esitysmuodosta
jollakin parametrin arvolla. Tässä tapauksessa jokaista suoran pistettä vastaa yksikäsitteinen 
parametrin $t$ arvo ja kääntäen, ts.\ parametrisointi \eqref{suora par} luo kääntäen 
yksikäsitteisen vastaavuuden $S\ \vast\ \R$.

Kun kirjoitetaan $\vec r = x \vec i + y \vec j$, niin yhtälölle \eqref{suora par} saadaan 
\kor{koordinaattimuoto}
\begin{equation} \label{suora komp taso}
\left\{ \begin{array}{ll}
x=x_0 + \alpha t, \\
y=y_0 + \beta t.
\end{array} \right.
\end{equation}
Eliminoimalla $t$ seuraa tästä $x$:n ja $y$:n välinen riippuvuus
\begin{equation} \label{suora perus}
ax+by+c=0,
\end{equation}
missä $a=\beta$, $b=-\alpha$ ja $c=-\beta x_0+\alpha y_0$. Tämä on suoran yhtälön
\index{perusmuoto!a@tason suoran yhtälön}%
\kor{perusmuoto} (ei-parametrinen muoto). Perusmuotoisen yhtälön \eqref{suora perus}
voi ilmaista myös vektoreiden avulla, sillä kun asetetaan
\[
\vec n = a \vec i + b \vec j \ \ (= \beta\vec i-\alpha\vec j\,),
\]
niin yhtälö on sama kuin
\begin{equation} \label{suora norm}
\vec n \cdot (\vec r - \vec r_0) = 0.
\end{equation}
Tässä $\vec n$ on $S$:n
\index{normaali(vektori)!a@suoran, käyrän}%
\kor{normaalivektori}, ts.\ $S$:n suuntavektoria $\vec v$ vastaan
kohtisuora ($\vec 0$:sta poikkeava) vektori. Normaalivektorin avulla suoran yhtälö voidaan
siis kirjoittaa suoraan ehtona $\vec r - \vec r_0 \perp \vec n$, kuten geometrisestikin
on helppo päätellä.
%\begin{figure}[H]
%\begin{center}
%\import{kuvat/}{kuvaII-17.pstex_t}
%\end{center}
%\end{figure}
\begin{figure}[H]
\setlength{\unitlength}{1cm}
\begin{center}
\begin{picture}(10,6)(-1,-0.5)
\put(0,0){\vector(1,2){2}} \put(0,0){\vector(2,1){6}} \put(2,4){\vector(1,4){0.3}}
\put(-1,4.75){\line(4,-1){10}} \put(2,4){\vector(4,-1){4}}
\path(1.84,4.04)(1.88,4.2)(2.04,4.16)
\put(5.8,3.2){$\vec r-\vec r_0$}
\put(-0.1,-0.5){$O$} \put(9.3,2.1){$S$}
\put(2,3.55){$\vec r_0$} \put(2.5,5){$\vec n$} \put(5.95,2.55){$\vec r$} 
\end{picture}
\end{center}
\end{figure}
\begin{Exa}
Suoran yhtälöstä $3x-2y+5=0$ nähdään heti, että suoran normaalivektori on
$\vec n=3\vec i -2\vec j$. Siis suoran eräs suuntavektori on $\vec v = 2\vec i + 3\vec j$. 
Koska esimerkiksi $P_0=(1,4)$ on suoralla, saadaan suoralle parametrisoitu esitys
\eqref{suora par}, missä
\[
\vec r_0 = \vec i + 4\vec j, \quad \vec v = 2\vec i + 3 \vec j.
\]
Parametrisoinnin koordinaattimuoto \eqref{suora komp taso} on tässä tapauksessa
\[
\left\{ \begin{array}{ll}
x=1+2t, \\
y=4+3t.
\end{array}\right. 
\]
Mainittu piste $P_0$ vastaa siis tässä parametrin arvoa $t=0$. \loppu
\end{Exa}
Parametrisointi \eqref{suora par} toimii myös avaruussuoralle. Jos
$\vec r_0=x_0\vec i + y_0 \vec j + z_0 \vec k$ ja 
$\vec v = \alpha \vec i + \beta \vec j + \gamma \vec k$, niin koordinaattimuotoiset
suoran yhtälöt ovat
\begin{equation} \label{suora komp avaruus}
\left\{ \begin{array}{ll}
x = x_0 + \alpha t, \\ 
y = y_0 + \beta t, \\
z = z_0 + \gamma t.
\end{array} \right.
\end{equation}
Eliminoimalla $t$ jää jäljelle kaksi ei-parametrista yhtälöä.
\begin{Exa}
Suora kulkee pisteiden $A=(1,1,1)$ ja $B=(1,-2,3)$ kautta. Mikä on suoran yhtälö?
\end{Exa}
\ratk Eräs suoran suuntavektori on
\[
\vec v = \overrightarrow{AB} = -3\vec j + 2\vec k
\]
ja siis eräs parametriesitys on muotoa \eqref{suora par}, missä esimerkiksi
\[
\vec r_0 = \overrightarrow{OA} = \vec i + \vec j + \vec k.
\]
Koordinaattimuodossa
\[
\left\{ \begin{array}{ll}
x=1, \\
y=1-3t, \\
z=1+2t.
\end{array} \right.
\]
Eliminoimalla $t$ saadaan saadaan suoran ei-parametrisiksi yhtälöiksi
\[
\left\{ \begin{array}{ll}
x &= 1, \\
2y+3z &=5.
\end{array} \right.
\]
Nämä voidaan kirjoittaa myös muotoon
\[
\left\{ \begin{array}{ll}
\vec n_1 \cdot (\vec r - \vec r_0)=0, \\
\vec n_2 \cdot (\vec r - \vec r_0)=0,
\end{array} \right.
\]
missä $\vec n_1 = \vec i$ ja $\vec n_2=2\vec j + 3\vec k$ ovat suoran normaalivektoreita.
Tämä vastaa tason suoran yhtälöä \eqref{suora norm}. \loppu
%\begin{figure}[H]
%\begin{center}
%\import{kuvat/}{kuvaII-18.pstex_t}
%\end{center}
%\end{figure}
\begin{figure}[H]
\setlength{\unitlength}{1cm}
\begin{center}
\begin{picture}(10,5.5)(-1,0)
\put(0,0){\vector(1,2){2}} \put(0,0){\vector(2,1){6}} 
\put(2,4){\vector(1,4){0.3}} \put(2,4){\vector(-2,-1){1}}
\put(-2,5){\line(4,-1){11}} \put(2,4){\vector(4,-1){4}}
\path(1.84,4.04)(1.88,4.2)(2.04,4.16)
\path(1.84,4.04)(1.72,3.98)(1.88,3.94)
\Thicklines 
\put(1,4.25){\vector(-4,1){2}} 
\thinlines
\put(0.93,4.18){$\scriptstyle{\bullet}$} \put(-1.07,4.68){$\scriptstyle{\bullet}$}
\put(1,4.4){$A$} \put(-1,4.9){$B$} \put(-0.95,4.25){$\vec v$} 
\put(5.8,3.2){$\vec r-\vec r_0$}
\put(-0.1,-0.5){$O$} \put(9.3,2.1){$S$}
\put(2,3.55){$\vec r_0$} \put(5.95,2.55){$\vec r$} 
\put(2.5,5){$\vec n_1$} \put(0.9,3.1){$\vec n_2$}
\end{picture}
\end{center}
\end{figure}
\begin{Exa}
Hävittäjä A nousee kentältä, joka on pisteessä $(5,7,0)$ (yksikkö=km), suuntaan 
$-\vec i - 2\vec j + \vec k$ ja hävittäjä B nousee toiselta kentältä, joka on pisteessä 
$(1,-9,0)$, suuntaan $-\vec i + \vec j + 2\vec k$. Kuinka lähelle toisiaan koneet voivat joutua
(pahin mahdollinen skenaario)\,?
\end{Exa}
\ratk Tässä on kyse nk. kahden suoran ongelmasta, jossa on etsittävä suorien lyhin etäisyys $d$.
Suorat ovat
\begin{align*}
S_1: \ \vec r &= 5\vec i + 7\vec j + t(-\vec i - 2\vec j + \vec k) =\vec r_1+t\vec v_1, \\
S_2: \ \vec r &= \vec i - 9\vec j + s(-\vec i + \vec j + 2\vec k)\ =\vec r_2+s\vec v_2.
\end{align*}
Lyhin etäisyys on $d=\abs{\overrightarrow{PQ}}$, missä $P \in S_1$, $Q \in S_2$ ja 
vektori $\Vect{PQ}$ on molempia suoria vastaan kohtisuora, ts. 
\[
\Vect{PQ}\cdot\vec v_1=\Vect{PQ}\cdot\vec v_2=0.
\]
Kun tässä $P \vastaa \vec r(t)$ ja $Q \vastaa \vec r(s)$ esitetään parametrisointien
mukaisesti, niin
\begin{align*}
\Vect{PQ} &= \vec r_2-\vec r_1+s\vec v_2-t\vec v_1 \\
                    &= (-4+t-s)\vec i + (-16+2t+s)\vec j + (-t+2s)\vec k,
\end{align*}
joten saadaan yhtälöryhmä
\begin{align*} 
&\begin{cases}
\,-1\cdot(-4+t-s)-2\cdot(-16+2t+s)+1\cdot(-t+2s) = 0 \\
\,-1\cdot(-4+t-s)+1\cdot(-16+2t+s)+2\cdot(-t+2s) = 0
\end{cases} \\
&\quad\qekv \begin{cases}
             \,6t-s = 36 \\ \,t-6s = -12
            \end{cases}
\end{align*}
Ratkaisemalla $t,s$ saadaan
\[
t=\frac{228}{35}\,, \quad s=\frac{108}{35}\,.
\]
Näillä arvoilla on
\[
\overrightarrow{PQ} = \frac{1}{35}(-20\vec i+4\vec j-12\vec k).
\]
Koska komponentin $\vec k$ kerroin on negatiivinen, lentää kone A kohtaamistilanteessa 
korkeammalla. Koneiden etäisyys on siis pahimman skenaarion mukaan
\[
d=\abs{\overrightarrow{PQ}} = 4/ \sqrt{35} \approx 0.676 = 676 \text{ m},
\]
ja pisteet $P$ ja $Q$ saadaan erikseenkin suorien parametriesityksistä:
\begin{align*}
S_1:\ \vec r\left(t=\frac{228}{35}\right) &\vastaa P \approx (-1.514,-6.029,6.514), \\
S_2:\ \vec r\left(s=\frac{108}{35}\right) &\vastaa Q \approx (-2.086,-5.914,6.171). \loppu
\end{align*}

\subsection{Taso}
\index{taso|vahv}

Tarkastellaan avaruuden tasoa $T$, joka kulkee pisteiden $P_0=(x_0,y_0,z_0)$,
$P_1=(x_1,y_1,z_1)$ ja $P_2=(x_2,y_2,z_2)$ kautta. Oletetaan, että pisteet eivät ole
samalla suoralla, jolloin vektorit
\begin{align*}
\vec v_1 &= \overrightarrow{P_0P_1} = \alpha_1 \vec i + \beta_1 \vec j + \gamma_1 \vec k, \\
\vec v_2 &= \overrightarrow{P_0P_2} = \alpha_2 \vec i + \beta_2 \vec j + \gamma_2 \vec k
\end{align*}
ovat lineaarisesti riippumattomat. Näiden, tason $T$
\index{suuntavektori}%
\kor{suuntavektoreiden} avulla saadaan
tasolle parametrisoitu esitysmuoto
\begin{equation} \label{taso par}
\vec r = \vec r_0 + t_1 \vec v_1 + t_2 \vec v_2, \quad t_1, t_2 \in \R.
\end{equation}
\index{parametri(sointi)!b@tason}%
Tässä siis reaalisia parametreja on kaksi. Jos parametri halutaan nähdä 'yhtenä', niin
parametriksi on tulkittava $\mathbf{t}=(t_1,t_2)\in\Rkaksi$. Parametrisointi \eqref{taso par}
luokin kääntäen yksikäsitteisen vastaavuuden $T\vast\Rkaksi$, ja $\{P_0,\vec v_1,\vec v_2\}$
voidaan tulkita $T$:n koordinaatistoksi, vrt.\ Luku \ref{tasonvektorit}. 

Yhtälön \eqref{taso par} koordinaattimuoto on
\[
\left\{ \begin{array}{lll}
x &= x_0 + \alpha_1 t_1 + \alpha_2 t_2, \\
y &= y_0 + \beta_1 t_1 + \beta_2 t_2,   \\
z &= z_0 + \gamma_1 t_1 + \gamma_2 t_2.
\end{array} \right.
\]
Parametrien eliminointi käy kuitenkin helpommin vektorimuodosta \eqref{taso par}, kun otetaan 
käyttöön vektori
\[
\vec n = \vec v_1 \times \vec v_2.
\]
Tämä on molempia vectoreita $\vec v_1, \vec v_2$ vastaan kohtisuora, eli kyseessä on
tason
\index{normaali(vektori)!b@tason, hypertason}%
\kor{normaalivektori}. Kertomalla yhtälö \eqref{taso par} skalaarisesti vektorilla
$\vec n$ ja huomioimalla, että $\vec n \cdot \vec v_1 = \vec n \cdot \vec v_2 = 0$, seuraa
tasolle yhtälö
\begin{equation} \label{taso norm}
\vec n \cdot (\vec r - \vec r_0)=0,
\end{equation}
joka siis on aivan samaa muotoa kuin tason suoran yhtälö \eqref{suora norm}. Jos
\[
\vec n = a\vec i + b \vec j + c\vec k, \quad d=-\vec n \cdot \vec r_0\,,
\]
niin \eqref{taso norm} on edelleen sama kuin
\begin{equation} \label{taso perus}
ax+by+cz+d=0.
\end{equation}
\index{perusmuoto!b@avaruustason yhtälön}%
Tämä on tason yhtälön \kor{perusmuoto}. Perusmuodosta nähdään siis suoraan, mikä on tason
normaalivektori.\footnote[2]{Tason \kor{normaali} on jokainen avaruussuora, jonka suuntavektori
= tason normaalivektori. Termiä käytetään usein myös normaalivektorista puhuttaessa. Esim:
'Yksikkövektori $\vec n$ on tason normaali.'}
\begin{Exa}
Taso $T_1$ kulkee pisteiden $A=(1,1,2)$, $B=(0,3,-1)$ ja $C=(-2,-2,-3)$ kautta. Taso $T_2$ 
kulkee pisteen $(-1,-1,-1)$ kautta ja sen normaalivektori on 
$\vec n_2 = \vec i - \vec j - \vec k$. Määrää tasojen yhtälöt perusmuodossa \eqref{taso perus}
ja tasojen leikkaussuoran parametriesitys muodoissa \eqref{suora par} ja
\eqref{suora komp avaruus}.
\end{Exa}
\ratk Tason $T_2$ yhtälö on
\[
\vec n_2 \cdot (\vec r + \vec i + \vec j + \vec k) = 0,
\]
eli
\[
T_2: \quad x-y-z-1=0.
\]
Tason $T_1$ normaalivektori on
\[
\vec n_1 = \Vect{AB} \times \Vect{AC} =
\left| \begin{array}{ccc}
\vec i & \vec j & \vec k \\
-1 & 2 & -3 \\
-3 & -3 & -5
\end{array} \right| =
-19\vec i + 4\vec j + 9\vec k,
\]
joten valitsemalla esimerkiksi $\vec r_0 \vastaa A$ saadaan tason $T_1$ yhtälöksi
\[
T_1: \quad \vec n_1 \cdot (\vec r - \vec i - \vec j - 2\vec k) = 0 \qekv -19x+4y+9z-3=0.
\]
Leikkaussuoran $S$ suuntavektori on
\[
\vec v = \vec n_1 \times \vec n_2 =
\left| \begin{array}{ccc}
\vec i & \vec j & \vec k \\
-19 & 4 & 9 \\
1 & -1 & -1 
\end{array} \right| =
5 \vec i - 10 \vec j + 15 \vec k.
\]
Suuntavektoriksi kelpaa myös $\frac{1}{5}\vec v = \vec i - 2 \vec j + 3 \vec k$. Vielä tarvitaan
yksi suoran piste $P_0 \vastaa \vec r_0$. Tämä löydetään esim. tasojen $T_1, T_2$ ja (sopivasti
valitun) kolmannen tason $T_3$ leikkauspisteenä. Valitaan kolmanneksi tasoksi $yz$-taso
\[
T_3: \ x=0,
\]
jolloin $P_0=(x_0,y_0,z_0)$ ratkeaa (jos ratkeaa) yhtälöryhmästä
\[
\left\{ \begin{array}{ll}
-19x_0+4y_0+9z_0 &= \ 3 \\
x_0-y_0-z_0 &= \ 1 \\
x_0 &= \ 0
\end{array} \right.
\]
Ratkaisu on $P_0=(0,-\frac{12}{5},\frac{7}{5})$, ja suoran $S$ parametriesitys näin muodoin 
\[
\vec r = \frac{1}{5}(-12 \vec j + 7 \vec k) + t(\vec i - 2 \vec j + 3 \vec k),
\]
eli koordinaattimuodossa
\[
S: \quad \left\{ \begin{array}{ll}
x &= \ t, \\
y &= \ -\frac{12}{5}-2t, \\
z &= \ \frac{7}{5} + 3t.
\end{array} \right. \loppu
\]
\begin{Exa} \label{pisteen etäisyys tasosta} Määritä annetun pisteen $P=(x_0,y_0,z_0)$ etäisyys
tasosta \newline $T:\ ax+by+cz+d=0$.
\end{Exa}
\ratk Jos $Q$ on pistettä $P$ lähinnä oleva piste tasolla $T$, niin ilmeisesti jollakin $t\in\R$
on $\overrightarrow{PQ}=t\vec n$, missä $\vec n=a\vec i+b\vec j+c\vec k$ on tason 
normaalivektori. Siis $Q=(x_0+ta,y_0+tb,z_0+tc)$ jollakin $t$. Koska $Q \in T$, niin $T$:n
yhtälön perusteella
\[
0\,=\,a(x_0+ta)+b(y_0+tb)+c(y_0+tc)+d\,=\,ax_0+by_0+cz_0+d+t\abs{\vec n}^2.
\]
Ratkaisemalla tästä $t$ saadaan etäisyyden laskukaavaksi
\[
h\,=\,\abs{t}{\abs{\vec n}}\,=\,\frac{\abs{ax_0+by_0+cz_0+d}}{\sqrt{a^2+b^2+c^2}}\,.
\]
Tason pisteen $(x_0,y_0)$ etäisyys suorasta $S:\ ax+by+c=0$ saadaan samasta kaavasta 
asettamalla ensin $c=0$ ja sitten $d$:n tilalle $c$. \loppu

\subsection{Ympyrä ja pallo}
\index{ympyrä|vahv} \index{pallo(pinta)|vahv}

Ympyrä on (harpin olomuodossa) euklidisen tason alkuperäisiä olioita. Jos ympyrän keskipiste
on $P_0\vastaa\vec r_0$ ja säde $R$, niin \kor{ympyrän} (ympyräviivan) \kor{yhtälö} on
\[
\abs{\vec r-\vec r_0} = R.
\]
Tämä on myös \kor{pallon} (pallopinnan)\footnote[2]{Ympyrällä ja pallolla saatetaan tarkoittaa 
myös euklidisen avaruuden 'täyteistä' joukkoa
\[
K = \{P\vastaa\vec r\, \mid \abs{\vec r-\vec r_0} \le R\}.
\]
Tämän täsmällisempi nimitys tasossa on \kor{kiekko} (engl.\ disc; ympyrä = circle), avaruudessa
\kor{kuula} (engl.\ ball; pallopinta = sphere). \index{kiekko|av} \index{kuula|av}} yhtälö.
Neliöitynä ja koordinaattien avulla kirjoitettuna yhtälöt ovat
\begin{align*}
\text{Ympyrä:} \quad &(x-x_0)^2+(y-y_0)^2=R^2. \\
\text{Pallo:}  \quad &(x-x_0)^2+(y-y_0)^2+(z-z_0)^2=R^2.
\end{align*}
\index{tangentti (käyrän)}%
\kor{Ympyrän tangentti} on suora, jolla on ympyrän kanssa täsmälleen yksi yhteinen piste. 
Sanotaan, että tangentti \kor{sivuaa} ympyrää ko.\ pisteessä. Jos sivuamispiste on annettu
piste $Q$, niin tangentti konstruoidaan geometrisesti piirtämällä ensin suora $S_1$ pisteen
$Q$ ja ympyrän keskipisteen kautta, jolloin tangentti $S_2$ on tätä vastaan kohtisuora. 
Sanotaan, että suora $S_1$ on
\index{normaali(vektori)!a@suoran, käyrän}%
\kor{ympyrän normaali} pisteessä $Q$ tai että suora $S_1$
leikkaa ympyrän \kor{kohtisuorasti}. Yleisemminkin voidaan ympyräviivan ja suoran välinen 
\index{leikkauskulma}%
\kor{leikkauskulma} määritellä ko.\ suoran ja leikkauspisteeseen asetetun tangentin
suuntavektoreiden välisenä kulmana.

Vastaavalla tavalla kuin ympyrän tangentti määritellään
\index{tangenttitaso}%
\kor{pallon tangenttitaso}. Pallon keskipisteen kautta kulkeva suora on
\index{normaali(vektori)!c@pinnan}%
\kor{pallon normaali}, eli se on kohtisuora tangenttitasoa
vastaan pallon ja suoran leikkauspisteessä. Jos pallolla ja tasolla $T$ on enemmän kuin yksi 
yhteinen piste, on yhteisten pisteiden joukko
\index{avaruusympyrä}%
\kor{avaruusympyrä} (ympyrä tasolla $T$). Jos $T$
kulkee pallon keskipisteen kautta, sanotaan leikkausviivaa pallon
\index{isoympyrä}%
\kor{isoympyräksi}.
\kor{Avaruusympyrän tangentti} on ympyrän kanssa samassa tasossa oleva, ympyrää sivuava
avaruussuora.
\begin{Exa} Pallon keskipiste on $(x_0,y_0,z_0)$ ja säde $=R$. Millä ehdolla taso 
$T:\ ax+by+cz+d=0$ sivuaa palloa\,?
\end{Exa}
\ratk Taso $T$ sivuaa palloa täsmälleen, kun pallon keskipisteen etäisyys tasosta $=R$.
Esimerkin \ref{pisteen etäisyys tasosta} perusteella ehto on
\[
\abs{ax_0+by_0+cz_0+d\,} =R\sqrt{a^2+b^2+c^2}. \loppu
\]
\begin{Exa} Yhtälöryhmä
\[ \begin{cases}
\,x^2+y^2+y^2=R^2 \\ \,x+y-z=1
\end{cases} \]
määrittelee origokeskisen pallon ja tason leikkausviivan $S$. Koska tason etäisyys origosta on
$h=1/\sqrt{3}$ (Esimerkki \ref{pisteen etäisyys tasosta}), niin päätellään, että $S$ on 
avaruusympyrä jos $R>1/\sqrt{3}$, piste jos $R=1/\sqrt{3}$, ja tyhjä joukko jos $R<1/\sqrt{3}$.
Ensiksi mainitussa tapauksessa $S$:n keskipiste on origon kautta kulkevalla suoralla, jonka 
suuntavektori on tason normaalivektori $\vec n=\vec i+\vec j-\vec k$. \loppu
\end{Exa}

\subsection{Lieriö ja kartio}
\index{lieriö|vahv} \index{kartio|vahv}

Olkoon $S$ avaruussuora, joka kulkee pisteen $P_0 \vastaa \vec r_0$ kautta ja jonka suuntavektori
on yksikkövektori $\vec e$. Tällöin pisteen $P \vastaa \vec r$ etäisyys suorasta $S$ voidaan
laskea kaavalla
\[
d = \abs{\vec r-\vec r_0}\abs{\sin\kulma(\vec r-\vec r_0,\vec e\,)} 
  = \abs{(\vec r-\vec r_0)\times\vec e\,}.
\]
Pinta, jolla tämä etäisyys on vakio $=R$ ($R>0$) on nimeltään \kor{lieriö}, tarkemmin
\kor{ympyrälieriö}. Suora $S$ on lieriön \kor{akseli} ja $R=$ lieriön \kor{säde}. Asettamalla
kaavassa $d=R$ ja neliöimällä saadaan lieriön yhtälölle muoto
\begin{equation} \label{lieriön y}
\abs{(\vec r-\vec r_0)\times\vec e\,}^2 = R^2.
\end{equation}
\begin{figure}[H]
\setlength{\unitlength}{1cm}
\begin{picture}(14,5)(-4,-1.5)
\put(0,-1.118){\line(2,1){6}} \put(-0.894,0.671){\line(2,1){6}}
\put(0,0){\vector(2,1){2}} \put(0,0){\vector(4,1){4.472}}
\dashline{0.1}(2.2,1.1)(4.6,2.3) \put(3,1.8){$S$}
\put(4.5,0.8){$\vec r-\vec r_0$}
\put(0,-0.5){$P_0$} \put(-0.07,-0.07){$\scriptstyle{\bullet}$}
\put(1.7,1.1){$\vec e$}
\put(0,0){\vector(-1,1){0.61}} \put(-0.4,0.5){$R$}
\renewcommand{\xscale}{.447} \renewcommand{\yscale}{.894}
\renewcommand{\xscaley}{-.447} \renewcommand{\yscalex}{.223}
\multiput(0,0)(4.6,2.3){2}{
\scaleput(0,0){
\curve(
 0.0,    -1.0,
-0.342, -0.940,
-0.643, -0.766,
-0.866, -0.5,
-0.940, -0.342,
-1.0,    0.0,
-0.940,  0.342,
-0.866,  0.5,
-0.643,  0.766,
-0.342,  0.940,
 0.0,    1.0)}
\curvedashes[1mm]{0,1,2}
\scaleput(0,0){
\curve(
0.0,   -1.0,
0.342, -0.940,
0.643, -0.766,
0.866, -0.5,
0.940, -0.342,
1.0,    0.0,
0.940,  0.342,
0.866,  0.5,
0.643,  0.766,
0.342,  0.940,
0.0,    1.0)}}
\end{picture}
\end{figure}
\begin{Exa} Olkoon lieriön akseli origon kautta kulkeva, yksikkövektorin
$\vec e=\frac{1}{3}(\vec i+2\vec j-2\vec k)$ suuntainen suora ja säde $R=2$. Tällöin on
$\vec r_0=\vec 0$ ja 
\[
\vec r\times\vec e\,=\,\frac{1}{3}\left|\begin{array}{rrr}
                                        \vec i&\ \vec j&\vec k\\x&\ y&z\\1&\ 2&-2
                                        \end{array}\right|
                  \,=\,\frac{1}{3}\left[(-2y-2z)\vec i+(2x+z)\vec j+(2x-y)\vec k\right],
\]
joten lieriön yhtälö on
\begin{align*}
&\qquad \frac{1}{9}[(2y+2z)^2+(2x+z)^2+(2x-y)^2]\,=\,4 \\[1mm]
&\ekv\quad 8x^2+5y^2+5z^2-4xy+4xz+8yz=36. \loppu
\end{align*}
\end{Exa}

Jos yhtälössä \eqref{lieriön y} korvataan ristitulo pistetulolla ja kirjoitetaan $R$:n tilalle
$\gamma\abs{\vec r -\vec r_0}$, missä $\gamma\in(0,1)$, niin yhtälö saa muodon
\begin{equation} \label{kartion y}
\abs{(\vec r-\vec r_0)\cdot\vec e\,}^2 = \gamma^2\abs{\vec r -\vec r_0}^2 \qekv
\abs{\cos\kulma(\vec r-\vec r_0,\vec e\,)}=\gamma.
\end{equation}
Tämä yhtälö määrittelee \kor{kartion}, tarkemmin \kor{ympyräkartion}. Piste 
$P_0\vastaa \vec r_0$ on kartion \kor{kärki}. Kartio koostuu kahdesta
\index{puolikartio}%
\kor{puolikartiosta}, joiden yhtälöt ovat
\[
\cos\kulma(\vec r-\vec r_0,\vec e\,)=\pm\gamma.
\]
\begin{figure}[H]
\setlength{\unitlength}{1cm}
\begin{picture}(14,4)(-4,-0.5)
\put(0,0){\vector(2,1){2}} \put(0,0){\vector(4,1){3.5}}
\put(3.5,0.4){$\vec r-\vec r_0$}
\put(0,-0.5){$P_0$} \put(-0.07,-0.07){$\scriptstyle{\bullet}$} 
\put(1.7,1.1){$\vec e$}
\path(-1,-0.813)(4,3.26) 
\put(0,0){\line(-4,-1){1.3}} \put(3.5,0.875){\line(4,1){1.5}}
\renewcommand{\xscale}{.447} \renewcommand{\yscale}{.894}
\renewcommand{\xscaley}{-.447} \renewcommand{\yscalex}{.223}
\put(4.025,2.012){
\scaleput(0,0){
\curve(
 0.0,    -1.0,
-0.342, -0.940,
-0.643, -0.766,
-0.866, -0.5,
-0.940, -0.342,
-1.0,    0.0,
-0.940,  0.342,
-0.866,  0.5,
-0.643,  0.766,
-0.342,  0.940,
 0.0,    1.0)}
\curvedashes[1mm]{0,1,2}
\scaleput(0,0){
\curve(
0.0,   -1.0,
0.342, -0.940,
0.643, -0.766,
0.866, -0.5,
0.940, -0.342,
1.0,    0.0,
0.940,  0.342,
0.866,  0.5,
0.643,  0.766,
0.342,  0.940,
0.0,    1.0)}}
\end{picture}
\end{figure}
Koordinaattien $x,y,z$ avulla esitettynä yhtälöt \eqref{lieriön y} ja \eqref{kartion y}
voidaan kumpikin kirjoittaa muotoon
\[
Ax^2+By^2+Cz^2+Dxy+Exz+Fyz+Gx+Hy+Iz+J=0,
\]
missä $A,B,$ jne.\ ovat reaalilukuja (vrt.\ esimerkki edellä). Tämä merkitsee, että lieriö ja 
kartio, myös pallo, ovat nk.\ 
\index{toisen asteen pinta}%
\kor{toisen asteen pintoja}. Toisen asteen pinnan ja avaruustason 
(esim.\ koordinaattitason) leikkausviiva on ko.\ tason
\index{toisen asteen käyrä}%
\kor{toisen asteen käyrä}. Tällainen on
esimerkiksi ympyrä. Yleinen toisen asteen käyrän yhtälö $xy$-tasolla on muotoa
\[
Ax^2+By^2+Cxy+Dx+Ey+F=0.
\]
Toisen asteen käyrien ja pintojen yleisempi luokittelu on matemaattinen ongelma, johon palataan
myöhemmin toisessa asiayhteydessä.

\pagebreak
\Harj
\begin{enumerate}

\item
a) Näytä, että yhtälöt esittävät samaa suoraa:
\[
\begin{cases} \,x=3+2t, \\ \,y=-11-6t \end{cases} \quad \text{ja} \quad
\begin{cases} \,x=-1-t, \\ \,y=1+3t \end{cases}
\]
b) Määritä suorien leikkauspiste:
\[
\begin{cases} \,x=3+2t, \\ \,y=-1-3t \end{cases} \quad \text{ja} \quad
\begin{cases} \,x=-1-t, \\ \,y=2(1+t) \end{cases}
\]
c) Määritä suoran $\,x=3+t,\ y=-2-t,\ z=4-2t\,$ ja koordinaattitasojen
leikkauspisteet.

\item
Määritä $\alpha$ siten, että vektori $\,\vec i+2\vec j+\alpha\vec k$ sekä avaruussuorat
\[
2(x-1)=1-y=2z-3 \quad \text{ja} \quad \begin{cases} x=17,\\y=7+3t,\\z=2t \end{cases}
\]
ovat saman tason suuntaiset.

\item
Määritä pisteet $P_1 \in S_1$ ja $P_2 \in S_2$ suorilla $\,S_1:\ x=-y=z$ ja 
$S_2:\ x+y-1=0,\ z=0$ siten, että vektori $\Vect{P_1P_2}$ on yhdensuuntainen vektorin
$2\vec i-\vec j-\vec k$ kanssa.

\item
Suora $S$ kulkee pisteen $(-1,1,3)$ ja sen janan keskipisteen kautta, jonka $xy$- ja $xz$-tasot
leikkaavat suorasta $\,x-1=2(y+1)=z+3$. Määritä $S$:n suuntavektori.

\item
Määritä $\alpha$ siten, että suorat $\,2(x-1)=y+1=2\alpha(z-1)$ ja $x+1=y-1=z$ leikkaavat.
Mikä on suorien lyhin etäisyys, jos $\alpha=1$\,?

\item
Määritä suorien
\[
\vec r=2\vec i+5\vec k+t(\vec i-2\vec j+2\vec k) \quad \text{ja} \quad
\begin{cases} 3x-2y=12\\x+2z=6 \end{cases}
\]
lyhin etäisyys ja lähinnä toisiaan olevat pisteet.

\item
Määritä sen suoran parametrimuotoinen yhtälö, joka leikkaa kohtisuorasti suorat
\[
\begin{cases} x=1+t, \\ y=1-2t, \\ z=t \end{cases} \quad \text{ja} \quad
\begin{cases} x=-1, \\ y=2s, \\ z=1-2s. \end{cases}
\]

\item
Laske pisteen $(2,3,-1)$ etäisyys suorasta
\[
\text{a)}\,\ \vec r=\vec i+2\vec j+3\vec k+t(\vec i-2\vec j+2\vec k) \qquad
\text{b)}\,\ \begin{cases} 3x-y+z=0\\x-2y+8=0 \end{cases}
\]

\item
a) Esitä tason $T$ yhtälö perusmuodossa $\,ax+by+cz+d=0$, kun parametrimuotoiset
yhtälöt ovat
\[ 
T:\ \begin{cases}
     x=2+2t_1+t_2, \\ y=-1+3t_1+2t_2, \\ z=3-t_1+t_2.
     \end{cases}
\]
b) Taso kulkee pisteen $P=(1,-13,-5)$ kautta ja sen suuntavektorit ovat
$\vec v_1=\vec i-\vec j$ ja $\vec v_2=\vec i+\vec j+\vec k$. Onko piste $Q=(3,-1,2)$
tasossa?

\item
Määritä tason yhtälö (perusmuoto!), kun tiedetään, että taso sisältää suoran
$x=3+t,\ y=1-2t,\ z=-2+t$ ja a) kulkee pisteen (0,2,1) kautta, b) on vektorin
$3\vec i+\vec j-2\vec k$ suuntainen

\item
Määritä seuraavien tasojen yhteiset pisteet: \newline
a) \ $x+y-z+2=0,\ 2x+y+2z-4=0,\ x-y+3z-2=0$ \newline
b) \ $x+y+z-6=0,\ x+2y-z-2=0,\  x+4y-5z+5=0$ \newline
c) \ $x+2y-2z-1=0,\ x-y+z-2=0,\ x+5y-5z=0$

\item Taso sisältää suoran $S_1:\ \vec r=(1+t)\vec i+(1+2t)\vec j+(1+3t)\vec k$ ja on suoran
$S_2:\ \vec r=(1+t)\vec i+(-1+t)\vec j+\vec k$ suuntainen. Johda tason yhtälö perusmuodossa.
Johda samoin sen tason yhtälö, joka sisältää suoran $S_2$ ja on $S_1$:n suuntainen.

\item
Johda sen tason yhtälö, joka puolittaa pisteen $P_0=(x_0,y_0,z_0)$ ja tason $T:\ ax+by+cz+d=0$
väliset janat.

\item
Määritä pisteen $(3,4,-2)$ kohtisuora projektio tasolla, jonka normaalivektori on
$\vec i-2\vec j+\vec k$ ja joka kulkee pisteen $(1,1,1)$ kautta.

\item
Määritä pisteen $(3,2,-4)$ peilikuvapiste tason $\,x+y-2z+5=0$ suhteen.

\item 
Määritä origon suurin mahdollinen etäisyys pisteiden $(0,1,0)$ ja $(2,2,-1)$ kautta
kulkevasta tasosta. Mikä taso antaa maksimietäisyyden?

\item
Määritä tasojen $\,x+y+z=3$ ja $3x-2y-z=1$ leikkaussuoran kautta kulkevat tasot, jotka
puolittavat tasojen välisen kulman.

\item
Kheopsin pyramidin (alkuperäinen) korkeus on $147$ m ja neliön muotoisen 
poh\-jan sivun pituus $230$ m. Sijoita pyramidi koordinaatistoon niin, että se
tuntee olonsa mahdollisimman mukavaksi, ja määritä tässä koordinaatistossa pyramidin
sivutasojen yhtälöt, mittayksikkönä $100$ m. Laske myös vierekkäisten sivujen välinen 
diedrikulma (vrt.\ Harj.teht \ref{ristitulo}:\ref{H-I-7:oktaedri}).

\item
Vektorin $2\vec i-\vec j+3\vec k$ suuntaan kulkeva valonsäde heijastuu tasosta $T$ pisteessä
$(1,2,-1)$. Heijastunut säde kulkee pisteen $(2,5,-3)$ kautta. Mikä on tason yhtälö?

\item
Positiivisen $z$-akselin suunnasta tuleva valonsäde osuu pisteessä $(1,2,3)$ tasolla
$3x+2y+z=10$ olevaan peiliin. Määritä heijastuneen säteen suunta. Missä pisteessä heijastunut
säde leikkaa (jos leikkaa) $xy$-tason?

\item
Painovoima vaikuttaa negatiivisen $z$-akselin suuntaan. Pisara putoaa pisteestä $(1,1,3)$ 
tasolle $\,3x-4y+12z=12$ ja alkaa valua tasoa pitkin alaspäin. Missä pisteessä pisara kohtaa
$xy$-tason?

\item
Tasangolta $z=0$ kohoaa vuorenrinne pitkin tasoa $x+2y+4z=0$. Rinteen pisteestä $P$, joka on 
korkeudella $h=10$, lähtee liikkeelle pistemäinen lumivyöry. Se etenee suoraviivaisesti 
rinnettä alas painovoimalakien mukaisesti. Tasangolle saavuttuaan se jatkaa suoraviivaista
liikettään vaakasuoraan, kunnes osuu mökkiin, joka on pisteessä $Q=(10,40,0)$.
Mikä oli piste $P$?

\item \index{zzb@\nim!Haukka ja kaksi kanaa} 
(Haukka ja kaksi kanaa) Universaalikoordinaatistossa maan pinta on 
taso $x+2y-3z=0$. Maan pinnalla käyskentelee kaksi pistemäistä kanaa $A$ ja 
$B$. Kanoja vaanii pistemäinen haukka, joka lentää maan pinnan suuntaisella
tasolla korkeudella $h=5$. Pistemäinen aurinko loistaa suunnassa
$-4\vec i +\vec k $. Hetkellä $H$ tapahtuu seuraavaa: Kanalta $A$ pääsee säikähtynyt 
'kot' sen huomatessa haukan varjon päällään. Kotkotuksen kuulee kana $B$,
joka vilkaisee samassa taivaalle ja näkee haukan suunnassa $\vec i -\vec j +\vec k $.
Määritä vektori $\Vect{AB}$ kyseisellä hetkellä $H$.

\item
Missä kulmassa tason suora $y=4x$ leikkaa ympyräviivan \newline
$x^2+y^2-2x-4y+4=0$\,?

\item
Avaruuskolmion kärjet ovat $(0,0,0)$, $(3,2,1)$ ja $(2,-1,3)$. Laske kolmion sisään piirretyn
(eli kaikkia sivuja sivuavan) ympyrän keskipiste ja säde.

\item
Avaruuden $E^3$ kolmen pisteen paikkavektorit ovat $\vec a$, $\vec b$ ja $\vec c$. Esitä
menettely, jolla voidaan määrittää pisteiden kautta kulkevan ympyrän keskipisteen paikkavektori.
Sovella menettelyä, kun pisteet ovat $(1,2,3)$, $(2,-5,3)$ ja $(-1,3,-6)$. Määritä myös
ympyrän säde ja ympyrän tason normaalivektori.

\item
Määritä avaruusympyrän
\[ \begin{cases}
\,x^2+y^2+z^2=49 \\ \,x+2y-z=10
\end{cases} \]
tangentti pisteessä $(-2,3,-6)$.

\item
Lieriön säde on $R=2$ ja akseli on pisteen $P_0=(1,1,1)$ kautta kulkeva suora, jonka 
suuntavektori on $-\vec i+\vec j+\vec k$. Määritä lieriön yhtälö toisen asteen pinnan yhtälön
perusmuodossa. Mitkä ovat lieriön ja suoran $x=y=z$ leikkauspisteet, ja mikä lieriön piste on
lähinnä origoa?

\item
Lieriöllä ja kartiolla on yhteisenä akselina suora $S:\ x=2y=-2z$ ja kumpikin kulkee pisteen
$(1,0,0)$ kautta. Kartion kärkenä on piste $(-2,-1,1)$. Laske lieriön säde ja kartion
(sivuprofiilin) aukeamiskulma sekä saata kummankin pinnan yhtälöt toisen asteen pinnan
yhtälön perusmuotoon. Määritä edelleen molempien pintojen ja $xy$-tason leikkauskäyrien
yhtälöt ja hahmottele näiden käyrien muoto.

\item (*)
Kuution, jonka sivun pituus $=4$, yksi kärki on origossa ja kolme muuta positiivisilla
koordinaattiakseleilla. Kuutiota katsotaan kaukaa vektorin $\vec i+2\vec j+3\vec k$
osoittamasta suunnasta kuvakulmassa, jossa $z$-akseli näkyy pystysuorana. Laske, millaisena
kuutio näkyy tästä kuvakulmasta. Kuva!

\item (*)
Millä tavoin saadaan selville avaruustasot, jotka sivuavat kolmea annettua palloa? Montako
tällaista tasoa on, jos pallot eivät leikkaa tai sivua toisiaan eikä mikään palloista ole 
toisen sisällä?

\item (*)
Näytä, että yhtälö $K:\ xy+yz+xz=0$ määrittelee kartion, ja määritä se $K$:n piste, joka on
lähinnä pistettä $(-1,2,3)$.

\item (*)
Kartion kärki on origossa, symmetria-akseli on vektorin $\vec i-2\vec j+2\vec k$ suuntainen
ja $y$-akseli on kartiopinnalla. Taso $T$ kulkee pisteen $(1,1,1)$ kautta ja sivuaa kartiota
pitkin avaruussuoraa. Määritä $T$:n yhtälö (kaksi ratkaisua!).

\end{enumerate}