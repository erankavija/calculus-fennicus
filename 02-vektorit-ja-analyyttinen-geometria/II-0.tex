\chapter{Vektorit ja analyyttinen geometria}

Ihmisten luoman matemaattisen kulttuurin peruspilareita on matemaattinen malli nimeltä 
\kor{euklidinen taso}. Termi viittaa kreikkalaisen 
\index{Eukleides}%
\hist{Eukleideen} n. 300 eKr kirjoittamaan teokseen Stoikheia eli 'Alkeet'. Eukleideen
pyrkimyksenä oli asettaa selkeät aksiomaattiset perusteet silloiselle matematiikalle,
erityisesti \kor{geometrian} nimellä tunnetulle matematiikan suuntaukselle. Geometria on ---
toisin kuin edellä tarkasteltu lukujen algebra --- näköaistimuksiin hyvin voimakkaasti vetoava
matematiikan laji. Algebra ja geometria edustavat nykyisessäkin matematiikassa kahta
matemaattisen ajattelun suurta päähaaraa.\footnote[2]{Algebran ja geometrian välinen rajanveto
matematiikassa ei ole aina helppoa. Ongelmaa kuvastaa esim.\ moderni matematiikan laji nimeltä
\kor{algebrallinen geometria}.}

Eukleideen teos on kaikkien aikojen menestynein matematiikan oppikirja, jonka vaikutus
varhaisena aksiomaattisen ajattelun esikuvana on ulottunut matematiikan ulkopuolellekin.
Käytännön kannalta 'tyhjästä' teos ei tietenkään syntynyt, vaan sitä edelsi itse asiassa
vuosituhantinen laskentamenetelmien perinne sekä Egyptissä että Babyloniassa. Esimerkiksi
Egyptissä geometrisia menetelmiä olivat kehittäneet (jos historioitsijoihin on uskominen) sekä
maanmittarit että papit.

Tässä luvussa lähdetään euklidisen geometrian perusteista tasossa ja avaruudessa ja paneudutaan
tarkastelemaan geometrian pohjalta syntyvää \kor{vektorin} käsitettä ja vektoreilla laskemista 
eli \kor{vektorialgebraa}. Vektoreiden avulla geometriset ongelmat on mahdollista 
'algebralisoida' eli muotoilla ja ratkaista vektorialgebran ongelmina. Tällaista --- 
historiallisesti melko myöhäsyntyistä --- lähestymistapaa geometriaan kutsutaan 
\kor{analyyttiseksi} geometriaksi.