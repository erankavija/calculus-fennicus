\subsection{Voiman momentti}

Ristituloa käytetään fysiikassa erityisesti pyörimisliikkeen kuvauksessa, kuten 
(\pain{kulmano}p\pain{euden}) ja pyörimisliikettä aikaansaavan voiman \pain{momentin} 
ilmaisemiseen. Olkoon jäykkä kappale tuettu pisteestä $O$ ja vaikuttakoon sen pinnan pisteessä 
$P$ voima $\vec F$ (vektori!). Tällöin jos merkitään $\vec r = \overrightarrow{OP}$, niin 
voiman momentti pisteen $O$ suhteen on vektori
\[
\vec M = \vec r \times \vec F.
\]
\begin{figure}[H]
\begin{center}
\import{kuvat/}{kuvaII-15.pstex_t}
\end{center}
\end{figure}
Jos voimia ja vaikutuspisteitä on useita, määrätään kokonaismomentti vektorien yhteenlaskulla:
\[
\vec M=(\vec r_1 \times \vec F_1) + (\vec r_2 \times \vec F_2) + \cdots = \sum_i\vec M_i.
\]
Vektori $\vec M$ (mikäli $\neq \vec 0$) määrää kappaleen pyörimisakselin (kulmanopeusvektorin)
suunnan voimien vaikuttaessa. (Ehto $\vec M = \vec 0$ on tasapainoehto.)