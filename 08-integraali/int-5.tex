\section{Riemannin integraali} \label{riemannin integraali}
\alku
\index{mzyzyrzy@määrätty integraali|vahv}

Tarkastellaan suljetulla välillä $[a,b]$ määriteltyä funktiota $f$, joka olkoon 
\index{rajoitettu!c@funktio}%
\kor{rajoitettu}, ts.\ on olemassa $M\in\R_+$ siten, että pätee
\[
\abs{f(x)}\leq M\quad\forall x\in [a,b].
\]
Välin $[a,b]$ yleistä 
\index{jako (osaväleihin)}%
\kor{jakoa} merkitään jatkossa symbolilla $\X$. Kuten edellisessä luvussa,
jako tarkoittaa äärellistä järjestettyä joukkoa $\X=\{x_0,x_1,\ldots,x_n\}$, missä 
$n\in\N$ ja
\[
a=x_0<x_1<\ldots <x_n=b.
\]
Jaon $\X$ keskeisin parametri on
\index{tiheysparametri}%
\kor{tiheysparametri}, joka määritellään
\[
h_{\X}=\max_{k=1\ldots n} (x_k-x_{k-1}).
\]
Jakoon $\X$ liittyen otetaan vielä käyttöön 
\index{vzy@välipiste(istö)}%
\kor{välipisteistö} $\Xi_\X=\{\xi_1,\ldots,\xi_n\}$,
missä $\xi_k\in [x_{k-1},x_k]$, $k=1\ldots n$ (muuten $\Xi_\X$ on vapaasti valittavissa).

Em. merkinnöin liitetään jokaiseen pariin $(\X,\Xi_\X)$ reaaliluku $\sigma(f,\X,\Xi_\X)$, joka
määritellään
\[
\sigma(f,\X,\Xi_\X)=\sum_{k=1}^n f(\xi_k)(x_k-x_{k-1}).
\]
Edellisen luvun tapaan käytetään raja-arvomerkintää $\Lim_{h_\X\kohti 0} \sigma(f,\X,\Xi_\X)=A$
kuvaamaan sellaista tilannetta, jossa summa lähestyy aina samaa raja-arvoa $A$ ($A\in\R$), kun
$h_\X\kohti 0$. Kuten funktion raja-arvo, myös raja-arvo 'Lim' voidaan määritellä
kahdella tavalla, joko lukujonojen avulla tai '$(\eps,\delta)$-määritelmänä', vrt.\
Määritelmä \ref{funktion raja-arvon määritelmä} ja Lause \ref{approksimaatiolause}.
Jälkimmäinen määrittelytapa muotoiltakoon jälleen lauseena. (Todistus sivuutetaan, vrt.\
Lauseen \ref{approksimaatiolause} todistus.)
\begin{Def} \label{raja-arvo Lim} \index{raja-arvo!f@määrätyn integraalin|emph}
$\ \Lim_{h_\X\kohti 0} \sigma(f,\X,\Xi_\X)=A \in\R,\ $
jos jokaiselle jonolle $\seq{\sigma(f,\X_n,\Xi_{\X_n})}$, jolle pätee $h_{\X_n} \kohti 0$,
on voimassa $\,\lim_{n\kohti\infty} \sigma(f,\X_n,\Xi_{\X_n}) = A$.
\end{Def}
\begin{*Lause} \label{Lim-kriteeri} \vahv{(Lim: $(\eps,\delta)$-kriteeri)} 
$\ \Lim_{h_\X\kohti 0} \sigma(f,\X,\Xi_\X)=A\ $
täsmälleen kun jokaisella $\eps>0$ on olemassa $\delta>0$ siten, että jokaiselle parille
$(\X,\Xi_\X)$, jolle $h_\X<\delta$, pätee
\[
\abs{\sigma(f,\X,\Xi_\X)-A}<\eps.
\]
\end{*Lause}
Määritelmän \ref{raja-arvo Lim} mukaisesti raja-arvo 'Lim' tarkoittaa kaikista mahdollisista
summista $\sigma(f,\X,\Xi)$ poimittujen, ehdon $h_{\X_n} \kohti 0$ täyttävien (ei muita ehtoja!)
\pain{luku}j\pain{ono}j\pain{en} $\seq{\sigma(f,\X_n,\Xi_{\X_n})}$ y\pain{hteistä}
\pain{ra}j\pain{a-arvoa}. Sikäli kuin tällainen yhteinen raja-arvo on olemassa, Määritelmä
\ref{raja-arvo Lim} tarjoaa myös algoritmin sen laskemiseksi: Valitaan mikä tahansa jono
tiheneviä (esim.\ tasavälisiä) jakoja $X_n$, liitetään kuhunkin $\X_n$ jokin välipisteistö 
$\Xi_n$ (esim.\ osavälien keskipisteet tai toinen päätepisteistä) ja lasketaan
$A=\lim_{n}\sigma(f,\X_n,\Xi_{\X_n})=\lim_n A_n$.
% Tässä $A_n$ on laskettavissa jokaisella
%$n\in\N$  ($n$ funktioevaluaatiota ja summaus!), joten kyseessä on toimiva algoritmi.

Edellisessä luvussa todettiin, että jos $f$:llä on välillä $(a,b)$ integraalifunktio $F$, ja $F$
on lisäksi jatkuva välillä $[a,b]$, niin jokaisella $\X$ on olemassa välipisteistö $\Xi_\X$
siten, että pätee
\[
\sigma(f,\X,\Xi_\X)=F(b)-F(a)=\sijoitus{a}{b} F(x)=\int_a^b f(x)\, dx.
\]
Edelleen näytettiin, että sikäli kuin määrätty integraali $\int_a^b f(x)\,dx$ määritellään tällä
tavoin (mainituin
oletuksin) ja lisäksi oletetaan, että $f$ on Lipschitz-jatkuva välillä $[a,b]$, niin sekä
jokaiselle jaolle $X$ että jokaiselle välipisteistölle $\Xi_\X$ pätee
(Lause \ref{summakaavalause})
\[
\sigma(f,\X,\Xi_\X)=\int_a^b f(x)\, dx + \ordoO{h_\X}.
\]
Määritelmän \ref{raja-arvo Lim} (tai Lauseen \ref{Lim-kriteeri}) perusteella todetaan, että
mainituilla (melko voimakkailla) oletuksilla pätee
\[
\Lim_{h_\X\kohti 0} \sigma(f,\X,\Xi_\X)=\int_a^b f(x)\,dx.
\]
Yleisemmin, jos $f$ mahdollisesti ei täytä mainittuja ehtoja, niin tehdään tästä laskukaavasta
integraalifunktiosta riippumaton \pain{määrät}y\pain{n} \pain{inte}g\pain{raalin}
\pain{määritelmä}. 
\begin{Def} \label{Riemannin integraali} \index{Riemannin!a@integraali|emph}
\index{Riemann-integroituvuus|emph} \vahv{(Riemannin\footnote[2]{Saksalainen matemaatikko 
\hist{Georg Friedrich Bernhard Riemann} (1826-1866) on 1800-luvun (ja kaikkienkin aikojen)
matematiikan suuria nimiä. Riemann oli 'puhdas matemaatikko' selvemmin kuin edeltäjänsä,
joista huomattavimmatkin (esim.\ Euler, Lagrange, Cauchy, Gauss) tutkivat matematiikan ohella
fysiikkaa tai muita matematiikan sovelluksia. Riemannin työt koskivat integraalien lisäksi mm.\
kompleksifunktioiden teoriaa (\kor{Riemannin pinnat}), alkulukujen jakautumista, ja geometrian
matemaattisia  perusteita. Monet Riemannin tuloksista olivat uraa uurtavia ja näyttivät suuntaa
myöhemmälle matematiikan tutkimukselle, joka 1800-luvulta lähtien erkaantui yhä selvemmin 
fysiikasta. \index{Riemann, G. F. B.|av}} integraali)} Olkoon $f$ määritelty ja rajoitettu
välillä $[a,b]$. Jos
\[
\Lim_{h_\X\kohti 0} \sigma(f,\X,\Xi_\X)=A\quad (A \in \R),
\]
niin sanotaan, että $f$ on \kor{Riemann-integroituva} (tai integroituva Riemannin mielessä)
välillä $[a,b]$. Lukua $A$ sanotaan $f$:n \kor{Riemannin integraaliksi} (Riemann-integraaliksi)
välillä $[a,b]$ ja merkitään
\[
A=\int_a^b f(x)\, dx.
\]
\end{Def}
Määritelmään liittyen sanotaan summia
\index{Riemannin!b@summa}%
$\sigma(f,\X,\Xi_\X)$ \kor{Riemannin summiksi}. Jatkossa
sanotaan yksinkertaisesti, että $f$ on \kor{integroituva} välillä $[a,b]$, jos $f$ on sekä
määritelty, rajoitettu että Riemann-integroituva ko.\ välillä .
\begin{Exa} \label{Riemann-ex1}
Olkoon $c\in\R$ ja määritellään
\[
f(x)=\begin{cases}
0, &\text{ kun } x<1/2 \\
c, &\text{ kun } x=1/2 \\
1, &\text{ kun } x>1/2
\end{cases}
\]
Tutki $f$:n integroituvuutta välillä $[0,1]$.
\end{Exa}
\ratk Jos $\X=\{x_0,\ldots ,x_n\}$, $n \ge 2$, on välin $[0,1]$ jako, niin jollakin $j\in\N$
on $x_{j-1}<1/2<x_{j+1}$. Tällöin on $f(\xi_k)=0$ kun $k\leq j-1$ ja $f(\xi_k)=1$ kun 
$k\geq j+2$, joten
\begin{align*}
\sigma(f,\X,\Xi_\X) &= \sum_{k=j}^{j+1}f(\xi_k)(x_k-x_{k-1})+\sum_{k=j+2}^n (x_k-x_{k-1}) \\
&=f(\xi_j)(x_j-x_{j-1})+f(\xi_{j+1})(x_{j+1}-x_j)+(1-x_{j+1}).
\end{align*}
Kun $h_\X \kohti 0$, niin $1-x_{j+1} \kohti 1/2$, koska $\,0<x_{j+1}-1/2<x_{j+1}-x_{j-1}<2h_\X$,
ja kolmioepäyhtälön nojalla
\begin{align*}
\abs{f(\xi_j)(x_j-x_{j-1})+f(\xi_{j+1})(x_{j+1}-x_j)} 
                     &\le \max \{\abs{c},1\}(x_{j+1}-x_{j-1}) \\
                     &\le 2\max \{\abs{c},1\} h_\X \kohti 0.
\end{align*}
Siis $f$ on integroituva välillä $[0,1]$ ja $\D\int_0^1 f(x)\, dx=\frac{1}{2}\,$. \loppu

\begin{Exa} \label{Riemann-ex2}
Dirichlet'n funktio
\[
f(x)=\begin{cases}
1, &\text{jos}\ x\in\Q \\
0, &\text{muulloin}
\end{cases}
\]
ei ole integroituva välillä $[0,1]$, sillä jos valitaan $\Xi_\X\subset\Q$, niin 
$\sigma(f,\X,\Xi_\X)=1$ ja jos valitaan $\Xi_\X\cap\Q=\emptyset$, niin $\sigma(f,\X,\Xi_\X)=0$,
olipa $\X$ mikä hyvänsä. \loppu
\end{Exa}
Esimerkeistä nähdään, että kaikki integroituvat funktiot eivät ole jatkuvia eivätkä kaikki
rajoitetut funktiot ole integroituvia. Esimerkissä \ref{Riemann-ex1} integraalin arvo ei riipu
funktion arvosta epäjatkuvuuspisteessä. --- Yleisemminkin pätee, että jos $f$ on välillä $[a,b]$
integroituva, niin integraalin arvo ei muutu, jos $f$ määritellään uudelleen äärellisen monessa
pisteessä (tai jopa suppenevassa jonossa pisteitä, ks.\ 
Harj.teht.\,\ref{H-int-5: funktion poikkeutus}). Integraali on siis 'tunnoton' funktion
yksittäisille pistearvoille samaan tapaan kuin funktion raja-arvo.

Näytetään nyt, että edellisessä luvussa todetut määrätyn integraalin keskeiset ominaisuudet,
eli additiivisuus integroimisvälin suhteen, lineaarisuus integroitavan funktion suhteen ja
integraalien vertialuperiaate, pysyvät voimassa myös Määritelmän \ref{Riemannin integraali}
mukaisille integraaleille.
\begin{Lause} \label{integraalin additiivisuus} \index{additiivisuus!a@integraalin}
\vahv{(Integraalin additiivisuus)} Jos $f$ on integroituva väleillä $[a,b]$ ja $[b,c]$
$(a<b<c)$, niin $f$ on integroituva välillä $[a,c]$, ja
\[
\int_a^c f(x)\, dx=\int_a^b f(x)\, dx+\int_b^c f(x)\, dx.
\]
\end{Lause}
\tod Tarkastellaan Riemannin summia $\sigma(f,\X,\Xi_\X)$ välillä $[a,c]$. Oletetaan aluksi,
että $b\in\X$, ja jaetaan summa kahteen osaan:
\begin{align*}
\sigma(f,\X,\Xi_\X) &= \sum_{k:\,x_k\leq b} f(\xi_k)(x_k-x_{k-1}) 
                         + \sum_{k:\,x_k> b} f(\xi_k)(x_k-x_{k-1}).
\end{align*}
Määritelmien \ref{Riemannin integraali} ja \ref{raja-arvo Lim} mukaan
\begin{align*}
\sum_{k:\,x_k\leq b} f(\xi_k)(x_k-x_{k-1}) 
                &\kohti \int_a^b f(x)\, dx, \quad \text{kun}\ h_\X \kohti 0, \\
\sum_{k:\,x_k> b} f(\xi_k)(x_k-x_{k-1})    
                &\kohti \int_b^c f(x)\, dx, \quad \text{kun}\ h_\X \kohti 0,
\end{align*}
koska $f$ oli integroituva väleillä $[a,b]$ ja $[b,c]$. Päätellään siis, että rajoituksen 
$b\in\X$ ollessa voimassa pätee
\[
h_\X\kohti 0 \ \impl \ \sigma(f,\X,\Xi_\X)\kohti\int_a^b f(x)\, dx+\int_b^c f(x)\, dx.
\]
Tarkastellaan seuraavaksi Riemannin summaa $\sigma(f,\X,\Xi_\X)$, missä $\X=\{x_k\}$ ja 
$x_{m-1}<b<x_m$ jollakin $m$ (jolloin $b\not\in\X$). Verrataan tätä summaan 
$\sigma(f,\X',\Xi_{\X'})$, missä $\X'$ on saatu $\X$:stä lisäämällä vain jakopiste $b$ ja
$\Xi_{\X'}$ on valittu siten, että välipisteet $\xi_k\in \Xi_{\X}$ ja $\xi_l'\in\Xi_{\X'}$ ovat
samat jaoille yhteisillä osaväleillä, eli jos $[x_{k-1},x_k]=[x_{l-1}',x_l']$. Tällöin on
$x_{m-1}'=x_{m-1}$, $x_m'=b$ ja $x_{m+1}'=x_m$, joten
\begin{align*}
\sigma(f,\X,\Xi_\X)&-\sigma(f,\X',\Xi_{\X'}) \\[1mm]
                   &=f(\xi_m)(x_m-x_{m-1})-f(\xi_m')(b-x_m)-f(\xi_{m+1}')(x_{m+1}-b).
\end{align*}
Jos $\abs{f(x)}\leq M \ \forall x\in [a,c]$, niin kolmioepäyhtälöä käyttäen päätellään
\begin{align*}
|\sigma(f,\X,\Xi_\X)-\sigma(f,\X',\Xi_{\X'})|
         &\,\le\, 2M(x_m-x_{m-1}) \\[1mm]
         &\,\le\, 2Mh_\X \kohti 0,\quad \text{kun}\ h_X \kohti 0.
\end{align*}
Summalla $\sigma(f,\X,\Xi_\X)$ on siis sama raja-arvo riippumatta siitä, onko $b\in\X$ tosi vai
ei. \loppu

Kun käytetään edellisessä luvussa sovittua integroimisrajojen vaihtosääntöä
\[
\int_b^a f(x)\, dx=-\int_a^b f(x)\, dx,
\]
niin nähdään, että Lauseen \ref{integraalin additiivisuus} väittämä on tosi lukujen $a,b,c$
suuruussuhteista rippumatta, kunhan $f$ on integroituva väleillä $[\min\{a,b\},\max\{a,b\}]$ ja
$[\min\{b,c\},\max\{b,c\}]$.
\begin{Lause} \label{integraalin lineaarisuus} \index{lineaarisuus!b@integroinnin|emph}
\vahv{(Integraalin lineaarisuus)} Jos $f$ ja $g$ ovat integroituvia välillä $[a,b]$, niin
$\alpha f+\beta g$ on integroituva välillä $[a,b]$ jokaisella $\alpha,\beta\in\R$ ja
\[
\int_a^b [\alpha f(x)+\beta g(x)]\,dx = \alpha\int_a^b f(x)\, dx + \beta\int_a^b g(x)\,dx.
\]
\end{Lause}
\tod Jos $\seq{X_n}$ ja $\seq{\Xi_n}$ ovat jako- ja välipisteistöt välillä $[a,b]$, niin
vastaaville Riemannin summille pätee
\[
\sigma(\alpha f+\beta g,X_n,\Xi_n)\,=\,\alpha\,\sigma(f,X_n,\Xi_n)+\beta\,\sigma(g,X_n,\Xi_n),
                                       \quad \alpha,\beta\in\R.
\]
Väitetty lineaarisuussääntö seuraa tästä Määritelmän \ref{Riemannin integraali} ja lukujonojen
raja-arvojen yhdistelysääntöjen (Lause \ref{raja-arvojen yhdistelysäännöt}) nojalla, kun
oletetaan $\lim_nh_{X_n}=0$. \loppu
\begin{Lause} \label{integraalien vertailuperiaate} \index{vertailuperiaate!a@integraalien|emph}
\vahv{(Integraalien vertailuperiaate)} Jos $f$ ja $g$ ovat integroituvia välillä $[a,b]$,
niin pätee
\[
f(x)\leq g(x)\quad\forall x\in [a,b] \ \impl \ \int_a^b f(x)\, dx\leq\int_a^b g(x)\, dx.
\]
\end{Lause}
\tod Myös tämä väittämä palautuu lukujonojen teoriaan: Vertailuperiaate on ilmeisen pätevä
Riemannin summille $\,\sigma(f,X_n,\Xi_n)\,$ ja $\,\sigma(g,X_n,\Xi_n)$, joten se pätee myös
raja-arvoille (Propositio \ref{jonotuloksia}\,[V1]). \loppu 

 
\subsection{*Riemannin ylä- ja alaintegraalit}

Määritelmä \ref{Riemannin integraali} Riemann-integroituvuudelle on sikäli konkreettiinen, että
se antaa suoraan myös integraalin numeeriseen laskemiseen soveltuvan algoritmin. Seuraavassa
lähestytään Riemannin integraalia toisella, hieman abstraktimmalla tavalla, mikä
johtaa vaihtoehtoiseen Riemann-integroituvuuden kriteeriin (Lause \ref{Riemann-integroituvuus}).
Jatkossa tämä kriteeri osoittautuu käteväksi erilaisissa teoreettisissa tarkasteluissa
(ks.\ myös Harj.teht.\,\ref{H-int-5: tasavälinen jako}--\ref{H-int-5: itseisarvon ja tulon
integroituvuus}).

Olkoon $f$ välillä $[a,b]$ määritelty ja rajoitettu funktio ja $\X = \{x_k,\ k = 0 \ldots n\}$
välin $[a,b]$ jako. Koska $f$ on rajoitettu, niin joukot 
$Y_k = \{f(x) \mid x \in [x_{k-1},x_k]\}$ ovat rajoitettuja. Siis on olemassa luvut 
(vrt.\ Luku \ref{reaalilukujen ominaisuuksia})
\[ 
\overline{f}_k = \sup Y_k, \quad \underline{f}_k = \inf Y_k, \quad k = 1 \ldots n.  
\]
Funktion $f$ jakoon $\X$ liittyviksi
\index{Riemannin!c@ylä- ja alasumma}%
\kor{Riemannin ylä- ja alasummiksi} sanotaan summia
\begin{align*}
\overline{\sigma}(f,\X)\ 
          &=\ \sum_{k=1}^n \overline{f}_k (x_k-x_{k-1}) \quad\ \text{(yläsumma)}, \\
\underline{\sigma}(f,\X)\ 
          &=\ \sum_{k=1}^n \underline{f}_k (x_k-x_{k-1}) \quad\ \text{(alasumma)}. 
\end{align*}
Huomattakoon, että jos $f$ on jatkuva välillä $[a,b]$, niin $f$ saavuttaa maksimi- ja 
minimiarvonsa jokaisella osavälillä $[x_{k-1},x_k]\subset[a,b]$ 
(Lause \ref{Weierstrassin peruslause}), jolloin $\overline{f}_k=f(\xi_k)$ ja 
$\underline{f}_k=f(\eta_k)$ joillakin $\xi_k,\eta_k\in[x_{k-1},x_k]$. Tässä tapauksessa siis
Riemannin ylä- ja alasummat ovat Riemannin summien erikoistapauksia.

Olkoon nyt $\mathcal{X}$ välin $[a,b]$ kaikkien mahdollisten jakojen $\X$ joukko. Tällöin
reaalilukujoukot $\{\overline{\sigma}(f,\X) \mid \X \in \mathcal{X}\}$ ja 
$\{\underline{\sigma}(f,\X) \mid \X \in \mathcal{X}\}$ ovat rajoitettuja, sillä jos
$\abs{f(x)} \le M,\ x \in [a,b]$, niin $\abs{\overline{f}_k} \le M$ ja 
$\abs{\underline{f}_k} \le M\ \ \forall k$, jolloin
\[ 
\abs{\overline{\sigma}(f,\X)},\ \abs{\underline{\sigma}(f,\X)}\ 
                       \le\ M \sum_{k=1}^n (x_k-x_{k-1})\ =\ M(b-a). 
\]
Mainituilla lukujoukoilla on siis sekä supremum että infimum.
\begin{Def} \label{ylä- ja alaintegraali} \index{Riemannin!d@ylä- ja alaintegraali|emph}
Jos $f$ on määritelty ja rajoitettu välillä $[a,b]$ ja
$\mathcal{X}$ on välin $[a,b]$ kaikkien mahdollisten jakojen $\X$ joukko, niin funktion $f$ 
\kor{Riemannin ylä- ja alaintegraalit} välillä $[a,b]$ määritellään
\begin{align*} \label{ylä- ja alaintegraalit}
\overline{\int_a^b} f(x)\,dx 
       &= \inf_{\X \in \mathcal{X}} \overline{\sigma}(f,\X) \quad\ \text{(yläintegraali)}, \\
\underline{\int_a^b} f(x)\,dx 
       &= \sup_{\X \in \mathcal{X}} \underline{\sigma}(f,\X) \quad\ \text{(alaintegraali)}.
\end{align*}
\end{Def}
Määritelmän \ref{ylä- ja alaintegraali} mukaisesti siis jokainen välillä $[a,b]$ määritelty ja
rajoitettu funktio on ko.\ välillä sekä 'yläintegroituva' että 'alaintegroituva'.
\begin{Exa} Jos $f(x)=1$, kun $x\in\Q$, ja $f(x)=0$ muulloin, niin
\[ 
\overline{\int_0^1} f(x)\, dx = 1, \quad \underline{\int_0^1} f(x)\,dx = 0. \loppu
\]
\end{Exa}
Esimerkki johdattelee seuraavaan hyvin eleganttiin Riemann-integroituvuuden kriteeriin.
Todistus esitetään luvun lopussa.
\begin{*Lause} \label{Riemann-integroituvuus} \vahv{(Riemann-integroituvuus)} 
\index{Riemann-integroituvuus|emph}
Välillä $[a,b]$ määritelty ja rajoitettu funktio $f$ on ko.\ välillä Riemann-integroituva
täsmälleen kun $f$:n ylä- ja alaintegraalit ko.\ välillä ovat samat, jolloin $f$:n Riemannin
integraali $=$ ylä- ja alaintegraalien yhteinen arvo. 
\end{*Lause}

Seuraavassa esitetään kaksi Riemannin ylä- ja alasummia koskevaa väittämää, joilla on
teoreettisissa tarkasteluissa käyttöä yhdessä Lauseen \ref{Riemann-integroituvuus} kanssa.
Ensinnäkin näytetään, että jaon
\index{tihennys (jaon)}%
\kor{tihennyksessä} yläsumma pienenee tai pysyy samana ja
vastaavasti alasumma suurenee tai pysyy samana.
\begin{Lause} \label{jaon tihennys} Jos $X$ ja $X'$ ovat välin $[a,b]$ jakoja ja $X' \supset X$ 
(eli $X'$ on $X$:n tihennys), niin
\[
\underline{\sigma}(f,X) \le \underline{\sigma}(f,X') \le
\overline{\sigma}(f,X') \le \overline{\sigma}(f,X).
\]
\end{Lause}
\tod Tihennys voidaan ajatella suoritetuksi lisäämällä jakoon $X$ yksi piste kerrallaan,
kunnes päädytään jakoon $X'$. Väittämä on tosi, jos se on tosi jokaiselle tällaiselle
osatihennykselle. Olkoon siis $X = \{x_k,\ k = 0 \ldots n\}$ ja olkoon $X'=X\cup\{x_k'\}$,
missä $x_k' \in (x_{k-1},x_k)$. Tällöin jos $\overline{f}_k$ ja $\underline{f}_k$ määritellään
kuten edellä (jakoon $X$ liittyen) ja merkitään
\[
\alpha_k = \sup_{x\in[x_{k-1},\,x_k']} f(x), \quad \beta_k = \sup_{x\in[x_k',\,x_k]} f(x),
\]
niin $\alpha_k \le \overline{f}_k$ ja $\beta_k \le \overline{f}_k$
(vrt.\ Harj.teht.\,\ref{reaalilukujen ominaisuuksia}:\ref{H-I-11: sup ja inf}c), joten
\[
\alpha_k(x_k'-x_{k-1})+\beta_k(x_k-x_k')\,\le\,\overline{f}_k(x_k-x_{k-1}).
\]
Koska jaot ovat samat välin $[x_{k-1},x_k]$ ulkopuolella, niin päätellään, että on oltava
$\overline{\sigma}(f,X') \le \overline{\sigma}(f,X)$. Vastaavalla tavalla näytetään, että
$\underline{\sigma}(f,X) \le \underline{\sigma}(f,X')$. Ylä- ja alasummien määritelmän 
perusteella on myös $\underline{\sigma}(f,X') \le \overline{\sigma}(f,X')$, joten väite 
seuraa. \loppu

Seuraava väittämä konkretisoi ylä- ja alaintegraalit lukujonojen avulla. Väittämä nojaa
oleellisesti edelliseen.  
\begin{*Lause} \label{ylä- ja alaintegraalit raja-arvoina} Olkoon $f$ määritelty ja rajoitettu
välillä $[a,b]$. Tällöin jokaiselle jonolle $\seq{X_n}$ välin $[a,b]$ jakoja, jolle
$h_{X_n} \kohti 0$, pätee
\[
\lim_n \overline{\sigma}(f,X_n)=\overline{\int_a^b} f(x)\,dx, \qquad
\lim_n \underline{\sigma}(f,X_n)=\underline{\int_a^b} f(x)\,dx.
\]
\end{*Lause}
\tod Olkoon $\seq{\X_n}$ oletusten mukainen jono ja olkoon $\eps>0$. Yläintegraalin määritelmän
nojalla on olemassa välin $[a,b]$ jako $X = \{x_k,\ k = 0 \ldots K\}$ siten, että
\[
0\ \le\ \overline{\sigma}(f,X)-\overline{\int_a^b} f(x)\,dx\ <\ \frac{\eps}{2}\,.
\]
Merktitään $X_n'=X_n \cup X$, jolloin $X_n'$ on jakojen $X_n$ ja $X$ tihennys. 
Tihennyksessä $X_n \ext X_n'$ jakautuvat jaon $X_n$ synnyttämistä osaväleistä täsmälleen ne,
joiden sisäpisteenä on ainakin yksi joukon $X$ piste. Tällaisia osavälejä on enintään $K-1$, 
ja näistä kunkin pituus on enintään $h_{X_n}$. Koska vain nämä osavälit vaikuttavat erotukseen
$\overline{\sigma}(f,X_n)-\overline{\sigma}(f,X_n')$, niin merkitsemällä mainittujen osavälien
osuuksia summissa $\overline{\sigma}(f,X_n)$ ja $\overline{\sigma}(f,X_n')$ symboleilla
$\overline{\sigma}_0(f,X_n)$ ja $\overline{\sigma}_0(f,X_n')$ ja olettaen, että
$|f(x)| \le M,\ x\in[a,b]$, voidaan Lauseen \ref{jaon tihennys} ja kolmioepäyhtälöm perusteella
arvioida
\begin{align*}
0\le\overline{\sigma}(f,X_n)-\overline{\sigma}(f,X_n')
 &\,=\,\left|\overline{\sigma}_0(f,X_n)-\overline{\sigma}_0(f,X_n')\right| \\ 
 &\,\le\,\left|\overline{\sigma}_0(f,X_n)\right|+\left|\overline{\sigma}_0(f,X_n')\right|
  \,\le\, (K-1) \cdot 2M \cdot h_{X_n}\,.
\end{align*}
Toisaalta koska $X_n'$ on myös $X$:n tihennys, niin pätee (Lause \ref{jaon tihennys})
\[
\overline{\sigma}(f,X_n')\le\overline{\sigma}(f,X).
\] 
Yhdistämällä epäyhtälöt seuraa
\begin{align*}
\overline{\int_a^b} f(x)\,dx\,
                \le\,\overline{\sigma}(f,X_n)\,
               &\le\,\overline{\sigma}(f,X_n')+2M(K-1)\,h_{X_n} \\[2mm]
               &\le\,\overline{\sigma}(f,X)+2M(K-1)\,h_{X_n} \\
               &<\,\overline{\int_a^b} f(x)\,dx +\frac{\eps}{2}+2M(K-1)\,h_{X_n}\,.
\end{align*}
Koska $h_{X_n} \kohti 0$ kun $n\kohti\infty$ ja koska $K$ on $n$:stä riippumaton, niin tässä 
on $\,2M(K-1\,h_{X_n}<\eps/2\,$ jostakin indeksistä $N$ eteenpäin, jolloin seuraa
\[
\overline{\int_a^b} f(x)\,dx\,\le\,\overline{\sigma}(f,X_n)\,
                                <\,\overline{\int_a^b} f(x)\,dx + \eps, \quad n>N.
\]
Tässä $\eps>0$ oli mielivaltainen ja $N\in\N$, joten lukujonon suppenemisen määritelmän nojalla
on todistettu väittämän yläsummia koskeva osa. Alasummia koskeva osaväittämä todistetaan 
vastaavalla tavalla. \loppu

Esimerkkinä Lauseiden \ref{Riemann-integroituvuus} ja \ref{ylä- ja alaintegraalit raja-arvoina}
soveltamisesta todistettakoon
\begin{Lause} \label{integroituvuus osavälillä} \vahv{(Integroituvuus osavälillä)}
Jos välillä $[a,b]$ määritelty ja rajoitettu funktio $f$ on ko.\ välillä  Riemann-integroituva,
niin $f$ on Riemann-integroituva myös jokaisella osavälillä $[c,d]\subset[a,b]$.
\end{Lause}
\tod Olkoon $\seq{X_n}$ jono välin jakoja siten, että $h_{X_n} \kohti 0$ ja
$c,d \in X_n\ \forall n$. Tällöin $X_n$ sisältää jokaisella $n$ välin $[c,d]$ jaon $X_n'$,
jolle $h_{X_n'} \le h_{X_n} \kohti 0$. Kun merkitään $X_n=$ $\{x_k,\ k=0 \ldots n\}$ ja
$X_n'=\{x_k,\ k=l-1 \ldots m\}$, niin
\begin{align*}
\overline{\sigma}(f,X_n)-\underline{\sigma}(f,X_n)
&\,=\, \sum_{k=1}^n (\overline{f}_k-\underline{f}_k)(x_k-x_{k-1})
 \,\ge\, \sum_{k=l}^m (\overline{f}_k-\underline{f}_k)(x_k-x_{k-1}) \\
&\,=\, \overline{\sigma}(f,X_n')-\underline{\sigma}(f,X_n') \,\ge\,0.
\end{align*}
Koska tässä pätee oletusten ja Lauseiden \ref{Riemann-integroituvuus} ja 
\ref{ylä- ja alaintegraalit raja-arvoina} perusteella 
\[
\overline{\sigma}(f,X_n)-\underline{\sigma}(f,X_n) 
\kohti \overline{\int_a^b} f(x)\,dx - \underline{\int_a^b} f(x)\,dx = 0,
\]
niin $\,\overline{\sigma}(f,X_n')-\underline{\sigma}(f,X_n') \kohti 0$ (Propositio
\ref{jonotuloksia} [V2]). Koska myös $h_{X_n'} \kohti 0$, niin Lauseen
\ref{ylä- ja alaintegraalit raja-arvoina} perusteella $f$:n ylä- ja alaintegraalit välillä 
$[c,d]$ ovat samat, eli $f$ on Riemann-integroituva ko.\ välillä
(Lause \ref{Riemann-integroituvuus}). \loppu

\vahv{Lauseen \ref{Riemann-integroituvuus} todistus}. \ \fbox{$\impl$} Oletetaan, että $f$ on
Riemann-integroituva välillä $[a,b]$ Määritelmän \ref{Riemannin integraali} mukaisesti. Olkoon
$\seq{X_n}$ jono välin $[a,b]$ jakoja, jolle $h_{X_n} \kohti 0$. Valitaan indeksi $n$ ja
merkitään $X_n = \{x_k,\ k = 0 \ldots K\}$. Nyt voidaan jokaisella $k = 1 \ldots K$ valita
$\xi_k,\xi_k'\in[x_{k-1},x_k]$ siten, että pätee
\begin{align*}
0\,&\le\,\overline{f}_k-f(\xi_k)\,<\,2^{-n}, \quad\ 
                \overline{f}_k=\sup_{x\in[x_{k-1},\,x_k]} f(x), \\
0\,&\le\,f(\xi_k')-\underline{f}_k\,<\,2^{-n}, \quad\ 
                \underline{f}_k=\inf_{x\in[x_{k-1},\,x_k]} f(x).
\end{align*}
Tällöin
\begin{align*}
0\,&\le\,\overline{\sigma}(f,X_n)-\sigma(f,\Xi_n,X_n)\,
                                <\,2^{-n}(b-a), \quad\ \Xi_n=\{\xi_k\}, \\
0\,&\le\,\sigma(f,\Xi_n',X_n)-\underline{\sigma}(f,X_n)\,
                                <\,2^{-n}(b-a), \quad\ \Xi_n'=\{\xi_k'\}.
\end{align*}
Koska tässä $\sigma(f,\Xi_n,X_n) \kohti A\in\R$ ja $\sigma(f,\Xi_n',X_n) \kohti A$ oletuksen
mukaan, niin seuraa, että myös $\overline{\sigma}(f,X_n) \kohti A$ ja 
$\underline{\sigma}(f,X_n) \kohti A$. Lauseen \ref{ylä- ja alaintegraalit raja-arvoina} mukaan
$A$ on tällöin $f$:n ylä- ja alaintegraalien yhteinen arvo.

\fbox{$\Leftarrow$} Oletetetaan, että $f$:n ylä- ja alaintegraaleilla välillä $[a,b]$ on
yhteinen arvo $A$. Tällöin jos $\seq{X_n}$ on jono välin $[a,b]$ jakoja ja $h_{X_n} \kohti 0$,
niin Lauseen \ref{ylä- ja alaintegraalit raja-arvoina} perusteella 
$\overline{\sigma}(f,X_n) \kohti A$ ja $\underline{\sigma}(f,X_n) \kohti A$. Tällöin myös 
$\sigma(f,\Xi_{\X_n},X_n) \kohti A\ \forall\,\Xi_{X_n}$, koska 
$\underline{\sigma}(f,X_n)\le\sigma(f,\Xi_{X_n},X_n)\le\overline{\sigma}(f,X_n)$
(Propositio \ref{jonotuloksia}\,[V2]). Siis $f$ on integroituva Määritelmän
\ref{Riemannin integraali} mukaisesti. \loppu

\Harj
\begin{enumerate}

\item
a) Olkoon $f(x)=x^2$, kun $x\in\Z$, ja $f(x)=0$, kun $x\in\R,\ x\not\in\Z$. Näytä, että $f$ on
integroituva jokaisella välillä $[a,b]\subset\R$ ja että $\int_a^b f(x)\,dx=0$. \newline
b) Olkoon $f(x)=1$, kun $x\in\{\,2^{-k},\ k=0,1,\ldots\,\}$ ja $f(x)=0$ muulloin. Laskemalla
yläsummat $\overline{\sigma}(f,X_n)$, missä $X_n$ on välin $[0,1]$ tasavälinen jako $2^n$
osaväliin, näytä, että $\int_0^1 f(x)\,dx=0$.

\item
a) Todista: Jos $f$ on integroituva välillä $[-a,a]$, niin pätee
\begin{align*}
&f\,\ \text{parillinen}   \qimpl \int_{-a}^a f(x)\,dx = 2\int_0^a f(x)\,dx, \\
&f\,\ \text{pariton} \quad\qimpl \int_{-a}^a f(x)\,dx =0.
\end{align*}
b) Todista: Jos $f$ on $L$-jaksoinen ja interoituva välillä $[0,L]$, niin $f$ on integroituva
välillä $[a,a+L]$ jokaisella $a\in\R$ ja
\[
\int_{a}^{a+L} f(x)\,dx=\int_0^L f(x)\,dx.
\]

\item \label{H-int-5: Riemann-1}
Olkoon
$\ \D f(x)= \begin{cases}
            1/x, &\text{kun}\ x>0 \\ 0, &\text{kun}\ x=0
\end{cases}$ \vspace{1mm}\newline
a) Totea, että $f$ on määritelty välillä $[0,1]$ mutta ei rajoitettu. \newline
b) Näytä, että $f$ ei ole Riemann-integroituva välillä $[0,1]$ konstruoimalla jono Riemannin
summia, jolle pätee $\,h_{\X_n} \kohti 0\,$ ja $\,\sigma(\X_n,\Xi_{X_n}) \kohti \infty$.

\item
Näytä, että jos $f$ on integroituva jokaisella välillä $[a,b]\subset\R$ ja integraali jokaisen
välin yli $=0$, niin $f(x)=0$ jokaisessa pisteessä, jossa $f$ on joko vasemmalta tai oikealta
jatkuva.

\item %\label{H-int-5: väittämiä}
Näytä Proposition \ref{jaon tihennys} avulla, että mikään Riemannin alasumma ei voi olla
suurempi kuin yläsumma, ts.\ kaikille välin $[a,b]$ jaoille $\X,\X'$ pätee
$\underline{\sigma}(f,\X)\le\overline{\sigma}(f,\X')$.
%b) Todista Korollaari \ref{integroituvuuskorollaari}. 

\item (*) \label{H-int-5: tasavälinen jako}
Näytä, että Määritelmä \ref{Riemannin integraali} ei muutu, jos välin $[a,b]$ jaot $X_n$
raja-arvossa 'Lim' (Määritelmä \ref{raja-arvo Lim}) rajoitetaan tasavälisiksi. Voidaanko myös
välipisteistöjen $\Xi_{\X_n}$ valintaa rajoittaa (esim.\ $\Xi_{\X_n}\subset\X_n$)
Riemann-integroituvuuden määritelmän muuttumatta?

\item (*) \label{H-int-5: Riemann-2}
Näytä, että jos $f$ on määritelty välillä $[a,b]$ mutta ei ole rajoitettu, niin on olemassa
jono Riemannin summia, jolle pätee $h_{\X_n} \kohti 0$ ja
$\left|\sigma(\X_n,\Xi_{X_n})\right| \kohti \infty$ (vrt.\ Harj.teht. \ref{H-int-5: Riemann-1}).
--- Mikä Määritelmän \ref{Riemannin integraali} oletus on siis tarpeeton?

\item (*) \label{H-int-5: funktion poikkeutus}
Olkoon $X$ suppenevan reaalilukujonon termeistä muodostettu (äärellinen tai numeroituva) joukko,
olkoon $f$ määritelty ja rajoitettu välillä $[a,b]$ ja olkoon $g$ määritelty välillä $[a,b]$
siten, että $g(x)=f(x)$, kun $x \not\in X$. Joukosta $\{g(x) \mid x\in X\cap[a,b]\}\subset\R$ 
tiedetään ainoastaan, että se on rajoitettu. Näytä, että jos $f$ on välillä $[a,b]$ 
Riemann-integroituva, niin samoin on $g$ ja $\int_a^b g(x)\,dx=\int_a^b f(x)\,dx$.

\item (*) \label{H-int-5: monotonisuus ja integroituvuus}
a) Olkoon $f$ määritelty ja rajoitettu välillä $[a,b]$ ja $\mathcal{X}$ välin $[a,b]$ jakojen
joukko. Liitetään jokaiseen $\X = \{x_k,\ k = 0 \ldots n\} \in \mathcal{X}$ luku
\[
\delta(f,\X)=\sum_{k=1}^n (\overline{f}_k-\underline{f}_k),
\]
missä $\overline{f}_k$ ja $\underline{f}_k$ ovat $f$:n supremum ja infimum osavälillä 
$[x_{k-1},x_k]$. Näytä, että jos joukko $\{\delta(f,\X)\,|\,\X \in \mathcal{X}\}$ on 
rajoitettu, niin $f$ on Riemann-integroituva välillä $[a,b]$. \vspace{1mm}\newline
b) Näytä  että jos $f$ on määritelty ja monotoninen (kasvava tai vähenevä) välillä $[a,b]$, niin
$f$ on Riemann-integroituva välillä $[a,b]$.

\item (*) \label{H-int-5: itseisarvon ja tulon integroituvuus}
a) Näytä, että jos $f$ on määritelty, rajoitettu ja Riemann-integroituva välillä $[a,b]$, niin
myös seuraavat funktiot ovat Riemann-integroituvia välillä $[a,b]$:
\[
f_+(x)=\max\{f(x),0\}, \quad f_-(x)=\min\{f(x),0\}.
\]
b) Näytä, että seuraavista väittämistä ensimmäinen on tosi, toinen epätosi:
\begin{align*}
f\ \ \text{Riemann-integroituva}       &\qimpl \abs{f}\,\ \text{Riemann-integroituva}, \\
\abs{f}\,\ \text{Riemann-integroituva} &\qimpl \,f\,\ \ \text{Riemann-integroituva}.
\end{align*}
c) Näytä, että jos $f$ ja $g$ ovat välillä $[a,b]$ määriteltyjä, rajoitettuja ja
Riemann-integroituvia, niin samoin on tulo $fg$. \kor{Vihje}: Kirjoita $f=f_++f_-$,
$g=g_++g_-$ (ks.\ a)-kohta).

\end{enumerate}