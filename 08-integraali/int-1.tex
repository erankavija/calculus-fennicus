\section{Integraalifunktio} \label{integraalifunktio}
\alku
\index{integraalifunktio|vahv}

Palautettakoon mieliin Luvusta \ref{väliarvolause 2}, että funktiota $F(x)$ sanotaan
funktion $f(x)$ \kor{integraalifunktioksi} avoimella välillä $(a,b)$, jos $F$ on derivoituva
välillä $(a,b)$ ja
\[
F'(x)=f(x),\quad x\in (a,b).
\]
Integraalifunktiota merkitään
\[
F(x)=\int f(x)\, dx.
\]
Hieman erikoinen merkintä viittaa integraalifunktion laskennalliseen määritelmään, joka 
esitetään myöhemmin (Luku \ref{määrätty integraali}). Integraalifunktiota sanotaan myös
\index{mzyzyrzyzy@määräämätön integraali}%
\kor{määräämättömäksi integraaliksi} (engl.\ indefinite integral, antiderivative). Tässä
'määräämätön' viittaa siihen, että integraalifunktio ei ole yksikäsitteinen, vaan siihen 
voidaan lisätä mielivaltainen vakio: Jos yksi integraalifunktio $F$ tunnetaan, niin kaikki 
\index{integroimisvakio}%
integraalifunktiot ovat esitettävissä nk.\ \kor{integroimisvakion} $C$ avulla muodossa
\[
\int f(x)\, dx=F(x)+C.
\]
Määrämätön integraali on siis itse asiassa funktiojoukko. Kyseessä on differentiaaliyhtälön
$\,y'=f(x)\,$ yleinen ratkaisu välillä $(a,b)$ (ks.\ Korollaari 
\ref{toiseksi yksinkertaisin dy}).
\begin{Exa} Määritä funktion
\[
f(x)=\begin{cases} 
     \,1, &\text{kun}\ x \le 2, \\[2mm] 
     \,\dfrac{10}{x}-2x, &\text{kun}\ x>2
     \end{cases}
\]
integraalifunktio $\R$:ssä. \end{Exa}
\ratk Koska $f(2^-)=f(2^+)=1=f(2)$, niin $f$ on jatkuva $\R$:ssä. Tunnettujen 
derivoimissääntöjen (Luku \ref{derivaatta}) perusteella on oltava
\[
F(x) = \begin{cases} 
       \,x+C_1\,, &\text{kun}\ x\in(-\infty,2), \\ 
       \,10\ln x-x^2+C_2\,, &\text{kun}\ x\in(2,\infty),
        \end{cases}
\]
jolloin on $F'(x)=f(x)$, kun $x \neq 2$. Jotta $F$ olisi derivoituva myös pisteessä $x=2$, on
$F$:n oltava tässä pisteessä ainakin jatkuva, eli on oltava
\[
2+C_1 = 10\ln 2-4+C_2 \qimpl C_2 = C_1-10\ln 2+6.
\]
Kirjoittamalla $C_1=C$ on saatu
\[
F(x)=\begin{cases} 
     \,x+C, &\text{kun}\ x \le 2, \\ 
     \,10\ln(x/2)-x^2+6+C, &\text{kun}\ x \ge 2. 
     \end{cases}
\]            
Funktion $F$ toispuoliset derivaatat pisteessä $x=2$ ovat
\[
D_-F(2)=f(2^-), \quad D_+F(2)=f(2^+).
\]
Koska $f(2^-)=f(2^+)$, niin $F$ on derivoituva myös pisteessä $x=2$ ja $F'(2)=f(2)$.
Integraalifunktio on siis löydetty. Kuvassa $f$ ja $F$, kun $C=0$. \loppu
\begin{figure}[H]
\setlength{\unitlength}{1.5cm}
\begin{center}
\begin{picture}(10,3)(-0.5,-0.7)
\put(-0.5,0){\vector(1,0){4}} \put(3.3,-0.35){$x$}
\put(0,-1){\vector(0,1){3}} \put(0.2,1.8){$f(x)$}
\put(0,1){\line(1,0){2}}
\curve(
2.0, 1.0,
2.1, 0.5619,
2.2, 0.1455,
2.3, -0.2522,
2.4, -0.6333,
2.5, -1.0)
\put(-0.3,0.9){$1$}
\put(2,0){\line(0,-1){0.1}} \put(1.9,-0.4){$2$}
\put(4.5,0){\vector(1,0){4}} \put(8.3,-0.35){$x$}
\put(5,-1){\vector(0,1){3}} \put(5.2,1.8){$F(x)$}
\put(5,0){\line(1,1){2}}
\curve(
7.0, 2.0,
7.1, 2.0779,
7.2, 2.1131,
7.3, 2.1076,
7.4, 2.0632,
7.5, 1.9814,
7.6, 1.8636,
7.7, 1.7110,
7.8, 1.5247,
7.9, 1.3056,
8.0, 1.0546,
8.1, 0.7725,
8.2, 0.4600,
8.3, 0.1177)
\put(5,1){\line(-1,0){0.1}} \put(4.7,0.9){$1$}
\put(7,0){\line(0,-1){0.1}} \put(6.9,-0.4){$2$}
\end{picture}
\end{center}
\end{figure}

'Integrointi', ymmärrettynä integraalifunktion etsimisenä, on siis derivoinnin käänteistoimi.
Sekä yksittäisten funktioiden integroimiskaavat että integraalifunktioiden yleisemmät
laskusäännöt ovatkin derivoimissääntöjä 'nurinpäin' luettuna. 
\begin{Exa} Koska (vrt.\ Luku \ref{kompleksinen eksponenttifunktio})
\[
D\ln(x+\sqrt{x^2+1}) = \frac{1}{\sqrt{x^2+1}}, \quad x\in\R,
\]
niin saadaan (koko $\R$:ssä pätevä) integroimissääntö
\[
\int \frac{1}{\sqrt{x^2+1}}\,dx \ =\ \ln(x+\sqrt{x^2+1})+C. \loppu
\]
\end{Exa}

Integrointimenetelmistä yksinkertaisin on 'taulukkomenetelmä' eli kaavakokoelma, joka perustuu
suoraan tunnettuihin derivoimissääntöihin. Seuraavaan taulukkoon on koottu tällaisista
integroimiskaavoista keskeisimmät (ml.\ esimerkin tulos). Nämä voidaan suoraan lukea Lukujen 
\ref{derivaatta}, \ref{kaarenpituus}, \ref{exp(x) ja ln(x)} ja
\ref{kompleksinen eksponenttifunktio} derivoimissäännöistä.

\[ \boxed{\begin{alignedat}[l]{2} \quad
&(1)  \quad \int x^\alpha\,dx & \ = \ 
     & \frac{\ykehys 1}{\alpha +1}\,x^{\alpha+1}+C,\quad \alpha\in\R, \ \alpha\neq -1. \quad \\
&(2)  \quad \int \frac{1}{x}\,dx & \ = \ &\ln\abs{x}+C. \\
&(3)  \quad \int \cos x\,dx & \ = \ & \sin x+C. \\
&(4)  \quad \int \sin x\,dx & \ = \ & -\cos x+C. \\
&(5)  \quad \int e^x \, dx & \ = \ & e^x+C. \\
&(6)  \quad \int \frac{1}{\cos^2 x}\,dx & \ = \ & \tan x+C. \\
&(7)  \quad \int \frac{1}{\sin^2 x}\,dx & \ = \ & -\cot x+C. \\
&(8)  \quad \int \frac{1}{\sqrt{1-x^2}}\,dx & \ = \ & \Arcsin x+C. \\
&(9)  \quad \int \frac{1}{\sqrt{x^2+1}}\,dx & \ = & \ln(x+\sqrt{x^2+1})+C. \quad \\
&(10) \quad \int \frac{1}{\sqrt{x^2-1}}\,dx & \ = \ & \ln\abs{x+\sqrt{x^2-1}}+C. \\
&(11) \quad \int \frac{1}{1+x^2}\,dx & \ = \ & \Arctan x+C. \\
&(12) \quad \int \frac{1}{\akehys 1-x^2}\,dx  & \ = \ & 
                 \frac{1}{2}\ln\left|\frac{1+x}{1-x}\right|+C.
\end{alignedat}} \]

Taulukkoa voitaisiin helposti jatkaa tunnettujen derivoimissääntöjen perusteella. Esimerkiksi
pätee (ks.\ Luku \ref{exp(x) ja ln(x)})\,:

\[ \boxed{\begin{alignedat}[l]{2} \quad
&(13) \quad \int\dfrac{\ykehys 1}{\sin x}\,dx & \ = \ & 
                     \ln\left|\dfrac{1-\cos x}{\sin x}\right|+C. \\
&(14) \quad \int\dfrac{1}{\cos x}\,dx & \ = \ & 
                    -\ln\left|\dfrac{1-\sin x}{\akehys\cos x}\right|+C. \hspace{35mm}
\end{alignedat}} \]

Sääntöjä (1)--(14) voidaan tarkemmin käyttää sellaisilla \pain{avoimilla} \pain{väleillä},
joilla annettu funktio $f$ ja integraalifunktio $F$ ovat molemmat määriteltyjä ja jälkimmäinen
on derivoituva. Esimerkiksi säännöt (3)--(5) ovat päteviä koko $\R$:ssä, sääntö (2) on pätevä 
väleillä $(-\infty,0)$ ja $(0,\infty)$, sääntö (8) välillä $(-1,1)$, ja sääntö (7) väleillä
$(k\pi,(k+1)\pi),\ k\in\Z$. Silloin kun integroimissäännön pätevyys on rajoitettu useammalle
pistevieraalle $\R$:n osavälille (kuten sännöissä (2) ja (7)), ei integroimisvakion tarvitse
olla eri osaväleillä sama.

Taulukkokaavat (9), (10) ja (12) voidaan ilmaista myös hyperbolisten käänteisfunktioiden
$\arsinh$, $\Arcosh$ ja $\artanh$ avulla, sillä pätee
(ks.\ Luku \ref{kompleksinen eksponenttifunktio})
\begin{align*}
&\ln(x+\sqrt{x^2+1}) \,=\, \arsinh x, \quad x\in\R, \\
&\ln|x+\sqrt{x^2-1}|\, = \begin{cases}
                         \,\Arcosh x, &\text{kun}\ x>1, \\ -\Arcosh|x|, &\text{kun}\ x<-1,
                         \end{cases} \\
&\frac{1}{2}\ln\left|\frac{1+x}{1-x}\right|\, = \begin{cases}
                                                \,\artanh x,      &\text{kun}\ |x|<1, \\
                                                \,\artanh\,(1/x), &\text{kun}\ |x|>1.
                                                \end{cases}
\end{align*}  

Integraalifunktion etsiminen --- silloin kun funktio ei löydy suoraan yksinkertaisimpien
derivoimissääntöjen perusteella --- on taitolaji, jota perinteisessä matematiikassa on 
harjoiteltu varsin paljon. Lajilla on urheilullista mielenkiintoa mm.\ siksi, että monien
varsin yksinkertaisten funktioiden, kuten 
\[
\frac{1}{x^4+1}\,, \quad \frac{\sqrt{x^2+1}}{x}
\]
integraalifunktioiden määrittäminen on työlästä. On myös paljon funktioita, esim.\ 
\[
e^{x^2}, \quad
\cos x^2, \quad
\frac{e^x}{x}\,, \quad
\frac{\sin x}{x}\,, \quad
\frac{1}{\ln x}\,, \quad
\sqrt{x}\sin x, \quad
\sqrt{x^4+1}, \quad
\sqrt{1+\sin^2 x},
\]
joiden integraalifunktiot eivät ole nk.\ 
\index{alkeisfunktio}%
\kor{alkeisfunktioita}, eli ne eivät ole 
polynomien tai rationaalifunktioiden, juurilausekkeiden, eksponentti- tai 
logaritmifunktioiden tai trigonometristen funktioiden tai niiden käänteisfuntioiden avulla
ilmaistavissa olevia lausekkeita. Sanotaan tällöin, että funktio ei ole integroitavissa
\index{suljettu muoto (integroinnin)}%
\kor{suljetussa muodossa} (engl.\ in closed form). Tällaisista integraalifunktioista on ennen
puhuttu kunnioittavaan sävyyn, kutsuen niitä mm. 'korkeammiksi transkendenttifunktioiksi'.
Nykyisin laskimet ja tietokoneet, ja niiden myötä numeerisen laskennan helppous, ovat
pudottaneet tällaisetkin funktiot tavallisten kuolevaisten joukkoon.

Tietokoneiden myötä on tullut myös symbolisen laskennan mahdollisuus, mikä on vähentänyt
perinteisen käsityötaidon merkitystä integraalifunktioiden etsinnässä. Tavallisimmat 
integroimissäännöt ja -menetelmät on silti edelleen syytä tuntea, mm.\ koska tällä 
laskutekniikalla on yleisempääkin käyttöä. Tässä ja kahdessa seuraavassa luvussa käydään läpi
integroimistekniikan yleisimmät menetelmät.

\subsection{Kolme yleistä integroimissääntöä}

Yleisiä derivoimissääntöjä 'nurinpäin' lukemalla saadaan yleisiä integroimissääntöjä. 
Seuraavista kolmesta säännöstä ensimäinen kertoo, että integrointi on derivoinnin tavoin 
\pain{lineaarinen} toimitus. Toinen ja kolmas sääntö ovat yhdistetyn funktion
derivoimissääntöön suoraan perustuvia.
\begin{align*}
&\int f(x)\, dx=F(x)+C\ \ \ja\,\ \int g(x)\, dx=G(x)+C \\
&\impl \quad \left\{ \begin{array}{lr}
     \int\, [\alpha f(x)+\beta g(x)]\, dx
                =\alpha F(x) +\beta G(x) +C,\quad \alpha,\beta\in\R, &\qquad \text{(I-1)} \\
     \int f(ax+b)\, dx 
                =\dfrac{1}{a}\,F(ax+b)+C,\quad a,b\in\R, \ a\neq 0, &\qquad \text{(I-2)} \\[2mm]
     \int f(g(x))g'(x)\, dx = F(g(x))+C.     &\qquad \text{(I-3)}
                     \end{array} \right.
\end{align*}
Näissä säännöissä (kuten säännöissä (1)--(14)) on huomioitava, että integroimisvakio $C$ on nk.\
\index{geneerinen vakio}%
\kor{geneerinen} vakio, joka voi eri yhteyksissä (kuten lausekkeissa $F(x)+C$ ja $G(x)+C$) aina
saada erilaisia arvoja, ellei toisin sovita.
\begin{Exa} \label{E10.1.2}
$\D\int\frac{1}{x^2+x}\, dx=\,?$
\end{Exa}
\ratk Hajotetaan integroitava funktio ensin kahden yksinkertaisemman funktion summaksi:
\begin{align*}
&\frac{1}{x^2+x} \,=\, \frac{1}{x(x+1)} \,=\, \frac{A}{x}+\frac{B}{x+1}
                                         \,=\, \frac{(A+B)x+A}{x(x+1)} \\ 
&\qquad\ \qimpl \begin{cases} \,A+B=0 \\ \,A=1 \end{cases}
         \ekv\quad \begin{cases} \,A=1 \\ \,B=-1  \end{cases} 
         \impl\quad \frac{1}{x^2+x} \,=\, \frac{1}{x}-\frac{1}{x+1} \\[2mm]
&\qquad\ \qimpl \int \frac{1}{x^2+x}\,dx\ 
                \overset{\text{(I-1)}}{=}\ \int \frac{1}{x}\, dx - \int\frac{1}{x+1}\, dx \\
&\hspace{48mm}\ \overset{(2),\,\text{(I-2)}}{=}\ \ln\abs{x}-\ln\abs{x+1}+C\
                    =\ \underline{\underline{\ln\left|\frac{x}{x+1}\right|+C}}. \loppu
\end{align*}
\begin{Exa} Johda taulukkokaava (12). \end{Exa}
\ratk Edellistä esimerkkiä mukaillen lasketaan
\begin{align*}
&\frac{1}{1-x^2} \,=\, \frac{1}{(1+x)(1-x)}
                \,=\, \frac{1}{2}\,\frac{1}{1+x}+\frac{1}{2}\,\frac{1}{1-x} \\
&\ \qimpl \int\frac{1}{1-x^2}\,dx \,=\, \frac{1}{2}\ln|1+x|-\frac{1}{2}\ln|1-x|+C
                                  \,=\, \frac{1}{2}\ln\left|\frac{1+x}{1-x}\right|+C. \loppu
\end{align*}

Esimerkeissä integraalifunktio löydettiin nk.
\index{osamurtokehitelmä}%
\kor{osamurtokehitelmän} avulla. Tämä on
yleisempiinkin rationaalifunktiohin soveltuva menetelmä, kuten nähdään jäljempänä
Luvussa \ref{osamurtokehitelmät}.
\begin{Exa} Säännön (I-3), taulukkokaavan (1) ja derivoimissääntöjen $\dif\ln|x|=1/x$ ja
$\dif\sin x=\cos x$ perusteella
\begin{align*}
&\int \frac{1}{x}\,\ln\abs{x}\,dx = \frac{1}{2}\,(\ln\abs{x})^2 + C, \\
&\int \sin^4 x \cos x\, dx = \frac{1}{5}\,\sin^5 x + C. \loppu
\end{align*}
\end{Exa}
\begin{Exa}
$\D\int\frac{1}{x^2+x+1}\, dx=\,?$
\end{Exa}
\ratk Koska
\[
\frac{1}{x^2+x+1}=\frac{1}{(x+\frac{1}{2})^2+\frac{3}{4}}
                 =\frac{4}{3}\,\frac{1}{(\frac{2x+1}{\sqrt{3}})^2+1}\,,
\]
niin
\begin{align*}
\int\frac{1}{x^2+x+1}\,dx\ 
&\,\ \ \overset{\text{(I-1)}}{=}\ \ 
          \frac{4}{3}\int \frac{1}{(\frac{2x+1}{\sqrt{3}})^2+1}\,dx \\
&\overset{\text{(I-2)},(11)}{=}\ 
          \frac{4}{3}\cdot\frac{\sqrt{3}}{2}\Arctan\left(\frac{2x+1}{\sqrt{3}}\right)+C \\
&\quad\, =\ \ 
 \underline{\underline{\frac{2}{\sqrt{3}}\Arctan\left(\frac{2x+1}{\sqrt{3}}\right)+C}}. \loppu
\end{align*}

\begin{Exa} \label{cos-integraaleja}
$\text{a)}\ \D\int\cos^2 x\, dx=\,? \quad\text{b)}\ \int\cos^3 x\, dx=\,?$
\end{Exa}
\ratk Koska
\[
\cos^2 x=\frac{1}{2}+\frac{1}{2}\cos 2x, \quad \cos^3 x=(1-\sin^2 x)\cos x,
\]
niin
\begin{align*}
&\text{a)}\ \int \cos^2 x\, dx\,=\,\frac{1}{2}\,x+\frac{1}{4}\sin 2x+C\,
                     =\, \underline{\underline{\frac{1}{2}\,(x+\cos x\sin x)+C}}, \\
&\text{b)}\ \int \cos^3 x\, dx\,=\,\int\left(\cos x-\sin^2 x\cos x\right)\,dx\,
                     =\, \underline{\underline{\sin x-\frac{1}{3}\sin^3 x+C}}. \loppu
\end{align*}

\Harj
\begin{enumerate}

\item 
Määritä se funktion $2 x - 3$ integraalifunktio, jonka kuvaaja sivuaa suoraa $x+y = 0$.

\item 
Ratkaise seuraavat alkuarvotehtävät välillä $(-\infty,\infty)$\,: \newline
a) \ $F'(x)=2x^2-\abs{x},\ F(0) = 1$. \newline
b) \ $F'(x)=\abs{x^2+4x},\ F(-4)=0$. \newline
c) \ $F'(x)=\max\{x,\,8x-x^2\},\ F(4)=-1$. \newline
d) \ $F'(x)=\min\{4-4x+x^2,\,40+2x-x^2\},\ F(0)=180$.

\item 
Määritä seuraavien funktioiden integraalifunktiot taulukkokaavoihin (1)--(12) ja sääntöihin
(I-1)--(I-3) vedoten. Muunna funktio tarvittaessa ensin integroinnin kannalta
soveliaampaan muotoon.
\begin{align*}
&\text{a)}\ \ (2x^2 - 5 x + 7)(4x-4) \qquad 
 \text{b)}\ \ \frac{1}{\sqrt{2x + 3}} \qquad
 \text{c)}\ \ \frac{3}{3-\pi x} \qquad
 \text{d)}\ \ \frac{x^2}{4x^3-1} \\
&\text{e)}\ \ \frac{2x^3+x}{(x^4+x^2+1)^3} \qquad
 \text{f)}\ \ \frac{x}{(x^2+1)\sqrt{x^2+1}} \qquad
 \text{g)}\ \ \frac{1}{x\ln\abs{x}} \qquad 
 \text{h)}\ \ \frac{e^x}{e^x+1} \\
&\text{i)}\ \ \frac{1}{x(\ln\abs{x})^2} \qquad
 \text{j)}\ \ \frac{1}{\sqrt{x^2-3}} \qquad
 \text{k)}\ \ \frac{1}{\sqrt{x^2+3}} \qquad 
 \text{l)}\ \ \frac{x+2}{\sqrt{x^2-1}} \\
&\text{m)}\ \ \frac{3x-2}{\sqrt{2x^2+5}} \qquad
 \text{n)}\ \ \frac{1}{x^2+14x+50} \qquad
 \text{o)}\ \ \frac{1}{x^2+14x+48} \\
&\text{p)}\ \ \frac{1}{x^2+2x+10} \qquad
 \text{q)}\ \ \frac{1}{x^2+3x-10} \qquad
 \text{r)}\ \ \frac{2x+1}{x^2+3x-10} \\
&\text{s)}\ \ \sin 2x\,\cos^4 x \qquad
 \text{t)}\ \ \frac{\sin x}{\cos^3 x} \qquad
 \text{u)}\ \ \tan x \qquad
 \text{v)}\ \ \tan^2 x \qquad
 \text{x)}\ \ \tan^3 x \\
&\text{y)}\ \ \frac{e^{\tan x}}{\cos^2 x} \qquad
 \text{z)}\ \ \sin^5 x \qquad
 \text{å)}\ \ \cosh^2 x \qquad 
 \text{ä)}\ \ \tanh x \quad\ \
 \text{ö)}\ \ \frac{\arsinh x}{\sqrt{x^2+1}}
\end{align*}

\item (*)
Määritä $a\in\R$ ja $\R$:ssä jaksollinen funktio $u$ siten, että $y(x)=ax+u(x)$ on 
alkuarvotehtävän $\,y'=\abs{\sin x},\ y(0)=0\,$ ratkaisu.

\item (*)
Olkoon $f(0)=0$ ja $f$:n määritelmä muualla kuin origossa
\[
\text{a)}\ \ f(x)=\frac{x}{\sqrt{\abs{x}}}\,, \qquad
\text{b)}\ \ f(x)=\frac{x}{\abs{x}}\,, \qquad
\text{c)}\ \ f(x)=\frac{1}{x}\left(\sin\frac{1}{x^2}+x^2\cos\frac{1}{x^2}\right).
\]
Konstruoi $f$:n integraalifunktio välillä $(-1,1)$ tai näytä, että ko.\ välillä $f$:llä
ei ole integraalifunktiota.

\end{enumerate}