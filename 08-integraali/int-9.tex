\section{Numeerinen integrointi} \label{numeerinen integrointi}
\alku
\index{numeerinen integrointi|vahv}

\kor{Numeerisella integroinnilla} tarkoitetaan määrätyn integraalin, eli reaaliluvun
\[
I(f,a,b)=\int_a^b f(x)\,dx
\]
laskemista numeerisin keinoin (likimäärin). Jatkossa integroitavalle funktiolle $f$ asetetaan
melko voimakkaita säännöllisyysvaatimuksia. Vähimmäisvaatimus on, että $f$ jatkuva välillä
$[a,b]$, jolloin $f$ on ko.\ välillä myös Riemann-integroituva
(Lause \ref{Analyysin peruslause}).

Numeerisen integroinnin menetelmät ovat yleensä nk.\
\index{kvadratuuri!numeerinen} \index{numeerinen kvadratuuri}%
\kor{numeerisia kvadratuureja}
(tai kvadratuurisääntöjä, engl.\ quadrature rule)\footnote[2]{Kvadratuuri tarkoittaa
kirjaimellisesti 'neliöimistä'. Termi viittaa integraalin ja pinta-alan väliseen yhteyteen.},
joissa integraalia approksimoidaan äärellisenä summana muotoa
\begin{equation} \label{yleinen kvadratuuri}
\int_a^b f(x)\,dx\approx\sum_{i=1}^N w_i f(x_i).
\end{equation}
\index{kvadratuuripiste, -paino}%
Tässä pisteitä $x_i,\ i=1 \ldots N$ (yleensä $x_i\in[a,b]$) sanotaan \kor{kvadratuuripisteiksi}
ja lukuja $w_i$ \kor{kvadratuuripainoiksi}. Kvadratuuripisteitä ja -painoja valittaessa 
integroimisväli jaetaan usein ensin osaväleihin $[x_{k-1},x_k],\ k=1 \ldots n\,$ 
(kuten määrätyn integraalin määrittelyssä alunperin). Tällöin integraalin additiivisuuden
nojalla on
\begin{equation} \label{integraalin hajotelma}
I(f,a,b)=\int_a^b f(x)\,dx = \sum_{k=1}^n \int_{x_{k-1}}^{x_k} f(x)\,dx = \sum_{k=1}^n I_k\,.
\end{equation}
Jos integroitava funktio on säännöllinen ja osavälit $[x_{k-1},x_k]$ riittävän lyhyitä, voidaan
osaintegraalit
\[
I_k=\int_{x_{k-1}}^{x_k} f(x)\,dx
\]
laskea likimäärin käyttämällä suhteellisen yksinkertaista kvadratuuria muotoa 
\eqref{yleinen kvadratuuri}, missä $a=x_{k-1}$ ja $b=x_{k}$. Perustana kvadratuurin
yksinkertaistumiselle lyhyellä välillä on Taylorin lause, jonka mukaan säännöllinen funktio on
likimain (matala-asteinen) polynomi lyhyellä välillä (ks.\ Luku \ref{taylorin lause}). 
Numeerisen integroinnin virheanalyysin lähtöajatuksena onkin juuri vertailu polynomeihin.

Jatkossa ajatellaan väli $[a,b]$ alkuperäisen integroimisvälin lyhyeksi osaväliksi ja $f$ ko.\
välillä (riittävän) säännölliseksi. Tällaisten lähtöoletusten vallitessa yksinkertaisimmat
numeeriset kvadratuurit ovat:

\vspace{3mm}

\begin{tabular}{ll}
\kor{Keskipistesääntö}: \index{keskipistesääntö}
        \quad & $\displaystyle{\int_a^b f(x)\,dx\approx (b-a)f(\frac{a+b}{2})}$. \\ \\
\kor{Puolisuunnikassääntö}: \index{puolisuunnikassääntö}
        \quad & $\displaystyle{\int_a^b f(x)\,dx\approx \frac{1}{2}(b-a)[f(a)+f(b)]}$. \\ \\
\kor{Simpsonin sääntö}: \index{Simpsonin sääntö}
        \quad & $\displaystyle{\int_a^b f(x)\,dx\approx 
                         \frac{1}{6}(b-a)\left[f(a)+4f(\frac{a+b}{2})+f(b)\right]}$.
\end{tabular}

\vspace{4mm}

Näistä etenkin Simpsonin sääntö (osaväleillä käytettynä) on laskinten ja tietokoneiden
yleisesti käyttämä numeerisen integroinnin menetelmä.\footnote[2]{Simpsonin sääntö on
numeerisen integroinnin klassikko, jota on aikanaan käytetty paljon käsinlaskussakin. Säännön
keksi englantilainen matemaatikko \hist{Thomas Simpson} (1710-1761). \index{Simpson, T.|av}}
Puolisuunnikassääntöön viitataan usein myös nimellä
\index{trapetsi}%
\kor{Trapetsi}. Nimensä mukaisesti sääntö
antaa integraalin arvoksi \kor{puolisuunnikkaan} (trapetsin) pinta-alan:
\begin{figure}[H]
\setlength{\unitlength}{1cm}
\begin{center}
\begin{picture}(10,5)(0,0)
\put(0,0){\vector(1,0){9}} \put(8.8,-0.4){$x$}
\put(1.5,-1){\vector(0,1){6}} \put(1.7,4.8){$y$}
\put(3.9,-0.5){$a$} \put(6.9,-0.5){$b$} 
\put(5.3,1.3){$A$} \put(8,3.7){$y=f(x)$}
\thicklines
\put(4,0){\line(0,1){2.8}} \put(7,0){\line(0,1){3.3179}} 
\put(4,0){\line(1,0){3}} \drawline(4,2.8)(7,3.3179)
\thinlines
\curve(
    1.0000,    3.0000,
    1.5000,    3.7635,
    2.0000,    4.0000,
    2.5000,    3.8848,
    3.0000,    3.5679,
    3.5000,    3.1733,
    4.0000,    2.8000,
    4.5000,    2.5211,
    5.0000,    2.3843,
    5.5000,    2.4117,
    6.0000,    2.6000,
    6.5000,    2.9202,
    7.0000,    3.3179,
    7.5000,    3.7130,
    8.0000,    4.0000)
\end{picture}
\end{center}
\end{figure}

Kun ym.\ sääntöjä käytetään osaväleillä hajotelmassa \eqref{integraalin hajotelma}, niin
tuloksena on alkuperäisen integraalin approksimaatio muotoa \eqref{yleinen kvadratuuri}.

Sanotaan tällöin, että kyseessä on \kor{yhdistetty} (engl. composite) kvadratuuri.
\index{yhdistetty!a@keskipistesääntö}%
Esimerkiksi \kor{yhdistetty keskipistesääntö} on kvadratuuri muotoa \eqref{yleinen kvadratuuri},
missä valitaan
\[
x_i=\frac{1}{2}(x_{i-1}+x_i), \quad w_i=x_{i}-x_{i-1}\,, \quad i=1 \ldots n=N.
\]
Yhdistetty keskipistesääntö on siis eräs integraalia approksimoivista Riemannin summista.
Suoraan Riemannin summiin perustuvista approksimaatioista tämä on yleensä tarkin 
(ks.\ virhearviot jäljempänä).

Jos hajotelmassa \eqref{integraalin hajotelma} jako on tasavälinen ja vastaavassa yhdistetyssä
kvadratuurissa \eqref{yleinen kvadratuuri} merkitään $\,h=x_i-x_{i-1}$ (= peräkkäisten
kvadratuuripisteiden väli) ja $f_i=f(x_i)$, niin \kor{yhdistetty trapetsi} saa muodon
\index{yhdistetty!b@trapetsi, Simpson}%
\[
\int_a^b f(x)\,dx\approx h\left[\frac{1}{2}f_0+f_1+\cdots + f_{n-1}+\frac{1}{2}f_n\right]
\]
ja \kor{yhdistetty Simpsonin sääntö} muodon
\[
\int_a^b f(x)\,dx
       \approx \frac{h}{3}\left[\,f_0+4f_1+2f_2+\cdots +2f_{2n-2}+ 4f_{2n-1}+f_{2n}\,\right].
\]
Yhdistetyssä Simpsonin säännössä joka toinen kvadratuuripiste on osavälijaon jakopiste ja joka
toinen (suurimmalla painokertoimella varustettu) on osavälin keskipiste. 

\begin{Exa} \label{Trapetsi vastaan Simpson 1}
Kun integraalia
\[
\int_0^1 xe^{-x^2}/(x+1)\,dx
\]
approksimoidaan käyttäen tasavälistä yhdistettyä trapetsia ja Simpsonia, saadaan seuraavat
tulokset:
\begin{center}
\begin{tabular}{lll}
pisteitä & Trapetsi & Simpson \\ \hline \\
3 & 0.17578506065833 & 0.20372346078015 \\
21 & 0.20230210857337 & 0.20256800959472 \\
201 & 0.20256528212086 & 0.20256794027440 \\
2001 \quad & 0.20256791368608 \quad & 0.20256794026753
\end{tabular}
\end{center}
Tässä Simpson on Trapetsia selvästi nopeampi. Tämä on odotettavissa, koska integroitava funktio
on sileä, ks.\ virhearviot jäljempänä. \loppu
\end{Exa}
\begin{Exa} \label{Trapetsi vastaan Simpson 2}
Sama vertailuasetelma kuin edellisessä esimerkissä. Lasketaan
\[
\int_0^1 e^{-x}/(\sqrt{x}+1)\,dx.
\]
Tulokset ovat tässä tapauksessa:
\begin{center}
\begin{tabular}{lll}
pisteitä & Trapetsi & Simpson \\ \hline \\
3 & 0.47363365 & 0.43418824 \\
21 & 0.40741212 & 0.40604377 \\
201 & 0.40520890 & 0.40516458 \\
2001 \quad & 0.40513820 \quad & 0.40513679
\end{tabular}
\end{center}
Konvergenssi on nyt selvästi hitaampaa kuin edellisessä esimerkissä, eikä eri menetelmillä
ole tässä merkittävää eroa. Syynä on ilmeisesti se, että integroitava funktio on vähemmän
säännöllinen kuin edellisessä esimerkissä (jatkuva, mutta ei jatkuvasti derivoituva
integroimisvälillä). \loppu
\end{Exa}
Em.\ esimerkeissä on merkille pantavaa, että kun Esimerkin 2 integraalissa tehdään sijoitus
$\sqrt{x}=t$, on tuloksena Esimerkin 1 integraali (kertoimella 2). Sijoitus siis kannattaa
tässä tehdä, vaikka se ei muuten (suljetussa muodossa integroimisen kannalta) tee tehtävää 
helpommaksi. Myös osittaisintegroinnilla, sarjakehitelmillä ym. voi usein manipuloida tehtävää
niin, että integroitavasta funktiosta tulee säännöllisempi, jolloin numeeriset kvadratuurit
toimivat paremmin.

\subsection{Virhearvioista}

Osavälijakoon perustuvan yhdistetyn kvadratuurin virhe arvioidaan ensin erikseen kullakin
osavälillä. Tarkastellaan jatkossa edellä esitettyä kolmea esimerkkiä: Keskipistesääntö, 
Trapetsi ja Simpson.

Arvioitaessa virhettä yksittäisellä osavälillä on ensimmäisenä tehtävänä tutkia, \pain{kuinka} 
\pain{korkea-asteisille} p\pain{ol}y\pain{nomeille} \pain{kvadratuuri} \pain{on} 
\pain{tarkka}, ts. on määrättävä indeksi $m$ siten, että kvadratuuri on tarkka polynomille
astetta $m$ mutta ei astetta $m+1$. Mainituissa esimerkkitapauksissa on tulos seuraava:
\begin{center}
\begin{tabular}{ll}
Sääntö & \hspace{2mm} Tarkka polynomeille astetta \\ \hline \\
Keskipistesääntö & $\quad m=1$ \\ Trapetsi & $\quad m=1$ \\ Simpson & $\quad m=3$
\end{tabular}
\end{center}

Aloitetaan keskipistesäännöstä. Oletetetaan, että funktio $f$ on tarkasteltavalla (osa)välillä
$[a,b]$ kahdesti jatkuvasti derivoituva. Merkitään $c=(a+b)/2$ ja käytetään
integroimisvirheelle symbolia $E(f)\,$:
\[ 
E(f) = \int_a^b f(x)\,dx - (b-a)f(c). 
\]
Integroimisvirheen arvioiminen perustuu kahteen perushavaintoon: Ensinnäkin nähdään, että pätee
\begin{equation} \label{havainto 1}
E(f+g) = E(f)+E(g). \tag{a} 
\end{equation}
(Yleisemmin $E(f)$ on lineaarinen $f$:n suhteen, eli 
$E(\alpha f + \beta g) = \alpha E(f) + \beta E(g)$, $\alpha,\beta \in \R$.) Toinen
perushavainto on em.\ taulukkoon perustuva:
\begin{equation} \label{havainto 2}
f(x) = A + Bx \quad (A,B \in \R) \qimpl E(f)=0. \tag{b}
\end{equation}
Olkoon nyt $p(x)=f$:n Taylorin polynomi astetta $n=1$ pisteessä $c$, eli
\[ 
p(x) = f(c) + f'(c)(x-c). 
\]
Tällöin on havaintojen \eqref{havainto 1}--\eqref{havainto 2} perusteella, ja koska
$(f-p)(c)=0$,
\[
E(f) = E(f-p+p) = E(f-p) + E(p) = E(f-p) = \int_a^b (f-p)(x)\,dx.
\]
Taylorin lauseen (Lause \ref{Taylor}) perusteella on
\[ 
f(x)-p(x) = \frac{1}{2}f''(\xi)(x-c)^2 = f''(\xi)\,\omega(x), \quad x \in [a,b], 
\]
missä $\xi = \xi(x) \in (a,b)$ ja $\,\omega(x)=\tfrac{1}{2}(x-c)^2 \ge 0$. Näin ollen jos
\[ 
M_1 = \min_{x\in[a,b]} f''(x), \quad M_2 = \max_{x\in[a,b]} f''(x), 
\]
niin voidaan päätellä
\begin{align*}
&M_1\,\omega(x) \,\le\, (f-p)(x) \,\le\, M_2\,\omega(x), \quad x \in [a,b] \\[1mm]
&\impl\quad M_1\,\int_a^b \omega(x)\,dx 
                \,\le\, \int_a^b (f-p)(x)\,dx \,\le\, M_2\,\int_a^b \omega(x)\,dx \\
&\ekv\quad \frac{1}{24}\,M_1\,(b-a)^3    
                \,\le\, \int_a^b (f-p)(x)\,dx \,\le\, \frac{1}{24}\,M_2\,(b-a)^3.
\end{align*}
Siis jollakin $\eta \in [M_1,M_2]$ pätee
\[ 
E(f) = \frac{\eta}{24}\,(b-a)^3. 
\]
Koska $f''$ on jatkuva välillä $[a,b]$, niin $M_1=f''(x_1)$ ja $M_2=f''(x_2)$ joillakin
$x_1,x_2\in[a,b]$ (Lause \ref{Weierstrassin peruslause}), jolloin $\eta = f''(\xi)$ jollakin 
$\xi \in [a,b]$ (Lause \ref{ensimmäinen väliarvolause}). Keskipistesäännölle on näin johdettu
virhekaava
\[
\boxed{\quad E(f) = \frac{1}{24}\,(b-a)^3 f''(\xi), \quad \xi \in [a,b]
                                     \quad\text{(Keskipistesääntö)}. \quad}
\]

Trapetsille voidaan johtaa virhekaava vastaavaan tapaan. Tässä tapauksessa on
vertailupolynomiksi $p$ kuitenkin syytä valita Taylorin polynomin sijasta ensimmäisen asteen
interpolaatiopolypainnomi, jolle pätee $p(a)=f(a)$ ja $p(b)=f(b)$
(ks.\ Luku \ref{interpolaatiopolynomit}). Virhekaavaksi saadaan
(Harj.teht.\,\ref{H-int-9: Trapetsin virhekaava})
\[
\boxed{\quad E(f) = - \frac{1}{12}\,(b-a)^3 f''(\xi), \quad \xi \in [a,b] 
                                      \quad \text{(Trapetsi)}. \quad}
\]

Keskipistesäännön ja trapetsin virhekaavoista nähdään, että jos $f$ on ylöspäin kaareutuva
($f''(x)>0$ välillä $[a,b$]), niin keskipistesääntö antaa integraalille liian pienen ja 
Trapetsi liian suuren arvon. Em.\ virheanalyysiin liittyen ilmiön voi selittää niin, että 
keskipistesääntö integroi oikein $f$:n Taylorin polynomin $T_1(x,c)$, kun taas Trapetsi
integroi oikein $f$:n interpolaatiopolynomin, vrt. kuvio.
%\begin{figure}[H]
%\begin{center}
%\import{kuvat/}{kuvaipol-6.pstex_t}
%\end{center}
%\end{figure}
\begin{figure}[H]
\setlength{\unitlength}{1cm}
\begin{center}
\begin{picture}(10,3)(0,0)
\put(2,0){\vector(1,0){7}} \put(8.8,-0.4){$x$}
\put(2,0){\vector(0,1){3}} \put(2.2,2.8){$y$}
\dashline{0.1}(4.0,1.8)(7.0,2.3179) \dashline{0.1}(4.0,1.1306)(7.0,1.7)
\put(4,0){\line(0,1){0.1}} \put(5.5,0){\line(0,1){0.1}} \put(7,0){\line(0,1){0.1}}
\put(3.9,-0.5){$a$} \put(6.9,-0.5){$b$} \put(5.4,-0.5){$c$} 
\put(7.2,1.652){$\scriptstyle{\text{Keskipistesääntö}}$}
\put(7.2,2.316){$\scriptstyle{\text{Trapetsi}}$}
\curve(
    4.0000,    1.8000,
    4.5000,    1.5211,
    5.0000,    1.3843,
    5.5000,    1.4117,
    6.0000,    1.6000,
    6.5000,    1.9202,
    7.0000,    2.3179)
\end{picture}
\end{center}
\end{figure}
Myös Simpsonin säännölle voidaan johtaa virhekaava samantyyppisellä päättelyllä kuin edellä.
Olettaen, että $f$ on neljä kertaa jatkuvasti derivoituva välillä $[a,b]$, saadaan
virhekaavaksi (Harj.teht.\,\ref{H-int-9: Simpsonin 2})
\[
\boxed{\quad E(f) = - \frac{1}{90}\,h^5 f^{(4)}(\xi), 
              \quad h=(b-a)/2, \quad \xi \in [a,b] \quad \text{(Simpson)}. \quad}
\]

Em.\ virhekaavoista voidaan edelleen johtaa virhearvioita yhdistetyille kvadratuureille.
Tarkastellaan jälleen esimerkkinä keskipistesääntöä, rajoittuen tasaväliseen jakoon eli
olettaen väli $[a,b]$ jaetuksi osaväleihin $[x_{k-1},x_k],\ k=1\ldots n$, missä
$x_k-x_{k-1}=h=(b-a)/n\ \forall k$. Olkoon funktio $f$ kahdesti jatkuvasti
derivoituva välillä $[a,b]$, ja olkoon $M_1$ ja $M_2$ $f''$:n minimi- ja maksimiarvot välillä
$[a,b]$. Tällöin jos $E_k(f) =$ keskipistesäännön integrointivirhe osavälillä $[x_{k-1},x_k]$,
niin em.\ virhekaavan perusteella
\[ 
\frac{1}{24}\,M_1 h^3 \le E_k(f) \le \frac{1}{24}\,M_2 h^3. 
\]
Summaamalla yli $k$:n ja huomioimalla, että $\,nh=b-a\,$ saadaan kokonaisvirheelle 
$E(f) = \sum_{k=1}^n E_k(f)$ arviot
\[ 
\frac{1}{24}\,(b-a)M_1 h^2 \le E(f) \le \frac{1}{24}\,(b-a)M_2 h^2. 
\]
Tässä on $M_1=f(x_1),\ M_2=f(x_2)\,$ joillakin $x_1,x_2 \in [a,b]$, joten päädytään 
väliarvomuotoiseen virhekaavaan kuten edellä. Tasavälisen trapetsin ja Simpsonin tapauksessa
menetellään vastaavasti, jolloin tuloksena ovat seuraavat virhekaavat:
\begin{center}
\fbox{\begin{tabular}{ll}
Tasavälinen, yhdistetty kp.-sääntö:    & $E(f)=\ \frac{1}{24}(b-a)f''(\xi)\,h^2$ \ykehys\\ \\
Tasavälinen, yhdistetty trapetsi:      & $E(f)=-\frac{1}{12}(b-a)f''(\xi)\,h^2$ \\ \\
Tasavälinen, yhdistetty Simpson:       & $E(f)=-\frac{1}{180}(b-a)f^{(4)}(\xi)\,h^4 \ $ \akehys
\end{tabular}}
\end{center}
Tässä on kaikissa tapauksissa $\xi \in [a,b]$, ja $h$ on lähimpien kvadratuuripisteiden väli,
eli joko $h=$ osavälin pituus (Keskipistesääntö, Trapetsi), tai $h=$ puolet osavälin
pituudesta (Simpson).

Muun kuin tasavälisen jaon tapauksessa voidaan em.\ yhdistettyjen kvadratuurien virhe arvioida
muodossa $\abs{E(f)} \le \ldots\,$, missä oikealla puolella on $\pm f^{(k)}(\xi)$:n sijasta
$\abs{f^{(k)}(x)}$:n maksimiarvo välillä $[a,b]$ ja $h$ on peräkkäisten kvadratuuripisteiden
\pain{suurin} väli. Yleisesti jos osavälijakoon perustuvan yhdistetyn numeerisen kvadratuurin
virhe on suuruusluokkaa $\ordoO{h^r}$ (integroitavan funktion $f$ ollessa riittävän
säännöllinen), mutta ei luokkaa $\ordoO{h^{r+1}}$ (vaikka $f$ olisi kuinka säännöllinen), niin
sanotaan, että ko. menetelmän (tarkkuuden)
\index{kertaluku!b@tarkkuuden}%
\kor{kertaluku} (engl.\ order of accuracy) on $r$.
Kuten em.\ virhetarkastelusta voi päätellä, on yleisesti $r=m+1$, jos numeerinen kvadratuuri
integroi osaväleillä tarkasti polynomin astetta $m$ mutta ei polynomia astetta $m+1$.
Yhdistetty keskipistesääntö ja trapetsi ovat siis toisen kertaluvun menetelmiä, ja yhdistetyn
Simpsonin kertaluku on $r=4$. Esimerkkejä ensimmäisen kertaluvun menetelmistä ovat Riemannin
summakaavat, joissa kvadratuuripisteet (eli välipisteet $\xi_k$, vrt.\ Luku
\ref{määrätty integraali}) eivät ole osavälien keskipisteitä
(ks.\ Harj.teht\,\ref{H-int-9: Riemannin summa}). Riemannin summiin perustuvista
approksimaatioista yhdistetty keskipistesääntö on siis tarkkuutensa puolesta omassa luokassaan.

\subsection{Sovellusesimerkki: Adaptiivinen Simpson}
\index{adaptiivinen Simpson|vahv}
\index{zza@\sov!Adaptiivinen Simpson|vahv}

Laskimissa ja numeeris--symbolisissa tietokoneohjelmistoissa numeerisen integroinnin komentojen
(esim. Mathematica: \verb|NIntegrate|) takana on usein yhdistetty Simpson, mahdollisesti myös
korkeamman kertaluvun yhdistettyjä kvadratuureja. Jakoa osaväleihin ei näissä ohjelmissa
yleensä suoriteta tasavälisesti, vaan integroitavaan funktioon sopeutuen \kor{adaptiivisesti}. 
Adaptiivisen Simpsonin menetelmän ideana on selvittää 'nuuskimalla', kuinka suuri on 
integroitavan funktion neljäs derivaatta $f^{(4)}(x)$ tarkasteltavalla osavälillä 
$[x_{k-1},x_k]$. Perusajatus on yksinkertainen: Jos osaväli on lyhyt (niinkuin laskun kuluessa
ennen pitkää on), niin voidaan olettaa, että $f^{(4)}(x)$ on ko.\ välillä likimain vakio $=M_k$.
Tällöin vakio $M_k$ saadaan selville tulkitsemalla itse algoritmin antamia tuloksia 
\index{a posteriori}%
\kor{a posteriori} (eli laskemisen jälkeen) seuraavasti: Sovelletaan ensin Simpsonin sääntöä
välillä $[x_{k-1},x_k]$ -- tulos $I_1$. Jaetaan sitten väli puoliksi ja sovelletaan Simpsonin
sääntöä kummallakin osavälillä erikseen -- tulos $I_2$. Jos nyt integraalin tarkka arvo $=I$ 
välillä $[x_{k-1},x_k]$, niin Simpsonin virhekaavan mukaan
\[
\begin{cases} 
 I-I_1=-\frac{1}{90}M_kh^5, \\ 
 I-I_2=-\frac{1}{90}M_k\cdot 2\cdot\left(\frac{h}{2}\right)^5
\end{cases}
\impl\ M_k=96h^{-5}(I_1-I_2), \quad h=\tfrac{1}{2}(x_k-x_{k-1}).
\]
Todellisuudessa $f^{(4)}$ ei ole aivan vakio edes lyhyellä välillä (jos olisi, saataisiin myös
integraalin tarkka arvo $I$ selville!), mutta saatu arvio on yleensä adaptiivisiin 
tarkoituksiin riittävä: Sen avulla voidaan ohjata algoritmia tihentämään jakoa siellä, missä 
laskun antama luku $M_k$ on itseisarvoltaan suuri. Algoritmi pyrkii tarkemmin tihentämään jakoa
niin, että \pain{virhetihe}y\pain{s}, eli integroimisvirhe osavälillä jaettuna osavälin 
pituudella, on suunnilleen sama jokaisella osavälillä. (Tällainen jako on osoitettavissa 
laskutyön kannalta optimaaliseksi.) 

Adaptiivinen algoritmi toimii käytännössä hämmästyttävän hyvin myös useimmissa sellaisissa
tilanteissa, joissa funktio ei ole lainkaan niin säännöllinen kuin em.\ laskussa oletetaan
(eli neljä kertaa jatkuvasti derivoituva). Näin käy vaikkapa Esimerkin 
\ref{Trapetsi vastaan Simpson 2} integroimistehtävässä, jossa integroitava funktio ei ole
edes jatkuvasti derivoituva ($f(x) \approx 1-\sqrt{x}$ pisteen $x=0$ lähellä). Laskemalla
lukuja $M_k$ ym.\ tavalla algoritmi päätyy tihentämään jakoa voimakkaasti origon lähellä
ja pystyy tämän 'koneälyn' ansiosta laskemaan integraalin arvon vaaditulla tarkkuudella lähes
yhtä nopeasti kuin Esimerkin \ref{Trapetsi vastaan Simpson 1} tilanteessa 
(ks.\ Harj.teht.\,\ref{H-int-9: sqrt-integraali}c).

\Harj
\begin{enumerate}

\item
Integraali $\int_0^1 f(x)\,dx$ lasketaan tasavälisellä, yhdistetyllä Trapetsilla jakamalla
integroimisväli $n$ osaväliin. Laske näin saadun likiarvon virhe tarkasti, kun \
a) $f(x)=x(1-x)$, \ b) $f(x)=e^x$.

\item
Laske seuraaville integraalille likiarvot käyttämällä $11$ pisteen tasavälistä, yhdistettyä
Trapetsia ($10$ osaväliä) ja Simpsonin sääntöä ($5$ osaväliä) ja vertaa tarkkaan arvoon.
\[
\text{a)}\,\ \int_0^1 \frac{1}{1+x^2}\,dx \qquad
\text{b)}\,\ \int_0^1 \frac{4x^3}{1+x}\,dx \qquad
\text{c)}\,\ \int_0^1 \frac{1}{1+\sqrt[4]{x}}\,dx
\]

\item \label{H-int-9: korjattu trapetsi} \index{korjattu trapetsikaava}
Numeerisen integroinnin nk.\ \kor{korjattu trapetsikaava} on
\[
\int_a^b f(x)\,dx \approx \frac{1}{2}(b-a)[f(a)+f(b)]-\frac{1}{12}(b-a)^2[f'(b)-f'(a)].
\]
Näytä, että kaava on tarkka polynomeille astetta $m=3$. Millainen on vastaava yhdistetty
kvadratuuri, jos jako osaväleihin on tasavälinen?

\item \label{H-int-9: Trapetsin virhekaava}
Olkoon $f$ kahdesti jatkuvasti derivoituva välillä $[a,b]$ ja $g=f-p$, missä $p$ on ensimmäisen
asteen (Lagrangen) interpolaatiopolynomi, jolle pätee $p(a)=f(a)$ ja $p(b)=f(b)$. Näytä
osittain integroimalla, että pätee
\[
-\frac{1}{2}\int_a^b (x-a)(b-x)g''(x)\,dx = \int_a^b g(x)\,dx.
\]
Johda Trapetsin virhekaava tästä tuloksesta.

\item \label{H-int-9: Simpson 1}
a) Olkoon $p$ funktion $f$ toisen asteen interpolaatiopolynomi, joka määritellään ehdoilla
$p(a-h)=f(a-h)$, $p(a)=f(a)$ ja $p(a+h)=f(a+h)$ välillä $[a-h,a+h]$. Näytä, että Simpsonin
sääntö ko.\ välillä on sama kuin approksimaatio
\[
\int_{a-h}^{a+h} f(x)\,dx \approx \int_{a-h}^{a+h} p(x)\,dx.
\]
b) Todista Simpsonin säännön virhekaava siinä tapauksessa, että $f^{(4)}$ on vakio 
välillä $[a-h,a+h]$ (eli $f$ on polynomi astetta $4$).

\item \label{H-int-9: Riemannin summa}
Integraalin approksimoiminen Riemannin summalla, jossa välipisteiksi osaväleillä $[x_{k-1},x]$
valitaan $\xi_k=x_{k-1}$, vastaa osaväleillä tehtyä approksimaatiota
$\int_a^b f(x)\,dx \approx (b-a)f(a)$. Johda tälle virhekaava
\[
E(f)=\frac{1}{2}(b-a)^2 f'(\xi), \quad \xi\in[a,b].
\]
Mikä on virhekaava koko välillä, jos jako osaväleihin on tasavälinen?

\item(*) \label{H-int-9: Arkhimedes} \index{zzb@\nim!Arkhimedeen algoritmi}
(Arkhimedeen algoritmi) Suora leikkaa paraabelin $K:\ y=x^2$ pisteissä $(a,a^2)$ ja 
$(b,b^2)$, jolloin suora erottaa paraabelista segmentin $A$ välillä $[a,b]$ ($a<b$). Olkoon 
$A_n$ segmentin pinta-alan $\mu(A)$ likiarvo, joka saadaan jakamalla väli $[a,b]$ tasavälisesti
$2^n$ osaväliin ($n\in\N$) ja käyttämällä yhdistettyä trapetsia. Näytä pelkin algebran keinoin,
että jono $\seq{A_n}$ on geometrinen sarja. Laske $\mu(A)$ ($=\lim_n A_n$) tällä perusteella.

\item (*) \label{H-int-9: Simpsonin 2}
a) (Vrt.\ Tehtävä \ref{H-int-9: Simpson 1}a.) Olkoon $p$ funktion $f$ yleistetty kolmannen
asteen interpolaatiopolynomi, joka määritellään ehdoilla $\,p(a-h)=f(a-h)$, $\,p(a+h)=f(a+h)$,
$\,p(a)=f(a)\,$ ja $\,p'(a)=f'(a)$ välillä $[a-h,a+h]$. Näytä, että Simpsonin sääntö ko.\
välillä on sama kuin approksimaatio
\[
\int_{a-h}^{a+h} f(x)\,dx \approx \int_{a-h}^{a+h} p(x)\,dx.
\]
b) Johda Simpsonin virhekaava. \kor{Vihje}: Lause \ref{usean pisteen Taylor}.

\item (*) \label{H-int-9: korjatun trapetsin virhe}
Olkoon $f$ neljä kertaa jatkuvasti derivoituva välillä $[a,b]$. Johda korjatulle
trapetsikaavalle (Harj.teht.\,\ref{H-int-9: korjattu trapetsi}) virhekaava
\[
E(f) = \frac{1}{720}\,(b-a)^5 f^{(4)}(\xi), \quad \xi\in[a,b].
\]
\kor{Vihje}: Hermiten interpolointi ja Lause \ref{usean pisteen Taylor}!

\item (*) \label{H-int-9: sqrt-integraali}
Tarkastellaan numeerisen integroinnin virhettä laskettaessa integraali $\int_0^1\sqrt{x}\,dx\,$
yhdistetyillä kvadratuureilla, joissa väli $[0,1]$ jaetaan eri tavoilla $n$ 
osaväliin. \vspace{1mm}\newline
\ a) Näytä, että jos jako on tasavälinen, niin sekä yhdistetyn trapetsin että yhdistetyn
Simpsonin virhe on \ $\sim n^{-3/2}$ suurilla $n$:n arvoilla. \newline 
b) Valitaan välin $[0,1]$ jakopisteiksi $x_k=(k/n)^{4/3},\ k=0 \ldots n$ ja käytetään
yhdistettyä trapetsia. Näytä, että virhe $\,=\mathcal{O}(n^{-2}\ln n)$. \newline
c) Valitaan välin $[0,1]$ jakopisteiksi $x_k=(k/n)^{8/3},\ k=0 \ldots n$ ja \mbox{käytetään}
yhdistettyä Simpsonia. Näytä, että virhe $\,=\mathcal{O}(n^{-4}\ln n)$. (Adaptiivinen Simpsonin
algoritmi päätyy likimain tähän jakoon suurilla $n$:n arvoilla.)
 
\item (*) \label{H-int-9: hidas sarja}
(Hitaan sarjan kiihdytys) Suppeneva integraali $\int_n^\infty f(x)\,dx$, missä $n\in\N$ on suuri
luku, voidaan laskea likimäärin käyttämällä joko a) yhdistettyä trapetsia tai b) yhdistettyä,
korjattua trapetsia (Harj.teht.\,\ref{H-int-9: korjattu trapetsi}) välin $[n,\infty)$ jaossa
osaväleihin $\,[k,k+1]$, $k\in\N$, $k \ge n$. Lähtien näistä ajatuksista ja mainittujen
kvadratuurien virhearvioista 
(ks.\ Harj.teht.\,\ref{H-int-9: korjatun trapetsin virhe}) johda seuraavat tulokset, kun
$f(x)=1/x^\alpha,\ \alpha>1$\,:
\begin{align*}
&\text{a)}\,\ \sum_{k=1}^\infty \frac{1}{k^\alpha} \,=\,
 \sum_{k=1}^n \frac{1}{k^\alpha}+\frac{1}{\alpha-1}\,n^{1-\alpha}-\frac{1}{2}\,n^{-\alpha}
 +\Ord{n^{-\alpha-1}}\,. \\
&\text{b)}\,\ \sum_{k=1}^\infty \frac{1}{k^\alpha} \,=\,
 \sum_{k=1}^n \frac{1}{k^\alpha}+\frac{1}{\alpha-1}\,n^{1-\alpha}-\frac{1}{2}\,n^{-\alpha}
 +\frac{\alpha}{12}\,n^{-\alpha-1}+\Ord{n^{-\alpha-3}}\,.
\end{align*}
\kor{Vihje}: Huomioi, että
\begin{align*}
\text{a)}\,\ \sum_{k=n+1}^\infty f(k)
&\,=\, \sum_{k=n}^\infty \frac{1}{2}\,\bigl[\,f(k)+f(k+1)\,\bigr] - \frac{1}{2}\,f(n), \\
\text{b)}\,\ \sum_{k=n+1}^\infty f(k)
&\,=\, \sum_{k=n}^\infty \left(\frac{1}{2}\,\bigl[\,f(k)+f(k+1)\,\bigr]
                             -\frac{1}{12}\,\bigl[\,f'(k+1)-f'(k)\,\bigr]\right) \\
&\quad -\frac{1}{2}\,f(n) - \frac{1}{12}\,f'(n).
\end{align*}

\end{enumerate}
 
