\section{Matriisin ominaisarvot} \label{matriisin ominaisarvot}
\alku

Aloitetaan määritelmästä.
\begin{Def} \index{matriisin ($\nel$neliömatriisin)!i@$\nel$ominaisarvo, -vektori|emph}
Jos $\mA$ on neliömatriisi kokoa $n\times n$, niin kompleksiluku $\lambda\in\C$ on $\mA$:n 
\kor{ominaisarvo} (engl.\ eigenvalue, ruots.\ egenvärde, saks.\ Eigenwert), ja vektori
$\mx\in\C^n$, $\mx\neq\mo$, on ominaisarvoon $\lambda$ liittyvä \kor{ominaisvektori}
(engl.\ eigenvector), jos pätee
\[
\mA\mx=\lambda\mx.
\]
\end{Def}
Määritelmässä voi $\mA$ olla yhtä hyvin reaalinen kuin kompleksinen matriisi. Määritelmä siis 
\pain{ei} yksinkertaistu muotoon $\lambda\in\R$, $\mx\in\R^n$ siinäkään tapauksessa, että $\mA$
on reaalinen.

Määritelmästä nähdään heti, että ominaisvektorit eivät ole yksikäsitteiset: Jos $\mx$ on 
ominaisvektori, niin samoin on $\alpha\mx$ jokaisella $\alpha\in\C$, $\alpha\neq 0$. 
\index{normeeraus (vektorin)}%
Ominaisvektorit voidaan siis \kor{normeerata} mielivaltaisella tavalla, esimerkiksi ehdolla 
$\abs{\mx}=1$.

\index{ominaisarvo, -vektori, -pari}%
Matriisin ominaisarvot ja -vektorit, eli \kor{ominaisparit} (engl. eigenpair) määrittelevää 
yhtälöä $\mA\mx=\lambda\mx$ sanotaan \kor{ominaisarvoyhtälöksi}. Ominaisparien laskeminen 
aloitetaan määrämällä tästä yhtälöstä ensin kaikki ominaisarvot. Ominaisarvojen joukkoa, jota
\index{spektri (matriisin)} \index{matriisin ($\nel$neliömatriisin)!i@$\nel$spektri}%
sanotaan $\mA$:n \kor{spektriksi}, merkitään jatkossa symbolilla $\sigma(\mA)\,$:
\[
\sigma(\mA)=\{\mA\text{:n ominaisarvot}\} \subset\C.
\]
Spektrin avulla saadaan matriisin säännöllisyydelle jälleen uusi luonnehdinta:
\[
\boxed{\kehys\quad \mA\ \text{säännöllinen} \qekv 0\not\in\sigma(\mA). \quad}
\]

Kun ominaisarvoyhtälö kirjoitetaan muotoon
\[
(\mA-\lambda\mI)\mx=\mo
\]
($\mI=\text{yksikkömatriisi}$), niin matriisialgebran ja determinanttiopin mukaan tällä on 
ei-triviaali ratkaisu, eli $\lambda\in\sigma(\mA)$, täsmällen kun
\[
\det(\mA-\lambda\mI)=0.
\]
Tämä on siis \kor{ominaisarvoehto}. Koska $\lambda$ esiintyy tässä ainoastaan 
$(\mA-\lambda\mI)$:n diagonaalilla, niin alideterminanttisäännön 
(Lause \ref{alideterminanttisääntö}) perusteella nähdään, että ominaisarvoehto purkautuu
yhtälöksi muotoa
\[
p(\lambda)=0,
\]
missä $p(\lambda)$ on polynomi astetta $n$, tarkemmin
\[
p(\lambda)=(-1)^n\lambda^n+\{\text{alempiasteisia termejä}\}.
\]
\index{karakteristinen yhtälö (polynomi)!b@neliömatriisin}
\index{matriisin ($\nel$neliömatriisin)!j@$\nel$karakteristinen polynomi}%
Ominaisarvoehtoa $p(\lambda)=0$ sanotaan $\mA$:n \kor{karakteristiseksi yhtälöksi} ja
polynomia $p(\lambda)=\det(\mA-\lambda\mI)$ $\mA$:n \kor{karakteristiseksi polynomiksi}.
Matriisin $\mA$ ominaisarvot ovat siis $\mA$:n karakteristisen polynomin juuret, ja näin ollen
$\mA$:n spektri $\sigma(\mA)$ koostuu vähintään yhdestä ja enintään $n$:stä kompleksiluvusta 
(Algebran peruslause). Jos $\mA$ on reaalinen matriisi, niin karakteristinen polynomi
$p(\lambda)$ on reaalikertoiminen, jolloin juuret ovat joko reaalisia tai muodostavat 
konjugaattipareja. Jos $\lambda\in\sigma(\mA)$ on $p(\lambda)$:n $m$-kertainen juuri, niin
lukua $m_a(\lambda)=m$ sanotaan ko. ominaisarvon
\index{kertaluku!e@ominaisarvon} \index{algebrallinen kertaluku}%
\kor{algebralliseksi kertaluvuksi}. Algebran peruslauseen mukaan on
\[
\sum_{\lambda\in\sigma(\mA)} m_a(\lambda)=n.
\]
Jos $m_a(\lambda)=1$, on kyseessä
\index{yksinkertainen!b@ominaisarvo}%
\kor{yksinkertainen} (engl. simple) ominaisarvo. 

Kun ominaisarvo $\lambda\in\sigma(\mA)$ on löydetty (mahdollisesti numeerisin apukeinoin),
saadaan vastaavat ominaisvektorit selville ominaisarvoyhtälöstä $(\mA-\lambda\mI)\mx=\mo$, joka
ominaisarvon määritelmän mukaisesti on singulaarinen yhtälöryhmä $\C^n$:ssä. Kun tämä 
yhtälöryhmä saatetaan Gaussin algoritmilla singulaariseen perusmuotoonsa, saadaan kaikki 
ratkaisut selville. Yleisesti jos $\mx_1 \in \C^n$ ja $\mx_2 \in \C^n$ ovat ominaisarvoyhtälön
ratkaisuja, niin samoin on $\alpha_1\mx_1+\alpha_2\mx_2\ \forall \alpha_1,\alpha_2 \in \C$, 
joten tiettyyn ominaisarvoon liittyvät ominaisvektorit muodostavat $\C^n$:n aliavaruuden. Tätä
aliavaruutta sanotaan ko.\ ominaisarvoon liittyväksi
\index{ominaisavaruus}%
\kor{ominaisavaruudeksi} (engl.\ eigenspace).
Jos ominaisavaruuden dimensio on $\nu$, eli jos samaan ominaisarvoon
$\lambda$ voidaan liittää täsmälleen $\nu$ lineaarisesti riippumatonta ominaisvektoria, niin
sanotaan, että luku $\nu$ (aina $\nu\geq 1$) on ko.\ ominaisarvon
\index{geometrinen kertaluku}%
\kor{geometrinen kertaluku}, ja merkitään
\[
m_g(\lambda)=\nu.
\]
Jokaiseen ominaisarvoon $\lambda\in\sigma(\mA)$ liittyy siis kaksi kertalukua, algebrallinen
kertaluku $m_a(\lambda)$ ja geometrinen kertaluku $m_g(\lambda)$, jotka molemmat ovat
luonnollisia lukuja. Voidaan myös osoittaa, että aina pätee
\[ 
m_g(\lambda)\leq m_a(\lambda), \quad \lambda\in\sigma(\mA).
\]
\begin{Exa} \label{eig-ex1}
Määritä matriisin
\[
\mA=\begin{rmatrix} 2 & 0 & 4 \\ 0 & 6 & 0 \\ 4 & 0 & 2 \end{rmatrix}
\]
ominaisarvot ja ominaisvektorit.
\end{Exa}
\ratk Ominaisarvojen determinanttiehto on
\begin{align*}
\det(\mA-\lambda\mI) &= \begin{vmatrix} 
                        2-\lambda & 0 & 4 \\ 0 & 6-\lambda & 0 \\ 4 & 0 & 2-\lambda 
                        \end{vmatrix} \\ 
                     &= (6-\lambda)(\lambda^2-4\lambda-12) = -(6-\lambda)^2(\lambda+2) = 0
\end{align*}
(determinantti purettu toisen vaaka- tai pystyrivin mukaan), joten ominaisarvot ja niiden 
algebralliset kertaluvut ovat
\[ 
\lambda_1=6, \quad m_a(\lambda_1)=2, \qquad \lambda_2=-2, \quad m_a(\lambda_2)= 1. 
\]
Ominaisvektorit saadaan ratkaisemalla ominaisarvoyhtälö kummallekin ominaisarvolle erikseen:
\begin{align*}
(\mA-\lambda_1\mI)\mx=\mo \ &\ekv \ \begin{rmatrix} 
                                    -4 & 0 & 4 \\ 0 & 0 & 0 \\ 4 & 0 & -4 
                                    \end{rmatrix}
\begin{bmatrix} x_1 \\ x_2 \\ x_3 \end{bmatrix} = \mo \\ \\
&\ekv \ \begin{rmatrix} -4 & 0 & 4 \\ 0 & 0 & 0 \\ 0 & 0 & 0 \end{rmatrix}
\begin{bmatrix} x_1 \\ x_2 \\ x_3 \end{bmatrix} = \mo. \\ \\
(\mA-\lambda_2\mI)\mx=\mo \ &\ekv \ \begin{rmatrix} 
                                    4 & 0 & 4 \\ 0 & 8 & 0 \\ 4 & 0 & 4 
                                    \end{rmatrix}
\begin{bmatrix} x_1 \\ x_2 \\ x_3 \end{bmatrix} = \mo \\ \\
&\ekv \ \begin{rmatrix} 4 & 0 & 4 \\ 0 & 8 & 0 \\ 0 & 0 & 0 \end{rmatrix}
\begin{bmatrix} x_1 \\ x_2 \\ x_3 \end{bmatrix} = \mo.
\end{align*}
Nähdään, että  ominaisarvoon $\lambda_1$ liittyy kaksi lineaarisesti riippumatonta 
ominaisvektoria, esimerkiksi
\[
\mx_1=\begin{rmatrix} 1 \\ 0 \\ 1 \end{rmatrix},\quad 
\mx_2=\begin{rmatrix} 0 \\ 1 \\ 0 \end{rmatrix},
\]
ja ominaisarvoon $\lambda_2$ liittyy yksi lineaarisesti riippumaton ominaisvektori, esimerkiksi
\[
\mx_3=\begin{rmatrix} -1 \\ 0 \\ 1 \end{rmatrix}.
\]
Ominaisarvojen geometriset kertaluvut ovat näin ollen
\[
m_g(\lambda_1)=2,\quad m_g(\lambda_2)=1. \loppu
\]
\begin{Exa} \label{eig-ex2}
Laske matriisin
\[
\mA=\begin{rmatrix} 1 & 0 & -2 \\ 0 & 1 & -2 \\ 0 & 1 & -1 \end{rmatrix}
\]
ominaisarvot ja ominaisvektorit.
\end{Exa}
\ratk Karakteristinen polynomi on
\[
p(\lambda)=(1-\lambda)(\lambda^2+1),
\]
joten ominaisarvot ovat
\[
\lambda_1=1,\quad \lambda_2=i,\quad \lambda_3=-i.
\]
Samaan tapaan laskien kuin edellisessä esimerkeissä saadaan ominaisvektoreiksi
\[
\mx_1=\begin{bmatrix} 1 \\ 0 \\ 0 \end{bmatrix},\quad
\mx_2=\begin{bmatrix} 1+i \\ 1+i \\ 1 \end{bmatrix},\quad
\mx_3=\begin{bmatrix} 1-i \\ 1-i \\ 1 \end{bmatrix}.
\]
Ominaisarvojen kertaluvut ovat
\[
m_a(\lambda_i)=m_g(\lambda_i)=1,\quad i=1,2,3. \loppu
\]

\begin{Exa} \label{eig-ex3} Laske diagonaalimatriisin $\mA=\text{diag}\, (d_i, \ i=1\ldots n)$
ominaisarvot ja ominaisvektorit. 
\end{Exa}
\ratk Karakteristinen yhtälö on
\[
\det(\mA-\lambda\mI)=\prod_{i=1}^n (d_i-\lambda)=0.
\]
Jos $d_i\neq d_j$, kun $i\neq j$, niin ominaisarvot ja niiden algebrallisest kertaluvut ovat
\[
\lambda_i=d_i,\quad m_a(\lambda_i)=1,\quad i=1\ldots n.
\]
Jos $d_i=d \ \forall i$ (jolloin $\mA=d\mI$), niin $\mA$:lla on yksi $n$-kertainen ominaisarvo:
\[
\lambda=d,\quad m_a(\lambda)=n.
\]
Jos yleisemmin luku $d$ esiintyy $\mA$:n diagonaalilla täsmälleen $m$ kertaa, niin ominaisarvon
$\lambda=d$ algebrallinen kertaluku on $m_a(\lambda)=m$. Myös geometrinen kertaluku on 
$m_g(\lambda)=m$, sillä ominaisarvoyhtälöstä nähdään, että ominaisvektoreita ovat $\R^n$:n 
peruskantavektorit $\me_i$ jokaisella $i$, jolla $d_i=d$. Erityisesti jos $d_i=d \ \forall i$,
niin jokainen vektori $\mx\in\R^n$, $\mx\neq \mo$, on ominaisvektori. \loppu

Tähänastisissa esimerkeissä ovat omainaisarvojen algebralliset ja geometriset kertaluvut olleet
samat. Seuraavasta esimerkistä nähdään, että näin ei ole aina.
\begin{Exa} \label{eig-ex4} Määrää $\mA$:n ominaisarvot ja ominaisvektorit, kun
\[
\mA=\begin{bmatrix} 
    a & 1 \\ & a & 1 \\ & & a & \ddots \\ & & & \ddots & 1 \\ & & & & a 
    \end{bmatrix}
\]
eli
\[
A=(a_{ij})_{i,j=1}^n,\quad a_{ij}
 = \begin{cases} 
   \,a, &\text{kun } i=j, \\ \,1, &\text{kun } j=i+1, \\ \,0, &\text{muulloin}.
   \end{cases}
\]
\end{Exa}
\ratk Kyseessä on kolmiomatriisi, jonka karakteristinen yhtälö on
\[
\det(\mA-\lambda\mI)=(a-\lambda)^n=0.
\]
Tämän mukaan $\lambda=a$ on $n$-kertainen ominaisarvo ($m_a(\lambda)=n$). Kun $\lambda=a$, niin
ominaisarvoyhtälöt ovat
\[
(\mA-\lambda\mI)\mx=\mo \ \ekv \ \left\{ \begin{aligned}
x_2 &= 0 \\
x_3 &= 0 \\
& \ \, \vdots \\
x_n &= 0
\end{aligned} \right.
\]
joten $\mA$:lla on vain yksi lineaarisesti riippumaton ominaisvektori $\mx=[1,0,\ldots,0]^T$. 
Siis ainoan ominaisarvon geometrinen kertaluku on $m_g(\lambda)=1$. \loppu

Todetaan tässä yhteydessä vielä matriisin ominaisvektoreiden lineaarista riippuvuutta
rajoittava yleinen tulos.
\begin{Lause} \label{ominaisvektorien lineaarinen riippumattomuus}
\index{lineaarinen riippumattomuus|emph} Matriisin eri ominaisarvoihin liittyvät ominaisvektorit
ovat keskenään lineaarisesti riippumattomat.
\end{Lause}
\tod Olkooon $\mA\ma_i=\lambda_i\ma_i,\ i=1 \ldots m$, missä $\ma_i\neq\mo$ ja 
$\lambda_i\neq\lambda_j$, kun $i \neq j$. Kun merkitään
\[
\mB=(\mA-\lambda_1\mI)(\mA-\lambda_2\mI) \cdots (\mA-\lambda_{m-1}\mI) 
   = \prod_{k=1}^{m-1}(\mA-\lambda_k\mI),
\]
niin oletusten perusteella
\[
\mB\ma_j = \prod_{k=1}^{m-1}(\lambda_j-\lambda_k)\,\ma_j 
         = \begin{cases} 
            \,\mo, \quad &j=1 \ldots m-1, \\ 
            \,\prod_{k=1}^{m-1}(\lambda_m-\lambda_k)\,\ma_m, \quad &j=m. 
           \end{cases} 
\]
Oletetaan nyt, että
\[
\sum_{j=1}^m x_j\ma_j = \mo.
\]
Tällöin seuraa
\[
\mo=\mB\sum_{j=1}^m x_j\ma_j\,=\,\sum_{j=1}^m x_j\mB\ma_j\,
                              =\,\prod_{k=1}^{m-1}(\lambda_m-\lambda_k)x_m\ma_m\,,
\]
joten on oltava $x_m=0$. Siis
\[
\sum_{j=1}^{m-1} x_j\ma_j = \mo.
\]
Jos $m=2$, niin tämä yhtälö toteutuu vain kun $x_1=0$, jolloin on päätelty, että $x_1=x_2=0$.
Jos $m>2$, niin kertomalla yhtälö puolittain matriisilla
\[
\mB=\prod_{k=1}^{m-2}(\mA-\lambda_k\mI)
\]
päätellään samalla tavoin kuin edellä, että $x_{m-1}=0$, jne. Siis vektorit 
$\ma_1, \ldots, \ma_m$ ovat lineaarisesti riippumattomat:
\[
\sum_{j=1}^m x_j\ma_j = \mo \qimpl x_m = x_{m-1} = \ldots = x_1 = 0. \loppu
\]

Lauseella \ref{ominaisvektorien lineaarinen riippumattomuus} on seuraava ilmeinen seuraamus:
\begin{Kor} Jos $\mA$ on kokoa $n\times n$ ja kaikki $\mA$:n ominaisarvot ovat yksinkertaisia,
niin $\mA$:n ominaisvektoreista voidaan muodostaa $\C^n$:n kanta.
\end{Kor}

\subsection{Symmetrinen matriisi}
\index{neliömatriisi!a@symmetrinen|vahv}

Reaalisen ja symmetrisen matriisin ominaisarvoteoria voidaan pelkistää seuraavaksi kauniiksi
lauseeksi. Lauseen täydellinen todistus sopii paremmin lineaarialgebran laajemman esittelyn
yhteyteen, joten rajoitutaan tässä todistamaan vain kaksi lauseeseen sisältyvää (helpompaa)
osaväittämää.
\begin{Lause} \label{symmetrisen matriisin ominaisarvoteoria} Olkoon $\mA$ reaalinen ja 
symmetrinen matriisi kokoa $n\times n$. Tällöin $\mA$:n ominaisarvot ovat reaaliset,
ominaisarvojen algebralliset ja geometriset kertaluvut ovat samat, ja $\mA$:n ominaisvektoreista
voidaan muodostaa $\R^n$:n ortonormeerattu kanta.
\end{Lause}
\tod (osittainen) \ \, Jos
\[
\mA\mx=\lambda\mx,\quad \lambda\in\C,\ \mx\in\C^n,\ \mx\neq\mo,
\]
niin
\[
\scp{\mA\mx}{\mx}=\lambda\scp{\mx}{\mx}=\lambda\abs{\mx}^2,
\]
missä $\scp{\cdot}{\cdot}$ on $\C^n$:n euklidinen skalaaritulo ja $\abs{\cdot}$ on euklidinen
normi:
\[
\scp{\mx}{\my}=\sum_{i=1}^n x_i\overline{y}_i,\quad \abs{\mx}=\sqrt{\scp{\mx}{\mx}}.
\]
Konjugoimalla saatu yhtälö puolittain saadaan
\[
\overline{\scp{\mA\mx}{\mx}}=\overline{\lambda\abs{\mx}^2}=\overline{\lambda}\abs{\mx}^2.
\]
Mutta $\mA$:n oletetun reaalisuuden ja symmetrian perusteella on
(vrt.\ Luku \ref{matriisialgebra})
\[
\overline{\scp{\mA\mx}{\mx}}=\scp{\mx}{\mA\mx}=\scp{\mA\mx}{\mx},
\]
joten $\lambda\abs{\mx}^2=\overline{\lambda}\abs{\mx}^2$. Tässä on $\abs{\mx}^2\in\R_+$, joten
$\lambda=\overline{\lambda}$, eli $\lambda\in\R$.

Olkoot $\lambda$ ja $\mu$ $\mA$:n kaksi eri suurta ominaisarvoa ja $\mx$ ja $\my$ vastaavat
ominaisvektorit:
\[
\mA\mx=\lambda\mx\,\ \ja\,\ \mA\my=\mu\my.
\]
Jo todetun perusteella on $\lambda,\mu\in\R$, joten seuraa
\begin{align*}
\scp{\mA\mx}{\my} &= \scp{\lambda\mx}{\my}=\lambda\scp{\mx}{\my}, \\
\scp{\mx}{\mA\my} &= \scp{\mx}{\mu\my}=\mu\scp{\mx}{\my}.
\end{align*}
Mutta koska $\mA$ on reaalinen ja symmetrinen, niin
\[
\scp{\mA\mx}{\my}=\scp{\mx}{\mA\my},
\]
joten seuraa
\[
(\lambda-\mu)\scp{\mx}{\my}=0 \ \impl \ \scp{\mx}{\my}=0.
\]
Yhdistämällä tähänastiset päätelmät on todistettu, että $\mA$:n ominaisarvot ovat reaaliset, ja
että eri ominaisarvoihin liittyvät ominaisvektorit ovat ortogonaaliset. Todistuksen loppuosa
sivuutetaan. \loppu

Lauseen \ref{symmetrisen matriisin ominaisarvoteoria} tuloksesta seuraa, että reaalisen ja 
symmetrisen matriisin tapauksessa ei kompleksilukuja tarvita myöskään ominaisvektoreissa. 
Nimittäin jos $\mx\in\C^n$ on ominaisvektori ja $\mx=\mpu+i\mpv$, $\mpu,\mpv\in\R^n$, niin 
$\mA$:n, $\lambda$:n, $\mpu$:n ja $\mpv$:n reaalisuuden perusteella
\begin{align*}
\mA\mx=\lambda\mx \ &\ekv \ \mA\mpu+i\mA\mpv=\lambda\mpu+i\lambda\mpv \\
&\ekv \ \mA\mpu=\lambda\mpu \ \ja \ \mA\mpv=\lambda\mpv.
\end{align*}
Reaalisen ja symmetrisen matriisin ominaisarvoyhtälö $\mA\mx=\lambda\mx$ voidaan siis ratkaista
kokonaan $\R^n$:ssä.
\begin{Exa} \label{eig-ex5} Symmetrisen matriisin
\[
\mA=\begin{rmatrix} 2 & 0 & 4 \\ 0 & 6 & 0 \\ 4 & 0 & 2 \end{rmatrix}
\]
ortonormeeratut ominaisvektorit ovat (vrt.\ Esimerkki \ref{eig-ex1})
\[
\mx_1=\frac{1}{\sqrt{2}}[1,0,1]^T, \quad 
\mx_2=[0,1,0]^T, \quad 
\mx_3=\frac{1}{\sqrt{2}}[-1,0,1]^T. \loppu
\]
\end{Exa}

\subsection{Diagonalisoituva matriisi}
\index{diagonalisointi!a@matriisin|vahv}

Olkoon $\mA$ reaalinen ja symmetrinen matriisi kokoa $n\times n$ ja olkoon $\mC$ reaalinen
matriisi, jonka sarakkeina ovat $\mA$:n ortonormeeratut ominaisvektorit:
\[
\mC=[\mc_1\ldots\mc_n],\quad 
         \mA\mc_i=\lambda_i\mc_i, \quad i=1\ldots n, \quad \scp{\mc_i}{\mc_j}=\delta_{ij}.
\]
Tässä voidaaan ominaisarvoyhtälöt $\mA\mc_i=\lambda_i\mc_i$ koota matriisiyhtälöksi
\[
[\mA\mc_1\ldots\mA\mc_n] \,=\, [\lambda_1\mc_1\ldots\lambda_n\mc_n]
                         \,=\, [\mC(\lambda_1\me_1)\ldots\mC(\lambda_n\me_n)]
\]
eli (vrt.\ Luku \ref{matriisialgebra})
\[
\mA\mC=\mC\boldsymbol{\Lambda},
\]
missä $\boldsymbol{\Lambda}=[\lambda_1\me_1\ldots\lambda_n\me_n]=\text{diag}(\lambda_i)$. Koska
$\mC$ on ortogonaalinen, on $\inv{\mC}=\mC^T$, joten seuraa
\[
\boxed{\kehys\quad \mC^T\mA\mC=\boldsymbol{\Lambda}
                          \qekv \mA=\mC\boldsymbol{\Lambda}\mC^T. \quad}
\]
Tässä tuloksessa siis $\mC$:n sarakkeina ovat $\mA$:n \pain{ortonormeeratut} ominaisvektorit 
jossakin järjestyksessä ja $\boldsymbol{\Lambda}$ on diagonaalimatriisi, jonka lävistäjällä ovat
$\mA$:n ominaisarvot $\mC$:n määräämässä järjestyksessä siten, että $\mA\mc_i=\lambda\mc_i$,
$\lambda_i=[\boldsymbol{\Lambda}]_{ii}$. Annetaan tässä yhteydessä hieman yleisempi määritelmä.
\begin{Def} (\vahv{Similariteettimuunnos}) \label{similaarimuunnos}
\index{similaarisuus|emph} \index{ortogonaalisuus!d@similariteettimuunnoksen|emph}
\index{neliömatriisi!g@diagonalisoituva|emph}
Jos $\mA$ ja $\mB$ ovat 
samankokoisia neliömatriiseja, ja on olemassa säännöllinen matriisi $\mC$ siten, että 
$\mB=\mC^{-1}\mA\mC$, niin sanotaan, että $\mA$ ja $\mB$ ovat \kor{similaariset}, ja kuvausta
$\mA \map \mB$ sanotaan \kor{similariteettimuunnokseksi}. Similariteettimuunnos on 
\kor{ortogonaalinen}, jos $\mC$ on ortogonaalinen matriisi. Jos $\mA$ on similaarinen 
diagonaalimatriisin kanssa, niin sanotaan, että $\mA$ on \kor{diagonalisoituva}.
\end{Def}
Reaalinen ja symmetrinen matriisi on siis määritelmän mukaisesti diagonalisoituva, ja tässä
tapauksessa matriisi $\mC$ on myös valittavissa niin, että diagonalisoiva similariteettimuunnos
on ortogonaalinen. Yleisemmin voidaan $\mC$:n sarakkeiksi valita mitkä tahansa $n$ lineaarisesti
riippumatonta ominaisvektoria. Tällaisen ominaisvektorisysteemin olemassaolo on myös
välttämätön ehto sille, että matriisi on diagonalisoituva:
\begin{Lause} \label{diagonalisoituva matriisi} Matriisi $\mA$ kokoa $n \times n$ on 
diagonalisoituva täsmälleen kun $\mA$:lla on $n$ lineaarisesti riippumatonta ominaisvektoria.
Tällöin jos matriisin $\mC$ sarakkeina ovat mitkä tahansa $\mA$:n lineaarisesti riippumattomat
ominaisvektorit ja diagonaalimatriisin $\boldsymbol{\Lambda}$ lävistäjäalkioina ovat vastaavat
ominaisarvot, niin $\mA=\mC\boldsymbol{\Lambda}\mC^{-1}$.
\end{Lause}
\tod Jos $\mA$:lla on lineaarisesti riippumattomat ominaisvektorit $\mc_i\,,\ i=1 \ldots n$,
niin muodostamalla nämä sarakkeina matriisi $\mC=[\mc_1 \ldots \mc_n]$ todetaan kuten edellä,
että $\mA\mC=\mC\boldsymbol{\Lambda}\ \ekv\ \mA=\mC\boldsymbol{\Lambda}\mC^{-1}$, missä
$\boldsymbol{\Lambda}=\text{diag}\,(\lambda_i)$ ja $\lambda_i=$ ominaisvektoriin $\mc_i$ 
liittyvä ominaisarvo. Tämä todistaa väittämän osan \fbox{$\Leftarrow$}\,. Osan \fbox{$\impl$} 
todistamiseksi olkoon $\mA=\mC^{-1}\mD\mC$, missä $\mC$ on säännöllinen ja 
$\mD=\text{diag}\,(d_i)$ on diagonaalimatriisi. Tällöin pätee
\[
\mD\me_i=d_i\me_i \qekv (\mC^{-1}\mD\mC)(\mC^{-1}\me_i)=d_i\mC^{-1}\me_i, \quad i=1 \ldots n,
\]
joten matriisilla $\mA=\mC^{-1}\mD\mC$ on ominaisvektorit $\mc_i=\mC^{-1}\me_i$ ja vastaavat
ominaisarvot $d_i$. Vektorit $\mc_i$ ovat lineaarisesti riippumattomat, sillä
\[
\sum_{i=1}^n x_i\mc_i=\mo \qekv \mC\sum_{i=1}^n x_i\mc_i=\sum_{i=1}^n x_i\me_i=\mo 
                          \qekv x_i=0,\ i= 1 \ldots n.
\]
Siis $\{d_i,\ i=1 \ldots n\}\subset\sigma(\mA)$ ja $\mA$:lla on $n$ lineaarisesti riippumatonta
ominaisvektoria vastaten näitä ominaisarvoja. Tällöin on myös oltava 
$\{d_i,\ i=1 \ldots n\}=\sigma(\mA)$, sillä muu johtaisi ristiriitaan Lauseen 
\ref{ominaisvektorien lineaarinen riippumattomuus} kanssa. \loppu
\begin{Exa} Matriisilla
\[
\mA=\begin{bmatrix} 1&1\\0&1 \end{bmatrix}
\]
on vain yksi ominaisvektori (Esimerkki \ref{eig-ex4}), joten $\mA$ ei ole diagonalisoituva
(Lause \ref{diagonalisoituva matriisi}). \loppu
\end{Exa}
\begin{Exa}
Määrittele diagonalisoiva muunnos $\mA\map\mD=\inv{\mC}\mA\mC$ matriisille
\[
\mA=\begin{rmatrix} 2 & 0 & 4 \\ 0 & 6 & 0 \\ 4 & 0 & 2 \end{rmatrix}.
\]
\end{Exa}
\ratk saadaan suoraan edellisen luvun Esimerkeistä \ref{eig-ex1} ja \ref{eig-ex5}:
\[
\begin{rmatrix} 6&0&0\\0&6&0\\0&0&-2 \end{rmatrix}=
\begin{rmatrix} 
\frac{1}{\sqrt{2}}&0&\frac{1}{\sqrt{2}}\\0&1&0\\-\frac{1}{\sqrt{2}}&0&\frac{1}{\sqrt{2}} 
\end{rmatrix}
\begin{rmatrix} 2&0&4\\0&6&0\\4&0&2 \end{rmatrix} 
\begin{rmatrix} 
\frac{1}{\sqrt{2}}&0&-\frac{1}{\sqrt{2}}\\0&1&0\\ \frac{1}{\sqrt{2}}&0&\frac{1}{\sqrt{2}} 
\end{rmatrix}.
\]
Tässä on valittu ortogonaalinen muunnos. Mahdollista on myös valita $\mC$:n sarakkeiksi
esim.\ vektorit $\mc_1=[1,1,1]^T$, $\mc_2=[1,0,-1]^T$ ja $\mc_3=[0,1,0]^T$, sillä nämäkin
ovat $\mA$:n lineaarisesti riippumattomia ominaisvektoreita (vastaten ominaisarvoja 
$\lambda_1=6$, $\lambda_2=-2$ ja $\lambda_3=6$, ks.\ edellisen luvun Esimerkki \ref{eig-ex1}).
Kysytyksi muunnokseksi saadaan tällä valinnalla
\[
\begin{rmatrix} 6&0&0\\0&-2&0\\0&0&6 \end{rmatrix}=
\begin{rmatrix} 1&1&0\\1&0&1\\1&-1&1 \end{rmatrix}^{-1}
\begin{rmatrix} 2&0&4\\0&6&0\\4&0&2 \end{rmatrix}
\begin{rmatrix} 1&1&0\\1&0&1\\1&-1&1 \end{rmatrix} \loppu
\]

\Harj
\begin{enumerate}

\item
Todista: \vspace{1mm}\newline
a) Jos $\lambda\in\sigma(\mA)$, niin $\lambda^p\in\sigma(\mA^p)\ \forall p\in\N$. \newline
b) Jos $\mA$ on säännöllinen ja $\lambda\in\sigma(\mA)$, niin 
   $\lambda^{-1}\in\sigma(\mA^{-1})$.
\newline
c) Jos $\mA$ on reaalinen, niin $\sigma(\mA^T)=\sigma(\mA)$. \newline
d) Jos $1\in\sigma(\mA)$ tai $2\in\sigma(\mA)$, niin $\mB=2\mI-3\mA+\mA^2$ on singulaarinen.

\item 
Määritä seuraavien matriisien ominaisarvot ja -vektorit sekä ominaisarvojen algebralliset ja
geometriset kertaluvut.
\begin{align*}
&\text{a)}\ \ \begin{rmatrix} 1&-2\\-2&1 \end{rmatrix} \qquad
 \text{b)}\ \ \begin{rmatrix} 1&-1\\1&1 \end{rmatrix} \qquad
 \text{c)}\ \ \begin{rmatrix} 4&4\\-1&0 \end{rmatrix} \\[2mm]
&\text{d)}\ \ \begin{rmatrix} 3&1&1\\1&3&1\\1&1&3 \end{rmatrix} \qquad
 \text{e)}\ \ \begin{rmatrix} -2&1&1\\0&0&4\\0&-1&-4 \end{rmatrix} \qquad
 \text{f)}\ \ \begin{rmatrix} 1&2&1\\0&2&-2\\1&0&3 \end{rmatrix} \\[2mm]
&\text{g)}\ \ \begin{rmatrix} 2&2&3\\4&1&0\\1&-2&0 \end{rmatrix} \qquad
 \text{h)}\ \ \begin{rmatrix} 1&0&0&1\\0&2&0&0\\9&0&3&0\\1&0&0&4 \end{rmatrix} \qquad
 \text{i)}\ \ \begin{rmatrix} 1&0&0&4\\0&1&3&0\\0&2&1&0\\1&0&0&1 \end{rmatrix}
\end{align*}

\item
Olkoon
\[
\mA=\begin{rmatrix} 1&1&0\\0&2&0\\2&1&-1 \end{rmatrix}, \quad
\mB=\begin{bmatrix} 1&t&t\\t&1&t\\t&t&1 \end{bmatrix}, \quad
\mC=\begin{bmatrix}
    \alpha&1&1&\beta \\ 4&\alpha&\beta&2 \\ 4&\beta&\alpha&2 \\ \beta&3&3&\alpha    
    \end{bmatrix}.
\]
a) Matriisin $\mA$ ominaisvektoreita ovat $[1,1,1]^T$, $[1,0,1]^T$ ja $[0,0,1]^T$. Laske tätä
tietoa hyväksi käyttäen $\mA^{11}\mx$, kun $\mx=[2,1,1]^T$. \vspace{1mm}\newline
b) Matriisin $\mB$ ominaisarvot ovat reaalifunktioita $t\map\lambda_i(t)$. Hahmottele näiden
kuvaajat! \vspace{1mm}\newline
c) Matriisilla $\mC$ on ominaisarvo $\lambda$ ja vastaava ominaisvektori $[0,1,-1,0]^T$.
Millainen ehto tästä seuraa luvuille $\alpha,\beta$ ja $\lambda$\,?

\item
a) Näytä, että jos neliömatriisille pätee $\mA^2=\mA$, niin $\sigma(\mA)\subset\{0,1\}$. Näytä
esimerkillä, että vaihtoehdot $\sigma(\mA)=\{0\}$, $\sigma(\mA)=\{1\}$ ja $\sigma(\mA)=\{0,1\}$
ovat kaikki mahdollisia. \vspace{1mm}\newline
b) Lineaarikuvaus $\mx\map\mA\mx$, missä $\mA$ on kokoa $3 \times 3$, vastaa projektiota
tasolle $T:\,x+3y-2z=0$ suoran $S:\,x=y=z$ sunnassa. Näytä, että $\sigma(\mA)=\{0,1\}$ ja
että $\mA$:lla on kolme lineaarisesti riippumatonta ominaisvektoria. Mitkä ovat ominaisarvojen
kertaluvut? Päteekö myös $\mA^2=\mA$\,?
\vspace{1mm} \newline
c) Olkoon $\mA=\ma\ma^T$, missä $\ma\in\R^n,\ n \ge 2$ ja $\abs{\ma}=1$. Näytä, että 
$\sigma(\mA)=\{0,1\}$. Mitkä ovat ominaisarvojen geometriset kertaluvut?

\item
Matriisilla kokoa $3 \times 3$ on ominaisuus: Matriisin ominaisarvot ovat $\lambda_1=-1$, 
$\lambda_2=1$, $\lambda_3=2$ ja vastaavat ominaisvektorit ovat $[1,1,1]^T$,
$[1,1,-1]^T$ ja $[0,1,2]^T$. Montako erilaista nämä ehdot täyttävää matriisia on olemassa?
Määritä yksi!

\item
Tutki, ovatko seuraavat matriisit $\mA$ diagonalisoituvia. Myönteisessä tapauksessa laske 
matriisille jokin (reaalinen tai kompleksinen) tulohajotelma muotoa $\mA=\mC\mD\mC^{-1}$, missä
$\mD$ on diagonaalinen.
\begin{align*}
&\text{a)}\ \ \begin{rmatrix} 1&0\\4&-3 \end{rmatrix} \qquad
 \text{b)}\ \ \begin{rmatrix} 3&1\\5&-1 \end{rmatrix} \qquad
 \text{c)}\ \ \begin{rmatrix} -3&2\\-2&1 \end{rmatrix} \qquad
 \text{d)}\ \ \begin{rmatrix} 1&2\\-4&-3 \end{rmatrix} \\[2mm]
&\text{e)}\ \ \begin{rmatrix} 1&1&0\\0&2&0\\-2&5&-1 \end{rmatrix} \qquad
 \text{f)}\ \ \begin{rmatrix} -2&-1&1\\1&1&0\\0&1&0 \end{rmatrix} \qquad
 \text{g)}\ \ \begin{rmatrix} 0&2&0\\-2&2&2\\1&1&0 \end{rmatrix} \\[2mm]
&\text{h)}\ \ \begin{rmatrix} -1&-1&0\\1&-2&3\\1&-2&-6 \end{rmatrix} \qquad
 \text{i)}\ \ \begin{rmatrix} 1&0&0&9\\0&1&0&0\\0&0&1&0\\1&0&0&1 \end{rmatrix} \qquad
 \text{j)}\ \ \begin{rmatrix} 1&0&0&3\\0&1&3&0\\0&1&1&0\\1&0&0&1 \end{rmatrix}
\end{align*}

\item
a) Näytä, että similaarisuus on samaa kokoa olevien neliömatriisien välinen
ekvivalenssirelaatio. Millainen on ekvivalenssiluokka, joka sisältää yksikkömatriisin 
$\mI$\,? \vspace{1mm}\newline
b) Olkoon $\mA$ ja $\mB$ lineaarikuvauksen $A:\,\R^n\kohti\R^n$ matriisit avaruuden $\R^n$
kahdessa eri kannassa. Näytä, että $\mA$ ja $\mB$ ovat similaariset. 

\item
Määritä seuraavien matriisien ominaisarvot ja muodosta ominaisvektoreista $\R^2$:n, $\R^3$:n tai
$\R^4$:n ortonormeerattu, positiivisesti suunnistettu kanta.
\begin{align*}
&\text{a)}\ \ \begin{rmatrix} 0&2\\2&0 \end{rmatrix} \qquad
 \text{b)}\ \ \begin{rmatrix} 1&1\\2&2 \end{rmatrix} \qquad
 \text{c)}\ \ \begin{rmatrix} 5&2\\2&8 \end{rmatrix} \qquad
 \text{d)}\ \ \begin{rmatrix} 0&-2\\-2&1 \end{rmatrix} \\[2mm]
&\text{e)}\ \ \begin{rmatrix} 0&-1&-1\\-1&0&1\\-1&1&0 \end{rmatrix} \qquad
 \text{f)}\ \ \begin{rmatrix} 4&2&2\\2&1&1\\2&1&1 \end{rmatrix} \qquad
 \text{g)}\ \ \begin{rmatrix} 1&2&0\\2&1&1\\0&1&1 \end{rmatrix} \\[2mm]
&\text{h)}\ \ \begin{rmatrix} 3&1&1\\1&3&1\\1&1&3 \end{rmatrix} \qquad
 \text{i)}\ \ \begin{rmatrix} 2&0&0&2\\0&1&0&0\\0&0&1&0\\2&0&0&1 \end{rmatrix} \qquad
 \text{j)}\ \ \begin{rmatrix} 1&0&0&2\\0&1&2&0\\0&2&1&0\\2&0&0&1 \end{rmatrix}
\end{align*}

\item (*) \index{Cayleyn--Hamiltonin lause}
\kor{Cayleyn--Hamiltonin lauseen} mukaan neliömatriisi toteuttaa oman karakteristisen
yhtälönsä, ts.\ jos $p(\lambda)$ on matriisin karakteristinen polynomi, niin $p(\mA)=\mo$. \ 
a) Todista lause siinä tapauksessa, että $\mA$ on diagonalisoituva. \ b) Näytä, että lause on
tosi myös Esimerkin \ref{eig-ex4} ei-diagonalisoituvalle matriisille

\end{enumerate}