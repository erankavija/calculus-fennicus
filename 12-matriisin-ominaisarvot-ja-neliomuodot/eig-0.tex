\chapter{Matriisin ominaisarvot ja neliömuodot}

Matriiseja ja matriisialgebraa on aiemmin (Luku \ref{matriisit}) tarkasteltu lähinnä
lineaaristen yhtälöryhmien teorian ja ratkaisemisen näkökulmasta. Tässä luvussa avataan
matriisilaskuun ja sen sovelluksiin kokonaan uusi näkökulma, kun tarkastelun kohteeksi otetaan
matriisiin liitettävät erityiset luvut ja vektorit, nimeltään \kor{ominaisarvot} ja
\kor{ominaisvektorit}.

Luvussa \ref{matriisin ominaisarvot} määritellään matriisin ominaisarvot ja -vektorit, eli
yhdistettynä \kor{ominaisparit}, sekä esitetään ominaisarvoteorian keskeisimmät väittämät.
Osoittautuu, että yleisen reaalisen matriisin ominaisarvot ja -vektorit voivat olla
komp\-leksisia. Symmetrisen matriisin ominaisarvo-ongelma sen sijaan ratkeaa kokonaan
$\R^n$:ssä, ja osoittautuu, että tässä tapauksessa ominaisvektoreista voidaan myös muodostaa
$\R^n$:n ortonormeerattu kanta. Tähän perustuva matriisin \kor{diagonalisointi} on keskeisellä
sijalla Luvussa \ref{diagonalisointi}, jossa symmetrisen matriisin ominais\-arvoteoriaa
sovelletaan $\R^n$:n \pain{neliömuotoihin}. Tällä tavoin ensinnäkin ratkeaa (ainakin osittain)
Luvussa \ref{usean muuttujan ääriarvotehtävät} avoimeksi jätetty ongelma: Onko funktion
$f(\mx)$ kriittinen piste paikallinen minimi, maksimi vai ei kumpikaan? Neliömuotoihin liittyen
ratkaistaan Luvussa \ref{diagonalisointi} myös toisen asteen käyrien ja ja toisen asteen
pintojen geometrisen luokittelun ongelmat matriisien ominaisarvoteorian avulla.    

Luvussa \ref{pinnan kaarevuus} osoitetaan, että symmetriset matriisit ja niiden
ominaisarvoteoria ovat avaimia myös \kor{pinnan kaarevuuden} käsitteeseen. Viimeisessä
osaluvussa esitellään vielä lyhyesti käsite \kor{tensori}, jollaiseksi esim.\ pinnan kaarevuus
on ymmärrettävissä. Tensori ja sen erilaiset esiintymismuodot johdattavat pohtimaan uudelleen
myös vektorin ja skalaarin käsitteitä.
