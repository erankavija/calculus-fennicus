\section{Affiinikuvaukset. Geometriset kuvaukset} \label{affiinikuvaukset} 
\sectionmark{Affiinikuvaukset}
\alku
\index{affiinikuvaus|vahv}
\index{funktio A!h@affiinikuvaus|vahv}

\kor{Affiinikuvaukset} ovat lineaarikuvauksia hieman yleisempiä kuvauksia tyyppiä
\[ 
\mf:\ \R^n \kohti \R^m, \quad \mf(\mx) = \mA\mx + \mb, 
\]
missä $\mA$ on matriisi kokoa $m \times n$ ja $\mb \in \R^m$. Affiinikuvaukset ovat oma tärkeä
luokkansa lineaaristen ($\mb = \mv{0}$) ja yleisempien
\index{epzy@epälineaarinen kuvaus}%
\kor{epälineaaristen} (= ei-lineaaristen)
kuvausten välimaastossa. Ominaisuuksiltaan affiinikuvaukset ovat lähellä lineaarikuvauksia. 
Esim.\ jos $\mf_1(\mx) = \mA_1 \mx + \mb_1$ on affiinikuvaus tyyppiä 
$\mf_1:\,\R^n \kohti \R^p$ ja $\mf_2(\mx) = \mA_2 \mx + \mb_2$ on affiinikuvaus typpiä 
$\mf_2:\,\R^p \kohti \R^m$, niin yhdistetty kuvaus $\mf_2\circ\mv{f}_1$ on määritelty ja myös 
affiininen:
\[ 
(\mf_2\circ\mf_1)(\mx) = \mA_2(\mA_1\mx + \mb_1) + \mb_2 
                       = \mA_2\mA_1\mx + (\mA_2\mb_1 + \mb_2), \quad \mx \in \R^n. 
\]
Samoin nähdään, että jos $\mf(\mx) =  \mA\mx + \mb$ on tyyppiä $\mf:\ \R^n \kohti \R^n$, niin 
$\mf$ on kääntyvä täsmälleen kun $\mA$ on säännöllinen matriisi, jolloin myös
\index{kzyzy@käänteiskuvaus}%
käänteiskuvaus $\mf^{-1}$ on affiininen.

Edellisessä luvussa todettiin, että $\R^n$:n kannan vaihtoa vastaava koordinaattimuunnos on
lineaarikuvaus. Jos kannan vaihtoon yhdistetään myös
\index{origon siirto}%
\kor{origon siirto}, niin koordinaattimuunnoksesta tulee affiinikuvaus.
\begin{Exa} Esitä affiinikuvauksina koordinaattien väliset muunnoskaavat $E^3$:n 
koordinaatistojen $(O,\vec i,\vec j,\vec k\,)$ ja $(O',\vec a,\vec b,\vec c\,)$ välillä, kun
ensin mainitussa koordinaatistossa on $O'=(1,-2,2)$ ja
\[
\vec a=\vec i+\vec j-\vec k,\ \ \vec b=\vec j+2\vec k,\ \ \vec c=\vec i+\vec k.
\]
\end{Exa}
\ratk Jos pisteen $P$ koordinaatit ovat $(x,y,z)$ ja $(x',y',z')$, niin on oltava voimassa
\[
x\vec i+y\vec j+z\vec k=\Vect{OO'}+x'\vec a+y'\vec b+z'\vec c.
\]
Tämä on tulkittavissa vektorimuotoiseksi muunnoskaavaksi $(x',y',z') \map (x,y,z)$, jonka
matriisimuoto on affiinikuvaus:
\[
\begin{bmatrix} x\\y\\z \end{bmatrix} = \begin{rmatrix} 1\\-2\\2 \end{rmatrix} +
\begin{rmatrix} 1&0&1\\1&1&0\\-1&2&1 \end{rmatrix} \begin{bmatrix} x'\\y'\\z' \end{bmatrix}.
\]
Ratkaisemalla tästä $[x',y',z']^T$ saadaan käänteismuunnokseksi
\[
\begin{bmatrix} x'\\y'\\z' \end{bmatrix} = 
\frac{1}{4}\begin{rmatrix} 1&2&-1\\-1&2&1\\3&-2&1 \end{rmatrix} 
           \begin{bmatrix} x\\y\\z \end{bmatrix} +
\frac{1}{4}\begin{rmatrix} 5\\3\\-9 \end{rmatrix}. \loppu
\]

\subsection{Geometriset kuvaukset}
\index{geometrinen kuvaus|vahv}

\kor{Geometrisilla kuvauksilla} ymmärretään yleensä euklidisen pisteavaruuden $E^2$ tai $E^3$
kuvauksia, joilla on jokin (verrattain yksinkertainen) geometrinen merkitys. 
Koordinaattiavaruuteen palauttamalla voidaan geometriset kuvaukset aina tulkita myös $\R^2$:n 
tai $\R^3$:n kuvauksina. Jatkossa rajoitutaan yksinkertaisimpiin geometrisiin kuvauksiin,
jotka ovat affiinisia. Tällaisia ovat mm.\ seuraavat kuvaukset $P \map Q$ tyyppiä 
$f:\ \Ekaksi\kohti\Ekaksi$ tai $f:\ \Ekolme\kohti\Ekolme$ tai näiden yhdistelyt.
\index{suuntaisprojektio} \index{yhdensuuntaisprojektio} \index{peilaus}
\index{skaalaus (homotetia)} \index{homotetia (skaalaus)} \index{translaatio (siirto)}
\index{siirto (translaatio)} \index{kierto!a@geom.\ kuvaus}%
\begin{itemize}
\item[---] \kor{siirto} eli \kor{translaatio}: $\ \Vect{OQ} = \Vect{OP} + \vec{b}\ $ 
           ($\vec{b}$ annettu)
\item[---] \kor{kierto} tasossa origon ympäri: $\ \Vect{OQ} = \Vect{OP}$ kierrettynä kulman 
            $\varphi$ verran vastapäivään ($\varphi \in \R$ annettu)
\item[---] \kor{yhdensuuntaisprojektio} avaruustasolle $T$ (tason suoralle $S$) suoran $L$,
           tai vektorin $\vec v=L$:n suuntavektori, suunnassa: $\Q \in T$ tai $Q \in S$ ja
           $\Vect{PQ}=\lambda\vec v,\ \lambda\in\R\ $ ($\,\vec v\,$ ei $T$:n/$S$:n suuntainen)
\item[---] \kor{suuntaisprojektio avaruussuoralle} $S$ tason $T$, tai vektorien
           $\vec v_1,\,\vec v_2=T$:n suuntavektorit, suunnassa : $Q \in S$ ja 
           $\Vect{PQ}=\lambda\vec{a}_1 + \mu\vec{a}_2,\ (\lambda,\mu)\in\R^2$
           ($\,\{\vec v,\vec{a}_1,\vec{a}_2\}$ lineaarisesti riippumaton, $\vec v=S$:n
           suuntavektori)
\item[---] \kor{peilaus} annetun tason (avaruudessa) tai suoran (tasossa) suhteen
\item[---] \kor{skaalaus} eli \kor{homotetia} origon suhteen: 
           $\ \Vect{OQ} = \lambda\,\Vect{OP}\ $ ($\lambda \in \R$ annettu)
\end{itemize}
Geometrirista kuvauksista yksinkertaisimmat ovat siirto ja homotetia, joiden affiinimuodot ovat 
$\mv{f}(\mx) = \mI\mx + \mb$ ja $\mv{f}(\mx)=\lambda\mI\mx$ ($\mI$ yksikkömatriisi). Yleinen
affinikuvaus $\mv{f}(\mx) = \mA\mx + \mb$ on lineaarikuvauksen ja siirron yhdistelmä:
\[ 
\mx \Map \mA\mx \Map \mA\mx + \mb. 
\]
Jos yhdensuuntaisprojektiossa taso $T$/suora $S$ tai suuntaisprojektiossa suora $S$ kulkee
origon kautta, on $\mf(\mo)=\mo$, jolloin projektio on lineaarikuvuas. Näiden projektioiden
erikoistapauksia ovat ortogonaaliprojektiot, joissa projisoidaan tasoa/suoraa vastaan
kohtisuorasti.
\begin{Exa} \label{3d-projektio matriiseilla} Taso $T$ kulkee origon kautta ja sen 
suuntavektorit ovat $\ \vec{v}_1 = \vec{i} + \vec{j}\ $ ja $\ \vec{v}_2 = \vec{j} + \vec{k}$. 
Piste $\ P\ \vastaa\ x_1 \vec{i} + x_2 \vec{j} + x_3 \vec{k}\ \vastaa\ [x_1,x_2,x_3]^T = \mx\ $
projisoidaan kohtisuorasti tason $T$ pisteeksi 
\[ 
Q\ \vastaa\ y_1 \vec{i} + y_2 \vec{j} + y_3 \vec{k}\ =\ z_1 \vec{v}_1 + z_2 \vec{v}_2. 
\]
Määritä vastaavien lineaarikuvausten
\[ 
A:\ \mx \Map \my = [y_1,y_2,y_3]^T, \quad\quad  B:\ \mx \Map \mz = [z_1,z_2]^T 
\]
matriisit $\mA,\mB$. 
\end{Exa}
\ratk Projektioehto on
\begin{align*}
(x_1 \vec{i} &+ x_2 \vec{j} + x_3 \vec{k} - z_1 \vec{v}_1 - z_2 \vec{v}_2) \cdot \vec{v}_i = 0,
                                                         \quad i = 1,2 \\[2mm]
             &\ekv \quad \left\{ \begin{aligned} 
                                  x_1 + x_2 - 2z_1 - z_2 &= 0 \\ 
                                  x_2 + x_3 -z_1 -2z_2 &= 0 
                                  \end{aligned} \right. \\
             &\ekv \quad \begin{rmatrix} 2&1\\1&2 \end{rmatrix} 
                         \begin{rmatrix} z_1\\z_2 \end{rmatrix}\ =\ 
                         \begin{rmatrix} 1&1&0\\0&1&1 \end{rmatrix} 
                         \begin{rmatrix} x_1\\x_2\\x_3 \end{rmatrix} \\[2mm]
             &\ekv \quad \mC\mz = \mv{D}\mx.
\end{align*}
Siis $\mz = \mB\mx$, missä
\[ 
\mB\ =\ \mC^{-1}\mv{D}\ =\ \frac{1}{3} \begin{rmatrix} 2&-1\\-1&2 \end{rmatrix} 
                                       \begin{rmatrix} 1&1&0\\0&1&1 \end{rmatrix}\ 
                        =\ \frac{1}{3} \begin{rmatrix} 2&1&-1\\-1&1&2 \end{rmatrix}. 
\]
Koska
\[ y_1 \vec{i} + y_2 \vec{j} + y_3 \vec{k}\ =\ z_1 \vec{v}_1 + z_2 \vec{v}_2\ 
                                            =\ z_1 \vec{i} + (z_1 + z_2) \vec{j} + z_2 \vec{k}, 
\]
niin
\[ \begin{rmatrix} y_1\\y_2\\y_3 \end{rmatrix}\ =\ 
   \begin{rmatrix} 1&0\\1&1\\0&1 \end{rmatrix} \begin{rmatrix} z_1\\z_2 \end{rmatrix}\ =\ 
   \begin{rmatrix} 1&0\\1&1\\0&1 \end{rmatrix} \mB \begin{rmatrix} x_1\\x_2\\x_3 \end{rmatrix}. 
\]
Siis $\my = \mA\mx$, missä
\[ 
\mA\ =\ \frac{1}{3} \begin{rmatrix} 1&0\\1&1\\0&1 \end{rmatrix} 
                    \begin{rmatrix} 2&1&-1\\-1&1&2 \end{rmatrix}\ =\ 
        \frac{1}{3} \begin{rmatrix} 2&1&-1\\1&2&1\\-1&1&2 \end{rmatrix}. 
\]
\pain{Tarkistus}: Soveltamalla yhtälöryhmään $\mA\mx = \mv{0}$ Gaussin algoritmia saadaan 
ratkaisuksi $\mx = t[1,-1,1]^T,\ t\in\R$. Tämä sopii yhteen sen tiedon kanssa, että tason $T$
normaalivektori on $\ \vec n = \vec i-\vec j+\vec k$. \quad \loppu

Ortogonaaliprojektion avulla voidaan myös peilauskuvaus määritellä helposti: Jos $P_0$ on
$P$:n projektiopiste, niin $\Vect{OQ} = \Vect{OP} + 2\,\Vect{PP}_0$ määrittää peilikuvan $Q$.
\jatko \begin{Exa} (jatko) Taso $T$ kulkee pisteen $(1,-1,2)$ kautta ja tason suuntavektoreita
ovat $\ \vec{v}_1 = \vec{i} + \vec{j}\ $ ja $\ \vec{v}_2 = \vec{j} + \vec{k}$. Määritä 
peilauskuvaus $(x_1,x_2,x_3) \triangleq P \Map Q \triangleq (y_1,y_2,y_3)$ 
affiinikuvauksena $\my = \mA\mx + \mb$. 
\end{Exa}
\ratk Merkitään $\ma = [1,-1,2]^T$ ja olkoon projektiopiste $P_0 \triangleq (z_1,z_2,z_3)$. 
Tällöin $\mz-\ma = A(\mx-\ma)$, missä $A$ on edellisessä esimerkissä laskettu lineaarikuvaus. 
Kun tämän matriisia merkitään nyt symbolilla $\mA_0$, niin on siis
\[ 
\mz - \ma = \mA_0(\mx - \ma), \quad  \mA_0\ =\ \frac{1}{3} \begin{rmatrix} 
                                                           2&1&-1\\1&2&1\\-1&1&2 
                                                           \end{rmatrix}. 
\]
Näin ollen
\begin{align*}
\my\ &=\ \mx + 2(\mz - \mx)\ =\ (2\mA_0-\mI)\mx + 2(\mI-\mA_0)\ma \\[2mm]
     &=\ \frac{1}{3} \begin{rmatrix} 1&2&-2\\2&1&2\\-2&2&1 \end{rmatrix} 
                     \begin{rmatrix} x_1\\x_2\\x_3 \end{rmatrix}
                              + \frac{8}{3} \begin{rmatrix} 1\\-1\\1 \end{rmatrix} \\[2mm]
     &=\ \mA\mx + \mb. \quad\quad \loppu
\end{align*}
\begin{Exa} Tulkitse kierto tasossa lineaarikuvauksina $\mx \map \my = \mA\mx$. \end{Exa}
\ratk  Vektorin kierto tasossa annetun kulman $\varphi$ verran on tulkittavissa joko
lineaarikuvauksena tyyppiä $A: \R^2\kohti\R^2$ tai kompleksitason kuvauksena
$z = x_1 + i x_2 \Map w = y_1 + iy_2$. Jälkimmäisen tulkinnan perusteella on
\begin{align*}
 y_1 + i y_2 &= (\cos\varphi + i \sin\varphi)(x_1 + i x_2) \\ 
             &= (x_1\cos\varphi - x_2\sin\varphi) + i(x_1\sin\varphi + x_2\cos\varphi),
\end{align*}
joten $A$:n matriisi on
\[ 
\mA\ =\ \begin{rmatrix} 
        \cos\varphi & -\sin\varphi \\ \sin\varphi & \cos\varphi 
        \end{rmatrix}. \qquad \loppu 
\]
\index{kierto!a@geom.\ kuvaus}%
Myös pisteen \kor{kierto avaruussuoran ympäri} on tulkittavissa affiinikuvauksena: Jos $P_0$ on
$P$:n kohtisuora projektio suoralle $S$ (erikoistapaus yhdensuuntaisprojektiosta 
avaruussuoralle), niin $\Vect{OQ} = \Vect{OP}_0 + \Vect{P_0 Q}$, missä $\Vect{P_0 Q} =$ vektori
$\Vect{P_0 P}$ kierrettynä pisteen $P_0$ kautta kulkevassa tasossa, jonka normaali on $S$. 
\jatko \begin{Exa} (jatko) Vektorin $\vec{r} = x_1 \vec{i} + x_2 \vec{j} + x_3 \vec{k}$ 
kiertyessä $x_1$-akselin ympäri pysyy komponentti  $x_1$ muuttumattomana ja $[x_2,x_3]^T$
muuntuu kuten tasokierrossa. Näin ollen kierto vastaa lineaarikuvausta 
$\mv{f}(\mx) = \mA\mx$, missä
\[ 
\mA\ =\ \begin{rmatrix} 
         1&0&0\\0&\cos\varphi&-\sin\varphi\\0&\sin\varphi&\cos\varphi 
         \end{rmatrix}. \qquad \loppu 
\] 
\end{Exa}

Siirron, taso- tai avaruuskierron, peilauksen ja skaalauksen yhdistelmiä $E^n$:ssä 
($n=2$ tai $n=3$) kutsutaan
\index{euklidinen kuvaus}%
\kor{euklidisiksi kuvauksiksi}\footnote[2]{Kuvausta $K \Map \mv{f}(K)$, missä $\mv{f}$ on
pelkästään siirtojen ja kiertojen yhdistelmä, sanotaan \kor{euklidiseksi liikkeeksi},
mekaniikassa (kun $K \subset \Ekolme$ edustaa kiinteää kappaletta) jäykän kappaleen
liikkeeksi. \index{euklidinen liike|av}}. Euklidisen tason kahta pistejoukkoa
$K_1 \subset E^2$ ja $K_2 \subset E^2$ (nämä voivat olla geometrisia kuvioita kuten kolmioita)
\index{yhdenmuotoisuus}%
sanotaan \kor{yhdenmuotoisiksi}, jos on olemassa euklidinen kuvaus $\mv{f}:\ E^2 \kohti E^2$
siten, että $K_2 = \mv{f}(K_1)$. Jos $\mv{f}$ koostuu vain siirroista, kierroista ja
peilauksista, niin sanotaan, että $K_1$ ja $\mv{f}(K_2)$ ovat
\index{yhtenevyys}%
\kor{yhtenevät}.

\subsection{*$\R^n$:n geometriaa}

Geometrinen ajattelu voidaan ulottaa yleisesti avaruuteen $\R^n$, jolloin $\R^n$:n alkiot ovat
(kuvitellun) euklidisen pisteavaruuden $E^n$ pisteiden vastineita. Sanonta 'pisteessä $\mx$' on
tällöin tulkittava niin, että piste $P \in E^n$ ja sen vastine $\mx \in \R^n$ 
($=P$:n koordinaatit $\R^n$:n valitussa ortonormeeratussa kannassa) samastetaan, tai
jälkimmäinen tulkitaan edellisen 'nimeksi'. Kun omaksutaan tämä ajattelutapa, niin geometrian
pisteet, janat, kolmiot ym.\ voidaan tuoda luontevasti suoraan avaruuteen $\R^n$.

Seuraavassa luetellaan eräitä tavallisimpia geometrian käsitteitä avaruuteen $\R^n$ tuotuina ja
yleistettyinä:
\begin{itemize}
\item[---] \kor{Piste} on alkio $\ma \in \R^n$. \index{piste ($\R^n$:n)}
\item[---] \kor{Suora} on pistejoukko \index{suora}
           $S = \{\mx \in \R^n \mid \mx = \mx_0 + t\mv{v},\ t \in \R\}$, missä \\ 
           $\mx_0,\mv{v} \in \R^n$ ja $\mv{v} \neq \mv{0}$.
\item[---] \kor{Hypertaso} (\kor{taso} tapauksessa $n=3$) on pistejoukko \\
           $T = \{\mx \in \R \mid \mx = \mx_0 + \sum_{i=1}^{n-1} t_i \mv{v}_i,\ \mv{t} 
              = (t_i) \in \R^{n-1}\}$, missä \\
           $\mx_0,\mv{v}_i \in \R^n$ ja $\{\mv{v}_1, \ldots \mv{v}_{n-1}\}$ on lineaarisesti 
           riippumaton. \index{taso!a@hypertaso, $m$-taso} \index{hypertaso}
\item[---] \kor{$m$-taso} ($m$-ulotteinen taso) on pistejoukko \\
           $T = \{\mx \in \R \mid \mx = \mx_0 + \sum_{i=1}^{m} t_i \mv{v}_i,\ \mv{t} 
              = (t_i) \in \R^m\}$, missä \\
           $2 \le m \le n-1,\ \mx_0,\mv{v}_i \in \R^n$ ja $\{\mv{v}_1, \ldots \mv{v}_m\}$ on 
           lineaarisesti riippumaton.
\item[---] \kor{Jana}  on pistejoukko 
           $S = \{\mx \in \R^n \mid \mx 
              = \lambda\ma_1 + (1-\lambda)\ma_2,\ \lambda \in [0,1]\}$, \\
           missä $\ma_1,\ma_2 \in \R^n$ ja $\ma_1 \neq \ma_2$. \index{jana ($\R^n$:n)}
\item[---] \kor{Kolmio} on pistejoukko \\
           $K = \{\mx \in \R^n \mid \mx = \sum_{i=1}^3 \lambda_i \ma_i,\ 
                             \lambda_i \in [0,1],\ \ \sum_{i=1}^3 \lambda_i = 1\}$, missä \\
           $\ma_1,\ma_2,\ma_3 \in \R^n$ ja $\{\ma_2-\ma_1,\ma_3-\ma_1\}$ on lineaarisesti 
           riippumaton. \index{kolmio ($\R^n$:n)}
\item[---] \kor{$m$-simpleksi} ($2 \le m \le n+1$) on pistejoukko \\
           $K = \{\mx \in \R^n \mid \mx 
              = \sum_{i=1}^m \lambda_i \ma_i,\ \lambda_i \in [0,1],\ \ 
                \sum_{i=1}^m \lambda_i = 1\}$, missä \\
           $\ma_i \in \R^n$ ja $\{\ma_2-\ma_1, \ldots, \ma_m-\ma_1\}$ on lineaarisesti 
           riippumaton. \index{simpleksi}
\item[---] \kor{$m$-suuntaissärmiö} ($2 \le m \le n$) on pistejoukko \\
           $K = \{\mx \in \R^n \mid \mx 
              = \mx_0+\sum_{i=1}^m \lambda_i \ma_i,\ \lambda_i \in [0,1]\}$, missä 
           $\mx_0,\ma_i\in\R^n$ ja $\{\ma_1,\ldots,\ma_m\}$ on lineaarisesti riippumaton
           \index{suuntaissärmiö ($\R^n$:n)}
\item[---] \kor{Pallo (kuula)} on pistejoukko 
           $B(\mx_0,R) = \{\mx \in \R^n \mid \abs{\mx-\mx_0} \le R\}$, missä \\
           $\mx_0 \in \R^n$ ja $R>0$. \index{kuula}
\item[---] \kor{Pallopinta} on pistejoukko $K = \{\mx \in \R^n \mid \abs{\mx-\mx_0} = R\}$, 
           missä \\
           $\mx_0 \in \R^n$ ja $R>0$. \index{pallo(pinta)}
\item[---] \kor{$n$-suorakulmio} (koordinaatiakselien mukaan suunnattu) on pistejoukko \\
           $K = \{\mx = (x_i) \in \R^n \mid x_i \in [a_i,b_i],\ i = 1 \ldots n\}$, missä \\
           $a_i,b_i \in \R,\ a_i < b_i,\ i = 1 \ldots n$. \index{n@$n$-suorakulmio}
\end{itemize}
Määritelmien  mukaisesti hypertaso = $(n-1)$--ulotteinen taso, $2$-simpleksi = jana, 
$3$-simpleksi = kolmio ja $\R^3$:n $4$-simpleksi = tetraedri. Hypertason ja $m$-tason 
määritelmissä sanotaan vektoreita $\mv{v}_i$ ko.\ tason
\index{suuntavektori}%
\kor{suuntavektoreiksi}. Simpleksin 
\index{kzy@kärkipiste (simpleksin)}%
määritelmässä pisteet $\ma_i$ ovat simpleksin \kor{kärkipisteet} (engl.\ vertex). Jos
$n$-suorakulmiossa on $b_i-a_i = a,\ i = 1 \ldots n$, niin kyseessä on 
\index{n@$n$-kuutio}%
\kor{$n$-kuutio} (sivun pituus $\,a$).

Ym.\ määritelmässä on hypertaso ja $m$-taso esitetty
\index{parametri(sointi)!e@hypertason, $m$-tason}%
\kor{parametrisessa} muodossa suuntavektoreittensa $\mv{v}_i$ avulla. Hypertasolle voidaan myös
kirjoittaa \kor{hypertason yhtälö} muodossa
\[ 
T:\ \ \scp{\mx-\mx_0}{\mv{n}} = \mv{n}^T(\mx-\mx_0) = 0, 
\]
missä vektori $\mv{n}$ on $T$:n
\index{normaali(vektori)!b@tason, hypertason}%
\kor{normaali}, eli kohtisuorassa vektoreita $\mv{v}_i$ vastaan.
Tällainen $\mv{n}$ on löydettävissä esim.\ ortogonaaliprojektion avulla, vrt.\ Esimerkki 
\ref{projektioesimerkki} edellisessä luvussa. Yleisemmin voidaan $m$-taso määritellä 
yhtälöryhmällä
\[ 
T:\ \ \scp{\mx-\mx_0}{\mv{n}_i} = 0, \quad i = 1 \ldots n-m, 
\]
missä vektorit $\mv{n}_i$ ovat keskenään lineaarisesti riippumattomia $T$:n normaalivektoreita.
Hypertasoihin ja $m$-tasoihin liittyviä geometrisia ongelmia voidaan tyypillisesti ratkaista 
matriisialgebran avulla.
\begin{Exa} Millä ehdolla hypertasot
\[ 
T_i:\ \ \scp{\mx-\ma_i}{\mv{n}_i} = 0, \quad i = 1, \ldots, n 
\]
leikkaavat täsmälleen yhdessä pisteessä? 
\end{Exa}
\ratk Matriisialgebran mukaan tasojen $T_i$ yhtälöt muodostavat lineaarisen yhtälöryhmän 
$\mA\mx = \mb$, missä \mA:n $i$:s rivi $= \mv{n}_i^T$ ja $[\mb]_i = \scp{\ma_i}{\mv{n}_i}$. 
Siis vastaus on: Täsmälleen sillä ehdolla, että $\mA$ on säännöllinen eli että normaalivektorit
$\mv{n}_i$ muodostavat lineaarisesti riippumattoman systeemin ($\R^n$:n kannan). \loppu  

\subsection{*Lineaarinen optimointi $\R^n$:ssä}
\index{lineaarinen optimointi|vahv} \index{optimointi!a@lineaarinen optimointi|vahv}

Reaaliarvoisen $\R^n$:n lineaarikuvauksen $L:\ \R^n \kohti \R$ yleinen muoto on
\[ 
L\mx = \ma^T \mx,
\]
missä $\ma \in \R^n$ (oletetaan jatkossa: $\ma\neq\mv{0}$). Tällaisen funktion maksimi- tai 
minimiarvon etsimistä annetussa joukossa $K \subset \R^n$ sanotaan \kor{lineaarisen optimoinnin}
tehtäväksi. Jos $T$ on hypertaso, jonka yhtälö on
\[ 
T:\ \ \scp{\ma}{\mx-\mx_0}=\ma^T(\mx-\mx_0)= 0, 
\]
niin $L\mx = \ma^T \mx_0=$ vakio, kun $\mx \in T$, eli tällaiset hypertasot ovat $L$:n 
\index{tasa-arvopinta}%
\kor{tasa-arvopintoja}. Toisaalta jos tarkastellaan $L$:n arvoja pisteen $\mv{c}$ kautta 
kulkevalla, tasa-arvopintoja vastaan kohtisuoralla suoralla
\[ 
S:\ \ \mx = \mv{c} + t \ma,\ \ t \in \R, 
\]
niin nähdään, että tällä suoralla on
\[
L\mx = L(\mv{c}+t\ma) = \alpha t+\beta, \quad \alpha = \abs{\ma}^2 > 0,\ \beta = \ma^T\mv{c}.
\]
Jos nyt $L$:n tasa-arvopintojen yhtälössä valitaan $\mx_0 = \mv{c}+t\ma \in S$, ja merkitään 
tätä tasa-arvopintaa (hypertasoa) $T_t$:llä, niin parametria $t$ vaihtelemalla tulee koko
avaruus $\R^n$ 'pyyhityksi' hypertasoilla $T_t$. Tällöin nähdään heti, että jos $L$:llä on 
pistejoukossa $K$ maksimi- tai minimiarvo, niin tämän on oltava vastaavasti pienin\,/\,suurin 
luku reaalilukujoukossa
\[
A = \{t\in\R \mid T_t \cap K \neq \emptyset\}.
\]
Optimointiongelma on näin 'ratkaistu' geometrisesti. --- Idean soveltaminen käytäntöön kylläkin
edellyttää, että joukko $A$ on määrättävissä. Seuraavassa esimerkissä $K$ on geometrisesti 
yksinkertainen ja $n$ kohtuullinen, jolloin ongelma ratkeaa 'käsipelillä'. 
\begin{Exa} Määritä funktion $f(x_1,x_2,x_3,x_4) = 2x_1-4x_2+5x_3-6x_4$ pienin ja suurin arvo
$\R^4$:n pallossa
\[
K= \{\mx\in\R^4 \mid \abs{\mx-\mx_0} \le R\}, \quad 
               \text{missä}\,\ \mx_0=(1,-1,3,2)\,\ \text{ja}\,\ R=3. 
\]
\end{Exa}
\ratk Tässä on $f(\mx)=L\mx=\ma^T\mx,\ \ma=[2,-4,5,-6]^T,\ \mx=[x_1,x_2,x_3,x_4]^T$. Jos $T_t$
on $\R^4$:n hypertaso, joka kulkee pisteen $P_t \vastaa \mx_0+t\ma$ kautta, niin pätee
\[
T_t \cap K \neq \emptyset \qekv \abs{t\ma}=\abs{t}\abs{\ma} 
                                          \le 3 \qekv \abs{t}\ 
                                          \le\ \frac{3}{\sqrt{2^2+4^2+5^2+6^2}}\ 
                                          =\ \frac{1}{3}\,.
\]
Siis $\{t\in\R \mid T_t \cap K \neq \emptyset\} = [-1/3,\,1/3]$. Koska hypertasolla $T_t$ on
\[
f(\mx)\ =\ \ma^T(\mx_0+t\ma)\ =\ \ma^T\mx_0+\abs{\ma}^2 t\ =\ 9+81 t, \quad \mx \in T_t\,,
\]
niin
\[
f_{min}=9+81\cdot\left(-\frac{1}{3}\right)=\underline{\underline{-18}}, \quad 
f_{max}=9+81\cdot\frac{1}{3}=\underline{\underline{36}}. \loppu
\]

Sovelluksissa hyvin yleinen on lineaarisen optimoinnin ongelma, jossa $K$ on määritelty 
\index{lineaarinen epäyhtälö}%
\kor{lineaaristen epäyhtälöiden} avulla muodossa
\[
K = \{\mx\in\R^n \mid \ma_i^T\mx \le c_i\,,\ i=1\,\ldots\,m\},
\]
missä $\ma_i\in\R^n,\ \ma_i\neq\mv{0}$, ja $c_i\in\R$. Joukkoa $K$ voi luonnehtia geometrisesti
\index{monitahokas ($\R^n$:n)}%
\kor{monitahokkaaksi}, jota hypertasot $T_i\,:\ \ma_i^T\mx=c_i$ rajoittavat (yleensä on $m>n$).
Tällaista ongelmaa on perinteisesti kutsuttu
\index{lineaarinen optimointi!lineaarinen ohjelmointi}%
\kor{lineaarisen ohjelmoinnin} (engl.\ linear programming) ongelmaksi. Ongelman sekä suoraan
(periaatteessa tarkkaan) että likimääräiseen iteratiiviseen ratkaisuun on kehitetty algoritmeja.

\Harj
\begin{enumerate}

\item
Millaiseksi muuntuu \, a) suorakulmio, jonka kärjet ovat pisteissä $(1,1)$, $(3,1)$, $(3,2)$ ja
$(1,2)$, \ b) yksikkympyrä $S:\,x^2+y^2=1$ affiinikuvauksessa
\[
\mf(x,y)=\begin{bmatrix} 1&2\\3&4 \end{bmatrix} \begin{bmatrix} x\\y \end{bmatrix} +
         \begin{bmatrix} 5\\6 \end{bmatrix} \ ?
\]
Piirrä kuva! Määritä myös käyrän $S'=\mf(S)$ yhtälö.

\item
Esitä affiinikuvuaksena: $\mf:\,\R^2\kohti\R^2$ tai $\mf:\,\R^3\kohti\R^3$\,: 
\vspace{1mm}\newline
a) \ Kohtisuora projektio tasossa suoralle $x+2y+1=0$ \newline
b) \ Projektio suoralle $x-3y-2=0$ suoran $y=2x$ suunnassa \newline
c) \ Kierto tasossa pisteen $(2,3)$ ympäri $60\aste$ vastapäivään \newline
d) \ Projektio tasolle $T:\,x+2z-3y+3=0$ vektorin $\vec i+\vec j-\vec k$ suunnassa \newline
e) \ Peilaus tason $T:\,3x-2y+z-5=0$ suhteen \newline
f) \ Peilaus avaruussuoran $S:\,\vec r=(1+t)\vec i+(t-2)\vec j+(2t-3)\vec k$ suhteen

\item
Taso $T$ kulkee origon kautta ja sen suuntavektoreita ovat $\vec a_1=\vec i+\vec j+\vec k$ ja
$\vec a_2=\vec j-2\vec k$. Suora $S$ kulkee pisteen $(0,1,2)$ kautta ja sen suuntavektori on
$\vec a_3=-\vec i+\vec j+\vec k$. \vspace{1mm}\newline
a) Olkoon $\my=[y_1,y_2,y_3]^T$ vektorin $x_i\vec i+x_2\vec j+x_3\vec k$ koordinaatit kannassa
$\{\vec a_1,\vec a_2,\vec a_3\}$. Määritä lineaarikuvaus $\mx\map\my$, ts.\ matriisi $\mC$
siten, että $\my=\mC\mx$. \newline
b) Käyttäen a-kohdan tulosta ja matriisialgebraa määritä affiinikuvaus $\mf(\mx)=\mA\mx+\mb$,
joka suorittaa pisteen $P=(x_1,x_2,x_3)$ suuntaisprojisoinnin suoralle $S$ tason $T$
suunnassa ($\mf:\,\R^3\kohti\R^3$). \newline
c) Määrittele affiinikuvaus tyyppiä $\mf:\,\R^3\kohti\R^2$, joka suorittaa pisteen
$P=(x_1,x_2,x_3)$ projisoinnin suoran $S$ sunnassa tason $T$ pisteeksi $Q$, joka on ilmoitettu
koordinaatteina kannassa $\{\ma_1,\ma_2\}$. 

\item
Vektori $\vec b$ saadaan vektorista $\vec a=2\vec i+\vec j+2\vec k$ kiertämällä $\vec a$:ta
ensin $x$-akselin ympäri kulma $30\aste$ ja näin saatua vektoria edelleen $y$-akselin ympäri
$60\aste$ (kierrot positiivisten koordinaattiakselien suunnasta katsoen vastapäivään). Määritä
kummankin kierto-operaation (lineaarikuvauksia) matriisit, näiden avulla koko operaation
matriisi $\mA$ ja $\mA$:n avulla vektori $\vec b$. Mitä vektoria mainituilla tavoilla
kierrettäessä saadaan lopputulokseksi $\vec b=\vec k$\,?

\item
Osoita lineaarikuvaus
\[
\mf(x,y)=\frac{1}{25}\begin{rmatrix} 7&-24\\-24&-7 \end{rmatrix} 
                     \begin{bmatrix} x\\y \end{bmatrix}
\]
peilaukseksi erään suoran suhteen ja määritä suoran yhtälö.

\item
Kolmion $A$ kärjet ovat pisteissä $(0,0)$, $(1,0)$ ja $(0,1)$ ja kolmion $B$ kärjet ovat 
pisteissä $(-1,1)$, $(-2,2)$ ja $(-3,0)$. Määritä kaikki kolmiot $C$, joille pätee $C=\mv{f}(B)$
ja $B=\mv{f}(A)$, missä $\mv{f}$ on tason affiinikuvaus.

\item
Seuraavat lineaarikuvaukset määrittelevät yhdensuuntaisprojektion origon kautta kulkevalle
tasolle $T$ suoran $S$ sunnassa. Määritä $T$:n yhtälö ja $S$:n suuntavektori.
\begin{align*}
&\text{a)}\ \ \begin{bmatrix} x'\\y'\\z' \end{bmatrix} =
              \frac{1}{5}\begin{rmatrix} 4&1&-3\\1&4&3\\-1&1&2 \end{rmatrix}
                         \begin{bmatrix} x\\y\\z \end{bmatrix} \\[3mm]
&\text{b)}\ \ \begin{bmatrix} x'\\y'\\z' \end{bmatrix} =
              \frac{1}{10}\begin{rmatrix} 7&-6&-9\\-2&6&-6\\-1&-2&7 \end{rmatrix}
                          \begin{bmatrix} x\\y\\z \end{bmatrix} 
\end{align*}

\item 
Näytä, että jos affiinimuunnos $\mf:\,\R^n\kohti\R^n$, missä $n=2$ tai $n=3$, on kääntyvä, niin
$\mf$  \vspace{1mm}\newline
a) kuvaa suoran suoraksi ($n=2,3$) ja tason tasoksi ($n=3$) \newline
b) kuvaa janan janaksi ja kolmion kolmioksi \newline
c) kuvaa tetraedrin tetraedriksi ($n=3$), \newline
d) säilyttää suorien ja tasojen yhdensuuntaisuuden, \newline
e) säilyttää kahden yhdensuuntaisen janan pituuksien suhteen. \newline
f) Miten ominaisuus c) muuttuu, jos $\mf$ ei ole kääntyvä?

\item
Missä joukon $A\subset\R^4$ pisteessä funktio
\[
f(\mx)=x_1-2x_2-x_3+2x_4
\]
saavuttaa suurimman arvonsa, kun $A$ on \vspace{1mm}\newline
a) yksikkökkuutio: $A=\{\mx\in\R^4 \mid \abs{x_i} \le 1,\ i=1 \ldots 4\}$, \newline
b) yksikköpallo: $A=\{\mx\in\R^4 \mid x_1^2+\ldots+x_4^2 \le 1\}$\,?

\item (*)
Määritä affiinikuvaus $\mf:\,\R^3\kohti\R^3$, joka suorittaa kierron suoran
$S:\ x=1+6t,\ y=1-3t,\ z=-2t\,$ ympäri, suunnasta $\vec v=6\vec i-3\vec j-2\vec j$ katsoen
kulman $\varphi$ verran vastapäivään. \kor{Vihje}: Määritä $\mv{f}$ ensin koordinaatistossa,
jossa $S$ on koordinaattiakseli --- ks.\
Harj.teht.\,\ref{lineaarikuvaukset}:\,\ref{H-m-6: kiertoja}a!

\item (*)
Ratkaise lineaarisen optimoinnin tehtävä $f(x,y,z)=x+2y+3x=$max! ehdoilla
\[
\begin{cases}
x \ge 0,\ y \ge 0,\ z \ge 0, \\ 2x+y-2z \le 6, \\ -x+5y+z \le 8, \\ 4x+2y+7z \le 23.
\end{cases}
\]
\kor{Vihje}: Riittää tutkia monitahokkaan kärkipisteet!

\item (*) \index{zzb@\nim!Illaksi kotiin}
(Illaksi kotiin) Liikemies näkee lentokoneesta kotitalonsa suunnassa
$-2\vec i-3\vec j-\vec k$. Talon perusosa on suorakulmainen särmiö, jonka korkeus $=1$ ja
jonka maan pinnalla olevat nurkat ovat $xy$-tason pisteissä $(0,0)$, $(3,0)$, $(3,2)$ ja
$(0,2)$. Talossa on symmetrinen harjakatto, harjan korkeus $=2$ ja suunta $=\vec i$. Räystäät
talon sivuilla ja päädyissä ulottuvat eäisyydelle $0.2$ seinistä. Määrittele affiinikuvaus
$\mf:\,\R^3\map\R^2$, joka muuntaa kolmiulotteisen todellisuuden liikemiehen näkemäksi
projektiokuvaksi (oleta luonteva pään asento). Piirrä kuva talosta sellaisena kuin liikemies
sen näkee.

\end{enumerate}